    \documentclass[11pt,
        usenames, % allows access to some tikz colors
        dvipsnames % more colors: https://en.wikibooks.org/wiki/LaTeX/Colors
    ]{article}
    \usepackage{
        amsmath,
        amssymb,
        fouriernc, % fourier font w/ new century book
        fancyhdr, % page styling
        lastpage, % footer fanciness
        hyperref, % various links
        setspace, % line spacing
        amsthm, % newtheorem and proof environment
        mathtools, % \Aboxed for boxing inside aligns, among others
        float, % Allow [H] figure env alignment
        enumerate, % Allow custom enumerate numbering
        graphicx, % allow includegraphics with more filetypes
        wasysym, % \smiley!
        upgreek, % \upmu for \mum macro
        listings, % writing TrueType fonts and including code prettily
        tikz, % drawing things
        booktabs, % \bottomrule instead of hline apparently
        xcolor, % colored text
        cancel % can cancel things out!
    }
    \usepackage[margin=1in]{geometry} % page geometry
    \usepackage[
        labelfont=bf, % caption names are labeled in bold
        font=scriptsize % smaller font for captions
    ]{caption}
    \usepackage[font=scriptsize]{subcaption} % subfigures

    \newcommand*{\scinot}[2]{#1\times10^{#2}}
    \newcommand*{\dotp}[2]{\left<#1\,\middle|\,#2\right>}
    \newcommand*{\rd}[2]{\frac{\mathrm{d}#1}{\mathrm{d}#2}}
    \newcommand*{\pd}[2]{\frac{\partial#1}{\partial#2}}
    \newcommand*{\rdil}[2]{\mathrm{d}#1 / \mathrm{d}#2}
    \newcommand*{\pdil}[2]{\partial#1 / \partial#2}
    \newcommand*{\rtd}[2]{\frac{\mathrm{d}^2#1}{\mathrm{d}#2^2}}
    \newcommand*{\ptd}[2]{\frac{\partial^2 #1}{\partial#2^2}}
    \newcommand*{\md}[2]{\frac{\mathrm{D}#1}{\mathrm{D}#2}}
    \newcommand*{\pvec}[1]{\vec{#1}^{\,\prime}}
    \newcommand*{\svec}[1]{\vec{#1}\;\!}
    \newcommand*{\bm}[1]{\boldsymbol{\mathbf{#1}}}
    \newcommand*{\uv}[1]{\hat{\bm{#1}}}
    \newcommand*{\ang}[0]{\;\text{\AA}}
    \newcommand*{\mum}[0]{\;\upmu \mathrm{m}}
    \newcommand*{\at}[1]{\left.#1\right|}
    \newcommand*{\bra}[1]{\left<#1\right|}
    \newcommand*{\ket}[1]{\left|#1\right>}
    \newcommand*{\abs}[1]{\left|#1\right|}
    \newcommand*{\ev}[1]{\left\langle#1\right\rangle}
    \newcommand*{\p}[1]{\left(#1\right)}
    \newcommand*{\s}[1]{\left[#1\right]}
    \newcommand*{\z}[1]{\left\{#1\right\}}

    \newtheorem{theorem}{Theorem}[section]

    \let\Re\undefined
    \let\Im\undefined
    \DeclareMathOperator{\Res}{Res}
    \DeclareMathOperator{\Re}{Re}
    \DeclareMathOperator{\Im}{Im}
    \DeclareMathOperator{\Log}{Log}
    \DeclareMathOperator{\Arg}{Arg}
    \DeclareMathOperator{\Tr}{Tr}
    \DeclareMathOperator{\E}{E}
    \DeclareMathOperator{\Var}{Var}
    \DeclareMathOperator*{\argmin}{argmin}
    \DeclareMathOperator*{\argmax}{argmax}
    \DeclareMathOperator{\sgn}{sgn}
    \DeclareMathOperator{\diag}{diag\;}

    \colorlet{Corr}{red}

    % \everymath{\displaystyle} % biggify limits of inline sums and integrals
    \tikzstyle{circ} % usage: \node[circ, placement] (label) {text};
        = [draw, circle, fill=white, node distance=3cm, minimum height=2em]
    \definecolor{commentgreen}{rgb}{0,0.6,0}
    \lstset{
        basicstyle=\ttfamily\footnotesize,
        frame=single,
        numbers=left,
        showstringspaces=false,
        keywordstyle=\color{blue},
        stringstyle=\color{purple},
        commentstyle=\color{commentgreen},
        morecomment=[l][\color{magenta}]{\#}
    }

\begin{document}

\section{Summary of J0045-7319 with MESA} % chktex 8

\subsection{Parameters \& Background}

PSR J0045-7319 is in the SMC\@. % chktex 8
The following properties are measured:
\begin{itemize}
    \item $L = \scinot{1.2}{4}L_{\odot}$.
    \item $T_{\rm eff} = \p{2.4 \pm 0.1} \times 10^4 \;\mathrm{K}$, giving $R =
        6.4R_{\odot}$.
    \item $P_{\rm orb} = 51.17\;\mathrm{day}$.
    \item $q \equiv M / M_{\rm NS} = 6.3$ the mass ratio.
    \item $\dot{P} = \scinot{-3.03}{-7}$.
    \item $v \sin i = 110\;\mathrm{km/s}$ surface rotation rate of the star.
        Breakup $v = \sqrt{GM/R} \approx 500\;\mathrm{km/s}$.
\end{itemize}

Kaspi+1996 (doi:10.1038/381584a0) assume that the NS has mass $1.4
M_{\odot}$\footnote{At the time, all NS mass measurements were consistent with
$1.4M_{\odot}$ so they took it as a fixed parameter.}, giving a stellar mass of
$8.8M_{\odot}$ and an inferred separation $a = 126R_{\odot}$. Kumar \& Quatart
(doi:10.1086/305091) use a ``Yale model'' to infer a convective core of extent
$r_{\rm c} = 0.23R = 1.38R_{\odot}$ and mass $M_{\rm c} = 3M_{\odot}$.

Later, a study by Thorsett \& Chakrabarty (doi:10.1086/306742) compare to
stellar models to infer a stellar mass of $10M_{\odot}$ and NS mass of
$1.58M_{\odot}$ instead.

Separately, the model of dynamical tides includes a density dependence as:
\begin{equation}
    T \propto \frac{\rho_{\rm c}}{\bar{\rho}_{\rm c}}\p{1 - \frac{\rho_{\rm
    c}}{\bar{\rho}_{\rm c}}}^2.
\end{equation}
This factor is not considered in previous works. It is maximized at $4/27$ when
$\rho_{\rm c} / \bar{\rho}_{\rm c} = 1/3$. Assuming this density ratio, the
convective core needs to be at least $0.93R_{\odot}$ in extent to generate the
observed $\dot{P}$ without rotating above \emph{core} breakup, i.e.\ satisfying
$\Omega_{\rm c} \leq \sqrt{GM_{\rm c} / R_{\rm c}^3}$.

\textbf{Puzzle:} Even for a very large core $R_{\rm c} \sim 1.5R_{\odot}$, the
core still needs to be rotating substantially faster than $\sqrt{GM/R^3}$, the
entire star's breakup rotation rate, see Figure~8 of the draft. This requires
substantial differential rotation. Is this believable?

\emph{NB:} This is in line with the Kumar \& Quataert result, which estimates a
star rotating quite near surface critical rotation but without the density
correct factor.

\subsection{My MESA Models}

\textbf{The current thinking is not to include these results unless there's a
very good chance of finding a useful, updated stellar structure.}

I'm not too confident that I've implemented the spin prescription
accurately/correctly, but the sorts of results I'm obtaining are shown in
Fig.~\ref{fig:2}. The three columns show: (i) finding the correct mass that
evolves through the observed $(T, R)$, (ii) showing the evolution of the stellar
radius (black), core radius (red), and core H$_1$ fraction (black) as the
best-fitting model evolves, with the vertical blue line showing the time that it
matches the observed stellar properties, and (iii) propagation diagram at the
time of best fit.
\begin{figure}
    \centering
    \includegraphics[width=0.3\columnwidth]{../../../../MESA/tutorials/4_J00457319/plots/lowm_nonrots_LT.png}
    \includegraphics[width=0.3\columnwidth]{../../../../MESA/tutorials/4_J00457319/plots/lowm_nonrots_9_4_radii.png}
    \includegraphics[width=0.3\columnwidth]{../../../../MESA/tutorials/4_J00457319/plots/lowm_nonrots_9_4_prop.png}

    \includegraphics[width=0.3\columnwidth]{../../../../MESA/tutorials/4_J00457319/plots/highalpha_sims_LT.png}
    \includegraphics[width=0.3\columnwidth]{../../../../MESA/tutorials/4_J00457319/plots/highalpha_sims_9_0_radii.png}
    \includegraphics[width=0.3\columnwidth]{../../../../MESA/tutorials/4_J00457319/plots/highalpha_sims_9_0_prop.png}

    \includegraphics[width=0.3\columnwidth]{../../../../MESA/tutorials/4_J00457319/plots/higha_004_sims_LT.png}
    \includegraphics[width=0.3\columnwidth]{../../../../MESA/tutorials/4_J00457319/plots/higha_004_sims_9_8_radii.png}
    \includegraphics[width=0.3\columnwidth]{../../../../MESA/tutorials/4_J00457319/plots/higha_004_sims_9_8_prop.png}
    \caption{The three sets of simulations I present here are shown in each row:
    (i) low metallicity ($Z = 0.0012$), non-rotating, (ii) rotating with large
    convective overshoot, (iii) same as (ii) but with metallicity $Z = 0.004$.
    The first column shows the $L$-$T$ diagram, with which we try to match the
    observed parameters, the second column shows the evolution of some
    parameters of the best-fitting star as a function of age, and the third
    column shows the propagation diagrams at the time of best fit (dashed lines
    denote negative values). The best fitting masses are (i) $M = 9.6M_{\odot}$,
    (ii) $M = 9.0M_{\odot}$, and (iii) $M = 9.8M_{\odot}$. }\label{fig:2}
\end{figure}

The problem with these simulations is that: (i) the convective core is never
more than $\sim 0.7R_{\odot}$, because some stellar evolution off the MS is
required to match the $(L, R)$, and (ii) the density ratio above is actually
closer to $0.75$, requiring a stellar core $\gtrsim R_{\odot}$ to produce the
required stellar torque. Since $T \propto R^5$, this bound is actually somewhat
tight.

\textbf{Puzzle:} It seems that the cores generated by MESA outright cannot
produce the expected tidal torque even if rotating at core breakup. Even if they
could, they would still require substantial differential rotation.

\end{document}

