    \documentclass[
        fleqn,
        usenatbib,
        % referee,
    ]{mnras}
    \usepackage{
        amsmath,
        amssymb,
        newtxtext,
        newtxmath,
        graphicx,
        ae, aecompl,
        booktabs,
        caption,
        subcaption,
    }
    \usepackage[T1]{fontenc}
    \captionsetup{compatibility=false}

    \newcommand*{\rd}[2]{\frac{\mathrm{d}#1}{\mathrm{d}#2}}
    \newcommand*{\pd}[2]{\frac{\partial#1}{\partial#2}}
    \newcommand*{\rdil}[2]{\mathrm{d}#1 / \mathrm{d}#2}
    \newcommand*{\pdil}[2]{\partial#1 / \partial#2}
    \newcommand*{\rtd}[2]{\frac{\mathrm{d}^2#1}{\mathrm{d}#2^2}}
    \newcommand*{\ptd}[2]{\frac{\partial^2 #1}{\partial#2^2}}
    \newcommand*{\md}[2]{\frac{\mathrm{D}#1}{\mathrm{D}#2}}
    \newcommand*{\pvec}[1]{\vec{#1}^{\,\prime}}
    \newcommand*{\svec}[1]{\vec{#1}\;\!}
    \newcommand*{\bm}[1]{\boldsymbol{\mathbf{#1}}}
    \newcommand*{\uv}[1]{\hat{\bm{#1}}}
    \newcommand*{\ang}[0]{\;\text{\AA}}
    \newcommand*{\mum}[0]{\;\upmu \mathrm{m}}
    \newcommand*{\at}[1]{\left.#1\right|}
    \newcommand*{\bra}[1]{\left<#1\right|}
    \newcommand*{\ket}[1]{\left|#1\right>}
    \newcommand*{\abs}[1]{\left|#1\right|}
    \newcommand*{\ev}[1]{\langle#1\rangle}
    \newcommand*{\p}[1]{\left(#1\right)}
    \newcommand*{\s}[1]{\left[#1\right]}
    \newcommand*{\z}[1]{\left\{#1\right\}}

    \DeclareMathOperator*{\sgn}{sgn}
    \DeclareMathOperator*{\argmax}{argmax}

\title[Eccentric Dynamical Tides]{Eccentric Dynamical Tides}
\author[Y. Su, D. Lai.]{
Yubo Su$^1$,
Dong Lai$^1$
\\
$^1$ Cornell Center for Astrophysics and Planetary Science, Department of
Astronomy, Cornell University, Ithaca, NY 14853, USA
}

\date{Accepted XXX\@. Received YYY\@; in original form ZZZ}

\pubyear{2020}

\begin{document}\label{firstpage}
\pagerange{\pageref{firstpage}--\pageref{lastpage}}
\maketitle

\begin{abstract}
    Abstract
\end{abstract}

\begin{keywords}
keywords % chktex 8
\end{keywords}

\section{Introduction}

In the course of their evolution, massive stellar binaries give rise to many
astrophysical systems of interest including high mass x-ray binaries (HMXRBs)
and compact object binaries (CITATION with evolutionary pathway). In general,
the more massive star undergoes a supernova before its less massive companion,
after which the binary consists of one massive star and one compact object in an
eccentric orbit (MS-CO binary). In such a system, the evolution is dominated by
the torque from the compact object on the massive star due to dynamical tides.
While this tidal torque is now well understood for circular binaries
\citep{kushnir}, it has not been carefully studied for binaries with substantial
eccentricities. The tidal evolution of such eccentric binaries sculpt the
population of HMXRBs (CITE) as well as the population of compact object binaries
\citep{vigna2020common}.

The dissipation due to the dynamical tide in a massive star's envelope under the
influence of a \emph{circular} perturber is traditionally understood via Zahn's
parameterized theory of dynamical tides \citep{zahn1975dynamical}. However,
Zahn's theory relies on a dimensionless parameter $E_2$ reflecting the detailed
stellar structure that varies over many orders of magnitudes for typical stars.
In general, the value of $E_2$ is taken from empirical fits to simplified
stellar models \citep{hurley2002evolution, vigna2020common}. The introduction of
this uncertain parameter $E_2$ is because the dynamical tidal torque arises due
to excitation of internal gravity waves at the radiative convective boundary
(RCB) \citep{goldreich1989tidal, savonije1983tidal}, but Zahn's formula is
evaluated at the \emph{stellar} radius. Instead, it is possible re-express the
tidal torque in terms of quantities evaluated at the RCB itself, for which
dimensionless parameters are generally of order unity for a wide range of stars
\citep{kushnir}. However, \citet{kushnir} only consider circular binaries.

To study the dynamical tide in eccentric binaries, it is natural to decompose
the perturbing potential into Fourier harmonics, each of which is analogous to a
perturber on a circular orbit \citep[e.g.][]{sl, vlf}. While accurate, such
decompositions are unwieldy to evaluate as the eccentricity increases, often
requiring summing hundreds of terms and lending little intuition to the broad
scalings of the tidal torque. In this work, we show that, for the circular
torque given by \citet{kushnir}, an accurate, approximate, closed form for the
dynamical tide in a highly eccentric MS-CO binary can be obtained. Contrary to
existing models of dynamical tides \citep[e.g.][]{vigna2020common}, our
formulation improves in accuracy as the binary eccentricity increases.  We give
closed forms for both the tidal torque and inspiral rate of such an MS-CO
binary.

In Section~\ref{s:background}, we review the relevant equations. In
Sections~\ref{s:hansens} and~\ref{s:eval}, we derive accurate, approximate
closed forms for the torque and energy transfer rate in the binary. In
Section~\ref{s:j00457319}, we apply results to the binary radio pulsar
J0045-7319. We conclude and discuss in Section~\ref{s:disc}.% chktex 8

\section{Summary of Relevant Work}\label{s:background}

\subsection{Tidal Torque in Massive Stars}

We first review the case where the MS-CO binary is circular. Let $M_2$ be the
mass of the CO, $a$ be the semimajor axis of the binary, and $\omega$ the
orbital angular frequency of the binary. The tidal torque exerted on the star by
the companion is \citep{kushnir}:
\begin{align}
    \tau ={}& \hat{\tau}(\omega) \sgn\p{1 - \frac{2\Omega_{\rm s}}{\omega}}
        \abs{1 - \frac{2\Omega_{\rm s}}{\omega}}^{8/3}
            \label{eq:kushnir_torque},\\
    \hat{\tau}(\omega) ={}& \frac{GM_2^2r_c^5}{a^6}
        \p{\frac{\omega}{\sqrt{GM_c/r_c^3}}}^{8/3}
        \s{\frac{r_c}{g_c}\p{\rd{N^2}{\ln r}}_{r = r_c}}^{-1/3}\nonumber\\
            &\times \frac{\rho_c}{\bar{\rho}_c}
                \p{1 - \frac{\rho_c}{\bar{\rho}_c}}^2
                \s{\frac{3}{2}\frac{3^{2/3}\Gamma^2(1/3)}{5 \cdot
                6^{4/3}} \frac{3}{4\pi}\alpha^2},\nonumber\\
        \equiv{}& \beta_2\frac{GM_2^2r_c^5}{a^6}
            \p{\frac{\omega}{\sqrt{GM_c/r_c^3}}}^{8/3}
            \frac{\rho_c}{\bar{\rho}_c} \p{1 - \frac{\rho_c}{\bar{\rho}_c}}^2.
\end{align}
Here, $\Omega_{\rm s}$ is the spin of the massive star, $N$ is the Br\"unt-Vaisala
frequency, $r$ is the radial coordinate within the star, and $r_c$, $M_c$,
$g_c$, $\rho_c$, $\bar{\rho}_c$ are the radius of the RCB, the mass contained
within the convective core, the gravitational acceleration at the RCB, the
stellar density at the RCB, and the average density of the convective core,
respectively. $\alpha$ and $\beta_2$ are numerical constants defined by
\citet{kushnir}, where $\beta_2 \approx 1$ is a good approximation for a large
range of stellar models. In Eq.~\eqref{eq:kushnir_torque}, we have written the
terms such that $\hat{\tau}$ contains all the spin-independent terms.

\subsection{Perturbation from an Eccentric Companion: Hansen Coefficients}

Separately, we review the general procedure for calculating tidal dissipation
due to an eccentric perturber. The gravitational potential of an eccentric
companion can be decomposed at quadrupolar order as a sum over circular orbits
\citep{sl,vlf}:
\begin{align}
    U &= \sum\limits_m U_{2m} \p{\vec{r}, t},\label{eq:u_ecc}\\
    U_{2m}\p{\vec{r}} &= -\frac{GM_2 W_{2m} R^2}{D(t)^3}
            e^{-imf(t)} Y_{2m}(\theta, \phi),\nonumber\\
        &= -\frac{GM_2W_{2m}R^2}{a^3}Y_{2m}\p{\theta, \phi}
            \sum\limits_{N = -\infty}^\infty F_{Nm}e^{-iN\Omega t},
            \label{eq:hansen_decomp}\\
    F_{Nm} &= \frac{1}{\pi}\int\limits_{0}^{\pi}
        \frac{\cos\s{N\p{E - e\sin E} - mf(E)}}
            {\p{1 - e\cos E}^2}\;\mathrm{d}E.
\end{align}
Here, $R$ is the radius of the star, $W_{2 \pm 2} = \sqrt{3\pi/10}$, $W_{2 \pm
1} = 0$, $W_{20} = -\sqrt{\pi / 5}$, $D(t)$ is the instantaneous distance
between the star and companion, $f$ is the true anomaly, $Y_{lm}$ denote the
spherical harmonics, and $\Omega$ is the mean motion of the companion. $F_{Nm}$
denote are the \emph{Hansen coefficients} for $l = 2$ \citep[also denoted
$X^n_{2m}$ in][]{murray1999solar}.

By considering the effect of each summand in Eq.~\eqref{eq:hansen_decomp}, the
total torque on the star, energy transfer in the inertial frame, and heating in
the star's corotating frame can be obtained \citep{sl, vlf}:
\begin{align}
    \tau ={}& \sum\limits_{N = -\infty}^\infty F_{N2}^2
        \hat{\tau}\p{\omega =
        N\Omega - 2\Omega_{\rm s}},\label{eq:tau_sum}
        \\
    \dot{E}_{\rm in} ={}&
        \frac{1}{2}\sum\limits_{N = -\infty}^\infty\Bigg\{
            \p{\frac{W_{20}}{W_{22}}}^2 N\Omega F_{N0}^2
            \hat{\tau}\p{\omega = N\Omega}\nonumber\\
            &+ N\Omega F_{N2}^2 \hat{\tau}\p{\omega =
            N\Omega - 2\Omega_{\rm s}}\Bigg\},\label{eq:edot_in}\\
    \dot{E}_{\rm rot} ={}& \dot{E}_{\rm in} - \Omega_{\rm s} \tau \label{eq:edot_rot},
\end{align}
where $\hat{\tau}(\omega)$ is the torque exerted by a perturber on a circular
trajectory with orbital frequency $\omega$. Our notation differs from that of
\citet{vlf}, where we write $\hat{\tau}(\omega) = T_0 \sgn(\omega)
\hat{F}(\abs{\omega})$ in their notation, to easier incorporate the results of
the previous section.

\subsection{Objective of This Paper}\label{ss:objective}

The objective of this paper is to study the effect of dynamical tides in an
eccentric MS-CO binary. First, we compute the tidal torque by substituting the
torque due to a CO on a circular orbit [Eq.~\eqref{eq:kushnir_torque}] into the
summation Eq.~\eqref{eq:tau_sum}, obtaining
\begin{equation}
    \tau = \sum_{N = -\infty}^{N = \infty} F_{N2}^2 \hat{\tau}(r_c)
        \mathrm{sgn}\left(N - \frac{2\Omega_{\rm s}}{ \Omega}\right) \left|N - \frac{2
        \Omega_{\rm s}}{\Omega}\right|^{8/3}.\label{eq:tau_explicit_sum}
\end{equation}
The energy transfer rate in the inertial frame is similarly obtained by
substituting Eq.~\eqref{eq:kushnir_torque} into Eq.~\eqref{eq:edot_in}
\begin{align}
    \dot{E}_{\rm in} ={}& \frac{\hat{\tau}(r_c, \Omega)}{2}
        \sum_{N = -\infty}^{\infty} \Bigg[
            N\Omega F_{N2}^2 \mathrm{sgn}\left(N - 2\Omega_{\rm s} / \Omega\right)
                    \left|N - 2 \Omega_{\rm s} / \Omega\right|^{8/3}\nonumber\\
            &+ \left(\frac{W_{20}}{W_{22}}\right)^2 \Omega
                    F_{N0}^2 |N|^{11/3}
            \Bigg].\label{eq:ein_explicit_sum}
\end{align}
These two expressions give the spin synchronization timescale of the star as
well as the inspiral time of the binary due to dynamical tides. While exact,
these expressions are difficult to evaluate for larger eccentricities, where one
often must sum hundreds or thousands of terms, each of which has a different
$F_{Nm}$. In the subsequent sections, our objective is to obtain closed-form
approximations to Eqs.~(\ref{eq:tau_explicit_sum}--\ref{eq:ein_explicit_sum}).

\section{Approximating Hansen Coefficients}\label{s:hansens}

Recall that the Hansen coefficients are defined as the Fourier series
coefficients of part of the companion's gravitational potential
\begin{equation}
    \frac{a^3}{D(t)^3} e^{-imf} = \sum\limits_{N = -\infty}^\infty
        F_{Nm} e^{-iN\Omega t}.\label{eq:hansen_series}
\end{equation}
Observe that $F_{(-N)m} = F_{N(-m)}$, so we will only study the Hansen
coefficient behavior for $m \geq 0$.

\subsection{$m=2$ Hansen Coefficient Behavior at High Eccentricity}

We first consider the case where $m = 2$. We give an example of the behavior of
the $F_{N2}$ for $e = 0.9$ in Fig.~\ref{fig:hansens}, and make the following
observations:
\begin{itemize}
    \item $F_{N2}$ is much larger for $N \geq 0$ than for $N < 0$.

    \item For $N \geq 0$, $F_{N2}$ has only one substantial peak.
        The only characteristic scale for $N$ is the pericenter harmonic
        \begin{equation}
            N_{\rm p} \equiv \lfloor \Omega_{\rm p} / \Omega\rfloor,
        \end{equation}
        where $\Omega_{\rm p}$ is the pericenter frequency, defined by
        \begin{equation}
            \Omega_{\rm p} \equiv \Omega \frac{\sqrt{1 + e}}{\p{1 - e}^{3/2}}.
        \end{equation}
        Thus, we expect that the peak of the $F_{N2}$ should occur at $\sim
        N_{\rm p}$, which is indeed the case.

    \item Since the left hand side of Eq.~\eqref{eq:hansen_series} is smooth,
        the Fourier coefficients must fall off exponentially for $N \gg N_{\rm
        p}$ by the Paley-Wiener theorem.

    \item Since there are no characteristic timescales between $\Omega$ and
        $\Omega_{\rm p}$, we anticipate the Hansen coefficients must be scale
        free between $N = 1$ and $N_{\rm p}$, i.e.\ a power law.
\end{itemize}
Motivated by these considerations, we assume the Hansen coefficients can be
approximated by
\begin{equation}
    F_{N2} \approx
    \begin{cases}
        C_2 N^{p}e^{-N/\eta_2} & N \geq 0,\\
        0 & N < 0,
    \end{cases}\label{eq:fn2_fit}
\end{equation}
for some fitting coefficients $C_2$, $p$, and $\eta_2$. By performing fits to
$F_{N2}$, we found that $p \approx 2$ is relatively constant for modest-to-large
large eccentricities. This is expected, as the left hand side of
Eq.~\eqref{eq:hansen_series} resembles the second derivative of a Dirac delta
for $N \lesssim N_{\rm p}$ when the eccentricity is substantial\footnote{
Note that the left hand side of Eq.~\eqref{eq:hansen_series} is
sharply peaked about $t = 0$, is periodic with period $P = 2\pi / \Omega$, and
has zero derivative three times every period (at $t = \epsilon$, $t = P / 2$,
and $t = P - \epsilon$ for some small $\epsilon \sim 1 / \Omega_{\rm p}$). This
characteristics imply that it can be well approximated by the second derivative
of a Gaussian with width $\sim 1 / \Omega_{\rm P}$. For frequencies $\Omega
\lesssim \Omega_{\rm p}$, this Gaussian is further well approximated by a Dirac
delta, which has a flat Fourier transform ($\propto N^0$). Since time
differentiation multiplies by $N$ in frequency space, we see indeed that $F_{N2}
\propto N^2$ for $N \lesssim N_{\rm p}$.}. For the remainder of this work, we
take $p = 2$ to be fixed.

To constrain the remaining two free parameters $\eta_2$ and $C_2$ the
normalization, we use the well known Hansen coefficient moments
\citep{hut81}
\begin{align}
    \sum\limits_{N = -\infty}^\infty F_{N2}^2
        &= \frac{1 + 3e^2 + 3e^4/8}{\p{1 - e^2}^{9/2}}
            \equiv \frac{f_5}{\p{1 - e^2}^{9/2}},\\
    \sum\limits_{N = -\infty}^\infty F_{N2}^2N
        &= \frac{2}{\p{1 - e^2}^6}\p{1 + \frac{15e^2}{2}
            + \frac{45 e^4}{8} + \frac{5e^6}{16}}\nonumber\\
        &\equiv \frac{2f_2}{\p{1 - e^2}^6}.
\end{align}
This fixes
\begin{align}
    \eta_2 &= \frac{4f_2}{5f_5\p{1 - e^2}^{3/2}},\label{eq:eta2}\\
    C_2^2\eta_2^5 &= \frac{4f_5}{3\p{1 - e^2}^{9/2}}.\label{eq:C2}
\end{align}
This parameterization is compared to explicit evaluation of the Hansen
coefficients in Fig.~\ref{fig:hansens}. The agreement is particularly striking
as this parameterization uses zero free parameters.
\begin{figure}
    \centering
    \includegraphics[width=\columnwidth]{../../scripts/eccentric_tides/hansens.png}
    \caption{Plot of Hansen coefficients $F_{N2}$ for $e = 0.9$, where dotted
    line denotes negative values. The green line is the parameterization
    of Eq.~\eqref{eq:fn2_fit} with $\eta_2$ and $C_2$ given by
    Eqs.~(\ref{eq:eta2}--\ref{eq:C2}). }\label{fig:hansens}
\end{figure}

\subsection{$m = 0$ Hansen Coefficient Behavior at High Eccentricity}

We now turn to the case where $m = 0$ in Eq.~\eqref{eq:hansen_series}. These
coefficients also only have one characteristic scale in harmonic space ($N_{\rm
p}$), but are also symmetric. Therefore, the natural ansatz is of form
\begin{equation}
    F_{N0} = C_0 e^{-\abs{N} / \eta_0}.\label{eq:fn0_fit}
\end{equation}
The two free parameters $C_0$ and $\eta_0$ are again constrained by the well known
moments \citep{hut81}
\begin{align}
    \sum\limits_{N = -\infty}^\infty F_{N0}^2 &= \frac{f_5}{\p{1 - e^2}^{9/2}}
        ,\\
    \sum\limits_{N = -\infty}^\infty F_{N0}^2 N^2
        &= \frac{9e^2}{2\p{1 - e^2}^{15/2}}
            f_3.
\end{align}
This fixes
\begin{align}
    \eta_0^2 &= \frac{9e^2f_3}{\p{1 - e^2}^{3}f_5},\label{eq:eta0}\\
    C_0^2\eta_0 &= \frac{f_5}{\p{1 - e^2}^{9/2}}.\label{eq:C0}
\end{align}

\begin{figure}
    \centering
    \includegraphics[width=0.9\columnwidth]{../../scripts/eccentric_tides/hansens/hansens0_90.png}
    \caption{Plot of $F_{N0}$ for $e = 0.9$. The green line denotes the
    parameterization of Eq.~\eqref{eq:fn0_fit} with $\eta_0$ and $C_0$ given by
    Eqs.~(\ref{eq:eta0}--\ref{eq:C0}).}\label{fig:fn0_fit}
\end{figure}

\section{Evaluating Torque and Energy Transfer}\label{s:eval}

We now return to the summations in Section~\ref{ss:objective} and simplify them
using the parameterized Hansen coefficients from the previous section.

\subsection{Tidal Torque}

We return now to Eq.~\eqref{eq:tau_explicit_sum}. By replacing the $F_{N2}$ with
the approximation Eq.~\eqref{eq:fn2_fit} and replacing the sum with an integral,
we obtain
\begin{equation}
    \tau = \hat{\tau} \int_0^\infty C_2^2 N^4 e^{-2N / \eta_2}
        \sgn\left(N - 2\Omega_{\rm s} / \Omega\right) \left|N - 2 \Omega_{\rm s} /
            \Omega\right|^{8/3}\;\mathrm{d}N.\label{eq:tau_int}
\end{equation}

This can be further simplified. Define $N_{\max} = 10\eta_2/3$ to be the $N$ for
which the integrand in Eq.~\eqref{eq:tau_int} is maximized if
$\Omega_{\rm s} = 0$ (if $\Omega_{\rm s}$ is large, the integrand is maximized
at $N = 2\eta_2 \simeq N_{\max}$). We first consider the high-spin limit where
$\abs{\Omega_{\rm s}} \gg N_{\max}\Omega / 2$. In this limit,
Eq.~\eqref{eq:tau_explicit_sum} can be evaluated directly with the known Hansen
coefficient moments, which gives
\begin{equation}
    \lim_{\Omega_{\rm s} \to \infty} \tau = -\hat{\tau} \sgn (\Omega_{\rm s})\;\left|2
        \Omega_{\rm s} / \Omega\right|^{8/3} \frac{f_5}{(1 -
        e^2)^{9/2}}.\label{eq:tau_highspin}
\end{equation}

Next, we consider the low-to-moderate spin regime $2\Omega_{\rm s}/\Omega \lesssim
N_{\max}$. To accommodate for moderate spins, we assume that $N -
2\Omega_{\rm s}/\Omega$ can be approximated by
\begin{equation}
    N - 2\Omega_{\rm s} / \Omega \simeq \frac{N}{N_{\max}}
        \left(N_{\max} - \frac{2\gamma_\tau
        \Omega_{\rm s}}{\Omega}\right)\label{eq:nmax_ansatz},
\end{equation}
for some free parameter $\gamma_\tau$. Using this prescription,
Eq.~\eqref{eq:tau_int} can be integrated analytically, since
$\int\limits_0^\infty x^pe^{-x}\;\mathrm{d}x = \Gamma(p - 1)$\footnote{The key
to the success of our approach is that sums of form $\sum\limits_{n =
-\infty}^\infty F_{N2}^2 N^p$ can be evaluated for non-integer $p$ in terms of
$\Gamma$, which is not possible with the traditional techniques used to evaluate
these sums for integer $p$.}. After integrating, $\gamma_\tau$ is constrained by
requiring our expression reproduce the high spin limit
[Eq.~\eqref{eq:tau_highspin}] when taking $\abs{\Omega_{\rm s}} \to \infty$.
Then, the torque becomes
\begin{align}
    \tau &= \hat{\tau} \frac{f_5 (\eta_2/2)^{8/3}}{(1 - e^2)^{9/2}}
        \;\mathrm{sgn}\left(1 - \gamma_\tau\frac{\Omega_{\rm s}}{\eta_2\Omega}\right)
            \left|1 - \gamma_\tau\frac{\Omega_{\rm s}}{\eta_2\Omega}\right|^{8/3}
            \frac{\Gamma(23/3)}{4!},\label{eq:tau_approx}\\
    \gamma_\tau &= 4\p{\frac{4!}{\Gamma(23/3)}}^{3/8} \approx 0.691.
\end{align}
See Figs.~\ref{fig:totals_ecc0} and~\ref{fig:totals_s} for the accuracy of this
prediction. As expected from the construction of this approximation, both the
low and high spin limits are well captured, and the scaling for intermediate
spins is also somewhat accurate.
\begin{figure}
    \centering
    \includegraphics[width=\columnwidth]{../../scripts/eccentric_tides/1totals_ecc_0.png}
    \includegraphics[width=\columnwidth]{../../scripts/eccentric_tides/1totals_ecc_400.png}
    \caption{Tidal torque on a non-rotating (top) and rapidly rotating
    ($\Omega_{\rm s} / \Omega = 400$; bottom) star with a companion having
    orbital eccentricity $e$. Blue plus signs represent explicit summation of
    Eq.~\eqref{eq:tau_explicit_sum}, blue crosses are evaluated using the
    integral approximation Eq.~\eqref{eq:tau_int}, and the green dashed line
    is Eq.~\eqref{eq:tau_approx}.}\label{fig:totals_ecc0}
\end{figure}
\begin{figure}
    \centering
    \includegraphics[width=\columnwidth]{../../scripts/eccentric_tides/1totals_s_0_9.png}
    \caption{Tidal torque as a function of spin for a highly eccentric $e = 0.9$
    companion. Pluses represent direct summation of Hansen coefficients, crosses
    represent the integral approximation, and solid lines represent the
    analytic closed form. Blue [red] means positive [negative]
    torque on the star.}\label{fig:totals_s}
\end{figure}

\subsection{Pseudosynchronization}

Eq.~\eqref{eq:tau_approx} shows that there is a single $\Omega_{\rm s}$ for which the
torque $\tau$ vanishes. This is the pseudosynchronization spin frequency, given
by
\begin{equation}
    \frac{\Omega_{\rm s, ps}}{\Omega} =
        \frac{\eta_2}{\gamma_\tau} = \frac{4f_2}{5\gamma_\tau f_5\p{1 -
        e^2}^{3/2}}.\label{eq:eta2_691}
\end{equation}
This has the expected scaling $\Omega_{\rm s, ps} \propto (1 - e^2)^{-3/2}$,
but is more accurate than $\Omega_{\rm s, ps} \simeq \Omega_{\rm p}$, as seen
in Fig.~\ref{fig:pseudosync}. By comparison, in standard weak friction theory of
equilibrium tides, the pseudo-synchronized rotation rate is given by
\citep{alexander73, hut81}
\begin{equation}
    \frac{\Omega_{\rm s, ps}^{\rm (Eq)}}{\Omega} = \frac{f_2}{f_5\p{1 -
        e^2}^{3/2}}.
\end{equation}
This differs from our Eq.~\eqref{eq:eta2_691} by a factor of $1.15$.
\begin{figure}
    \centering
    \includegraphics[width=\columnwidth]{../../scripts/eccentric_tides/1pseudosynchronous.png}
    \caption{Pseudosynchronization frequency. Integral form is obtained by
    performing root finding using the integral form for the torque. We see that
    Eq.~\eqref{eq:eta2_691} is a very good approximation.}\label{fig:pseudosync}
\end{figure}

Note that, very near the pseudosynchronized spin frequency,
Eq.~\eqref{eq:tau_approx} predicts that $\rdil{\tau}{\Omega_{\rm s}} \approx 0$.
This is inaccurate and is an artifact of our factorization ansatz in
Eq.~\eqref{eq:nmax_ansatz}.

\subsection{Closed Form for Energy Transfer}

We now simplify Eq.~\eqref{eq:edot_in} and replace $F_{N2}$ and $F_{N0}$ with
their approximations Eqs.~(\ref{eq:fn2_fit},~\ref{eq:fn0_fit}) to obtain
\begin{align}
    \dot{E}_{\rm in} ={}&
        \frac{\hat{\tau} \Omega}{2} \int\limits_0^\infty \Big[
            C_2^2 N^5 e^{-2N/\eta_2} \mathrm{sgn}\left(N - 2\Omega_{\rm s} /
                \Omega\right) \left|N - 2 \Omega_{\rm s} /
                \Omega\right|^{8/3}\nonumber\\
            &+ 2 \p{\frac{W_{20}}{W_{22}}}^2 C_0^2 e^{-2N / \eta_0} N^{11/3}
        \Big]\;\mathrm{d}N.\label{eq:ein_int}
\end{align}
We evaluate the $m = 2$ and $m = 0$ components of this expression separately.

We first examine the $m = 2$ contribution using a very similar procedure to what
we used for the torque. Again, we define $N_{\max} = 23\eta_2 / 6$ the $N$ for
which the integrand is maximized. The high spin limit $\Omega_{\rm s} \gg
N_{\max}\Omega / 2$ comes out to be
\begin{equation}
    \lim_{\Omega_{\rm s} \to \infty} \dot{E}_{\rm in}^{(m=2)} =
        -\frac{\hat{\tau}\Omega}{2} \; \mathrm{sgn}(\Omega_{\rm s})|2\Omega_{\rm
        s}/\Omega|^{8/3} \frac{2f_2}{(1 - e^2)^6}.\label{eq:ein_highs}
\end{equation}
We then consider the low-to-moderate spin limit and factorize
\begin{equation}
    N - 2\Omega_{\rm s} / \Omega \simeq \frac{N}{N_{\max}}
        \left(N_{\max} - \frac{2\gamma_E
        \Omega_{\rm s}}{\Omega}\right)\label{eq:nmax_ansatz},
\end{equation}
where $\gamma_E$ is a free parameter. This lets us integrate
Eq.~\eqref{eq:ein_int} analytically, and we can constrain $\gamma_E$ by
requiring the integral agree with Eq.~\eqref{eq:ein_highs}. We obtain that
\begin{align}
    \dot{E}_{\rm in}^{(m=2)}
        ={}& -\frac{\hat{\tau}\Omega f_5 (\eta_2/2)^{11/3}}{2(1-e^2)^{9/2}}
            \frac{\Gamma(26/3)}{4!}\nonumber\\
        &\times \mathrm{sgn}\p{1 - \gamma_E\frac{\Omega_s}{\eta_2 \Omega}}
            \left|1 - \gamma_E\frac{\Omega_s}{\eta_2 \Omega}\right|^{8/3},\\
    \gamma_E ={}& \p{\frac{5! 2^{16/3}}{\Gamma\p{26/3}}}^{3/8}
        \approx 0.5886.
\end{align}

The $m = 0$ contribution to Eq.~\eqref{eq:ein_int} is much more straightforward
and can be directly integrated using the parameterization
Eq.~\eqref{eq:fn0_fit}. We finally obtain the total energy transfer rate
\begin{align}
    \dot{E}_{\rm in} ={}& -\frac{\hat{\tau} \Omega}{2}\Bigg[\frac{f_5
        (\eta_2/2)^{11/3}}{(1-e^2)^{9/2}} \frac{\Gamma(26/3)}{4!}\nonumber\\
        &\times \mathrm{sgn}\p{1 - \gamma_E\frac{\Omega_s}{\eta_2 \Omega}}
            \left|1 - \gamma_E\frac{\Omega_s}{\eta_2 \Omega}\right|^{8/3}
            \nonumber\\
        &+
    \frac{f_5 \Gamma(14 / 3)}{(1 - e^2)^{10}} \left(\frac{3}{2}\right)^{8/3}
            \left(\frac{e^2 f_3}{f_5}\right)^{11/6}\Bigg].
            \label{eq:total_heating}
\end{align}

We make plots in the two $\Omega_{\rm s}$ regimes as a function of eccentricity in
Fig.~\ref{fig:e0} and~\ref{fig:e_spin}. Agreement is good again.
\begin{figure}
    \centering
    \includegraphics[width=\columnwidth]{../../scripts/eccentric_tides/1totals_e_0.png}
    \includegraphics[width=\columnwidth]{../../scripts/eccentric_tides/1totals_e_400.png}
    \caption{Plot of $\dot{E}_{\rm in}$ for a non-rotating (top) and rapidly
    rotating ($\Omega_{\rm s} / \Omega = 400$; bottom) star. Blue pluses
    represent explicit summation of the Hansen coefficients, crosses the
    integral form Eq.~\eqref{eq:ein_int}, and the green dashed line the closed
    form Eq.~\eqref{eq:total_heating}.}\label{fig:e0}
\end{figure}
\begin{figure}
    \centering
    \includegraphics[width=\columnwidth]{../../scripts/eccentric_tides/1totals_NRG_e_0_9.png}
    \caption{$\dot{E}_{\rm in}$ as a function of spin for a highly eccentric $e
    = 0.9$ companion. Pluses represent direct summation of Hansen coefficients,
    crosses represent the integral approximation, and solid lines represent
    the analytic closed form. Blue [red] means positive [negative]
    energy transfer into the stellar spin.}\label{fig:e_spin}
\end{figure}

\section{Example System: J0045+7319}\label{s:j00457319}

We study J0045+7319 \citep{bell1995psr}.

We can arrive at an upper bound for the $\dot{P}$ by setting the spin frequency
equal to the breakup frequency. Then, taking the correct parameters and
evaluating $\dot{E}_{\rm in}$, we obtain
\begin{equation}
    \frac{\dot{P}}{2\pi} \lesssim
        -\frac{3}{(1 + q)^2}\beta_2 \left(\frac{r_c}{a}\right)^5
        \frac{\rho_c}{\bar{\rho}_c} \left(1 -
        \frac{\rho_c}{\bar{\rho}_c}\right)^2 2^{8/3}\frac{f_2}{(1 - e^2)^6}.
\end{equation}
This comes out to be $r_c \gtrsim 1.799R_{\odot}$ for J0045+7319, larger than
the prior values in the literature and much larger than MESA models.

\subsection{Stellar Structure Simulations}

In previous literature, the primary star was initially taken to have $M_1 =
8.8M_{\odot}$ \citep{kumar1998,lai1996}, by taking $M_2 \simeq 1.4M_{\odot}$
characteristic mass for NSs and multiplying by the observed $q$. A later, but
still very old, study proposed $M_1 \approx 10M_{\odot}$ by comparison with
stellar models. Using MESA to generate updated stellar models (TODO all
citations), we compute an updated $M_1$ estimate, giving detailed stellar
structure measurements.

At the level of approximation of this paper, our procedure is as follows: we
use a few different values of convective overshooting and mixing, and
metallicities, and compute stellar structures for both both non-rotating and
highly rotating $\sim 0.95\Omega_{\rm s, c}$ stars. We find that, in general, to
match the observed $T, L$ we must let the star evolve to $\sim 80\%$ of the way
to complete core hydrogen depletion. We then sample a range of stellar masses,
and find the stellar mass that best reproduces the observed $T, L$.

For all of these systems, $r_c \lesssim R_{\odot}$, in tension with our bound
above.

\section{Conclusion and Discussion}\label{s:disc}

The primary results of the paper is Eq.~\eqref{eq:tau_approx}, shown in
Figs.~\ref{fig:totals_ecc0} and~\ref{fig:totals_s} to be accurate across a range
of spins and eccentricities. The energy dissipation rate is also computed using
similar techniques and show good agreement (see Figs.~\ref{fig:e0}
and~\ref{fig:e_spin}).

\begin{itemize}
    \item Thanks to some references (Barker \& Ogilvie, my work), there seems to
        be some evidence that hydrodynamic wave breaking could cause all IGW to
        break and not reflect, once the pericenter wave reaches nonlinear
        amplitudes.

    \item As noted in the text, the approximate forms enforce
        $\rdil{\tau}{\Omega_{\rm s}} = 0$, which the actual torque does not satisfy.
        This introduces some slight errors in the exact value of the torque very
        near pseudosynchronization.
\end{itemize}

\section{Acknowledgements}

We thank Michelle Vick and Christopher O'Connor for fruitful discussions. YS is
supported by the NASA FINESST grant 19-ASTRO19-0041.% chktex 8

\bibliographystyle{mnras}
\bibliography{Su_eccentric_tides}

% \clearpage
% \onecolumn

\bsp
\label{lastpage} % chktex 24
\end{document}
