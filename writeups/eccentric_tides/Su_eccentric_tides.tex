% for i in hansens hansens/hansens0_90 1totals_ecc_0 1totals_ecc_400 1totals_s_0_9 1pseudosynchronous 1totals_e_0 1totals_e_400 1totals_NRG_e_0_9 1_7319_disps; do cp ../../scripts/eccentric_tides/$i.png .; done
% chktex-file 8
    \documentclass[
        fleqn,
        usenatbib,
        % referee,
    ]{mnras}
    \usepackage{
        amsmath,
        amssymb,
        newtxtext,
        newtxmath,
        graphicx,
        ae, aecompl,
        booktabs,
        xcolor,
    }
    \usepackage[T1]{fontenc}

    \newcommand*{\rd}[2]{\frac{\mathrm{d}#1}{\mathrm{d}#2}}
    \newcommand*{\pd}[2]{\frac{\partial#1}{\partial#2}}
    \newcommand*{\rdil}[2]{\mathrm{d}#1 / \mathrm{d}#2}
    \newcommand*{\pdil}[2]{\partial#1 / \partial#2}
    \newcommand*{\rtd}[2]{\frac{\mathrm{d}^2#1}{\mathrm{d}#2^2}}
    \newcommand*{\ptd}[2]{\frac{\partial^2 #1}{\partial#2^2}}
    \newcommand*{\md}[2]{\frac{\mathrm{D}#1}{\mathrm{D}#2}}
    \newcommand*{\pvec}[1]{\vec{#1}^{\,\prime}}
    \newcommand*{\svec}[1]{\vec{#1}\;\!}
    \newcommand*{\bm}[1]{\boldsymbol{\mathbf{#1}}}
    \newcommand*{\uv}[1]{\hat{\bm{#1}}}
    \newcommand*{\ang}[0]{\;\text{\AA}}
    \newcommand*{\mum}[0]{\;\upmu \mathrm{m}}
    \newcommand*{\at}[1]{\left.#1\right|}
    \newcommand*{\bra}[1]{\left<#1\right|}
    \newcommand*{\ket}[1]{\left|#1\right>}
    \newcommand*{\abs}[1]{\left|#1\right|}
    \newcommand*{\ev}[1]{\langle#1\rangle}
    \newcommand*{\p}[1]{\left(#1\right)}
    \newcommand*{\s}[1]{\left[#1\right]}
    \newcommand*{\z}[1]{\left\{#1\right\}}
    \colorlet{Corr}{red}

    \DeclareMathOperator*{\sgn}{sgn}
    \DeclareMathOperator*{\argmax}{argmax}

\title[Eccentric Dynamical Tides]{Dynamical Tides in Eccentric, Massive Stellar
Binaries}
\author[Y. Su, D. Lai.]{
Yubo Su$^1$,
Dong Lai$^1$
\\
$^1$ Cornell Center for Astrophysics and Planetary Science, Department of
Astronomy, Cornell University, Ithaca, NY 14853, USA
}

\date{Accepted XXX\@. Received YYY\@; in original form ZZZ}

\pubyear{2020}

\begin{document}\label{firstpage}
\pagerange{\pageref{firstpage}--\pageref{lastpage}}
\maketitle

\begin{abstract}
    The long-term orbital evolution of binaries in which at least one component
    is a massive, main-sequence star is predominantly due to dynamical tides.
    However, most understanding of dynamical tides is only applicable when the
    binary is nearly circular. In this work, we study the effect of dynamical
    tides in massive stellar binaries with large eccentricities and derive
    analytical expressions for the orbital decay, circularization, and spin
    synchronization rates. We apply our results to the radio pulsar J0045-7319,
    which has a massive B star companion and a rapidly decreasing orbital
    period. In order to reproduce the measured orbital decay via dynamical
    tides, we find that the core is likely rotating significantly faster than
    the measured surface rotation rate. This tentatively suggests a long
    core-envelope coupling timescale in the B star in J0045-7319. Our analytical
    approach can also be applied to other scenarios of tidal dissipation in
    eccentric binaries.
\end{abstract}

\begin{keywords}
keywords
\end{keywords}

\section{Introduction}

% 2nd para:  The original expression by Zahn (1975) involved unknown function that is not conveninet.....  More more transparent derivation of the tidal torque was goven by Kushnor et al...    In Kushnir et al (2017b) the authors use the new expression to study tidal synchrnization in ... systems (progentors ot merging BH binaries...

% 3rd para: These previous works all apply to circular binaries. However, massive stars can be found in high-eccentrcity systems. A common example is MS star + NS systems. The NS is formed with large kick, giving rise to an high eccentricty binary --- several such high-e systems have been discovered...
% An important issue is to udnerstand whether such hogh-e systemc an be circularizated prior to mass transfer or common-enevelope phase... (Vigna-Gomez...2020)..

% 4th para: The purpose of this paper is to derive an analytical expression for ...
% In section 2, we....

The physics of tidal dissipation in massive, main-sequence (MS) stars (i.e.\
having a convective core and radiative envelope) under the gravitational
influence of a companion was first studied by \citet{zahn1975dynamical}
\citep[see also][]{savonije1983tidal, goldreich1989tidal}. The dominant
dissipation mechanism is through the \emph{dynamical tide}, in which the
time-dependent tidal potential of the companion excites internal gravity waves
at the radiative-convective boundary (RCB). As the wave propagates towards the
surface, its amplitude grows, and the wave dissipates efficiently
\citep{zahn1975dynamical, goldreich1989tidal, su2020}. The contribution due to
the \emph{equilibrium tide}, in which the tidal potential induces a static
deformation of the star that undergoes viscous dissipation \citep[see
e.g.][]{alexander73, hut81}, is expected to be subdominant to the dynamical tide
in stars with radiative envelopes \citep{zahn1977tidal, goldreich1989tidal}.
\textcolor{red}{(NB for Dong: \citet{vigna2020common} say that the dynamical
tide is less efficient in virtually all binaries, are there any recent
references for the strength of the dynamical tide in stars with radiative
envelopes?)}

The original expression describing the torque due to dynamical tides by
\citet{zahn1975dynamical} is very sensitive to the global properties of the
star. \citet{kushnir} present an updated derivation of the tidal torque
dependent only on the local stellar properties near the RCB, eliminating many
uncertainties from the Zahn's original expression. In
\citet{zaldarriaga2018expected}, the authors use the new expression to study
tidal synchronization in binaries consisting of a Wolf-Rayet star and a black
hole, progenitors to black hole binaries that may be observed by LIGO/VIRGO\@.

These previous works all apply to circular binaries. However, massive stars can
be found in high-eccentricity (high-e) systems, such as binaries consisting of
one MS star and one neutron star (NS). The NS is formed with a large kick,
giving rise to a high-e binary. Several such high-e systems have been discovered
\citep[e.g.][]{johnston1994radio, bell1995psr, champion2008eccentric}. An
important issue is to understand whether such high-e systems can circularize
prior to mass transfer or a common envelope phase \citep{vigna2020common}.

The purpose of this paper is to derive an analytical expression for the effect
of dynamical tides for high-e binaries with a massive MS star. In
Section~\ref{s:background}, we review the effect of dynamical tides in circular
binaries and existing techniques for studying high-e systems. In
Section~\ref{ss:objective}, we evaluate the effect of dynamical tides in
high-e systems containing a massive MS star, including the torque and orbital
decay rate. In Section~\ref{s:j00457319}, we apply our results to the pulsar-MS
binary J0045-7319, for which a non-zero orbital decay rate has been measured.
Finally, we summarize and discuss in Section~\ref{s:disc}.

\section{Dynamical Tides in Massive Stars}\label{s:background}

\subsection{Circular Orbit}\label{ss:2_circ}

We first review the case where the binary is circular. Let $M$ be the mass of
the MS star, $M_2$ the mass of the companion, $a$ the semimajor axis of the binary, and
$\Omega$ the mean motion of the binary. The tidal torque exerted on the star by
the companion is \citep{kushnir}:
\begin{align}
    T_{\rm circ}(\omega) ={}& T_0 \sgn\p{\omega}
        \Big|\,\frac{\omega}{\Omega}\,\Big|^{8/3} \label{eq:kushnir_torque},\\
    T_0 \equiv{}& \beta_2\frac{GM_2^2r_{\rm c}^5}{a^6}
            \p{\frac{\Omega}{\sqrt{GM_{\rm c}/r_{\rm c}^3}}}^{8/3}
            \frac{\rho_{\rm c}}{\bar{\rho}_{\rm c}} \p{1 - \frac{\rho_{\rm
            c}}{\bar{\rho}_{\rm c}}}^2,\\
    \beta_2 \equiv{}&
        \s{\frac{r_{\rm c}}{g_{\rm c}}
            \p{\rd{N^2}{\ln r}}_{r = r_{\rm c}}}^{-1/3}
                \s{\frac{3}{2}\frac{3^{2/3}\Gamma^2(1/3)}{5 \cdot
                6^{4/3}} \frac{3}{4\pi}\alpha^2}.
\end{align}
Here, $\omega \equiv 2\Omega - 2\Omega_{\rm s}$ is the tidal forcing frequency,
$\Omega_{\rm s}$ is the spin of the MS star, $N$ is the Br\"unt-Vaisala
frequency, $r$ is the radial coordinate within the star, and $r_{\rm c}$,
$M_{\rm c}$, $g_{\rm c}$, $\rho_{\rm c}$, and $\bar{\rho}_{\rm c}$ are the
radius of the RCB, the mass contained within the convective core, the
gravitational acceleration at the RCB, the stellar density at the RCB, and the
average density of the convective core respectively. $\alpha$ is a numerical
constant of order unity given by Eq.~(A32) of \citet{kushnir}, and $\beta_2
\approx 1$ for a large range of stellar models (Fig.~2 of \citealp{kushnir}). In
Eq.~\eqref{eq:kushnir_torque}, we have written the terms such that $T_0$
contains all the spin-independent terms.

Note that we use Eq.~\eqref{eq:kushnir_torque} from \citet{kushnir} instead of
the classic expression from \citet{zahn1975dynamical}, which is given by:
\begin{equation}
    T_{\rm circ}^{\rm (Zahn)}(\omega) = \frac{3}{2} \frac{GM^2 R^5}{a^6}E_2
        \omega^{8/3},
\end{equation}
where $M$ and $R$ are the mass and radius of the MS star and $E_2$ is a
numerical parameter obtained by integrating over the entire star. The fitting
formula $E_2 = 1.592 \times 10^{-9}\p{M / M_{\odot}}^{2.84}$ as given by
\citet{hurley2002evolution} is commonly used, which varies by many orders of
magnitude for different stars. Moreover, $T_{\rm circ}^{\rm (Zahn)}$ depends on
$M$ and $R$, properties of the entire star, when the tidal torque is entirely
generated at the RCB\@. For these reasons, the expression by \citet{kushnir} is
preferred.

\subsection{Eccentric Binaries}\label{ss:2_eccentric}

The gravitational potential of an eccentric companion at the quadrupole order
can be decomposed as a sum over circular orbits \citep[e.g.][]{sl,vlf}:
\begin{align}
    U &= \sum\limits_{m=-2}^2 U_{2m} \p{\vec{r}, t},\label{eq:u_ecc}\\
    U_{2m}\p{\vec{r}, t} &= -\frac{GM_2 W_{2m} r^2}{D(t)^3}
        Y_{2m}(\theta, \phi) e^{-imf\!\!\!\:(t)},\nonumber\\
        &= -\frac{GM_2W_{2m} r^2}{a^3}Y_{2m}\p{\theta, \phi}
            \sum\limits_{N = -\infty}^\infty \!\!F_{Nm}e^{-iN\Omega t}
            \label{eq:hansen_decomp}.
\end{align}
Here, the coordinate system is centered on the MS star, $(r, \theta, \phi)$ are
the radial, polar, and azimuthal coordinates of $\vec{r}$ respectively, $W_{2
\pm 2} = \sqrt{3\pi/10}$, $W_{2 \pm 1} = 0$, $W_{20} = -\sqrt{\pi / 5}$, $D(t)$
is the instantaneous distance to the companion, $f$ is the true anomaly,
and $Y_{lm}$ denote the spherical harmonics. $F_{Nm}$ denote the \emph{Hansen
coefficients} for $l = 2$ \citep[also denoted $X^n_{2m}$ in][]{murray1999solar},
which are the Fourier coefficients of the perturbing function, i.e.
\begin{equation}
    \frac{a^3}{D(t)^3} e^{-imf\!\!\!\:(t)} = \sum\limits_{N = -\infty}^\infty
        \!\!F_{Nm} e^{-iN\Omega t}.\label{eq:hansen_series}
\end{equation}
The $F_{Nm}$ can be written explicitly as an integral over the eccentric anomaly
\citep{murray1999solar, sl}:
\begin{equation}
    F_{Nm} = \frac{1}{\pi}\int\limits_{0}^{\pi}
        \frac{\cos\s{N\p{E - e\sin E} - mf(E)}}
            {\p{1 - e\cos E}^2}\;\mathrm{d}E.\label{eq:hansen_integral}
\end{equation}

By considering the effect of each summand in Eq.~\eqref{eq:u_ecc}, the total
torque on the star, energy transfer in the inertial frame, and heating in the
star's corotating frame can be obtained \citep{sl, vlf}:
\begin{align}
    T ={}& \sum\limits_{N = -\infty}^\infty F_{N2}^2
        T_{\rm circ}\p{N\Omega - 2\Omega_{\rm s}},\label{eq:tau_sum}
        \\
    \dot{E}_{\rm in} ={}&
        \frac{1}{2}\sum\limits_{N = -\infty}^\infty\Bigg\{
            \p{\frac{W_{20}}{W_{22}}}^2 N\Omega F_{N0}^2 T_{\rm circ}\p{N\Omega}
                \nonumber\\
            &+ N\Omega F_{N2}^2 T_{\rm circ}\p{N\Omega - 2\Omega_{\rm s}}
            \Bigg\},\label{eq:edot_in}\\
    \dot{E}_{\rm rot} ={}& \dot{E}_{\rm in} - \Omega_{\rm s} T
        \label{eq:edot_rot}.
\end{align}
Here, dots indicate time derivatives.

These can be used to express the binary orbital decay and circularization times
using:
\begin{align}
    \frac{\dot{a}}{a} &= -\frac{2a\dot{E}_{\rm in}}{GMM_2},
        \label{eq:dota}\\
    \frac{\dot{e}e}{1 - e^2} &= -\frac{a\dot{E}_{\rm in}}{GMM_2} +
        \frac{T}{L_{\rm orb}},\label{eq:dote}
\end{align}
where $L_{\rm orb} = MM_2 \s{Ga(1 - e^2) / (M + M_2)}^{1/2}$ is the orbital
angular momentum. The stellar synchronization time can also be computed assuming
that the star rotates rigidly:
\begin{equation}
    \dot{\Omega}_{\rm s}
        = \frac{T}{kMR^2},\label{eq:dots}
\end{equation}
where $kMR^2$ is the moment of inertia of the MS star.

\section{Analytic Evaluation of Tidal Torque and Energy Transfer}\label{ss:objective}

We can combine the results given in Sections~\ref{ss:2_circ}
and~\ref{ss:2_eccentric} to compute the torque and energy transfer rate due to
dynamical tides in an eccentric binary. The tidal torque is obtained by
evaluating Eq.~\eqref{eq:tau_sum} with the circular torque set to
Eq.~\eqref{eq:kushnir_torque}, giving:
\begin{equation}
    T = \sum_{N = -\infty}^{N = \infty} F_{N2}^2 T_0
        \,\mathrm{sgn}\left(N - \frac{2\Omega_{\rm s}}{ \Omega}\right) \left|N -
        \frac{2 \Omega_{\rm s}}{\Omega}\right|^{8/3}.\label{eq:tau_explicit_sum}
\end{equation}
The energy transfer rate in the inertial frame is obtained by evaluating
Eq.~\eqref{eq:edot_in} in the same way, giving:
\begin{align}
    \dot{E}_{\rm in} ={}& \frac{T_0}{2}
        \sum_{N = -\infty}^{\infty} \Bigg[
            N\Omega F_{N2}^2 \mathrm{sgn}\left(N - 2\Omega_{\rm s} / \Omega\right)
                    \left|N - 2 \Omega_{\rm s} / \Omega\right|^{8/3}\nonumber\\
            &+ \left(\frac{W_{20}}{W_{22}}\right)^2 \Omega
                    F_{N0}^2 |N|^{11/3}
            \Bigg].\label{eq:ein_explicit_sum}
\end{align}
These two expressions can be used to obtain the orbital decay, circularization, and
spin synchronization timescales using Eqs.~(\ref{eq:dota}--\ref{eq:dots}).

While exact, these two sums are difficult to evaluate for larger eccentricities,
where one often must sum hundreds or thousands of terms, each of which has a
different $F_{Nm}$. In the following sections, our objective is to obtain
closed-form approximations to
Eqs.~(\ref{eq:tau_explicit_sum}--\ref{eq:ein_explicit_sum}) when the
eccentricity is sufficiently large that the sums cannot be approximated by just
one or two terms.

\subsection{Approximating Hansen Coefficients}\label{ss:hansens}

Towards simplifying
Eqs.~(\ref{eq:tau_explicit_sum}--\ref{eq:ein_explicit_sum}), we seek tractable
approximations for both $F_{N2}$ and $F_{N0}$. Note that while the Hansen
coefficients can be evaluated using the integral expression
Eq.~\eqref{eq:hansen_integral}, this requires calculating a separate integral
for each $N$. Instead, it is more convenient to use the discrete Fourier
Transform of the left hand side of Eq.~\eqref{eq:hansen_series} to calculate
arbitrarily many $N$ at once \citep[as pointed out
by][]{correia2014deformation}. Since $F_{(-N)m} = F_{N(-m)}$, we will only study
the Hansen coefficient behavior for $m \geq 0$.

\subsubsection{$m=2$ Hansen Coefficients}

Figure~\ref{fig:hansens} shows the $F_{N2}$ when $e = 0.9$. First, we note that
$F_{N2}$ is much larger when $N \geq 0$ than for $N < 0$, so we focus on the
behavior for $N \geq 0$. Here, $F_{N2}$ has only one substantial peak. There are
only two characteristic frequency scales: $\Omega$ and $\Omega_{\rm p}$ the
pericenter frequency, defined by
\begin{equation}
    \Omega_{\rm p} \equiv \Omega \frac{\sqrt{1 + e}}{\p{1 - e}^{3/2}}.
        \label{eq:Wperi}
\end{equation}
For convenience, we also define $N_{\rm p}$ as the floor of $\Omega_{\rm p} /
\Omega$, i.e.
\begin{equation}
    N_{\rm p} \equiv \lfloor \Omega_{\rm p} / \Omega\rfloor,
\end{equation}
We find that the peak of the $F_{N2}$ occurs at $\sim N_{\rm p}$, the only
characteristic scale in $N$ over which $F_{N2}$ can vary. When $N \gg N_{\rm
p}$, the Fourier coefficients must fall off exponentially by the Paley-Wiener
theorem, as the left hand side of Eq.~\eqref{eq:hansen_series} is smooth
\citep[e.g.][]{stein2009real}. When instead $N \ll N_{\rm p}$, there are no
characteristic frequencies between $\Omega$ and $\Omega_{\rm p}$, so we expect
the Hansen coefficients to be free between $N = 1$ and $N_{\rm p}$, i.e.\ a
power law in $N$. The expected behaviors in both of these regimes are in
agreement with Fig.~\ref{fig:hansens}.

Motivated by these considerations, we assume the Hansen coefficients can be
approximated by a function of form:
\begin{equation}
    F_{N2} \approx
    \begin{cases}
        C_2 N^{p}e^{-N/\eta_2} & N \geq 0,\\
        0 & N < 0,
    \end{cases}\label{eq:fn2_fit}
\end{equation}
for some fitting coefficients $C_2$, $p$, and $\eta_2$. By performing fits to
$F_{N2}$, we found that $p \approx 2$ for substantial
eccentricities, and we take $p = 2$ to be fixed for the remainder of this
work\footnote{There is good reason to expect that $p = 2$ for $N \ll N_{\rm p}$
as long as the eccentricity is sufficiently large, as then the left hand side of
Eq.~\eqref{eq:hansen_series} resembles the second derivative of a Dirac delta
function within each orbital period: It is both sharply peaked about $t = 0$ and
has zero derivative three times every period (at $t = \epsilon$, $t = P / 2$,
and $t = P - \epsilon$ for some small $\epsilon \sim \Omega_{\rm p}^{-1}$).
These two characteristics describe the second derivative of a Gaussian with
width $\sim \Omega_{\rm p}^{-1}$. Then, for timescales longer than its width, a
Gaussian resembles a Dirac delta function, which has a flat Fourier spectrum
($\propto N^0$). Finally, since time differentiation multiplies by $N$ in
frequency space, the second derivative of a Gaussian has a Fourier spectrum
$\propto N^2$ for sufficiently small $N \lesssim N_{\rm p}$. As $F_{N2}$ is the
$N$th Fourier coefficient for a function resembling the second derivative of a
Dirac delta function for $N \lesssim N_{\rm p}$, we do indeed expect $F_{N2}
\propto N^2$ in this regime.}.

To constrain the remaining two free parameters $\eta_2$ and $C_2$ the
normalization, we use the well known Hansen coefficient moments
\citep{hut81}
\begin{align}
    \sum\limits_{N = -\infty}^\infty F_{N2}^2 &= \frac{f_5}{\p{1 - e^2}^{9/2}},
        \\
    f_5 &\equiv 1 + 3e^2 + \frac{3e^4}{8},\\
    \sum\limits_{N = -\infty}^\infty F_{N2}^2N
        &= \frac{2f_2}{\p{1 - e^2}^6},\\
    f_2 &\equiv 1 + \frac{15e^2}{2}
            + \frac{45 e^4}{8} + \frac{5e^6}{16}.
\end{align}
We have defined the common functions $f_2$ and $f_5$. This fixes
\begin{align}
    \eta_2 &= \frac{4f_2}{5f_5\p{1 - e^2}^{3/2}},\label{eq:eta2}\\
    C_2 &= \s{\frac{4f_5}{3\p{1 - e^2}^{9/2}\eta_2^5}}^{1/2}.\label{eq:C2}
\end{align}
Figure~\ref{fig:hansens} illustrates the aggrement of Eq.~\eqref{eq:fn2_fit}
using these two values of $\eta_2$ and $C_2$ with the numerical $F_{N2}$. The
good agreement is especially impressive as there are no fitting parameters in
Eq.~\eqref{eq:fn2_fit}, as $C_2$, $\eta_2$, and $p$ are all analytically
constrained. Finally, note that the maximum of the $F_{N2}$ occurs at $N =
\lfloor 2\eta_2 \rfloor$, and $2\eta_2 \propto \p{1 - e}^{3/2} \Omega_{\rm p}$,
so the maximum occurs at $N \sim N_{\rm p}$ as argued above.
\begin{figure}
    \centering
    \includegraphics[width=\columnwidth]{hansens.png}
    \caption{Plot of Hansen coefficients $F_{N2}$ for $e = 0.9$. The red circles
    denote negative $N$, while the black circles and crosses denote positive and
    negative $F_{N2}$. The blue line is the formula given by
    Eq.~\eqref{eq:fn2_fit} with $\eta_2$ and $C_2$ given by
    Eqs.~(\ref{eq:eta2}--\ref{eq:C2}). }\label{fig:hansens}
\end{figure}

\subsubsection{$m = 0$ Hansen Coefficients}

We now turn to the $m = 0$ Hansen coefficients, $F_{N0}$, which are shown in
Fig.~\ref{fig:fn0_fit}. We know that $F_{N0} = F_{(-N)0}$, so we consider only
$N \geq 0$. From the figure, we see that the $F_{N0}$ decay exponentially. There
is only one characteristic scale available for this decay, namely $N_{\rm p}$.
Therefore, we naturally assume the $F_{N0}$ coefficients can be approximated by
a function of form:
\begin{equation}
    F_{N0} = C_0 e^{-\abs{N} / \eta_0}.\label{eq:fn0_fit}
\end{equation}
The two free parameters $C_0$ and $\eta_0$ are constrained by the well known
moments \citep{hut81}
\begin{align}
    \sum\limits_{N = -\infty}^\infty F_{N0}^2 &= \frac{f_5}{\p{1 - e^2}^{9/2}}
        ,\\
    \sum\limits_{N = -\infty}^\infty F_{N0}^2 N^2
        &= \frac{9e^2}{2\p{1 - e^2}^{15/2}}
            f_3,\\
    f_3 &= \frac{1}{2} + \frac{15e^2}{8} + \frac{15 e^4}{16}
        + \frac{5e^6}{128}.
\end{align}
We have defined the common function $f_3$. This then requires
\begin{align}
    \eta_0 &= \s{\frac{9e^2f_3}{\p{1 - e^2}^{3}f_5}}^{1/2},\label{eq:eta0}\\
    C_0 &= \s{\frac{f_5}{\p{1 - e^2}^{9/2}\eta_0}}^{1/2}.\label{eq:C0}
\end{align}
Figure~\ref{fig:fn0_fit} illustrates the agreement of Eq.~\eqref{eq:fn0_fit}
using these two values of $\eta_0$ and $C_0$, where again good agreement is
observed.
\begin{figure}
    \centering
    \includegraphics[width=0.9\columnwidth]{hansens0_90.png}
    \caption{Plot of $F_{N0}$ (black circles) for $e = 0.9$. Since $F_{N0} =
    F_{(-N)0}$, we only show positive $N$. The blue line is given by
    Eq.~\eqref{eq:fn0_fit} with $\eta_0$ and $C_0$ given by
    Eqs.~(\ref{eq:eta0}--\ref{eq:C0}).}\label{fig:fn0_fit}
\end{figure}

\subsection{Approximate Expressions for Torque and Energy Transfer}\label{s:eval}

Having found good approximations for the Hansen coefficients, we now apply them
to simplify the formulas for the torque and the energy transfer rate in
Eqs.~(\ref{eq:tau_explicit_sum}--\ref{eq:ein_explicit_sum}).

\subsubsection{Tidal Torque}\label{ss:torque_eval}

To simplify the torque, given by Eq.~\eqref{eq:tau_explicit_sum}, we replace
$F_{N2}$ with Eq.~\eqref{eq:fn2_fit} and the sum with an integral, obtaining
\begin{equation}
    T \approx T_0 \int_0^\infty C_2^2 N^4 e^{-2N / \eta_2}
        \sgn\left(N - 2\Omega_{\rm s} / \Omega\right) \left|N - 2 \Omega_{\rm s} /
            \Omega\right|^{8/3}\;\mathrm{d}N.\label{eq:tau_int}
\end{equation}

This expression is already easier to evaluate than
Eq.~\eqref{eq:tau_explicit_sum}, but we can use further approximations to obtain
a closed form. We first analyze Eq.~\eqref{eq:tau_int} in the zero spin limit,
where it can be integrated analytically\footnote{The key to the success of our
approach is that sums of form $\sum_{n = -\infty}^\infty F_{N2}^2 N^p$ can be
approximated for non-integer $p$ in terms of $\Gamma$, since $\int_0^\infty
x^pe^{-x}\;\mathrm{d}x = \Gamma(p - 1)$. This is not possible with existing
analytical techniques.}, giving
\begin{align}
    \lim_{\Omega_{\rm s} \to 0} T &= T_0 \frac{f_5 (\eta_2/2)^{8/3}}{(1 -
        e^2)^{9/2}} \frac{\Gamma(23/3)}{4!}.\label{eq:tau_lowspin}
\end{align}
The top panel of Fig.~\ref{fig:totals_ecc0} compares this formula to the
integral of Eq.~\eqref{eq:tau_int} and to the direct sum of
Eq.~\eqref{eq:tau_explicit_sum} as a function of the eccentricity. It can be see
that both the integral and the analytic closed form perform well for
moderate-to-large eccentricities, but both overpredict the torque at small $e
\lesssim 0.3$. This discrepancy is expected: there are only a few nonnegligible
summands in Eq.~\eqref{eq:tau_explicit_sum} when $e$ is small, so replacing the
sum over $N$ with an integral is expected to introduce significant inaccuracy
that appears in both the integral and closed form expression.

Eq.~\eqref{eq:tau_lowspin} is valid so long as $\abs{\Omega_{\rm s} / \Omega} \ll
N_{\max}$, where $N_{\max} = 10 \eta_2/3$ is where the integrand is maximized.
If instead $\abs{\Omega_{\rm s} / \Omega} \gg N_{\max}$, the torque can be
evaluated directly using Eq.~\eqref{eq:tau_explicit_sum} and the known Hansen
coefficient moments, giving:
\begin{equation}
    \lim_{\Omega_{\rm s} \to \infty} T = -T_0 \sgn (\Omega_{\rm s})\;\left|2
        \Omega_{\rm s} / \Omega\right|^{8/3} \frac{f_5}{(1 -
        e^2)^{9/2}}.\label{eq:tau_highspin}
\end{equation}
The bottom panel of Fig.~\ref{fig:totals_ecc0} compares this formula to the
integral of Eq.~\eqref{eq:tau_int} and to the direct sum of
Eq.~\eqref{eq:tau_explicit_sum} as a function of the eccentricity, where
$\Omega_{\rm s} / \Omega = 400$. Here, $400 \gg N_{\max}$ for all eccentricities
shown. We see that direct summation, the integral expression, and
Eq.~\eqref{eq:tau_highspin} all agree very well for all eccentricities.

We now have obtained the asymptotic forms of Eq.~\eqref{eq:tau_int} for small
and large spins, but we can further derive a single expression joining these two
limits. To do this, we first assume that the spin is small but non-negligible.
In this regime, we make the approximation
\begin{equation}
    N - 2\Omega_{\rm s} / \Omega \simeq \frac{N}{N_{\max}}
        \left(N_{\max} - \frac{2\gamma_T
        \Omega_{\rm s}}{\Omega}\right)\label{eq:nmax_ansatz},
\end{equation}
for some free parameter $\gamma_T$, after which we can integrate
Eq.~\eqref{eq:tau_int} in closed form. $\gamma_T$ is fixed by requiring our
expression reproduce the large spin limit (Eq.~\ref{eq:tau_highspin}) when
taking $\abs{\Omega_{\rm s}} \to \infty$. This procedure gives an expression for
the torque that agrees with both asymptotic forms
(Eq.~\ref{eq:tau_highspin}--\ref{eq:tau_lowspin}) and is given by:
\begin{align}
    T &= T_0 \frac{f_5 (\eta_2/2)^{8/3}}{(1 - e^2)^{9/2}}
        \;\mathrm{sgn}\left(1 - \gamma_T\frac{\Omega_{\rm s}}{\eta_2\Omega}\right)
            \left|
                \frac{4}{\gamma_T}
                \p{1 - \gamma_T\frac{\Omega_{\rm s}}{\eta_2\Omega}}
            \right|^{8/3} ,\label{eq:tau_approx}\\
    \gamma_T &= 4\p{\frac{4!}{\Gamma(23/3)}}^{3/8} \approx 0.691.
\end{align}
Figure~\ref{fig:totals_s} compares this expression to the integral of
Eq.~\eqref{eq:tau_int} and to the direct sum of Eq.~\eqref{eq:tau_explicit_sum}
at fixed $e = 0.9$ and varying $\Omega_{\rm s}$. Eq.~\eqref{eq:tau_approx}
agrees well with both the integral and sum for large and small spins, as is
expected from how it is constructed, and is also somewhat accurate for
intermediate spins. However, Eq.~\eqref{eq:tau_int} is more accurate than
Eq.~\eqref{eq:tau_approx} when $T$ changes signs and $\abs{T}$ is small. This is
also expected: $T$ changes signs when the spin approaches
\emph{pseudosynchronization} (Section~\ref{ss:pseudosynchronization}) because
large contributions to the sum in Eq.~\eqref{eq:tau_explicit_sum} have opposite
signs and mostly cancel out. Thus, small inaaccuracies in the summand result in
significant discrepancies in the total torque. The integral approximation,
Eq.~\eqref{eq:tau_int}, is expected to be in good agreement with the direct sum,
Eq.~\eqref{eq:tau_explicit_sum}, as the accuracy of the Hansen coefficient
approximation in Section~\ref{ss:hansens} is good for large eccentricities, thus
guaranteeing term-by-term accuracy. On the other hand, the closed form
expression, Eq.~\eqref{eq:tau_approx}, is a crude extrapolation where
$\abs{\Omega_{\rm s} / \Omega} \simeq N_{\max}$ and is not expected to be
accurate in this regime. In fact, Eq.~\eqref{eq:tau_approx} predicts that
$\rdil{T}{\Omega_{\rm s}} \approx 0$ near pseudosynchronization. This is not
accurate and is an artifact of our factorization ansatz in
Eq.~\eqref{eq:nmax_ansatz}.

In summary, the tidal torque must be evaluated with explicit summation
(Eq.~\ref{eq:tau_explicit_sum}) when $e\lesssim 0.3$ (see discussion after
Eq.~\ref{eq:tau_lowspin}), can be approximated by the
integral expression (Eq.~\ref{eq:tau_int}) when $e$ is large even when
$\Omega_{\rm s} \simeq \Omega_{\rm p}$, and otherwise can be approximated by the
closed-form expression given by Eq.~\eqref{eq:tau_approx}. Recall that when $e$
is small, explicit summation of Eq.~\eqref{eq:tau_explicit_sum} is quite simple,
as good accuracy can be obtained with just the first few terms in the summation.
\begin{figure}
    \centering
    \includegraphics[width=\columnwidth]{1totals_ecc_0.png}
    \includegraphics[width=\columnwidth]{1totals_ecc_400.png}
    \caption{Tidal torque on a non-rotating (top) and rapidly rotating (bottom)
    star due to a companion with orbital eccentricity $e$. Black circles
    represent direct summation of Eq.~\eqref{eq:tau_explicit_sum}, green
    crosses are evaluated using the integral approximation
    Eq.~\eqref{eq:tau_int}, and the blue line is Eq.~\eqref{eq:tau_approx}. In
    the small and large spin limits, Eq.~\eqref{eq:tau_approx} reduces to
    Eq.~\eqref{eq:tau_lowspin} and Eq.~\eqref{eq:tau_highspin} respectively.
    }\label{fig:totals_ecc0}
\end{figure}
\begin{figure}
    \centering
    \includegraphics[width=\columnwidth]{1totals_s_0_9.png}
    \caption{Tidal torque as a function of spin for a highly eccentric $e = 0.9$
    companion. Black circles represent direct summation of Hansen coefficients
    (Eq.~\ref{eq:tau_explicit_sum}), green crosses the integral approximation
    (Eq.~\ref{eq:tau_int}), and solid lines represent the analytic closed form
    (Eq.~\ref{eq:tau_approx}). The spin is normalized by $\Omega_{\rm p}
    \approx 43 \Omega$ (Eq.~\ref{eq:Wperi}). }\label{fig:totals_s}
\end{figure}

\subsubsection{Pseudosynchronization}\label{ss:pseudosynchronization}

In general, the exact torque as given by Eq.~\eqref{eq:tau_explicit_sum}
vanishes for a single $\Omega_{\rm s}$, which we call the
\emph{pseudosynchronized} spin frequency. An approximation for the
pseudosynchronized spin can be directly read off of Eq.~\eqref{eq:tau_approx}:
\begin{equation}
    \frac{\Omega_{\rm ps}}{\Omega} =
        \frac{\eta_2}{\gamma_T} = \frac{4f_2}{5\gamma_T f_5\p{1 -
        e^2}^{3/2}}.\label{eq:eta2_691}
\end{equation}
This has the expected scaling $\Omega_{\rm ps} \simeq \Omega_{\rm p}$.
Figure~\ref{fig:pseudosync} compares these two predictions for the
pseudosynchronized spin to the exact one obtained by applying a root finding
algorithm to Eq.~\eqref{eq:tau_explicit_sum}. We see that $\Omega_{\rm ps}$ is a
good approximation for the pseudosynchronized spin frequency when $e \lesssim
0.1$.
\begin{figure}
    \centering
    \includegraphics[width=\columnwidth]{1pseudosynchronous.png}
    \caption{Calculations of the pseudosynchronization spin frequencies
    $\Omega_{\rm sync}$ normalized by the pericenter frequency $\Omega_{\rm p}$
    (Eq.~\ref{eq:Wperi}) as a function of eccentricity. The blue line is given
    by Eq.~\eqref{eq:eta2_691}. The black line shows the exact solution,
    obtained using finding a root finding algorithm to solve for the zero of
    Eq.~\eqref{eq:tau_sum}. The red dashed line shows the pseudosynchronization
    spin frequency predicted by the weak friction theory of equilibrium tides
    (Eq.~\ref{eq:weaktide}).
    }\label{fig:pseudosync}
\end{figure}

In passing, we note that, in the standard weak friction theory of equilibrium
tides, the pseudosynchronized spin is given by \citep{alexander73, hut81}
\begin{equation}
    \frac{\Omega_{\rm ps}^{\rm (Eq)}}{\Omega} = \frac{f_2}{f_5\p{1 -
        e^2}^{3/2}}.\label{eq:weaktide}
\end{equation}
Though describing a different tidal phenomenon, this only differs from
Eq.~\eqref{eq:eta2_691} by a factor of $4 / (5\gamma_{T}) \approx 1.15$. We show
it for comparison as the red dotted line in Fig.~\ref{fig:pseudosync}.

\subsubsection{Energy Transfer}

We now turn our attention to Eq.~\eqref{eq:edot_in} and replace $F_{N2}$ and
$F_{N0}$ with their respective approximations (Eqs.~\ref{eq:fn2_fit}
and~\ref{eq:fn0_fit}) to obtain
\begin{align}
    \dot{E}_{\rm in} ={}&
        \frac{T_0 \Omega}{2} \int\limits_0^\infty \Big[
            C_2^2 N^5 e^{-2N/\eta_2} \mathrm{sgn}\left(N - 2\Omega_{\rm s} /
                \Omega\right) \left|N - 2 \Omega_{\rm s} /
                \Omega\right|^{8/3}\nonumber\\
            &+ 2 \p{\frac{W_{20}}{W_{22}}}^2 C_0^2 e^{-2N / \eta_0} N^{11/3}
        \Big]\;\mathrm{d}N.\label{eq:ein_int}
\end{align}
We evaluate the $m = 2$ and $m = 0$ contributions to this expression separately.

We first examine the $m = 2$ contribution using the same procedure that was used
in Section~\ref{ss:torque_eval} for the torque. If the spin is moderate,
i.e.\ $\abs{\Omega_{\rm s} / \Omega} \lesssim N_{\max}$ where now $N_{\max} = 23
\eta_2 / 6$, we make the approximation
\begin{equation}
    N - 2\Omega_{\rm s} / \Omega \simeq \frac{N}{N_{\max}}
        \left(N_{\max} - \frac{2\gamma_E
        \Omega_{\rm s}}{\Omega}\right)\label{eq:nmax_ansatz},
\end{equation}
where $\gamma_E$ is a free parameter. This lets us integrate the $m = 2$
component of Eq.~\eqref{eq:ein_int} analytically. We constrain $\gamma_E$
by requiring agreement with the large-spin limit, where $\abs{\Omega_{\rm s} /
\Omega} \gg N_{\max}$ and we obtain
\begin{equation}
    \lim_{\Omega_{\rm s} \to \infty} \dot{E}_{\rm in}^{(m=2)} =
        -\frac{T_0\Omega}{2} \; \mathrm{sgn}(\Omega_{\rm s})|2\Omega_{\rm
        s}/\Omega|^{8/3} \frac{2f_2}{(1 - e^2)^6}.\label{eq:ein_highs}
\end{equation}
This fixes $\gamma_E$ and we obtain the complete $m = 2$ contribution:
\begin{align}
    \dot{E}_{\rm in}^{(m=2)}
        ={}& \frac{T_0\Omega f_5 (\eta_2/2)^{11/3}}{2(1-e^2)^{9/2}}
            \nonumber\\
        &\times \mathrm{sgn}\p{1 - \gamma_E\frac{\Omega_s}{\eta_2 \Omega}}
            \left|
                \frac{4}{\gamma_E}
                \p{1 - \gamma_E\frac{\Omega_s}{\eta_2 \Omega}}
            \right|^{8/3},\\
    \gamma_E ={}& 4\p{\frac{5!}{\Gamma\p{26/3}}}^{3/8}
        \approx 0.5886.
\end{align}

The $m = 0$ contribution to Eq.~\eqref{eq:ein_int} can be straightforwardly
integrated using the parameterization Eq.~\eqref{eq:fn0_fit}. The sum of the two
contributions then gives the total energy transfer rate:
\begin{align}
    \dot{E}_{\rm in} ={}& \frac{T_0 \Omega}{2}\Bigg[\frac{f_5
        (\eta_2/2)^{11/3}}{(1-e^2)^{9/2}} \frac{\Gamma(26/3)}{4!}\nonumber\\
        &\times \mathrm{sgn}\p{1 - \gamma_E\frac{\Omega_s}{\eta_2 \Omega}}
            \left|1 - \gamma_E\frac{\Omega_s}{\eta_2 \Omega}\right|^{8/3}
            \nonumber\\
        &+
    \frac{f_5 \Gamma(14 / 3)}{(1 - e^2)^{10}} \left(\frac{3}{2}\right)^{8/3}
            \left(\frac{e^2 f_3}{f_5}\right)^{11/6}\Bigg].
            \label{eq:ein_dot_tot}
\end{align}

The two panels of Fig.~\ref{fig:e0} compare this expression with the integral
form Eq.~\eqref{eq:ein_int} and the direct sum Eq.~\eqref{eq:ein_explicit_sum}
for small and large spins as a function of $e$. Again, when the spin and
eccentricity are both negligible, both the integral and closed form expressions
overpredict the energy dissipation rate. Figure~\ref{fig:e_spin} compares these
three expressions as a function of spin when $e = 0.9$. The performance of the
Eq.~\eqref{eq:ein_dot_tot} degrades when the system is near
pseudosynchronization, i.e.\ $\Omega_{\rm s} \simeq \Omega_{\rm p}$, but
generally captures the correct scaling, while Eq.~\eqref{eq:ein_int} is accurate
for all spins. As was the case with the tidal torque, we see that evaluation of
the energy transfer rate when $e \lesssim 0.3$ requires direct summation
(Eq.~\ref{eq:ein_explicit_sum}), evaluation when $e$ is substantial but the spin
is near pseudosynchronization can be performed using the integral approximation
(Eq.~\ref{eq:ein_int}), and otherwise evaluation can be performed using the
closed-form expression (Eq.~\ref{eq:ein_dot_tot}).
\begin{figure}
    \centering
    \includegraphics[width=\columnwidth]{1totals_e_0.png}
    \includegraphics[width=\columnwidth]{1totals_e_400.png}
    \caption{Plot of $\dot{E}_{\rm in}$ for a non-rotating (top) and a rapidly
    rotating (bottom) star. Black circles represent direct summation of the
    Hansen coefficients as in Eq.~\eqref{eq:ein_explicit_sum}, green crosses
    the integral form Eq.~\eqref{eq:ein_int}, and the blue line the closed form
    Eq.~\eqref{eq:ein_dot_tot}. }\label{fig:e0}
\end{figure}
\begin{figure}
    \centering
    \includegraphics[width=\columnwidth]{1totals_NRG_e_0_9.png}
    \caption{$\dot{E}_{\rm in}$ as a function of spin [normalized by
    $\Omega_{\rm p}$; Eq.~\eqref{eq:Wperi}] for a highly eccentric $e = 0.9$
    companion. Black circles represent direct summation of Hansen coefficients,
    green crosses the integral approximation, and the blue line the analytic
    closed form. }\label{fig:e_spin}
\end{figure}

\section{Example System: PSR J0045+7319}\label{s:j00457319}

As an example of our calculations above, we consider the pulsar-MS binary PSR
J0045-7319 \citep{bell1995psr} and attempt to explain its orbital decay via
dissipation due to dynamical tides. Previous attempts to do so used Zahn's
parameterized theory of dynamical tides and thus are more susceptible to
inaccuracies \citep[e.g.][]{lai1996, kumar1998}. The binary was initially
reported to have pulsar mass $M_2 = 1.4M_{\odot}$, mass ratio $q = 6.3$, $e =
0.808$, orbital period $P = 51.17\;\mathrm{days}$, and exhibit orbital decay at
the rate $\dot{P} = -3.03\times 10^{-7}$ \citep{kaspi1996params}. From these, a
MS mass of $M = 8.8M_{\odot}$ and an orbital separation of $a = 126R_{\odot}$
can be inferred. Furthermore, the measured luminosity $L = 1.2 \times
10^4L_{\odot}$ and surface temperature $T_{\rm surf} = (24000 \pm
1000)\;\mathrm{K}$ of the MS star its radius to be $R = 6.4R_{\odot}$
\citep{kaspi1996params}\footnote{Later studies obtain different MS properties in
J0045-7319, e.g.\ \citet{thorsett1999neutron} obtain a MS mass of $M = 10.0
M_{\odot}$ and $M_2 = 1.58M_{\odot}$. For better comparison with existing work
on the effect of dynamical tides in this system, we use the same parameters as
earlier works used.}. The internal structure of the star can be obtained by
comparison to detailed stellar structure calculations, yielding $M_{\rm c}
\approx 3M_{\odot}$ and $r_{\rm c} \approx 1.38R_{\odot}$ \citep{kumar1998}. As
the stellar structure may be somewhat uncertain, we take this value $M_{\rm c}$
to be fixed and consider a range of $r_{\rm c} \in [0.7, 1.5]$. In order to
compute the orbital decay due to dynamical tides, we also require the ratio
$\rho_{\rm c} / \bar{\rho}_{\rm c}$, which can only be obtained via stellar
structure simulations. We take $\rho_{\rm c} / \bar{\rho}_{\rm c} \approx 1/3$
as a fiducial value (though in reality this likely changes as $r_{\rm c}$
changes).

Figure~\ref{fig:j0045_fid} shows $\dot{P}$ as a function of $\Omega_{\rm s}$,
using Eq.~\eqref{eq:ein_dot_tot}, evaluated using four different $r_{\rm c}$.
The measured $\dot{P}$ is shown by the horizontal dashed line. Note that for the
most compact core radius $r _{\rm c} = 0.7R_{\odot}$, there are no solutions for
$\Omega_{\rm s}$; even a maximally spinning core cannot generate enough tidal
dissipation to match the observed $\dot{P}$. In general, substantial retrograde
rotation of the stellar core is required to match the observed orbital decay
rate, a few times faster than the critical rotation rate of the star as a whole.
The required core rotation rate decreases when its radius is increased. The MS
star has a projected surface rotation rate of $v \sin i = 113 \pm
10\;\mathrm{km/s}$ \citep{bell1995psr}, which is substantially slower than the
breakup surface rotation rate of $\sqrt{GM / R} = 512\;\mathrm{km/s}$. Thus,
strong differential rotation is required in the MS star in J0045-7319 to
generate the required orbital decay via dynamical tides alone.

\begin{figure}
    \centering
    \includegraphics[width=\columnwidth]{1_7319_disps.png}
    \caption{$\dot{P}$ as a function of $\Omega_{\rm s}$ for the canonical
    parameters for J0045-7319, as evaluated by direct summation of
    Eq.~\eqref{eq:ein_explicit_sum}, for four different values of $r_{\rm c}$
    (legend, in units of $R_{\odot}$). The measured $\dot{P} = -3.03\times
    10^{-7}$ is shown by the horizontal dashed line. The vertical black line
    denotes a surface spin rate of $160\;\mathrm{km/s}$, corresponding to the
    observed $v \sin i = 110\;\mathrm{km/s}$ with an estimated $\sin i \simeq
    2^{-1/2}$. The vertical shaded region is the region where $\Omega_{\rm s}$
    is less than the breakup rotation rate of the star as a whole, given by $(GM
    / R^3)^{1/2}$. Each $r_{\rm c}$ is only shown for $\abs{\Omega_{\rm s}} \leq
    \Omega_{\rm s, c} \equiv (GM_{\rm c} / r_{\rm c}^3)^{1/2}$ the \emph{core}
    breakup rotation rate.
    }\label{fig:j0045_fid}
\end{figure}

% \subsection{Constraints on Stellar Structure}

% As discovered above, requiring that the observed $\dot{P}$ lie within the
% range attainable via tidal dissipation imposes constraints on $r_{\rm c}$. We
% can compute the range of attainable $\dot{P}$ by setting the spin frequency
% equal to the breakup frequency in Eq.~\eqref{eq:ein_int}, giving
% \begin{equation}
%     \dot{P} \lesssim
%         -\frac{6\pi}{q}\beta_2 \left(\frac{r_{\rm c}}{a}\right)^5
%         \frac{\rho_{\rm c}}{\bar{\rho}_{\rm c}} \left(1 -
%         \frac{\rho_{\rm c}}{\bar{\rho}_{\rm c}}\right)^2 2^{8/3}\frac{f_2}{(1 -
%         e^2)^6}.\label{eq:dotp_rc}
% \end{equation}
% For J0045+7319, requiring that $\dot{P}$ equal its observed value gives $r_{\rm
% c} \gtrsim 0.93R_{\odot}$ when using $\rho_{\rm c} / \bar{\rho}_{\rm c} = 1/3$.
% Indeed, in Fig.~\ref{fig:j0045_fid}, there is a solution equal to the observed
% $\dot{P}$ when $r_{\rm c} \geq R_{\odot}$ but not when $r_{\rm c} =
% 0.7R_{\odot}$. The strong dependence of the right hand side on
% Eq.~\eqref{eq:dotp_rc} on $r_{\rm c}$ suggests that this constraint is somewhat
% insensitive to other system uncertanties.

\section{Conclusion and Discussion}\label{s:disc}

In this paper, we have calculated the torque and orbital decay rate due to
dynamical tides in a massive, main-sequence (MS) star under the gravitational
influence of an eccentric companion. For general eccentricities, these are
given by the sums Eqs.~\eqref{eq:tau_explicit_sum}
and~\eqref{eq:ein_explicit_sum} respectively. However, when the eccentricity is
large, these sums require the evaluation of many terms to be accurate. We
present approximations for two possible regimes:
\begin{itemize}
    \item For $e \gtrsim 0.3$, we show that the torque and orbital decay rate
        can be accurately approximated by the integral expressions
        Eqs.~\eqref{eq:tau_int} and~\eqref{eq:ein_int} respectively.

    \item If furthermore the spin of the stellar core is not near its
        pseudosynchronized value (i.e.\ where the torque vanishes; see
        Section~\ref{ss:pseudosynchronization}), we show that these two integral
        expressions can be approximated by the closed-form expressions
        Eqs.~\eqref{eq:tau_approx} and~\eqref{eq:ein_dot_tot} respectively.
\end{itemize}
The accuracy of these approximations for the torque and orbital decay rate as
functions of the spin of the stellar core and orbit eccentricity are illustrated
in Figs.~\ref{fig:totals_ecc0},~\ref{fig:totals_s},~\ref{fig:e0},
and~\ref{fig:e_spin}.

We then apply our results to the radio pulsar J0045-7319, which has a MS B star
companion and a measured orbital decay rate. In Fig.~\ref{fig:j0045_fid}, we
show that the required stellar core spin to generate the observed orbital decay
is a few times the breakup rotation rate of the entire star. This suggests that
the core-envelope coupling timescale in the MS star is at least the
characteristic age of the pulsar, $\approx 3\;\mathrm{Myr}$
\citep{kaspi1996params}.

We address a few potential caveats of our tidal model. In our derivations, we
have assumed that the stellar spin and orbit axes are aligned. If the stellar
obliquity is substantial, then the orbital decay rate is generally expected to
be decreased \citep[see e.g.][]{lai2012tidal} \textcolor{red}{(NB Dong: is this
a good reference? Or do I not need one.)}. As such, our results give the
\emph{maximum} tidal torque and orbital decay rates due to dynamical tides.
Additionally, in our work, we have assumed that all Fourier harmonics $N$ excite
internal gravity waves at the radiative-convective boundary that damp as they
propagate towards the stellar surface. This is known to be the case when the
normalized tidal forcing frequency satisfies $\abs{2\Omega - 2\Omega_{\rm s}}
\ll \sqrt{GM / R^3}$ \citep{zahn1975dynamical, kushnir}. In the present
scenario, this inequality is always violated for sufficiently large $N$.
However, these higher frequency waves are still likely to deposit most of their
angular momentum near the stellar surface, as either their amplitudes grow large
and they undergo wave breaking or they encounter a critical layer
(\citealp{goldreich1989tidal}; though there may be significant reflected angular
momentum flux, see \citealp{su2020}). In this scenario, significant differential
rotation is expected, as the wave breaking process tends to steepen shear flows
\citep{su2020}. This violates the assumption of rigid rotation (see discussion
before Eq.~\ref{eq:dots}) but only changes the spin evolution of the star and
not the orbital evolution of the binary.

We also discuss potential caveats of our results regarding the pulsar system
J0045-7319. We have used the stellar parameters derived by comparison with MS
stellar models \citep{kumar1998}. However, it is possible that earlier phases of
the binary's evolution have resulted in a B star with stellar structure
different than MS stars. If this is the case, the discussion in
Section~\ref{s:j00457319} suggests that the core is likely \emph{larger} than MS
stellar models predict. Another possible caveat is the accuracy of the original
stellar models: a recent study of intermediate and high-mass eclipsing binaries
suggests that convective core masses are underpredicted by stellar structure
codes \citep{larger_conv_masses}.

\subsection{Applicability to Other Tidal Models}

\textcolor{red}{[NB Dong: Do you think this is worth pointing out? I think it's
worth at least these few sentences, but maybe it scatters the message too much
(though I think that the paper isn't very busy as is\dots)]}

While the detailed expressions in this work are specific to dynamical tides, the
approach of using the Hansen coefficient approximations given by
Eq.~\eqref{eq:fn2_fit} to evaluate the torque as given by
Eq.~\eqref{eq:tau_sum} for a given circular torque $T_{\rm circ}$ is general,
i.e.:
\begin{equation}
    T = \int\limits_0^\infty C_2^2 N^{4}e^{-2N / \eta_2}T_{\rm circ}
        \p{N\Omega - 2\Omega_{\rm s}}\;\mathrm{d}N.
\end{equation}
This can be further simplified analytically for simple $T_{\rm circ}$, e.g.\ in
white dwarfs, $T_{\rm circ}(\omega) \propto \omega^5$
\citep{fuller2012dynamical}.

% There is also reason to expect that the Hansen coefficients for higher-order
% multiple expansions can be approximated with similarly simple expressions,
% following the arguments presented in Section~\ref{ss:hansens}: the perturbing
% function is infinitely differentiable and is peaked at pericenter at all
% multiple orders.

\section{Acknowledgements}

We thank Michelle Vick, Christopher O'Connor, and Matteo Cantiello for fruitful
discussions. YS is supported by the NASA FINESST grant 19-ASTRO19-0041.

\bibliographystyle{mnras}
\bibliography{Su_eccentric_tides}

% \clearpage
% \onecolumn

\bsp
\label{lastpage} % chktex 24
\end{document}
