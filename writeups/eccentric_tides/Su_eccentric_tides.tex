 % chktex-file 8
    \documentclass[
        fleqn,
        usenatbib,
        % referee,
    ]{mnras}
    \usepackage{
        amsmath,
        amssymb,
        newtxtext,
        newtxmath,
        graphicx,
        ae, aecompl,
        booktabs,
        xcolor,
    }
    \usepackage[T1]{fontenc}

    \newcommand*{\rd}[2]{\frac{\mathrm{d}#1}{\mathrm{d}#2}}
    \newcommand*{\pd}[2]{\frac{\partial#1}{\partial#2}}
    \newcommand*{\rdil}[2]{\mathrm{d}#1 / \mathrm{d}#2}
    \newcommand*{\pdil}[2]{\partial#1 / \partial#2}
    \newcommand*{\rtd}[2]{\frac{\mathrm{d}^2#1}{\mathrm{d}#2^2}}
    \newcommand*{\ptd}[2]{\frac{\partial^2 #1}{\partial#2^2}}
    \newcommand*{\md}[2]{\frac{\mathrm{D}#1}{\mathrm{D}#2}}
    \newcommand*{\pvec}[1]{\vec{#1}^{\,\prime}}
    \newcommand*{\svec}[1]{\vec{#1}\;\!}
    \newcommand*{\bm}[1]{\boldsymbol{\mathbf{#1}}}
    \newcommand*{\uv}[1]{\hat{\bm{#1}}}
    \newcommand*{\ang}[0]{\;\text{\AA}}
    \newcommand*{\mum}[0]{\;\upmu \mathrm{m}}
    \newcommand*{\at}[1]{\left.#1\right|}
    \newcommand*{\bra}[1]{\left<#1\right|}
    \newcommand*{\ket}[1]{\left|#1\right>}
    \newcommand*{\abs}[1]{\left|#1\right|}
    \newcommand*{\ev}[1]{\langle#1\rangle}
    \newcommand*{\p}[1]{\left(#1\right)}
    \newcommand*{\s}[1]{\left[#1\right]}
    \newcommand*{\z}[1]{\left\{#1\right\}}
    \colorlet{Corr}{red}

    \DeclareMathOperator*{\sgn}{sgn}
    \DeclareMathOperator*{\argmax}{argmax}

\title[Eccentric Dynamical Tides]{Eccentric Dynamical Tides}
\author[Y. Su, D. Lai.]{
Yubo Su$^1$,
Dong Lai$^1$
\\
$^1$ Cornell Center for Astrophysics and Planetary Science, Department of
Astronomy, Cornell University, Ithaca, NY 14853, USA
}

\date{Accepted XXX\@. Received YYY\@; in original form ZZZ}

\pubyear{2020}

\begin{document}\label{firstpage}
\pagerange{\pageref{firstpage}--\pageref{lastpage}}
\maketitle

\begin{abstract}
    Massive stars in eccentric binaries are very important, being the progenitors
    of neutron star and high-mass X-ray binaries. In such systems, dynamical
    tides plays a crucial role. However, previous studies of dynamical tides in
    massive stellar binaries have primarily focused on the case where these
    binaries are circular. In this work, we revisit the effect of dynamical
    tides in eccentric, massive stellar binaries and derive analytical
    expressions that can be used to study binary evolution with arbitrary
    eccentricities. We apply our results to the radio pulsar J0045--7319, which
    has a massive B star companion and whose orbital period is rapidly
    decreasing. In order to reproduce this orbital decay via dynamical tides, we
    find that the core is likely rotating significantly faster than the measured
    surface rotation rate. This implies a long core-envelope coupling timescale
    in B stars.
\end{abstract}

\begin{keywords}
keywords
\end{keywords}

\section{Introduction}

In the course of their evolution, massive stellar binaries give rise to many
astrophysical systems of interest including high mass x-ray binaries (HMXRBs)
and compact object binaries (CITATION with evolutionary pathway). In general,
the more massive star undergoes a supernova before its less massive companion,
after which the binary consists of one massive star and one compact object in an
eccentric orbit (MS-CO binary). In such a system, the evolution is dominated by
the torque from the compact object on the massive star due to dynamical tides.
While this tidal torque is now well understood for circular binaries
\citep{kushnir}, it has not been carefully studied for binaries with substantial
eccentricities. The tidal evolution of such eccentric binaries sculpt the
population of HMXRBs (CITE) as well as the population of compact object binaries
\citep{vigna2020common}.

The dissipation due to the dynamical tide in a massive star's envelope under the
influence of a \emph{circular} perturber is traditionally understood via Zahn's
parameterized theory of dynamical tides \citep{zahn1975dynamical}. However,
Zahn's theory relies on a dimensionless parameter $E_2$ reflecting the detailed
stellar structure that varies over many orders of magnitudes for typical stars.
In general, the value of $E_2$ is taken from empirical fits to simplified
stellar models \citep{hurley2002evolution, vigna2020common}. The introduction of
this uncertain parameter $E_2$ is because the dynamical tidal torque arises due
to excitation of internal gravity waves at the radiative convective boundary
(RCB) \citep{goldreich1989tidal, savonije1983tidal}, but Zahn's formula is
evaluated at the \emph{stellar} radius. Instead, it is possible re-express the
tidal torque in terms of quantities evaluated at the RCB itself, for which
dimensionless parameters are generally of order unity for a wide range of stars
\citep{kushnir}. However, \citet{kushnir} only consider circular binaries.

To study the dynamical tide in eccentric binaries, it is natural to decompose
the perturbing potential into Fourier harmonics, each of which is analogous to a
perturber on a circular orbit \citep[e.g.][]{sl, vlf}. While accurate, such
decompositions are unwieldy to evaluate as the eccentricity increases, often
requiring summing hundreds of terms and lending little intuition to the broad
scalings of the tidal torque. In this work, we show that, for the circular
torque given by \citet{kushnir}, an accurate, approximate, closed form for the
dynamical tide in a highly eccentric MS-CO binary can be obtained. Contrary to
existing models of dynamical tides \citep[e.g.][]{vigna2020common}, our
formulation improves in accuracy as the binary eccentricity increases.  We give
closed forms for both the tidal torque and inspiral rate of such an MS-CO
binary.

In Section~\ref{s:background}, we review the relevant equations. In
Sections~\ref{s:hansens} and~\ref{s:eval}, we derive accurate, approximate
closed forms for the torque and energy transfer rate in the binary. In
Section~\ref{s:j00457319}, we apply results to the binary radio pulsar
J0045-7319. We conclude and discuss in Section~\ref{s:disc}.

\section{Summary of Relevant Work}\label{s:background}

\subsection{Tidal Torque in Massive Stars}

We first review the case where the MS-CO binary is circular. Let $M_2$ be the
mass of the CO, $a$ be the semimajor axis of the binary, and $\Omega$ the
orbital angular frequency of the binary. The tidal torque exerted on the star by
the companion is \citep{kushnir}:
\begin{align}
    T_{\rm circ}(\omega) ={}& T_0 \sgn\p{\omega}
        \Big|\,\frac{\omega}{\Omega}\,\Big|^{8/3} \label{eq:kushnir_torque},\\
    T_0 ={}& \frac{GM_2^2r_{\rm c}^5}{a^6}
        \p{\frac{\Omega}{\sqrt{GM_{\rm c}/r_{\rm c}^3}}}^{8/3}
        \s{\frac{r_{\rm c}}{g_{\rm c}}
            \p{\rd{N^2}{\ln r}}_{r = r_{\rm c}}}^{-1/3}\nonumber\\
            &\times \frac{\rho_{\rm c}}{\bar{\rho}_{\rm c}}
                \p{1 - \frac{\rho_{\rm c}}{\bar{\rho}_{\rm c}}}^2
                \s{\frac{3}{2}\frac{3^{2/3}\Gamma^2(1/3)}{5 \cdot
                6^{4/3}} \frac{3}{4\pi}\alpha^2},\nonumber\\
        \equiv{}& \beta_2\frac{GM_2^2r_{\rm c}^5}{a^6}
            \p{\frac{\Omega}{\sqrt{GM_{\rm c}/r_{\rm c}^3}}}^{8/3}
            \frac{\rho_{\rm c}}{\bar{\rho}_{\rm c}} \p{1 - \frac{\rho_{\rm
            c}}{\bar{\rho}_{\rm c}}}^2.
\end{align}
Here, $\omega \equiv \Omega - 2\Omega_{\rm s}$ is the tidal forcing frequency,
$\Omega_{\rm s}$ is the spin of the massive star, $N$ is the Br\"unt-Vaisala
frequency, $r$ is the radial coordinate within the star, and $r_{\rm c}$,
$M_{\rm c}$, $g_{\rm c}$, $\rho_{\rm c}$, $\bar{\rho}_{\rm c}$ are the radius of
the RCB, the mass contained within the convective core, the gravitational
acceleration at the RCB, the stellar density at the RCB, and the average density
of the convective core, respectively. $\alpha$ and $\beta_2$ are numerical
constants defined by \citet{kushnir}, where $\beta_2 \approx 1$ is a good
approximation for a large range of stellar models. In
Eq.~\eqref{eq:kushnir_torque}, we have written the terms such that $T_0$
contains all the spin-independent terms.

\subsection{Perturbation from an Eccentric Companion: Hansen Coefficients}

Separately, we review the general procedure for calculating tidal dissipation
due to an eccentric perturber. The gravitational potential of an eccentric
companion to quadrupolar order can be decomposed as a sum over circular orbits
\citep[e.g.][]{sl,vlf}:
\begin{align}
    U &= \sum\limits_{m=-2}^2 U_{2m} \p{\vec{r}, t},\label{eq:u_ecc}\\
    U_{2m}\p{\vec{r}, t} &= -\frac{GM_2 W_{2m} r^2}{D(t)^3}
        Y_{2m}(\theta, \phi) e^{-imf(t)},\nonumber\\
        &= -\frac{GM_2W_{2m} r^2}{a^3}Y_{2m}\p{\theta, \phi}
            \sum\limits_{N = -\infty}^\infty F_{Nm}e^{-iN\Omega t}
            \label{eq:hansen_decomp}.
\end{align}
Here, $(r, \theta, \phi)$ are the radial, polar, and azimuthal coordinates of
$\vec{r}$ respectively, $W_{2 \pm 2} = \sqrt{3\pi/10}$, $W_{2 \pm 1} = 0$,
$W_{20} = -\sqrt{\pi / 5}$, $D(t)$ is the instantaneous distance between the
star and companion, $f$ is the true anomaly, $Y_{lm}$ denote the spherical
harmonics, and $\Omega$ is the mean motion of the companion. $F_{Nm}$ denote
the \emph{Hansen coefficients} for $l = 2$ \citep[also denoted $X^n_{2m}$
in][]{murray1999solar}, which are defined implicitly in
Eq.~\eqref{eq:hansen_decomp} to be the Fourier coefficients of the perturbing
function, i.e.
\begin{equation}
    \frac{a^3}{D(t)^3} e^{-imf(t)} = \sum\limits_{N = -\infty}^\infty
        F_{Nm} e^{-iN\Omega t}.\label{eq:hansen_series}
\end{equation}
The $F_{Nm}$ can be written explicitly as an integral over the eccentric anomaly
\citep{murray1999solar, sl}, but this requires separate integral for each $N$.
Instead, taking a discrete Fourier Transform of the left hand side of
Eq.~\eqref{eq:hansen_series} permits computation of arbitrarily many $N$ at once
\citep[as pointed out by][]{correia2014deformation}, which is what is done in
this work.

By considering the effect of each summand in Eq.~\eqref{eq:hansen_decomp}, the
total torque on the star, energy transfer in the inertial frame, and heating in
the star's corotating frame can be obtained \citep{sl, vlf}:
\begin{align}
    T ={}& \sum\limits_{N = -\infty}^\infty F_{N2}^2
        T_{\rm circ}\p{N\Omega - 2\Omega_{\rm s}},\label{eq:tau_sum}
        \\
    \dot{E}_{\rm in} ={}&
        \frac{1}{2}\sum\limits_{N = -\infty}^\infty\Bigg\{
            \p{\frac{W_{20}}{W_{22}}}^2 N\Omega F_{N0}^2 T_{\rm circ}\p{N\Omega}
                \nonumber\\
            &+ N\Omega F_{N2}^2 T_{\rm circ}\p{N\Omega - 2\Omega_{\rm s}}
            \Bigg\},\label{eq:edot_in}\\
    \dot{E}_{\rm rot} ={}& \dot{E}_{\rm in} - \Omega_{\rm s} T
        \label{eq:edot_rot}.
\end{align}

\subsection{Objective of This Paper}\label{ss:objective}

The objective of this paper is to study the effect of dynamical tides in an
eccentric MS-CO binary. First, we compute the tidal torque by substituting the
torque due to a CO on a circular orbit [Eq.~\eqref{eq:kushnir_torque}] into the
summation Eq.~\eqref{eq:tau_sum}, obtaining
\begin{equation}
    T = \sum_{N = -\infty}^{N = \infty} F_{N2}^2 T_0
        \,\mathrm{sgn}\left(N - \frac{2\Omega_{\rm s}}{ \Omega}\right) \left|N -
        \frac{2 \Omega_{\rm s}}{\Omega}\right|^{8/3}.\label{eq:tau_explicit_sum}
\end{equation}
The energy transfer rate in the inertial frame is similarly obtained by
substituting Eq.~\eqref{eq:kushnir_torque} into Eq.~\eqref{eq:edot_in}
\begin{align}
    \dot{E}_{\rm in} ={}& \frac{T_0}{2}
        \sum_{N = -\infty}^{\infty} \Bigg[
            N\Omega F_{N2}^2 \mathrm{sgn}\left(N - 2\Omega_{\rm s} / \Omega\right)
                    \left|N - 2 \Omega_{\rm s} / \Omega\right|^{8/3}\nonumber\\
            &+ \left(\frac{W_{20}}{W_{22}}\right)^2 \Omega
                    F_{N0}^2 |N|^{11/3}
            \Bigg].\label{eq:ein_explicit_sum}
\end{align}
These two expressions give the spin synchronization timescale of the star as
well as the inspiral time of the binary due to dynamical tides. While exact,
these expressions are difficult to evaluate for larger eccentricities, where one
often must sum hundreds or thousands of terms, each of which has a different
$F_{Nm}$. In the subsequent sections, our objective is to obtain closed-form
approximations to Eqs.~(\ref{eq:tau_explicit_sum}--\ref{eq:ein_explicit_sum}).

\section{Approximating Hansen Coefficients}\label{s:hansens}

In this section, we derive scalings for the Hansen coefficients in the
high-eccentricity limit. Since $F_{(-N)m} = F_{N(-m)}$, so we will only study
the Hansen coefficient behavior for $m \geq 0$.

\subsection{$m=2$ Hansen Coefficient Behavior at High Eccentricity}

We first consider the case where $m = 2$. Figure~\ref{fig:hansens} shows
the $F_{N2}$ for $e = 0.9$. First, we note that $F_{N2}$ is much larger when $N
\geq 0$ than for $N < 0$, so we focus on the behavior for $N \geq 0$. Here,
$F_{N2}$ has only one substantial peak. There are two characteristic frequency
scales, $\Omega$ and $\Omega_{\rm p}$ the pericenter frequency, defined by
\begin{equation}
    \Omega_{\rm p} \equiv \Omega \frac{\sqrt{1 + e}}{\p{1 - e}^{3/2}}.
        \label{eq:Wperi}
\end{equation}
For convenience, we also define
\begin{equation}
    N_{\rm p} \equiv \lfloor \Omega_{\rm p} / \Omega\rfloor,
\end{equation}
Since $N_{\rm p}$ is the largest harmonic scale, we expect that the peak of the
$F_{N2}$ should occur at $\sim N_{\rm p}$. When $N \gg N_{\rm p}$, the Fourier
coefficients must fall off exponentially by the Paley-Wiener theorem since the
left hand side of Eq.~\eqref{eq:hansen_series} is smooth. On the other hand,
when $N \ll N_{\rm p}$, since there are no characteristic frequenciees between
$\Omega$ and $\Omega_{\rm p}$, we expect that the Hansen coefficients must be
scale free between $N = 1$ and $N_{\rm p}$, i.e.\ a power law. Both of these
characteristics are reflected in Fig.~\ref{fig:hansens}.

Motivated by these considerations, we assume the Hansen coefficients can be
approximated by
\begin{equation}
    F_{N2} \approx
    \begin{cases}
        C_2 N^{p}e^{-N/\eta_2} & N \geq 0,\\
        0 & N < 0,
    \end{cases}\label{eq:fn2_fit}
\end{equation}
for some fitting coefficients $C_2$, $p$, and $\eta_2$. By performing fits to
$F_{N2}$, we found that $p \approx 2$ is relatively constant for modest-to-large
large eccentricities\footnote{This can be understood, as the left hand side of
Eq.~\eqref{eq:hansen_series} resembles the second derivative of a Dirac delta
when the eccentricity is substantial: it is sharply peaked about $t = 0$, is
periodic with period $P = 2\pi / \Omega$, and has zero derivative three times
every period (at $t = \epsilon$, $t = P / 2$, and $t = P - \epsilon$ for some
small $\epsilon \sim \Omega_{\rm p}^{-1}$). This characteristics suggest that it
resembles the second derivative of a Gaussian with width $\sim \Omega_{\rm
p}^{-1}$. For timescales $\gtrsim \Omega_{\rm p}^{-1}$, this Gaussian further
resembles the Dirac delta function, which has a flat Fourier spectrum ($\propto
N^0$). Since time differentiation multiplies by $N$ in frequency space, we see
indeed that $F_{N2} \propto N^2$ for $N \lesssim N_{\rm p}$.}. For
the remainder of this work, we
take $p = 2$ to be fixed.

To constrain the remaining two free parameters $\eta_2$ and $C_2$ the
normalization, we use the well known Hansen coefficient moments
\citep{hut81}
\begin{align}
    \sum\limits_{N = -\infty}^\infty F_{N2}^2 &= \frac{f_5}{\p{1 - e^2}^{9/2}},
        \\
    f_5 &\equiv 1 + 3e^2 + \frac{3e^4}{8},\\
    \sum\limits_{N = -\infty}^\infty F_{N2}^2N
        &= \frac{2f_2}{\p{1 - e^2}^6},\\
    f_2 &\equiv 1 + \frac{15e^2}{2}
            + \frac{45 e^4}{8} + \frac{5e^6}{16}.
\end{align}
This fixes
\begin{align}
    \eta_2 &= \frac{4f_2}{5f_5\p{1 - e^2}^{3/2}},\label{eq:eta2}\\
    C_2^2\eta_2^5 &= \frac{4f_5}{3\p{1 - e^2}^{9/2}}.\label{eq:C2}
\end{align}
Despite having zero free parameters, this formula accurately describes the
$F_{N2}$ as can be seen in Fig.~\ref{fig:hansens}.
\begin{figure}
    \centering
    \includegraphics[width=\columnwidth]{../../scripts/eccentric_tides/hansens.png}
    \caption{Plot of Hansen coefficients $F_{N2}$ for $e = 0.9$. The red dots
    denote negative $N$, while the black dots and crosses denote positive and
    negative $F_{N2}$. The green line is the formula given by
    Eq.~\eqref{eq:fn2_fit} with $\eta_2$ and $C_2$ given by
    Eqs.~(\ref{eq:eta2}--\ref{eq:C2}). }\label{fig:hansens}
\end{figure}

\subsection{$m = 0$ Hansen Coefficient Behavior at High Eccentricity}

We now turn to the $m = 0$ Hansen coefficients, $F_{N0}$, which are shown in
Fig.~\ref{fig:fn0_fit}. We know that $F_{N0} = F_{(-N)0}$, so we consider only
$N \geq 0$. From the figure, we see that the $F_{N0}$ decay exponentially. By
dimensional analysis, this decay must occur over scales $\sim N_{\rm p}$.
Therefore, we naturally assume the $F_{N0}$ scale as
\begin{equation}
    F_{N0} = C_0 e^{-\abs{N} / \eta_0}.\label{eq:fn0_fit}
\end{equation}
The two free parameters $C_0$ and $\eta_0$ are constrained by the well known
moments \citep{hut81}
\begin{align}
    \sum\limits_{N = -\infty}^\infty F_{N0}^2 &= \frac{f_5}{\p{1 - e^2}^{9/2}}
        ,\\
    \sum\limits_{N = -\infty}^\infty F_{N0}^2 N^2
        &= \frac{9e^2}{2\p{1 - e^2}^{15/2}}
            f_3,\\
    f_3 &= \frac{1}{2} + \frac{15e^2}{8} + \frac{15 e^4}{16}
        + \frac{5e^6}{128}.
\end{align}
We have defined the common functions $f_3$ and $f_5$. This then requires
\begin{align}
    \eta_0^2 &= \frac{9e^2f_3}{\p{1 - e^2}^{3}f_5},\label{eq:eta0}\\
    C_0^2\eta_0 &= \frac{f_5}{\p{1 - e^2}^{9/2}}.\label{eq:C0}
\end{align}
The good agreement of this analytic formula can be seen in
Fig.~\ref{fig:fn0_fit}.

\begin{figure}
    \centering
    \includegraphics[width=0.9\columnwidth]{../../scripts/eccentric_tides/hansens/hansens0_90.png}
    \caption{Plot of $F_{N0}$ for $e = 0.9$. Since $F_{N0} = F_{(-N)0}$, we only
    show positive $N$. The green line is given by Eq.~\eqref{eq:fn0_fit} with
    $\eta_0$ and $C_0$ given by
    Eqs.~(\ref{eq:eta0}--\ref{eq:C0}).}\label{fig:fn0_fit}
\end{figure}

\section{Evaluating Torque and Energy Transfer}\label{s:eval}

Having found good approximations for the Hansen coefficients, we now apply them
to simplify the formulas for the torque and the energy transfer rate in
Section~\ref{ss:objective}.

\subsection{Tidal Torque}\label{ss:torque_eval}

Towards simplifying the torque, given by Eq.~\eqref{eq:tau_explicit_sum}, we
replace $F_{N2}$ with Eq.~\eqref{eq:fn2_fit} and the sum with an integral,
obtaining
\begin{equation}
    T = T_0 \int_0^\infty C_2^2 N^4 e^{-2N / \eta_2}
        \sgn\left(N - 2\Omega_{\rm s} / \Omega\right) \left|N - 2 \Omega_{\rm s} /
            \Omega\right|^{8/3}\;\mathrm{d}N.\label{eq:tau_int}
\end{equation}

To further simplify this expression, we analyze it in the large and small spin
limits. To distinguish between the two limits, we first define $N_{\max}$ to be
the $N$ for which the integrand in Eq.~\eqref{eq:tau_int} is largest assuming
$\Omega_{\rm s} = 0$, so $N_{\max} = 10\eta_2/3$. Note that if $\Omega_{\rm s}$
is large instead, the integrand is maximized at $2\eta_2 \simeq N_{\max}$.
The large-spin limit is then where $\abs{\Omega_{\rm s}} \gg N_{\max}\Omega / 2$.
In this limit, Eq.~\eqref{eq:tau_explicit_sum} can be evaluated directly with
the known Hansen coefficient moments, which gives
\begin{equation}
    \lim_{\Omega_{\rm s} \to \infty} T = -T_0 \sgn (\Omega_{\rm s})\;\left|2
        \Omega_{\rm s} / \Omega\right|^{8/3} \frac{f_5}{(1 -
        e^2)^{9/2}}.\label{eq:tau_highspin}
\end{equation}
The accuracy of this formula is shown in the top panel of
Fig.~\ref{fig:totals_ecc0} for $\Omega_{\rm s} / \Omega = 400 \gg N_{\max}$. For
comparison, we also illustrate the torque predicted by direct integration of
Eq.~\eqref{eq:tau_explicit_sum}. It is seen that both the integral approximation
and the asymptotic closed form match the direct summation of Hansen coefficients
accurately.

We can also evaluate the small spin limit $\abs{\Omega_{\rm s}} \ll
N_{\max}\Omega / 2$. Eq.~\eqref{eq:tau_int} can be integrated
analytically\footnote{The key to the success of our approach is that sums of
form $\sum_{n = -\infty}^\infty F_{N2}^2 N^p$ can be approximated for
non-integer $p$ in terms of $\Gamma$, since $\int\limits_0^\infty
x^pe^{-x}\;\mathrm{d}x = \Gamma(p - 1)$. This is not possible with existing
analytical techniques.}, giving
\begin{align}
    \lim_{\Omega_{\rm s} \to 0} T &= T_0 \frac{f_5 (\eta_2/2)^{8/3}}{(1 -
        e^2)^{9/2}} \frac{\Gamma(23/3)}{4!}.\label{eq:tau_lowspin}
\end{align}
The accuracy of this formula is shown in the bottom panel of
Fig.~\ref{fig:totals_ecc0} for $\Omega_{\rm s} / \Omega = 0$. Again, both the
integral approximation and the asymptotic closed form accurately track the
explicit sum.

We now have the asymptotic forms of Eq.~\eqref{eq:tau_int} for large and small
spins, but we can derive an approximation valid for all spins. To do this, we
first analyze the regime where the spin is small but non-negligible. In this
regime, we approximate
\begin{equation}
    N - 2\Omega_{\rm s} / \Omega \simeq \frac{N}{N_{\max}}
        \left(N_{\max} - \frac{2\gamma_T
        \Omega_{\rm s}}{\Omega}\right)\label{eq:nmax_ansatz},
\end{equation}
for some free parameter $\gamma_T$. Using this, we can integrate
Eq.~\eqref{eq:tau_int}, and then $\gamma_T$ is fixed by requiring our
expression reproduce the large spin limit [Eq.~\eqref{eq:tau_highspin}] when
taking $\abs{\Omega_{\rm s}} \to \infty$. This procedure thus connects the two
asymptotic forms Eq.~(\ref{eq:tau_highspin}--\ref{eq:tau_lowspin}), and we
obtain
\begin{align}
    T &= T_0 \frac{f_5 (\eta_2/2)^{8/3}}{(1 - e^2)^{9/2}}
        \;\mathrm{sgn}\left(1 - \gamma_T\frac{\Omega_{\rm s}}{\eta_2\Omega}\right)
            \left|1 - \gamma_T\frac{\Omega_{\rm s}}{\eta_2\Omega}\right|^{8/3}
            \frac{\Gamma(23/3)}{4!},\label{eq:tau_approx}\\
    \gamma_T &= 4\p{\frac{4!}{\Gamma(23/3)}}^{3/8} \approx 0.691.
\end{align}
Fig.~\ref{fig:totals_s} demonstrates the accuracy of
this prediction as a function. As expected from the construction of this approximation, both
the large and small spin limits are well captured, and the scaling for
intermediate spins is also rather accurate. It is worth noting that integration
of Eq.~\eqref{eq:tau_int} is markedly more accurate than the closed form near
pseudosynchronization. Note that the integral handles the $F_{N2}$
approximately but handles the spin dependence exactly. As such, we see that the
Hansen coefficient approximations proposed in Section~\ref{s:background}
introduce less inaccuracy than the approximations used to evaluate the spin
dependence near pseudosynchronization [Eq.~\eqref{eq:nmax_ansatz}].
\begin{figure}
    \centering
    \includegraphics[width=\columnwidth]{../../scripts/eccentric_tides/1totals_ecc_0.png}
    \includegraphics[width=\columnwidth]{../../scripts/eccentric_tides/1totals_ecc_400.png}
    \caption{Tidal torque on a non-rotating (top) and rapidly rotating
    ($\Omega_{\rm s} / \Omega = 400$; bottom) star with a companion having
    orbital eccentricity $e$. Blue plus signs represent explicit summation of
    Eq.~\eqref{eq:tau_explicit_sum}, blue crosses are evaluated using the
    integral approximation Eq.~\eqref{eq:tau_int}, and the green line is
    Eq.~\eqref{eq:tau_approx}. In the large and small spin limits,
    Eq.~\eqref{eq:tau_approx} reduces to Eq.~\eqref{eq:tau_highspin} and
    Eq.~\eqref{eq:tau_lowspin} respectively. }\label{fig:totals_ecc0}
\end{figure}
\begin{figure}
    \centering
    \includegraphics[width=\columnwidth]{../../scripts/eccentric_tides/1totals_s_0_9.png}
    \caption{Tidal torque as a function of spin for a highly eccentric $e = 0.9$
    companion. Pluses represent direct summation of Hansen coefficients, crosses
    represent the integral approximation, and solid lines represent the
    analytic closed form. Blue [red] means positive [negative]
    torque on the star. The vertical line denotes $\Omega_{\rm s} = \Omega_{\rm
    p}$, where $\Omega_{\rm p}$ is given by Eq.~\eqref{eq:Wperi}. }\label{fig:totals_s}
\end{figure}

\textcolor{Corr}{Finally, the synchronization time can be computed? It's a bit
complicated, since dissipation happens in the envelope but is driven by the
core. Assuming differential rotation, synchronization is really enforced by
magnetic winding or something? Ask Dong}
\begin{equation}
    \tau_{\rm sync} = \frac{\Omega_{\rm s}}{T}kMR^2
\end{equation}

\subsection{Pseudosynchronization}

There is a single $\Omega_{\rm s}$ for which the torque, given by
Eq.~\eqref{eq:tau_approx}, vanishes, which we call the pseudosynchronization
spin frequency. It is given by
\begin{equation}
    \frac{\Omega_{\rm s, ps}}{\Omega} =
        \frac{\eta_2}{\gamma_T} = \frac{4f_2}{5\gamma_T f_5\p{1 -
        e^2}^{3/2}}.\label{eq:eta2_691}
\end{equation}
This has the expected scaling $\Omega_{\rm s, ps} \propto (1 - e)^{-3/2} \propto
\Omega_{\rm p}$. By comparison, in standard weak friction theory of equilibrium
tides, the pseudo-synchronized rotation rate is given by
\citep{alexander73, hut81}
\begin{equation}
    \frac{\Omega_{\rm s, ps}^{\rm (Eq)}}{\Omega} = \frac{f_2}{f_5\p{1 -
        e^2}^{3/2}}.\label{eq:weaktide}
\end{equation}
This differs from our Eq.~\eqref{eq:eta2_691} by a factor of $1.15$.
Figure~\ref{fig:pseudosync} compares these two predictions, as well as
$\Omega_{\rm p}$, to applying a root finding algorithm to
Eq.~\eqref{eq:tau_sum}. $\Omega_{\rm p}$ is very nearly equal to $\Omega_{\rm s,
ps}$, and both are slightly better estimates for the pseudosynchonization spin
frequency than $\Omega_{\rm s, ps}^{\rm (Eq)}$.
\begin{figure}
    \centering
    \includegraphics[width=\columnwidth]{../../scripts/eccentric_tides/1pseudosynchronous.png}
    \caption{Calculations of the pseudosynchronization spin frequencies
    $\Omega_{\rm s, sync}$ normalized by the orbital frequency $\Omega$ as a
    function of eccentricity (the scale is uniform in $1 - e^2$). The green,
    blue, and red line are given by Eqs.~\eqref{eq:Wperi},~\eqref{eq:eta2_691},
    and~\eqref{eq:weaktide} respectively. The exact solution, given in the black
    line, is obtained using finding algorithm to solve for the zero of
    Eq.~\eqref{eq:tau_sum}.}\label{fig:pseudosync}
\end{figure}

Note that, very near the pseudosynchronized spin frequency,
Eq.~\eqref{eq:tau_approx} predicts that $\rdil{T}{\Omega_{\rm s}} \approx 0$.
This is not physically accurate and is an artifact of our factorization ansatz
in Eq.~\eqref{eq:nmax_ansatz}.

\subsection{Closed Form for Energy Transfer}

We now turn our attention to Eq.~\eqref{eq:edot_in} and replace $F_{N2}$ and
$F_{N0}$ with their approximations [Eqs.~\eqref{eq:fn2_fit}
and~\eqref{eq:fn0_fit}] to obtain
\begin{align}
    \dot{E}_{\rm in} ={}&
        \frac{T_0 \Omega}{2} \int\limits_0^\infty \Big[
            C_2^2 N^5 e^{-2N/\eta_2} \mathrm{sgn}\left(N - 2\Omega_{\rm s} /
                \Omega\right) \left|N - 2 \Omega_{\rm s} /
                \Omega\right|^{8/3}\nonumber\\
            &+ 2 \p{\frac{W_{20}}{W_{22}}}^2 C_0^2 e^{-2N / \eta_0} N^{11/3}
        \Big]\;\mathrm{d}N.\label{eq:ein_int}
\end{align}
We evaluate the $m = 2$ and $m = 0$ components of this expression separately.

We first examine the $m = 2$ contribution using the same procedure that was used
in Section~\ref{ss:torque_eval} for the torque. Here, the integrand is maximized
at $N_{\max} = 23\eta_2 / 6$ when $\Omega_{\rm s} = 0$. In the large spin limit
where $\Omega_{\rm s} \gg N_{\max}\Omega / 2$, the $m = 2$ contribution
evaluates to
\begin{equation}
    \lim_{\Omega_{\rm s} \to \infty} \dot{E}_{\rm in}^{(m=2)} =
        -\frac{T_0\Omega}{2} \; \mathrm{sgn}(\Omega_{\rm s})|2\Omega_{\rm
        s}/\Omega|^{8/3} \frac{2f_2}{(1 - e^2)^6}.\label{eq:ein_highs}
\end{equation}
The small spin limit can be similarly computed; it is omitted here for brevity.
Instead, we directly seek an approximation valid for all spins, beginning by
making the approximation
\begin{equation}
    N - 2\Omega_{\rm s} / \Omega \simeq \frac{N}{N_{\max}}
        \left(N_{\max} - \frac{2\gamma_E
        \Omega_{\rm s}}{\Omega}\right)\label{eq:nmax_ansatz},
\end{equation}
where $\gamma_E$ is a free parameter. This lets us integrate the $m = 2$
component of Eq.~\eqref{eq:ein_int} analytically, and we can constrain
$\gamma_E$ by requiring the integral agree with Eq.~\eqref{eq:ein_highs}. This
yields
\begin{align}
    \dot{E}_{\rm in}^{(m=2)}
        ={}& \frac{T_0\Omega f_5 (\eta_2/2)^{11/3}}{2(1-e^2)^{9/2}}
            \frac{\Gamma(26/3)}{4!}\nonumber\\
        &\times \mathrm{sgn}\p{1 - \gamma_E\frac{\Omega_s}{\eta_2 \Omega}}
            \left|1 - \gamma_E\frac{\Omega_s}{\eta_2 \Omega}\right|^{8/3},\\
    \gamma_E ={}& \p{\frac{5! 2^{16/3}}{\Gamma\p{26/3}}}^{3/8}
        \approx 0.5886.
\end{align}

The $m = 0$ contribution to Eq.~\eqref{eq:ein_int} is much more straightforward
and can be directly integrated using the parameterization
Eq.~\eqref{eq:fn0_fit}. So we obtain the total energy transfer rate
\begin{align}
    \dot{E}_{\rm in} ={}& \frac{T_0 \Omega}{2}\Bigg[\frac{f_5
        (\eta_2/2)^{11/3}}{(1-e^2)^{9/2}} \frac{\Gamma(26/3)}{4!}\nonumber\\
        &\times \mathrm{sgn}\p{1 - \gamma_E\frac{\Omega_s}{\eta_2 \Omega}}
            \left|1 - \gamma_E\frac{\Omega_s}{\eta_2 \Omega}\right|^{8/3}
            \nonumber\\
        &+
    \frac{f_5 \Gamma(14 / 3)}{(1 - e^2)^{10}} \left(\frac{3}{2}\right)^{8/3}
            \left(\frac{e^2 f_3}{f_5}\right)^{11/6}\Bigg].
            \label{eq:ein_dot_tot}
\end{align}

In Fig.~\ref{fig:e0}, we show the agreement of Eq.~\eqref{eq:ein_dot_tot} in the
large and small spin limits, where good agreement is observed both with the
integral form Eq.~\eqref{eq:ein_int} and the closed form
Eq.~\eqref{eq:ein_dot_tot}. In Fig.~\ref{fig:e_spin}, we show the energy
transfer rate as a function of spin. The agreement degrades near
pseudosynchronization but generally captures the correct scaling. Again,
explicit integration is visibly more accurate than the analytic closed form.
\begin{figure}
    \centering
    \includegraphics[width=\columnwidth]{../../scripts/eccentric_tides/1totals_e_400.png}
    \includegraphics[width=\columnwidth]{../../scripts/eccentric_tides/1totals_e_0.png}
    \caption{Plot of $\dot{E}_{\rm in}$ for a rapidly rotating ($\Omega_{\rm s}
    / \Omega = 400$; top) and non-rotating (bottom) star. Black pluses represent
    explicit summation of the Hansen coefficients, blue crosses the integral
    form Eq.~\eqref{eq:ein_int}, and the green line the closed form
    Eq.~\eqref{eq:ein_dot_tot}. }\label{fig:e0}
\end{figure}
\begin{figure}
    \centering
    \includegraphics[width=\columnwidth]{../../scripts/eccentric_tides/1totals_NRG_e_0_9.png}
    \caption{$\dot{E}_{\rm in}$ as a function of spin for a highly eccentric $e
    = 0.9$ companion. Pluses represent direct summation of Hansen coefficients,
    crosses represent the integral approximation, and solid lines represent the
    analytic closed form. Blue [red] means positive [negative] energy transfer
    into the spin of the star. The vertical line denotes $\Omega_{\rm s} =
    \Omega_{\rm p}$, where $\Omega_{\rm p}$ is given by Eq.~\eqref{eq:Wperi}.
    }\label{fig:e_spin}
\end{figure}

\textcolor{Corr}{With this, a circularization time can also be computed:}
\begin{equation}
    \tau_{\rm circ} = XXXX.
\end{equation}

\section{Example System: PSR J0045+7319}\label{s:j00457319}

As an example of our calculations above, we consider PSR J0045-7319 binary
system \citep{bell1995psr}. The system was initially reported to have pulsar
mass $M_2 = 1.4M_{\odot}$, $q = 6.3$, $e = 0.808$, orbital period $P =
51.17\;\mathrm{days}$, and $\dot{P} = -3.03\times 10^{-7}$
\citep{kaspi1996params}. This gives a mass of $M = 8.8M_{\odot}$ for the massive
star. Furthermore, the measured luminosity $L = 1.2 \times 10^4L_{\odot}$ and
surface temperature $T_{\rm surf} = (24000 \pm 1000)\;\mathrm{K}$ give the
radius of the star to be $R = 6.4R_{\odot}$ \citep{kaspi1996params}.

\subsection{Spin of the Massive Star}

From these system parameters, we can also calculate the orbital separation $a =
126R_{\odot}$. The internal structure of the star can be obtained by comparison
to detailed stellar structure calculations, and yield $M_{\rm c} \approx
3M_{\odot}$ and $r_{\rm c} \approx 1.38R_{\odot}$ \citep{kumar1998}. To account
for uncertainties in the stellar structure, we take this $M_{\rm c}$ to be fixed
and consider a range of $r_{\rm c} \in [0.7, 1.5]$. With these parameters, the
spin of the massive star can be computed using our calculations above. By
conservation of energy, $\dot{E}_{\rm in} + \dot{E}_{\rm g} = 0$, where
$\dot{E}_{\rm g}$ is the change in the gravitational binding energy, given by
\begin{equation}
    \dot{E}_{\rm g} = \frac{GqM_2^2}{3a}\frac{\dot{P}}{P}.
\end{equation}

To relate $\dot{E}_{\rm in}$ and $\Omega_{\rm s}$ to $\dot{P}$, we need to make
assumptions about the ratio $\rho_{\rm c} / \bar{\rho}_{\rm c}$, which can only
be obtained via stellar structure simulations. We take $\rho_{\rm c} /
\bar{\rho}_{\rm c} \approx 1/3$ as a fiducial value (though in reality this
likely varies with $r_{\rm c}$). Figure~\ref{fig:j0045_fid} shows
$\abs{\dot{P}}$ as a function of $\Omega_{\rm s}$, using
Eq.~\eqref{eq:ein_dot_tot}, evaluated using four different $r_{\rm c}$. The
measured $\dot{P}$ is shown by the horizontal dashed line. Substantial
retrograde rotation ($\abs{\Omega_{\rm c}} \gg \Omega_{\rm p},
\sqrt{GM/R^3}$) is required to match the predictions. Note furthermore that for the
most compact core radius $r _{\rm c} = 0.7R_{\odot}$, there are no solutions for
$\Omega_{\rm s}$; even a maximally spinning core cannot generate enough tidal
dissipation to match the observed $\dot{P}$.
\begin{figure}
    \centering
    \includegraphics[width=\columnwidth]{../../scripts/eccentric_tides/1_7319_disps.png}
    \caption{$\dot{P}$ as a function of $\Omega_{\rm s}$ for the canonical
    parameters for J0045-7319, as evaluated by explicit summation of
    Eq.~\eqref{eq:ein_explicit_sum}, for four different values of $r_{\rm c}$
    (legend, in units of $R_{\odot}$). The measured $\dot{P} = -3.03\times
    10^{-7}$ is shown by the horizontal dashed line. The vertical shaded region
    is the region where $\Omega_{\rm s}$ is less than the breakup rotation rate
    of the star as a whole, given by $\sqrt{GM / R^3}$. Each $r_{\rm c}$ is only
    shown for $\abs{\Omega_{\rm s}} \leq \Omega_{\rm s, c}$ the \emph{core}
    breakup rotation rate.
    }\label{fig:j0045_fid}
\end{figure}

\subsection{Constraints on Stellar Structure}

As discovered above, requiring that the observed $\dot{P}$ lie within the
range attainable via tidal dissipation imposes constraints on $r_{\rm c}$. We
can compute the range of attainable $\dot{P}$ by setting the spin frequency
equal to the breakup frequency in Eq.~\eqref{eq:ein_int}, giving
\begin{equation}
    \dot{P} \lesssim
        -\frac{6\pi}{q}\beta_2 \left(\frac{r_{\rm c}}{a}\right)^5
        \frac{\rho_{\rm c}}{\bar{\rho}_{\rm c}} \left(1 -
        \frac{\rho_{\rm c}}{\bar{\rho}_{\rm c}}\right)^2 2^{8/3}\frac{f_2}{(1 -
        e^2)^6}.\label{eq:dotp_rc}
\end{equation}
For J0045+7319, requiring that $\dot{P}$ equal its observed value gives $r_{\rm
c} \gtrsim 0.93R_{\odot}$ when using $\rho_{\rm c} / \bar{\rho}_{\rm c} = 1/3$.
Indeed, in Fig.~\ref{fig:j0045_fid}, there is a solution equal to the observed
$\dot{P}$ when $r_{\rm c} \geq R_{\odot}$ but not when $r_{\rm c} =
0.7R_{\odot}$. The strong dependence of the right hand side on
Eq.~\eqref{eq:dotp_rc} on $r_{\rm c}$ suggests that this constraint is somewhat
insensitive to other system uncertanties.

% \subsection{Stellar Structure Simulations}

% In previous literature, the primary star was initially taken to have $M =
% 8.8M_{\odot}$ \citep{kumar1998,lai1996}, by taking $M_2 \simeq 1.4M_{\odot}$
% characteristic mass for NSs and multiplying by the observed $q$. A later, but
% still very old, study proposed $M \approx 10M_{\odot}$ by comparison with
% stellar models \citep{thorsett1999neutron}. Using MESA to generate updated
% stellar models (TODO all citations), we compute an updated $M$ estimate, giving
% detailed stellar structure measurements.

% At the level of approximation of this paper, our procedure is as follows: we
% use a few different values of convective overshooting and mixing, and
% metallicities, and compute stellar structures for both both non-rotating and
% highly rotating $\sim 0.95\Omega_{\rm s, c}$ stars. We find that, in general, to
% match the observed $T, L$ we must let the star evolve to $\sim 80\%$ of the way
% to complete core hydrogen depletion. We then sample a range of stellar masses,
% and find the stellar mass that best reproduces the observed $T, L$.

% For all of these systems, $r_{\rm c} \lesssim R_{\odot}$, in tension with our
% bound above.

\section{Conclusion and Discussion}\label{s:disc}

\textcolor{Corr}{This section is not complete.}

The primary results of the paper are approximate expressions for the torque,
Eq.~\eqref{eq:tau_approx}, and the energy dissipation rate,
Eq.~\eqref{eq:ein_dot_tot}, in closed form.

\begin{itemize}
    \item Thanks to some references (Barker \& Ogilvie, my work), there seems to
        be some evidence that hydrodynamic wave breaking could cause all IGW to
        break and not reflect, once the pericenter wave reaches nonlinear
        amplitudes.

    \item As noted in the text, the approximate forms enforce
        $\rdil{T}{\Omega_{\rm s}} = 0$, which the actual torque does not satisfy.
        This introduces some slight errors in the exact value of the torque very
        near pseudosynchronization.

    \item It is pointed out that the $v \sin i$ for J0045-7319 is $110 \pm 10
        \;\mathrm{km/s}$, while the breakup frequency is $\sqrt{GM / R} =
        512\;\mathrm{km/s}$. Thus, it is expected that the stellar surface is
        rotating at an appreciable fraction of breakup.
\end{itemize}

\section{Acknowledgements}

We thank Michelle Vick, Christopher O'Connor, and Matteo Cantiello for fruitful
discussions. YS is supported by the NASA FINESST grant 19-ASTRO19-0041.

\bibliographystyle{mnras}
\bibliography{Su_eccentric_tides}

% \clearpage
% \onecolumn

\bsp
\label{lastpage} % chktex 24
\end{document}
