    \documentclass[
        fleqn,
        usenatbib,
        referee,
    ]{mnras}
    \usepackage{
        amsmath,
        amssymb,
        newtxtext,
        newtxmath,
        graphicx,
        ae, aecompl,
        booktabs,
        caption,
        subcaption,
    }
    \usepackage[T1]{fontenc}
    \captionsetup{compatibility=false}

    \newcommand*{\rd}[2]{\frac{\mathrm{d}#1}{\mathrm{d}#2}}
    \newcommand*{\pd}[2]{\frac{\partial#1}{\partial#2}}
    \newcommand*{\rdil}[2]{\mathrm{d}#1 / \mathrm{d}#2}
    \newcommand*{\pdil}[2]{\partial#1 / \partial#2}
    \newcommand*{\rtd}[2]{\frac{\mathrm{d}^2#1}{\mathrm{d}#2^2}}
    \newcommand*{\ptd}[2]{\frac{\partial^2 #1}{\partial#2^2}}
    \newcommand*{\md}[2]{\frac{\mathrm{D}#1}{\mathrm{D}#2}}
    \newcommand*{\pvec}[1]{\vec{#1}^{\,\prime}}
    \newcommand*{\svec}[1]{\vec{#1}\;\!}
    \newcommand*{\bm}[1]{\boldsymbol{\mathbf{#1}}}
    \newcommand*{\uv}[1]{\hat{\bm{#1}}}
    \newcommand*{\ang}[0]{\;\text{\AA}}
    \newcommand*{\mum}[0]{\;\upmu \mathrm{m}}
    \newcommand*{\at}[1]{\left.#1\right|}
    \newcommand*{\bra}[1]{\left<#1\right|}
    \newcommand*{\ket}[1]{\left|#1\right>}
    \newcommand*{\abs}[1]{\left|#1\right|}
    \newcommand*{\ev}[1]{\langle#1\rangle}
    \newcommand*{\p}[1]{\left(#1\right)}
    \newcommand*{\s}[1]{\left[#1\right]}
    \newcommand*{\z}[1]{\left\{#1\right\}}

    \DeclareMathOperator*{\sgn}{sgn}
    \DeclareMathOperator*{\argmax}{argmax}

\title[Eccentric Dynamical Tides]{Eccentric Dynamical Tides}
\author[Y. Su, D. Lai.]{
Yubo Su$^1$,
Dong Lai$^1$
\\
$^1$ Cornell Center for Astrophysics and Planetary Science, Department of
Astronomy, Cornell University, Ithaca, NY 14853, USA
}

\date{Accepted XXX\@. Received YYY\@; in original form ZZZ}

\pubyear{2020}

\begin{document}\label{firstpage}
\pagerange{\pageref{firstpage}--\pageref{lastpage}}
\maketitle

\begin{abstract}
    Abstract
\end{abstract}

\begin{keywords}
keywords % chktex 8
\end{keywords}

\section{Introduction}

Many massive stars are in binaries. Likely progenitors of BNS, HMXRB\@. Open
question: after first SN, does the binary circularize before second SN\@?
Dynamical tides in the radiative envelope are likely to be important, but are
hard to calculate in eccentric binaries.

The dissipation due to the dynamical tide in a massive star's envelope under the
influence of a \emph{circular} perturber is well understood \citep{kushnir}.

It is also well known that the perturbing potential of an eccentric companion
can be decomposed into Fourier harmonics \citep[e.g.][]{sl, vlf}.

In this paper, we show that the dynamical tide due to an eccentric companion can
be approximated in closed form for a wide range of physical parameters. In
Section~\ref{s:background}, we review the relevant equations. In
Section~\ref{s:hansens}, we present novel approximations for the Hansen
coefficients. In Section~\ref{s:eval}, we apply these approximations to obtain
closed forms for the torque and energy transfer rate in the binary. In
Section~\ref{s:j00457319}, we apply our theory to observations. In
Section~\ref{s:disc}, we discuss and conclude.

\section{Summary of Existing Work}\label{s:background}

\subsection{Decomposition of Perturbation from an Eccentric Companion}

Consider a star subject to the perturbing potential of a companion star with
mass $M_2$, For a general eccentric orbit, the potential to quadrupolar order
can be decomposed into a sum over circular orbits \citep{sl,vlf}:
\begin{align}
    U &= \sum\limits_m U_{2m} \p{\vec{r}, t},\label{eq:u_ecc}\\
    U_{2m}\p{\vec{r}} &= -\frac{GM_2 W_{2m} r^2}{D(t)^3}
            e^{-imf(t)} Y_{2m}(\theta, \phi),\nonumber\\
        &= -\frac{GM_2W_{2m}r^2}{a^3}Y_{2m}\p{\theta, \phi}
            \sum\limits_{N = -\infty}^\infty F_{Nm}e^{-iN\Omega t},\\
    F_{Nm} &= \frac{1}{\pi}\int\limits_{0}^{\pi}
        \frac{\cos\s{N\p{E - e\sin E} - mf(E)}}
            {\p{1 - e\cos E}^2}\;\mathrm{d}E.
\end{align}
We denote $W_{2 \pm 2} = \sqrt{3\pi/10}$, $W_{2 \pm 1} = 0$, $W_{20} =
-\sqrt{\pi / 5}$, $D(t)$ the instantaneous distance between the star and
companion, $f(t)$ the true anomaly, $Y_{lm}$ the spherical harmonics, and
$\Omega$ the mean motion of the companion. Note that $F_{Nm}$ are the
\emph{Hansen coefficients} for $l = 2$. The total torque on the star, energy
transfer in the inertial frame, and heating in the star's corotating frame are
given respectively
\citep{vlf}:
\begin{align}
    \tau &= \sum\limits_{N = -\infty}^\infty F_{N2}^2
        \hat{\tau}\p{\omega =
        N\Omega - 2\Omega_s},\label{eq:tau_each}
        \\
    \dot{E}_{\rm in} &=
        \frac{1}{2}\sum\limits_{N = -\infty}^\infty\s{
            \p{\frac{W_{20}}{W_{22}}}^2 N\Omega F_{N0}^2
            \hat{\tau}\p{\omega = N\Omega}
            + N\Omega F_{N2}^2 \hat{\tau}\p{\omega =
            N\Omega - 2\Omega_s}},\label{eq:edot_in}\\
    \dot{E}_{\rm rot} &= \dot{E}_{\rm in} - \Omega_s \tau \label{eq:edot_rot},
\end{align}
where $\hat{\tau}(\omega)$ is the torque exerted by a perturber on a circular
trajectory with orbital frequency $\omega$. Compared to \citet{vlf}, we use
$\tau_N(\omega) = T_0 \sgn(\omega) \hat{F}(\abs{\omega})$, for better
integration with the next section.

\subsection{Tidal Torque in Massive Stars}

For a circular orbit with orbital frequency $\omega$ and fixed semimajor axis
$a$, the tidal torque exerted on the star by the companion is given by
\citealt{kushnir}:
\begin{align}
    \tau(\omega) &= \hat{\tau}(r_c, \omega) \sgn\p{1 - \frac{2\Omega_s}{\omega}}
        \abs{1 - \frac{2\Omega_s}{\omega}}^{8/3}
            \label{eq:kushnir_torque},\\
    \hat{\tau}(r_c, \omega) &= \frac{GM_2^2r_c^5}{a^6}
        \p{\frac{\omega}{\sqrt{GM_c/r_c^3}}}^{8/3}
        \s{\frac{r_c}{g_c}\p{\rd{N^2}{\ln r}}_{r = r_c}}^{-1/3}
            \frac{\rho_c}{\bar{\rho}_c} \p{1 - \frac{\rho_c}{\bar{\rho}_c}}^2
            \s{\frac{3}{2}\frac{3^{2/3}\Gamma^2(1/3)}{5 \cdot
                6^{4/3}} \frac{3}{4\pi}\alpha^2},\nonumber\\
        &\equiv \beta_2\frac{GM_2^2r_c^5}{a^6}
            \p{\frac{\omega}{\sqrt{GM_c/r_c^3}}}^{8/3}
            \frac{\rho_c}{\bar{\rho}_c} \p{1 - \frac{\rho_c}{\bar{\rho}_c}}^2.
\end{align}
Here, $1 - \frac{2\Omega_s}{\omega}$ is the dimensionless pattern frequency,
$\alpha$ is defined in Equation A32 of \citealt{kushnir}, $r_c$ is the radius of
the core, $M_c$ the mass of the core, $g_c$ is the gravitational acceleration at
the radiative-convective boundary (RCB), $N^2$ is the Brunt-Vaisala frequency,
$r$ is the radial coordinate within the star, $\rho_c$ is the density at the
RCB, $\bar{\rho}_c$ is the average density of the convective core, and $\beta_2
\approx 1$ is a good approximation for a large range of stellar models
\citep{kushnir}.

\subsection{Objective of This Paper}

Putting the two results from the previous two sections together, we can
immediately obtain the tidal torque due to an eccentric companion:
\begin{equation}
    \tau = \sum_{N = -\infty}^{N = \infty} F_{N2}^2 \tau_N,
\end{equation}
where
\begin{equation}
    \tau_N(r_c) = \hat{\tau}(r_c) \mathrm{sgn}\left(N - \frac{2\Omega_s}{
        \Omega}\right) \left|N - \frac{2 \Omega_s}{\Omega}\right|^{8/3},
\end{equation}
and the corresponding energy transfer rate
\begin{equation}
    \dot{E}_{\rm in} = \frac{\hat{\tau}(r_c, \Omega)}{2}
        \sum_{N = -\infty}^{\infty} \left[
            N\Omega F_{N2}^2 \mathrm{sgn}\left(N - 2\Omega_s / \Omega\right)
                    \left|N - 2 \Omega_s / \Omega\right|^{8/3}
            + \left(\frac{W_{20}}{W_{22}}\right)^2 \Omega
                    F_{N0}^2 |N|^{11/3}
            \right].
\end{equation}
While exact, these expressions are difficult to evaluate for larger
eccentricities, where one often must sum hundreds or thousands of terms. In the
subsequent sections, we present closed-form approximations to the above
equations.

\section{Approximating Hansen Coefficients}\label{s:hansens}

\subsection{$m=2$ Hansen Coefficient Behavior at High Eccentricity}

Recall that the Hansen coefficients are defined as the Fourier series coffecients
of part of the disturbing function
\begin{equation}
    \frac{a^3}{D(t)^3} e^{-imf} = \sum\limits_{N = -\infty}^\infty
        F_{Nm} e^{-iN\Omega t}.\label{eq:hansen_series}
\end{equation}
Observe that $F_{(-N)m} = F_{N(-m)}$. We further observe the following facts about
the Hansen coefficients $F_{N2}$:
\begin{itemize}
    \item For substantial eccentricies, $F_{N2}$ has only one substantial peak.
        The only characteristic scale for $N$ is $N_{\rm p}$ the pericenter
        harmonic, so indeed we find the peak of $F_{N2}$ occurs at $\sim N_{\rm
        p}$ (see Fig.~\ref{fig:hansens}). Furthermore, for $N < 0$, $F_{N2}
        \approx 0$ to good accuracy.

    \item Since the left hand side of Eq.~\eqref{eq:hansen_series} is smooth in
        time, the Fourier coefficients must fall off exponentially for
        sufficiently large $N$.

    \item Since there are no characteristic timescales between $\Omega$ and
        $\Omega_{\rm p}$, we anticipate the Hansen coefficients must be scale
        free between $N = 1$ and $N_{\rm p}$, i.e.\ a power law.
\end{itemize}
As such, we make ansatz for the scalings of the Hansen coefficients for $N \geq
0$:
\begin{equation}
    F_{N2} \approx C_2 N^{p}e^{-N/\eta_2},\label{eq:fn2_fit}
\end{equation}
Eq.~\eqref{eq:fn2_fit} has the advantage that $N_{\max} = p\eta_2$
immediately. Since we expect the Hansen coefficients to peak at $N \sim
\Omega_{\rm p} / \Omega$, where
\begin{equation}
    \Omega_{\rm p} \equiv \simeq \Omega \frac{\sqrt{1 + e}}{\p{1 - e}^{3/2}},
\end{equation}
is the pericenter orbital frequency, we expect $\eta_2 \propto \p{1 - e}^{-3/2}$
as well.

Note that at moderate eccentricities $e \lesssim 0.7$, $p$ is very poorly
constrained, since only the harmonics $N \lesssim N_{\max}$ contribute to the
fit. Thus, we fix $p = 2$ by fitting $F_{N2}$ for large eccentricities ($e
\gtrsim 0.9$) and assume it is universal. We found indeed this proves robust to
smaller eccentricities\footnote{There is some numerical justification for $p$
being exactly $2$. We note the function $\p{a/D(t)}^3e^{-imf}$ resembles the
second derivative of a Gaussian: as it is sharply peaked at $t = 0$, but goes
negative at some small, positive time, and only becomes positive again near $t =
P$ the orbital period. A Gaussian with vanishing width (corresponding to $e \to
1$) approximates the Dirac delta, which has a flat frequency spectrum. This
implies the second derivative of the Dirac delta has Fourier coefficients with
increasing magnitude $\propto N^2$. Thus, as $e \to 1$, we expect the $F_{Nm}$
to also scale like $N^2$ for $N \ll \Omega_{\rm p} / \Omega$.}.

To constrain the remaining two free parameters $\eta_2$ and $C_2$ the
normalization, we use the well known Hansen coefficient homements
\citep{hut,sl,vlf}:
\begin{align}
    \sum\limits_{N = -\infty}^\infty F_{N2}^2
        &= \frac{1 + 3e^2 + 3e^4/8}{\p{1 - e^2}^{9/2}}
            \equiv \frac{f_5}{\p{1 - e^2}^{9/2}},\\
    \sum\limits_{N = -\infty}^\infty F_{N2}^2N
        &= \frac{2}{\p{1 - e^2}^6}\p{1 + \frac{15e^2}{2}
            + \frac{45 e^4}{8} + \frac{5e^6}{16}},\\
        &= \frac{2f_2}{\p{1 - e^2}^6}.
\end{align}
This fixes
\begin{align}
    \eta_2 &= \frac{4f_2}{5f_5\p{1 - e^2}^{3/2}},\\
    C_2^2\eta_2^5 &= \frac{4f_5}{3\p{1 - e^2}^{9/2}}.
\end{align}
Note that indeed $\eta_2 \propto \p{1 - e^2}^{-3/2}$. The agreement of this fit
of the Hansen coefficients can be seen in Fig.~\ref{fig:hansens}.
\begin{figure}
    \centering
    \includegraphics[width=\columnwidth]{../../scripts/eccentric_tides/hansens.png}
    \caption{Plot of Hansen coefficients $F_{N2}$, where dotted lines denote
    negative values. The green line is the fitted function of form
    Eq.~\eqref{eq:fn2_fit}.}\label{fig:hansens}
\end{figure}

\subsection{$m = 0$ Hansen Coefficient Behavior at High Eccentricity}

These coefficients have only one characteristic scale in harmonic space, namely
$\Omega / \Omega_{\rm p}$, but must be symmetric, therefore, the natural
ansatz is of form
\begin{equation}
    F_{N0} = C_0 e^{-\abs{N} / \eta_0}.\label{eq:fn0_fit}
\end{equation}
Here, again, we expect $\eta_0 \simeq \Omega / \Omega_{\rm p}$. The two free
parameters $C_0$ and $\eta_0$ are again fixed by the well known moments
\citep{hut,sl,vlf}:
\begin{align}
    \sum\limits_{N = -\infty}^\infty F_{N0}^2 &= \frac{f_5}{\p{1 - e^2}^{9/2}}
        ,\\
    \sum\limits_{N = -\infty}^\infty F_{N0}^2 N^2
        &= \frac{9e^2}{2\p{1 - e^2}^{15/2}}
            f_3.
\end{align}
Integration of the ansatz fixes the two free parameters
\begin{align}
    \eta_0^2 &= \frac{9e^2f_3}{\p{1 - e^2}^{3}f_5},\\
    C_0^2\eta_0 &= \frac{f_5}{\p{1 - e^2}^{9/2}}.
\end{align}

\begin{figure}
    \centering
    \includegraphics[width=0.9\columnwidth]{../../scripts/eccentric_tides/hansens/hansens0_90.png}
    \caption{$F_{N0}$ for $e = 0.9$, with fit [Eq.~\eqref{eq:fn0_fit}]
    overlaid.}\label{fig:fn0_fit}
\end{figure}

\section{Evaluating Torque and Energy Transfer}\label{s:eval}

\subsection{Tidal Torque}

To obtain a closed form for the tidal torque experienced by a massive star due
to an eccentric companion, as a first estimate we can use
Eq.~\eqref{eq:kushnir_torque} as the torque generated by each mode in
Eq.~\eqref{eq:tau_each}. This gives
\begin{equation}
    \tau = \hat{T}(r_c, \Omega) \sum\limits_{N = -\infty}^\infty
        F_{N2}^2 \sgn\p{N - 2\frac{\Omega_s}{\Omega}}
            \abs{N - 2\frac{\Omega_s}{\Omega}}^{8/3}.\label{eq:tau_sum}
\end{equation}
Using our integral approximations, this immediately becomes
\begin{equation}
    \tau = \hat{\tau} \int_0^\infty C_2^2 N^4 e^{-2N / \eta_2}
        \sgn\left(N - 2\Omega_s / \Omega\right) \left|N - 2 \Omega_s /
            \Omega\right|^{8/3}\;\mathrm{d}N.\label{eq:tau_int}
\end{equation}
Note that strictly speaking, this is an upper bound on the tidal torque, as it
assumes all excited IGWs damp effectively, but there is evidence this bound may
be saturated (see Section~\ref{s:disc}).

Towards evaluating Eq.~\eqref{eq:tau_sum}, we first consider the limit
$\abs{\Omega_s} \to \infty$ for simplicity. This requires only the known Hansen
coefficient moments, and we obtain
\begin{equation}
    \tau = -\hat{\tau} \sgn (\Omega_s)\;\left|2 \Omega_s /
        \Omega\right|^{8/3} \frac{f_5}{(1 - e^2)^{9/2}}.
\end{equation}

Next, we consider the case where $N_{\max} \gtrsim 2\Omega_s/\Omega$, where
$N_{\max}$ is the $N$ for which the magnitude of the summand in
Eq.~\eqref{eq:tau_sum} is largest. We make the ansatz that $N -
2\Omega_s/\Omega$ can be effectively factorized as
\begin{equation}
    N - 2\Omega_s / \Omega \simeq \frac{N}{N_{\max}}
        \left(N_{\max} - \frac{2\alpha \Omega_s}{\Omega}\right),
\end{equation}
In this case, $N_{\max} = 10\eta_2 / 3$. $\alpha$ is constrained by requiring
our expression reproduce the high spin limit when taking $\abs{\Omega_s} \to
\infty$. Then, the torque becomes
\begin{equation}
    \tau = \hat{\tau} \frac{f_5 \eta_2^{8/3}}{(1 - e^2)^{9/2}}
        \;\mathrm{sgn}\left(1 - 0.691\frac{\Omega_s}{\eta_2\Omega}\right)
            \left|1 - 0.691\frac{\Omega_s}{\eta_2\Omega}\right|^{8/3}
            \frac{\Gamma(23/3)}{4!} \frac{1}{2^{8/3}}.\label{eq:tau_approx}
\end{equation}

See Figs.~\ref{fig:totals_ecc0},~\ref{fig:totals_ecc400},
and~\ref{fig:totals_s} for the accuracy of this prediction.
\begin{figure}
    \centering
    \includegraphics[width=\columnwidth]{../../scripts/eccentric_tides/1totals_ecc_0.png}
    \caption{Tidal torque on a non-rotating star with a companion having
    orbital eccentricity $e$. Blue plus signs represent explicit summation of
    Eq.~\eqref{eq:tau_sum}, blue crosses are evaluated using the integral
    approximation Eq.~\eqref{eq:tau_int}, while the green dashed line is
    Eq.~\eqref{eq:tau_approx}.}\label{fig:totals_ecc0}
\end{figure}
\begin{figure}
    \centering
    \includegraphics[width=\columnwidth]{../../scripts/eccentric_tides/1totals_ecc_400.png}
    \caption{Same as Fig.~\ref{fig:totals_ecc0} but for $\Omega_{s} \gg
    \Omega_{\rm p}$.}\label{fig:totals_ecc400}
\end{figure}
\begin{figure}
    \centering
    \includegraphics[width=\columnwidth]{../../scripts/eccentric_tides/1totals_s_0_9.png}
    \caption{Tidal torque as a function of spin for a highly eccentric $e = 0.9$
    companion. Pluses represent direct summation of Hansen coefficients, crosses
    represent the integral approximation, while solid lines represent the
    analytic closed form. Blue [red] means positive [negative]
    torque.}\label{fig:totals_s}
\end{figure}

\subsection{Pseudosynchronization}

In fact, Eq.~\eqref{eq:tau_approx} gives a very concrete prediction for the
pseudosynchronization frequency:
\begin{equation}
    \frac{\Omega_{\rm s, sync}}{\Omega} =
        \frac{\eta_2}{0.691}.\label{eq:eta2_691}
\end{equation}
This has the expected scaling $\Omega_{\rm s, sync} \propto (1 - e^2)^{-3/2}$,
but is more accurate than $\Omega_{\rm s, sync} \simeq \Omega_{\rm p}$, as seen
in Fig.~\ref{fig:pseudosync}.
\begin{figure}
    \centering
    \includegraphics[width=\columnwidth]{../../scripts/eccentric_tides/1pseudosynchronous.png}
    \caption{Pseudosynchronization frequency. Integral form is obtained by
    performing root finding using the integral form for the torque. We see that
    Eq.~\eqref{eq:eta2_691} is a very good approximation.}\label{fig:pseudosync}
\end{figure}

\subsection{Closed Form for Energy Transfer}

The integral form is
\begin{equation}
    \dot{E}_{\rm in} =
        \frac{\hat{\tau}(r_c, \Omega) \Omega}{2} \int\limits_0^\infty \left[
            C_2^2 N^5 e^{-2N/\eta_2} \mathrm{sgn}\left(N - 2\Omega_s /
                \Omega\right) \left|N - 2 \Omega_s / \Omega\right|^{8/3}
            + 2 \frac{2}{3} C_0^2 e^{-2N / \eta_0} N^{11/3}
        \right]\;\mathrm{d}N.\label{eq:heating_int}
\end{equation}
This comes out to be (elaborate)
\begin{align}
    \dot{E}_{\rm in} ={}& \frac{\hat{\tau}\Omega}{2}\Bigg[
    \sgn\left(1 - 0.5886\frac{\Omega_s}{\eta_2 \Omega}\right)\;\left|1 -
        0.5886\frac{\Omega_s}{\eta_2 \Omega}\right|^{8/3}
            \frac{f_5}{(1 - e^2)^{9/2}}
            \frac{\Gamma(26/3)}{4!} \left(\frac{\eta_2}{2}\right)^{11/3}
            \nonumber\\
        &+
    \frac{f_5 \Gamma(14 / 3)}{(1 - e^2)^{10}} \left(\frac{3}{2}\right)^{8/3}
            \left(\frac{e^2 f_3}{f_5}\right)^{11/6}\Bigg].
            \label{eq:total_heating}
\end{align}

We make plots in the two $\Omega_s$ regimes as a function of eccentricity in
Figs.~\ref{fig:e0} and~\ref{fig:e400}. Agreement is good again.
\begin{figure}
    \centering
    \includegraphics[width=\columnwidth]{../../scripts/eccentric_tides/1totals_e_0.png}
    \caption{Plot of $\dot{E}_{\rm in}$ for a non-rotating star. Blue pluses
    represent explicit summation of the Hansen coefficients, crosses the
    integral form Eq.~\eqref{eq:heating_int}, and the green dashed line the
    closed form Eq.~\eqref{eq:total_heating}.}\label{fig:e0}
\end{figure}
\begin{figure}
    \centering
    \includegraphics[width=\columnwidth]{../../scripts/eccentric_tides/1totals_e_400.png}
    \caption{Same as Fig.~\ref{fig:e0} but for a rapidly spinning
    star.}\label{fig:e400}
\end{figure}

\section{Example System: J0045+7319}\label{s:j00457319}

We can arrive at an upper bound for the $\dot{P}$ by setting the spin frequency
equal to the breakup frequency. Then, taking the correct parameters and
evaluating $\dot{E}_{\rm in}$, we obtain
\begin{equation}
    \frac{\dot{P}}{2\pi} \lesssim
        -\frac{3}{(1 + q)^2}\beta_2 \left(\frac{r_c}{a}\right)^5
        \frac{\rho_c}{\bar{\rho}_c} \left(1 -
        \frac{\rho_c}{\bar{\rho}_c}\right)^2 2^{8/3}\frac{f_2}{(1 - e^2)^6}.
\end{equation}
This comes out to be $r_c \gtrsim 1.799R_{\odot}$ for J0045+7319, larger than
the prior values in the literature and much larger than MESA models.

\section{Conclusion and Discussion}\label{s:disc}

The primary results of the paper are Eq.~\eqref{eq:tau_approx}, shown in
Figs.~\ref{fig:totals_ecc0},~\ref{fig:totals_ecc400},
and~\ref{fig:totals_s} to be reasonably accurate across a range of spins
and eccentricities. The energy dissipation rate is also computed using similar
techniques and show good agreement (see Figs.~\ref{fig:e0}
and~\ref{fig:e400}).

\begin{itemize}
    \item Thanks to some references (Barker \& Ogilvie, my work), there seems to
        be some evidence that hydrodynamic wave breaking could cause all IGW to
        break and not reflect, once the pericenter wave reaches nonlinear
        amplitudes.

    \item As noted in the text, the approximate forms enforce
        $\rdil{\tau}{\Omega_s} = 0$, which the actual torque does not satisfy.
        This introduces some slight errors in the exact value of the torque very
        near pseudosynchronization.
\end{itemize}


\bibliographystyle{mnras}
\bibliography{Su_eccentric_tides}

% \clearpage
% \onecolumn

\bsp
\label{lastpage} % chktex 24
\end{document}
