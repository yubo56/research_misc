    \documentclass[
        fleqn,
        usenatbib,
        referee,
    ]{mnras}
    \usepackage{
        amsmath,
        amssymb,
        newtxtext,
        newtxmath,
        graphicx,
        ae, aecompl,
        booktabs,
        caption,
        subcaption,
    }
    \usepackage[T1]{fontenc}
    \captionsetup{compatibility=false}

    \newcommand*{\rd}[2]{\frac{\mathrm{d}#1}{\mathrm{d}#2}}
    \newcommand*{\rtd}[2]{\frac{\mathrm{d}^2#1}{\mathrm{d}#2^2}}
    \newcommand*{\pd}[2]{\frac{\partial#1}{\partial#2}}
    \newcommand*{\md}[2]{\frac{\mathrm{D}#1}{\mathrm{D}#2}}
    \newcommand*{\at}[1]{\left.#1\right|}
    \newcommand*{\abs}[1]{\left|#1\right|}
    \newcommand*{\ev}[1]{\langle#1\rangle}
    \newcommand*{\bm}[1]{\boldsymbol{\mathbf{#1}}}
    \newcommand*{\p}[1]{\left(#1\right)}
    \newcommand*{\s}[1]{\left[#1\right]}
    \newcommand*{\z}[1]{\left\{#1\right\}}
    \DeclareMathOperator*{\argmin}{argmin}
    \DeclareMathOperator*{\argmax}{argmax}
    \DeclareMathOperator*{\med}{med}
    \DeclareMathOperator*{\sgn}{sgn}

\title[Eccentric Dynamical Tides]{Eccentric Dynamical Tides}
\author[Y. Su, D. Lai.]{
Yubo Su$^1$,
Dong Lai$^1$
\\
$^1$ Cornell Center for Astrophysics and Planetary Science, Department of
Astronomy, Cornell University, Ithaca, NY 14853, USA
}

\date{Accepted XXX\@. Received YYY\@; in original form ZZZ}

\pubyear{2019}

\begin{document}\label{firstpage}
\pagerange{\pageref{firstpage}--\pageref{lastpage}}
\maketitle

\begin{abstract}
    Abstract
\end{abstract}

\begin{keywords}
keywords % chktex 8
\end{keywords}

\section{Introduction}

This problem is important as it will be important for self-driving cars, curing
cancer, and the search for extraterrestrial intelligence.

\section{Theory}

The primary goal of the paper is to evaluate Eq.~\eqref{eq:tau_sum}, the total
torque on a star due to dynamical tides excited by an eccentric companion. We
assume all frequencies excite outgoing waves (no standing modes) to simplify,
and our results are generally most applicable for larger eccentricities $e
\gtrsim 0.5$.

The primary results of the paper are Eq.~\eqref{eq:tau_approx}, shown in
Figs.~\ref{fig:totals_ecc0},~\ref{fig:totals_ecc400},
and~\ref{fig:totals_s} to be reasonably accurate across a range of spins
and eccentricities. The energy dissipation rate is also computed using similar
techniques and show good agreement (see Figs.~\ref{fig:e0}
and~\ref{fig:e400}).

\subsection{Summary of Existing Work}

\subsubsection{Decomposition of Perturbation from an Eccentric Companion}

Consider a star subject to the perturbing potential of a companion star with
mass $M_2$, For a general eccentric orbit, the potential to quadrupolar order
can be decomposed into a sum over circular orbits \citep{sl,vlf}:
\begin{align}
    U &= \sum\limits_m U_{2m} \p{\vec{r}, t},\label{eq:u_ecc}\\
    U_{2m}\p{\vec{r}} &= -\frac{GM_2 W_{2m} r^2}{D(t)^3}
            e^{-imf(t)} Y_{2m}(\theta, \phi),\nonumber\\
        &= -\frac{GM_2W_{2m}r^2}{a^3}Y_{2m}\p{\theta, \phi}
            \sum\limits_{N = -\infty}^\infty F_{Nm}e^{-iN\Omega t},\\
    F_{Nm} &= \frac{1}{\pi}\int\limits_{0}^{\pi}
        \frac{\cos\s{N\p{E - e\sin E} - mf(E)}}
            {\p{1 - e\cos E}^2}\;\mathrm{d}E.
\end{align}
We denote $W_{2 \pm 2} = \sqrt{3\pi/10}$, $W_{2 \pm 1} = 0$, $W_{20} =
-\sqrt{\pi / 5}$, $D(t)$ the instantaneous distance between the star and
companion, $f(t)$ the true anomaly, $Y_{lm}$ the spherical harmonics, and
$\Omega$ the mean motion of the companion. Note that $F_{Nm}$ are the
\emph{Hansen coefficients} for $l = 2$. The total torque on the star, energy
transfer in the inertial frame, and heating in the star's corotating frame are
given respectively
\citep{vlf}:
\begin{align}
    \tau &= \sum\limits_{N = -\infty}^\infty F_{N2}^2
        \hat{\tau}\p{\omega =
        N\Omega - 2\Omega_s},\label{eq:tau_each}
        \\
    \dot{E}_{\rm in} &=
        \frac{1}{2}\sum\limits_{N = -\infty}^\infty\s{
            \p{\frac{W_{20}}{W_{22}}}^2 N\Omega F_{N0}^2
            \hat{\tau}\p{\omega = N\Omega}
            + N\Omega F_{N2}^2 \hat{\tau}\p{\omega =
            N\Omega - 2\Omega_s}},\label{eq:edot_in}\\
    \dot{E}_{\rm rot} &= \dot{E}_{\rm in} - \Omega_s \tau \label{eq:edot_rot},
\end{align}
where $\hat{\tau}(\omega)$ is the torque exerted by a perturber on a circular
trajectory with orbital frequency $\omega$. Compared to \citet{vlf}, we use
$\tau_N(\omega) = T_0 \sgn(\omega) \hat{F}(\abs{\omega})$, for better
integration with the next section.

\subsubsection{Tidal Torque in Massive Stars}

For a circular orbit with orbital frequency $\omega$ and fixed semimajor axis
$a$, the tidal torque exerted on the star by the companion is given by
\citealt{kushnir}:
\begin{align}
    \hat{\tau}(\omega) &= \hat{T}(r_c, \omega) \sgn\p{1 - \frac{2\Omega_s}{\omega}}
        \abs{1 - \frac{2\Omega_s}{\omega}}^{8/3}
            \label{eq:kushnir_torque},\\
    \hat{T}(r_c, \omega) &= \frac{GM_2^2r_c^5}{a^6}
        \p{\frac{\omega}{\sqrt{GM_c/r_c^3}}}^{8/3}
        \s{\frac{r_c}{g_c}\p{\rd{N^2}{\ln r}}_{r = r_c}}^{-1/3}
            \frac{\rho_c}{\bar{\rho}_c} \p{1 - \frac{\rho_c}{\bar{\rho}_c}}^2
            \s{\frac{3}{2}\frac{3^{2/3}\Gamma^2(1/3)}{5 \cdot
                6^{4/3}} \frac{3}{4\pi}\alpha^2},\nonumber\\
        &\equiv \beta_2\frac{GM_2^2r_c^5}{a^6}
            \p{\frac{\omega}{\sqrt{GM_c/r_c^3}}}^{8/3}
            \frac{\rho_c}{\bar{\rho}_c} \p{1 - \frac{\rho_c}{\bar{\rho}_c}}^2.
\end{align}
Here, $1 - \frac{2\Omega_s}{\omega}$ is the dimensionless pattern frequency,
$\alpha$ is defined in Equation A32 of \citealt{kushnir}, $r_c$ is the radius of
the core, $M_c$ the mass of the core, $g_c$ is the gravitational acceleration at
the radiative-convective boundary (RCB), $N^2$ is the Brunt-Vaisala frequency,
$r$ is the radial coordinate within the star, $\rho_c$ is the density at the
RCB, $\bar{\rho}_c$ is the average density of the convective core, and $\beta_2
\approx 1$ is a good approximation for a large range of stellar models
\citep{kushnir}.

\subsection{Tidal Torque}

To obtain a closed form for the tidal torque experienced by a massive star due
to an eccentric companion, as a first estimate we can use
Eq.~\eqref{eq:kushnir_torque} as the torque generated by each mode in
Eq.~\eqref{eq:tau_each}. This gives
\begin{equation}
    \tau = \hat{T}(r_c, \Omega) \sum\limits_{N = -\infty}^\infty
        F_{N2}^2 \sgn\p{N - 2\frac{\Omega_s}{\Omega}}
            \abs{N - 2\frac{\Omega_s}{\Omega}}^{8/3}.\label{eq:tau_sum}
\end{equation}
Note that strictly speaking, this is an upper bound on the tidal torque, as it
assumes all excited IGWs damp effectively, but there is evidence this bound may
be saturated (see Section~\ref{s:disc}).

To understand the scalings of this tidal torque, we aim to express
Eq.~\eqref{eq:tau_sum} in simple closed form. Using the results of
Section~\ref{app:hansens} to sum over the $F_{N2}$, we obtain final form
\begin{equation}
    \frac{\tau}{\hat{T}\p{r_c, \Omega}} \approx \alpha_{2}^{8/3}
        \frac{f_5\p{1 + e}^{4/3}}{\p{1 - e^2}^{17/2}} \times
    \begin{cases}
        \abs{1 - 1.3818\frac{\Omega_s}{N_{\max}\Omega}}^{8/3}
            \frac{\Gamma(23/3)}{4!}\p{\frac{1}{4}}^{8/3},
            & \Omega_s < N_{\max}\Omega / 2,\\[5pt]
        -\abs{1 - 2\frac{\Omega_s}{N_{\max}\Omega}}^{8/3},
            & \Omega_s > N_{\max}\Omega / 2.
    \end{cases}\label{eq:tau_approx}
\end{equation}
Here, we have defined
\begin{align}
    f_5 &= 1 + 3e^2 + 3e^4/8,\label{eq:f5}\\
    \alpha_{2} &\equiv N_{\max} / N_{\rm p} \approx 2(1 + e),\\
    N_{\rm p} &\equiv \left\lfloor \frac{\Omega_{\rm p}}{\Omega} \right\rfloor
        = \left\lfloor \frac{\sqrt{1 + e}}{\p{1 - e^2}^{3/2}} \right\rfloor.
\end{align}
The exact expression for $\alpha_{2}$ is given in Section~\ref{app:hansens},
and $N_{\rm p}$ is the pericenter harmonic. Note that pseudosynchronized spin
occurs for $\frac{\Omega_s}{\Omega} \sim N_{\rm p}$ as is consistent with
literature. We showcase three plots showing the accuracy of this piecewise
description of $\tau\p{e, \Omega_s / \Omega}$:
Figs.~\ref{fig:totals_ecc0},~\ref{fig:totals_ecc400},
and~\ref{fig:totals_s}:
\begin{figure}
    \centering
    \includegraphics[width=0.6\textwidth]{../scripts/eccentric_tides/totals_ecc_0.png}
    \caption{Tidal torque on a slowly spinning star with a companion having
    orbital eccentricity $e$. Blue dots represent explicit summation of
    Eq.~\eqref{eq:tau_sum}, while the blue dashed line is
    Eq.~\eqref{eq:tau_approx}.}\label{fig:totals_ecc0}
\end{figure}
\begin{figure}
    \centering
    \includegraphics[width=0.6\textwidth]{../scripts/eccentric_tides/totals_ecc_400.png}
    \caption{Tidal torque on a rapidly spinning star with a companion having
    orbital eccentricity $e$. Blue dots represent explicit summation of
    Eq.~\eqref{eq:tau_sum}, while the blue dashed line is
    Eq.~\eqref{eq:tau_approx}.}\label{fig:totals_ecc400}
\end{figure}
\begin{figure}
    \centering
    \includegraphics[width=0.6\textwidth]{../scripts/eccentric_tides/totals_s_0_9.png}
    \caption{Tidal torque as a function of spin for a highly eccentric $e = 0.9$
    companion. Blue dots represent explicit summation of Eq.~\eqref{eq:tau_sum},
    while the blue dashed line is the piecewise prediction of
    Eq.~\eqref{eq:tau_approx}. The vertical black line is the analytical $N_{\rm
    p} = 43$. While the pseudo-synchronization frequency differs somewhat
    from $N_{\rm p}$ and the prediction of the piecewise torque, the
    qualitative behavior is very well captured.}\label{fig:totals_s}
\end{figure}

\subsection{Closed Form for Energy Transfer}

We obtain the energy transfer into the star due to this torque (including the $m
= 0$ component) by substituting Eq.~\eqref{eq:kushnir_torque} into
Eq.~\eqref{eq:edot_in}:
\begin{align}
     \dot{E}_{\rm in} = \frac{1}{2}\hat{T}\p{r_c, \Omega}
         \sum\limits_{N = -\infty}^\infty\s{
            N\Omega F_{N2}^2 \sgn \p{\sigma} \abs{\sigma}^{8/3}
            + \p{\frac{W_{20}}{W_{22}}}^2\Omega F_{N0}^2 \abs{N}^{11/3}}.
            \label{eq:e_sum}
\end{align}
Using similar techniques to the previous section for the $m=2$ terms and further
approximations for the $m = 0$ terms given in Section~\ref{app:hansens}, we
obtain expression
\begin{align}
    \frac{\dot{E}_{in}}{\hat{T}\Omega} \approx{}&
            \frac{1}{2^{11/3}}
            \frac{f_5 \alpha_{0}^{11/3}\p{1 + e}^{11/6}}{3\p{1 - e^2}^{10}}
            \Gamma\p{14/3}\nonumber\\
        & + \frac{\alpha_{2}^{11/3}}{2}
                \frac{f_5\p{1 + e}^{11/6}}{\p{1 - e^2}^{10}}
                \times\nonumber\\
        &\begin{cases}
            \abs{1 - 1.1772\frac{\Omega_s}{N_{\max}\Omega}}^{8/3}
                \frac{\Gamma(26/3)}{5!}\frac{1}{4^{8/3}},
                & \Omega_s < N_{\max}\Omega/2,\\[5pt]
            -\abs{1 - 2\frac{\Omega_s}{N_{\max}\Omega}}^{8/3}\frac{5}{4},
                & \Omega_s > N_{\max}\Omega/2.
        \end{cases}\label{eq:total_heating}
\end{align}
Here, $\alpha_0 \approx 2.5e$ is another order-unity constant defined in
Section~\ref{app:hansens} (Eq.~\eqref{eq:a0def})

We make plots in the two $\Omega_s$ regimes as a function of eccentricity in
Figs.~\ref{fig:e0} and~\ref{fig:e400}. Agreement is good again.
\begin{figure}
    \centering
    \includegraphics[width=0.6\textwidth]{../scripts/eccentric_tides/totals_e_0.png}
    \caption{Plot of $\dot{E}_{\rm in}$ for a slowly spinning star. Blue points
    denote explicit summation of Eq.~\eqref{eq:e_sum} and the red dotted line
    represents the closed form Eq.~\eqref{eq:total_heating}.}\label{fig:e0}
\end{figure}
\begin{figure}
    \centering
    \includegraphics[width=0.6\textwidth]{../scripts/eccentric_tides/totals_e_400.png}
    \caption{Same as Fig.~\ref{fig:e0} but for a rapidly spinning
    star.}\label{fig:e400}
\end{figure}

\section{Conclusion and Discussion}\label{s:disc}

\begin{itemize}
    \item Thanks to some references (Barker \& Ogilvie, my work), there seems to
        be some evidence that hydrodynamic wave breaking could cause all IGW to
        break and not reflect, once the pericenter wave reaches nonlinear
        amplitudes.
\end{itemize}

\appendix

\section{Evaluation of Sums over Hansen Coefficients}\label{app:hansens}

We seek closed form approximations to sums of form
\begin{equation}
    S_{mp} = \sum\limits_{N = -\infty}^\infty
        F_{Nm}^2 \sgn\p{N - 2\frac{\Omega_s}{\Omega}}
            \abs{N - 2\frac{\Omega_s}{\Omega}}^p.\label{eq:gen_sum}
\end{equation}
In this paper, we only consider sums where the summand contains $F_{Nm}^2$, but
our approximation holds for arbitrary powers of $F_{Nm}$.

\subsection{$m=2$ Hansen Coefficient Behavior at High Eccentricity}

Recall that the Hansen coefficients are defined as the Fourier series coffecients
of part of the disturbing function
\begin{equation}
    \frac{a^3}{D(t)^3} e^{-imf} = \sum\limits_{N = -\infty}^\infty
        F_{Nm} e^{-iN\Omega t}.\label{eq:hansen_series}
\end{equation}
Observe that $F_{(-N)m} = F_{N(-m)}$. We further observe the following facts about
the Hansen coefficients $F_{N2}$:
\begin{itemize}
    \item For substantial eccentricies, $F_{N2}$ has only one substantial peak.
        The only characteristic scale for $N$ is $N_{\rm p}$ the pericenter
        harmonic, so indeed we find the peak of $F_{N2}$ occurs at $\sim N_{\rm
        p}$ (see Fig.~\ref{fig:hansens}). Furthermore, for $N < 0$, $F_{N2}
        \approx 0$ to good accuracy.

    \item We expect two more general properties about $F_{N2}$: (i) since the
        left hand side of Eq.~\eqref{eq:hansen_series} is smooth in time, the
        Fourier coefficients must have an exponential tail; (ii) since there are
        no characteristic timescales between $\Omega$ and $\Omega_{\rm p}$, we
        anticipate the Hansen coefficients must be scale free between $N = 1$
        and $N_{\rm p}$, i.e.\ a power law. As such, we make ansatz for the
        scalings of the Hansen coefficients for $N \geq 0$:
        \begin{equation}
            F_{N2} \approx C_2 N^{p}e^{-N/\eta_2},\label{eq:hansen_fit}
        \end{equation}
        Eq.~\eqref{eq:hansen_fit} has the advantage that the peak has simple
        form $\argmax\limits_N F_{N2} = p\eta_2$ which we expect $\sim N_{\rm p}$.
        To quantify this, we define
        \begin{equation}
            \alpha_{2} \equiv \frac{p\eta_2}{N_{\rm p}}.\label{eq:alpha2_def}
        \end{equation}
        We expect $\alpha_{{2}}$ must be of order unity.

        Note that at moderate eccentricities $e \lesssim 0.7$, $p$ is very
        poorly constrained, since it is is only constrained by $N \lesssim
        N_{\max}$, a small range for smaller eccentricies. Thus, we fixed $p =
        2$ by fitting $F_{N2}$ for large eccentricities ($e \gtrsim 0.9$) and
        assumed it is universal. We found indeed this proves robust to smaller
        eccentricities.
\end{itemize}
To constrain the remaining two free parameters $\alpha_{2}$ and $C_2$ the
normalization, we use the well known sums of $S_{2p}$ for $p = 0$ and $p = 1$,
which can be explicitly calculated \citep{hut,sl,vlf}:
\begin{align}
    \sum\limits_{N = -\infty}^\infty F_{N2}^2
        &= \frac{1 + 3e^2 + 3e^4/8}{\p{1 - e^2}^{9/2}}
            \equiv \frac{f_5}{\p{1 - e^2}^{9/2}},\\
    \sum\limits_{N = -\infty}^\infty F_{N2}^2N
        &= \frac{2}{\p{1 - e^2}^6}\p{1 + \frac{15e^2}{2}
            + \frac{45 e^4}{8} + \frac{5e^6}{16}}
        \equiv \frac{2f_2}{\p{1 - e^2}^6}.\label{eq:f2}
\end{align}
This fixes
\begin{align}
    \alpha_{2} &= \frac{8f_2}{5f_5\sqrt{1 + e}},\\
    C_2 &= \sqrt{\frac{f_5}{\p{\eta/2}^5 4! \p{1 - e^2}^{9/2}}}.
\end{align}

The agreement of this fit of the Hansen coefficients can be seen in
Fig.~\ref{fig:hansens}.
\begin{figure}
    \centering
    \includegraphics[width=0.6\textwidth]{../scripts/eccentric_tides/hansens.png}
    \caption{Plot of Hansen coefficients $F_{N2}$ for $N > 0$, where red dots
    denote negative values. The red dashed line is the fitted function of form
    Eq.~\eqref{eq:hansen_fit} with the fit values overlaid in red text. Also shown
    in the blue vertical line is the location of $N_{\rm
    p}$.}\label{fig:hansens}
\end{figure}

\subsection{$m = 0$ Hansen Coefficient Behavior at High Eccentricity}

These coefficients have only one characteristic scale in harmonic space, namely
$N_{\rm p}$, but must be symmetric, therefore, the natural ansatz is of form
\begin{equation}
    F_{N0} = C_0 e^{-\abs{N} / \eta_0}.
\end{equation}
Here, again, we expect $\eta_0 \sim N_{\rm p}$, and so we define
\begin{equation}
    \alpha_0 \equiv \eta_0 / N_{\rm p},\label{eq:a0def}
\end{equation}
The two free parameters $C_0$ and $\alpha_0$ are constrained using well known
analytical sums \citep{hut,sl,vlf}:
\begin{align}
    \sum\limits_{N = -\infty}^\infty F_{N0}^2 &= \frac{f_5}{\p{1 - e^2}^{9/2}}
        ,\\
    \sum\limits_{N = -\infty}^\infty F_{N0}^2 N^2
        &= \frac{9e^2}{2\p{1 - e^2}^{15/2}}
            \frac{1}{2}\p{1 + \frac{15}{4}e^2 + \frac{15}{8}e^4
            + \frac{5}{64}e^6}
        = \frac{9e^2}{2\p{1 - e^2}^{15/2}}f_3.
\end{align}
These fix the two free parameters
\begin{align}
    \alpha_0 &= \frac{3e\sqrt{f_3/f_5}}{\sqrt{1 + e}},\\
    C_0 &= \sqrt{\frac{2f_5}{\alpha_0\sqrt{1 + e}\p{1 - e^2}^3}}.
\end{align}

\subsection{Evaluating Torque}

This section and the next are copy pasted from my notes for the time being, need
to revise:

To evaluate the torque, we need to compute $S_{2(8/3)}$, which has form
\begin{equation}
    \frac{\tau}{\hat{T}} = \sum\limits_{N = -\infty}^\infty
        F_{N2}^2 \sgn\p{N - 2\frac{\Omega_s}{\Omega}}
            \abs{N - 2\frac{\Omega_s}{\Omega}}^{8/3}.
\end{equation}
Recall that $F_{N2}$ falls off for either $N \ll N_{\max}, N \gg N_{\max}$, so
we have two regimes for the sum:
\begin{itemize}
    \item Consider $2\Omega_s \gg N_{\max}\Omega$, then the sign is always
        negative and $\abs{N - 2\Omega_s/\Omega} \approx
        \abs{\frac{2\Omega_s}{\Omega} - N_{\max}}$. We can then use the results
        of the preceeding sections and obtain
        \begin{align}
            \hat{\tau}
                &\approx -\abs{\frac{2\Omega_s}{\Omega} - N_{\max}}^{8/3}
                    \sum\limits_{N = -\infty}^\infty F_{N2}^2,\\
                &\approx -\abs{1 - \frac{2\Omega_s}{N_{\max}\Omega}}^{8/3}
                    \frac{f_5}{\p{1 - e^2}^{9/2}}\p{\alpha\p{\frac{1 + e}{\p{1 -
                    e^2}^3}}^{1/2}}^{8/3},\\
                &\approx -\abs{1 - \frac{2\Omega_s}{N_{\max}\Omega}}^{8/3}
                    \frac{f_5\alpha^{8/3}\p{1 + e}^{4/3}}{\p{1 - e^2}^{17/2}}.
        \end{align}

    \item Consider $2\Omega_s \ll N_{\max} \Omega$, then the sign is the
        sign of $N_{\max}$ ($>0$). To apply the results of the previous
        sections, we seek to factorize the $\abs{N - 2\Omega_s/\Omega}^{8/3}$
        term to pull out the factor of $N^{8/3}$, of form
        \begin{equation}
            \abs{N - \frac{2\Omega_s}{\Omega}} \approx \abs{N}\abs{1 -
                \frac{2\Omega_s}{\beta_T N_{\max}\Omega}}.
        \end{equation}
        Such a factorization must have the correct asymptotics as $\Omega_s \to
        -\infty$, that is:
        \begin{equation}
            \sum\limits_{N} F_{N2}^2 \abs{\frac{-2\Omega_s}{\Omega}}^{8/3}
                = \abs{\frac{2\Omega_s}{\beta_T N_{\max}\Omega}}^{8/3}
                    \sum\limits_{N}F_{N2}^2 N^{8/3}.
        \end{equation}

        This equates to requiring that the two regimes $\Omega_s \gg
        N_{\max}\Omega$ and $-\Omega_s \ll -N_{\max}\Omega$ be symmetric. We
        find $\beta_T = \p{\frac{\Gamma\p{23/3}}{4!}}^{3/8}/4 \approx 1.447$,
        and so $2/\beta_T \approx 1.3818$. Thus,
        \begin{align}
            \hat{\tau}
                &\approx \abs{1 - 1.3818\frac{\Omega_s}{N_{\max}\Omega}}^{8/3}
                    \sum\limits_{N = -\infty}^\infty F_{N2}^2 \abs{N}^{8/3},\\
                &\approx \abs{1 - 1.3818\frac{\Omega_s}{N_{\max}\Omega}}^{8/3}
                    \frac{f_5}{\p{1 - e^2}^{17/2}}
                    \frac{\Gamma(23/3)}{4!}\p{\frac{\alpha}{4}}^{8/3}
                        \p{1 + e}^{4/3}.
        \end{align}
\end{itemize}
Thus, we arrive at final answer
\begin{equation}
    \hat{\tau} \approx \alpha^{8/3}
        \frac{f_5\p{1 + e}^{4/3}}{\p{1 - e^2}^{17/2}} \times
    \begin{cases}
        \abs{1 - 1.3818\frac{\Omega_s}{N_{\max}\Omega}}^{8/3}
            \frac{\Gamma(23/3)}{4!}\p{\frac{1}{4}}^{8/3},
            & \Omega_s < N_{\max}\Omega / 2,\\[5pt]
        -\abs{1 - 2\frac{\Omega_s}{N_{\max}\Omega}}^{8/3},
            & \Omega_s > N_{\max}\Omega / 2.
    \end{cases}\label{eq:traveling_torque}
\end{equation}

\subsection{Evaluating Energy Transfer}

In the case of the energy transfer in the inertial frame, we recall
\begin{align}
     \dot{E}_{in} = \frac{1}{2}\hat{T}\p{r_c, \Omega}\s{
         \sum\limits_{N = -\infty}^\infty
            N\Omega F_{N2}^2 \sgn \p{N - \frac{2\Omega_s}{\Omega}} \abs{N -
            \frac{2\Omega_s}{\Omega}}^{8/3}
            + \p{\frac{W_{20}}{W_{22}}}^2\Omega F_{N0}^2 \abs{N}^{11/3}},
\end{align}
Recall $\p{W_{20} / W_{22}}^2 = 2/3$.

The second term has no casework and is very clean:
\begin{align}
    \frac{\dot{E}_{m=0}}{\hat{T}\Omega}
        &= \frac{1}{3} \sum\limits_{N = -\infty}^\infty
            F_{N0}^2\abs{N}^{11/3},\\
        &\approx \frac{1}{2^{11/3}}
            \frac{f_5 \alpha_0^{11/3}\p{1 + e}^{11/6}}{3\p{1 - e^2}^{10}}
            \Gamma\p{14/3}.
\end{align}

The first term is the same procedure as for the torque, where we must expand and
match consistency. The basic form is
\begin{equation}
    \frac{\dot{E}_{m=2}}{\hat{T}\Omega} = \frac{1}{2}
        \sum\limits_{N = -\infty}^\infty NF_{N2}^2 \mathrm{sgn}\p{N -
            \frac{2\Omega_s}{\Omega}} \abs{N - \frac{2\Omega_s}{\Omega}}^{8/3}.
\end{equation}
The two regimes again are:
\begin{itemize}
    \item In the limit where $\Omega_s \gg N_{\max}\Omega$, then the sign is
        always negative and $\abs{N - 2\Omega_s / \Omega} \approx
        \abs{2\Omega_s / \Omega - N_{\max}}$ and we have straightforwardly
        \begin{align}
            \frac{\dot{E}_{m=2}}{\hat{T}\Omega}
                &= -\frac{1}{2}
                    \abs{\frac{2\Omega_s}{\Omega} - N_{\max}}^{8/3}
                    \sum\limits_{N = -\infty}^\infty NF_{N2}^2,\\
                &\approx -\frac{1}{2}
                    \abs{1 - \frac{2\Omega_s}{N_{\max}\Omega}}^{8/3}
                    \alpha_2^{11/3}
                    \frac{f_5}{\p{1 - e^2}^{10}}\frac{5\p{1 + e}^{11/6}}{4}.
        \end{align}
    \item Again, in the other limit where $2\Omega_s \ll N_{\max}\Omega$, we
        seek some factorization + summation $\abs{N - \frac{2\Omega_s}{\Omega}}
        \approx \abs{N}\abs{1 - \frac{2\Omega_s}{\beta_E N_{\max}\Omega}}$ such
        that the asymptotics are correct:
        \begin{align}
            \abs{\frac{2\Omega_s}{\Omega}}^{8/3} \sum\limits_N F_{N2}^2 N
                &= \abs{\frac{2\Omega_s}{\beta_E N_{\max}\Omega}}^{8/3}
                    \sum\limits_N F_{N2}N^{11/3},\\
            \beta_E
                &= \p{\frac{\Gamma(26/3)}{5!}}^{3/8} \p{\frac{1}{4}}
                    \approx 1.699.
        \end{align}

        Thus, we can now approximate the sum
        \begin{align}
            \frac{\dot{E}_{m=2}}{\hat{T}\Omega}
                &= \frac{1}{2}
                    \abs{1 - 1.1772\frac{\Omega_s}{\Omega N_{\max}}}^{8/3}
                    \sum\limits_{N = -\infty}^\infty \abs{N}^{11/3}F_{N2}^2,\\
                &= \frac{1}{2}
                    \abs{1 - 1.1772\frac{\Omega_s}{N_{\max}\Omega}}^{8/3}
                    \frac{f_5}{\p{1 - e^2}^{10}}\frac{5\p{1 + e}^{11/6}}{4}
                    \p{\frac{\alpha_2}{4}}^{11/3}
                    \frac{\Gamma(26/3)}{5!}.
        \end{align}
\end{itemize}
Thus, we have total energy transfer rate
\begin{align}
    \frac{\dot{E}_{in}}{\hat{T}\Omega} ={}&
            \frac{1}{2^{11/3}}
            \frac{f_5 \alpha_0^{11/3}\p{1 + e}^{11/6}}{3\p{1 - e^2}^{10}}
            \Gamma\p{14/3}\nonumber\\
        & + \frac{\alpha^{11/3}}{2}
                \frac{f_5\p{1 + e}^{11/6}}{\p{1 - e^2}^{10}}
                \times\nonumber\\
        &\begin{cases}
            \abs{1 - 1.1772\frac{\Omega_s}{N_{\max}\Omega}}^{8/3}
                \frac{\Gamma(26/3)}{5!}\frac{1}{4^{8/3}},
                & \Omega_s < N_{\max}\Omega/2,\\[5pt]
            -\abs{1 - 2\frac{\Omega_s}{N_{\max}\Omega}}^{8/3}\frac{5}{4},
                & \Omega_s > N_{\max}\Omega/2.
        \end{cases}\label{eq:edot_in}
\end{align}


\bibliographystyle{mnras}
\bibliography{Su_eccentric_tides}

% \clearpage
% \onecolumn

\bsp
\label{lastpage} % chktex 24
\end{document}
