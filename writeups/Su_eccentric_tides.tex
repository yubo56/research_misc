    \documentclass[
        fleqn,
        usenatbib,
        referee,
    ]{mnras}
    \usepackage{
        amsmath,
        amssymb,
        newtxtext,
        newtxmath,
        graphicx,
        ae, aecompl,
        booktabs,
        caption,
        subcaption,
    }
    \usepackage[T1]{fontenc}
    \captionsetup{compatibility=false}

    \newcommand*{\rd}[2]{\frac{\mathrm{d}#1}{\mathrm{d}#2}}
    \newcommand*{\rtd}[2]{\frac{\mathrm{d}^2#1}{\mathrm{d}#2^2}}
    \newcommand*{\pd}[2]{\frac{\partial#1}{\partial#2}}
    \newcommand*{\md}[2]{\frac{\mathrm{D}#1}{\mathrm{D}#2}}
    \newcommand*{\at}[1]{\left.#1\right|}
    \newcommand*{\abs}[1]{\left|#1\right|}
    \newcommand*{\ev}[1]{\langle#1\rangle}
    \newcommand*{\bm}[1]{\boldsymbol{\mathbf{#1}}}
    \newcommand*{\p}[1]{\left(#1\right)}
    \newcommand*{\s}[1]{\left[#1\right]}
    \newcommand*{\z}[1]{\left\{#1\right\}}
    \DeclareMathOperator*{\argmin}{argmin}
    \DeclareMathOperator*{\argmax}{argmax}
    \DeclareMathOperator*{\med}{med}
    \DeclareMathOperator*{\sgn}{sgn}

\title[Eccentric Dynamical Tides]{Eccentric Dynamical Tides}
\author[Y. Su, D. Lai.]{
Yubo Su$^1$,
Dong Lai$^1$
\\
$^1$ Cornell Center for Astrophysics and Planetary Science, Department of
Astronomy, Cornell University, Ithaca, NY 14853, USA
}

\date{Accepted XXX\@. Received YYY\@; in original form ZZZ}

\pubyear{2019}

\begin{document}\label{firstpage}
\pagerange{\pageref{firstpage}--\pageref{lastpage}}
\renewcommand*{\sectionautorefname}{Section}
% \renewcommand*{\subsectionautorefname}{Subsection}
\maketitle

\begin{abstract}
    Abstract
\end{abstract}

\begin{keywords}
keywords % chktex 8
\end{keywords}

\section{Introduction}

This problem is important.

\section{Theory}

The primary goal of the paper is to evaluate \autoref{eq:tau_sum}, the total
torque on a star due to dynamical tides excited by an eccentric perturber. We
assume all frequencies excite outgoing waves (no standing modes) to simplify,
and our results are generally most applicable for larger eccentricities $e
\gtrsim 0.5$.

The primary results of the paper are \autoref{eq:tau_approx_1} and
\autoref{eq:tau_approx_2}, which are shown in \autoref{fig:totals_ecc0},
\autoref{fig:totals_ecc400}, and \autoref{fig:totals_s} to be reasonably
accurate across a range of spins and eccentricities. The energy dissipation rate
is also computed using similar techniques and show good agreement (see
\autoref{fig:e0} and \autoref{fig:e400}).

\subsection{Summary of Existing Work}

\subsubsection{Decomposition of Perturbation from an Eccentric Companion}

Consider a star subject to the perturbing potential of a perturbing star with
mass $M_2$, For a general eccentric orbit, the potential to quadrupolar order
can be decomposed into a sum over circular orbits \citep{sl,vlf}:
\begin{align}
    U &= \sum\limits_m U_{2m} \p{\vec{r}, t},\label{eq:u_ecc}\\
    U_{2m}\p{\vec{r}} &= -\frac{GM_2 W_{2m} r^2}{D(t)^3}
            e^{-imf(t)} Y_{2m}(\theta, \phi),\nonumber\\
        &= -\frac{GM_2W_{2m}r^2}{a^3}Y_{2m}\p{\theta, \phi}
            \sum\limits_{N = -\infty}^\infty F_{Nm}e^{-iN\Omega t},\\
    F_{Nm} &= \frac{1}{\pi}\int\limits_{0}^{\pi}
        \frac{\cos\s{N\p{E - e\sin E} - mf(E)}}
            {\p{1 - e\cos E}^2}\;\mathrm{d}E.
\end{align}
We denote $W_{2 \pm 2} = \sqrt{3\pi/10}$, $W_{2 \pm 1} = 0$, $W_{20} =
-\sqrt{\pi / 5}$, $D(t)$ the instantaneous distance between the star and
perturber, $f(t)$ the true anomaly, $Y_{lm}$ the spherical harmonics, and
$\Omega$ the mean motion of the companion. Note that $F_{Nm}$ are the
\emph{Hansen coefficients} for $l = 2$. The total torque on the star is known
\citep{vlf}:
\begin{equation}
    \tau = \sum\limits_{N = -\infty}^\infty F_{N2}^2 \tau_N,\label{eq:tau_each}
\end{equation}
where $\tau_N$ is the torque exerted by any particular $N$ mode on the star.

\subsubsection{Tidal Torque in Massive Stars}

For a circular orbit at fixed semimajor axis $a$, the tidal torque exerted on
the star by the companion is given by \citealt{kushnir}:
\begin{align}
    \tau &= \hat{T}(r_c, \omega) \sgn\p{1 - \frac{2\Omega_s}{\omega}}
        \abs{1 - \frac{2\Omega_s}{\omega}}^{8/3}
            \label{eq:kushnir_torque},\\
    \hat{T}(r_c, \omega) &= \frac{GM_2^2r_c^5}{a^6}
        \p{\frac{\omega}{\sqrt{GM_c/r_c^3}}}^{8/3}
        \s{\frac{r_c}{g_c}\p{\rd{N^2}{\ln r}}_{r = r_c}}^{-1/3}
            \frac{\rho_c}{\bar{\rho}_c} \p{1 - \frac{\rho_c}{\bar{\rho}_c}}^2
            \s{\frac{3}{2}\frac{3^{2/3}\Gamma^2(1/3)}{5 \cdot
                6^{4/3}} \frac{3}{4\pi}\alpha^2},\nonumber\\
        &\equiv \beta_2\frac{GM_2^2r_c^5}{a^6}
            \p{\frac{\omega}{\sqrt{GM_c/r_c^3}}}^{8/3}
            \frac{\rho_c}{\bar{\rho}_c} \p{1 - \frac{\rho_c}{\bar{\rho}_c}}^2,
\end{align}
where $1 - \frac{2\Omega_s}{\omega}$ is the dimensionless pattern frequency of a
companion with orbital frequency $\omega$, $\alpha \simeq 1$ is defined in
Equation A32 of \citealt{kushnir}, $r_c$ is the radius
of the core, $M_c$ the mass of the core, $g_c$ is the gravitational acceleration at
the radiative-convective boundary (RCB), $N^2$ is the Brunt-Vaisala frequency, $r$
is the radial coordinate within the star, $\rho_c$ is the density at the RCB,
$\bar{\rho}_c$ is the average density of the convective core, and $\beta_2
\approx 1$ as implicitly defined above.

\subsection{Tidal Torque}

To obtain a closed form for the tidal torque experienced by a massive star due
to an eccentric companion, as a first estimate we can use
Eq.~\eqref{eq:kushnir_torque} as the torque generated by each mode in
Eq.~\eqref{eq:tau_each}. This gives
\begin{equation}
    \tau = \hat{T}(r_c, \Omega) \sum\limits_{N = -\infty}^\infty
        F_{N2}^2 \sgn\p{N - 2\frac{\Omega_s}{\Omega}}
            \abs{N - 2\frac{\Omega_s}{\Omega}}^{8/3}.\label{eq:tau_sum}
\end{equation}
Note that strictly speaking, this is an upper bound on the tidal torque, as it
assumes all excited IGWs damp effectively, but there is evidence this bound may
be saturated (see Section~\ref{s:disp}).

To understand the scalings of this tidal torque, we aim to express
\autoref{eq:tau_sum} in simple closed form. Using the results of
Section~\ref{app:hansens} to sum over the $F_{N2}$, we obtain final form
\begin{equation}
    \frac{\tau}{\hat{T}\p{r_c, \Omega}} \approx \alpha^{8/3}
        \frac{f_5\p*{1 + e}^{4/3}}{\p*{1 - e^2}^{17/2}} \times
    \begin{cases}
        \abs*{1 - 1.38\frac{\Omega_s}{N_{\max}\Omega}}^{8/3}
            \frac{\Gamma(23/3)}{4!}\p*{\frac{1}{4}}^{8/3},
            & \Omega_s < N_{\max}\Omega / 2,\\[5pt]
        -\abs*{1 - 2\frac{\Omega_s}{N_{\max}\Omega}}^{8/3},
            & \Omega_s > N_{\max}\Omega / 2.
    \end{cases}\label{eq:tau_approx}
\end{equation}

The existence of a pseudo-synchronized state at $\Omega_s \sim N_{\rm peri}
\Omega$ is already evident, as the sign of the torque changes in the two
regimes. We showcase three plots showing the accuracy of this piecewise
description of $\tau\p{e, \Omega_s / \Omega}$, \autoref{fig:totals_ecc0},
\autoref{fig:totals_ecc400}, \autoref{fig:totals_s}.:
\begin{figure}
    \centering
    \includegraphics[width=0.6\textwidth]{../scripts/eccentric_tides/totals_ecc_0.png}
    \caption{Tidal torque on a slowly spinning star with a companion having
    orbital eccentricity $e$. Blue dots represent explicit summation of
    \autoref{eq:tau_sum}, while the blue dashed line is
    \autoref{eq:tau_approx_1}.}\label{fig:totals_ecc0}
\end{figure}
\begin{figure}
    \centering
    \includegraphics[width=0.6\textwidth]{../scripts/eccentric_tides/totals_ecc_400.png}
    \caption{Tidal torque on a rapidly spinning star with a companion having
    orbital eccentricity $e$. Blue dots represent explicit summation of
    \autoref{eq:tau_sum}, while the blue dashed line is
    \autoref{eq:tau_approx_2}.}\label{fig:totals_ecc400}
\end{figure}
\begin{figure}
    \centering
    \includegraphics[width=0.6\textwidth]{../scripts/eccentric_tides/totals_s_0_9.png}
    \caption{Tidal torque as a function of spin for a highly eccentric $e = 0.9$
    companion. Blue dots represent explicit summation of \autoref{eq:tau_sum},
    while the blue dashed line is the piecewise prediction of
    \autoref{eq:tau_approx_1} and \autoref{eq:tau_approx_2} together. The
    vertical black line is the analytical $N_{\rm peri} = 43$. While the
    pseudo-synchronization frequency differs somewhat from $N_{\rm peri}$ and
    the prediction of the piecewise torque, the qualitative behavior is very
    well captured.}\label{fig:totals_s}
\end{figure}

\subsection{Closed Form for Heating}

Energy transfer into the star due to this torque (including the $m = 0$
component) in the inertial frame is given by \citealt{vlf}:
\begin{align}
     \dot{E}_{\rm in} = \frac{1}{2}\hat{T}\p{r_c, \Omega}
         \sum\limits_{N = -\infty}^\infty\s{
            N\Omega F_{N2}^2 \sgn \p{\sigma} \abs{\sigma}^{8/3}
            + \p{\frac{W_{20}}{W_{22}}}^2\Omega F_{N0}^2 \abs{N}^{11/3}}.
            \label{eq:e_sum}
\end{align}
Using similar techniques to the previous section for the $m=2$ terms and further
approximations given in Section~\ref{app:hansens}, we obtain approximation
\begin{align}
    \frac{\dot{E}_{in}}{\hat{T}\Omega} ={}&
            \frac{1}{2^{11/3}}
            \frac{f_5 \beta^{11/3}\p*{1 + e}^{11/6}}{3\p*{1 - e^2}^{10}}
            \Gamma\p*{14/3}\nonumber\\
        & + \frac{\alpha^{11/3}}{2}
                \frac{f_5\p*{1 + e}^{11/6}}{\p*{1 - e^2}^{10}}
                \times\nonumber\\
        &\begin{cases}
            \abs*{1 - 1.1772\frac{\Omega_s}{N_{\max}\Omega}}^{8/3}
                \frac{\Gamma(26/3)}{5!}\frac{1}{4^{8/3}},
                & \Omega_s < N_{\max}\Omega/2,\\[5pt]
            -\abs*{1 - 2\frac{\Omega_s}{N_{\max}\Omega}}^{8/3}\frac{5}{4},
                & \Omega_s > N_{\max}\Omega/2.
        \end{cases}\label{eq:total_heating}
\end{align}

We make plots in the two $\Omega_s$ regimes as a function of eccentricity in
\autoref{fig:e0} and \autoref{fig:e400}. Agreement is good again.
\begin{figure}
    \centering
    \includegraphics[width=0.6\textwidth]{../scripts/eccentric_tides/totals_e_0.png}
    \caption{Plot of $\dot{E}_{\rm in}$ for a slowly spinning star. Blue points
    denote explicit summation of \autoref{eq:e_sum} and the red dotted line
    represents the closed form \autoref{eq:e_dot_tot}.}\label{fig:e0}
\end{figure}
\begin{figure}
    \centering
    \includegraphics[width=0.6\textwidth]{../scripts/eccentric_tides/totals_e_400.png}
    \caption{Same as \autoref{fig:e0} but for a rapidly spinning
    star.}\label{fig:e400}
\end{figure}

\section{Conclusion and Discussion}\label{s:disc}

\appendix

\section{Evaluation of Sums over Hansen Coefficients}\label{app:hansens}

We seek closed form approximations to sums of form
\begin{equation}
    S_{mp} = \sum\limits_{N = -\infty}^\infty
        F_{Nm}^2 \sgn\p*{N - 2\frac{\Omega_s}{\Omega}}
            \abs*{N - 2\frac{\Omega_s}{\Omega}}^p.\label{eq:gen_sum}
\end{equation}
In this paper, we only consider sums where the summand contains $F_{Nm}^2$, but
our approximation holds for arbitrary powers of $F_{Nm}$.

\subsection{$m=2$ Hansen Coefficient Behavior at High Eccentricity}

Recall that the Hansen coefficients are defined as the Fourier series coffecients
of part of the disturbing function
\begin{equation}
    \frac{a^3}{D(t)^3} e^{-imf} = \sum\limits_{N = -\infty}^\infty
        F_{Nm} e^{-iN\Omega t}.\label{eq:hansen_series}
\end{equation}
Observe that $F_{(-N)m} = F_{N(-m)}$. We further observe the following facts about
the Hansen coefficients $F_{N2}$:
\begin{itemize}
    \item For substantial eccentricies, $F_{N2}$ has only one substantial peak.
        Defining
        \begin{equation}
            N_{\rm peri} \equiv \frac{\Omega_p}{\Omega}.
        \end{equation}
        where $\frac{\Omega_p}{\Omega} \equiv \frac{\sqrt{1 + e}}{\p{1 -
        e}^{3/2}}$, is the orbital frequency at pericenter, we find the peak of
        $F_{N2}$ occurs at $\simeq \sqrt{2}N_{\rm peri}$ (see
        \autoref{fig:hansens}). In fact, for $N < 0$, $F_{N2} \approx 0$ to good
        accuracy.

    \item We seek an analytically tractable formula for the Hansen coefficients.
        We guess form
        \begin{equation}
            F_{N2} \approx cN^{p}e^{-N/\eta}.\label{eq:hansen_fit}
        \end{equation}
        Two primary considerations enter into this fitting law: (i) since the
        left hand side of \autoref{eq:hansen_series} is smooth in time, the
        Fourier coefficients must have an exponential tail; (ii) since there are
        no characteristic timescales between $\Omega$ and $\Omega_p$, we
        anticipate the Hansen coefficients must be scale free between $N = 1$
        and $N \sim \Omega_p / \Omega$, i.e.\ a power law. Empirically, we find
        that $p = 2$ to good approximation.

        Eq.~\eqref{eq:hansen_fit} has the advantage that the peak has simple
        form $\argmax\limits_N F_{N2} = p\eta \sim N_{\rm peri}$. We define
        \begin{equation}
            \alpha^{(2)} \equiv \frac{p\eta}{N_{\rm peri}}.\label{eq:alpha2_def}
        \end{equation}

    \item Finally, we can lastly constrain $c$ as well: the sum in
        Eq.~\eqref{eq:gen_sum} has been evaluated for $m = 0$ and $m = 2$ for
        $p = 0, 1, 2$ \citep{hut,sl,vlf} by an application of Parseval's
        theorem. The relevant sums are
        \begin{align}
            \sum\limits_N F_{N2}^2N
                &= \frac{2}{\p*{1 - e^2}^6}\p*{1 + \frac{15e^2}{2}
                    + \frac{45 e^4}{8} + \frac{5e^6}{16}}
                \equiv \frac{}{}
        \end{align}
\end{itemize}
The agreement of this fit of the Hansen coefficients as well as the accuracy
$\p{C, p, \eta}$ parameterization as a function of $e$ can be seen in
\autoref{fig:hansens} and \autoref{fig:hansen_params} respectively. Note that at
moderate eccentricities $e \lesssim 0.7$, $p$ is very poorly constrained, since
it is dominant only in the regime $N \lesssim N_{\max}$, which has fewer values
for smaller eccentricies. Thus, we fixed $p = 2$ from our high-eccentricity fits
($e \gtrsim 0.9$) and assumed it is universal. We found indeed this does not
significantly worsen fits at moderate eccentricities.
\begin{figure}
    \centering
    \includegraphics[width=0.6\textwidth]{../scripts/eccentric_tides/hansens.png}
    \caption{Plot of Hansen coefficients $F_{N2}$ for $N > 0$, where red dots
    denote negative values. The red dashed line is the fitted function of form
    \autoref{eq:hansen_fit} with the fit values overlaid in red text. Also shown
    in the blue vertical line is the location of $N_{\rm
    peri}$.}\label{fig:hansens}
\end{figure}
\begin{figure}
    \centering
    \includegraphics[width=0.6\textwidth]{../scripts/eccentric_tides/hansen_params.png}
    \caption{Plot of $C$, $\eta$ as a function of $e$. Blue dots indicate the
    numerical fit to $F_{N2}$ for $N > 0$ while red dashed lines reflect the
    predictions of \autoref{eq:a_approx} and \autoref{eq:c_approx} respectively.
    Note that $p = 2$ is taken to be fixed in the numerical
    fit.}\label{fig:hansen_params}
\end{figure}

\bibliographystyle{mnras}
\bibliography{Su_eccentric_tides}

% \clearpage
% \onecolumn

\bsp
\label{lastpage} % chktex 24
\end{document}
