% rm -f LK90_plots/*.png; for i in M1_M2_thetasl_thetae_70 5_dWs_inner 5_dWs 6bin_comp 6bin_comp_zoom 4qslscan/qslscan 4sims_scan/deviations_one 4sims/4sim_90_200_zoom 4sims/4sim_90_350_supp 4sims/4sim_90_350_zoom 4sims/4sim_90_350 7_3vec_cropped 4sims/good_quants; do cp ../scripts/lk90/$i.png LK90_plots; done
% rename "Harmonic" and "nonad" to equation numbers in plots
    \documentclass[
        twocolumn,
        twocolappendix
    ]{aastex63}
    \usepackage{
        amsmath,
        amssymb,
        newtxtext,
        newtxmath,
        graphicx,
        ae,
        aecompl,
        booktabs,
        wasysym
    }
    \usepackage[T1]{fontenc}

    \newcommand*{\rd}[2]{\frac{\mathrm{d}#1}{\mathrm{d}#2}}
    \newcommand*{\rtd}[2]{\frac{\mathrm{d}^2#1}{\mathrm{d}#2^2}}
    \newcommand*{\pd}[2]{\frac{\partial#1}{\partial#2}}
    \newcommand*{\md}[2]{\frac{\mathrm{D}#1}{\mathrm{D}#2}}
    \newcommand*{\rdil}[2]{\mathrm{d}#1 / \mathrm{d}#2}
    \newcommand*{\pdil}[2]{\partial#1 / \partial#2}
    \newcommand*{\at}[1]{\left.#1\right|}
    \newcommand*{\abs}[1]{\left|#1\right|}
    \newcommand*{\ev}[1]{\langle#1\rangle}
    \renewcommand*{\bm}[1]{\boldsymbol{\mathbf{#1}}}
    \newcommand*{\uv}[1]{\hat{\bm{#1}}}
    \newcommand*{\p}[1]{\left(#1\right)}
    \newcommand*{\s}[1]{\left[#1\right]}
    \newcommand*{\z}[1]{\left\{#1\right\}}
    \DeclareMathOperator*{\argmin}{argmin}
    \DeclareMathOperator*{\argmax}{argmax}
    \DeclareMathOperator*{\med}{med}
    \DeclareMathOperator*{\sgn}{sgn}
    \let\Re\undefined
    \let\Im\undefined
    \DeclareMathOperator{\Re}{Re}
    \DeclareMathOperator{\Im}{Im}

    \newcommand\aastex{AAS\TeX}
    \newcommand\latex{La\TeX}

    \received{XXXX}
    \revised{XXXX}
    \accepted{XXXX}
    \submitjournal{ApJ}

\shorttitle{BH Triple Spin-Orbit Dynamics}
\shortauthors{Y.\ Su et.\ al.}

\begin{document}

\title{Spin Dynamics in Hierarchical Black Hole Triples: Predicting Final
Spin-Orbit Misalignment Angle From Initial Conditions}

\correspondingauthor{Yubo Su}
\email{yubosu@astro.cornell.edu}

\author[0000-0001-8283-3425]{Yubo Su}% chktex 8
\affiliation{Cornell Center for Astrophysics and Planetary Science, Department
of Astronomy, Cornell University, Ithaca, NY 14853, USA}

\author[0000-0002-1934-6250]{Dong Lai}% chktex 8
\affiliation{Cornell Center for Astrophysics and Planetary Science, Department
of Astronomy, Cornell University, Ithaca, NY 14853, USA}

\author[0000-0002-0643-8295]{Bin Liu}% chktex 8
\affiliation{Cornell Center for Astrophysics and Planetary Science, Department
of Astronomy, Cornell University, Ithaca, NY 14853, USA}

\begin{abstract}
    Abstract
\end{abstract}

\keywords{keywords}

\section{Introduction}\label{s:intro}

As the Laser Interferometer Gravitational-wave Observatory (LIGO) continues to
detect mergers of black hole (BH) binaries
\citep[e.g.][]{LIGOScientific:2020stg, Abbott:2017gyy, Abbott:2017oio,
Abbott:2017vtc, Abbott:2016nmj, TheLIGOScientific:2016pea, Abbott:2016blz}, it
is increasingly important to systematically study various formation channels of
BH binaries and their observable signatures. The canonical channel consists of
isolated binary evolution, in which mass transfer and friction in the common
envelope phase cause the binary orbit to shrink sufficiently that it
subsequently merges via emission of gravitational waves (GW) within a Hubble
time \citep[e.g.][]{dominik2012double}. BH binaries formed via isolated binary
evolution are expected to be slowly rotating \citep{fuller2019most}.

However, \citep{S190521g} reports that the GW event candidate S190521g has at
least one significantly spinning component. This is consistent with the
tertiary-induced dynamical formation channel, in which a BH binary that is too
far separated to merge in isolation is induced to merge by a tertiary companion,
often a supermassive black hole (SMBH). The SMBH induces Lidov-Kozai (LK)
oscillations in the BH binary that excite the eccentricity to very large values,
which causes significant GW emission at pericenter that allows the BH binary to
coalesce and merge within a Hubble time.

Because eccentricity is efficiently damped by gravitational wave (GW) radiation,
the vast majority of BH binaries are expected to be circular when entering the
LIGO sensitivity band regardless of their formation channel. It has been
suggested that the BH spin and the spin-orbit misalignment angle may carry
information on the binary formation history. In particular, through the phase
shift in the binary inspiral waveform, one can directly measure the
mass-weighted average of the dimensionless spin parameter
\begin{equation}
    \chi_{\rm eff} \equiv \frac{m_1\bm{\chi}_1 + m_2\bm{\chi}_2}{m_1 + m_2}
        \cdot \uv{L},
\end{equation}
where $m_{1,2}$ are the masses of the BHs, $\bm{\chi}_{1,2} = c\bm{S}_{1,2} /
(Gm_{1,2}^2)^2$ are the dimensionless BH spins, and $\uv{L}$ is the unit orbital
angular momentum vector.

In a recent paper, \citet{bin2} found that LK-induced mergers can often yield a
$90^\circ$ spin attractor: when the inner binary starts with $\theta_{\rm
sl}^{\rm i} = 0$, the individual BH spins often evolve towards spin-orbit
misalignment angle $\theta_{\rm sl}^{\rm f} = 90^\circ$, where $\theta_{\rm sl}$
is the angle between the BH spin and $\uv{L}$. They found that the attractor
exists when (i) the LK-induced orbital decay is slow, and (ii) the octupole
effect is unimportant. Fig.~\ref{fig:4sim_90_350} gives an example of this
attractor, where $\theta_{\rm sl}$ rapidly converges to $\approx 90^\circ$ at
late times in the bottom right panel. Fig.~\ref{fig:qslscan} shows how
$\theta_{\rm sl}^{\rm f}$ varies when the initial inclination of the inner orbit
$I_0$ (relative to the tertiary orbit) is varied. Note that for longer merger
times, corresponding to $I_0$ farther from $90^\circ$, the final $\theta_{\rm
sl}^{\rm f} \approx 90^\circ$. In both of these plots, the tertiary's
eccentricity is taken to be zero, for which octupole effects are negligible.

The physical origin of this $90^\circ$ attractor is not well understood.
\citet{bin2} proposed an explanation based on analogy with an adiabatic
invariant in systems where the inner binary remains circular through the
inspiral \citep{bin1}. However, this analogy is not justified, as significant
eccentricity excitation is a necessary ingredient in LK-induced mergers. In
addition, the results in \citet{bin1} show no $90^\circ$ attractor even though
the orbital evolution is slow and regular.

In this paper, we study an analytic theory that reproduces the $90^\circ$
attractor as well as its regime of validity. In Sections~\ref{s:setup_orbital}
and~\ref{s:setup_spin}, we set up the relevant equations of motion for the
orbital and spin evolution of the system. In Sections~\ref{s:fast_merger}
and~\ref{s:harmonic}, we develop an analytic theory and compute its regime of
validity for LK-induced mergers. In Section~\ref{s:lk_enhanced}, we comment on
the consistency of our results with \citet{bin1}. We discuss and conclude in
Section~\ref{s:discussion}.

\begin{figure}
    \centering
    \includegraphics[width=\columnwidth]{LK90_plots/4sim_90_350.png}
    \caption{Sample orbital and spin evolution for the fiducial parameter
    regime. The unit of time $t_{\rm LK, 0}$ is the LK timescale
    [Eq.~\eqref{eq:t_lk}] evaluated for the initial conditions. The inner binary
    is taken to have $a = 100\;\mathrm{AU}$, $m_1 = 30M_{\odot}$, $m_2 =
    20M_{\odot}$, $I_0 = 90.35^\circ$, and $e_0 = 0.001$, while the tertiary
    SMBH has $\tilde{a}_3 = 2.2\;\mathrm{pc}$, $m_3 = 3 \times 10^7 M_{\odot}$.
    We take $\theta_{\rm sl}^{\rm i} = 0$. The top two panels show $a$ and $e$
    respectively. The bottom two panels show the inclination of the inner
    binary, both instantaneous ($I$) and appropriately averaged following
    Eq.~\eqref{eq:barI} ($\bar{I}$), and the instantaneous spin-orbit
    misalignment angle $\theta_{\rm sl}$. It can be seen that $\theta_{\rm
    sl}^{\rm f}$ converges rapidly to $\sim 90^\circ$ at the end of the
    simulation.}\label{fig:4sim_90_350}
\end{figure}
\begin{figure}
    \centering
    \includegraphics[width=\columnwidth]{LK90_plots/qslscan.png}
    \caption{Plot of merger time of the inner binary and $\theta_{\rm sl}^{\rm
    f}$ for the fiducial parameters, with $\theta_{\rm sl}^{\rm
    i} = 0$ and using a restricted range of $I_0$ \citep[analogous to the
    bottom-most panel in Fig.~3 of][]{bin2}, where the blue line is taken from
    numerical simulations. It is clear that for $I_0$ sufficiently far from
    $90^\circ$, the resulting $\theta_{\rm sl}^{\rm f}$ are all quite near
    $90^\circ$\citep{bin2}.}\label{fig:qslscan}
\end{figure}

\section{LK-Induced Mergers: Orbital Evolution}\label{s:setup_orbital}

We study Lidov-Kozai (LK) oscillations due to an external perturber to
quadrupole order and include apsidal precession and gravitational wave
radiation due to general relativity. Consider an black hole (BH) binary
with masses $m_1$ and $m_2$ having total mass $m_{12}$ and reduced mass $\mu$
orbiting a supermassive black hole (SMBH) with mass $m_3 \gg m_1$ and $m_2$.
Call $a_3$ and $e_3$ the semimajor axis and eccentricity of the orbit of the
inner binary around the SMBH, and define effective semimajor axis
\begin{equation}
    \tilde{a}_3 \equiv a_3\sqrt{1 - e_3^2}.
\end{equation}
Finally, call $\bm{L}_{\rm out} \equiv L_{\rm out} \uv{L}_{\rm out}$ the angular
momentum of the SMBH relative to the center of mass of the inner BH binary, and
call $\bm{L} \equiv L \uv{L}$ the orbital angular momentum of the inner BH
binary. We take $\bm{L}_{\rm out}$ to be fixed.

We then consider the motion of the inner binary, described by the Keplerian
orbital elements $(a, e, \ascnode, I, \omega)$ (respectively: semimajor axis,
eccentricity, longitude of the ascending node, inclination, and argument of
periapsis). The equations describing the motion of these orbital elements are
\citep{peters1964,storch,bin2}
\begin{align}
    \rd{a}{t} &= \p{\rd{a}{t}}_{\rm GW},\\
    \rd{e}{t} &= \frac{15}{8t_{\rm LK}} e\sqrt{1 - e^2}\sin 2\omega
        \sin^2 I + \p{\rd{e}{t}}_{\rm GW}\label{eq:dedt},\\
    \rd{\ascnode}{t} &= \frac{3}{4t_{\rm LK}}
        \frac{\cos I\p{5e^2 \cos^2\omega - 4e^2 - 1}}{\sqrt{1 - e^2}}
            \label{eq:dWdt},\\
    \rd{I}{t} &= \frac{15}{16}\frac{e^2\sin 2\omega \sin 2I}{
        \sqrt{1 - e^2}},\\
    \rd{\omega}{t} &= \frac{3}{4t_{\rm LK}}
        \frac{2\p{1 - e^2} + 5\sin^2\omega
            (e^2 - \sin^2 I)}{\sqrt{1 - e^2}}
        + \Omega_{\rm GR},\label{eq:dwdt}
\end{align}
where we define
\begin{align}
    t_{\rm LK}^{-1} &= n\p{\frac{m_3}{m_{12}}}\p{\frac{a}{\tilde{a}_3}}^3,
        \label{eq:t_lk}\\
    \p{\rd{a}{t}}_{\rm GW} &= -\frac{a}{t_{\rm GW}(e)},\\
    t_{\rm GW}^{-1}(e) &= \frac{64}{5}\frac{G^3 \mu m_{12}^2}{c^5a^4}
            \frac{1}{\p{1 - e^2}^{7/2}}\p{1 + \frac{73}{24}e^2
                + \frac{37}{96}e^4},\label{eq:dadt_gw}\\
    \p{\rd{e}{t}}_{\rm GW} &= -\frac{304}{15}\frac{G^3 \mu m_{12}^2}{c^5a^4}
        \frac{1}{\p{1 - e^2}^{5/2}}\p{1 + \frac{121}{304}e^2}\label{eq:dedt_gw}
            ,\\
    \Omega_{\rm GR}(e) &= \frac{3Gnm_{12}}{c^2a\p{1 - e^2}},
\end{align}
and $n = \sqrt{Gm_{12}/a^3}$ is the mean motion of the inner binary. We will
often refer to $e_{\min}$ and $e_{\max}$ the minimum/maximum eccentricity in a
single LK cycle and notate $j(e) = \sqrt{1 - e^2}$.

For concreteness, we adopt fiducial parameters similar to those from
\citet{bin2}: the inner binary has $a = 100\;\mathrm{AU}$, $m_1 = 30M_{\odot}$,
$m_2 = 20M_{\odot}$, and initial $e_0 = 0.001$ with varying $I_0$. We take the
SMBH tertiary companion to have $m_3 = 3 \times 10^{7} M_{\odot}$ and
$\tilde{a}_3 = 4.5 \times 10^5\;\mathrm{AU} = 2.2\;\mathrm{pc}$. This gives the
same $t_{\rm LK}$ as do the parameters used in \citet{bin2}. We refer to this as
the fiducial parameter regime, and all simulations use these parameters unless
otherwise noted.

We summarize a few key analytical properties of this orbital evolution below:
\begin{itemize}
    \item Neglecting GR effects (GW radiation and apsidal precession), the
        above equations describe periodic oscillations in $e$ and $I$. There are
        two conserved quantities, the total angular momentum $\bm{L}_{\rm out} +
        \bm{L}$ and the ``Kozai constant'' \citep{lidov,kozai}:
        \begin{equation}
            K \equiv j(e) \cos I.\label{eq:K_def}
        \end{equation}
        This implies that $e$ is a function of $\omega$ only
        \citep{kinoshita,storch}.

        Note that an eccentricity maxiumum occurs every half-period of $\omega$,
        at $\omega = \pi/2$ and $\omega = 3\pi/2$ \citep{kinoshita, storch}.
        It is most convenient to define the LK period $P_{\rm LK}$ as the
        \emph{half}-period of $\omega$, and the corresponding frequency
        $\Omega_{\rm LK}$:
        \begin{align}
            \pi &= \int\limits_0^{P_{\rm LK}}\rd{\omega}{t}\;\mathrm{d}t,\\
            \Omega_{\rm LK} &\equiv \frac{2\pi}{P_{\rm LK}}.\label{eq:WLK_def}
        \end{align}

        The conservation laws above can be combined to calculate the maximum
        eccentricity as a function of $I$. If the minimum eccentricity is
        negligible and the inclination at the start of the LK cycle is $I_0$,
        then
        \begin{equation}
            e_{\max} \equiv \sqrt{1 - \frac{5}{3}\cos^2 I_0}.\label{eq:emax}
        \end{equation}
        Note that these results assume $L_{\rm out} \gg L$ and that both
        octupole and GR effects are negligible; see \citet{bin2} for generalized
        forms of the above results.

        Finally, when $e_{\min} \ll e_{\max}$, the binary spends a fraction
        $\sim j(e_{\max})$ of the LK cycle near $e \simeq e_{\max}$
        \citep{anderson2016formation}.

    \item When GR effects are considered, the system gradually coalesces due
        to GW radiation, primarily in the $j(e_{\max})P_{\rm LK}$ window at
        eccentricity maxima. During this process, there are three competing
        timescales: (i) $\ascnode$ precession driven by the tertiary ($\sim
        t_{\rm LK}$), (ii) apsidal precession due to post-Newtonian effects
        ($\sim \Omega_{\rm GR}^{-1}(e_{\max}) / j(e_{\max})$), and (iii) orbital
        decay due to GW radiation ($\sim t_{\rm GW}(e_{\max}) / j(e_{\max})$).
        At early times, (i) is the shortest, but as $a$ decreases, (ii) and
        (iii) become dominant and eccentricity oscillations become suppressed,
        which we term ``eccentricity freezing''. We consider the two conditions
        under which this happens.

        Apsidal precession becomes important when $\Omega_{\rm GR}$ is the
        dominant contribution in Eq.~\eqref{eq:dwdt}. Quantitatively, this
        occurs when
        \begin{equation}
            \frac{\epsilon_{\rm GR}}{j(e_{\max})} \gg 1,
        \end{equation}
        where we define
        \begin{equation}
            \epsilon_{\rm GR} \equiv \frac{3Gm_{12}^2 \tilde{a}_3^3}{
                c^2m_3a^4}.
        \end{equation}

        GW radiation can also inhibit eccentricity oscillations when $t_{\rm
        LK}j(e) \gtrsim t_{\rm GW}(e)$ [see Eq.~\eqref{eq:dedt}]. In order for
        GW radiation to be the dominant cause of eccentricity freezing, we
        require
        \begin{align}
            1 \lesssim \frac{t_{\rm GW}^{-1}(e)}{\Omega_{\rm GR}}
                &\sim \p{\frac{Gm_{12}}{a(1 - e^2)c^2}}^{3/2} \frac{1}{1 - e^2}
                    ,\\
                &\sim \p{\frac{v_p}{c}}^3\frac{1}{1 - e^2},
        \end{align}
        where $v_p$ is the pericenter velocity of the inner binary. Since $v_p/c
        \ll 1$, we find that apsidal precession is usually more effective at
        freezing the eccentricity.
\end{itemize}
The behavior of $a$, $e$, and $I$ for a characteristic LK-induced merger can be
seen in Fig.~\ref{fig:4sim_90_350}. At early times, $e$ and $I$ have large
oscillations while $a$ slowly decreases. At later times, $e$ and $I$ stop
oscillating as apsidal precession freezes the eccentricity, and both $a$ and $e$
decrease under GW radiation. Since $P_{\rm LK}$ is defined as the half-period in
$\omega$, $\Omega_{\rm LK}$ asymptotes to $\Omega_{\rm GR}$ at late times even
though the eccentricity is frozen (bottom right panel of
Fig.~\ref{fig:4sim_90_350}).

\section{Spin Dynamics: Equations}\label{s:setup_spin}

We are ultimately interested in the spin orientations of the inner BHs at merger
as a function of initial conditions. Since they evolve independently to leading
post-Newtonian order, we focus on the dynamics of a single BH spin $\uv{S}$.
Since the spin magnitude does not enter into the dynamics, we write $\bm{S}
\equiv \uv{S}$ for brevity (i.e.\ $\bm{S}$ is a unit vector). Neglecting
spin-spin interactions, $\bm{S}$ undergoes de Sitter precession about $\bm{L}$
as
\begin{align}
    \rd{\bm{S}}{t} &= \Omega_{\rm SL}\hat{\bm{L}} \times \bm{S}
            \label{eq:dsdt},\\
        \Omega_{\rm SL} &= \frac{3Gn\p{m_2 + \mu/3}}{2c^2a\p{1 - e^2}}.
\end{align}

To analyze the dynamics of the spin vector, we go to co-rotating frame with
$\uv{L}$ about $\uv{L}_{\rm out}$, where Eq.~\eqref{eq:dsdt} becomes
\begin{align}
    \p{\rd{\bm{S}}{t}}_{\rm rot}
        &= \p{\Omega_{\rm L}\uv{L}_{\rm out}
            + \Omega_{\rm SL}\uv{L}} \times \bm{S},\\
        &= \bm{\Omega}_{\rm e} \times \bm{S}\label{eq:dsdt_weff},
\end{align}
where we define
\begin{align}
    \bm{\Omega}_{\rm e} &\equiv \Omega_{\rm L}\uv{L}_{\rm out} + \Omega_{\rm SL}
            \uv{L},\label{eq:weff_def}\\
    \Omega_{\rm L} &\equiv -\rd{\ascnode}{t}.\label{eq:Wldef}
\end{align}

\subsection{Nondissipative Dynamics}

Despite its simple form, Eq.~\eqref{eq:dsdt_weff} is difficult to analyze, since
$\bm{\Omega}_{\rm e}$ varies significantly within each LK cycle and evolves over
timescale $\sim t_{\rm GW}(e)$. For simplicity, we first consider the limit
where dissipation via GW radiation can be completely neglected ($t_{\rm GW}(e)
\to \infty$). Then $\bm{\Omega}_{\rm e}$ is exactly periodic with period $P_{\rm
LK}$. We can rewrite Eq.~\eqref{eq:dsdt_weff} in Fourier components
\begin{equation}
    \p{\rd{\bm{S}}{t}}_{\rm rot}
        = \s{\bm{\overline{\Omega}}_{\rm e} + \sum\limits_{N = 1}^\infty
            \bm{\Omega}_{\rm eN}\cos \p{N\Omega_{\rm LK}t}}
            \times \bm{S}.\label{eq:dsdt_fullft}
\end{equation}
We write $\bm{\overline{\Omega}}_{\rm e} \equiv \bm{\Omega}_{\rm e0}$ for
convenience, where the bar denotes an average over an LK cycle. We adopt
the convention where $t = 0$ is the time of maximum eccentricity of the LK
cycle.

This system resemebles that studied in \citet{storch}, where a star's obliquity
varies chaotically when driven by a hot Jupiter undergoing LK oscillations. One
strong indicator of chaos in their study is the presence of fine structure in a
bifurcation diagram [Fig.~1 of \citet{storch}] that shows the oscillation
amplitude of a misalignment angle $\theta_{\rm sl}$ when varying system
parameters in the ``transadiabatic'' regime, where some adiabaticity parameter
crosses $1$. In our system, the most natural angle to examine is
\begin{equation}
    \cos \theta_{\rm e} = \frac{\bm{\overline{\Omega}}_{\rm
        e}}{\overline{\Omega}_{\rm e}}\cdot \bm{S}.\label{eq:q_eff_inst}
\end{equation}
% and adiabaticity parameter $\mathcal{A}_0$ following \citet{bin1,bin2}
% \begin{equation}
%     \mathcal{A}_0 \equiv \abs{\frac{\Omega_{\rm SL}}{\Omega_{\rm L}}\sin 2I}
%         _{e = 0}
%         = \frac{4G(m_2 + \mu/3) m_{12}\tilde{a}_3^3}{c^2m_3a^4}.
% \end{equation}
% $\mathcal{A}_0 \simeq 1$ is a useful criterion for identifying where interesting
% behavior arises, the ``transadiabatic'' regime studied in \citet{storch,bin2}.

To generate an analogous bifurcation diagram for our problem, we begin with
$\bm{S} \parallel \bm{L}$, then evolve Eq.~\eqref{eq:dsdt_weff} while sampling
both $\theta_{\rm sl}$ and $\theta_{\rm e}$ at eccentricity maxima. We repeat
this procedure while varying the mass ratio $m_1 / m_{12}$ of the inner binary,
which only changes $\Omega_{\rm SL}$ without changing the orbital evolution.
Note that the fiducial parameters do not serve this purpose well because GR
effects are too weak initially, and the system is not in the desired regime.
Our choice of parameters and our results are shown in
Fig.~\ref{fig:bifurcation_70}.
\begin{figure}
    \centering
    \includegraphics[width=\columnwidth]{LK90_plots/M1_M2_thetasl_thetae_70.png}
    \caption{Bifurcation diagram for $I_0 = 70^\circ$, for physical parameters
    $m_1 + m_2 = 60M_{\odot}$, $m_3 = 3 \times 10^7 M_{\odot}$, $a_{\rm in} =
    0.1\;\mathrm{AU}$, $\tilde{a}_3 = 300\;\mathrm{AU}$, and $e_3 = 0$. For each
    mass ratio $m_1 / m_{12}$, Eq.~\eqref{eq:dsdt_weff} is solved over $500$ LK
    cycles, and both $\theta_{\rm e}$ and $\theta_{\rm sl}$ are sampled at every
    eccentricity maximum and are plotted. The top axis shows the adiabaticity
    parameter $\mathcal{A}$ as defined by Eq.~\eqref{eq:abar_def}, the natural
    adiabaticity parameter for this problem. Interesting behavior is expected
    for $\mathcal{A} \simeq 1$.}\label{fig:bifurcation_70}
\end{figure}

While our bifurcation diagram has some structure, it does not exhibit small
scale variability, suggesting that Eq.~\eqref{eq:dsdt_weff} cannot generate
chaotic behavior. We can understand this using Floquet theory, as
Eq.~\eqref{eq:dsdt_weff} is a linear system with periodic coefficients (the
system studied in SL15 is nonlinear). Floquet's theorem says that when a linear
system with periodic coefficients is integrated over a period, the evolution can
be described by a linear transformation, the \emph{monodromy matrix}
$\bm{\tilde{M}}$, or
\begin{equation}
    \bm{S}\p{t + P_{\rm LK}} = \bm{\tilde{M}} \bm{S}(t),
\end{equation}
where $\bm{\tilde{M}}$ is independent of $\bm{S}$.

While $\bm{\tilde{M}}$ can be easily defined, it cannot be evaluated in
closed form. Thankfully, it suffices to reason directly about the properties of
$\bm{\tilde{M}}$. In our problem, $\bm{M}$ must be a proper orthogonal matrix,
or a rotation matrix, as it represents the effect of many infinitesimal
rotations about $\bm{\Omega}_{\rm e}$\footnote{More formally, $\bm{\tilde{M}} =
\bm{\tilde{\Phi}}(P_{\rm LK})$ where $\bm{\tilde{\Phi}}(t)$ is the
\emph{principal fundamental matrix solution}: the columns of $\bm{\tilde{\Phi}}$
are solutions to Eq.~\eqref{eq:dsdt_weff} and $\bm{\tilde{\Phi}}(0)$ is the
identity. By linearity, the columns of $\bm{\tilde{\Phi}}(t)$ remain
orthonormal, while its determinant does not change, so $\bm{\tilde{M}}$ is a
proper orthogonal matrix, or a rotation matrix.}. Therefore, every $P_{\rm LK}$,
the dynamics of Eq.~\eqref{eq:dsdt_weff} are equivalent to a rotation about a
fixed axis, prohibiting chaotic behavior.

One other traditional indicator of chaos is a positive Lyapunov exponent,
obtained when the separation between nearby trajectories diverges
\emph{exponentially} in time. In Floquet theory, the Lyapunov exponent is the
logarithm of the largest eigenvalue of the monodromy matrix. Since
$\bm{\tilde{M}}$ is a rotation matrix here, the Lyapunov exponent must be $0$,
indicating no chaos. Numerically, we were able to verify the separation between
nearby trajectories does not grow in time.

\subsection{With GW Dissipation}

When $t_{\rm GW}$ is finite, the coefficients $\bm{\Omega}_{\rm eN}$ change over
timescales $\sim t_{\rm GW}(e_{\max}) / j(e_{\max})$ (see discussion in
Section~\ref{s:setup_orbital}). However, when the system satisfies both $t_{\rm
GW}(e_{\max}) \gg \Omega_{\rm GR}^{-1}(e_{\max})$ and $t_{\rm LK} j(e_{\max})$,
we can approximate its evolution as a sequence of nondissipative LK cycles. This
is a good approximation when coalescence takes place over more than a few
$P_{\rm LK}$, which is the case for most of the systems we study here.

Once $a$ is sufficiently small that $\Omega_{\rm SL} \gg \Omega_{\rm L}$ (this
also gives $\epsilon_{\rm GR} \gg 1$, implying the LK cycles are suppressed), it
can be seen from Eq.~\eqref{eq:dsdt_weff} that $\theta_{\rm e} = \theta_{\rm
sl}$ is constant (see bottom right panel of Fig.~\ref{fig:4sim_90_350}). For the
fiducial parameters, we stop the simulation at $a \leq 0.5\;\mathrm{AU}$, as
$\theta_{\rm sl}$ has converged to its final value.

\subsection{Component Form}

For later analysis, it is also useful to write Eq.~\eqref{eq:dsdt_fullft} in
components. To do so, it is next useful to define inclination angle
$\bar{I}_{\rm e}$ as the angle between $\bm{\overline{\Omega}}_{\rm e}$ and
$\bm{L}_{\rm out}$ as shown in Fig.~\ref{fig:3vec}. To express $\bar{I}_{\rm e}$
algebraically, we first define LK-averaged quantities
\begin{figure}
    \centering
    \includegraphics[width=0.5\columnwidth]{LK90_plots/7_3vec_cropped.png}
    \caption{Definition of angles, shown in plane of the two angular momenta
    $\bm{L}_{\rm out}$ and $\bm{L}$. Note that for $I_0 > 90^\circ$ (and
    $\bar{I} > 90^\circ$), we have $\bar{I}_{\rm e} < 0$ since $\Omega_{\rm L} <
    0$. The bottom right shows our choice of coordinate axes.}\label{fig:3vec}
\end{figure}
\begin{align}
    \overline{\Omega_{\rm SL} \sin I} &\equiv
            \overline{\Omega}_{\rm SL} \sin \bar{I},\\
    \overline{\Omega_{\rm SL} \cos I} &\equiv
            \overline{\Omega}_{\rm SL} \cos \bar{I}.\label{eq:barI}
\end{align}
It then follows that
\begin{equation}
    \tan \bar{I}_{\rm e} = \frac{\mathcal{A}\sin \bar{I}}{
        1 + \mathcal{A}\cos \bar{I}},\label{eq:ie_def}
\end{equation}
where
\begin{equation}
    \mathcal{A} \equiv \frac{\overline{\Omega}_{\rm SL}}{
        \overline{\Omega}_{\rm L}}.\label{eq:abar_def}
\end{equation}

We now can choose non-inertial coordinate system where $\uv{z} \propto
\bm{\overline{\Omega}}_{\rm e}$ and $\uv{x}$ lies in the plane of $\bm{L}_{\rm
out}$ and $\bm{L}$ with positive component along $\bm{L}$ (see
Fig.~\ref{fig:3vec}). In this reference frame, the polar coordinate is just
$\bar{\theta}_{\rm e}$ as defined above in Eq.~\eqref{eq:q_eff}, and the
equation of motion becomes
\begin{align}
    \rd{\bm{S}}{t} &= \s{\overline{\Omega}_{\rm e}\uv{z}
         + \sum\limits_{N = 1}^\infty
            \bm{\Omega}_{\rm eN}\cos \p{N\Omega_{\rm LK}t }}
        \times \bm{S}
        - \dot{\bar{I}}_{\rm e} \uv{y} \times \bm{S}\label{eq:eom_prime}.
\end{align}
One further simplification lets us cast this vector equation of motion into
scalar form. Break $\bm{S}$ into components $\bm{S} = S_x\uv{x} + S_y \uv{y} +
\cos \bar{\theta}_{\rm e} \uv{z}$ and define complex variable
\begin{equation}
    S_\perp \equiv S_x + iS_y.
\end{equation}
Then, we can rewrite Eq.~\eqref{eq:eom_prime} as
\begin{align}
    \rd{S_{\perp}}{t} ={}& i\overline{\Omega}_{\rm e}S_\perp
            - \dot{\bar{I}}_{\rm e} \cos \bar{\theta}_{\rm e}
        + \sum\limits_{N = 1}^\infty\left[
            \cos \p{\Delta I_{\rm eN}}S_\perp\right. \nonumber\\
        &- \left.i\cos \theta \sin \p{\Delta I_{\rm eN}}\right]
            \Omega_{\rm eN}\cos N\Omega_{\rm LK} t.\label{eq:formal_eom_allgen}
\end{align}
Here, for each $\bm{\Omega}_{\rm eN}$ Fourier harmonic, we denote its magnitude
$\Omega_{\rm eN}$ and its inclination angle relative to $\bm{L}_{\rm out}$ as
$I_{\rm eN}$ using the same convention as Fig.~\ref{fig:3vec} (where
$\bar{I}_{\rm e} \equiv I_{\rm e0}$), and $\Delta I_{\rm eN} = \bar{I}_{\rm e} -
I_{\rm eN}$.

\section{Analysis: Deviation from Adiabaticity}\label{s:fast_merger}

In general, Eq.~\eqref{eq:dsdt_fullft} is difficult to study analytically. Two
approximations can be made: (i) the effect of the $N \geq
1$ harmonic terms is negligible, and the dominant effect in the system is motion
of $\bm{\overline{\Omega}}_{\rm e}$; or (ii) the $\bm{\Omega}_{\rm eN}$ Fourier
coefficients are nonnegligible, and the dominant effect in the system arises
from resonant interactions due to the Fourier coefficients. In this section, we
analyze the former approximation and provide accurate analytic descriptions of
dynamics. The effect of the harmonic terms is studied in
Section~\ref{s:harmonic}.

\subsection{The Adiabatic Invariant}

When neglecting the $N \geq 1$ harmonic terms, the equation of motion is
modified to
\begin{equation}
    \p{\rd{\bm{\overline{S}}}{t}}_{\rm rot}
        = \bm{\overline{\Omega}}_{\rm e}
            \times \bm{\overline{S}}.\label{eq:dsdt_0only}
\end{equation}

It is not obvious in what capacity analysis of Eq.~\eqref{eq:dsdt_0only} is
applicable to Eq.~\eqref{eq:dsdt_fullft}. Empirically, we find that the
LK-average of $\bm{S}$ (which has magnitude $\leq 1$) often evolves following
Eq.~\eqref{eq:dsdt_0only}, motivating our notation $\bm{\overline{S}}$.
Over timescales $\lesssim P_{\rm LK}$, Eq.~\eqref{eq:dsdt_0only} loses accuracy
as the evolution of $\bm{S}$ itself is dominated by the $N \geq 1$ harmonics we
have neglected. An intuitive interpretation of this result is that the $N \geq
1$ harmonics vanish when integrating Eq.~\eqref{eq:dsdt_fullft} over an LK
cycle.

Eq.~\eqref{eq:dsdt_0only} has one desirable property: $\bar{\theta}_{\rm e}$,
given by
\begin{equation}
    \cos \bar{\theta}_{\rm e} \equiv
        \bm{\overline{S}} \cdot \frac{\bm{\overline{\Omega}}_{\rm
            e}}{\overline{\Omega}_{\rm e}},
        \label{eq:q_eff}
\end{equation}
is an adiabatic invariant. The adiabaticity condition requires the precession
axis evolve slowly compared to the precession frequency at all times:
\begin{equation}
    \rd{\bar{I}_{\rm e}}{t} \ll \overline{\Omega}_{\rm e}.\label{eq:ad_constr}
\end{equation}
For the simulation shown in Fig.~\ref{fig:4sim_90_350}, the values of
$\dot{\bar{I}}_{\rm e}$ and $\overline{\Omega}_{\rm e}$ are shown in the top
panel of Fig.~\ref{fig:4sim_90_350_supp}. Note that $\dot{\bar{I}}_{\rm e} \ll
\overline{\Omega}_{\rm e}$ at all times, and the total change in
$\bar{\theta}_{\rm e}$ in this simulation is $0.01^\circ$, small as expected.

% Here, $\bar{I}_{\rm e}$ changes on timescale $\sim t_{\rm GW}(e_{\max}) /
% j(e_{\max})$ (see discussion at the end of Section~\ref{s:setup_orbital}) while
% \begin{align}
%     \overline{\Omega}_{\rm e} &\approx \max\p{\Omega_{\rm SL}, \Omega_{\rm
%             L}},\\
%         &\sim \max\p{\frac{\Omega_{\rm SL}(e_{\max})}{
%             j(e_{\max})}, t_{\rm LK}^{-1}}.
% \end{align}
% Since $\Omega_{\rm SL} \sim \Omega_{\rm GR}$, we see that the adiabatic
% assumption and the Fourier decomposition have similar regimes of validity.

\subsection{Calculating Deviation from Adiabaticity}\label{ss:eom_0}

In real systems, the particular extent to which $\bar{\theta}_{\rm e}$ is
conserved depends on how well Eq.~\eqref{eq:ad_constr} is satisfied. In this
subsection, we derive a loose bound on the total non-conservation of
$\bar{\theta}_{\rm e}$, then in the next subsection we show this bound can be
estimated from initial conditions.

When neglecting harmonic terms, the scalar equation of motion
Eq.~\eqref{eq:formal_eom_allgen} becomes
\begin{align}
    \rd{S_{\perp}}{t} &= i\overline{\Omega}_{\rm e}S_\perp
            - \dot{\bar{I}}_{\rm e} \cos \bar{\theta}_{\rm e}.
\end{align}
This can be solved in closed form using an integrating factor. Defining
\begin{equation}
    \Phi(t) \equiv \int\limits^t \overline{\Omega}_{\rm e}\;\mathrm{d}t,
\end{equation}
we obtain solution up between initial time $t_{\rm i}$ and final time $t_{\rm f}$
\begin{equation}
    e^{-i\Phi}S_{\perp}\bigg|_{t_{\rm i}}^{t_{\rm f}}
        = -\int\limits_{t_{\rm i}}^{t_{\rm f}}
            e^{-i\Phi(\tau)}\dot{\bar{I}}_{\rm e} \cos \theta\;\mathrm{d}\tau.
            \label{eq:formal_sol_0}
\end{equation}
Recalling $\abs{S_{\perp}} = \sin \bar{\theta}_{\rm e}$ and analyzing
Eq.~\eqref{eq:formal_sol_0}, we see that $\bar{\theta}_{\rm e}$ oscillates about
its initia value $\theta_{\rm e}^{\rm i}$ with amplitude
\begin{equation}
    \abs{\Delta \bar{\theta}_{\rm e}} \sim
        \frac{\dot{\bar{I}}_{\rm e}}{\overline{\Omega}_{\rm
        e}}.\label{eq:nonad_dqeff}
\end{equation}
It can be seen that, in the adiabatic limit [Eq.~\eqref{eq:ad_constr}],
$\bar{\theta}_{\rm e}$ is indeed conserved, as the right hand side goes to zero.

This is compared to the $\Delta \bar{\theta}_{\rm e}$ for the fiducial
simulation in the bottom panel of Fig.~\ref{fig:4sim_90_350_supp}. Note that
$\bar{\theta}_{\rm e}$ is indeed mostly constant where
Eq.~\eqref{eq:nonad_dqeff} predicts small oscillations.

Furthermore, if we denote $\abs{\Delta \bar{\theta}_{\rm e}}^{\rm f}$ to be the
total change in $\bar{\theta}_{\rm e}$ over $t \in [t_{\rm i}, t_{\rm f}]$, we
can give loose bound\footnote{Given the complicated evolution of
$\overline{\Omega}_{\rm e}$ and $\dot{\bar{I}}_{\rm e}$, it is difficult to give
a more exact bound on the deviation from adiabaticity. In practice, deviations
$\lesssim 1^\circ$ are observationally indistinguishable, so the exact scaling
in this regime is negligible.}
\begin{equation}
    \abs{\Delta \bar{\theta}_{\rm e}}^{\rm f} \lesssim
        \abs{\frac{\dot{\bar{I}}_{\rm e}}{\overline{\Omega}_{\rm e}}}_{\max}.
        \label{eq:nonad_dqeff_tot}
\end{equation}

Inspection of Fig.~\ref{fig:4sim_90_350_supp} indicates that the dynamics are
mostly uninteresting except near the peak of $\dot{\bar{I}}_{\rm e}$, which is
where $\Omega_{\rm SL} \simeq \Omega_{\rm L}$. We present a zoomed in view of
the behavior of all dynamical quantities near the peak of $\dot{\bar{I}}_{\rm
e}$ in the fiducial simulation in Fig.~\ref{fig:4sim_90_350_zoom}. In
particular, in the bottom-rightmost plot, we see that the fluctuations in
$\bar{\theta}_{\rm e}$ are dominated by a second contribution, the subject of
the discussion in Section~\ref{s:harmonic}. For comparison, we show a more rapid
coalescence, where $I_0 = 90.2^\circ$, for which $\abs{\Delta \theta_{\rm
e}}^{\rm f} \approx 2^\circ$, in Fig.~\ref{fig:4sim_90_200_zoom}. If we again
examine the bottom-rightmost plot, we see that the fluctuations in $\Delta
\bar{\theta}_{\rm e}$ are somewhat better described by Eq.~\eqref{eq:nonad_dqeff}.

\begin{figure}
    \centering
    \includegraphics[width=\columnwidth]{LK90_plots/4sim_90_350_supp.png}
    \caption{The same simulation as Fig.~\ref{fig:4sim_90_350} but showing
    calculated quantities relevant to the theory of the paper over the course of
    the simulation. Top: the important frequencies of the system. Middle: the
    frequency ratios between the zeroth and first Fourier components of
    $\bm{\Omega}_{\rm e}$ to the LK frequency $\Omega_{\rm LK}$. Bottom: Plot of
    $\theta_{\rm e}$ [grey; Eq.~\eqref{eq:q_eff_inst}], $\bar{\theta}_{\rm e}$
    [red dots; Eq.~\eqref{eq:q_eff}], as well as estimates of the deviations from
    perfect conservation of $\bar{\theta}_{\rm e}$ due to non-adiabatic effects
    [green, Eq.~\eqref{eq:nonad_dqeff}] and due to resonances with harmonic
    terms [blue, Eq.~\eqref{eq:harmonic_dqeff}].}\label{fig:4sim_90_350_supp}
\end{figure}
\begin{figure*}
    \centering
    \includegraphics[width=0.9\textwidth]{LK90_plots/4sim_90_350_zoom.png}
    \caption{The same simulation as Fig.~\ref{fig:4sim_90_350} but shown
    focusing on the region where $\mathcal{A} \simeq 1$. The first three panels
    depict $a$, $e$, $I$ and $\bar{I}$ as before, while the fourth shows $I_{\rm
    e}$ [Eq.~\eqref{eq:ie_def}] and $I_{\rm e1}$. The bottom four panels depict
    $\theta_{\rm sl}$; the five characteristic frequencies of the system
    [Eqs.~\ref{eq:weff_def} and~\eqref{eq:Wldef}]; and the same quantities as
    the top and bottom panels of Fig.~\ref{fig:4sim_90_350_supp}. In the final
    panel, it is clear that oscillations in $\bar{\theta}_{\rm e}$ are
    dominantly driven by interactions with the $N = 1$
    harmonic.}\label{fig:4sim_90_350_zoom}
\end{figure*}

\begin{figure*}
    \centering
    \includegraphics[width=0.9\textwidth]{LK90_plots/4sim_90_200_zoom.png}
    \caption{Same as Fig.~\ref{fig:4sim_90_350_zoom} except for $I_0 =
    90.2^\circ$, corresponding to a faster coalescence and a total change in
    $\bar{\theta}_{\rm e}$ of $\approx 2^\circ$. In the bottom middle panel, the
    nonadiabatic contribution is more significant and causes much poorer
    conservation of $\bar{\theta}_{\rm e}$.}\label{fig:4sim_90_200_zoom}
\end{figure*}

\subsection{Estimate of Deviation from Adiabaticity from Initial Conditions}

To estimate $\abs{\dot{\bar{I}}_{\rm e} / \overline{\Omega}_{\rm e}}_{\max}$ from
initial conditions, we first differentiate Eq.~\eqref{eq:ie_def},
\begin{equation}
    \dot{\bar{I}}_{\rm e} = \p{\frac{\dot{\mathcal{A}}}{
            \mathcal{A}}}
        \frac{\mathcal{A} \sin \bar{I}}{
            1 + 2\mathcal{A}\cos \bar{I}
                + \mathcal{A}^2}.
\end{equation}
It also follows from Eq.~\eqref{eq:weff_def} that
\begin{equation}
    \overline{\Omega}_{\rm e} = \overline{\Omega}_{\rm L}
        \p{1 + 2\mathcal{A}\cos \bar{I}
            + \overline{\mathcal{A}}^2}^{1/2},
\end{equation}
from which we obtain
\begin{equation}
    \abs{\frac{\dot{\bar{I}}_{\rm e}}{\overline{\Omega}_{\rm e}}}
        = \p{\frac{\dot{\mathcal{A}}}{
            \mathcal{A}}}
        \frac{1}{\abs{\overline{\Omega}_{\rm L}}}
        \frac{\mathcal{A} \sin \bar{I}}{
            \p{1 + 2\mathcal{A}\cos \bar{I}
                + \mathcal{A}}^{3/2}}.\label{eq:idot_over_W_med}
\end{equation}
Moreover, if we approximate $e \approx 1$ and $\cos^2 \omega \simeq 1/2$ in
$\rdil{\ascnode}{t}$, we can estimate
\begin{align}
    \mathcal{A} &\simeq
        \frac{3Gn\p{m_2 + \mu/3}}{
            2c^2a j^2}
                \s{\frac{15\cos \bar{I}}{
                    8t_{\rm LK}j}}^{-1},\\
        &\simeq \frac{4}{5}\frac{G(m_2 + \mu/3) m_{12}\tilde{a}_3^3}{
            c^2m_3a^4 j \cos \bar{I}},\label{eq:abar_eq1}\\
    \frac{\dot{\mathcal{A}}}{\mathcal{A}}
        &= -4\p{\frac{\dot{a}}{a}}_{\rm GW}
            + \frac{e}{j^2}\p{\rd{e}{t}}_{\rm GW}.
\end{align}
With this, we see that Eq.~\eqref{eq:idot_over_W_med} is small except when
$\mathcal{A} \simeq 1$, and so we obtain that the maximum $\dot{\bar{I}}_{\rm e}
/ \overline{\Omega}_{\rm e}$ is given roughly by
\begin{equation}
    \abs{\frac{\dot{\bar{I}}_{\rm e}}{\overline{\Omega}_{\rm e}}}_{\max}
        \simeq \p{\frac{\dot{\mathcal{A}}}{\mathcal{A}}}
            \frac{1}{\abs{\overline{\Omega}_{\rm L}}}
            \frac{\sin \bar{I}}{\p{2 + 2\cos \bar{I}}^{3/2}}.
            \label{eq:idot_over_W}
\end{equation}

To evaluate this, we make two assumptions: (i) $\bar{I}$ is approximately
constant (see third panels of Figs.~\ref{fig:4sim_90_350_zoom}
and~\ref{fig:4sim_90_200_zoom}), and (ii) $j(e)$ evaluated at $\mathcal{A}
\simeq 1$ can be approximated as a constant multiple of the initial
$j(e_{\max})$, so that
\begin{equation}
    j_{\star} \equiv j(e_{\star}) = f
        \sqrt{\frac{5}{3}\cos^2 I_0},\label{eq:jstar_ansatz}
\end{equation}
for some unknown factor $f > 1$; we use star subscripts to denote evaluation at
$\mathcal{A} \simeq 1$. $f$ turns out to be relatively insensitive to $I_0$.
This can be as systems with lower $e_{\max}$ values taking more cycles to attain
$\mathcal{A} \simeq 1$, resulting in all systems experiencing a similar amount
of decay due to GW radiation.

For simplicity, let's first assume $\mathcal{A} \simeq 1$ is satisfied when the
LK oscillations are mostly suppressed, and $e_\star \approx 1$ is a constant
throughout the LK cycle (see second panels of Figs.~\ref{fig:4sim_90_350_zoom}
and~\ref{fig:4sim_90_200_zoom}; we will later see that the scalings are the same
in the LK-oscillating regime). Approximating $e_\star \approx 1$ in
Eqs.~\eqref{eq:dadt_gw} and~\eqref{eq:dedt_gw} gives
\begin{equation}
    \s{\frac{\dot{\mathcal{A}}}{\mathcal{A}}}_{\star}
        \simeq \frac{G^3 \mu m_{12}^2}{c^5a_\star^4j_\star^7} \frac{595}{3}.
\end{equation}
Again assuming $e_\star \approx 1$ and averaging $\cos^2 \omega \simeq 1/2$ in
$\rdil{\ascnode}{t}$ Eq.~\eqref{eq:dWdt} gives
\begin{equation}
    \overline{\Omega}_{\rm L, \star}
        \approx \frac{15\cos \bar{I}}{8t_{\rm LK, \star}j_\star}.
\end{equation}
Lastly, to fix $a_\star$, we require Eq.~\eqref{eq:abar_eq1} to give
$\mathcal{A} = 1$ for $a_\star$ and $j_\star$. Taking all of these together, we
obtain
\begin{align}
    \abs{\frac{\dot{\bar{I}}_{\rm e}}{\overline{\Omega}_{\rm e}}}_{\max}
        \approx{}& \frac{952}{9}\frac{G^3 \mu m_{12}^3 \tilde{a}_3^3}{c^5
        \sqrt{Gm_{12}} m_3}
            \p{\frac{5c^2 m_3 \cos \bar{I}}{4G(m_2 + \mu / 3) m_{12}
                \tilde{a}_3^3}}^{11/8}\nonumber\\
        &\times \p{j_\star}^{-37/8}
            \frac{\tan \bar{I}}{
            \p{2 + 2 \cos \bar{I}}^{3/2}},\\
        \approx{}& XXX \p{\frac{\tilde{a}_3}{2.2\;\mathrm{pc}}}^{-9/8}
            \dots\label{eq:prediction}
\end{align}
We take constant $\bar{I} = 120^\circ$ (Figs.~\ref{fig:4sim_90_350_zoom}
and~\ref{fig:4sim_90_200_zoom} show that this is a serviceable approximation
that holds across a range of $I_0$). Then Eq.~\eqref{eq:prediction} has only one
free parameter, $f$, which enters through $j_\star$ following
Eq.~\eqref{eq:jstar_ansatz}. We can also calculate $\dot{\bar{I}}_{\rm e} /
\overline{\Omega}_{\rm e}$ from numerical simulations. Fitting $f$ to the data
from numerical simulations gives $f = 2.6$. The result of this fit is shown in
Fig.~\ref{fig:good_quants}, where it is clear the scaling predicted by
Eq.~\eqref{eq:prediction} is remarkably accurate.
\begin{figure}
    \centering
    \includegraphics[width=\columnwidth]{LK90_plots/good_quants.png}
    \caption{Comparison of $\abs{\dot{\bar{I}}_{\rm e} / \overline{\Omega}_{\rm
    e}}_{\max}$ extracted from simulations and using Eq.~\eqref{eq:prediction},
    where we take $f = 2.6$ in Eq.~\eqref{eq:jstar_ansatz}. The coalescence time
    $T_{\rm m}$ is shown along the top axis of the plot in units of the
    characteristic LK timescale at the start of inspiral $t_{\rm LK, 0}$; the LK
    period is initially of order a few $t_{\rm LK, 0}$. The agreement is
    remarkable for systems that coalesce over many $t_{\rm LK, 0}$.
    }\label{fig:good_quants}
\end{figure}

Above, we assumed that the system evolves through $\mathcal{A} \simeq 1$  when
the eccentricity is mostly frozen (see Fig.~\ref{fig:4sim_90_350} for an
indication of how accurate this is for the parameter space explored in
Fig.~\ref{fig:good_quants}). It is also possible that $\mathcal{A} \simeq 1$
occurs when the eccentricity is still undergoing substantial oscillations. In
fact, Eq.~\eqref{eq:prediction} is still accurate in this regime when replacing
$e$ with $e_{\max}$, due to the following analysis. Recall that when $e_{\min}
\ll e_{\max}$, the binary spends a fraction $\sim j(e_{\max})$ of the LK cycle
near $e \simeq e_{\max}$ \citep{anderson2016formation}. This fraction of the LK
cycle dominates both GW dissipation and $\overline{\Omega}_{\rm e}$ precession.
Thus, both $\dot{\bar{I}}_{\rm e}$ and $\overline{\Omega}_{\rm e}$ in the
oscillating-$e$ regime are evaluated by setting $e \approx e_{\max}$ and adding
a prefactor of $j(e_{\max})$. The prefactors of $j(e_{\max})$ cancel when
computing the quotient $\dot{\bar{I}}_{\rm e} / \overline{\Omega}_{\rm e}$.

The accuracy of Eq.~\eqref{eq:prediction} in bounding $\abs{\Delta
\bar{\theta}_{\rm e}}^{\rm f}$ is shown in Fig.~\ref{fig:deviations}. Note that
conservation of $\bar{\theta}_{\rm e}$ is generally much better than
Eq.~\eqref{eq:prediction} predicts. This is because cancellation of phases in
Eq.~\eqref{eq:formal_sol_0} is generally more efficient than
Eq.~\eqref{eq:prediction} assumes.
\begin{figure}
    \centering
    \includegraphics[width=\columnwidth]{LK90_plots/deviations_one.png}
    \caption{Total change in $\bar{\theta}_{\rm e}$ over inspiral as a function of
    initial inclination $I_0$, where the initial $\bm{\overline{\Omega}}_{\rm
    e}$ is computed without GW dissipation. For each $I_0$, $100$ simulations
    are run for $\bm{S}$ on a uniform, isotropic grid. Plotted for comparison is
    the bound $\abs{\Delta \bar{\theta}_{\rm e}}^{\rm f} \lesssim
    \abs{\dot{\bar{I}}_{\rm e} /
    \overline{\Omega}_{\rm e}}_{\max}$, using the analytical scaling given by
    Eq.~\eqref{eq:prediction}. It is clear that the given bound is not tight but
    provides an upper bound for non-conservation of $\bar{\theta}_{\rm e}$ due to
    nonadiabatic effects. At the right of the plot, the accuracy saturates: this
    is because neglecting GW dissipation causes inaccuracies when computing the
    average $\bm{\overline{\Omega}}_{\rm e}$.}\label{fig:deviations}
\end{figure}

\subsection{$\theta_{\rm sl}^{\rm f}$ Behavior}\label{eq:effect}

In Fig.~\ref{fig:qslscan}, we plotted $\theta_{\rm sl}^{\rm f}$ as a function of
$I_0$ when $\theta_{\rm sl}^{\rm i} = 0$. With the results developed in the
previous sections, we can understand very well the behavior seen in this plot.

Note that, for the fiducial parameter space, $\overline{\Omega}_{\rm L} \gg
\overline{\Omega}_{\rm SL}$ initially (Fig.~\ref{fig:4sim_90_350_supp}), and so
by Eq.~\eqref{eq:weff_def} we have
\begin{equation}
    \frac{\bm{\overline{\Omega}}_{\rm e}}{\overline{\Omega}_{\rm e}} \approx
        \mathrm{sgn}\p{\overline{\Omega}_{\rm L}} \uv{L}_{\rm
        out}.\label{eq:Weff_approx}
\end{equation}
If then $\theta_{\rm sl}^{\rm i} = 0^\circ$, we have $\bar{\theta}_{\rm e}^{\rm i}
\approx I_0$, and when $\bar{\theta}_{\rm e}$ is conserved, we obtain the desired
result $\bar{\theta}_{\rm e}^{\rm f} = \theta_{\rm sl}^{\rm f} \approx 90^\circ$. Note
that this is a general feature for LK-induced mergers if $\theta_{\rm sl}^{\rm
i} = 0^\circ$, since $I_0 \approx 90^\circ$ is required for sufficient
$e$-excitation to cause the binary to merge. Careful examination of
Fig.~\ref{fig:qslscan} shows the $\theta_{\rm sl}^{\rm f}$ values for slower
mergers is slightly below $90^\circ$. This is because
$\bm{\overline{\Omega}}_{\rm e}$ differs slightly from $\bm{L_{\rm out}}$ when
$\overline{\Omega}_{\rm SL}$ is nonzero, and indeed $\bar{\theta}_{\rm e} \approx
88.5^\circ$ for $I_0 = 90.5^\circ$ and $I_0 = 89.5^\circ$.

This also reproduces a recent result that $\theta_{\rm sl}^{\rm f}$ either
equals $\theta_{\rm sb}^{\rm i}$ or $180^\circ - \theta_{\rm sb}^{\rm i}$, where
where $\theta_{\rm sb}^{\rm i}$ is the initial angle between $\bm{L}_{\rm out}$
and $\bm{S}$ \citep{yu2020spin}. Noting $\mathrm{sgn}\p{\overline{\Omega}_{\rm
L}}$ is positive for $I_0 < 90^\circ$ and negative for $I_0 > 90^\circ$, we get
\begin{equation}
    \bar{\theta}_{\rm e}^{\rm i} \approx
    \begin{cases}
        \theta_{\rm sb}^{\rm i} & I_0 < 90^\circ,\\
        180^\circ - \theta_{\rm sb}^{\rm i} & I_0 > 90^\circ,
    \end{cases}
\end{equation}
Following the above analysis, $\bar{\theta}_{\rm e}$ conservation then
immediately gives
\begin{equation}
    \theta_{\rm sl}^{\rm f} \approx
        \begin{cases}
            \theta_{\rm sb}^{\rm i} & I_0 < 90^\circ,\\
            180^\circ - \theta_{\rm sb}^{\rm i} & I_0 > 90^\circ.
        \end{cases}
\end{equation}

Having understood the behavior of the $\theta_{\rm sl}^{\rm f}$ distribution for
adiabatic mergers and the $90^\circ$ attractor, we next turn to the structure of
the deviations in Fig.~\ref{fig:qslscan}. The deviations grow towards $I_0 =
90^\circ$. This follows the shape predicted by Eq.~\eqref{eq:prediction} and shown
in Fig.~\ref{fig:deviations}. The oscillatory nature of these deviations can
also be understood: Eq.~\eqref{eq:nonad_dqeff_tot} only gives the maximum of the
absolute value of the change in $\bar{\theta}_{\rm e}$, while the actual change
depends on the initial and final complex phases, denoted $\Phi(t_{\rm i})$ and
$\Phi(t_{\rm f})$. When $\theta_{\rm sl}^{\rm i} = 0$, we have $\Phi(t_{\rm i})
= 0$, as $\bm{S}$ starts in the $\uv{x}$-$\uv{z}$ plane. Then, as $I_0$ is
smoothly varied, the final phase $\Phi\p{t_{\rm f}}$ will also smoothly vary, so
the total phase difference between the initial and final values of $S_{\perp}$
vary smoothly. This means the total change in $\bar{\theta}_{\rm e}$ will
fluctuate smoothly between $\pm \abs{\Delta \bar{\theta}_{\rm e}}^{\rm f}$ as
$I_0$ is smoothly varied, giving rise to the sinusoidal shape seen in
Fig.~\ref{fig:qslscan}.

\section{Analysis: Effect of Resonances}\label{s:harmonic}

In the previous section, we neglected the $N \geq 1$ Fourier harmonics in
Eq.~\eqref{eq:dsdt_fullft}, and showed that the final $\theta_{\rm sl}^{\rm f}$
behavior could be completely explained. In this section, we study one effect of
the Fourier harmonics that occurs when two frequencies become commensurate.
We show that, while this effect can be neglected for the fiducial parameter
regime, it becomes important to explain some of the results in \citet{bin1}.
This connection is discussed in Section~\ref{s:lk_enhanced}.

For simplicity, we ignore the effects of GW dissipation in this section and
assume the system is exactly periodic (so $\dot{\bar{I}}_{\rm e} = 0$). The
scalar equation of motion Eq.~\eqref{eq:formal_eom_allgen} is then:
\begin{align}
    \rd{S_{\perp}}{t} ={}& i\overline{\Omega}_{\rm e}S_\perp
        + \sum\limits_{N = 1}^\infty[
            \cos \p{\Delta I_{\rm eN}}S_\perp \nonumber\\
        &- i\cos \theta \sin \p{\Delta I_{\rm eN}}]
            \Omega_{\rm eN}\cos (N\Omega_{\rm LK} t).\label{eq:formal_sol_gen}
\end{align}
Resonances can occur when $\overline{\Omega}_{\rm e} = N\Omega_{\rm LK}$.
Numerically, we find that $\overline{\Omega}_{\rm e} \lesssim \Omega_{\rm LK}$
for most regions of parameter space (see Fig.~\ref{fig:dWs}, and recall that
LK-induced mergers only complete within a Hubble time when $I_{\min} \approx
90^\circ$). Accordingly, we restrict our analysis to resonances with the $N = 1$
component. For simplicity, we also ignore the modulation of the forcing
frequency in Eq.~\eqref{eq:formal_sol_gen}. While this fails to capture the
possibility of a parametric resonance, we find no evidence for parametric
resonances in our simulations. With these two simplifications,
Eq.~\eqref{eq:formal_sol_gen} further reduces to
\begin{align}
    \rd{S_{\perp}}{t} &\approx i\overline{\Omega}_{\rm e}S_\perp
        - i\cos \bar{\theta}_{\rm e} \sin \p{\Delta I_{\rm e1}} \Omega_{\rm e1}
            \cos \p{\Omega_{\rm LK} t}.
\end{align}
We can approximate $\cos \p{\Omega_{\rm LK}t} \approx e^{i\Omega_{\rm LK} t} /
2$, as the $e^{-i\Omega_{\rm LK} t}$ component is far from resonance. Then we
can write down solution as before
\begin{align}
    e^{-i\overline{\Omega}_{\rm e}t}S_{\perp}\bigg|_{t_{\rm i}}^{t_{\rm f}}
        &= -\int\limits_{t_{\rm i}}^{t_{\rm f}}
            \frac{i\sin\p{\Delta I_{\rm e1}} \Omega_{\rm e1}}{2}
                e^{-i\overline{\Omega}_{\rm e}t + i\Omega_{\rm LK} \tau} \cos
                \bar{\theta}_{\rm e}
            \;\mathrm{d}\tau.\label{eq:harmonic_dS}
\end{align}
Thus, similarly to Section~\ref{ss:eom_0}, the instantaneous oscillation
amplitude $\abs{\Delta \bar{\theta}_{\rm e}}$ can be bound by
\begin{equation}
    \abs{\Delta \bar{\theta}_{\rm e}} \sim \frac{1}{2}
        \abs{\frac{\sin \p{\Delta I_{\rm e1}} \Omega_{\rm e1}}{
            \Omega_{\rm LK} - \overline{\Omega}_{\rm e}}}.
        \label{eq:harmonic_dqeff}
\end{equation}
We see that if $\overline{\Omega}_{\rm e} < \Omega_{\rm LK}$ by a sufficient
margin for all times, then the conservation of $\bar{\theta}_{\rm e}$ in the
fiducial parameter regime cannot be significantly affected by this resonance.
The ratio $\overline{\Omega}_{\rm e} / \Omega_{\rm LK}$ is shown in the middle
panel of Fig.~\ref{fig:4sim_90_350_supp}, and the amplitude of oscillation of
$\bar{\theta}_{\rm e}$ it generates [Eq.~\eqref{eq:harmonic_dqeff}] is given in
blue in the bottom panel of Fig.~\ref{fig:4sim_90_350_supp}. We see that the
total effect of the harmonic terms never exceeds a few degrees.

Furthermore, we see from Fig.~\ref{fig:4sim_90_350_zoom} (the fourth and seventh
panels) that the interesting dynamics, occuring when $\bm{\Omega}_{\rm e1}$ and
$\Delta I_{\rm e1}$ are both nonzero [necessary for
Eq.~\eqref{eq:harmonic_dqeff} to be nonzero], occur in the regime $\mathcal{A}
\simeq 1$. Thn, the bottom-rightmost panel of Fig.~\ref{fig:4sim_90_350_zoom}
compares the detailed behavior of $\bar{\theta}_{\rm e}$ and its two
contributions, the nonadiabatic and harmonic effects. We see that
Eq.~\eqref{eq:harmonic_dqeff} describes the oscillations in $\bar{\theta}_{\rm
e}$ very well. The agreement is poorer in the bottom-rightmost panel of
Fig.~\ref{fig:4sim_90_200_zoom}, as the nonadiabatic effect is much stronger.
However, note that the theory presented in the previous section captures the
final deviations $\abs{\Delta \bar{\theta}_{\rm e}}^{\rm f}$ very well. This
suggests that oscillations in $\bar{\theta}_{\rm e}$ due to
Eq.~\eqref{eq:harmonic_dqeff} of up to a few degrees do not affect final
nonconservation by more than $\sim 0.01^\circ$.

\begin{figure}
    \centering
    \includegraphics[width=\columnwidth]{LK90_plots/5_dWs.png}
    \caption{$e_{\max}$ and $\overline{\Omega}_{\rm e} / \Omega_{\rm LK}$ as a
    function of $I_{\min}$, the inclination of the inner binary at eccentricity
    minimum, for varying values of $e_{\min}$ (different colors) for the
    fiducial parameter regime. }\label{fig:dWs}
\end{figure}

\section{Lidov-Kozai Enhanced Mergers}\label{s:lk_enhanced}

In \citet{bin1}, a different parameter regime is considered,
where the inner binary is sufficiently close in ($\sim 0.1\;\mathrm{AU}$) that
it can merge in isolation via GW radiation, given by: $m_1 = m_2 = m_3 =
30M_{\odot}$, $a_{\rm in} = 0.1\;\mathrm{AU}$, $\tilde{a}_3 = 3\;\mathrm{AU}$,
and $e_3 = 0$. Note that $m_3$ here is not an SMBH\@. However, our results can
still be applied judiciously to this parameter regime and yield interesting
insights.

First, we recall the $\theta_{\rm sl}^{\rm f}$ distribution obtained via
numerical simulation, shown in Fig.~\ref{fig:bin_comp}. \citet{bin1} derives an
adiabatic invariant assuming the inner binary does not undergo eccentricity
oscillations. Our result, based on $\bar{\theta}_{\rm e}$ conservation, is a
generalization of their result, giving the same result when the inner orbit
remains circular. Very near $I_0 \approx 90^\circ$, the data are offset somewhat
from our result, because we have assumed the tertiary's angular momentum is
fixed, but accounting for the offset, our theory captures the scaling of
$\theta_{\rm sl}^{\rm f}$, as seen in Fig.~\ref{fig:bin_comp_zoom}.

\begin{figure}
    \centering
    \includegraphics[width=\columnwidth]{LK90_plots/6bin_comp.png}
    \caption{Plot of $\theta_{\rm sl}^{\rm f}$ for the LK-enhanced parameter
    regime, i.e.\ $m_1 = m_2 = m_3 = 30M_{\odot}$,
    $a_{\rm in} = 0.1\;\mathrm{AU}$, $\tilde{a}_3 = 3\;\mathrm{AU}$, and $e_3
    = 0$. Conservation of $\bar{\theta}_{\rm e}$ gives the green line. Agreement near
    $I_0 = 90^\circ$ is good when accounting for the effects of a finite
    $L_{\rm out}$ (see Fig.~\ref{fig:bin_comp_zoom}). For $I_0$ in the two
    intervals $[40^\circ, 80^\circ]$ and $[100^\circ, 140^\circ]$, a further
    effect causes $\theta_{\rm sl}^{\rm f}$ to fluctuate
    unpredictably.}\label{fig:bin_comp}
\end{figure}

\begin{figure}
    \centering
    \includegraphics[width=\columnwidth]{LK90_plots/6bin_comp_zoom.png}
    \caption{Zoomed in version of Fig.~\ref{fig:bin_comp} near $I_0 \approx
    90^\circ$ while adding an $I_0$ offset to account for differences between
    the data and theory due to finite $L_{\rm out}$ effects. The green shaded
    area shows the expected range of deviations due to resonant perturbations
    following Eq.~\eqref{eq:harmonic_dqeff} (evaluated for the initial system
    parameters). The data nearer $90^\circ$ have less spread than predicted, but
    the transition to a larger $\theta_{\rm sl}^{\rm f}$ spread roughly follows
    the prediction of the green line.}\label{fig:bin_comp_zoom}
\end{figure}

However, as can be seen in Fig.~\ref{fig:bin_comp}, intermediate
inclinations $I_0 \in [50, 80]$ and $I_0 \in [100, 130]$ exhibit very volatile
behavior in $\theta_{\rm sl}^{\rm f}$. This is unlike the plots generated in the
fiducial parameter regime (Fig.~\ref{fig:qslscan}), as this inclination regime
corresponds to neither the fastest nor slowest merging systems. We attribute the
origin of this volatility to a stronger resonant interaction. By examining
Fig.~\ref{fig:dWs_inner}, it is evident that, for the same $e_{\min}$, systems
with $I_{\min}$ further from $90^\circ$ are closer to the
$\overline{\Omega}_{\rm e} = \Omega_{\rm LK}$ resonance. Outside of the LK
window, $\overline{\Omega}_{\rm e}$ also goes swiftly to zero, as seen in
Fig.~\ref{fig:dWs_inner}, so this explanation is consistent with the
ranges of inclinations that exhibit volatile $\theta_{\rm sl}^{\rm f}$ behavior.
We forgo further investigation of this mechanism because it is not expected to
play an important role in any LK-induced BH binary mergers for reasons discussed
below.

\begin{figure}
    \centering
    \includegraphics[width=\columnwidth]{LK90_plots/5_dWs_inner.png}
    \caption{Same as Fig.~\ref{fig:dWs} but for the compact parameter regime.
    Cmopared to Fig.~\ref{fig:dWs}, we see that $e_{\max}$ is smaller due to
    stronger pericenter precession $\Omega_{\rm GR}$ in this regime, so the
    $\overline{\Omega}_{\rm e} / \Omega_{\rm LK} = 1$ resonance is accessible
    for a wide range of parameter space. In particular, both smaller $e_{\min}$
    and $e_{\max}$ values bring the system closer to the
    resonance.}\label{fig:dWs_inner}
\end{figure}

At first, it seems clear from Fig.~\ref{fig:dWs} that $\overline{\Omega}_{\rm e}$
is significantly smaller than $\Omega_{\rm LK}$ near $I_{\min} \approx 90^\circ$
for LK-induced mergers. However, this is not sufficient to guarantee that the
frequency ratio remains small for the entire evolution, as GR effects become
stronger as the binary coalesces. Instead, a more careful analysis of the
relevant quantities in Eq.~\eqref{eq:harmonic_dqeff} proves useful:
\begin{itemize}
    \item $\sin \p{\Delta I_{\rm e1}}$ is small unless $\mathcal{A} \simeq
        1$. Otherwise, $\bm{\Omega}_{\rm e}$ does not nutate appreciably within
        an LK cycle, and all the $\bm{\Omega}_{\rm eN}$ are aligned with
        $\bm{\overline{\Omega}}_{\rm e}$, implying all the $\Delta I_{\rm eN}
        \approx 0$.

    \item Smaller values of $e_{\min}$ increase $\overline{\Omega}_{\rm e} /
        \Omega_{\rm LK}$, as shown in Fig.~\ref{fig:dWs_inner}.
\end{itemize}
However, LK-driven coalescence causes $\mathcal{A}$ to increase on a similar
timescale to that of $e_{\min}$ increase (see Fig.~\ref{fig:4sim_90_350}). This
implies that, if $\mathcal{A} \ll 1$ initially, which is the case for LK-induced
mergers, then $e_{\min}$ will be very close to $1$ when $\mathcal{A}$ grows to
be $\simeq 1$, and the contribution predicted by Eq.~\eqref{eq:harmonic_dqeff}
must be small.

\section{Conclusion and Discussion}\label{s:discussion}

In this paper, we consider the evolution of the spin-orbit misalignment angle
$\theta_{\rm sl}$ of a black hole (BH) binary that merges under gravitational
wave (GW) radiation during Lidov-Kozai (LK) oscillations induced by a tertiary
supermassive black hole (SMBH). We show that, when the gravitational potential
of the SMBH is handled at quadrupolar order, the spin vectors of the inner BHs
obey the simple equation of motion Eq.~\eqref{eq:dsdt_weff}. Analysis of this
equation yields the following conclusions:
\begin{itemize}
    \item Since Eq.~\eqref{eq:dsdt_weff} is a linear system with periodically
        varying coefficients, it cannot give rise to chaotic dynamics by
        Floquet's Theorem.

    \item For most parameters of astrophysical relevance, the angle
        $\bar{\theta}_{\rm e}$ [Eq.~\eqref{eq:q_eff}] is an adiabatic invariant.
        Since the inner BH binary merges in finite time, $\bar{\theta}_{\rm e}$
        is only conserved to finite accuracy; we show that the deviation from
        perfect adiabaticity can be predicted from initial conditions.

    \item When the resonant condition $\bar{\Omega}_{\rm e} \approx \Omega_{\rm
        LK}$ is satisfied, significant oscillations in $\bar{\theta}_{\rm e}$
        can arise. We derive an analytic estimate of this oscillation amplitude.
        This estimate both demonstrates that the resonance is unimportant for
        ``LK-induced'' mergers and tentatively explains the scatter in
        $\theta_{\rm sl}^{\rm f}$ seen by \citet{bin1}.
\end{itemize}

\bibliography{Su_LK90}
\bibliographystyle{aasjournal}

\end{document}
