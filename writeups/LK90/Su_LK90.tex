% rm -f LK90_plots/*.png; for i in M1_M2_thetasl_thetae_70 5_dWs_inner 5_dWs 6bin_comp 6bin_comp_zoom 4qslscan/qslscan 4sims_scan/deviations_one 4sims/4sim_90_200_zoom 4sims/4sim_90_350_supp 4sims/4sim_90_350_zoom 4sims/4sim_90_350 7_3vec_cropped 4sims/good_quants; do cp ../../scripts/lk90/$i.png LK90_plots; done
% check signs of all omega_e, I_e in zoom plots/equations
% rewrite Dong note
    \documentclass[
        twocolumn,
        twocolappendix
    ]{aastex63}
    \usepackage{
        amsmath,
        amssymb,
        newtxtext,
        newtxmath,
        graphicx,
        ae,
        aecompl,
        booktabs,
        wasysym
    }
    \usepackage[T1]{fontenc}

    \newcommand*{\rd}[2]{\frac{\mathrm{d}#1}{\mathrm{d}#2}}
    \newcommand*{\rtd}[2]{\frac{\mathrm{d}^2#1}{\mathrm{d}#2^2}}
    \newcommand*{\pd}[2]{\frac{\partial#1}{\partial#2}}
    \newcommand*{\md}[2]{\frac{\mathrm{D}#1}{\mathrm{D}#2}}
    \newcommand*{\rdil}[2]{\mathrm{d}#1 / \mathrm{d}#2}
    \newcommand*{\pdil}[2]{\partial#1 / \partial#2}
    \newcommand*{\at}[1]{\left.#1\right|}
    \newcommand*{\abs}[1]{\left|#1\right|}
    \newcommand*{\ev}[1]{\langle#1\rangle}
    \renewcommand*{\bm}[1]{\boldsymbol{\mathbf{#1}}}
    \newcommand*{\uv}[1]{\hat{\bm{#1}}}
    \newcommand*{\p}[1]{\left(#1\right)}
    \newcommand*{\s}[1]{\left[#1\right]}
    \newcommand*{\z}[1]{\left\{#1\right\}}
    \DeclareMathOperator*{\argmin}{argmin}
    \DeclareMathOperator*{\argmax}{argmax}
    \DeclareMathOperator*{\med}{med}
    \DeclareMathOperator*{\sgn}{sgn}
    \let\Re\undefined
    \let\Im\undefined
    \DeclareMathOperator{\Re}{Re}
    \DeclareMathOperator{\Im}{Im}
    \colorlet{Corr}{red}

    \received{XXXX}
    \revised{XXXX}
    \accepted{XXXX}
    \submitjournal{ApJ}

\shorttitle{BH Triple Spin-Orbit Dynamics}
\shortauthors{Y.\ Su et.\ al.}

\begin{document}

\title{Spin Dynamics in Hierarchical Black Hole Triples: Predicting Final
Spin-Orbit Misalignment Angle From Initial Conditions}

\correspondingauthor{Yubo Su}
\email{yubosu@astro.cornell.edu}

\author[0000-0001-8283-3425]{Yubo Su}% chktex 8
\affiliation{Cornell Center for Astrophysics and Planetary Science, Department
of Astronomy, Cornell University, Ithaca, NY 14853, USA}

\author[0000-0002-1934-6250]{Dong Lai}% chktex 8
\affiliation{Cornell Center for Astrophysics and Planetary Science, Department
of Astronomy, Cornell University, Ithaca, NY 14853, USA}

\author[0000-0002-0643-8295]{Bin Liu}% chktex 8
\affiliation{Cornell Center for Astrophysics and Planetary Science, Department
of Astronomy, Cornell University, Ithaca, NY 14853, USA}

\begin{abstract}
    Abstract
\end{abstract}

\keywords{keywords}

\section{Introduction}\label{s:intro}

As the LIGO/VIRGO collaboration continues to detect mergers of black hole (BH)
binaries \citep[e.g.][]{Abbott:2016blz, abbott2019binary}, it is increasingly
important to systematically study various formation channels of BH binaries and
their observable signatures. The canonical channel consists of isolated binary
evolution, in which mass transfer and friction in the common envelope phase
cause the binary orbit to shrink sufficiently that it subsequently merges via
emission of gravitational waves (GW) within a Hubble time
\citep[e.g.][]{lipunov1997black, lipunov2017first, podsiadlowski2003formation,
belczynski2010effect, belczynski2016first, dominik2012double, dominik2013double,
dominik2015double}. BH binaries formed via isolated binary evolution are
generally expected to have small misalignment between the BH spin axis and the
orbital angular momentum axis \citep{postnov2019black,
belczynski2020evolutionary}. On the other hand, various flavors of dynamical
formation channels of BH binaries have also been studied. These involve either
strong gravitational scatterings in dense clusters
\citep[e.g.][]{zwart1999black, o2006binary, miller2009mergers,
banerjee2010stellar, downing2010compact, ziosi2014dynamics, rodriguez2015binary,
samsing2017assembly, samsing2018black, rodriguez2018post, gondan2018eccentric}
or more gentle ``tertiary-induced mergers'' \citep[e.g.][]{blaes2002kozai,
miller2002four, wen2003eccentricity, antonini2012secular, antonini2017binary,
silsbee2016lidov, bin1, bin2, randall2018induced, hoang2018black}. The dynamical
formation channels generally produce BH binaries with misaligned spins.

GW observations of binary inspirals can put constraints on BH masses and spins.
Typically, spin constraints come in the form of two dimensionless mass-weighted
combinations of the component BH spins: (i) the aligned spin parameter
\begin{equation}
    \chi_{\rm eff} \equiv \frac{m_1 \chi_1 \cos \theta_{\rm s_1, l}
            + m_2 \chi_2 \cos \theta_{\rm s_2, l}}{m_1 + m_2}
\end{equation}
where $m_{1,2}$ are the masses of the BHs, $\theta_{\rm s_i, l}$ is the angle
between the $i$th spin and the binary orbital angular momentum axis, and
$\chi_i \equiv cS_i / (Gm_i^2)$ is the dimensionless Kerr spin parameter; and
(ii) the perpendicular spin parameter \citep{schmidt2015towards}
\begin{equation}
    \chi_{\rm p} \equiv \max\z{
        \chi_1 \sin \theta_{\rm s_1,l}, \frac{q\p{4q + 3}}{4 + 3q} \chi_2 \sin
        \theta_{\rm s_2,l}},
\end{equation}
where $q \equiv m_2 / m_1$. The systems detected in the O1 and O2 observing runs
have small $\chi_{\rm eff}$, which is consistent with either small $\chi_1$ and
$\chi_2$ or highly misaligned BH spins ($\theta_{\rm s_i, l} \approx 90^\circ$).

\textcolor{Corr}{TODO Update this paragraph later with references.} In the
recent GW detection GW190521.1, the component BHs have dimensionless spins
$\chi_{1, 2} = S_{1,2} / M_{1, 2}^2 \sim 0.7$, yet $\chi_{\rm eff} \approx 0$
while $\chi_{\rm p} \sim 1$. This requires both BH spins be misaligned with the
orbital angular momentum, i.e.\ $\theta_{\rm s_i, l} \approx 90^\circ$.

\citet[][hereafter LL17, LL18]{bin1, bin2}, and \citet{bin3} carried out a
systematic study of binary BH mergers in the presence of a tertiary companion.
LL17 pointed out the important effect of spin-orbit coupling (de-Sitter
precession) in determining the final spin-orbit misalignment angles of BH
binaries in triple systems. They considered binaries with sufficiently compact
orbits (so that mergers are possible even without a tertiary) and showed that
the combination of LK oscillations (induced by a modestly inclined tertiary) and
spin-orbit coupling gives rise to a broad range of final spin-orbit misalignment
in the merging binary BHs. We call these mergers \emph{LK-enhanced mergers}.
LL18 considered the most interesting case of \emph{ LK-induced mergers}, in
which an initially wide BH binary (too wide to merge in isolation) is pushed
to extreme eccentricities (close to unity) by a highly inclined tertiary and
merges within a few Gyrs. LL18 examined a wide range of orbital and spin
evolution behaviors and found that LK-induced mergers can sometimes yield a
``$90^\circ$ attractor'': when the BH spin is initially aligned with the inner
binary angular momentum axis ($\theta_{\rm sl}^{\rm i} = 0$), it evolves towards
a perpendicular state ($\theta_{\rm sl}^{\rm f} = 90^\circ$) near merger.
Qualitatively, they found that the attractor exists when the LK-induced orbital
decay is sufficiently ``gentle'' and the octupole effect is unimportant.
Fig.~\ref{fig:4sim_90_350} gives an example of a system evolving towards this
attractor, where $\theta_{\rm sl}$ converges to $\approx 90^\circ$ at
late times in the bottom right panel. Fig.~\ref{fig:qslscan} shows how
$\theta_{\rm sl}^{\rm f}$ varies when the initial inclination of the tertiary
orbit $I_0$ (relative to the inner orbit) is varied. Note that for rapid mergers
(when $I_0$ is close to $90^\circ$), the attractor does not exist; as $I_0$
deviates more from $90^\circ$, the merger time increases and $\theta_{\rm
sl}^{\rm f}$ is close to $90^\circ$.  This $90^\circ$ attractor gives rise to a
peak around $\chi_{\rm eff} = 0$ in the final $\chi_{\rm eff}$ distribution in
tertiary-induced mergers (LL18; \citealp{bin3}). This peak was also found in the
population studies of \citet{antonini2018precessional}.

The physical origin of this $90^\circ$ attractor and under what conditions it
can be achieved are not well understood. LL18 proposed an explanation based on
analogy with an adiabatic invariant in systems where the inner binary remains
circular throughout the inspiral (LL17). However, this analogy is not justified,
as significant eccentricity excitation is a necessary ingredient in LK-induced
mergers. In addition, the LK-enhanced mergers considered in LL17 show no
$90^\circ$ attractor even though the orbital evolution is slow and regular.

In this paper, we study an analytic theory that reproduces the $90^\circ$
attractor and characterizes its regime of validity. In
Sections~\ref{s:setup_orbital} and~\ref{s:setup_spin}, we set up the relevant
equations of motion for the orbital and spin evolution of the system. In
Sections~\ref{s:fast_merger} and~\ref{s:harmonic}, we develop an analytic theory
and compute its regime of validity for LK-induced mergers. In
Section~\ref{s:lk_enhanced}, we comment on the LK-enhanced scenario. We discuss
and conclude in Section~\ref{s:discussion}.

\begin{figure}
    \centering
    \includegraphics[width=\columnwidth]{LK90_plots/4sim_90_350.png}
    \caption{An example of the $90^\circ$ spin attractor in LK-induced BH binary
    mergers. The four panes show the time evolution of the binary semi-major
    axis $a$, eccentricity $e$, inclination $I$ [the red line denotes the
    averaged $\bar{I}$ given by Eq.~\eqref{eq:barI}], and spin-orbit
    misalignment angle $\theta_{\rm sl}$. The unit of time $t_{\rm LK, 0}$ is
    the LK timescale [Eq.~\eqref{eq:t_lk}] evaluated for the initial conditions.
    The inner binary has $m_1 = 30M_{\odot}$, $m_2 = 20M_{\odot}$, and initial
    $a_0 = 100\;\mathrm{AU}$, $e_0 = 0.001$, $I_0 = 90.35^\circ$ (with respect
    to the outer binary), and $\theta_{\rm sl}^{\rm i} = 0$. The tertiary SMBH
    has $a_{\rm out} = 2.2\;\mathrm{pc}$, $e_{\rm out} = 0$, and $m_3 = 3 \times
    10^7 M_{\odot}$. It can be seen that $\theta_{\rm sl}$ evolves to $\sim
    90^\circ$ as $a$ shrinks to smaller values.}\label{fig:4sim_90_350}
\end{figure}
\begin{figure}
    \centering
    \includegraphics[width=\columnwidth]{LK90_plots/qslscan.png}
    \caption{The merger time and the final spin-orbit misalignment angle
    $\theta_{\rm sl}^{\rm f}$ as a function of the initial inclination $I_0$ for
    LK-induced mergers. The other parameters are the same as those in
    Fig.~\ref{fig:4sim_90_350}. For $I_0$ somewhat far away from $90^\circ$, the
    resulting $\theta_{\rm sl}^{\rm f}$ are all quite near $90^\circ$. The black
    dashed line in the lower panel shows Eq.~\eqref{eq:qslf_plot_black}, which
    provides an estimate for the deviation from the $90^\circ$
    attractor.}\label{fig:qslscan}
\end{figure}

\section{LK-Induced Mergers: Orbital Evolution}\label{s:setup_orbital}

In this section we summarize the key features and relevant equations for
LK-induced mergers to be used for our analysis in later sections. Consider a
black hole (BH) binary with masses $m_1$ and $m_2$ having total mass $m_{12}$,
reduced mass $\mu = m_1 m_2 / m_{12}$, semimajor axis $a$ and eccentricity $e$.
This inner binary orbits around a tertiary with mass $m_3$, semimajor axis
$a_{\rm out}$ and eccentricity $e_{\rm out}$ in a hierachical configuration
($a_{\rm out}\gg a$). Unless explicitly stated, we assume $m_3 \gg m_1, m_2$ (so
the tertiary is a supermassive black hole, or SMBH), although our analysis can
be easily generalized to comparable masses. It is also convenient to define the
effective outer semimajor axis
\begin{equation}
    \tilde{a}_{\rm out} \equiv a_{\rm out}\sqrt{1 - e_{\rm out}^2}.
\end{equation}
We denote the orbital angular momentum of the inner binary by $\bm{L} \equiv L
\uv{L}$ and the angular momentum of the outer binary by $\bm{L}_{\rm out} \equiv
L_{\rm out} \uv{L}_{\rm out}$. Since $L_{\rm out} \gg L$, we take $\bm{L}_{\rm
out}$ to be fixed.

The equations of motion governing the orbital elements $a$, $e$, $\ascnode$,
$I$, $\omega$ (where $\ascnode$, $I$, $\omega$ are the longitude of the
ascending node, inclination, and argument of periapsis respectively) of the
inner binary are
\begin{align}
    \rd{a}{t} &= \p{\rd{a}{t}}_{\rm GW}\label{eq:dadt},\\
    \rd{e}{t} &= \frac{15}{8t_{\rm LK}} e\,j(e)\sin 2\omega
        \sin^2 I + \p{\rd{e}{t}}_{\rm GW}\label{eq:dedt},\\
    \rd{\ascnode}{t} &= \frac{3}{4t_{\rm LK}}
        \frac{\cos I\p{5e^2 \cos^2\omega - 4e^2 - 1}}{j(e)}
            \label{eq:dWdt},\\
    \rd{I}{t} &= \frac{15}{16}\frac{e^2\sin 2\omega \sin
        2I}{j(e)},\label{eq:dIdt}\\
    \rd{\omega}{t} &= \frac{3}{4t_{\rm LK}}
        \frac{2j^2(e) + 5\sin^2\omega (e^2 - \sin^2 I)}{j(e)}
        + \Omega_{\rm GR},\label{eq:dwdt}
\end{align}
where we have defined
\begin{align}
    j(e) &= \sqrt{1 - e^2},\\
    t_{\rm LK}^{-1} &\equiv n\p{\frac{m_3}{m_{12}}}
        \p{\frac{a}{\tilde{a}_{\rm out}}}^3,\label{eq:t_lk}
\end{align}
with $n \equiv \sqrt{G m_{12} / a^3}$ the mean motion of the inner binary. The
GR-induced apsidal precession of the inner binary is given by
\begin{equation}
    \Omega_{\rm GR}(e) = \frac{3Gnm_{12}}{c^2aj^2(e)}.
\end{equation}
The dissipative terms due to gravitational radiation are
\begin{align}
    \p{\rd{a}{t}}_{\rm GW} &= -\frac{a}{t_{\rm GW}(e)},\label{eq:dadt_gw}\\
    \p{\rd{e}{t}}_{\rm GW} &= -\frac{304}{15}\frac{G^3 \mu m_{12}^2}{c^5a^4}
        \frac{1}{j^{5/2}(e)}\p{1 + \frac{121}{304}e^2}\label{eq:dedt_gw},
\end{align}
where
\begin{equation}
    t_{\rm GW}^{-1}(e) \equiv \frac{64}{5}\frac{G^3 \mu m_{12}^2}{c^5a^4}
            \frac{1}{j^{7/2}(e)}\p{1 + \frac{73}{24}e^2
                + \frac{37}{96}e^4}.\label{eq:t_gw}
\end{equation}

Fig.~\ref{fig:4sim_90_350} depicts an example of LK-induced mergers as
caclulated using the above equations. We adopt the following fiducial
parameters: the inner binary has $m_1 = 30M_{\odot}$, $m_2 = 20M_{\odot}$,
initial $a = 100\;\mathrm{AU}$, $e_0 = 0.001$, and $I_0 = 90.35^\circ$. We take
the SMBH tertiary companion to have $m_3 = 3 \times 10^{7} M_{\odot}$ and
$\tilde{a}_{\rm out} = 4.5 \times 10^5\;\mathrm{AU} = 2.2\;\mathrm{pc}$. Since
$\tilde{a}_{\rm out} \gg a$, the octupole effects are negligible and have been
omitted in Eqs.~(\ref{eq:dedt}--\ref{eq:dwdt}). Note that these parameters give
the same $t_{\rm LK}$ as Fig.~4 of LL18. We refer to this as the fiducial
parameter regime, and our analysis in later sections will be based on this
example unless otherwise noted.

We next discuss the key analytical properties of the orbital evolution.

\subsection{Analytical Results Without GW Radiation}

First, neglecting the GW radiation terms, the system admits two
conservation laws, the ``Kozai constant'' and energy conservation,
\begin{align}
    &j(e) \cos I = \mathrm{const},\label{eq:const1}\\
    &\frac{3}{8}\s{2e^2 + j^2(e) \cos^2 I - 5e^2 \sin^2 I \sin^2 \omega}
        + \frac{\epsilon_{\rm GR}}{j(e)} =
            \mathrm{const},\label{eq:const2}
\end{align}
(see \citet{anderson2016formation}, LL18 for more general expressions
when $L_{\rm out}$ is comparable to $L$), where
\begin{equation}
    \epsilon_{\rm GR} \equiv \p{\Omega_{\rm GR} t_{\rm LK}}_{e = 0}
        = \frac{3Gm_{12}^2 \tilde{a}_{\rm out}^3}{c^2m_3a^4}.
\end{equation}
The conservation laws can be combined to obtain the maximum eccentricity
$e_{\max}$ as a function of the initial $I_0$ (and initial $e_0 \ll 1$).
The largest value of $e_{\max}$ occurs at $I_0 = 90^\circ$ and is given
by
\begin{equation}
    j(e_{\max})_{I_0 = 90^\circ} = (8/9) \epsilon_{\rm GR}.
\end{equation}
Eccentricity excitation then requires $\epsilon_{\rm GR} < 9/8$. Our fiducial
examples in Figs.~\ref{fig:4sim_90_350} and~\ref{fig:qslscan} satisfy
$\epsilon_{\rm GR} \ll 1$ at $a = a_0$, leading to $e_{\max} \sim 1$ within a
narrow inclination window around $I_0 = 90^\circ$.

Eqs.~\eqref{eq:const1} and~\eqref{eq:const2} imply that $e$ is a fuction
of $\sin^2\omega$ alone \citep[see][for exact forms]{kinoshita, storch},
so an eccentricity maximum occurs every half period of $\omega$. We
define the period and angular frequency of eccentricity oscillation via
\begin{align}
    \pi &= \int\limits_0^{P_{\rm LK}} \rd{\omega}{t}\;\mathrm{d}t,&
    \Omega_{\rm LK} &\equiv \frac{2\pi}{P_{\rm LK}}.\label{eq:PLK_def}
\end{align}

In LK cycles, the inner binary oscillates between the eccentricity minimum
$e_{\min}$ and maximum $e_{\max}$.  The oscillation is ``uneven'': when
$e_{\min} \ll e_{\max}$, the binary spends a fraction $\sim j(e_{\max})$ of the
LK cycle, or time $\Delta t \sim t_{\rm LK} j\p{e_{\max}}$, near $e \simeq
e_{\max}$ (see Eq.~\eqref{eq:dwdt}).

\subsection{Behavior with GW Radiation}

Including the effect of GW radiation, orbital decay predominantly occurs at $e
\simeq e_{\max}$ with the timescale of $t_{\rm GW}\p{e_{\max}}$ [see
Eq.~\eqref{eq:t_gw}]. On the other hand, Eq.~\eqref{eq:dwdt} implies that, when
$\epsilon_{\rm GR} \ll 1$, the binary spends only a small fraction ($\sim
j(e_{\max})$) of the time near $e \simeq e_{\max}$. Thus, we expect two
qualitatively different merger behaviors:
\begin{itemize}
    \item ``Rapid mergers'': When $t_{\rm GW}\p{e_{\max}} \lesssim t_{\rm
        LK}j(e_{\max})$, the binary is ``pushed'' into high eccentricity and
        exhibits a ``one shot merger'' without any $e$-oscillations.

    \item ``Smooth mergers'': When $t_{\rm GW}\p{e_{\max}} \gtrsim t_{\rm
        LK}j(e_{\max})$, the binary goes through a phase of eccentricity
        oscillations while the orbit gradually decays. In this case, the
        LK-averaged orbital decay rate is $\sim j(e_{\max})t_{\rm
        GW}^{-1}(e_{\max})$. As $a$ decreases, $e_{\max}$ decreases slightly
        while the minimum eccentricity increases, approaching $e_{\max}$ (see
        Fig.~\ref{fig:4sim_90_350}). This eccentricity oscillation ``freeze''
        ($e_{\min} \sim e_{\max}$) is due to GR-induced apsidal precession
        ($\epsilon_{\rm GR}$ increases as $a$ decreases), and occurs when
        $\epsilon_{\rm GR}(a) \gg j(e_{\max})$. After the eccentricity is
        frozen, the binary circularizes and decays on the timescale $t_{\rm
        GW}\p{e_{\max}}$.
\end{itemize}

\section{Spin Dynamics: Equations}\label{s:setup_spin}

We are interested in the spin orientations of the inner BHs at merger
as a function of initial conditions. Since they evolve independently to leading
post-Newtonian order, we focus on the dynamics of $\uv{S}_1 = \uv{S}$, the spin
vector of $m_1$. Since the spin magnitude does not enter into the dynamics, we
write $\bm{S} \equiv \uv{S}$ for brevity (i.e.\ $\bm{S}$ is a unit vector).
Neglecting spin-spin interactions, $\bm{S}$ undergoes de Sitter precession about
$\bm{L}$ as
\begin{align}
    \rd{\bm{S}}{t} &= \Omega_{\rm SL}\hat{\bm{L}} \times \bm{S},
            \label{eq:dsdt}
\end{align}
with
\begin{align}
    \Omega_{\rm SL} &= \frac{3Gn\p{m_2 + \mu/3}}{2c^2aj^2(e)}.
\end{align}

In the presence of a tertiary companion, the orbital axis $\uv{L}$ of the inner
binary precesses around $\uv{L}_{\rm out}$ with rate $\rdil{\ascnode}{t}$ and
nutates with varying $I$ [see Eqs.~\eqref{eq:dWdt} and~\eqref{eq:dIdt}]. To
analyze the dynamics of the spin vector, we go to the co-rotating frame with
$\uv{L}$ about $\uv{L}_{\rm out}$, in which Eq.~\eqref{eq:dsdt} becomes
\begin{align}
    \p{\rd{\bm{S}}{t}}_{\rm rot}
        &= \bm{\Omega}_{\rm e} \times \bm{S}\label{eq:dsdt_weff},
\end{align}
where we have defined an effective rotation vector
\begin{align}
    \bm{\Omega}_{\rm e} &\equiv \Omega_{\rm L}\uv{L}_{\rm out} + \Omega_{\rm SL}
            \uv{L},\label{eq:weff_def}
\end{align}
with [see Eq.~\eqref{eq:dWdt}]
\begin{align}
    \Omega_{\rm L} &\equiv -\rd{\ascnode}{t}.\label{eq:Wldef}
\end{align}
In this rotating frame, the plane spanned by $\uv{L}_{\rm out}$ and $\uv{L}$ is constant
in time, only the inclination angle $I$ can vary.

\subsection{Nondissipative Dynamics}

We first consider the limit where dissipation via GW radiation is completely
neglected ($t_{\rm GW}(e) \to \infty$). Then $\bm{\Omega}_{\rm e}$ is exactly
periodic with period $P_{\rm LK}$ [see Eq.~\eqref{eq:PLK_def}] We can rewrite
Eq.~\eqref{eq:dsdt_weff} in Fourier components
\begin{equation}
    \p{\rd{\bm{S}}{t}}_{\rm rot}
        = \s{\bm{\overline{\Omega}}_{\rm e} + \sum\limits_{N = 1}^\infty
            \bm{\Omega}_{\rm eN}\cos \p{N\Omega_{\rm LK}t}}
            \times \bm{S}.\label{eq:dsdt_fullft}
\end{equation}
We write $\bm{\overline{\Omega}}_{\rm e} \equiv \bm{\Omega}_{\rm e0}$ for
convenience, where the bar denotes an average over a LK cycle. We have adopted
the convention where $t = 0$ is the time of maximum eccentricity of the LK
cycle, so that Eq.~\eqref{eq:dsdt_fullft} does not have $\sin\p{N\Omega_{\rm
LK}t}$ terms.

This system superficially resembles that considered in \citet{storch} (SL15),
who studied the dynamics of the spin axis of a star when driven by a giant
planet undergoing LK oscilations \citep[see also][]{storch2014chaotic,
storch2017dynamics}. In their system, the spin-orbit coupling arises from
Newtonian interaction between the planet ($M_{\rm p}$) and the rotation-induced
stellar quadrupole ($I_{\rm out}-I_1$), and the spin precession frequency is
\begin{equation}
    \Omega_{\rm SL}^{\rm (Newtonian)} = -\frac{3GM_{\rm p} \p{I_{\rm out} -
        I_1}}{ 2a^3j^3(e)}\frac{\cos \theta_{\rm sl}}{I_3 \Omega_{\rm s}},
\end{equation}
where $I_3 \Omega_{\rm s}$ is the spin angular momentum of the star. SL15 showed
that under some conditions that depend on a dimensionless adiabaticity parameter
(roughly the ratio between the magnitudes of $\Omega_{\rm SL}^{\rm (Newtonian)}$
and $\Omega_{\rm L}$ when factoring out the eccentricity and obliquty
dependence), the stellar spin axis can vary chaotically. One strong indicator of
chaos in their study is the presence of irregular, fine structure in a
bifurcation diagram [Fig.~1 of \citet{storch}] that shows the values of the
spin-orbit misalignment angle $\theta_{\rm sl}$ when varying system parameters
in the ``transadiabatic'' regime, where the adiabaticity parameter crosses
unity.

To generate an analogous bifurcation diagram for our problem, we consider a
sample system with $m_{12} = 60M_{\odot}$, $m_3 = 3 \times 10^7 M_{\odot}$, $a =
0.1\;\mathrm{AU}$, $e_0 = 10^{-3}$, $I_0 = 70^\circ$, $a_{\rm out} =
300\;\mathrm{AU}$, $e_{\rm out} = 0$, and initial $\theta_{\rm sl} = 0$. We then
evolve Eq.~\eqref{eq:dsdt_weff} together with the orbital evolution equations
[Eqs.~(\ref{eq:dadt}--\ref{eq:dwdt}) without the GW terms] while sampling both
$\theta_{\rm sl}$ and $\theta_{\rm e}$ at eccentricity maxima, where
$\theta_{\rm e}$ is given by
\begin{equation}
    \cos \theta_{\rm e} = \frac{\bm{\overline{\Omega}}_{\rm e}}{
        \overline{\Omega}_{\rm e}}\cdot \bm{S}.\label{eq:q_eff_inst}
\end{equation}
We repeat this procedure with different mass ratios $m_1 / m_{12}$ of the inner
binary, which only changes $\Omega_{\rm SL}$ without changing the orbital
evolution (note that the LK oscillation depends only on $m_{12}$ and not on
individual masses of the inner binary). Analogous to SL15, we consider systems
with a range of the adibaticity parameter $\mathcal{A}$ [to be defined later in
Eq.~\eqref{eq:abar_def}] that crosses order unity. The fiducial system of
Fig.~\ref{fig:4sim_90_350} does not serve this purpose because the initial
$\Omega_{\rm SL}$ is too small. Our result is depicted in
Fig.~\ref{fig:bifurcation_70}.
\begin{figure}
    \centering
    \includegraphics[width=\columnwidth]{LK90_plots/M1_M2_thetasl_thetae_70.png}
    \caption{Bifurcation diagram for the BH spin orientation during LK
    oscillations. The physical parameters are $m_{12} = 60M_{\odot}$, $m_3 = 3
    \times 10^7 M_{\odot}$, $a = 0.1\;\mathrm{AU}$, $e_0 = 10^{-3}$, $I_0 =
    70^\circ$, $a_{\rm out} = 300\;\mathrm{AU}$, $e_{\rm out} = 0$, and initial
    condition $\theta_{\rm sl}^{\rm i} = 0$. For each mass ratio $m_1 / m_{12}$,
    the orbit-spin system is solved over $500$ LK cycles, and both $\theta_{\rm
    sl}$ (the angle between $\bm{S}$ and $\uv{L}$) and $\theta_{\rm e}$ [defined
    by Eq.~\eqref{eq:q_eff_inst}] are sampled at every eccentricity maximum and
    are plotted. The top axis shows the adiabaticity parameter $\mathcal{A}$ as
    defined by Eq.~\eqref{eq:abar_def}. Note that for a given $m_{12}$, changing
    the mass ratio $m_1 / m_{12}$ only changes the spin evolution and not the
    orbital evolution.}\label{fig:bifurcation_70}
\end{figure}

While our bifurcation diagram has interesting structure, the features are all
regular. This is in contrast to the star-planet system studied by SL15 (see
their Fig.~1). A key difference is that in our system, $\Omega_{\rm SL}$ does
not depend on $\theta_{\rm sl}$, while for the planet-star system, $\Omega_{\rm
SL}^{\rm (Newtonian)}$ does, and this latter feature introduces nonlinearity to
the dynamics.

A more formal understanding of the dynamical behavior of our spin-orbit system
comes from Floquet theory\citep{floquet1883equations, kuchment2012floquet}, as
Eq.~\eqref{eq:dsdt_weff} is a linear system with periodic coefficients (the
system studied in SL15 is nonlinear). Floquet's theorem says that when a linear
system with periodic coefficients is integrated over a period, the evolution can
be described by the linear transformation
\begin{equation}
    \bm{S}\p{t + P_{\rm LK}} = \bm{\tilde{M}} \bm{S}(t),
\end{equation}
where $\bm{\tilde{M}}$ is called the \emph{monodromy matrix} and is independent
of $\bm{S}$.

For our system, while $\bm{\tilde{M}}$ can be easily defined, it cannot be evaluated in closed
form. Thankfully, it suffices to reason directly about the general properties of
$\bm{\tilde{M}}$: it must be a proper orthogonal
matrix, or a rotation matrix, as it represents the effect of many infinitesimal
rotations, each about the instantaneous $\bm{\Omega}_{\rm e}$\footnote{More
formally, $\bm{\tilde{M}} = \bm{\tilde{\Phi}}(P_{\rm LK})$ where
$\bm{\tilde{\Phi}}(t)$ is the \emph{principal fundamental matrix solution}: the
columns of $\bm{\tilde{\Phi}}$ are solutions to Eq.~\eqref{eq:dsdt_weff} and
$\bm{\tilde{\Phi}}(0)$ is the identity. By linearity, the columns of
$\bm{\tilde{\Phi}}(t)$ remain orthonormal, while its determinant does not
change, so $\bm{\tilde{M}}$ is a proper orthogonal matrix, or a rotation
matrix.}. Therefore, over each period $P_{\rm LK}$, the dynamics of
$\bm{S}$ are equivalent to a rotation about a fixed axis,
prohibiting chaotic behavior.

Another traditional indicator of chaos is a positive Lyapunov exponent, obtained
when the separation between nearby trajectories diverges \emph{exponentially} in
time. In Floquet theory, the Lyapunov exponent is the logarithm of the largest
eigenvalue of the monodromy matrix. Since $\bm{\tilde{M}}$ is a rotation matrix
in our problem, the Lyapunov exponent must be $0$, indicating no chaos. We have
verified this numerically.

\subsection{Spin Dynamics With GW Dissipation}

When $t_{\rm GW}$ is finite, the coefficients $\bm{\Omega}_{\rm eN}$, including
$\bm{\overline{\Omega}}_{\rm e} = \bm{\Omega}_{\rm e0}$ [see
Eq.~\eqref{eq:dsdt_fullft}], are no longer constant, but change over time. For
``smooth'' mergers (satisfying $t_{\rm GW}\p{e_{\max}} \gg t_{\rm LK}
j(e_{\max})$; see Section~\ref{s:setup_orbital}), the binary goes through a
sequence of LK cycles, and the coefficents vary on the LK-averaged orbital decay
time $t_{\rm GW}\p{e_{\max}} / j\p{e_{\max}}$. As the LK oscillation freezes, we
have $\bm{\Omega}_{\rm e} \simeq \bm{\overline{\Omega}}_{\rm e}$ (and
$\bm{\Omega}_{\rm eN} \simeq 0$ for $N \geq 1$), which evolves on timesale
$t_{\rm GW}(e)$ as the orbit decays and circularizes.

Once $a$ is sufficiently small such that $\Omega_{\rm SL} \gg \Omega_{\rm L}$
(this also corresponds to $\epsilon_{\rm GR} \gg 1$ since $\Omega_{\rm SL} \sim
\Omega_{\rm GR}$, implying the LK cycles are suppressed), it can be seen from
Eqs.~(\ref{eq:dsdt_weff}--\ref{eq:weff_def}) that $\theta_{\rm e} = \theta_{\rm
sl}$ is constant, i.e.\ the spin-orbit misalignment angle is frozen (see bottom
right panel of Fig.~\ref{fig:4sim_90_350}). This is the  ``final'' spin-orbit
misalignment, although the binary may still be far from the final merger. For
the fiducial examples depicted in
Figs.~\ref{fig:4sim_90_350}--\ref{fig:qslscan}, we stop the simulation at $a =
0.5\;\mathrm{AU}$, as $\theta_{\rm sl}$ has converged to its final value.

\subsection{Spin Dynamics Equation in Component Form}

For later analysis, it is useful to write Eq.~\eqref{eq:dsdt_fullft} in
component form. To do so, we define inclination angle $\bar{I}_{\rm e}$ as the
angle between $\bm{\overline{\Omega}}_{\rm e}$ and $\bm{L}_{\rm out}$ as shown in
Fig.~\ref{fig:3vec}. To express $\bar{I}_{\rm e}$ algebraically, we define
LK-averaged quantities
\begin{figure}
    \centering
    \includegraphics[width=0.5\columnwidth]{LK90_plots/7_3vec_cropped.png}
    \caption{Definition of angles in the problem, shown in plane of the two
    angular momenta $\bm{L}_{\rm out}$ and $\bm{L}$. Here, $\overline{\bm{L}}$
    is the suitably averaged $\bm{L}$ with inclination $\bar{I}$ defined by
    Eq.~\eqref{eq:barI}, $\overline{\Omega}_{\rm e}$ is the LK-averaged
    $\Omega_{\rm e}$, and $\Omega_{\rm e1}$ is the first harmonic component (see
    Eqs.~\eqref{eq:weff_def} and~\eqref{eq:dsdt_fullft}). Note that for $I_0 >
    90^\circ$ (and $\bar{I} > 90^\circ$), we have $\bar{I}_{\rm e} \in (90^
    \circ, 180^\circ)$ since $\Omega_{\rm L} < 0$. The bottom right shows our
    choice of coordinate axes with $\uv{z} \propto \overline{\Omega}_{\rm e}$.
    }\label{fig:3vec}
\end{figure}
\begin{align}
    \overline{\Omega_{\rm SL} \sin I} &\equiv
            \overline{\Omega}_{\rm SL} \sin \bar{I},&
    \overline{\Omega_{\rm SL} \cos I} &\equiv
            \overline{\Omega}_{\rm SL} \cos \bar{I}.\label{eq:barI}
\end{align}
It then follows from Eq.~\eqref{eq:weff_def} that
\begin{equation}
    \tan \bar{I}_{\rm e} = \frac{\mathcal{A}\sin \bar{I}}{
        1 + \mathcal{A}\cos \bar{I}},\label{eq:ie_def}
\end{equation}
where $\mathcal{A}$ is the adiabaticity parameter, given by
\begin{equation}
    \mathcal{A} \equiv \frac{\overline{\Omega}_{\rm SL}}{
        \overline{\Omega}_{\rm L}}.\label{eq:abar_def}
\end{equation}
Note that in Eq.~\eqref{eq:ie_def}, $\bar{I}_{\rm e}$ is defined in the domain
$[0^\circ, 180^\circ]$, i.e.\ $\bar{I}_{\rm e} \in (0, 90)$ when $\tan
\bar{I}_{\rm e} > 0$ and $\bar{I}_{\rm e} \in (90,180)$ when $\tan \bar{I}_{\rm
e} < 0$.

We now choose a non-inertial coordinate system where $\uv{z} \propto
\bm{\overline{\Omega}}_{\rm e}$ and $\uv{x}$ lies in the plane of $\bm{L}_{\rm
out}$ and $\bm{L}$ (see Fig.~\ref{fig:3vec}). In this reference frame, the spin
orientation is specified by the polar angle $\bar{\theta}_{\rm e}$ as
defined above in Eq.~\eqref{eq:q_eff}, and the equation of motion becomes
\begin{align}
    \p{\rd{\bm{S}}{t}}_{\rm xyz} &= \s{\overline{\Omega}_{\rm e}\uv{z}
         + \sum\limits_{N = 1}^\infty
            \bm{\Omega}_{\rm eN}\cos \p{N\Omega_{\rm LK}t }}
        \times \bm{S}
        - \dot{\bar{I}}_{\rm e} \uv{y} \times \bm{S}\label{eq:eom_prime}.
\end{align}
One further simplification lets us cast this vector equation of motion into a
scalar form. Break $\bm{S}$ into components $\bm{S} = S_x\uv{x} + S_y \uv{y} +
\cos \bar{\theta}_{\rm e} \uv{z}$ and define complex variable
\begin{equation}
    S_\perp \equiv S_x + iS_y.
\end{equation}
Then, we can rewrite Eq.~\eqref{eq:eom_prime} as
\begin{align}
    \rd{S_{\perp}}{t} ={}& i\overline{\Omega}_{\rm e}S_\perp
            - \dot{\bar{I}}_{\rm e} \cos \bar{\theta}_{\rm e}
        + \sum\limits_{N = 1}^\infty\Big[
            \cos \p{\Delta I_{\rm eN}}S_\perp \nonumber\\
        &- i\cos \bar{\theta}_{\rm e} \sin \p{\Delta I_{\rm eN}}\Big]
            \Omega_{\rm eN}\cos \p{N\Omega_{\rm LK} t},
            \label{eq:formal_eom_allgen}
\end{align}
where $\Omega_{\rm eN}$ is the magnitude of the vector $\bm{\Omega}_{\rm eN}$
(see Eq.~\eqref{eq:dsdt_fullft}) and $\Delta I_{\rm eN} = I_{\rm eN} -
\bar{I}_{\rm e}$ where $I_{\rm eN}$ is the angle between $\bm{\Omega}_{\rm eN}$
and $\bm{L}_{\rm out}$ (see Fig.~\ref{fig:3vec}).

\section{Analysis: Approximate Adiabatic Invariant}\label{s:fast_merger}

In general, Eqs.~\eqref{eq:dsdt_fullft} and~\eqref{eq:formal_eom_allgen} are
difficult to study analytically. In this section, we neglect the hamonic terms
and focus on how the varying $\bm{\overline{\Omega}}_{\rm e}$ affects the
evolution of the BH spin axis. The effect of the harmonic terms is studied in
Section~\ref{s:harmonic}.

\subsection{The Adiabatic Invariant}

When neglecting the $N \geq 1$ harmonic terms, Eq.~\eqref{eq:dsdt_fullft}
reduces to
\begin{equation}
    \p{\rd{\bm{\overline{S}}}{t}}_{\rm rot}
        = \bm{\overline{\Omega}}_{\rm e}
            \times \bm{\overline{S}}.\label{eq:dsdt_0only}
\end{equation}

It is not obvious to what extent the analysis of Eq.~\eqref{eq:dsdt_0only} is
applicable to Eq.~\eqref{eq:dsdt_fullft}. From our numerical calculations, we
find that the LK-average of $\bm{S}$ often evolves following
Eq.~\eqref{eq:dsdt_0only}, motivating our notation $\bm{\overline{S}}$. Over
timescales shorter than the LK period $P_{\rm LK}$, Eq.~\eqref{eq:dsdt_0only}
loses accuracy as the evolution of $\bm{S}$ itself is dominated by the $N \geq
1$ harmonics we have neglected. An intuitive interpretation of this result is
that the $N \geq 1$ harmonics vanish when integrating Eq.~\eqref{eq:dsdt_fullft}
over a LK cycle.

Eq.~\eqref{eq:dsdt_0only} has one desirable property: $\bar{\theta}_{\rm e}$,
given by
\begin{equation}
    \cos \bar{\theta}_{\rm e} \equiv
        \bm{\overline{S}} \cdot \frac{\bm{\overline{\Omega}}_{\rm
            e}}{\overline{\Omega}_{\rm e}},
        \label{eq:q_eff}
\end{equation}
is an adiabatic invariant. The adiabaticity condition requires the precession
axis evolve slowly compared to the precession frequency at all times, i.e.
\begin{equation}
    \abs{\rd{\bar{I}_{\rm e}}{t}} \ll \overline{\Omega}_{\rm e}.\label{eq:ad_constr}
\end{equation}
For our fiducial example depicted in Fig.~\ref{fig:4sim_90_350}, the values of
$\dot{\bar{I}}_{\rm e}$ and $\overline{\Omega}_{\rm e}$ are shown in the top
panel of Fig.~\ref{fig:4sim_90_350_supp}, and the evolution of
$\bar{\theta}_{\rm e}$ in the bottom panel. The net change in $\bar{\theta}_{\rm
e}$ in this simulation is $0.01^\circ$, small as expected since
$\abs{\dot{\bar{I}}_{\rm e}} \ll \abs{\overline{\Omega}_{\rm e}}$ at all times.

\subsection{Deviation from Adiabaticity}\label{ss:eom_0}

The extent to which $\bar{\theta}_{\rm e}$ is conserved depends on how well
Eq.~\eqref{eq:ad_constr} is satisfied. In this subsection, we derive a bound on
the total non-conservation of $\bar{\theta}_{\rm e}$, then in the next
subsection we show how this bound can be estimated from initial conditions.

When neglecting harmonic terms, the scalar equation of motion
Eq.~\eqref{eq:formal_eom_allgen} becomes
\begin{align}
    \rd{S_{\perp}}{t} &= i\overline{\Omega}_{\rm e}S_\perp
            - \dot{\bar{I}}_{\rm e} \cos \bar{\theta}_{\rm e}.
\end{align}
This can be solved in closed form. Defining
\begin{equation}
    \Phi(t) \equiv \int\limits^t \overline{\Omega}_{\rm e}\;\mathrm{d}t,
        \label{eq:Phi_t}
\end{equation}
we obtain the solution between the initial time $t_{\rm i}$ and the final time
$t_{\rm f}$:
\begin{equation}
    e^{-i\Phi}S_{\perp}\bigg|_{t_{\rm i}}^{t_{\rm f}}
        = -\int\limits_{t_{\rm i}}^{t_{\rm f}}
            e^{-i\Phi(\tau)}\dot{\bar{I}}_{\rm e} \cos \bar{\theta}_{\rm e}
                \;\mathrm{d}\tau.\label{eq:formal_sol_0}
\end{equation}
Recalling $\abs{S_{\perp}} = \sin \bar{\theta}_{\rm e}$ and analyzing
Eq.~\eqref{eq:formal_sol_0}, we see that $\bar{\theta}_{\rm e}$ oscillates about
its initial value with amplitude
\begin{equation}
    \abs{\Delta \bar{\theta}_{\rm e}} \sim
        \abs{\frac{\dot{\bar{I}}_{\rm e}}{\overline{\Omega}_{\rm
        e}}}.\label{eq:nonad_dqeff}
\end{equation}
In the adiabatic limit [Eq.~\eqref{eq:ad_constr}], $\bar{\theta}_{\rm e}$ is
indeed conserved, as the right-hand side of Eq.~\eqref{eq:nonad_dqeff} goes to
zero. The bottom panel of Fig.~\ref{fig:4sim_90_350_supp} shows $\Delta
\bar{\theta}_{\rm e}$ for the fiducial example. Note that $\bar{\theta}_{\rm
e}$ is indeed mostly constant where Eq.~\eqref{eq:nonad_dqeff} predicts small
oscillations.

If we denote $\abs{\Delta \bar{\theta}_{\rm e}}^{\rm f}$ to be the net change
in $\bar{\theta}_{\rm e}$ over $t \in [t_{\rm i}, t_{\rm f}]$, we can give a
loose bound
\begin{equation}
    \abs{\Delta \bar{\theta}_{\rm e}}^{\rm f} \lesssim
        \abs{\frac{\dot{\bar{I}}_{\rm e}}{\overline{\Omega}_{\rm e}}}_{\max}.
        \label{eq:nonad_dqeff_tot}
\end{equation}

Inspection of Fig.~\ref{fig:4sim_90_350_supp} indicates that the spin dynamics
are mostly uninteresting except near the peak of $\abs{\dot{\bar{I}}_{\rm e}}$,
which occurs where $\bar{\Omega}_{\rm SL} \simeq \abs{\Omega_{\rm L}}$. We
present a zoomed-in view of dynamical quantities near the peak of
$\dot{\bar{I}}_{\rm e}$ in Fig.~\ref{fig:4sim_90_350_zoom}. In particular, in
the bottom-rightmost panel, we see that the fluctuations in $\bar{\theta}_{\rm
e}$ are dominated by a second contribution, the subject of
the discussion in Section~\ref{s:harmonic}.

For comparison, we show in Fig.~\ref{fig:4sim_90_200_zoom} a more rapid
binary merger starting with $I_0 = 90.2^\circ$, for which $\abs{\Delta
\theta_{\rm e}}^{\rm f} \approx 2^\circ$. If we again examine the
bottom-rightmost panel, we see that the net $\abs{\Delta \bar{\theta}_{\rm
e}}^{\rm f}$ obeys Eq.~\eqref{eq:nonad_dqeff_tot}.

\begin{figure}
    \centering
    \includegraphics[width=\columnwidth]{LK90_plots/4sim_90_350_supp.png}
    \caption{The same simulation as depicted in Fig.~\ref{fig:4sim_90_350} but
    showing the calculated quantities relevant to the theory of the spin
    evolution. Top: the four characteristic frequencies of the
    system and $\rdil{\bar{I}_{\rm e}}{t}$. Middle: the frequency ratios between
    the zeroth and first Fourier components of $\bm{\Omega}_{\rm e}$ to the LK
    frequency $\Omega_{\rm LK}$. Bottom: Time evolution of $\theta_{\rm e}$
    [grey line; Eq.~\eqref{eq:q_eff_inst}], $\bar{\theta}_{\rm e}$ [red dots;
    Eq.~\eqref{eq:q_eff}], as well as estimates of the deviations from perfect
    conservation of $\bar{\theta}_{\rm e}$ due to non-adiabaticity [green,
    Eq.~\eqref{eq:nonad_dqeff}] and due to resonances with harmonic terms [blue,
    Eq.~\eqref{eq:harmonic_dqeff}].}\label{fig:4sim_90_350_supp}
\end{figure}
\begin{figure*}
    \centering
    \includegraphics[width=\textwidth]{LK90_plots/4sim_90_350_zoom.png}
    \caption{The same simulation as Fig.~\ref{fig:4sim_90_350} but zoomed in on
    the region around $\mathcal{A} \equiv \overline{\Omega}_{\rm SL} /
    \overline{\Omega}_{\rm L} \simeq 1$ and showing a wide range of relevant
    quantities. The first three panels in the upper row depict $a$, $e$, $I$ and
    $\bar{I}$ as in Fig.~\ref{fig:4sim_90_350}, while the fourth shows
    $\bar{I}_{\rm e}$ [Eq.~\eqref{eq:ie_def}] and $I_{\rm e1}$. The bottom four
    panels depict $\theta_{\rm sl}$, the four characteristic frequencies of the
    system and $\rdil{\bar{I}_{\rm e}}{t}$ [Eqs.~\ref{eq:weff_def}
    and~\eqref{eq:Wldef}] (as in the top panel of
    Fig.~\ref{fig:4sim_90_350_supp}), the relevant frequency ratios (as in the
    middle panel of Fig.~\ref{fig:4sim_90_350_supp}), and the deviation of
    $\bar{\theta}_{\rm e}$ from its initial value compared to the predictions of
    Eqs.~\eqref{eq:nonad_dqeff}
    and~\eqref{eq:harmonic_dqeff}.}\label{fig:4sim_90_350_zoom}
\end{figure*}

\begin{figure*}
    \centering
    \includegraphics[width=\textwidth]{LK90_plots/4sim_90_200_zoom.png}
    \caption{Same as Fig.~\ref{fig:4sim_90_350_zoom} except for $I_0 =
    90.2^\circ$ (and all other parameters are the same as in
    Fig.~\ref{fig:4sim_90_350}), corresponding to a faster coalescence. The
    total change in $\bar{\theta}_{\rm e}$ for this simulation is $\approx
    2^\circ$.}\label{fig:4sim_90_200_zoom}
\end{figure*}

\subsection{Estimate of Deviation from Adiabaticity from Initial Conditions}

To estimate Eq.~\eqref{eq:nonad_dqeff_tot} as a function of initial conditions,
we first differentiate Eq.~\eqref{eq:ie_def},
\begin{equation}
    \dot{\bar{I}}_{\rm e} = \p{\frac{\dot{\mathcal{A}}}{
            \mathcal{A}}}
        \frac{\mathcal{A} \sin \bar{I}}{
            1 + 2\mathcal{A}\cos \bar{I}
                + \mathcal{A}^2}.
\end{equation}
It also follows from Eq.~\eqref{eq:weff_def} that
\begin{equation}
    \overline{\Omega}_{\rm e} = \abs{\overline{\Omega}_{\rm L}}
        \p{1 + 2\mathcal{A}\cos \bar{I}
            + \mathcal{A}^2}^{1/2},
\end{equation}
from which we obtain
\begin{equation}
    \abs{\frac{\dot{\bar{I}}_{\rm e}}{\overline{\Omega}_{\rm e}}}
        = \p{\frac{\dot{\mathcal{A}}}{
            \mathcal{A}}}
        \frac{1}{\abs{\overline{\Omega}_{\rm L}}}
        \frac{\mathcal{A} \sin \bar{I}}{
            \p{1 + 2\mathcal{A}\cos \bar{I}
                + \mathcal{A}^2}^{3/2}}.\label{eq:idot_over_W_med}
\end{equation}
Moreover, if we assume the eccentricity is frozen around $e \simeq 1$ and use
$\overline{\cos^2 \omega} \simeq 1/2$ in $\abs{\Omega_{\rm L}} =
\abs{\rdil{\ascnode}{t}}$, we obtain the estimate
\begin{align}
    \mathcal{A} &\simeq
        \frac{3Gn\p{m_2 + \mu/3}}{
            2c^2a j^2(e)}
                \s{\frac{15\cos \bar{I}}{
                    8t_{\rm LK}j(e)}}^{-1}\nonumber\\
        &\simeq \frac{4}{5}\frac{G(m_2 + \mu/3) m_{12}\tilde{a}_{\rm out}^3}{
            c^2m_3a^4 j(e) \cos \bar{I}},\label{eq:abar_eq1}\\
    \frac{\dot{\mathcal{A}}}{\mathcal{A}}
        &= -4\p{\frac{\dot{a}}{a}}_{\rm GW}
            + \frac{e}{j^2(e)}\p{\rd{e}{t}}_{\rm GW}.\label{eq:abar_dot}
\end{align}
With these, we see that Eq.~\eqref{eq:idot_over_W_med} is largest around
$\mathcal{A} \simeq 1$, and so we find that the maximum $\abs{\dot{\bar{I}}_{\rm
e} / \overline{\Omega}_{\rm e}}$ is given by
\begin{equation}
    \abs{\frac{\dot{\bar{I}}_{\rm e}}{\overline{\Omega}_{\rm e}}}_{\max}
        \simeq \abs{\frac{\dot{\mathcal{A}}}{\mathcal{A}}}
            \frac{1}{\abs{\overline{\Omega}_{\rm L}}}
            \frac{\sin \bar{I}}{\p{2 + 2\cos \bar{I}}^{3/2}}.
            \label{eq:idot_over_W}
\end{equation}

To evaluate this, we make two assumptions: (i) $\bar{I}$ is approximately
constant (see the third panels of Figs.~\ref{fig:4sim_90_350_zoom}
and~\ref{fig:4sim_90_200_zoom}), and (ii) $j(e)$ evaluated at $\mathcal{A}
\simeq 1$ can be approximated as a constant multiple of the initial
$j(e_{\max})$, i.e.
\begin{equation}
    j_{\star} \equiv j(e_{\star}) = f
        \sqrt{\frac{5}{3}\cos^2 I_0},\label{eq:jstar_ansatz}
\end{equation}
where the star subscript denotes evaluation at $\mathcal{A} \simeq 1$ and $f>1$
is a constant. Eq.~\eqref{eq:jstar_ansatz} assumes that $I_0$ far enough from
$90^\circ$ that the GR effect is unimportant in determining $e_{\max}$. The
value of f turns out to be relatively insensitive to $I_0$.

Using Eq.~\eqref{eq:abar_dot} and approximating $e_\star \approx 1$ in
Eqs.~\eqref{eq:dadt_gw} and~\eqref{eq:dedt_gw} give
\begin{equation}
    \s{\frac{\dot{\mathcal{A}}}{\mathcal{A}}}_{\star}
        \simeq \frac{G^3 \mu m_{12}^2}{c^5a_\star^4j_\star^7} \frac{595}{3}.
        \label{eq:adot_star}
\end{equation}
To determine $a_\star$, we require Eq.~\eqref{eq:abar_eq1} to give $\mathcal{A}
= 1$ for $a_\star$ and $j_\star$. Taking this and Eq.~\eqref{eq:adot_star}, we
rewrite Eq.~\eqref{eq:idot_over_W_med} as
% Exponent: (8000 * G^9 * (50 solar mass)^9 * (3e7 solar mass)^3 * (12 solar mass)^8 / ((2.2 pc)^9 * c^18 * (24 solar mass)^11))^(1/8)
% trig part: sin(120) * (cos(60))^(3/8) / (cos(120) + 1)^(3/2)
% together: 180 / (2pi) * (2.72^2 * 5 * cos^2(90.3 degrees) / 3)^(-37/16) * 1.033×10^-11 * 1.888 * 595 / 36
\begin{align}
    \abs{\frac{\dot{\bar{I}}_{\rm e}}{\overline{\Omega}_{\rm e}}}_{\max}
        \approx
            \frac{595 \sin\bar{I} \abs{\cos\bar{I}}^{3/8}}{36
                \left(\cos\bar{I} + 1\right)^{\frac{3}{2}}}
        \left[\frac{8000 G^{9} m_{12}^{9} m_3^{3}
            \mu^{8}}{\tilde{a}_{\rm out}^{9} j_\star^{37}c^{18} (m_2 + \mu /
            3)^{11}}\right]^{1/8}.
\end{align}
We can also calculate $\dot{\bar{I}}_{\rm e} / \overline{\Omega}_{\rm e}$ from
numerical simulations. Taking characteristic $\bar{I} \approx 120^\circ$
(Figs.~\ref{fig:4sim_90_350_zoom} and~\ref{fig:4sim_90_200_zoom} show that this
holds across a range of $I_0$), we fit the last remaining free parameter $f$
[Eq.~\eqref{eq:jstar_ansatz}] to the data from numerical simulations. This
yields $f \approx 2.72$, leading to
\begin{align}
    \abs{\frac{\dot{\bar{I}}_{\rm e}}{\overline{\Omega}_{\rm e}}}_{\max}
        \simeq{}& 0.98^\circ \p{\frac{\cos I_0}{\cos (90.3^\circ)}}^{-37/8}
            \p{\frac{\tilde{a}_{\rm out}}{2.2\;\mathrm{pc}}}^{-9/8}\nonumber\\
        & \times \p{\frac{m_3}{3 \times 10^7\;M_{\odot}}}^{3/8}
            \p{\frac{m_{12}^9 \mu^8 / (m_2 + \mu/3)^{11}}{\p{28.64M_{\odot}}^6}}
                ^{1/8}.
            \label{eq:prediction}
\end{align}
Fig.~\ref{fig:good_quants} shows that when the merger time $T_{\rm m}$ is much
larger than the initial LK timescale, Eq.~\eqref{eq:prediction} provides an
accurate estimate for $\abs{\dot{I}_{\rm e} / \bar{\Omega}_{\rm e}}_{\max}$ when
compared with numerical results.
\begin{figure}
    \centering
    \includegraphics[width=\columnwidth]{LK90_plots/good_quants.png}
    \caption{Comparison of $\abs{\dot{\bar{I}}_{\rm e} / \overline{\Omega}_{\rm
    e}}_{\max}$ obtained from simulations and from the analytical expression
    Eq.~\eqref{eq:prediction}, where we take $f = 2.72$ in
    Eq.~\eqref{eq:jstar_ansatz}. The coalescence time $T_{\rm m}$ is shown along
    the top axis of the plot in units of the characteristic LK timescale at the
    start of inspiral $t_{\rm LK, 0}$. The agreement between the analytical and
    numerical results is excellent for $T_{\rm m} \gg t_{\rm
    LK,0}$.}\label{fig:good_quants}
\end{figure}

In the above, we have assumed that the system evolves through $\mathcal{A}
\simeq 1$  when the eccentricity is mostly frozen (see
Fig.~\ref{fig:4sim_90_350} for an indication of how accurate this is for the
parameter space explored in Fig.~\ref{fig:good_quants}). It is also possible
that $\mathcal{A} \simeq 1$ occurs when the eccentricity is still undergoing
substantial oscillations. In fact, Eq.~\eqref{eq:prediction} remains accurate
in this case when replacing $e$ with $e_{\max}$, due to the following
analysis. Recall that when $e_{\min} \ll e_{\max}$, the binary spends a fraction
$\sim j(e_{\max})$ of the LK cycle near $e \simeq e_{\max}$. This fraction of
the LK cycle dominates both GW dissipation and $\overline{\Omega}_{\rm e}$
precession. Thus, both $\dot{\bar{I}}_{\rm e}$ and $\overline{\Omega}_{\rm e}$
in the oscillating-$e$ regime can be evaluated by setting $e \approx e_{\max}$
and multiplying by a prefactor of $j(e_{\max})$. This factor cancels when
computing the quotient $\dot{\bar{I}}_{\rm e} / \overline{\Omega}_{\rm e}$.

The accuracy of Eq.~\eqref{eq:prediction} in bounding $\abs{\Delta
\bar{\theta}_{\rm e}}^{\rm f}$ is shown in Fig.~\ref{fig:deviations}, where we
carry out simulations for a range of $I_0$, and for each $I_0$ we consider $100$
different, isotropically distributed initial orientations for $\bm{S}$ (thus
sampling a wide range of initial initial $\bar{\theta}_{\rm e}$). Note that
conservation of $\bar{\theta}_{\rm e}$ is generally much better than
Eq.~\eqref{eq:prediction} predicts. This is because cancellation of phases in
Eq.~\eqref{eq:formal_sol_0} is generally more efficient than
Eq.~\eqref{eq:prediction} assumes (recall that Eq.~\eqref{eq:nonad_dqeff} only
provides an estimate for the amplitude of ``local'' oscillations of
$\bar{\theta}_{\rm e}$). Nevertheless, it is clear that
Eq.~\eqref{eq:prediction} provides a robust upper bound of
$\abs{\Delta\bar{\theta}_{\rm e}}^{\rm f}$, and serves as a good indicator for
the breakdown of adiabatic invariance.
\begin{figure}
    \centering
    \includegraphics[width=\columnwidth]{LK90_plots/deviations_one.png}
    \caption{Net change in $\bar{\theta}_{\rm e}$ over the binary inspiral as a
    function of initial inclination $I_0$. For each $I_0$, $100$ simulations are
    run for $\bm{S}$ on a uniform, isotropic grid. Plotted for comparison is the
    bound $\abs{\Delta \bar{\theta}_{\rm e}}^{\rm f} \lesssim
    \abs{\dot{\bar{I}}_{\rm e} / \overline{\Omega}_{\rm e}}_{\max}$, using the
    analytical expression given by Eq.~\eqref{eq:prediction}. It is clear that the
    expression provides a robust upper bound for the non-conservation of
    $\bar{\theta}_{\rm e}$ due to nonadiabatic effects. Note that at the right of the
    plot, the numerical $\abs{\Delta \bar{\theta}_{\rm e}}^{\rm f}$ saturates;
    this is because we compute the initial $\bar{\Omega}_e$ (in order to
    evaluate the initial $\bar\theta_e$) without GW dissipation, and such a
    procedure inevitably introduces fuzziness in $\bar{\theta}_{\rm e}$.
    }\label{fig:deviations}
\end{figure}

\subsection{Origin of the $\theta_{\rm sl}^{\rm f} = 90^\circ$ Attractor
}\label{eq:effect}

In Fig.~\ref{fig:qslscan}, we show $\theta_{\rm sl}^{\rm f}$ as a function of
$I_0$ when $\theta_{\rm sl}^{\rm i} = 0$. With the results developed in the
previous sections, we can understand the behavior seen in this plot.

First, while $\bar{\theta}_{\rm e}$ is our proposed adiabatic invariant,
$\bar{\theta}_{\rm e}^{\rm i}$ is not measurable. As such, we re-express
$\bar{\theta}_{\rm e}$ in terms of physical quantities. We work in the
inertial frame and choose spherical coordinate system where $\bm{L}_{\rm out}
\propto \uv{Z}$. We then specify $\bm{S}$ by the polar and azimuthal angles
$\theta_{\rm s3}$ and $\phi_{ \rm s3}$ respectively. We follow the same
convention as before and choose $\uv{X}$ to point along $\uv{L}$. Finally, we
will consider the case of nonzero $\eta \equiv L / L_{\rm out}$ here, to better
parallel the discussion in LL18. In this case, $\bm{\overline{\Omega}}_{\rm e}$
is given by \citep{bin1}
\begin{equation}
    \bm{\overline{\Omega}}_{\rm e} = \overline{\Omega}_{\rm SL} \uv{L}
        + \overline{\Omega}_{\rm L}\frac{\bm{L}_{\rm tot}}{L_{\rm out}},
\end{equation}
where
\begin{equation}
    \bm{L}_{\rm tot} \equiv \bm{L} + \bm{L}_{\rm out}.
\end{equation}
It can then be shown that
\begin{equation}
    \bm{\overline{\Omega}}_{\rm e} = \overline{\Omega}_{\rm L}\s{\p{
        \mathcal{A} + \eta}\sin \bar{I}\,\uv{X} + \p{\mathcal{A}\cos \bar{I} +
        \eta\cos \bar{I} + 1} \uv{Z}}.
\end{equation}
Then, for some $\bm{S} = \cos \theta_{\rm s3} \uv{Z} + \sin \theta_{\rm
s3}\p{\cos \phi_{\rm s3}\uv{X} + \sin \phi_{\rm s3}\uv{Y}}$, we can evaluate and
find
\begin{equation}
    \cos \bar{\theta}_{\rm e} =
        \frac{\p{\mathcal{A} + \eta}\cos \theta_{\rm sl} \pm \cos \theta_{\rm
        s3}}{\sqrt{(\mathcal{A} + \eta)^2 \pm 2(\mathcal{A} + \eta)\cos \bar{I}
            + 1}}.\label{eq:cos_qe_explicit}
\end{equation}
The sign in the above equation is equal to
$\mathrm{sgn}\p{\overline{\Omega}_{\rm L}}$, i.e.\ positive for $I_0 < 90^\circ$
and negative for $I_0 > 90^\circ$. Our equation is in agreement with Eq.~(73) of
LL18 when taking $\theta_{\rm sl} = 0$ [since they evaluate for $e_0 \approx 0$,
their $\mathcal{A}_0 / 2 \cos I_0$ is our $\mathcal{A}$; see their Eq.~(72)].

Note that, for the fiducial parameter space, $\mathcal{A} \ll 1$ initially
(Fig.~\ref{fig:4sim_90_350_supp}), and $\eta \ll 1$ as well, so
Eq.~\eqref{eq:cos_qe_explicit} simply gives
\begin{equation}
    \bar{\theta}_{\rm e}^{\rm i} \approx
    \begin{cases}
        \theta_{\rm sb}^{\rm i} & I_0 < 90^\circ,\\
        180^\circ - \theta_{\rm sb}^{\rm i} & I_0 > 90^\circ,
    \end{cases}
\end{equation}
On the other hand, at late times, $\mathcal{A} \gg 1$, and $\bar{\theta}_{\rm
e}^{\rm f} = \theta_{\rm sl}^{\rm f}$. Thus, when $\bar{\theta}_{\rm e}$ is
conserved, we obtain
\begin{equation}
    \theta_{\rm sl}^{\rm f} \approx
    \begin{cases}
        \theta_{\rm sb}^{\rm i} & I_0 < 90^\circ,\\
        180^\circ - \theta_{\rm sb}^{\rm i} & I_0 > 90^\circ,
    \end{cases}\label{eq:qsl_relation}
\end{equation}
This shows that we expect $\theta_{\rm sl}^{\rm f} \in [89.5^\circ, 90^\circ]$
for mergers in Fig.~\ref{fig:qslscan} sufficiently far from $I_0 = 90^\circ$,
which is indeed observed. This is the origin of the $90^\circ$ attractor.
This is further in agreement with the recent results in
\citet{yu2020spin}, where they studied arbitrary initial $\bm{S}_{\rm
i}$ orientations and obtained exactly Eq.~\eqref{eq:qsl_relation} [see their
Eq.~(46), bottom panel of Fig.~4].

So far, we have analyzed the $\theta_{\rm sl}^{\rm f}$ distribution for smooth
mergers. Next, we can consider rapid mergers, for which $\bar{\theta}_{\rm e}$
conservation is imperfect. We expect
\begin{equation}
    \abs{\theta_{\rm sl}^{\rm f} - \bar{\theta}_{\rm e}^{\rm i}}
        \lesssim \abs{\Delta \bar{\theta}_{\rm e}}^{\rm
        f}.\label{eq:qslf_plot_black}
\end{equation}
where $\abs{\Delta \bar{\theta}_{\rm e}}^{\rm f}$ is given by
Eq.~\eqref{eq:prediction}. This is overplotted as the black dotted line in
Fig.~\ref{fig:qslscan}, and we see it mostly captures the deviation of
$\theta_{\rm sl}^{\rm f}$ from $\sim 90^\circ$ except very near $I_0 =
90^\circ$. This is expected, as Eq.~\eqref{eq:prediction} is singular for $I_0 =
90^\circ$, and it is expected to lose accuracy in this regime as shown in
Fig.~\ref{fig:deviations}.

When $I_0 = 90^\circ$ exactly, we can show that $\theta_{\rm sl}^{\rm f} =
\theta_{\rm sl}^{\rm i}$: when $I_0 = 90^\circ$, $\rdil{I}{t} = 0$ by
Eq.~\eqref{eq:dIdt}, so $I = 90^\circ$ for all time. Then, $\rdil{\ascnode}{t} =
0$ for $I = 90^\circ$ [Eq.~\eqref{eq:dWdt}], implying that $\bm{L}$ is constant.
Thus, $\bm{S}$ precesses around fixed $\bm{L}$, and $\theta_{\rm sl}$ can never
change. For Fig.~\ref{fig:qslscan}, we take $\theta_{\rm sl}^{\rm i} = 0$, and
indeed we see that $\theta_{\rm sl}^{\rm f} = 0$ at $I_0 = 90^\circ$ in the
figure.

Finally, Fig.~\ref{fig:qslscan} shows that the true $\theta_{\rm sl}^{\rm f}$
are oscillatory within the envelope bounded by Eq.~\eqref{eq:qslf_plot_black}
above. This can also be understood: Eq.~\eqref{eq:nonad_dqeff_tot} only bounds
the maximum of the absolute value of the change in $\bar{\theta}_{\rm e}$, while
the actual change depends on the initial and final complex phases of $S_{\perp}$
in Eq.~\eqref{eq:formal_sol_0}, denoted $\Phi(t_{\rm i})$ and $\Phi(t_{\rm f})$.
When $\theta_{\rm sl}^{\rm i} = 0$, we have $\Phi(t_{\rm i}) = 0$, as $\bm{S}$
starts in the $\uv{x}$-$\uv{z}$ plane. Then, as $I_0$ is smoothly varied, the
final phase $\Phi\p{t_{\rm f}}$ must also vary smoothly [since
$\overline{\Omega}_{\rm e}$ in Eq.~\eqref{eq:Phi_t} is a continuous function,
$\Phi\p{t}$ must be as well], so the total phase difference between the initial
and final values of $S_{\perp}$ varies smoothly. This means the total change in
$\bar{\theta}_{\rm e}$ will fluctuate smoothly between $\pm \abs{\Delta
\bar{\theta}_{\rm e}}^{\rm f}$ as $I_0$ is changed, giving rise to the
sinusoidal shape seen in Fig.~\ref{fig:qslscan}.

% I can't plot this since my plot (905_htol) uses with-GW to calculate \Omega_e.
% Neglecting GW isn't sensitive enough to capture the Phi dependence. Sad.
%
% In fact, this analysis suggests that, for the same $I_0$ (and therefore same
% orbital evolution), different $\bm{S}_{\rm i}$ with the same $\Phi(t_{\rm i})$
% will have the same value of $\Delta \bar{\theta}_{\rm e}$, which is in agreement
% with our numerical simulations

\section{Analysis: Effect of Resonances}\label{s:harmonic}

In the previous section, we neglected the $N \geq 1$ Fourier harmonics in
Eq.~\eqref{eq:dsdt_fullft}, and showed that the final $\theta_{\rm sl}^{\rm f}$
behavior could be completely explained. In this section, we study one effect of
the Fourier harmonics that occurs when two frequencies become commensurate. We
show this effect can be neglected for the fiducial parameter regime. A regime
for which the Fourier harmonics cannot be neglected is discussed separately in
Section~\ref{s:lk_enhanced}.

For simplicity, we ignore the effects of GW dissipation in this section and
assume the system is exactly periodic (so $\dot{\bar{I}}_{\rm e} = 0$). The
scalar equation of motion Eq.~\eqref{eq:formal_eom_allgen} is then:
\begin{align}
    \rd{S_{\perp}}{t} ={}& i\overline{\Omega}_{\rm e}S_\perp
        + \sum\limits_{N = 1}^\infty[
            \cos \p{\Delta I_{\rm eN}}S_\perp \nonumber\\
        &- i\cos \theta \sin \p{\Delta I_{\rm eN}}]
            \Omega_{\rm eN}\cos (N\Omega_{\rm LK} t).\label{eq:formal_sol_gen}
\end{align}
Resonances can occur when $\overline{\Omega}_{\rm e} = N\Omega_{\rm LK}$.
Numerically, we find that $\overline{\Omega}_{\rm e} \lesssim \Omega_{\rm LK}$
for most regions of parameter space (see Fig.~\ref{fig:dWs}, and recall that
LK-induced mergers only complete within a Hubble time when $I_{\min} \approx
90^\circ$). Accordingly, we restrict our analysis to resonances with the $N = 1$
component. For simplicity, we also ignore the modulation of the forcing
frequency in Eq.~\eqref{eq:formal_sol_gen}. While this fails to capture the
possibility of a parametric resonance, we find no evidence for parametric
resonances in our simulations. With these two simplifications,
Eq.~\eqref{eq:formal_sol_gen} further reduces to
\begin{align}
    \rd{S_{\perp}}{t} &\approx i\overline{\Omega}_{\rm e}S_\perp
        - i\cos \bar{\theta}_{\rm e} \sin \p{\Delta I_{\rm e1}} \Omega_{\rm e1}
            \cos \p{\Omega_{\rm LK} t}.
\end{align}
We can approximate $\cos \p{\Omega_{\rm LK}t} \approx e^{i\Omega_{\rm LK} t} /
2$, as the $e^{-i\Omega_{\rm LK} t}$ component is far from resonance. Then we
can write down solution as before
\begin{align}
    e^{-i\overline{\Omega}_{\rm e}t}S_{\perp}\bigg|_{t_{\rm i}}^{t_{\rm f}}
        &= -\int\limits_{t_{\rm i}}^{t_{\rm f}}
            \frac{i\sin\p{\Delta I_{\rm e1}} \Omega_{\rm e1}}{2}
                e^{-i\overline{\Omega}_{\rm e}t + i\Omega_{\rm LK} t} \cos
                \bar{\theta}_{\rm e}
            \;\mathrm{d}t.\label{eq:harmonic_dS}
\end{align}
Thus, similarly to Section~\ref{ss:eom_0}, the instantaneous oscillation
amplitude $\abs{\Delta \bar{\theta}_{\rm e}}$ can be bound by
\begin{equation}
    \abs{\Delta \bar{\theta}_{\rm e}} \sim \frac{1}{2}
        \abs{\frac{\sin \p{\Delta I_{\rm e1}} \Omega_{\rm e1}}{
            \Omega_{\rm LK} - \overline{\Omega}_{\rm e}}}.
        \label{eq:harmonic_dqeff}
\end{equation}
We see that if $\overline{\Omega}_{\rm e} < \Omega_{\rm LK}$ by a sufficient
margin for all times, then the conservation of $\bar{\theta}_{\rm e}$ in the
fiducial parameter regime cannot be significantly affected by this resonance.
The ratio $\overline{\Omega}_{\rm e} / \Omega_{\rm LK}$ is shown in the middle
panel of Fig.~\ref{fig:4sim_90_350_supp}, and the amplitude of oscillation of
$\bar{\theta}_{\rm e}$ it generates [Eq.~\eqref{eq:harmonic_dqeff}] is given in
blue in the bottom panel of Fig.~\ref{fig:4sim_90_350_supp}. We see that the
total effect of the harmonic terms never exceeds a few degrees.

Furthermore, we see from Fig.~\ref{fig:4sim_90_350_zoom} (top rightmost panel and
third panel of bottom row) that the interesting dynamics, occuring when
$\bm{\Omega}_{\rm e1}$ and $\Delta I_{\rm e1}$ are both nonzero [necessary for
Eq.~\eqref{eq:harmonic_dqeff} to be nonzero], occur in the regime $\mathcal{A}
\simeq 1$. Then, the bottom-rightmost panel of Fig.~\ref{fig:4sim_90_350_zoom}
compares the detailed behavior of $\bar{\theta}_{\rm e}$ and its two
contributions, the nonadiabatic and harmonic effects. We see that
Eq.~\eqref{eq:harmonic_dqeff} describes the oscillations in $\bar{\theta}_{\rm
e}$ very well. The agreement is poorer in the bottom-rightmost panel of
Fig.~\ref{fig:4sim_90_200_zoom}, as the nonadiabatic effect is much stronger.
However, note that the theory presented in the previous section captures the
final deviations $\abs{\Delta \bar{\theta}_{\rm e}}^{\rm f}$ very well (see
Fig.~\ref{fig:deviations} for $I_0 = 90.35^\circ$). This suggests that
oscillations in $\bar{\theta}_{\rm e}$ due to harmonic perturbations of up to a
few degrees do not affect final nonconservation by more than $\sim 0.01^\circ$.

\begin{figure}
    \centering
    \includegraphics[width=\columnwidth]{LK90_plots/5_dWs.png}
    \caption{$e_{\max}$ and $\overline{\Omega}_{\rm e} / \Omega_{\rm LK}$ as a
    function of $I_{\min}$, the inclination of the inner binary at eccentricity
    minimum, for varying values of $e_{\min}$ (different colors) for the
    fiducial parameter regime. }\label{fig:dWs}
\end{figure}

\section{Lidov-Kozai Enhanced Mergers}\label{s:lk_enhanced}

In LL17, a different parameter regime is considered,
where the inner binary is sufficiently close in ($\sim 0.1\;\mathrm{AU}$) that
it can merge in isolation via GW radiation, given by: $m_1 = m_2 = m_3 =
30M_{\odot}$, $a_{\rm in} = 0.1\;\mathrm{AU}$, $\tilde{a}_{\rm out} = 3\;\mathrm{AU}$,
and $e_{\rm out} = 0$. Note that $m_3$ here is not an SMBH\@. However, our results can
still be applied judiciously to this parameter regime and yield interesting
insights.

First, we recall the $\theta_{\rm sl}^{\rm f}$ distribution obtained via
numerical simulation, shown in Fig.~\ref{fig:bin_comp}. LL17 derives an
adiabatic invariant assuming the inner binary does not undergo eccentricity
oscillations. Our result, based on $\bar{\theta}_{\rm e}$ conservation, is a
generalization of their result, giving the same result when the inner orbit
remains circular. Very near $I_0 \approx 90^\circ$, the data are offset somewhat
from our result, because we have assumed the tertiary's angular momentum is
fixed, but accounting for the offset, our theory captures the scaling of
$\theta_{\rm sl}^{\rm f}$, as seen in Fig.~\ref{fig:bin_comp_zoom}.

\begin{figure}
    \centering
    \includegraphics[width=\columnwidth]{LK90_plots/6bin_comp.png}
    \caption{Plot of $\theta_{\rm sl}^{\rm f}$ for the LK-enhanced parameter
    regime, i.e.\ $m_1 = m_2 = m_3 = 30M_{\odot}$,
    $a_{\rm in} = 0.1\;\mathrm{AU}$, $\tilde{a}_{\rm out} = 3\;\mathrm{AU}$, and $e_{\rm out}
    = 0$. Conservation of $\bar{\theta}_{\rm e}$ gives the green line. Agreement near
    $I_0 = 90^\circ$ is good when accounting for the effects of a finite
    $L_{\rm out}$ (see Fig.~\ref{fig:bin_comp_zoom}). For $I_0$ in the two
    intervals $[40^\circ, 80^\circ]$ and $[100^\circ, 140^\circ]$, a further
    effect causes $\theta_{\rm sl}^{\rm f}$ to fluctuate
    unpredictably.}\label{fig:bin_comp}
\end{figure}

\begin{figure}
    \centering
    \includegraphics[width=\columnwidth]{LK90_plots/6bin_comp_zoom.png}
    \caption{Zoomed in version of Fig.~\ref{fig:bin_comp} near $I_0 \approx
    90^\circ$ while adding an $I_0$ offset to account for differences between
    the data and theory due to finite $L_{\rm out}$ effects. The green shaded
    area shows the expected range of deviations due to resonant perturbations
    following Eq.~\eqref{eq:harmonic_dqeff} (evaluated for the initial system
    parameters). The data nearer $90^\circ$ have less spread than predicted, but
    the transition to a larger $\theta_{\rm sl}^{\rm f}$ spread roughly follows
    the prediction of the green line.}\label{fig:bin_comp_zoom}
\end{figure}

However, as can be seen in Fig.~\ref{fig:bin_comp}, intermediate
inclinations $I_0 \in [50, 80]$ and $I_0 \in [100, 130]$ exhibit very volatile
behavior in $\theta_{\rm sl}^{\rm f}$. This is unlike the plots generated in the
fiducial parameter regime (Fig.~\ref{fig:qslscan}), as this inclination regime
corresponds to neither the fastest nor slowest merging systems. We attribute the
origin of this volatility to a stronger resonant interaction. By examining
Fig.~\ref{fig:dWs_inner}, it is evident that, for the same $e_{\min}$, systems
with $I_{\min}$ further from $90^\circ$ are closer to the
$\overline{\Omega}_{\rm e} = \Omega_{\rm LK}$ resonance. Outside of the LK
window, $\overline{\Omega}_{\rm e}$ also goes swiftly to zero, as seen in
Fig.~\ref{fig:dWs_inner}, so this explanation is consistent with the
ranges of inclinations that exhibit volatile $\theta_{\rm sl}^{\rm f}$ behavior.
We forgo further investigation of this mechanism because it is not expected to
play an important role in any LK-induced BH binary mergers for reasons discussed
below.

\begin{figure}
    \centering
    \includegraphics[width=\columnwidth]{LK90_plots/5_dWs_inner.png}
    \caption{Same as Fig.~\ref{fig:dWs} but for the compact parameter regime.
    Cmopared to Fig.~\ref{fig:dWs}, we see that $e_{\max}$ is smaller due to
    stronger pericenter precession $\Omega_{\rm GR}$ in this regime, so the
    $\overline{\Omega}_{\rm e} / \Omega_{\rm LK} = 1$ resonance is accessible
    for a wide range of parameter space. In particular, both smaller $e_{\min}$
    and $e_{\max}$ values bring the system closer to the
    resonance.}\label{fig:dWs_inner}
\end{figure}

At first, it seems clear from Fig.~\ref{fig:dWs} that $\overline{\Omega}_{\rm e}$
is significantly smaller than $\Omega_{\rm LK}$ near $I_{\min} \approx 90^\circ$
for LK-induced mergers. However, this is not sufficient to guarantee that the
frequency ratio remains small for the entire evolution, as GR effects become
stronger as the binary coalesces. Instead, a more careful analysis of the
relevant quantities in Eq.~\eqref{eq:harmonic_dqeff} proves useful:
\begin{itemize}
    \item $\sin \p{\Delta I_{\rm e1}}$ is small unless $\mathcal{A} \simeq
        1$. Otherwise, $\bm{\Omega}_{\rm e}$ does not nutate appreciably within
        an LK cycle, and all the $\bm{\Omega}_{\rm eN}$ are aligned with
        $\bm{\overline{\Omega}}_{\rm e}$, implying all the $\Delta I_{\rm eN}
        \approx 0$.

    \item Smaller values of $e_{\min}$ increase $\overline{\Omega}_{\rm e} /
        \Omega_{\rm LK}$, as shown in Fig.~\ref{fig:dWs_inner}.
\end{itemize}
However, LK-driven coalescence causes $\mathcal{A}$ to increase on a similar
timescale to that of $e_{\min}$ increase (see Fig.~\ref{fig:4sim_90_350}). This
implies that, if $\mathcal{A} \ll 1$ initially, which is the case for LK-induced
mergers, then $e_{\min}$ will be very close to $1$ when $\mathcal{A}$ grows to
be $\simeq 1$, and the contribution predicted by Eq.~\eqref{eq:harmonic_dqeff}
must be small.

\section{Conclusion and Discussion}\label{s:discussion}

In this paper, we consider the evolution of the spin-orbit misalignment angle
$\theta_{\rm sl}$ of a black hole (BH) binary that merges under gravitational
wave (GW) radiation during Lidov-Kozai (LK) oscillations induced by a tertiary
supermassive black hole (SMBH). We show that, when the gravitational potential
of the SMBH is handled at quadrupolar order, the spin vectors of the inner BHs
obey the simple equation of motion Eq.~\eqref{eq:dsdt_weff}. Analysis of this
equation yields the following conclusions:
\begin{itemize}
    \item Since Eq.~\eqref{eq:dsdt_weff} is a linear system with periodically
        varying coefficients, it cannot give rise to chaotic dynamics by
        Floquet's Theorem.

    \item For most parameters of astrophysical relevance, the angle
        $\bar{\theta}_{\rm e}$ [Eq.~\eqref{eq:q_eff}] is an adiabatic invariant.
        Since the inner BH binary merges in finite time, $\bar{\theta}_{\rm e}$
        is only conserved to finite accuracy; we show that the deviation from
        perfect adiabaticity can be predicted from initial conditions.

    \item When the resonant condition $\bar{\Omega}_{\rm e} \approx \Omega_{\rm
        LK}$ is satisfied, significant oscillations in $\bar{\theta}_{\rm e}$
        can arise. We derive an analytic estimate of this oscillation amplitude.
        This estimate both demonstrates that the resonance is unimportant for
        ``LK-induced'' mergers and tentatively explains the scatter in
        $\theta_{\rm sl}^{\rm f}$ seen by LL17.
\end{itemize}

\bibliography{Su_LK90}
\bibliographystyle{aasjournal}

\end{document}
