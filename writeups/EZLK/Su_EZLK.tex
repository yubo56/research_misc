    \documentclass[
        fleqn,
        usenatbib,
        % referee,
    ]{mnras}
    \usepackage{
        amsmath,
        amssymb,
        newtxtext,
        newtxmath,
        ae, aecompl,
        graphicx,
        booktabs,
        xcolor,
    }
    \usepackage[T1]{fontenc}
    \usepackage[
        labelfont=bf, % caption names are labeled in bold
        font=scriptsize % smaller font for captions
    ]{caption}
    \usepackage[caption=false]{subfig} % subfigures

    \newcommand*{\scinot}[2]{#1\times10^{#2}}
    \newcommand*{\rd}[2]{\frac{\mathrm{d}#1}{\mathrm{d}#2}}
    \newcommand*{\rtd}[2]{\frac{\mathrm{d}^2#1}{\mathrm{d}#2^2}}
    \newcommand*{\pd}[2]{\frac{\partial#1}{\partial#2}}
    \newcommand*{\ptd}[2]{\frac{\partial^2#1}{\partial#2^2}}
    % inline
    \newcommand*{\mdil}[2]{\mathrm{D}#1/\mathrm{D}#2}
    \newcommand*{\pdil}[2]{\partial#1/\partial#2}
    \newcommand*{\rdil}[2]{\mathrm{d}#1/\mathrm{d}#2}
    \newcommand*{\md}[2]{\frac{\mathrm{D}#1}{\mathrm{D}#2}}
    \newcommand*{\at}[1]{\left.#1\right|}
    \newcommand*{\abs}[1]{\left|#1\right|}
    \newcommand*{\ev}[1]{\left\langle#1\right\rangle}
    \newcommand*{\p}[1]{\left(#1\right)}
    \newcommand*{\s}[1]{\left[#1\right]}
    \newcommand*{\z}[1]{\left\{#1\right\}}
    \newcommand*{\bm}[1]{\mathbf{#1}}
    \newcommand*{\uv}[1]{\hat{\mathbf{#1}}}
    \DeclareMathOperator*{\med}{med}
    \DeclareMathOperator*{\erf}{erf}

    \colorlet{Corr}{red}

    \newlength{\colummwidth}
    \setlength{\colummwidth}{246.0pt} % columnwidth for reprint

\title[Mass Ratio Distribution]{The Mass Ratio Distribution of
Black Hole Mergers Induced by a Comparable Mass Tertiary}
% \author[Y. Su et\ al.]{
% Yubo Su,$^1$,
% Dong Lai$^1$
% \\
% $^1$ Cornell Center for Astrophysics and Planetary Science, Department of
% Astronomy, Cornell University, Ithaca, NY 14853, USA
% }

\date{Accepted XXX\@. Received YYY\@; in original form ZZZ}

\pubyear{2021}

\begin{document}\label{firstpage}
\pagerange{\pageref{firstpage}--\pageref{lastpage}}
\maketitle

\begin{abstract}
    Abstract
\end{abstract}

\begin{keywords}
binaries:close -- stars:black holes % chktex 8
\end{keywords}

% so that \cite[see][for q]{QQ} gives (see QQ for q) without comma after QQ
% \setcitestyle{notesep={ }}

\section{Introduction}\label{s:intro}

We study the von Zeipel-Lidov-Kozai effect (ZLK) for eccentric perturbers to
octupole order, also sometimes known as the eccentric Kozai mechanism
\citep[e.g.][]{lithwick2011eccentric}.

The mass ratio distribution among ZLK-induced BBH mergers has already been noted
\citep[see Fig.~10 of][]{silsbee2017lidov}, but the origin of the effect has not
been carefully studied.

\section{Dynamics Without Gravitational Wave Radiation}\label{s:background}

We first briefly review results in regimes of increasing complexity.

We use the octupole-order, double-averaged vectorial ZLK equations from LML15
\citep{LML15}, including 1PN de Sitter apsidal precession. In this section, we
neglect 2.5PN GW radiation \citep{peters1964}, but will consider it later in
Section~\ref{s:with_gw}.

\subsection{Quadrupole-Order ZLK}

When considering the EOM to quadrupole order, the eccentricity of the inner
binary cycles regularly over $\sim t_{\rm ZLK}$ with
\begin{equation}
    t_{\rm LK} = \frac{1}{n_1}\frac{m_{12}}{m_3}
            \p{\frac{a_{\rm out, eff}}{a}}^3,
\end{equation}
where $a_{\rm out, eff} \equiv a_{\rm out}\sqrt{1 - e_{\rm out}^2}$. Our
fiducial parameters are: $a = 100\;\mathrm{AU}$, $a_{\rm out, eff} =
3600\;\mathrm{AU}$, $m_{12} = 50M_{\odot}$, $m_3 = 30M_{\odot}$.

The dynamics to quadrupole order depend on the ratio of the angular momenta,
given by
\begin{equation}
    \eta \equiv \at{\frac{L}{L_{\rm out}}}_{e = 0}
        = \frac{\mu}{\mu_{\rm out}}\s{\frac{m_{12}a}
            {m_{123}a_{\rm out}\p{1 - e_{\rm out}^2}}}^{1/2}.\label{eq:def_eta}
\end{equation}
Note that to quadrupolar order, $e_{\rm out}$ and thus $\eta$ are constant.

We further have the following results:
\begin{itemize}
    \item The system spends a fraction $\sim j(e_{\max})$ of each ZLK
        eccentricity cycle near $e_{\max}$ \citep{anderson2016formation}.

    \item Two conserved quantities, energy and $K \equiv j \cos I - \eta e^2 /
        2$. $K$ is the general conserved quantity which when $\eta = 0$ reduces
        to the classical ``Kozai Constant'' \citep{LML15}.
\end{itemize}

When 1PN is considered, the strength is measured by
\begin{equation}
    \epsilon_{\rm GR} \equiv \frac{3Gm_{12}}{c^2}
        \frac{m_{12}}{m_3}\frac{a_{\rm out, eff}^3}{a^4}.
\end{equation}
This gives further results:
\begin{itemize}
    \item It can be shown that $e_{\max}$ and $j_{\min} \equiv j(e_{\max})$ for
        a given $I_{\rm 0}$ obeys \citep{LML15, anderson2016formation}:
        \begin{align}
            \frac{3}{8}\frac{j_{\min}^2 - 1}{j_{\min}^2}\Big[&
                5\p{\cos I_0 + \frac{\eta}{2}}^2
                - \p{3 + 4\eta \cos I_0 + \frac{9}{4}\eta^2}j_{\min}^2
                    \nonumber\\
                &+ \eta^2 j_{\min}^4
            \Big] + \epsilon_{\rm GR}\p{1 - 1 / j_{\min}} = 0.\label{eq:emax_quad}
        \end{align}
        The minimum value of $j_{\min}$, denoted $j_{\lim}$, occurs when $I_0
        = I_{\rm 0, \lim}$, where
        \begin{equation}
            \cos I_{\rm 0, \lim} = \frac{\eta}{2}\p{\frac{4}{5}j_{\lim}^2 -
                1}.\label{eq:I0lim}
        \end{equation}
        Note that $I_{\rm 0, \lim} \geq 90^\circ$ with equality only when $\eta
        = 0$. Substituting Eq.~\eqref{eq:I0lim} into Eq.~\eqref{eq:emax_quad}
        gives
        \begin{equation}
            \frac{3}{8}\p{j_{\lim}^2 - 1}\s{-3 + \frac{\eta^2}{4}
                \p{\frac{4}{5}j_{\lim}^2 - 1}}
                + \epsilon_{\rm GR}\p{1 - 1 / j_{\lim}} = 0.
                \label{eq:def_elim}
        \end{equation}
\end{itemize}

\subsection{Octupole-Order ZLK, Test-Particle Limit}

Well-studied in test particle limit where $m_2 = 0$ and $\eta = 0$, we review a
few relevant results. In this limit, define
\begin{equation}
    \epsilon_{\rm oct}^{\rm (tp)} = \frac{a}{a_{\rm
        out}} \frac{e_{\rm out}}{1 - e_{\rm out}^2}.\label{eq:eps_oct_tp}
\end{equation}

\begin{itemize}
    \item In this limit, $K = j \cos I$.

    \item When $0 < \epsilon_{\rm oct}$, orbit flips become possible, and
        during these orbit flips, $e = e_{\lim}$ is attained
        \citep{lithwick2011eccentric, LML15}. The range of inclinations for
        which this is possible is wider than the quadrupole-only case.

    \item \citet{katz2011long} show that, with $0 < \epsilon_{\rm oct} \ll 1$,
        $K$ oscillates in a well-behaved way. An approximate analytic
        calculation gives an accurate prediction of this orbit-flipping window
        when $\epsilon_{\rm oct} \ll 1$ and when $\omega$, the argument of
        pericenter of the inner orbit, is circulating. Timescale is given by
        \citet{antognini2015timescales}.

    \item \citep{MLL16} shows that the octupole-active region can be well fit by
        a fitting formula for $\epsilon_{\rm oct}$ as large as $0.07$. The
        blue dots in top panel of Fig.~\ref{fig:composites_tp} show that indeed
        the maximum eccentricity reaches $e_{\lim}$ within this inclination
        range.
\end{itemize}

\subsection{Octupole-Order ZLK, General Masses}

\begin{figure}
    \centering
    \includegraphics[width=\colummwidth]{../../scripts/octlk/1nogw_sims/1nogw_vec.png}
    \caption{Fiducial simulation showing orbit flips, which occur when $K$
    crosses the dotted line $K = K_{\rm c} \approx \eta / 2$. $I_0 =
    93.5^\circ$. When $\omega$ is circulating, the angle $\Omega_{\rm e}$ (the
    co-longitude of the inner eccentricity vector) changes slowly
    \citep{katz2011long}. }\label{fig:nogw_fiducial}
\end{figure}
When $m_1, m_2$ are comparable, Eq.~\eqref{eq:eps_oct_tp} generalizes to
\begin{equation}
    \epsilon_{\rm oct} = \frac{m_1 - m_2}{m_{12}} \frac{a}{a_{\rm out}}
        \frac{e_{\rm out}}{1 - e_{\rm out}^2}.\label{eq:eps_oct}
\end{equation}
The comparable-mass regime is qualitatively different from the test-particle
regime (e.g. Rodet et al, 2021). We show a fiducial simulation in
Fig.~\ref{fig:nogw_fiducial}

\begin{itemize}
    \item No analytic solution to predict amplitude of $K$ (generalized version)
        oscillations, though qualitative behavior is the same.

    \item Asymmetric octupole-active region (see top panels of
        Figs.~\ref{fig:composites_dist}--\ref{fig:composites_e91p5}).

    \item We describe the behavior with two characteristic eccentricities:
        $e_{\max}$ and an effective $e_{\rm eff}$. The latter is defined via an
        average over many ZLK cycles (denoted by angle brackets) and $j_{\rm
        eff} \equiv \sqrt{1 - e_{\rm eff}^2}$:
        \begin{align}
            \ev{\rd{\ln a}{t}} &\approx -\frac{1}{t_{\rm GW, 0}}
                    \ev{\frac{1 + 73e_{\max}^2/24 + 37e_{\max}^4/96}
                        {j_{\min}^6}}\nonumber\\
                &\equiv -\frac{421/96}{t_{\rm GW, 0}}\frac{1}{j_{\rm eff}^6}.
                \label{eq:def_e_eff}
        \end{align}
\end{itemize}

\section{With GW}\label{s:with_gw}

We turn on GW, binaries will either merge or not within $10\;\mathrm{Gyr}$,
produces middle panels of Fig.~\ref{fig:composites_dist}.
\begin{figure}
    \centering
    \includegraphics[width=\colummwidth]{../../scripts/octlk/1sweepbin/composite_tp.png}
    \caption{Fiducial parameters ($a = 100\;\mathrm{AU}$, $a_{\rm out, eff} =
    3600\;\mathrm{AU}$, $m_{12} = 50M_{\odot}$, $m_3 = 30M_{\odot}$) where
    $e_{\rm out} = 0.6$ and $q = 0.01$, i.e.\ in the test-particle regime
    $\eta \ll 1$ [Eq.~\eqref{eq:def_eta}]. In the top panel, for each of
    $1000$ initial inclinations, we choose $5$ different random $\omega$,
    $\omega_{\rm out}$, and $\Omega$ as initial conditions, then run for
    $500t_{\rm ZLK}$ without gravitational wave radiation. The
    maximum eccentricity $e_{\max}$ (blue dots) as well as the effective
    eccentricity [Eq.~\eqref{eq:def_e_eff}; green dots] over this period are
    displayed. For comparison, $e_{\rm eff, c}$ [Eq.~\eqref{eq:def_e_eff_c}] is
    given by the horizontal green dashed line, $e_{\rm os}$
    [Eq.~\eqref{eq:def_e_os}] is shown in the horizontal blue line, and
    $e_{\lim}$ [Eq.~\eqref{eq:def_elim}] is shown in the horizontal blue dashed
    line. The vertical purple lines are the fitting formula of
    \citet{MLL16} and accurately capture the range of initial $I_0$ for which
    $e_{\max}$ reaches $e_{\lim}$. In the bottom panel, we simulate the same
    initial conditions with gravitational wave radiation and record the
    outcome, which is either a successful merger ($T_{\rm m} <
    10\;\mathrm{Gyr}$; green dots) or not ($T_{\rm m} > 10\;\mathrm{Gyr}$; blue
    triangles). Since each $I_0$ is run multiple times with random $\omega$,
    $\omega_{\rm out}$, and $\Omega$, we can estimate a merger probability for
    each $I_0$, depicted in the middle panel. In the top panel, systems are
    expected to merge according to the GW-free simulations if $e_{\max} >
    e_{\rm os}$ (blue dots below blue line) or if $e_{\rm eff} > e_{\rm eff,
    c}$ (green dots below green line). This criterion accurately captures the
    regions for which the dissipative simulations predict nonzero merger
    probabilities (middle panel).
    }\label{fig:composites_tp}
\end{figure}
\begin{figure}
    \centering
    \includegraphics[width=\colummwidth]{../../scripts/octlk/1sweepbin/composite_1p5dist.png}
    \caption{Same as Fig.~\ref{fig:composites_tp} but where instead $q = 0.5$,
    so $\eta \approx 0.087$ is nonnegligible, and $\epsilon_{\rm oct} \approx
    0.007$. The \citet{MLL16} fitting formula (the vertical purple lines in the
    top panel) no longer accurately captures the octupole-active inclinations,
    and none of the panels are symmetric about $I_0 = 90^\circ$. However, the
    $e_{\max}$ and $e_{\rm eff}$ criteria still accurately predict the regions
    where the merger probability is nonzero. }\label{fig:composites_dist}
\end{figure}
\begin{figure}
    \centering
    \includegraphics[width=\colummwidth]{../../scripts/octlk/1sweepbin/composite_1p2dist.png}
    \caption{Same as Fig.~\ref{fig:composites_dist} but for $q = 0.2$, $\eta
    \approx 0.054$, and $\epsilon_{\rm oct} \approx 0.014$. Note that prograde
    perturbers ($I_0 < 90^\circ$) are able to induce mergers, unlike in
    Fig.~\ref{fig:composites_dist}. A ``gap'' in merger probabilities for $I_0
    \approx 90^\circ$ is evident, which is discussed further in
    Appendix~\ref{app:gap}. }\label{fig:composites_1p2}
\end{figure}
\begin{figure}
    \centering
    \includegraphics[width=\colummwidth]{../../scripts/octlk/1sweepbin/composite_e91p5dist.png}
    \caption{Same as Fig.~\ref{fig:composites_dist} but $e_{\rm out} = 0.9$
    while holding $a_{\rm out, eff} = 3600\;\mathrm{AU}$ constant. Here, $\eta
    \approx 0.12$ while $\epsilon_{\rm oct} \approx 0.019$. Note that even
    though $\epsilon_{\rm oct}$ is larger here than in
    Fig.~\ref{fig:composites_1p2}, no mergers are possible here with a prograde
    perturber. This is because $\eta$ is more than twice as large for these
    parameters.
    }\label{fig:composites_e91p5}
\end{figure}

\begin{itemize}
    \item Define $e_{\rm os}$, and $j_{\rm os} = \sqrt{1 - e_{\rm os}^2}$, the
        $e_{\max}$ required to produce a one-shot merger ZLK\@:
        \begin{align}
            j\p{e_{\rm os}}\at{\rd{\ln a}{t}}_{e = e_{\rm os}} &\sim
                \frac{1}{t_{\rm ZLK}},\\
            j_{\rm os}^6
                &= \frac{256}{5}\frac{G^3 \mu m_{12}^3}{m_3c^5a^4n}
                    \p{\frac{a_{\rm out, eff}}{a}}^3.
                    \label{eq:def_e_os}
        \end{align}
        If an IC has $e_{\max} \gtrsim e_{\rm os}$, then gives one-shot merger.
        See comparison between top and middle panels for
        Figs.~\ref{fig:composites_tp}--\ref{fig:composites_e91p5}.

    \item Define $e_{\rm eff, c}$ and $j_{\rm eff, c} \equiv \sqrt{1 - e_{\rm
        eff, c}^2}$, the critical effective eccentricity, such that
        \begin{equation}
            \ev{\rd{\ln a}{t}} = \frac{421/96}{t_{\rm GW, 0}j_{\rm eff, c}^6}
                = 10\;\mathrm{Gyr}.
                \label{eq:def_e_eff_c}
        \end{equation}
        If an IC has $e_{\rm eff} \gtrsim e_{\rm eff, c}$, then gives gradual
        merger.

        For wide binaries, systems generally satisfy $e_{\max} \gtrsim e_{\rm
        os}$ if and only if they satisfy $e_{\rm eff} \gtrsim e_{\rm eff, c}$.
        However, this need not be the case, see
        Fig.~\ref{fig:composite_bindist}.
\end{itemize}
\begin{figure}
    \centering
    \includegraphics[width=\colummwidth]{../../scripts/octlk/1sweepbin/bindist.png}
    \caption{Same as Figs.~\ref{fig:composites_tp}--\ref{fig:composites_1p2}
    but for a compact inner binary; the parameters are $a = 10\;\mathrm{AU}$,
    $a_{\rm out, eff} = 700\;\mathrm{AU}$, $m_{12} = 50M_{\odot}$, $m_3 =
    30M_{\odot}$, and $e_{\rm out} = 0.9$, $q = 0.4$. Note that when the
    perturber is prograde ($I_0 < 90^\circ$), $e_{\max} <
    e_{\rm os}$ but $e_{\rm eff} > e_{\rm eff, c}$, which predicts that systems
    are able to merge. This prediction is borne out by simulations with GW
    radiation (middle and bottom panels). }\label{fig:composite_bindist}
\end{figure}

\section{Mass Ratio Signature in a Population of Merging BBHs}

A physically representative ensemble is beyond the scope of this paper, and we
instead focus on simple, illustrative populations of BBH to demonstrate the
effect.

\subsection{Fixed Tertiary Eccentricity}

We first consider the simplified case where the tertiary eccentricity is fixed,
and seek the merger probability as a function of mass ratio $q$. For these, we
let $\cos I_0$ be drawn uniformly from the range $[-1, 1]$, and fix everything
else. See Fig.~\ref{fig:total_merger_fracs}. This suggests that low mass ratio
mergers should be more common.
\begin{figure*}
    \centering
    \includegraphics[width=0.8\textwidth]{../../scripts/octlk/1sweepbin/total_merger_fracs.png}
    \caption{From Figs.~\ref{fig:composites_dist}--\ref{fig:composites_e91p5},
    we can compute the total merger probability in the presence of GW radiation
    assuming $\cos I_0$ is uniformly distributed $\in [-1, 1]$. We do this for
    three values of $e_{\rm out}$ and six values of $q$ and show the results
    with solid dots. The X's show the results when GW-free simulations are
    instead used to predict the total merger probability, i.e.\ using the
    criteria shown in the top panels of
    Figs.~\ref{fig:composites_dist}--\ref{fig:composites_e91p5}; good agreement
    is observed. }\label{fig:total_merger_fracs}
\end{figure*}
\begin{itemize}
    \item For sufficiently small $\epsilon_{\rm oct}$, the merger probability
        does not depend on $e_{\rm out}$ and $q$ separately but only on their
        combination via $\epsilon_{\rm oct}$.

    \item However, when $\epsilon_{\rm oct}$ is sufficiently large, an $e_{\rm
        out}$ dependence is introduced. This is because the size of the gap (see
        e.g.\ Fig.~\ref{fig:composites_e91p5}), and the value of $\epsilon_{\rm
        oct}$ when prograde systems are able to merge, depend on $q$ through
        $\eta$. A discussion of this gap is in Appendix~\ref{app:gap}.
\end{itemize}

We also examine what happens when $a = 50\;\mathrm{AU}$; \textcolor{Corr}{This
figure is still being generated}.

\subsection{Distribution of Tertiary Eccentricities}

We can also draw $e_{\rm out} \in [0, 0.9]$. We try either uniform probability
or thermal $P(e_{\rm out}) \propto e_{\rm out}$. The results are given for our
fiducial $a_{\rm out, eff} = 3600\;\mathrm{AU}$ in Fig.~\ref{fig:popsynth}. For
comparison, we also show the results when $a_{\rm out, eff} =
5500\;\mathrm{AU}$.
\begin{figure}
    \centering
    \includegraphics[width=\columnwidth]{../../scripts/octlk/1popsynth/a2eff3600.png}
    \caption{Merger fractions with the fiducial parameters obtained by randomly
    drawing $\cos I_0$ uniformly distributed $\in [-1, 1]$ and drawing $e_{\rm
    out}$ from either a uniform distribution ($e_{\rm out} \in [0, 0.9]$) or a
    thermal distribution ($P(e_{\rm out}) \propto e_{\rm out}$, $e_{\rm out}
    \in [0, 0.9]$). In the top panel, we show the resulting merger
    probabilities for a uniform $e_{\rm out}$ distribution (black) and a
    thermal one (red). Green dots use the same initial conditions but use the
    GW-free merger criteria. The blue line instead samples a dense, uniform
    grid in $\cos I_0$ and $q \in [0.01, 1]$ using the GW-free simulations,
    confirming that the non-monotonicity of the other dots is due to the random
    sampling of $\cos I_0$. The middle panel shows the merger time for
    successful mergers (the median is denoted with the large black dot). The
    bottom panel shows the binary eccentricity in the LISA band
    ($0.1\;\mathrm{Hz}$) and in the LIGO band ($10 \;\mathrm{Hz}$).
    }\label{fig:popsynth}
\end{figure}
\begin{figure}
    \centering
    \includegraphics[width=\columnwidth]{../../scripts/octlk/1popsynth/a2eff5500.png}
    \caption{Same as Fig.~\ref{fig:popsynth} but for $a_{\rm out, eff} =
    5500\;\mathrm{AU}$. }\label{fig:popsynth5500}
\end{figure}

\begin{itemize}
    \item Note: LIGO-band eccentricities are small, but nonetheless show a
        small $q$ dependence, which may be detectable in the LISA band.
\end{itemize}

\subsection{Effect of Smaller Mass Ratios}

Intuitively, $q \to 0$ means $t_{\rm GW, 0} \to \infty$ and no mergers. This
would predict more large-mass-ratio mergers, opposite to our observed trend. Why
is this?

When orbital flips occur, $e_{\max}$ attains $e_{\lim}$. This induces one-shot
mergers when $e_{\lim} > e_{\rm os}$, or when
\begin{align}
    \p{\frac{a}{a_{\rm out, eff}}} \gtrsim{}&
        0.0118
        \p{\frac{a_{\rm out, eff}}{3600\;\mathrm{AU}}}^{-7/37}
        \p{\frac{m_{12}}{50M_{\odot}}}^{17/37}\nonumber\\
        &\times\p{\frac{30M_{\odot}}{m_3}}^{10/37}
        \p{\frac{q / (1 + q)^2}{1/4}}^{-2/37}.\label{eq:q_237}
\end{align}
This shows that $q$ must be changed by many orders of magnitude to transition
between a parameter regime where one-shot mergers occur at orbital flips to one
where such mergers fail to occur.

\section{Conclusion}

\begin{itemize}
    \item Primordial $q$ distribution in massive stellar binaries: tends to be
        skewed towards smaller $q$ to varying degrees \citep{sana2012binary,
        el2019discovery}.

    \item Eccentricity distribution of tertiary? Likely to be between uniform
        and thermal.

    \item LIGO O3a bounds on the BBH $q$ distribution: if $P(q) = q^{\beta_q}$,
        then $\beta_q > 0$ at $89\%$ or more, depending on the model
        \citep{LIGOO3a}, implying favoring equal-mass binaries.

    \item Our parameters are not particularly fine-tuned. If the inner binary is
        tighter or the perturber farther, then the MLL16 fits are better and
        give the same qualitative result. If the inner binary is looser, flybys
        become important. If the perturber is closer, then DA breaks down.

        However, if $e_{\lim} < e_{\rm os}$ [i.e.\ Eq.~\eqref{eq:q_237} is
        violated], the size of the effect can be much weaker, as most
        octupole-active systems no longer successfully merge within
        $10\;\mathrm{Gyr}$ unless they are near the quadrupole merger window.
        This implies the merger window only widens slightly even if
        $\epsilon_{\rm oct}$ is appreciable. Note that this does not change the
        trend of the effect, only its magnitude.
\end{itemize}

\section{Acknowledgements}\label{s:ack}

YS is supported by the NASA FINESST grant 19-ASTRO19-0041.%chktex 8

\bibliographystyle{mnras}
\bibliography{Su_EZLK}

\clearpage
\onecolumn

\appendix

\section{Origin of Octupole-Inactive Gap}\label{app:gap}

\begin{figure}
    \centering
    \includegraphics[width=\colummwidth]{../../scripts/octlk/1sweepbin_emax/1p2dist.png}
    \caption{Top panel is the same as the top panel of
    Fig.~\ref{fig:composites_1p2}. The bottom panel shows the range of
    oscillation in $K$, denoted by $K_{\min}$ and $K_{\max}$, for the same
    parameters. The critical $K = \eta / 2$ for orbit flipping is shown with the
    horizontal red line. It can be seen that when $K_{\min} < \eta / 2 <
    K_{\max}$ that $e_{\max} \simeq e_{\lim}$ in the top panel. It is therefore
    clear that the gap in $e_{\max}$ excitation is due to a limited range of
    oscillation in $K$.
    }\label{fig:kdist}
\end{figure}

\begin{itemize}
    \item Show the $K_{\min}$ and $K_{\max}$ plot, Fig.~\ref{fig:kdist}.

    \item Is because librating! example simulation where librating,
        Fig.~\ref{fig:nogw_circ}.

    \item Point out that the character of the circulation-libration transition
        changes when $\epsilon_{\rm oct}$ is substantial, to not only depend on
        $\omega$, Fig.~\ref{fig:dW}.
\end{itemize}

\begin{figure}
    \centering
    \includegraphics[width=\colummwidth]{../../scripts/octlk/1nogw_sims/1nogw_vec88.png}
    \caption{Example simulation where $\Omega_{\rm e}$ is primarily circulating,
    which suppresses the amplitude of oscillation of $K$. As a result, the orbit
    does not flip.}\label{fig:nogw_circ}
\end{figure}
\begin{figure}
    \centering
    \includegraphics[width=\colummwidth]{../../scripts/octlk/2dW_sweeps/2_dWsweeps6_2_dual.png}
    \caption{Plot of $\Delta \Omega_{\rm e}$, the change in the co-longitude of
    the inner eccentricity vector $\Omega_{\rm e}$ over a ZLK cycle, for
    different initial conditions and when octupole terms are off/on. In the top
    panel, the octupole terms are neglected, and $\omega_{\rm 0} = 0$ results in
    circulation ($\Delta \omega = 0$ and $\Delta \Omega_{\rm e} = 180^\circ$)
    while $\omega_{\rm 0} = \pi / 2$ results in libration, as expected. In the
    bottom panel, the octupole terms are included, and it is seen that the
    circulation-libration transition no longer depends on $\omega_{\rm 0}$ but
    instead on $I_0$. }\label{fig:dW}
\end{figure}

\label{lastpage} % chktex 24
\end{document}
