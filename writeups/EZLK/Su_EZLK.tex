% for i in 3qhist 1sweepbin/composite_tp 1nogw_sims/1nogw_vec 1sweepbin/composite_1p5dist 1sweepbin/composite_1p3dist 1sweepbin/composite_1p2dist 1sweepbin/composite_e91p5dist 1sweepbin/bindist 1sweepbin/total_merger_fracs 1sweepbin_simple/simpleouter 1popsynth/a2eff3600 1popsynth/a2eff5500 1popsynth/a2eff_nogw_lowq3600 1sweepbin/completeness 1sweepbin_emax_long/1p2dist 1nogw_sims/1nogw_vec88 2dW_sweeps/2_dWsweeps6_2_dual; do cp ../../scripts/octlk/$i.png .; done
    \documentclass[
        fleqn,
        usenatbib,
        % referee,
    ]{mnras}
    \usepackage{
        amsmath,
        amssymb,
        newtxtext,
        newtxmath,
        ae, aecompl,
        graphicx,
        booktabs,
        xcolor,
    }

    \newcommand*{\scinot}[2]{#1\times10^{#2}}
    \newcommand*{\rd}[2]{\frac{\mathrm{d}#1}{\mathrm{d}#2}}
    \newcommand*{\rtd}[2]{\frac{\mathrm{d}^2#1}{\mathrm{d}#2^2}}
    \newcommand*{\pd}[2]{\frac{\partial#1}{\partial#2}}
    \newcommand*{\ptd}[2]{\frac{\partial^2#1}{\partial#2^2}}
    % inline
    \newcommand*{\mdil}[2]{\mathrm{D}#1/\mathrm{D}#2}
    \newcommand*{\pdil}[2]{\partial#1/\partial#2}
    \newcommand*{\rdil}[2]{\mathrm{d}#1/\mathrm{d}#2}
    \newcommand*{\md}[2]{\frac{\mathrm{D}#1}{\mathrm{D}#2}}
    \newcommand*{\at}[1]{\left.#1\right|}
    \newcommand*{\abs}[1]{\left|#1\right|}
    \newcommand*{\ev}[1]{\left\langle#1\right\rangle}
    \newcommand*{\p}[1]{\left(#1\right)}
    \newcommand*{\s}[1]{\left[#1\right]}
    \newcommand*{\z}[1]{\left\{#1\right\}}
    \newcommand*{\bm}[1]{\mathbf{#1}}
    \newcommand*{\uv}[1]{\hat{\mathbf{#1}}}
    \DeclareMathOperator*{\med}{med}
    \DeclareMathOperator*{\erf}{erf}

    \colorlet{Corr}{red}

    \newlength{\colummwidth}
    \setlength{\colummwidth}{246.0pt} % columnwidth for reprint

\title[Mass Ratio Distribution]{The Mass Ratio Distribution of
Black Hole Mergers Induced by a Comparable Mass Tertiary}
\author[Y. Su et\ al.]{
Yubo Su,$^1$,
Bin Liu,$^{1,2}$,
Dong Lai$^1$
\\
$^1$ Cornell Center for Astrophysics and Planetary Science, Department of
Astronomy, Cornell University, Ithaca, NY 14853, USA\\
$^2$ Niels Bohr International Academy, Niels Bohr Institute, Blegdamsvej 17,
2100 Copenhagen, Denmark.
}

\date{Accepted XXX\@. Received YYY\@; in original form ZZZ}

\pubyear{2021}

\begin{document}\label{firstpage}
\pagerange{\pageref{firstpage}--\pageref{lastpage}}
\maketitle

\begin{abstract}
    Many proposed scenarios for black hole mergers consist of a tertiary
    companion that induces von Zeipel-Lidov-Kozai eccentricity cycles in the
    inner binary. An attractive feature of such mechanisms is the enhanced
    merger rate when the octupole-order corrections, also known as the eccentric
    Kozai mechanism, are important. This can be the case when the tertiary is of
    comparable mass to the binary components. However, in this work, we show
    that this enhancement is preferentially more efficient for binaries with
    smaller mass ratios, sometimes by as much as $20\times$. We use a
    combination of numerical and semi-analytical approaches to fully
    characterize these octupole-induced binary black hole mergers and provide
    analytical criteria for quickly estimating the strength of this enhancement.
    We show that the resulting observational signature is in tension with the
    mass ratio distribution obtained by the LIGO/VIRGO collaboration.
\end{abstract}

\begin{keywords}
binaries:close -- stars:black holes % chktex 8
\end{keywords}

\section{Introduction}\label{s:intro}

As LIGO/VIRGO continues to detect mergers of black hole (BH) binaries
\citep[e.g.][]{Abbott:2016blz, abbott2019binary, LIGOO3a}, it is important
to systematically study various formation channels of BH binaries and their
observable signatures. The canonical channel consists of isolated binary
evolution, in which mass transfer and friction in the common envelope phase
cause the binary orbit to decay sufficiently that it subsequently merges via
emission of gravitational waves (GW) within a Hubble time
\citep[e.g.][]{lipunov1997black, lipunov2017first, podsiadlowski2003formation,
belczynski2010effect, belczynski2016first, dominik2012double, dominik2013double,
dominik2015double}. BH binaries formed via isolated binary evolution are
generally expected to have large mass ratios, i.e.\ $q \equiv m_2 / m_1 > 0.5$,
where $m_1$ is the mass of the primary (more massive) binary component and $m_2$
is the mass of the secondary \citep{belczynski2016first, olejak2020}. On
the other hand, various flavors of dynamical formation channels of BH binaries
have also been studied. These involve either strong gravitational scatterings in
dense clusters \citep[e.g.][]{zwart1999black, o2006binary, miller2009mergers,
banerjee2010stellar, downing2010compact, ziosi2014dynamics, rodriguez2015binary,
samsing2017assembly, samsing2018black, rodriguez2018post, gondan2018eccentric}
or more gentle ``tertiary-induced mergers'' \citep[e.g.][]{ bin_misc1,
bin_misc2, bin_misc3, bin_misc4, bin_misc5, blaes2002kozai, miller2002four,
wen2003eccentricity, antonini2012secular, antonini2017binary, silsbee2017lidov,
bin1, bin2, randall2018induced, hoang2018black}. The dynamical formation
channels exhibit a range of distributions for the binary mass ratio
\citep[e.g.][]{silsbee2017lidov, fragione2019}.

GW observations of merging binaries are able to put constraints on the
individual component masses and therefore $q$. A histogram of the $q$ for all
binary BH mergers in the LIGO O3a dataset is shown in Fig.~\ref{fig:qhist}. A
clear preference for larger $q$ is evident, though events with small $q$ have
also been observed \citep[GW190814 and GW190412,][]{LIGOO3a}.

\begin{figure}
    \centering
    \includegraphics[width=\colummwidth]{3qhist.png}
    \caption{Histogram of the mass ratios $q \equiv m_2 / m_1$ of binary BH
    mergers in the O3a dataset, excluding the two NS-NS mergers but including
    GW190814, whose $2.5M_{\odot}$ secondary may be a BH \citep{LIGOO3a}. There
    is a preference for mergers with $q \gtrsim 0.5$.
    }\label{fig:qhist}
\end{figure}

In the tertiary-induced merger channel, different studies have obtained
different binary mass ratio distributions due to differing assumptions about the
properties of the inner BH binary at its formation. However, these studies
generally show that the mass ratio distribution for merging binaries is more
weighted towards smaller mass ratios than the initial distribution
\citep{silsbee2017lidov, fragione2019}. This implies that binaries with lower
mass ratios merge more readily via the tertiary-induced channel than do ones
with higher mass ratios. The origin of this preference is not well studied.

To understand this effect, we study the tertiary-induced merger channel where
the tertiary induces phases of extreme eccentricity in the inner binary via the
octupole-order von Zeipel-Lidov-Kozai (ZLK) effect, also sometimes known as the
eccentric Kozai mechanism \citep[e.g.][]{lithwick2011eccentric}. The
octupole-order terms are important when the system is only mildly hierarchical
and when the tertiary is on an eccentric orbit, which can be the case when the
tertiary is of mass comparable to the binary components, and they generally
enhance the rate of successful binary mergers \citep{ll18}. We show that this
enhancement is stronger for binaries with smaller mass ratios. We review
existing results in Section~\ref{s:background}, describe the basic behavior of
the triple system in Section~\ref{s:with_gw}, calculate the merger fractions as
a function of mass ratio in Section~\ref{s:merger_frac}, and conclude and
discuss in Section~\ref{s:conclusion}.

\section{Dynamics Without Gravitational Wave Radiation}\label{s:background}

Consider two BHs orbiting each other with masses $m_1$ and $m_2$ on a orbit with
semimajor axis $a$, eccentricity $e$, and angular momentum vector $\bm{L} \equiv
L\uv{L}$. Additionally, consider a third BH of mass $m_3$ orbiting this binary
with semimajor axis $a_{\rm out}$ and eccentricity $e_{\rm out}$, with angular
momentum vector $\bm{L}_{\rm out} \equiv L_{\rm out} \uv{L}_{\rm out}$. We
define $q \equiv m_2 / m_1$, the mass ratio, and $m_{12} = m_1 + m_2$, the total
mass of the inner binary. Our fiducial parameters are: $a = 100\;\mathrm{AU}$,
$a_{\rm out, eff} = 4500\;\mathrm{AU}$, $e_{\rm out} = 0.6$, $m_{12} =
50M_{\odot}$, $m_3 = 30M_{\odot}$. We take the initial inner binary eccentricity
to be $e_0 = 10^{-3}$.

To define the coordinate system, we choose the $\uv{z}$ axis pointing along the
total angular momentum, $\bm{L}_{\rm tot} = \bm{L} + \bm{L}_{\rm out}$. In this
coordinate system, we denote the longitude of the ascending node and argument of
pericenter of the inner and outer orbits by $\Omega$, $\omega$, $\Omega_{\rm
out}$, and $\omega_{\rm out}$ respectively. Note that, by conservation of
angular momentum, $\Omega_{\rm out} = \Omega + \pi$.

We study this system using the octupole-order, double-averaged vectorial ZLK
equations from \citet{LML15}, including general relativistic apsidal precession,
a first post-Newtonian order (1PN) effect. For the remainder of this section, we
review a few results in the absence of gravitational wave (GW) radiation, a
2.5PN effect (to be considered in Section~\ref{s:with_gw}). We group these
results by increasing order of approximation, starting by ignoring the
octupole-order effects entirely.

\subsection{Quadrupole-Order ZLK}

The primary feature of the quadrupole-order dynamics are cycles of the inner
binary's eccentricity from $e \ll 1$ to a consistent maximum $e_{\rm \max, q}$.
These cycles occur over characteristic time scale $\sim t_{\rm ZLK}$, where
\begin{equation}
    t_{\rm ZLK} = \frac{1}{n}\frac{m_{12}}{m_3}
            \p{\frac{a_{\rm out, eff}}{a}}^3,\label{eq:def_tzlk}
\end{equation}
where $n \equiv \sqrt{Gm_{12} / a^3}$ is the mean motion of the inner binary,
and $a_{\rm out, eff} \equiv a_{\rm out}\sqrt{1 - e_{\rm out}^2}$. During these
eccentricity cycles, there are two conserved quantities, the total energy and
the total orbital angular momentum. Through some manipulation, the total angular
momentum can be written in terms of the conserved quantity $K$ \citep{LML15},
where
\begin{equation}
    K \equiv j(e) \cos I - \eta e^2 / 2. \label{eq:def_K}
\end{equation}
Here, $j(e) \equiv \sqrt{1 - e^2}$, $I \equiv \cos^{-1}(\uv{L} \cdot
\uv{L}_{\rm out})$ is the mutual inclination, and $\eta$ is the ratio of the
magnitudes of the angular momenta at zero inner binary eccentricity:
\begin{equation}
    \eta \equiv \at{\frac{L}{L_{\rm out}}}_{e = 0}
        = \frac{\mu}{\mu_{\rm out}}\s{\frac{m_{12}a}
            {m_{123}a_{\rm out}\p{1 - e_{\rm out}^2}}}^{1/2},\label{eq:def_eta}
\end{equation}
where $\mu \equiv m_1m_2 / m_{12}$, $m_{123} = m_{12} + m_3$, and $\mu_{\rm out}
\equiv m_{12} m_3 / m_{123}$. Note that when $\eta = 0$, $K$ reduces to the
classical ``Kozai constant'', $K = j(e) \cos I$.

During these eccentricity cycles, the binary spends a fraction $\sim
j(e_{\rm \max, q})$ of each ZLK eccentricity cycle near $e_{\rm \max, q}$
\citep{anderson2016formation}. The exact value of $e_{\rm \max, q}$ depends on the
general relativistic apsidal precession, which induces precession of the
eccentricity vector $\bm{e}$ of the inner binary as
\begin{equation}
    \at{\rd{\bm{e}}{t}}_{\rm GR} = \Omega_{\rm GR}\uv{L} \times \bm{e}
        = \frac{3Gnm_{12}}{c^2aj^2(e)}\uv{L} \times \bm{e}.
\end{equation}
It is useful to quantify the relative strength of this precession via the
parameter
\begin{equation}
    \epsilon_{\rm GR} \equiv \p{\Omega_{\rm GR} t_{\rm ZLK}}_{e = 0}
        = \frac{3Gm_{12}}{c^2} \frac{m_{12}}{m_3}\frac{a_{\rm out, eff}^3}{a^4}.
\end{equation}
It can then be shown that $e_{\rm \max, q}$ and the initial mutual inclination $I_{\rm
0}$ satisfy \citep{LML15, anderson2016formation}:
\begin{align}
    \frac{3}{8}\frac{j^2(e_{\rm \max, q}) - 1}{j^2(e_{\rm \max, q})}\Big[&
        5\p{\cos I_0 + \frac{\eta}{2}}^2
        - \p{3 + 4\eta \cos I_0 + \frac{9}{4}\eta^2}j^2(e_{\rm \max, q})
            \nonumber\\
        &+ \eta^2 j^4(e_{\rm \max, q})
    \Big] + \epsilon_{\rm GR}\s{1 - 1 / j(e_{\rm \max, q})} = 0.\label{eq:emax_quad}
\end{align}
This only has solutions when $(\cos I_{\rm 0})_- \leq \cos I_{\rm 0} \leq (\cos
I_{\rm 0})_+$ where
\begin{equation}
    \p{\cos I_{\rm 0}}_{\pm} = \frac{1}{10}\p{-\eta \pm \sqrt{\eta^2 + 60 -
        20\epsilon_{\rm GR}}}.\label{eq:I0bounds}
\end{equation}
For $I_{\rm 0}$ outside of this range, there are no ZLK oscillations. This
condition reduces to the well-known $\cos^2 I_{\rm 0} \leq 3/5$ when $\eta =
\epsilon_{\rm GR} = 0$.

Additionally, the maximum value of $e_{\rm \max, q}$ across all $I_0$, denoted
$e_{\lim}$, occurs when $I_{\rm 0} = I_{\rm 0, \lim}$, where
\begin{equation}
    \cos I_{\rm 0, \lim} = \frac{\eta}{2}\s{\frac{4}{5}j^2(e_{\lim}) -
        1}.\label{eq:I0lim}
\end{equation}
Note that $I_{\rm 0, \lim} \geq 90^\circ$ with equality only when $\eta
= 0$. Substituting Eq.~\eqref{eq:I0lim} into Eq.~\eqref{eq:emax_quad}
gives
\begin{align}
    \frac{3}{8}\s{j^2(e_{\lim}) - 1}&\s{-3 + \frac{\eta^2}{4}
        \p{\frac{4}{5}j^2(e_{\lim}) - 1}}\nonumber\\
        &+ \epsilon_{\rm GR}\s{1 - 1 / j(e_{\lim})} = 0.
        \label{eq:def_elim}
\end{align}

\subsection{Octupole-Order ZLK, Test-Particle Limit}

\begin{figure}
    \centering
    \includegraphics[width=\colummwidth]{composite_tp.png}
    \caption{Plot of the maximum eccentricity achieved for a binary with
    starting mutual inclination $50^\circ < I_{\rm 0} < 130^\circ$ [spanning
    most of the ZLK-active region, Eq.~\eqref{eq:I0bounds}] for the fiducial
    parameters ($a = 100\;\mathrm{AU}$, $a_{\rm out, eff} = 3600\;\mathrm{AU}$,
    $m_{12} = 50M_{\odot}$, $m_3 = 30M_{\odot}$) where $e_{\rm out} = 0.6$ and
    $q \equiv m_2 / m_1 = 0.01$, i.e.\ in the test-particle regime $\eta \ll 1$
    [Eq.~\eqref{eq:def_eta}]. Simulations are run for $2000t_{\rm ZLK}$
    [Eq.~\eqref{eq:def_tzlk}] including octupole-order terms, for $1000$ initial
    inclinations. Each $I_{\rm 0}$ is further simulated five times with the
    initial orbital elements $\omega$, $\omega_{\rm out}$, and $\Omega =
    \Omega_{\rm out} - \pi$ chosen randomly $\in [0, 2\pi)$ % chktex 9
    for each simulation. The dotted black line shows $e_{\rm \max, q}$ given by
    Eq.~\eqref{eq:emax_quad}, and $e_{\lim}$ [Eq.~\eqref{eq:def_elim}] is shown
    as the horizontal red line. The vertical purple lines denote the boundary of
    the predicted octupole-active inclinations in the test-particle regime using
    the fitting formula from \citet{MLL16} [Eq.~\eqref{eq:I_oct_MLL}].
    }\label{fig:composite_tp}
\end{figure}

To begin to understand the ZLK dynamics at octupole order, we first restrict our
discussion to the well-studied test-particle limit, where $m_2 = \eta = 0$. In
this limit, the equations of motion are symmetric about $I_{\rm 0} = 90^\circ$,
so $I_{\rm 0, \lim} = 90^\circ$. The strength of the octupole-order effects is
quantified by the dimensionless parameter \citep{katz2011long,
lithwick2011eccentric, naoz2012formation}
\begin{equation}
    \epsilon_{\rm oct}^{\rm (tp)} = \frac{a}{a_{\rm
        out}} \frac{e_{\rm out}}{1 - e_{\rm out}^2}.\label{eq:eps_oct_tp}
\end{equation}

When $\epsilon_{\rm oct} > 0$, $K = j(e) \cos I$ is no longer conserved, and the
system becomes non-integrable and chaotic \citep{ford2000secular, katz2011long,
lithwick2011eccentric}. As a result, $e_{\rm \max, q}$ is no longer consistent
among cycles, and orbit flips become possible, where $I$ flips from prograde ($I
< 90^\circ$) to retrograde ($I > 90^\circ$) and back. During these orbit flips,
$e$ generally attains the limiting value $e_{\lim}$ even when $I_{\rm 0} \neq
I_{\rm 0, \lim}$ \citep{lithwick2011eccentric, LML15}. The origin of these orbit
flips is due to oscillations in $K$ over long timescales $\gg t_{\rm ZLK}$, as
if $K$ changes signs, the sign of $\cos I$ must also change, corresponding to an
orbit flip. \citet{katz2011long} calculate the amplitude of these oscillations
analytically, assuming $\omega$ is circulating, and use their calculation to
predict a a range of inclinations $I_{\rm flip, -} \lesssim I_0 \lesssim I_{\rm
flip, +}$ over which orbit flips occur and $e_{\lim}$ is attained; we refer to
this as an \emph{octupole-active window}. However, their calculation is only
accurate for $\epsilon_{\rm oct} \ll 1$; a more general fitting formula is given
by \citep{MLL16}
\begin{equation}
    \cos^2 I_{\rm flip, \pm} = \begin{cases}
        0.26\p{\frac{\epsilon_{\rm oct}^{\rm(tp)}}{0.1}}
            - 0.536\p{\frac{\epsilon_{\rm oct}^{\rm(tp)}}{0.1}}^2\\
            \quad + 12.05\p{\frac{\epsilon_{\rm oct}^{\rm(tp)}}{0.1}}^3
            - 16.78\p{\frac{\epsilon_{\rm oct}^{\rm(tp)}}{0.1}}^4
            & \epsilon_{\rm oct}^{\rm(tp)} \lesssim 0.05,\\
        0.45 & \epsilon_{\rm oct}^{\rm(tp)} \gtrsim 0.05.
    \end{cases} \label{eq:I_oct_MLL}
\end{equation}
The accuracy of this fitting formula is displayed in
Fig.~\ref{fig:composite_tp}, where we evolve a system for which $m_2 \ll m_1$
for $2000t_{\rm ZLK}$ using a range of $I_0$. We record the maximum eccentricity
$e_{\max}$ obtained by the inner binary\footnote{Note that we use $e_{\rm \max,
q}$ to denote the eccentricity maximum for a single quadrupole-order ZLK cycle,
and $e_{\max}$ to denote the maximum eccentricity over longer periods of time,
e.g.\ $t_{\rm ZLK, oct}$.},
and find that it attains $e_{\lim}$ if and
only if $I_0$ is within the octupole-active window as given by
Eq.~\eqref{eq:I_oct_MLL}.

Finally, the characteristic timescale of these orbit flips is approximately
\citep{antognini2015timescales}
\begin{equation}
    t_{\rm ZLK, oct} = t_{\rm ZLK}\frac{128\sqrt{10}}{
        15\pi\sqrt{\epsilon_{\rm oct}^{\rm(tp)}}}.\label{eq:def_tzlkoct}
\end{equation}

\subsection{Octupole-Order ZLK, General Masses}\label{ss:oct_gen}

\begin{figure}
    \centering
    \includegraphics[width=\colummwidth]{1nogw_vec.png}
    \caption{Example octupole-order, finite-mass simulation showing orbit
    flipping. We use mostly the same system parameters as in
    Fig.~\ref{fig:composite_tp} but have $e_{\rm out} = 0.6$, $q = 0.2$, and
    $I_0 = 93.5^\circ$. The four panels show the inner orbit eccentricity, the
    mutual inclination, $K$ [Eq.~\eqref{eq:def_K}], and the azimuthal angle of
    the inner eccentricity vector $\Omega_{\rm e} \equiv \tan^{-1}(e_y / e_x)$.
    By comparing the second and third panels, we see that orbit flips occur when
    $K$ crosses the dotted line $K = K_{\rm c} \equiv -\eta / 2$. Furthermore,
    note that the times when the angle $\Omega_{\rm e}$ changes slowly (which
    occurs when $\omega$ is circulating) are when $K$ oscillates sinusoidally,
    similar to the test-particle regime \citep{katz2011long}.
    }\label{fig:nogw_fiducial}
\end{figure}
When $m_1$ and $m_2$ are comparable, Eq.~\eqref{eq:eps_oct_tp} generalizes to
\citep{LML15, anderson2016formation, ll18}
\begin{equation}
    \epsilon_{\rm oct} = \frac{m_1 - m_2}{m_{12}} \frac{a}{a_{\rm out}}
        \frac{e_{\rm out}}{1 - e_{\rm out}^2}.\label{eq:eps_oct}
\end{equation}
As we will see, the comparable-mass regime exhibits much new behavior compared
to the test-particle regime \citep[see also][]{rodet_inprep}, but some results
from the previous section still hold. In particular, $K$ still slowly fluctuates
when $\omega$ is circulating, and orbit flips occur when $K$ crosses $K = K_{\rm
c} \equiv -\eta / 2$ [by Eq.~\eqref{eq:def_K}, this is where $\cos I$ changes
signs]; see Fig.~\ref{fig:nogw_fiducial} for an example simulation showing this
behavior in the comparable-mass case.

However, in this regime, the dynamics are no longer symmetric about $I_0 =
90^\circ$. This significantly complicates the simple octupole-active inclination
window shown in Fig.~\ref{fig:composite_tp}, which can be contrasted with the
analogous top panels of
Figs.~\ref{fig:composite_dist}--\ref{fig:composite_bindist} that illustrate the
behavior of the maximum inner binary eccentricity for a variety of system
parameters. The system always has an octupole-active window in the vicinity of
$I_{\rm 0, \lim}$, but only sometimes achieves $e_{\lim}$ for prograde
inclinations. In fact, the $e_{\max}$ curve is clearly double-valued, such that
even when $\epsilon_{\rm oct}$ is very large (Fig.~\ref{fig:composite_1p2}),
there are inclination ranges for which $e_{\lim}$ is only attained for some
fraction of all possible initial conditions of the system. In between these two
inclination ranges, there is a persistent range of inclinations $I_{\rm 0}
\approx 90^\circ$ for which $e_{\max} = e_{\rm \max, q}$ in spite of a substantial
$\epsilon_{\rm oct}$. We refer to this as the ``octupole-inactive gap'' and
further investigate its origin in Appendix~\ref{app:gap}.

\section{Dynamics With Gravitational Wave Radiation}\label{s:with_gw}

Emission of gravitational waves (GWs) also affects the evolution of $\bm{L}$ and
$\bm{e}$, which can be incorporated into the equations of motion
\citep{peters1964, ll18}. The associated orbital and eccentricity decay rates
are
\begin{align}
    \at{\frac{1}{a}\rd{a}{t}}_{\rm GW} &\equiv \frac{1}{t_{\rm GW}}\nonumber\\
        &= -\frac{64}{5}\frac{G^3 \mu m_{12}^2}{c^5a^4j^7(e)}
            \p{1 + \frac{73}{24}e^2 + \frac{37}{96}e^4}\label{eq:def_tgw},\\
    \at{\rd{e}{t}}_{\rm GW} &= -\frac{304}{15}\frac{G^3 \mu m_{12}^2}{c^5a^4}
        \frac{1}{j^{5}(e)}\p{1 + \frac{121}{304}e^2}\label{eq:dedt_gw}.
\end{align}

\subsection{Merger Windows and Probability}

If the eccentricity maxima of the ZLK cycles are sufficiently large, GW
radiation at $e \simeq e_{\rm \max, q}$ can be sufficiently enhanced that even very
wide binaries such as those studied here ($a = 100\;\mathrm{AU}$) can be induced
to merge within $10\;\mathrm{Gyr}$, a Hubble time \citep{ll18, LL19}. For the
same range of initial conditions considered in Section~\ref{s:background}, we
simulate the octupole-order ZLK equations and include GW radiation to understand
what systems successfully merge in under $10\;\mathrm{Gyr}$. To be precise, a
system is considered to successfully merge if it reaches $a = 0.5\;\mathrm{AU}$.
The resulting merger times $T_{\rm m}$ as a function of $I_0$ are shown in the
middle panels of Figs.~\ref{fig:composite_dist}--\ref{fig:composite_bindist}.

When including the chaotic octupole-order ZLK terms, the system's starting
inclination alone may not be sufficient to determine whether it merges within a
Hubble time (e.g.\ if the system is in the range where the $e_{\max}$ behavior
is double-valued, as mentioned in Section~\ref{ss:oct_gen}). Instead, multiple
simulations using different $\omega$, $\omega_{\rm out}$, and $\Omega$ can be
used to compute a merger probability
\begin{equation}
    P_{\rm merge}\p{I_0; q, e_{\rm out}} = P\p{T_{\rm m} < 10\;\mathrm{Gyr}}
        \label{eq:def_pmerge}
\end{equation}
To explore $P_{\rm merge}$ with sufficient accuracy and resolution, we first run
simulations over a coarse grid in $I_0$. For each range of $I_0$ where mergers
are possible, we then sample the range of inclinations finely and run $20$
simulations for each $I_0$ ($5$ for the more computationally expensive $a =
10\;\mathrm{AU}$ case in Fig.~\ref{fig:composite_bindist}). The resulting
$P_{\rm merge}$  are shown in the bottom panels of
Figs.~\ref{fig:composite_dist}--\ref{fig:composite_bindist}. The
octupole-inactive gaps first identified in Section~\ref{ss:oct_gen} are also
present as gaps in $P_{\rm merge}$.

There are two general categories of BH mergers in our simulations: (i) the
binary merges after a single large burst of GW radiation during an
extreme-eccentricity ZLK cycle (``one-shot merger'') or (ii) the binary merges
gradually by emitting a varying amount of GW radiation at every eccentricity
maximum (``smooth merger''). One-shot mergers merge upon reaching $e_{\lim}$,
and so merge with characteristic $T_{\rm m} \sim t_{\rm ZLK, oct}$, while smooth
mergers exhibit a broad spectrum of merger times $\gtrsim t_{\rm ZLK, oct}$. For
retrograde $I_0$ in Figs.~\ref{fig:composite_dist}--\ref{fig:composite_e91p5},
there are comparatively few mergers with $T_{\rm m} \gtrsim 1\;\mathrm{Gyr}$, so
the fiducial parameter regime tends to merge via one-shot mergers. On the other
hand, prograde inclinations in Figs.~\ref{fig:composite_1p3}
and~\ref{fig:composite_bindist} produce a broad range of $T_{\rm m}$ and
represent an example of smooth mergers.

\subsection{Semi-Analytic Calculation of Merger
Probability}\label{ss:nogw_merger}

It is clear that features in the $e_{\max}$ curve, obtained without GW
radiation, are correlated with behavior in the $P_{\rm merge}$ and $T_{\rm m}$
plots, obtained with GW radiation. Here, we further develop this connection and
show that the non-dissipative simulations can indeed be used to predict the
outcomes of simulations with GW dissipation rather reliably.

As mentioned above, we observe both one-shot and smooth mergers in our
simulations. Towards understanding the one-shot mergers, we first define $e_{\rm
os}$ to be the $e_{\rm \max, q}$ required to dissipate an order-unity fraction of
the binary's orbital energy via GW emission in a single ZLK cycle. Recalling
that a binary spends a fraction $\sim j(e_{\rm \max, q})$ of each ZLK cycle near
$e_{\rm \max, q}$, we write
\begin{align}
    j\p{e_{\rm os}}\at{\rd{\ln a}{t}}_{e = e_{\rm os}} &\sim
        \frac{1}{t_{\rm ZLK}},\\
    j^6(e_{\rm os})
        &\equiv \frac{842}{15}
            \frac{G^3 \mu m_{12}^3}{m_3c^5a^4n}
            \p{\frac{a_{\rm out, eff}}{a}}^3.
            \label{eq:def_e_os}
\end{align}
We have approximated $e_{\rm os} \approx 1$ in Eq.~\eqref{eq:def_tgw}. Then, if
a system satisfies $e_{\max} > e_{\rm os}$, then it is expected attain a
sufficiently large eccentricity to undergo a one-shot merger.

Towards understanding smooth mergers, we require a characteristic eccentricity
that captures GW emission over many ZLK cycles. We define $e_{\rm eff}$ as an
effective $e_{\rm \max, q}$. In other words,
\begin{align}
    \ev{\rd{\ln a}{t}} &\approx -\frac{1}{t_{\rm GW, 0}}
            \ev{\frac{1 + 73e_{\rm \max, q}^2/24 + 37e_{\rm \max, q}^4/96}
                {j^6(e_{\rm \max, q})}}\nonumber\\
        &\equiv -\frac{421/96}{t_{\rm GW, 0}}\frac{1}{j^6(e_{\rm eff})},
        \label{eq:def_e_eff}
\end{align}
where $t_{\rm GW, 0} = \p{t_{\rm GW}}_{\rm e = 0}$ [see Eq.~\eqref{eq:def_tgw}].
Here, the angle brackets denote averaging over many ZLK cycles; we average over
the full duration of the non-dissipative simulations, $2000 t_{\rm ZLK}$. Define
next the critical effective eccentricity $e_{\rm eff, c}$ such that the inspiral
time is a Hubble time:
\begin{equation}
    \ev{\rd{\ln a}{t}} \equiv \frac{421/96}{t_{\rm GW, 0}j^6(e_{\rm eff, c})}
        = \p{10\;\mathrm{Gyr}}^{-1}. \label{eq:def_e_eff_c}
\end{equation}
If a system is evolved using the non-dissipative equations of motion and
satisfies $e_{\rm eff} > e_{\rm eff, c}$, then it is expected to successfully
undergo a smooth merger within a Hubble time.

Finally, a system can be predicted to merge successfully if it satisfies either
the one-shot or smooth merger criteria, so we obtain the semi-analytic merger
probability:
\begin{equation}
    P_{\rm merge, SA} = P\p{e_{\rm eff} > e_{\rm eff, c} \;\;\text{or}\;\;
        e_{\max} > e_{\rm os}}.\label{eq:def_pmerge_sa}
\end{equation}
In Figs.~\ref{fig:composite_dist}--\ref{fig:composite_bindist}, we compare
$P_{\rm merge}$ obtained via dissipative simulations with $P_{\rm merge, SA}$
obtained via Eq.~\eqref{eq:def_pmerge_sa}. We generally observe good
agreement, though $P_{\rm merge, SA}$ slightly but systematically underpredicts
$P_{\rm merge}$. This is due to the restricted integration time of $2000 t_{\rm
ZLK} \approx 3\;\mathrm{Gyr}$ used for the non-dissipative simulations. We see
that if an even shorter integration time of $500 t_{\rm ZLK}$ is used for the
non-dissipative simulations \citep[as is used in][]{oconnor_wd}, the
underprediction is even more severe. A detailed discussion of this
underprediction can be found in Section~\ref{ss:completeness}.

We make an important observation about our one-shot merger criterion: if
$e_{\lim} \gtrsim e_{\rm os}$ for a particular system architecture, then all
initial conditions that undergo orbit flips (i.e.\ in an octupole-active window)
also execute one-shot mergers. When $e_{\lim} \approx 1$,
Eq.~\eqref{eq:def_elim} reduces to $j(e_{\lim}) \approx 4\epsilon_{\rm GR} / 9$,
which lets us rewrite the constraint $e_{\lim} \gtrsim e_{\rm os}$ as
\begin{align}
    \p{\frac{a}{a_{\rm out, eff}}} \gtrsim{}&
        0.0149
        \p{\frac{a_{\rm out, eff}}{3600\;\mathrm{AU}}}^{-7/37}
        \p{\frac{m_{12}}{50M_{\odot}}}^{17/37}\nonumber\\
        &\times\p{\frac{30M_{\odot}}{m_3}}^{10/37}
        \p{\frac{q / (1 + q)^2}{1/4}}^{-2/37}.\label{eq:q_237}
\end{align}
For the system architecture considered in
Figs.~\ref{fig:composite_dist}--\ref{fig:composite_e91p5}, this constraint is
indeed satisfied, and we see indeed that wherever the top panel suggests orbit
flipping ($e_{\max} = e_{\lim}$), the bottom panel shows $f_{\rm merge} \approx
1$. When this criterion is not satisfied, one-shot mergers are not
possible, and $P_{\rm merge}$ is generally only nonzero for a small window about
$I_{\rm 0, \lim}$.

A few other observations about Eq.~\eqref{eq:def_pmerge_sa} can be made.
First, it explains why some prograde systems merge probabilistically ($0 <
P_{\rm merge} < 1$), such as in Fig.~\ref{fig:composite_1p3}: for some
inclinations, the double-valued feature in the $e_{\max}$ curve pointed out in
the previous section represents a sub-population of systems that do not satisfy
Eq.~\eqref{eq:def_pmerge_sa}, and for other inclinations, $e_{\rm eff} \simeq
e_{\rm eff, c}$, so the detailed merger outcome depends on the initial
conditions. Second, $e_{\max} > e_{\rm os}$ often ensures $e_{\rm eff} > e_{\rm
eff, c}$ in practice, as the averaging in Eq.~\eqref{eq:def_e_eff} is heavily
weighted towards extreme eccentricities. As such, $e_{\rm eff} > e_{\rm eff, c}$
alone is often a sufficient condition in Eq.~\eqref{eq:def_pmerge_sa}.

\begin{figure}
    \centering
    \includegraphics[width=\colummwidth]{composite_1p5dist.png}
    \caption{A plot of the system dynamics for the fiducial parameter regime ($a
    = 100\;\mathrm{AU}$, $a_{\rm out, eff} = 3600\;\mathrm{AU}$, $m_{12} =
    50M_{\odot}$, $m_3 = 30M_{\odot}$) with $q = 0.5$ and $e_{\rm out} = 0.6$.
    Here, $\eta \approx 0.087$ is nonnegligible, and $\epsilon_{\rm oct} \approx
    0.007$. In the top panel, for each of $1000$ initial inclinations, we choose
    $5$ different random $\omega$, $\omega_{\rm out}$, and $\Omega$ as initial
    conditions, then run for $2000t_{\rm ZLK}$ without gravitational wave
    radiation. The effective eccentricity $e_{\rm eff}$
    [Eq.~\eqref{eq:def_e_eff}; green dots] as well as the maximum eccentricity
    $e_{\max}$ (blue dots) over this period are displayed. For comparison,
    $e_{\rm eff, c}$ [Eq.~\eqref{eq:def_e_eff_c}] is given by the horizontal
    green dashed line, $e_{\rm os}$ [Eq.~\eqref{eq:def_e_os}] is shown in the
    horizontal blue line, and $e_{\lim}$ [Eq.~\eqref{eq:def_elim}] is shown in
    the horizontal red dashed line. The vertical purple lines are the fitting
    formula of \citet{MLL16} (we evaluate for $\epsilon_{\rm oct}^{\rm (tp)} =
    \epsilon_{\rm oct}$) and no longer accurately describe the
    $e_{\lim}$-attaining inclination window. The black dashed line is given by
    Eq.~\eqref{eq:emax_quad}. In the middle panel, we show the binary merger
    times when including gravitational wave radiation and using the same range
    of initial conditions. Simulations are terminated when $T_{\rm m} >
    10\;\mathrm{Gyr}$ and marked as unsuccessful mergers. The horizontal dashed
    line denotes $t_{\rm ZLK}$ [Eq.~\eqref{eq:def_tzlk}] while the horizontal
    dash-dotted line indicates $t_{\rm ZLK, oct}$ [Eq.~\eqref{eq:def_tzlkoct};
    we evaluate for $\epsilon_{\rm oct}^{\rm (tp)} = \epsilon_{\rm oct}$]. Here,
    each $I_{\rm 0}$ is run $20$ times, so we can estimate a merger probability
    $P_{\rm merge}$ [Eq.~\eqref{eq:def_pmerge}] for each $I_{\rm 0}$, which is
    shown as the black line in the bottom panel. As described in
    Section~\ref{ss:nogw_merger}, the merger probability of systems can be
    predicted semi-analytically using the results of the top panel and
    Eq.~\eqref{eq:def_pmerge_sa}, and is denoted $P_{\rm merge, SA}$. The thick
    green line shows the prediction when using an integration time of
    $2000 t_{\rm ZLK} \approx 3\;\mathrm{Gyr}$ for the non-dissipative
    simulations, and thin green line shows the prediction using an integration
    time of $500 t_{\rm ZLK}$. The agreement with $P_{\rm merge}$ is good and
    improves when using the longer integration time.
    }\label{fig:composite_dist}
\end{figure}
\begin{figure}
    \centering
    \includegraphics[width=\colummwidth]{composite_1p3dist.png}
    \caption{Same as Fig.~\ref{fig:composite_dist} but for $q = 0.3$, $\eta
    \approx 0.07$, and $\epsilon_{\rm oct} \approx 0.011$. A ``gap'' in merger
    fractions for $I_0 \approx 90^\circ$ is evident, which is discussed further
    in Appendix~\ref{app:gap}. There is a substantial range of prograde
    inclinations for which the $e_{\max}$ behavior is double-valued and for
    which the merger fraction $0 < f_{\rm merge} < 1$; the origin of this
    behavior is unknown. Over this inclination range, the improvement when using
    an integration time of $2000 t_{\rm ZLK}$ for $P_{\rm merge, SA}$ as opposed
    to $500 t_{\rm ZLK}$ is very pronounced. }\label{fig:composite_1p3}
\end{figure}
\begin{figure}
    \centering
    \includegraphics[width=\colummwidth]{composite_1p2dist.png}
    \caption{Same as Fig.~\ref{fig:composite_dist} but for $q = 0.2$, $\eta
    \approx 0.054$, and $\epsilon_{\rm oct} \approx 0.014$. Note that a large
    range of prograde $I_0$ are able to guarantee mergers, unlike in
    Figs.~\ref{fig:composite_dist} and~\ref{fig:composite_1p3}.
    }\label{fig:composite_1p2}
\end{figure}
\begin{figure}
    \centering
    \includegraphics[width=\colummwidth]{composite_e91p5dist.png}
    \caption{Same as Fig.~\ref{fig:composite_dist} but with $q = 0.5$ and
    instead $e_{\rm out} = 0.9$ while holding $a_{\rm out, eff} =
    3600\;\mathrm{AU}$ constant. Here, $\eta \approx 0.12$ while $\epsilon_{\rm
    oct} \approx 0.019$. Note that even though $\epsilon_{\rm oct}$ is larger
    here than in Fig.~\ref{fig:composite_1p2}, no mergers are possible here with
    a prograde perturber. This is because $\eta$ is more than twice as large.
    }\label{fig:composite_e91p5}
\end{figure}
\begin{figure}
    \centering
    \includegraphics[width=\colummwidth]{bindist.png}
    \caption{Same as Figs.~\ref{fig:composite_dist}--\ref{fig:composite_1p2}
    but for a compact inner binary; the parameters are $a = 10\;\mathrm{AU}$,
    $a_{\rm out, eff} = 700\;\mathrm{AU}$, $m_{12} = 50M_{\odot}$, $m_3 =
    30M_{\odot}$, and $e_{\rm out} = 0.9$, $q = 0.4$. Note that when the
    perturber is prograde ($I_0 < 90^\circ$), $e_{\max} <
    e_{\rm os}$ but $e_{\rm eff} > e_{\rm eff, c}$, which predicts that systems
    are able to merge as smooth octupole-induced mergers. This prediction is
    in qualitative agreement with the dissipative simulations (see bottom
    panel). }\label{fig:composite_bindist}
\end{figure}

\section{Merger Fraction as a Function of Mass Ratio}\label{s:merger_frac}

Using the merger probability $P_{\rm merge}\p{I_0; q, e_{\rm out}}$, it is
natural to next seek the mass ratio distribution of merging BH binaries via the
comparable-mass tertiary-induced channel. However, this relies on many uncertain
quantities, such as the primordial $q$ distribution in BH binaries. Instead, we
aim to characterize the effect described in the preceeding sections by studying
the fraction of BH binaries that successfully merge under various conditions,
which we call the merger fraction.

\subsection{Merger Fraction for Fixed Tertiary Eccentricity}

We first consider the simplest case where $e_{\rm out}$ is fixed at a few
specific values and compute the merger fraction as a function of the mass ratio
$q$. We let $\cos I_0$ be drawn uniformly from the range $[-1, 1]$, let
$\omega$, $\omega_{\rm out}$, and $\Omega$ be drawn from the range $[0,
2\pi)$ as before % chktex 9
and adopt the fiducial parameters. The merger fraction is defined as:
\begin{equation}
    f_{\rm merge}\p{q, e_{\rm out}} \equiv
        \frac{1}{2}\int\limits_{-1}^1P_{\rm merge}\p{I_0; q, e_{\rm out}}
            \;\mathrm{d}\cos I_0.\label{eq:def_fmerge}
\end{equation}
This is plotted for various choices of $q$ and $e_{\rm out}$ as the solid dots in
Fig.~\ref{fig:total_merger_fracs}, where each point represents the integration
of a $P_{\rm merge}$ curve like the black lines in the bottom panels of
Figs.~\ref{fig:composite_dist}--\ref{fig:composite_e91p5}. For each simulation,
we also use semi-analytic criteria introduced in Section~\ref{ss:nogw_merger} to
predict the outcome and generate a merger fraction. This is computed by using
$P_{\rm merge, SA}$ as the integrand in Eq.~\eqref{eq:def_fmerge} and is shown
as the crosses and dashed lines in Fig.~\ref{fig:total_merger_fracs}. It is
clear that the merger fractions increase steeply for smaller $q$ and that the
semi-analytic merger fraction tracks $f_{\rm merge}$ well. Choosing instead $a =
50\;\mathrm{AU}$ and $a_{\rm out, eff} = 1800\;\mathrm{AU}$ does not
significantly change Fig.~\ref{fig:total_merger_fracs}. This is because
$\epsilon_{\rm oct}$ and $\eta$ are unchanged, while Eq.~\eqref{eq:q_237} only
changes very little.

If instead, we choose $a = 50\;\mathrm{AU}$ but keep $a_{\rm out, eff} =
3600\;\mathrm{AU}$, we obtain Fig.~\ref{fig:sweepbin_simpleouter}. According to
Eq.~\eqref{eq:q_237}, this new parameter regime no longer satisfies $e_{\lim}
\gtrsim e_{\rm os}$, so the merger fraction is expected to diminish strongly and
vary much more weakly with $q$, as one-shot mergers are no longer possible, in
agreement with the results.

In the right panel of Fig.~\ref{fig:total_merger_fracs}, we see that the merger
fractions for the three $e_{\rm out}$ values overlap for sufficiently small
$\epsilon_{\rm oct}$. This implies that $f_{\rm merge}$ depends only on
$\epsilon_{\rm oct}$ in this regime, and not the values of $q$ and $e_{\rm out}$
independently. From Fig.~\ref{fig:composite_dist}, we see that this suggests
that the size of the retrograde merger window only depends on $\epsilon_{\rm
oct}$, much like Eq.~\eqref{eq:I_oct_MLL} in the test-particle limit.
However, for larger $\epsilon_{\rm oct}$, the three curves diverge. This is
attributed to their different $\eta$ values: while for sufficiently small
$\epsilon_{\rm oct}$, no prograde initial inclinations successfully merge (e.g.\
Fig.~\ref{fig:composite_dist}), the size of the octupole-inactive gap is
different for different $\eta$, so a prograde merger window forms at smaller
values of $\epsilon_{\rm oct}$ if $\eta$ is also smaller ($e_{\rm out}$ is
larger). This again illustrates the importance of the octupole-inactive gap,
which we comment on in Appendix~\ref{app:gap}.

\begin{figure*}
    \centering
    \includegraphics[width=\textwidth]{total_merger_fracs.png}
    \caption{From Figs.~\ref{fig:composite_dist}--\ref{fig:composite_e91p5},
    we can compute the total merger fraction in the presence of GW radiation
    assuming $\cos I_0$ is uniformly distributed $\in [-1, 1]$ for the fiducial
    parameter regime. We do this for three values of $e_{\rm out}$ and six
    values of $q$ and show the results with solid dots. The X's show the results
    when using Eq.~\eqref{eq:def_pmerge_sa} and an integration time of $2000
    t_{\rm ZLK}$; good agreement is observed. }\label{fig:total_merger_fracs}
\end{figure*}
\begin{figure*}
    \centering
    \includegraphics[width=\textwidth]{simpleouter.png}
    \caption{Same as Fig.~\ref{fig:total_merger_fracs} but for $a =
    50\;\mathrm{AU}$; the semi-analytic merger fractions have been omitted for
    clarity. Note that the $f_{\rm merge}$ enhancement for smaller $q$ is
    smaller, as the condition Eq.~\eqref{eq:q_237} is no longer satisfied.
    }\label{fig:sweepbin_simpleouter}
\end{figure*}

\subsection{Merger Fraction for a Distribution of Tertiary Eccentricities}

We can also draw $e_{\rm out} \in [0, 0.9]$ with both a uniform probability
distribution and a thermal one $P(e_{\rm out}) \propto e_{\rm out}$, and examine
$f_{\rm merge}$ as a function of $q$, where now
\begin{equation}
    f_{\rm merge}(q) \propto
        \int\limits_{0}^{0.9}\int\limits_{-1}^1
            P_{\rm merge}\p{I_0; q, e_{\rm out}} P(e_{\rm out})
            \;\mathrm{d}\cos I_0\;\mathrm{d}e_{\rm out}.
\end{equation}
The results are shown in Fig.~\ref{fig:popsynth}, where each $q$ has $1000$
realizations using random $e_{\rm out}$, $\cos I_0$, $\omega$, $\omega_{\rm
out}$, and $\Omega$. If $e_{\rm out}$ is thermally distributed (red), binaries
with smaller $q$ are even more likely to merge, since $\epsilon_{\rm oct}$ tends
to be larger. To understand the impact of our random sampling, we also compute a
merger fraction using the semi-analytic merger criterion of
Eq.~\eqref{eq:def_pmerge_sa} while using a dense grid of initial conditions
(uniformly sampling $17$ values of $e_{\rm out}$ and $41$ values of $\cos I_0$),
shown as the blue dotted line. Again, the semi-analytic curve tracks the
dissipative results well. The distributions and medians of the merger time and
eccentricity in the LISA and LIGO bands are also shown. For the LISA and LIGO
band eccentricities, the inner binaries are evolved from when they reach
$0.5\;\mathrm{AU}$ to physical merger using
Eqs.~(\ref{eq:def_tgw}--\ref{eq:dedt_gw}). While all of these eccentricities are
quite small, this has to do with our using the double-averaged equations of
motion. Both the single-averaged and the full N-body equations of motion produce
larger eccentricities in the LISA and LIGO bands \citep{LL19}.

For comparison, we also show the results when $a_{\rm out, eff} =
5500\;\mathrm{AU}$, with all other parameters unchanged, in
Fig.~\ref{fig:popsynth5500}. While $f_{\rm merge}$ is lower than it is for
$a_{\rm out, eff} = 3600\;\mathrm{AU}$, there is still a large increase between
large and small $q$. This is expected, since $e_{\lim} > e_{\rm os}$ is still
satisfied [Eq.~\eqref{eq:q_237}].
\begin{figure}
    \centering
    \includegraphics[width=\columnwidth]{a2eff3600.png}
    \caption{Merger fractions with the fiducial parameters obtained by randomly
    drawing $\cos I_0$ uniformly distributed $\in [-1, 1]$ and drawing $e_{\rm
    out}$ from either a uniform distribution ($e_{\rm out} \in [0, 0.9]$; black)
    or a thermal distribution [$P(e_{\rm out}) \propto e_{\rm out}$; red]. The
    blue dotted line instead samples a dense, uniform grid in $\cos I_0$,
    $e_{\rm out}$, and $q \in [0.2, 1]$ using Eq.~\eqref{eq:def_pmerge_sa} for
    an integration time of $2000 t_{\rm ZLK}$, which is in good agreement with
    the dissipative simulations while being smoother. The middle panel shows the
    merger time for successful mergers (the median is denoted with the large
    black dot). The bottom panel shows the binary eccentricity in the LISA band
    ($0.1\;\mathrm{Hz}$; blue) and in the LIGO band ($10 \;\mathrm{Hz}$; red),
    with medians marked with large dots. }\label{fig:popsynth}
\end{figure}
\begin{figure}
    \centering
    \includegraphics[width=\columnwidth]{a2eff5500.png}
    \caption{Same as Fig.~\ref{fig:popsynth} but for $a_{\rm out, eff} =
    5500\;\mathrm{AU}$. }\label{fig:popsynth5500}
\end{figure}

\subsection{Effect of Smaller Mass Ratios}

The above results may seem somewhat counterintuitive at first glance, since
$t_{\rm GW} \propto \mu$ which should grow as $q$ is decreased. This effect
should increase the merger times and decrease the merger fractions, but the data
illustrate otherwise.

The key insight is that, since $e_{\lim} > e_{\rm os}$ for our fiducial
parameter regime, most binaries execute one-shot mergers when undergoing an
orbit flip. Since $t_{\rm oct, ZLK} \ll 10\;\mathrm{Gyr}$, this implies that
octupole-ZLK-induced binary merger fractions are primarily determined by what
initial conditions execute orbit flips and only weakly depend on the detailed GW
radiation rate. Indeed, Eq.~\eqref{eq:q_237} shows that, while $e_{\lim} >
e_{\rm os}$ is indeed violated if $q$ is decreased sufficiently, the dependence
is extremely weak. As such, $f_{\rm merge}$ is expected to be very nearly
constant in $q$ for all physical values of $q$. We verify this using the
semi-analytic merger criterion in Fig.~\ref{fig:popsynth_lowq}.
\begin{figure}
    \centering
    \includegraphics[width=\colummwidth]{a2eff_nogw_lowq3600.png}
    \caption{Same as blue dashed line of the top panel of
    Fig.~\ref{fig:popsynth} but extended to very small $q$. Due to the very weak
    $q$ dependence in Eq.~\eqref{eq:q_237}, $f_{\rm merge}$ is expected to
    depend very weakly on $q$ when $q \ll 1$ (such that $\epsilon_{\rm oct}$ is
    approximately constant), which agrees with the simulation
    results.}\label{fig:popsynth_lowq}
\end{figure}

\subsection{Limitations of Semi-Analytic Calculation}\label{ss:completeness}

It is evident from Figs.~\ref{fig:total_merger_fracs}--\ref{fig:popsynth5500}
that the semi-analytic merger fractions are systematically lower than the values
obtained from the simulations that include GW radiation. This can also be
directly seen in the bottom panels of
Figs.~\ref{fig:composite_dist}--\ref{fig:composite_bindist}, particularly at the
edges of the merger windows.

The reason this arises is because the non-dissipative simulations used to
compute $e_{\rm eff}$ and $e_{\max}$ are only run for $2000 t_{\rm LK} \approx
3\;\mathrm{Gyr}$, while the simulations including GW dissipation are run for
$10\;\mathrm{Gyr}$. Owing to the chaotic nature of the octupole-order ZLK
effect, this means that, if an initial condition only reaches more extreme
eccentricities after a few Gyr, the $e_{\rm eff}$ and $e_{\max}$ are
underpredicted by the non-dissipative simulations. Additionally, phases where
$\omega$ is librating can last an unpredictable amount of time, during which
orbit flips are strongly suppressed \citep{katz2011long}. This suggests that the
semi-analytic merger criteria becomes more complete as the integration time is
increased.

To quantify this, we compute the ``completeness'' of the semi-analytic merger
fraction via the ratio $f_{\rm merge, SA} / f_{\rm merge}$, where $f_{\rm merge,
SA}$ is computed using non-dissipative simulations with varying integration
times. We focus on just the fiducial parameter regime for demonstrative
purposes, and we compute a completeness for each of the $q$ and $e_{\rm out}$
combinations shown in Fig.~\ref{fig:total_merger_fracs}.
Figure~\ref{fig:completeness} shows the completeness for each of these
simulations in light grey lines and their mean in the thick black line. We see
that the completeness is still increasing even as the non-dissipative simulation
time is increased to $2000t_{\rm ZLK}$, so we expect that even longer
integration times would give even better agreement with the dissipative
simulations.
\begin{figure}
    \centering
    \includegraphics[width=\colummwidth]{completeness.png}
    \caption{Completeness of the semi-analytic merger fraction, defined as
    $f_{\rm merge, SA} / f_{\rm merge}$, as a function of the integration time
    used for the non-dissipative simulations, in the fiducial
    parameter regime while $e_{\rm out}$ is fixed at a few values. The thin grey
    lines indicate the completeness for particular combinations of $(q, e_{\rm
    out})$, and the thick black line denotes their average. We see that
    completeness is still increasing as the integration time approaches $2000
    t_{\rm ZLK} \approx 3\;\mathrm{Gyr}$. }\label{fig:completeness}
\end{figure}

\section{Conclusion and Discussion}\label{s:conclusion}

We have considered the dynamics of a comparable-mass tertiary-induced binary BH
merger by studying the ZLK effect at octupole order. We show that for system
architectures where the merger fraction is substantial, most merging systems
have small mass ratios. For instance, for the fiducial system architecture ($a =
100\;\mathrm{AU}$, $a_{\rm out} = 3600\;\mathrm{AU}$, inner binary mass
$50M_{\odot}$ and tertiary mass $30M_{\odot}$), the merger fraction varies by
$\sim 20\times$ between $q = 0.2$ and $q = 1.0$ (see Fig.~\ref{fig:popsynth}).
On the other hand, where the merger fraction only varies weakly with $q$, the
absolute merger fractions must also be small (see
Fig.~\ref{fig:sweepbin_simpleouter}). We demonstrate that the difference between
these two regimes lies in whether octupole-induced one-shot mergers are
possible. An analytical criterion for disambiguating between these two regimes
is given by Eq.~\eqref{eq:q_237}.

We discuss the implications of our results in context of the observed binary BH
mergers from LIGO/VIRGO\@. When including the latest data from the O3a observing
run, the observed distribution in $q$ significantly prefers larger mass ratios.
In particular, if $P(q) = q^{\beta_q}$, then $\beta_q > 0$ at $89\%$ or more
\citep{LIGOO3a}. On the other hand, the primordial $q$ distribution in massive
stellar binaries is generally expected to roughly uniform or prefer smaller $q$
\citep[e.g.][]{sana2012binary, duchene2013, kobulnicky2014, moe2017mind}.
This suggests that the initial distribution of $q$ in wide BH binaries is also
likely to be close to uniform or to prefer smaller $q$ \citep[for reasonable
assumptions on BH remnant masses after supernovae, see e.g.][]{farmer_bhmass}.
As our results suggest the $q$ distribution at merger is further skewed towards
small $q$, we see that comparable-mass-tertiary induced BH mergers produces a
distribution of $q$ than is in tension with the $P(q)$ found by LIGO/VIRGO\@. It
is possible that the initial mass ratio distribution in BH binaries strongly
prefers large $q$, or that they all have small $\epsilon_{\rm oct}$ such that
the $q$ enhancement is weak.

Note that the double-averaged (DA) equations of motion are used throughout this
work. However, most of the qualitative behavior we observe is expected to
persist when using either single-averaged equations or N-body simulations, e.g.\
\citet{LL19} find that $f_{\rm merge}$ is larger for the single-averaged and
N-body regimes, but the qualitative trends are unchanged.

We also comment that one-shot mergers can restore the validity of the DA
equations: recall that the DA approximation reqires the orbital period $P_{\rm
out}$ of the outer binary be much longer than all dynamical timescales of the
inner binary, i.e.
\begin{equation}
    t_{\rm ZLK}j\p{e_{\max, q}} \lesssim P_{\rm out},
\end{equation}
For our fiducial parameter regime, this is not satisfied when $e_{\max, q} =
e_{\lim}$, but it is satisfied when $e_{\max, q} = e_{\rm os} < e_{\lim}$. Thus,
in our simulations, the inner binary will begin to decay significantly as a
one-shot merger before the DA approximation breaks down, and use of the DA
equations of motion is justified.

\section{Acknowledgements}\label{s:ack}

YS is supported by the NASA FINESST grant 19-ASTRO19-0041.%chktex 8

\bibliographystyle{mnras}
\bibliography{Su_EZLK}

\clearpage
\onecolumn

\appendix

\section{Origin of Octupole-Inactive Gap}\label{app:gap}

In this section, we investigate the origin of the ``octupole-inactive gap'', an
inclination range $I_0 \approx 90^\circ$ for which $e_{\max}$ does not attain
$e_{\lim}$ despite being in between two octupole-active windows. This gap was
first identified in Section~\ref{ss:oct_gen}, and exists even without GW
radiation. We take Fig.~\ref{fig:composite_1p2} to be our primary example, and
reproduce the relevant portions in Fig.~\ref{fig:kdist}.

\begin{figure}
    \centering
    \includegraphics[width=\colummwidth]{1p2dist.png}
    \caption{Plot illustrating the octupole-inactive gap. The top panel shows
    $e_{\max}$ (blue dots), $e_{\lim}$ [Eq.~\eqref{eq:def_elim}, horizontal
    solid line], and $e_{\rm \max, q}$ [Eq.~\eqref{eq:emax_quad}, dashed line]
    from the top panel of Fig.~\ref{fig:composite_1p2}. The vertical black line
    is the empirically-determined center of the gap, denoted $I_{\rm 0, gap}$.
    Here, $I_{\rm 0, gap} \approx 88.32^\circ$. The bottom panel shows the range of
    oscillation in $K$ [Eq.~\eqref{eq:def_K}], denoted by $K_{\min}$ and
    $K_{\max}$, for the same parameters, as well as $K_0$, the initial $K$ for a
    given $I_0$, in the black dashed line. The critical $K_{\rm c} = -\eta / 2$
    for orbit flipping is shown by the horizontal red line. It can be seen that
    when $e_{\max}$ attains $e_{\lim}$ in the top panel correspond to the ranges
    where $K_{\min} < K_{\rm c} < K_{\max}$. It is therefore clear that the
    octupole-inactive gap is due to a suppression of oscillations in $K$, and
    indeed, the center of the octupole-inactive gap corresponds to the
    inclination for which $K$ oscillations are completely suppressed.
    }\label{fig:kdist}
\end{figure}

At the octupole order, \citet{katz2011long} showed in the test particle limit
that $K = j(e) \cos I$ oscillates over long timescales when $\omega$, the
argument of pericenter of the inner orbit, is circulating. This then leads to
orbit flips between prograde and retrograde inclinations when $K$ changes signs:
since $j(e)$ is nonnegative, the sign of $K$ determines the sign of $\cos I$.
They find that the evolution of $K$ is coupled with that of $\Omega_{\rm e}$,
the azimuthal angle of the inner eccentricity vector in the inertial reference
frame. The amplitude of oscillation of $K$ can then be analytically computed,
and the octupole-active window (the range of $I_0$ over which orbit flips occur)
is the region for which the $K$ oscillation amplitude exceeds $\abs{K_0}$, the
initial $K$ \citep{katz2011long}.

As discussed in Section~\ref{ss:oct_gen}, we find that the relation between $K$
oscillations and orbit flipping can be generalized even when $\eta$ is nonzero:
$K$, given by Eq.~\eqref{eq:def_K}, is again oscillatory when $\omega$ is
circulating, and when it crosses $K_{\rm c} \equiv -\eta / 2$, the inner orbit
flips. Here, orbit flips entail the range of inclination oscillations changing
from $\p{\cos I_0}_- < \cos I < \cos I_{0, \lim}$ to $\cos I_{0, \lim} < \cos I
< \p{\cos I_0}_+$ or vice versa, where $\p{\cos I_0}_{\pm}$ are given by
Eq.~\eqref{eq:I0bounds} and $I_{0, \lim}$ satisfies Eq.~\eqref{eq:I0lim}.

However, a detailed examination of the bottom panel of Fig.~\ref{fig:kdist}
reveals novel behavior due to the broken symmetry of $I_0$ about $90^\circ$. If
we call the center of the gap $I_{\rm 0, gap}$ (shown as the vertical black line in
Fig.~\ref{fig:kdist}), then we find that $K$ oscillates about $K(I_{\rm 0, gap}) >
0$, and the oscillation amplitude goes to zero at $I_{\rm 0, gap}$. Since $K_{\rm
c} < 0$, there will then always be a range of $I_0$ about $I_{\rm 0, gap}$ for
which the oscillation amplitude is smaller than $K\p{I_{\rm 0, gap}} - K_{\rm c}$,
and this range gives rise to the octupole-inactive gap.

This analysis has simply transplanted our lack of understanding into a new
quantity: why are $K$ oscillations suppressed in the neighborhood of $I_{\rm
0, gap}$? A quantitative answer to this question is beyond the scope of this
paper, but for a qualitative understanding, we can examine the evolution of a
system in the octupole-inactive gap. In Fig.~\ref{fig:nogw_circ}, we show a
figure analogous to Fig.~\ref{fig:nogw_fiducial} but for $I_0 = 88^\circ$, very
near $I_{\rm 0, gap}$. It is clear that the oscillations in $K$ (third panel)
are much suppressed, and no orbit flips occur. Most interestingly, the fourth
panel shows that the evolution of $\Omega_{\rm e}$ is much less smooth than in
Fig.~\ref{fig:nogw_fiducial}, jumping at almost every other eccentricity
maximum. \citet{katz2011long} have already pointed out that jumps in
$\Omega_{\rm e}$ occur when $\omega$ is \emph{librating}, rather than
circulating.

\begin{figure}
    \centering
    \includegraphics[width=\colummwidth]{1nogw_vec88.png}
    \caption{Same as Fig.~\ref{fig:nogw_fiducial} but for $I_0 = 88^\circ$. As a
    result, the orbit does not flip. Here, we have used $\omega_0 = 0$, but the
    evolution is very similar for $\omega_0 = \pi / 2$.}\label{fig:nogw_circ}
\end{figure}

When the octupole-order terms are neglected, the circulation-libration
boundary is a boundary in $\omega$: as long as the ZLK separatrix exists in the
$e$-$\omega$ plane and $e_0 > 0$, then an initial $\omega_0 = 0$ causes $\omega$
to circulate, while an initial $\omega_0 = \pi/2$ causes $\omega$ to librate
\citep[e.g.][]{kinoshita, shevchenko2016lidov}. However, when including
octupole-order terms, the dynamics complicate significantly. To illustrate this,
for a range of $I_0$ and both $\omega_0 = 0$ and $\omega_0 = \pi$, we evolve the
fiducial system parameters for a single ZLK cycle, using $q = 0.2$ as is used
for Figs.~\ref{fig:kdist} and~\ref{fig:nogw_circ}, and considered both the
dynamics with and without the octupole-order terms. The resulting changes in
$\Omega_{\rm e}$ over a single ZLK period are shown in Fig.~\ref{fig:dW}.

Two observations can be made: (i) $I_{\rm 0, gap}$ is approximately where
$\Delta \Omega_{\rm e} = 0$ for circulating initial conditions, and (ii) the
inclusion of the octupole-order terms seem to cause $\Omega_{\rm e}$ to
exclusively vary slowly except for $I_{\rm 0, gap} < I_0 < I_{0, \lim}$. The
former is intuitive, as any equilibrium in $K$-$\Omega_{\rm e}$ space must have
$\Delta \Omega_{\rm e} = 0$. The latter suggests that the assumption of
circulation in \citet{katz2011long} is more accurate than might initially be
expected when octupole-order effects are included except in the
octupole-inactive gap. Finally, if we assume that $I_{\rm 0, gap}$ is indeed
where $\Delta \Omega_{\rm e} = 0$, then the gap can be numerically located, and
the width of the gap is just
\begin{equation}
    \text{Gap Width} = 2\p{I_{0, \lim} - I_{\rm 0, gap}}.
\end{equation}
This also explains why the gap does not exist in the test-particle regime, as
$I_{0, \lim} = I_{\rm 0, gap} = 90^\circ$ by symmetry.

It is clear from the preceeding discussion and Fig.~\ref{fig:dW} that the
octupole-order, finite-$\eta$ dynamics are complex, and our preceeding
discussion can only be considered heuristic. Nevertheless, in the absence of a
closed form solution to the octupole-order ZLK equations of motion or a full
generalization of the work of \citet{katz2011long}, they provide an accurate
characterization of the octupole-inactive gap.

\begin{figure}
    \centering
    \includegraphics[width=\colummwidth]{2_dWsweeps6_2_dual.png}
    \caption{Plot of $\Delta \Omega_{\rm e}$, the change in $\Omega_{\rm e}$
    over a single ZLK cycle, for $q = 0.2$ and the fiducial parameters using
    different initial conditions. We consider both if octupole-order terms are
    ignored or not. In the top panel, the octupole terms are neglected, and
    $\omega_0 = 0$ yields gradual change in $\Omega_{\rm e}$ while $\omega_{\rm
    0} = \pi / 2$ yields jumps in $\Omega_{\rm e}$. This is in agreement with
    the results of \citet{katz2011long}: librating $\omega$ give large changes
    in $\Omega_{\rm e}$ while circulating $\omega$ give gradual changes. In the
    bottom panel, the octupole terms are included, and it is seen that the
    $\Delta \Omega_{\rm e}$ behavior is much more complex. The two vertical
    black lines denote $I_{\rm 0, gap}$ and $I_{0, \lim}$ on the left and right,
    respectively. The former is approximately where $\Delta \Omega_{\rm e} =
    0$.}\label{fig:dW}
\end{figure}

\label{lastpage} % chktex 24
\end{document} % chktex 17
