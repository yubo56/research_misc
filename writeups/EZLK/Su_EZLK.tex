% for i in 3qhist 1sweepbin/composite_tp 1nogw_sims/1nogw_vec_q02_short 1nogw_sims/1nogw_vec_q02 1sweepbin/composite_1p5dist 1sweepbin/composite_1p3dist 1sweepbin/composite_1p2dist 1sweepbin/composite_e91p5dist 1sweepbin/bindist 1sweepbin/total_merger_fracs 1sweepbin_simple/simpleouter 1popsynth/a2eff3600 1popsynth/a2eff5500 1popsynth/a2eff_nogw_lowq3600 1sweepbin/completeness 1sweepbin_emax_long/1p2dist 1nogw_sims/1nogw_vec88 2dW_sweeps/2_dWsweeps6_2_dual; do cp ../../scripts/octlk/$i.png .; done
    \documentclass[
        fleqn,
        usenatbib,
        % referee,
    ]{mnras}
    \usepackage{
        amsmath,
        amssymb,
        newtxtext,
        newtxmath,
        ae, aecompl,
        graphicx,
        booktabs,
        xcolor,
    }

    \newcommand*{\scinot}[2]{#1\times10^{#2}}
    \newcommand*{\rd}[2]{\frac{\mathrm{d}#1}{\mathrm{d}#2}}
    \newcommand*{\rtd}[2]{\frac{\mathrm{d}^2#1}{\mathrm{d}#2^2}}
    \newcommand*{\pd}[2]{\frac{\partial#1}{\partial#2}}
    \newcommand*{\ptd}[2]{\frac{\partial^2#1}{\partial#2^2}}
    % inline
    \newcommand*{\mdil}[2]{\mathrm{D}#1/\mathrm{D}#2}
    \newcommand*{\pdil}[2]{\partial#1/\partial#2}
    \newcommand*{\rdil}[2]{\mathrm{d}#1/\mathrm{d}#2}
    \newcommand*{\md}[2]{\frac{\mathrm{D}#1}{\mathrm{D}#2}}
    \newcommand*{\at}[1]{\left.#1\right|}
    \newcommand*{\abs}[1]{\left|#1\right|}
    \newcommand*{\ev}[1]{\left\langle#1\right\rangle}
    \newcommand*{\p}[1]{\left(#1\right)}
    \newcommand*{\s}[1]{\left[#1\right]}
    \newcommand*{\z}[1]{\left\{#1\right\}}
    \newcommand*{\bm}[1]{\mathbf{#1}}
    \newcommand*{\uv}[1]{\hat{\mathbf{#1}}}
    \DeclareMathOperator*{\med}{med}
    \DeclareMathOperator*{\erf}{erf}

    \newlength{\colummwidth}
    \setlength{\colummwidth}{246.0pt} % columnwidth for reprint

\title[Mass Ratio Distribution]{The Mass Ratio Distribution of
Tertiary Induced Binary Black Hole Mergers}
\author[Y. Su et\ al.]{
Yubo Su,$^1$,
Bin Liu,$^{1,2}$,
Dong Lai$^1$
\\
$^1$ Cornell Center for Astrophysics and Planetary Science, Department of
Astronomy, Cornell University, Ithaca, NY 14853, USA\\
$^2$ Niels Bohr International Academy, Niels Bohr Institute, Blegdamsvej 17,
2100 Copenhagen, Denmark.
}

\date{Accepted XXX\@. Received YYY\@; in original form ZZZ}

\pubyear{2021}

\begin{document}\label{firstpage}
\pagerange{\pageref{firstpage}--\pageref{lastpage}}
\maketitle

\begin{abstract}
    Many proposed scenarios for black hole mergers consist of a tertiary
    companion that induces von Zeipel-Lidov-Kozai eccentricity cycles in the
    inner binary. An attractive feature of such mechanisms is the enhanced
    merger rate when the octupole-order corrections, also known as the eccentric
    Kozai mechanism, are important. This can be the case when the tertiary is of
    comparable mass to the binary components. However, in this work, we show
    that this enhancement is preferentially more efficient for binaries with
    smaller mass ratios, sometimes by as much as $20\times$. We use a
    combination of numerical and analytical approaches to fully
    characterize these octupole-induced binary black hole mergers and provide
    analytical criteria for quickly estimating the strength of this enhancement.
    We show that the resulting observational signature is in tension with the
    mass ratio distribution obtained by the LIGO/VIRGO collaboration.
\end{abstract}

\begin{keywords}
binaries:close -- stars:black holes % chktex 8
\end{keywords}

\section{Introduction}\label{s:intro}

The $50$ or so black hole (BH) binary mergers detected by the LIGO/VIRGO
collaboration to date \citep{LIGOO3a} continue to motivate theoretical studies of their
formation channels. These range from the traditional isolated binary evolution,
in which mass transfer and friction in the common envelope phase cause the
binary orbit to decay sufficiently that it subsequently merges via emission of
gravitational waves (GWs) \citep[e.g.,][]{lipunov1997black,
lipunov2017first, podsiadlowski2003formation, belczynski2010effect,
belczynski2016first, dominik2012double, dominik2013double, dominik2015double},
to various flavors of dynamical formation channels that involve either strong
gravitational scatterings in dense clusters \citep[e.g.,][]{zwart1999black,
o2006binary, miller2009mergers, banerjee2010stellar, downing2010compact,
ziosi2014dynamics, rodriguez2015binary, samsing2017assembly, samsing2018black,
rodriguez2018post, gondan2018eccentric} or mergers in isolated triple and
quadruple systems induced by distant companions \citep[e.g.,][]{miller2002four,
wen2003eccentricity, antonini2012secular, antonini2017binary, silsbee2017lidov,
bin1, LL18, randall2018induced, hoang2018black, fragione2019, fragione2019loeb,
LL19, bin_misc5, bin_misc1, bin_misc2}.

\textcolor{red}{Given the large number of merger events to be detected in the
coming years, it is important to search for observational signatures to
distinguish various BH binary formation channels. One possible indicator is
merger eccentricity: previous studies find that dynamical binary-single
interactions in dense clustures \citep[e.g.,][]{samsing2017assembly,
rodriguez2018post, samsing2018black, fragione2019bromberg} or in galactic
triples \citep{silsbee2017lidov, antonini2017binary, fragione2019loeb} may lead
to BH binaries that enter the LIGO band with modest eccentricities. Another
possible indicator is the spin-orbit misalignment of the binary. In particular,
the mass-weighted projection of the BH spins, $\chi_{\rm eff} = \p{m_1
\bm{\chi}_1 + m_2\bm{\chi}_2} / (m_1 + m_2) \cdot \uv{L}$, can be measured
through the binary inspiral waveform [here, $m_{1,2}$ is the BH mass,
$\bm{\chi}_{1,2} = c\bm{S}_{1,2} / \p{Gm_{1,2}^2}$ is the dimensionless BH spin,
and $\uv{L}$ is the unit orbital angular momentum vector of the binary].
Different formation histories yield different distributions of $\chi_{\rm eff}$
\citep{LL17, LL18, LL19, antonini2018precessional, rodriguez2018post,
su2020spin}.}

A third possible indicator is the distribution of masses and mass ratio of
merging BHs. In Fig.~\ref{fig:qhist}, we show the distribution of the mass
ratio $q \equiv m_2 / m_1$, where $m_1 \geq m_2$, for all LIGO/VIRGO binaries
detected as of the O3a data release \citep{LIGOO3a}. The distribution distinctly
peaks around $q \sim 0.7$. BH binaries formed via isolated binary evolution are
generally expected to have $q \gtrsim 0.5$ \citep{belczynski2016first,
olejak2020}. On the other hand, dynamical formation channels may produce a
larger variety of distributions for the binary mass ratio
\citep[e.g.,][]{silsbee2017lidov, fragione2019}.

\begin{figure}
    \centering
    \includegraphics[width=\colummwidth]{3qhist.png}
    \caption{Histogram of the mass ratios $q \equiv m_2 / m_1$ of binary BH
    mergers in the O3a dataset, excluding the two NS-NS mergers but including
    GW190814, whose $2.5M_{\odot}$ secondary may be a BH \citep{LIGOO3a}.
    }\label{fig:qhist}
\end{figure}

In this paper, we study in detail the mass ratio distribution for BH mergers
induced by tertiary companions in triple systems. In this scenario, a tertiary
BH on a sufficiently inclined (outer) orbit induces phases of extreme
eccentricity in the inner binary via the von Zeipel-Lidov-Kozai
(ZLK;\@\citealp{zeipel, lidov, kozai}), leading to efficient gravitational
radiation and orbital decay.  While the original ZLK effect relies on the
leading-oder, quadrupolar gravitational perturbation from the tertiary on the
inner binary, the octupole order terms can become important when the triple
system is mildly hierachical, the outer orbit is eccentric ($e_{\rm out} \neq 0$)
and the inner binary BHs have unequal masses \citep[e.g.,][]{ford2000secular,
blaes2002kozai, lithwick2011eccentric, LML15}. The strength of the
octupole effect depends on the dimensionless parameter
\begin{equation}
    \epsilon_{\rm oct} = \frac{m_1 - m_2}{m_1 + m_2} \frac{a}{a_{\rm out}}
        \frac{e_{\rm out}}{1 - e_{\rm out}^2}.\label{eq:eps_oct}
\end{equation}
where $a, a_{\rm out}$ are the semi-major axes of the inner and outer binaries,
respectively. Previous studies have shown that the octupole terms generally
increase the inclination window for extreme eccentricity excitation, and thus
enhance the rate of successful binary mergers \citep{LL18}. As $\epsilon_oct
\propto (1-q)/(1+q)$ increases with decreasing $q$, we expect that ZLK-induced
BH mergers favor binaries with smaller mass ratios. The main goal of this paper
is to quantify the dependence of the merger fraction/probability on $q$, using a
combination of analytical and numerical calculations.

We review some analytical results of ZLK oscillations in
Section~\ref{s:background}, study tertiary-induced BH mergers in
Section~\ref{s:with_gw}, calculate the merger fractions as a function of mass
ratio in Section~\ref{s:merger_frac}, and conclude and discuss in
Section~\ref{s:conclusion}.

\section{Von Zeipel-Lidov-Kozai (ZLK) Oscillations: Analytical
Results}\label{s:background}

Consider two BHs orbiting each other with masses $m_1$ and $m_2$ on a orbit with
semi-major axis $a$, eccentricity $e$, and angular momentum $\bm{L}$. An external,
tertiary BH of mass $m_3$ orbits this inner binary with semi-major axis $a_{\rm
out}$, eccentricity $e_{\rm out}$, and angular momentum $\bm{L}_{\rm out}$. The
reduced masses of the inner and outer binaries are $\mu \equiv m_1m_2 / m_{12}$
and $\mu_{\rm out} \equiv m_{12} m_3 / m_{123}$ respectively, where $m_{12} =
m_1 + m_2$ and $m_{123} = m_{12} + m_3$. These two binary orbits are further
described by three angles: the inclinations $i$ and $i_{\rm out}$, the arguments
of pericenters $\omega$ and $\omega_{\rm out}$, and the longitudes of the
ascending nodes $\Omega$ and $\Omega_{\rm out}$. These angles are defined in a
coordinate system where the $z$ axis is aligned with the total angular momentum
$\bm{J} = \bm{L} + \bm{L}_{\rm out}$ (i.e.\ the invariant plane is perpendicular
to $\bm{J}$). The mutual inclination between the two orbits is denoted $I \equiv
i + i_{\rm out}$. Note that $\Omega_{\rm out} = \Omega + 180^\circ$.

To study the evolution of the inner binary under the influence of the tertiary
BH, we use the double-averaged secular equations of motion, including the
interactions between the inner binary and the tertiary up to the octupole level
of approximation as given by \citet{LML15}. For the remainder of this section,
we include general relativistic apsidal precession of the inner binary, a first
order post-Newtonian (1PN) effect, but omit the emission of GWs, a 2.5PN
effect---this will be considered in Section~\ref{s:with_gw}. We group the
results by increasing order of approximation, starting by ignoring the
octupole-order effects entirely.

\subsection{Quadrupole Order}

At the quadrupole order, the tertiary induces eccentricity oscillations in the
inner binary on the characteristic timescale
\begin{equation}
    t_{\rm ZLK} = \frac{1}{n}\frac{m_{12}}{m_3}
            \p{\frac{a_{\rm out, eff}}{a}}^3,\label{eq:def_tzlk}
\end{equation}
where $n \equiv \sqrt{Gm_{12} / a^3}$ is the mean motion of the inner binary,
and $a_{\rm out, eff} \equiv a_{\rm out}\sqrt{1 - e_{\rm out}^2}$. During these
oscillations, there are two conserved quantities, the total energy and the total
orbital angular momentum. Through some manipulation, the total angular momentum
can be written in terms of the conserved quantity $K$ \citep{LML15}, given by
\begin{equation}
    K \equiv j(e) \cos I - \eta e^2 / 2. \label{eq:def_K}
\end{equation}
Here, $j(e) \equiv \sqrt{1 - e^2}$ and $\eta$ is the ratio of the magnitudes of
the angular momenta at zero inner binary eccentricity
(\textcolor{red}{formatting of the below equation?}):
\begin{equation}
    \eta \equiv \p{\frac{L}{L_{\rm out}}}_{e = 0}
        = \frac{\mu}{\mu_{\rm out}}\s{\frac{m_{12}a}
            {m_{123}a_{\rm out}(1 - e_{\rm out}^2)}}^{1/2}.\label{eq:def_eta}
\end{equation}
Note that when $\eta = 0$, $K$ reduces to the classical ``Kozai constant'', $K =
j(e) \cos I$.

The maximum eccentricity $e_{\max}$ attained in these ZLK oscillations can be
computed analytically at the quadrupolar order. It depends on the
``competition'' between the 1PM apsidal precession rate $\dot{\omega}_{\rm GR}$
and the ZLK rate $t_{\rm ZLK}^{-1}$. The relevant dimensionless parameter is
\begin{equation}
    \epsilon_{\rm GR} \equiv \p{\dot{\omega}_{\rm GR} t_{\rm ZLK}}_{e = 0}
        = \frac{3Gm_{12}}{c^2} \frac{m_{12}}{m_3}\frac{a_{\rm out, eff}^3}{a^4}.
\end{equation}
It can then be shown that, for an initially circular inner binary, $e_{\max}$ is
related to the initial mutual inclination $I_{\rm 0}$ by \citep{LML15,
anderson2016formation}:
\begin{align}
    \frac{3}{8}\frac{j^2(e_{\max}) - 1}{j^2(e_{\max})}\Big[&
        5\p{\cos I_0 + \frac{\eta}{2}}^2
        - \p{3 + 4\eta \cos I_0 + \frac{9}{4}\eta^2}j^2(e_{\max})
            \nonumber\\
        &+ \eta^2 j^4(e_{\max})
    \Big] + \epsilon_{\rm GR}\s{1 - \frac{1}{j(e_{\max})}} = 0.
    \label{eq:emax_quad}
\end{align}
In the limit $\eta \to 0$ and $\epsilon_{\rm GR} \to 0$, we recover the
well-known result $e_{\max} = \sqrt{1 - (5/3) \cos^2 I_0}$. For general $\eta$,
$e_{\max}$ attains its limiting value $e_{\lim}$ when $I_{\rm 0} = I_{\rm 0,
\lim}$, where
\begin{equation}
    \cos I_{\rm 0, \lim} = \frac{\eta}{2}\s{\frac{4}{5}j^2(e_{\lim}) -
        1}.\label{eq:def_I0lim}
\end{equation}
Note that $I_{\rm 0, \lim} \geq 90^\circ$ with equality only when $\eta
= 0$. Substituting Eq.~\eqref{eq:def_I0lim} into Eq.~\eqref{eq:emax_quad}, we find
that $e_{\lim}$ satisfies
\begin{align}
    \frac{3}{8}\s{j^2(e_{\lim}) - 1}&\s{-3 + \frac{\eta^2}{4}
        \p{\frac{4}{5}j^2(e_{\lim}) - 1}}\nonumber\\
        &+ \epsilon_{\rm GR}\s{1 - \frac{1}{j(e_{\lim})}} = 0.
        \label{eq:def_elim}
\end{align}
On the other hand, eccentricity excitation ($e_{\max} \geq 0$) is only possible
when $(\cos I_{\rm 0})_- \leq \cos I_{\rm 0} \leq (\cos I_{\rm 0})_+$ where
\begin{equation}
    \p{\cos I_{\rm 0}}_{\pm} = \frac{1}{10}\p{-\eta \pm \sqrt{\eta^2 + 60 -
        \frac{80}{3}\epsilon_{\rm GR}}}.\label{eq:I0bounds}
\end{equation}
For $I_{\rm 0}$ outside of this range, no eccentricity excitation is possible.
This condition reduces to the well-known $\cos^2 I_{\rm 0} \leq 3/5$ when $\eta
= \epsilon_{\rm GR} = 0$.

\subsection{Octupole Order: Test-particle Limit}\label{ss:oct_tp}

\begin{figure}
    \centering
    \includegraphics[width=\colummwidth]{composite_tp.png}
    \caption{The maximum eccentricity achieved for an inner binary in the
    test-particle limit as a function of the initial inclination angle $I_0$ of
    the tertiary companion. The triple system parameters are:
    $a = 100\;\mathrm{AU}$, $a_{\rm out, eff} = 3600\;\mathrm{AU}$, $m_{12}
    = 50M_{\odot}$, $m_3 = 30M_{\odot}$, and $e_{\rm out} = 0.6$; the
    corresponding octupole strength parameter is $\epsilon_{\rm oct} = 0.02$.
    The octupole-level secular equations of motion are integrated for
    $2000t_{\rm ZLK}$ (see Eq.~\ref{eq:def_tzlk}), and the maximum
    eccentricity attained during this time is recorded and shown as a blue dot
    for each initial condition We consider $1000$ initial inclinations in the
    range $50^\circ \leq I_0 \leq 130^\circ$, and each $I_0$ is simulated five
    times, with the initial orbital elements $\omega$, $\omega_{\rm out}$, and
    $\Omega = \Omega_{\rm out} - \pi$ chosen randomly $\in [0, 2\pi)$ % chktex 9
    for each simulation. The dotted black line shows the quadrupole-level result
    (Eq.~\ref{eq:emax_quad} with $\eta = 0$), and $e_{\lim}$
    (Eq.~\ref{eq:def_elim}) is shown as the horizontal red line. The vertical
    purple lines denote the boundary of the octupole-active inclination window,
    based on the fitting formula from \citet{MLL16} (Eq.~\ref{eq:I_oct_MLL}).
    }\label{fig:composite_tp}
\end{figure}

The relative strength of the octupole-order potential to the quadrupole-order
potential is determined by the dimensionless parameter $\epsilon_{\rm oct}$
(Eq.~\ref{eq:eps_oct}). When $\epsilon_{\rm oct}$ is non-negligible, $K$ is no
longer conserved, and the system evolution becomes chaotic
\citep{ford2000secular, katz2011long, lithwick2011eccentric, li2014chaos,
LML15}. As a result, analytic (and semi-analytic) results have only been given
for the test-particle limit, where $m_2 = \eta = 0$. We briefly review these
results below.

Due to the non-conservation of $K$, $e_{\max}$ evolves irregularly ZLK cycles,
and the orbit may even flip between prograde ($I < 90^\circ$) and retrograde ($I
> 90^\circ$) if $K$ changes sign (in the test-particle limit, $K = j(e) \cos
I$). During these orbit flips, the eccentricity maxima reach their largest
values but do not exceed $e_{\lim}$ \citep{lithwick2011eccentric, LML15,
anderson2016formation}. These orbit flips occur on characteristic timescale
$t_{\rm ZLK, oct}$, given by \citep{antognini2015timescales}
\begin{equation}
    t_{\rm ZLK, oct} = t_{\rm ZLK}\frac{128\sqrt{10}}{
        15\pi\sqrt{\epsilon_{\rm oct}}}.\label{eq:def_tzlkoct}
\end{equation}
The octupole potential tends to widen the inclination range for which the
eccentricity can reach $e_{\lim}$; we refer to this widened range as the
\emph{octupole-active window}. \citet{katz2011long} show that this window can be
approximated using analytical arguments when $\epsilon_{\rm oct} \ll 1$.
\citet{MLL16} give a more general numerical fitting formula describing the
octupole-active window for arbitrary $\epsilon_{\rm oct}$. They find that orbit
flips and extreme eccentricity excitation occur for $I_{\rm flip, -} \lesssim
I_0 \lesssim I_{\rm flip, +}$ where
\begin{equation}
    \cos^2 I_{\rm flip, \pm} = \begin{cases}
        0.26\p{\frac{\epsilon_{\rm oct}}{0.1}}
            - 0.536\p{\frac{\epsilon_{\rm oct}}{0.1}}^2\\
            \quad + 12.05\p{\frac{\epsilon_{\rm oct}}{0.1}}^3
            - 16.78\p{\frac{\epsilon_{\rm oct}}{0.1}}^4
            & \epsilon_{\rm oct} \lesssim 0.05,\\
        0.45 & \epsilon_{\rm oct} \gtrsim 0.05.
    \end{cases} \label{eq:I_oct_MLL}
\end{equation}
Figure~\ref{fig:composite_tp} shows an example of the maximum eccentricity
$e_{\max}$ achieved by the inner binary when integrated for $2000t_{\rm ZLK}$ as
a function of $I_0$. We see that with the octupole effect included, $e_{\max}$
indeed attains $e_{\lim}$ when $I_0$ is within the broad octupole-active window
given by Eq.~\eqref{eq:I_oct_MLL}.

\subsection{Octupole Order: General Masses}\label{ss:oct_gen}

\begin{figure}
    \centering
    \includegraphics[width=\colummwidth]{1nogw_vec_q02_short.png}
    \caption{An example of the triple evolution for a system with significant
    octupole effects and finite $\eta$ (see Eq.~\ref{eq:def_eta}). We use the
    same system parameters as in Fig.~\ref{fig:composite_tp} except for $q =
    0.2$, corresponding to $\eta \approx 0.087$ and $\epsilon_{\rm oct} \approx
    0.007$, and $I_0 = 93.5^\circ$. The three panels show the inner orbit
    eccentricity, the mutual inclination, and the generalized ``Kozai constant''
    $K$ (Eq.~\ref{eq:def_K}). In the first panel, $e_{\lim}$ is denoted by the
    black dashed line. By comparing the second and third panels, we see that
    orbit flips occur when $K$ crosses the dotted line, given by $K = K_{\rm c}
    \equiv -\eta / 2$. }\label{fig:nogw_fiducial}
\end{figure}

For general inner binary masses, when the angular momentum ratio $\eta$ is
non-negligible, the octupole-level ZLK behavior is less well-studied
\citep[see][]{LML15}. Figure~\ref{fig:nogw_fiducial} shows an example of the
evolution of a triple system with significant $\eta$ and $\epsilon_{\rm oct}$.
Many aspects of the evolution discussed in Section~\ref{ss:oct_tp} are still
observed: the ZLK eccentricity maxima and $K$ evolve over timescales $\gg t_{\rm
ZLK}$; the eccentricity never exceeds $e_{\lim}$; when $K$ crosses $K_{\rm c}
\equiv -\eta / 2$, an orbit flip occurs (this follows by inspection of
Eq.~\ref{eq:def_K}).

However, Eq.~\eqref{eq:I_oct_MLL} no longer describes the octupole-active window
as $\eta$ is non-negligible \citep[see also][]{rodet_inprep}. In the top panel
of Fig.~\ref{fig:composite_dist}, the blue dots show the maximum achieved
eccentricity of a system with the same parameters as Fig.~\ref{fig:composite_tp}
except with $q = 0.5$ (so $\epsilon_{\rm oct} = 0.007$ and $\eta = 0.087$).
Here, it can be seen that no prograde systems can attain $e_{\lim}$, and only a
small range of retrograde inclinations $\geq I_{\rm 0, \lim}$ (see
Eq.~\ref{eq:def_I0lim}) are able to reach $e_{\lim}$. In fact, there is even a
clear double valued feature around $I \approx 75^\circ$ in the top panel of
Fig.~\ref{fig:composite_dist} that is not present in
Fig.~\ref{fig:composite_tp}. If $q$ is decreased to $0.3$
(Fig.~\ref{fig:composite_1p3}) or further to $0.2$
(Fig.~\ref{fig:composite_1p2}), $\epsilon_{\rm oct}$ increases while $\eta$
decreases. This permits a larger number of prograde systems to reach $e_{\lim}$,
though a small range of inclinations near $I_0 = 90^\circ$ still do not reach
$e_{\lim}$; we call this range of inclinations the ``octupole-inactive gap''. On
the other hand, if $q$ is held at $0.5$ as in Fig.~\ref{fig:composite_dist} and
$e_{\rm out}$ is increased to $0.9$ while holding $a_{\rm out, eff} =
3600\;\mathrm{AU}$ constant, both $\epsilon_{\rm oct}$ and $\eta$ increase; the
top panel of Fig.~\ref{fig:composite_e91p5} shows that prograde systems still
fail to reach $e_{\lim}$ for these parameters, despite the increase in
$\epsilon_{\rm oct}$. The top panel of Fig.~\ref{fig:composite_bindist}
illustrates the behavior when the inner binary is substantially more compact ($a
= 10\;\mathrm{AU}$): even though $\epsilon_{\rm oct}$ is larger than it is
in any of Figs.~\ref{fig:composite_dist}--\ref{fig:composite_e91p5}, we
see that prograde perturbers fail to attain $e_{\lim}$. All of these examples
(top panels of Figs.~\ref{fig:composite_dist}--\ref{fig:composite_bindist})
illustrate importance of $\eta$ in determining the range of inclinations for the
system to be able to reach $e_{\lim}$.

\textcolor{red}{In general, we find that a symmetric octupole-active window (as
in Eq.~\ref{eq:I_oct_MLL}) can be realized for sufficiently small $\eta$. In
\citet{rodet_inprep}, they find that $\eta \lesssim 0.1$ is sufficient for a
symmetric octupole-active window, but in this work, $\eta \lesssim 0.01$ is
necessary ($\eta = 0.004$ in Fig.~\ref{fig:composite_tp} while $\eta = 0.05$ in
Fig.~\ref{fig:composite_1p2}). Thus, the critical $\eta$ above which the
symmetry of the octupole-active window is significantly broken likely depends on
$e_{\lim}$ (in \citet{rodet_inprep}, $e_{\lim} \sim 10^{-3}$). When $\eta$ is
non-negligible, the octupole-active window is split into two intervals: a
prograde interval whose existence depends on the specific values of $\eta$ and
$\epsilon_{\rm oct}$, and a retrogade interval that always exists.}

\begin{figure}
    \centering
    \includegraphics[width=\colummwidth]{composite_1p5dist.png}
    \caption{Eccentricity excitation and merger windows for the fiducial BH
    triple system ($a = 100\;\mathrm{AU}$, $a_{\rm out, eff} =
    3600\;\mathrm{AU}$, $m_{12} = 50M_{\odot}$, $m_3 = 30M_{\odot}$) with $q =
    0.5$ and $e_{\rm out} = 0.6$, corresponding to $\eta \approx 0.087$ and
    $\epsilon_{\rm oct} \approx 0.007$. In the top panel, for each of $1000$
    initial inclinations, we choose $5$ different random $\omega$, $\omega_{\rm
    out}$, and $\Omega$ as initial conditions and evolve the system for
    $2000t_{\rm ZLK}$ without GW radiation. The effective eccentricity $e_{\rm
    eff}$ (Eq.~\ref{eq:def_e_eff}; green dots) as well as the maximum
    eccentricity $e_{\max}$ (blue dots) over this period are displayed. For
    comparison, $e_{\rm eff, c}$ (Eq.~\ref{eq:def_e_eff_c}) is given by the
    horizontal green dashed line, $e_{\rm os}$ (Eq.~\ref{eq:def_e_os}) is shown
    as the horizontal blue line, and $e_{\lim}$ (Eq.~\ref{eq:def_elim}) is shown
    as the horizontal red dashed line. The vertical purple lines denote the
    test-mass octupole-active window and are given by the fitting formula of
    \citet{MLL16}; they do not longer accurately describe the
    $e_{\lim}$-attaining inclination window because $\eta$ is finite. The black
    dashed line is is the quadrupole-level result as given by
    Eq.~\eqref{eq:emax_quad}. In the middle panel, we show the binary merger
    times when including GW radiation and using the same range of initial
    conditions. Numerical integrations are terminated when $T_{\rm m} >
    10\;\mathrm{Gyr}$ and marked as unsuccessful mergers. The horizontal dashed
    line denotes $t_{\rm ZLK}$ (Eq.~\ref{eq:def_tzlk}) while the horizontal
    dash-dotted line indicates $t_{\rm ZLK, oct}$ (Eq.~\ref{eq:def_tzlkoct}).
    Here, each $I_{\rm 0}$ is run $20$ times with uniform distributions of
    $\omega$, $\omega_{\rm out}$, and $\Omega$, so we can estimate the merger
    probability $P_{\rm merger}$ (Eq.~\ref{eq:def_pmerge}) for each $I_{\rm
    0}$---$P_{\rm merger}$ is shown as the black line in the bottom panel. As
    described in Section~\ref{ss:nogw_merger}, the merger probability can be
    predicted analytically using the results of the top panel and
    Eq.~\eqref{eq:def_pmerge_sa}, and is denoted by $P_{\rm merger}^{\rm an}$.
    In the bottom panel, the thick green line shows $P_{\rm merger}^{\rm an}$
    when using an integration time of $2000 t_{\rm ZLK} \approx 3\;\mathrm{Gyr}$
    for the non-dissipative simulations, and thin green line shows the
    prediction using an integration time of $500 t_{\rm ZLK}$. The agreement of
    $P_{\rm merger}^{\rm an}$ with $P_{\rm merger}$ is good and improves when
    using the longer integration time.
    }\label{fig:composite_dist}
\end{figure}
\begin{figure}
    \centering
    \includegraphics[width=\colummwidth]{composite_1p3dist.png}
    \caption{Same as Fig.~\ref{fig:composite_dist} but for $q = 0.3$,
    corresponding to $\eta
    \approx 0.07$ and $\epsilon_{\rm oct} \approx 0.011$.}\label{fig:composite_1p3}
\end{figure}
\begin{figure}
    \centering
    \includegraphics[width=\colummwidth]{composite_1p2dist.png}
    \caption{Same as Fig.~\ref{fig:composite_dist} but for $q = 0.2$,
    corresponding to $\eta
    \approx 0.054$ and $\epsilon_{\rm oct} \approx 0.014$.
    }\label{fig:composite_1p2}
\end{figure}
\begin{figure}
    \centering
    \includegraphics[width=\colummwidth]{composite_e91p5dist.png}
    \caption{Same as Fig.~\ref{fig:composite_dist} but for $e_{\rm out} = 0.9$
    while holding $a_{\rm out, eff}$ the same, correpsonding to $\eta =
    0.118$ and $\epsilon_{\rm oct} = 0.019$. }\label{fig:composite_e91p5}
\end{figure}
\begin{figure}
    \centering
    \includegraphics[width=\colummwidth]{bindist.png}
    \caption{Same as Fig.~\ref{fig:composite_dist} but for a more compact inner
    binary; the parameters are $a_0 = 10\;\mathrm{AU}$, $a_{\rm out, eff} =
    700\;\mathrm{AU}$, $m_{12} = 50M_{\odot}$, $m_3 = 30M_{\odot}$, $e_{\rm
    out} = 0.9$, and $q = 0.4$, corresponding to $\eta = 0.118$ and
    $\epsilon_{\rm oct} = 0.029$. Here, each $P_{\rm merger}$ is computed with
    only $5$ integrations (for random $\omega$, $\omega_{\rm out}$, and
    $\Omega$) per $I_0$.}\label{fig:composite_bindist}
\end{figure}

\section{Tertiary-Induced Black Hole Mergers}\label{s:with_gw}

Emission of gravitational waves (GWs) affects the evolution of the inner
binary, which can be incorporated into the secular equations of motion for the
triple \citep[e.g.,]{peters1964, LL18}. The associated orbital and eccentricity
decay rates are \citep{peters1964}:
\begin{align}
    \at{\frac{1}{a}\rd{a}{t}}_{\rm GW} &\equiv -\frac{1}{t_{\rm GW}}\nonumber\\
        &= -\frac{64}{5}\frac{G^3 \mu m_{12}^2}{c^5a^4j^7(e)}
            \p{1 + \frac{73}{24}e^2 + \frac{37}{96}e^4}\label{eq:def_tgw},\\
    \at{\rd{e}{t}}_{\rm GW} &= -\frac{304}{15}\frac{G^3 \mu m_{12}^2}{c^5a^4}
        \frac{1}{j^{5}(e)}\p{1 + \frac{121}{304}e^2}\label{eq:dedt_gw}.
\end{align}
GW emission can cause the orbit to decay significantly when extreme
eccentricities are reached during the ZLK cycles described in the previous
section. This allows even wide binaries ($\sim 100\;\mathrm{AU}$) to merge
efficiently within a Hubble time. While various numerical examples of such
tertiary-induced mergers have been given before (e.g.\ \citet{LL18}; see also
\citet{LL19} for ``population synthesis''), in this section we examine the
dynamical process in detail in order to develop an analytical understanding.
Our fiducial system parameters are as in Fig.~\ref{fig:nogw_fiducial}:
$a_{\rm out, eff} = 4500\;\mathrm{AU}$, $e_{\rm out} =
0.6$, $m_{12} = 50M_{\odot}$ (with varying $q$), $m_3 = 30M_{\odot}$, and the
inner binary has initial $a_0 = 100\;\mathrm{AU}$ and $e_0 = 10^{-3}$.

\subsection{Merger Windows and Probability: Numerical Results}\label{ss:windows}

To understand what initial conditions lead to successful mergers within a Hubble
time, we integrate the double-averaged octupole-order ZLK equations including GW
radiation. We terminate each integration if either $a = 0.005a_0$ (a successful
merger) or the system age reaches $10\;\mathrm{Gyr}$. The middle panel of
Fig.~\ref{fig:composite_dist} shows the merger time $T_{\rm m}$ as a function of
$I_0$ for our fiducial parameters with $q = 0.5$. We note that only retrograde
inclinations lead to successful mergers, and almost all successful mergers are
rapid, with $T_{\rm m} \lesssim t_{\rm ZLK, oct}$. These are the result of a
system merging by emitting a single large burst of GW radiation during an
extreme-eccentricity ZLK cycle, which we term a ``one-shot merger''. In
Fig.~\ref{fig:composite_1p3}, $q$ is decreased to $0.3$, and some prograde
systems are also able to merge successfully. However, these prograde systems
exhibit a broad range of merger times, with $T_{\rm m} \gtrsim t_{\rm ZLK,
oct}$. These occur when a system gradually emits a small amount of GW radiation
at every eccentricity maximum---we term this a ``smooth merger''. Additionally,
the octupole-inactive gap near $I_0 = 90^\circ$ is visible in the merger time
plot (middle panel of Fig.~\ref{fig:composite_1p3}). The middle panels of
Figs.~\ref{fig:composite_1p2}--\ref{fig:composite_bindist} show the behavior of
$T_{\rm m}$ for the other parameter regimes and also exhibit these two
categories of mergers and the octupole-inactive gap.

Due to the chaotic nature of the octupole-order ZLK effect, the initial
inclination $I_0$ alone is not sufficient to determine with certainty whether a
system can merge within a Hubble time. Instead, for a given $I_0$, we can use
numerical integrations with various $\omega$, $\omega_{\rm out}$, and $\Omega$
to compute a merger probability, denoted by
\begin{equation}
    P_{\rm merger}\p{I_0; q, e_{\rm out}} = P\p{T_{\rm m} < 10\;\mathrm{Gyr}},
        \label{eq:def_pmerge}
\end{equation}
where the notation $P_{\rm merger}\p{I_0; q, e_{\rm out}}$ highlights the
dependence of $P_{\rm merger}$
on $q$ and $e_{\rm out}$, two of the key factors that determine the strength of the
octupole effect (of course $P_{\rm merger}$ depends on other system parameters
such as $m_12$, $a_0$, $a_{\rm out}$, etc.)
The bottom panels of Figs.~\ref{fig:composite_dist}--\ref{fig:composite_bindist}
show our numerical results. In all of these plots, there is a retrograde
inclination window for which successful merger is guaranteed. In
Fig.~\ref{fig:composite_1p3}, it can be seen that a large range of prograde
inclinations have a probabilistic outcome. In Fig.~\ref{fig:composite_1p2},
while the enhanced octupole strength allows for most of the prograde
inclinations to merge with certainty, there is still a region around $I_0 \approx
80^\circ$ where $P_{\rm merger} < 1$.

\subsection{Merger Probability: Analytic Criteria}\label{ss:nogw_merger}

By comparing the top and bottom panels of
Figs.~\ref{fig:composite_dist}--\ref{fig:composite_bindist}, it is clear that
their features are correlated: in all five cases, the retrograde merger window
occupies the same inclination range as the retrograde octupole-active window,
while $P_{\rm merger}$ is only nonzero for prograde inclinations where
$e_{\max}$ nearly attains $e_{\lim}$. Here, we further develop this connection
and show that the non-dissipative simulations can be used to predict the
outcomes of simulations with GW dissipation rather reliably.

In Section~\ref{ss:windows}, we identified both one-shot and smooth mergers in
our simulations. Towards understanding the one-shot mergers, we first define
$e_{\rm os}$ to be the $e_{\max}$ required to dissipate an order-unity fraction
of the binary's orbital energy via GW emission in a single ZLK cycle. Since a
binary spends a fraction $\sim j(e_{\max})$ of each ZLK cycle near $e_{\max}$
\citep[e.g.,][]{anderson2016formation}, we set
\begin{equation}
    j\p{e_{\rm os}}\at{\rd{\ln a}{t}}_{e = e_{\rm os}} \sim
        -\frac{1}{t_{\rm ZLK}},
\end{equation}
where $\rdil{(\ln a)}{t}$ is given by Eq.~\eqref{eq:def_tgw}. This yields
\begin{equation}
    j^6(e_{\rm os})
        \equiv \frac{842}{15}
            \frac{G^3 \mu m_{12}^3}{m_3c^5a^4n}
            \p{\frac{a_{\rm out, eff}}{a}}^3,
            \label{eq:def_e_os}
\end{equation}
where we have approximated $e_{\rm os} \approx 1$. Then, if a system satisfies
$e_{\max} > e_{\rm os}$ with $e_{\max}$ based on non-dissipative integration, it
is expected attain a sufficiently large eccentricity to undergo a one-shot
merger.

Towards understanding smooth mergers, we seek a characteristic eccentricity that
captures GW emission over many ZLK cycles. We define $e_{\rm eff}$ as an
effective ZLK maximum eccentricity, i.e.
\begin{align}
    \ev{\rd{\ln a}{t}} &= -\frac{1}{t_{\rm GW, 0}}
            \ev{\frac{1 + 73e^2/24 + 37e^4/96}
                {j^7(e)}}\nonumber\\
        &\equiv -\frac{421/96}{t_{\rm GW, 0}}\frac{1}{j^6(e_{\rm eff})},
        \label{eq:def_e_eff}
\end{align}
where $t_{\rm GW, 0} = \p{t_{\rm GW}}_{\rm e = 0}$ (see Eq.~\ref{eq:def_tgw}),
and the angle brackets denote averaging over many ZLK cycles.
In the second line of Eq.~\eqref{eq:def_e_eff}, we have essentially replaced the ZLK-avergaed
orbital decay rate by $\rdil{(\ln a)}{t}$ evaluated at $e_{\rm eff}$ multiplied
by $j(e_{\rm eff})$.
In practice (see Figs.~\ref{fig:composite_dist}--\ref{fig:composite_bindist}),
we typically average over $2000 t_{\rm ZLK}$ of
the non-dissipative simulations to compute $e_{\rm eff}$.

With $e_{\rm eff}$ computed using Eq.~\eqref{eq:def_e_eff}, we can define the
critical effective eccentricity $e_{\rm eff, c}$ such that the inspiral time is
a Hubble time:
\begin{equation}
    \ev{\rd{\ln a}{t}} \equiv \frac{421/96}{t_{\rm GW, 0}j^6(e_{\rm eff, c})}
        = \p{10\;\mathrm{Gyr}}^{-1}. \label{eq:def_e_eff_c}
\end{equation}
Thus, if a system is evolved using the non-dissipative equations of motion and
satisfies $e_{\rm eff} > e_{\rm eff, c}$, then it is expected to successfully
undergo a smooth merger within a Hubble time.

Therefore, a system can be predicted to merge successfully if it satisfies either
the one-shot or smooth merger criteria. The analytic merger probability (as a
function of $I_0$ and other parameters) is:
\begin{equation}
    P_{\rm merger}^{\rm an}\p{I_0; q, e_{\rm out}} =
        P\p{e_{\rm eff} > e_{\rm eff, c} \;\;\text{or}\;\;
        e_{\max} > e_{\rm os}}.\label{eq:def_pmerge_sa}
\end{equation}
Although not fully analytic (since numerical integrations of non-dissipative
systems are needed to obtain $e_{\rm eff}$ and $e_{\max}$ in general),
Eq.~\eqref{eq:def_pmerge_sa} provides efficient computation of the merger
probability without full numerical integrations including GW radiation.

The top panels of Figs.~\ref{fig:composite_dist}--\ref{fig:composite_bindist}
show $e_{\rm eff}$ and $e_{\max}$, and their critical values, $e_{\rm eff, c}$
and $e_{\rm os}$. Using these, we compute the analytic merger probability, shown
as the thick green lines in the bottom panels of
Figs.~\ref{fig:composite_dist}--\ref{fig:composite_bindist}. We generally
observe good agreement with the numerical $P_{\rm merger}$. However, $P_{\rm
merger}^{\rm an}$ slightly but systematically underpredicts $P_{\rm merger}$ for
some configurations, such as the prograde inclinations in
Figs.~\ref{fig:composite_1p3} and~\ref{fig:composite_bindist}. These regions
coincide with the inclinations for which the merger outcome is uncertain. This
underprediction is due to the restricted integration time of $2000 t_{\rm ZLK}
\approx 3\;\mathrm{Gyr}$ used for the non-dissipative simulations. To illustrate
this, we also calculate $P_{\rm merge}^{\rm an}$ using a shorter integration
time of $500 t_{\rm ZLK}$ for our non-dissipative simulations. The results are
shown as the light green lines in the bottom panels of
Figs.~\ref{fig:composite_dist}--\ref{fig:composite_bindist}, performing visibly
worse. A more detailed discussion of this issue can be found in
Section~\ref{ss:completeness}.

A few observations about Eq.~\eqref{eq:def_pmerge_sa} can be made. First, it
explains why some prograde systems merge probabilistically ($0 < P_{\rm merger}
< 1$): for the prograde inclinations in Fig.~\ref{fig:composite_1p3},
the $e_{\rm eff}$ values scatter widely around $e_{\rm eff,c}$ [or more
precisely, $j(e_{\rm eff})$ scatters around $j(e_{\rm eff,c})$], even for a
given $I_0$, so the detailed merger outcome depends on the initial conditions.
For the prograde inclinations in Fig.~\ref{fig:composite_1p2}, the double-valued
feature in the $e_{\max}$ plot (the top panel) pointed out in
Section~\ref{ss:oct_gen} represents a sub-population of systems that do not
satisfy Eq.~\eqref{eq:def_pmerge_sa}. Second, $e_{\max} > e_{\rm os}$ often
ensures $e_{\rm eff} > e_{\rm eff, c}$ in practice, as the averaging in
Eq.~\eqref{eq:def_e_eff} is heavily weighted towards extreme eccentricities. As
such, $e_{\rm eff} > e_{\rm eff, c}$ alone is often a sufficient condition in
Eq.~\eqref{eq:def_pmerge_sa}.

The one-shot merger criterion ($e_{\max} > e_{\rm os}$) can also be used to distinguish two different
types of system architectures: if $e_{\lim} \gtrsim e_{\rm os}$ for a particular
architecture, then all initial conditions leading to orbit flips (i.e.\ in an
octupole-active window) also execute one-shot mergers. For $e_{\lim} \approx
1$, Eq.~\eqref{eq:def_elim} reduces to
\begin{equation}
    j(e_{\lim}) \approx \frac{8\epsilon_{\rm GR}}{9}\p{1 +
        \frac{\eta^2}{12}}^{-1}.
\end{equation}
which lets us rewrite the constraint $e_{\lim} \gtrsim e_{\rm os}$ as
\begin{align}
    \p{\frac{a}{a_{\rm out, eff}}} \gtrsim{}&
        0.0186
        \p{\frac{a_{\rm out, eff}}{3600\;\mathrm{AU}}}^{-7/37}
        \p{\frac{m_{12}}{50M_{\odot}}}^{17/37}\nonumber\\
        &\times\p{\frac{30M_{\odot}}{m_3}}^{10/37}
        \p{\frac{q / (1 + q)^2}{1/4}}^{-2/37}.\label{eq:q_237}
\end{align}
For the system architecture considered in
Figs.~\ref{fig:composite_dist}--\ref{fig:composite_e91p5}, this condition is
indeed satisfied, and we see indeed that wherever the top panel suggests orbit
flipping ($e_{\max} = e_{\lim}$), the bottom panel shows $f_{\rm merger} \approx
1$. When the condition (Eq.~\ref{eq:q_237}) is not satisfied, one-shot mergers are
not possible, and $P_{\rm merger}$ is generally only nonzero for a small range
about $I_{\rm 0, \lim}$.

\section{Merger Fraction as a Function of Mass Ratio}\label{s:merger_frac}

In this section, we characterize the effect described in the preceeding sections
by studying the fraction of BH binaries that successfully merge under various
conditions, which we call the merger fraction.

\subsection{Merger Fraction for Fixed Tertiary Eccentricity}

We first consider the simplest case where $e_{\rm out}$ is fixed at a few
specific values and compute the merger fraction as a function of the mass ratio
$q$. We let $\cos I_0$ be drawn uniformly from the range $[-1, 1]$, let
$\omega$, $\omega_{\rm out}$, and $\Omega$ be drawn from the range $[0,
2\pi)$ as before % chktex 9
and adopt the fiducial parameters. The merger fraction is defined as:
\begin{equation}
    f_{\rm merger}\p{q, e_{\rm out}} \equiv
        \frac{1}{2}\int\limits_{-1}^1P_{\rm merger}\p{I_0; q, e_{\rm out}}
            \;\mathrm{d}\cos I_0.\label{eq:def_fmerge}
\end{equation}
Intuitively, this corresponds to the integral of the black lines in the bottom
panels of Figs.~\ref{fig:composite_dist}--\ref{fig:composite_e91p5}. For each
simulation, we can also use analytic criteria introduced in
Section~\ref{ss:nogw_merger} to predict the outcome and merger fraction. This is
computed by using $P_{\rm merger}^{\rm an}$ as the integrand in
Eq.~\eqref{eq:def_fmerge}, or by evaluating the integral of the thick green
lines in the bottom panels of
Figs.~\ref{fig:composite_dist}--\ref{fig:composite_e91p5}.
Figure~\ref{fig:total_merger_fracs} shows $f_{\rm merger}$ and the analytical
estimates for all combinations
of $q \in \z{0.2, 0.3, 0.4, 0.5, 0.7, 1.0}$ and $e_{\rm out} \in \z{0.6, 0.8,
0.9}$. It is clear both that the analytic merger fraction tracks $f_{\rm
merger}$ well and that the merger fractions increase steeply for smaller $q$.

Looking carefully at the right panel of Fig.~\ref{fig:total_merger_fracs}, we
see that the merger fractions for the three $e_{\rm out}$ values overlap for
small $\epsilon_{\rm oct}$. This implies that $f_{\rm merger}$ depends only on
$\epsilon_{\rm oct}$ in this regime, and not the values of $q$ and $e_{\rm out}$
independently. From Fig.~\ref{fig:composite_dist}, we see that this suggests
that the size of the retrograde merger window only depends on $\epsilon_{\rm
oct}$, much like Eq.~\eqref{eq:I_oct_MLL} shows for the test-particle limit.
However, once $\epsilon_{\rm oct}$ is increased sufficiently, the three curves
cease to overlap. This can be attributed to their different $\eta$ values: for
sufficiently small $\epsilon_{\rm oct}$, no prograde initial inclinations
successfully merge (e.g.\ Fig.~\ref{fig:composite_dist}), and the merger
fraction is solely determined by the size of the retrograde octupole-active
window. But once $\epsilon_{\rm oct}$ is sufficiently large, prograde mergers
become possible, and the merger fraction is also affected by the size of the
octupole-inactive gap, which depends on $\eta$. This again illustrates the
importance of the octupole-inactive gap, which we comment on in
Appendix~\ref{app:gap}.

Figure~\ref{fig:sweepbin_simpleouter} depicts the merger fractions if instead $a
= 50\;\mathrm{AU}$. According to Eq.~\eqref{eq:q_237}, this new parameter regime
no longer satisfies $e_{\lim} \gtrsim e_{\rm os}$, so the merger fraction is
expected to diminish strongly and vary much more weakly with $q$, as one-shot
mergers are no longer possible. This is indeed observed, particularly for
$e_{\rm out} = 0.6$.

\begin{figure*}
    \centering
    \includegraphics[width=\textwidth]{total_merger_fracs.png}
    \caption{From Figs.~\ref{fig:composite_dist}--\ref{fig:composite_e91p5}, we
    can compute the total merger fraction in the presence of GW radiation
    assuming $\cos I_0$ is uniformly distributed $\in [-1, 1]$ for the fiducial
    parameter regime. We do this for all combinations of $q \in \z{0.2, 0.3,
    0.4, 0.5, 0.7, 1.0}$ and $e_{\rm out} \in \z{0.6, 0.8, 0.9}$ and show the
    results with solid dots. The X's show the results when using
    Eq.~\eqref{eq:def_pmerge_sa} and an integration time of $2000 t_{\rm ZLK}$.
    }\label{fig:total_merger_fracs}
\end{figure*}
\begin{figure*}
    \centering
    \includegraphics[width=\textwidth]{simpleouter.png}
    \caption{Same as Fig.~\ref{fig:total_merger_fracs} but for $a =
    50\;\mathrm{AU}$; the analytic merger fractions have been omitted for
    clarity.
    }\label{fig:sweepbin_simpleouter}
\end{figure*}

\subsection{Merger Fraction for a Distribution of Tertiary Eccentricities}

We can also draw $e_{\rm out} \in [0, 0.9]$ with both a uniform probability
distribution and a thermal one $P(e_{\rm out}) \propto e_{\rm out}$, and examine
the merger fraction as a function of $q$. We denote this
\begin{equation}
    \eta_{\rm merger}(q) \propto
        \int\limits_{0}^{0.9}\int\limits_{-1}^1
            P_{\rm merger}\p{I_0; q, e_{\rm out}} P(e_{\rm out})
            \;\mathrm{d}\cos I_0\;\mathrm{d}e_{\rm out}.
\end{equation}
The top panel of Fig.~\ref{fig:popsynth} gives $\eta_{\rm merger}$ (black
dots) for the fiducial parameter regime, where each $q$ has $1000$ realizations
using random $e_{\rm out}$, $\cos I_0$, $\omega$, $\omega_{\rm out}$, and
$\Omega$. If $e_{\rm out}$ is thermally distributed (red dots), binaries with
smaller $q$ are even more likely to merge, since $\epsilon_{\rm oct}$ tends to
be larger. To understand the impact of our random sampling, we also compute a
merger fraction using the analytic merger criterion of
Eq.~\eqref{eq:def_pmerge_sa} on a dense grid of initial conditions uniform in
$e_{\rm out}$ and $\cos I_0$; this is shown as the blue dotted line, which is in
good agreement with the uniform-$e_{\rm out}$ simulation results (black). The
distributions and medians of the merger time and eccentricity in the LISA and
LIGO bands are also shown in the middle and bottom panels respectively. For the
LISA and LIGO band eccentricities, the inner binaries are evolved from when they
reach $0.005 a_0$ to physical merger using
Eqs.~(\ref{eq:def_tgw}--\ref{eq:dedt_gw}). While all of these eccentricities are
quite small, this has to do with our using the double-averaged equations of
motion. Both the single-averaged and the full N-body equations of motion produce
larger eccentricities in the LISA and LIGO bands \citep{LL19}.

For comparison, Figure~\ref{fig:popsynth5500} illustrates the outcome when
$a_{\rm out, eff} = 5500\;\mathrm{AU}$ with all other parameters unchanged.
While $f_{\rm merger}$ is lower than it is for $a_{\rm out, eff} =
3600\;\mathrm{AU}$, there is still a large increase between large and small $q$.
Since Eq.~\eqref{eq:q_237} is still satisfied, this is expected.
\begin{figure}
    \centering
    \includegraphics[width=\columnwidth]{a2eff3600.png}
    \caption{Merger fractions with the fiducial parameters obtained by randomly
    drawing $\cos I_0$ uniformly distributed $\in [-1, 1]$ and drawing $e_{\rm
    out}$ from either a uniform distribution ($e_{\rm out} \in [0, 0.9]$; black)
    or a thermal distribution [$P(e_{\rm out}) \propto e_{\rm out}$; red]. The
    blue dotted line instead samples a dense, uniform grid in $\cos I_0$,
    $e_{\rm out}$, and $q \in [0.2, 1]$ using Eq.~\eqref{eq:def_pmerge_sa} for
    an integration time of $2000 t_{\rm ZLK}$, which is in good agreement with
    the dissipative simulations while being smoother. The middle panel shows the
    merger time for successful mergers (the median is denoted with the large
    black dot). The bottom panel shows the binary eccentricity in the LISA band
    ($0.1\;\mathrm{Hz}$; blue) and in the LIGO band ($10 \;\mathrm{Hz}$; red),
    with medians marked with large dots. }\label{fig:popsynth}
\end{figure}
\begin{figure}
    \centering
    \includegraphics[width=\columnwidth]{a2eff5500.png}
    \caption{Same as Fig.~\ref{fig:popsynth} but for $a_{\rm out, eff} =
    5500\;\mathrm{AU}$. }\label{fig:popsynth5500}
\end{figure}

\subsection{Effect of Smaller Mass Ratios}

Even though the octupole strength increases as $q$ is decreased, the efficiently
of GW radiation also decreases. It is therefore natural to ask at what $q$ these
competing effects become equal and merger fraction is maximized. We show that
this does not happen except for unphysically small $q$ in our fiducial parameter
regime.

The key insight is that, since $e_{\lim} > e_{\rm os}$ for our fiducial
parameter regime, most binaries execute one-shot mergers when undergoing an
orbit flip. Since $t_{\rm oct, ZLK} \ll 10\;\mathrm{Gyr}$, this implies that
octupole-ZLK-induced binary merger fractions are primarily determined by what
initial conditions execute orbit flips and only weakly depend on the detailed GW
radiation rate. Indeed, Eq.~\eqref{eq:q_237} shows that, while $e_{\lim} >
e_{\rm os}$ is indeed violated if $q$ is decreased sufficiently, the dependence
is extremely weak. As such, $f_{\rm merger}$ is expected to be very nearly
constant in $q$ for all physical values of $q$. We verify this using the
analytic merger criterion in Fig.~\ref{fig:popsynth_lowq}.
\begin{figure}
    \centering
    \includegraphics[width=\colummwidth]{a2eff_nogw_lowq3600.png}
    \caption{Same as blue dashed line of the top panel of
    Fig.~\ref{fig:popsynth} but extended to very small $q$. Due to the very weak
    $q$ dependence in Eq.~\eqref{eq:q_237}, $f_{\rm merger}$ is expected to
    depend very weakly on $q$ when $q \ll 1$ (such that $\epsilon_{\rm oct}$ is
    approximately constant), which agrees with the simulation
    results.}\label{fig:popsynth_lowq}
\end{figure}

\subsection{Limitations of Analytic Calculation}\label{ss:completeness}

It can be seen in Fig.~\ref{fig:total_merger_fracs}
that the analytic merger fractions is systematically lower than the values
obtained from the direct simulations. This underprediction was first pointed out
in Section~\ref{ss:nogw_merger}.

The reason this arises is because the non-dissipative simulations used to
compute $e_{\rm eff}$ and $e_{\max}$ are only run for $2000 t_{\rm LK} \approx
3\;\mathrm{Gyr}$, while the simulations including GW dissipation are run for
$10\;\mathrm{Gyr}$. Owing to the chaotic nature of the octupole-order ZLK
effect, this means that, if an initial condition reaches extreme eccentricities
after many Gyrs, the $e_{\rm eff}$ and $e_{\max}$ are underpredicted by the
non-dissipative simulations. Additionally, phases where the eccentricity vector
of the inner binary is librating, during which orbit flips are strongly
suppressed \citep{katz2011long}, can last an unpredictable amount of time. This
suggests that the analytic merger criteria becomes more complete as the
integration time is increased.

We quantify the ``completeness'' of the analytic merger fraction via the ratio
$f_{\rm merger}^{\rm an} / f_{\rm merger}$ as a function of non-dissipative
integration time. We focus on just the fiducial parameter regime for
demonstrative purposes and compute a completeness for each of the $q$ and
$e_{\rm out}$ combinations shown in Fig.~\ref{fig:total_merger_fracs}.
Figure~\ref{fig:completeness} shows the completeness for each of these
simulations in light grey lines and their mean in the thick black line. We see
that the completeness is still increasing even as the non-dissipative simulation
time is increased to $2000t_{\rm ZLK}$, so we expect that even longer
integration times would give even better agreement with the dissipative
simulations.
\begin{figure}
    \centering
    \includegraphics[width=\colummwidth]{completeness.png}
    \caption{Completeness of the analytic merger fraction, defined as
    $f_{\rm merger}^{\rm an} / f_{\rm merger}$, as a function of the integration time
    used for the non-dissipative simulations, in the fiducial
    parameter regime while $e_{\rm out}$ is fixed at a few values. The thin grey
    lines indicate the completeness for particular combinations of $(q, e_{\rm
    out})$, and the thick black line denotes their average. We see that
    completeness is still increasing as the integration time approaches $2000
    t_{\rm ZLK} \approx 3\;\mathrm{Gyr}$. }\label{fig:completeness}
\end{figure}

\section{Conclusion and Discussion}\label{s:conclusion}

We have considered the dynamics of a comparable-mass tertiary-induced binary BH
merger by studying the ZLK effect at the octupole order. We show that for system
architectures where the merger fraction is substantial, most merging systems
have small mass ratios. For instance, for the fiducial system architecture ($a =
100\;\mathrm{AU}$, $a_{\rm out} = 3600\;\mathrm{AU}$, inner binary mass
$50M_{\odot}$ and tertiary mass $30M_{\odot}$), the merger fraction varies by
$\sim 20\times$ between $q = 0.2$ and $q = 1.0$ (see Fig.~\ref{fig:popsynth}).
On the other hand, where the merger fraction only varies weakly with $q$, the
absolute merger fractions must also be small (see
Fig.~\ref{fig:sweepbin_simpleouter}). We demonstrate that the difference between
these two regimes lies in whether octupole-induced one-shot mergers are
possible. An analytical criterion for disambiguating between these two regimes
is given by Eq.~\eqref{eq:q_237}.

We discuss the implications of our results in context of the observed binary BH
mergers from LIGO/VIRGO\@. When including the latest data from the O3a observing
run, the observed distribution in $q$ significantly prefers larger mass ratios.
In particular, if $P(q) = q^{\beta_q}$, then $\beta_q > 0$ at $89\%$ or more
\citep{LIGOO3a}. On the other hand, the primordial $q$ distribution in massive
stellar binaries is generally expected to roughly uniform or log-uniform in $q$
\citep[e.g.,][]{sana2012binary, duchene2013, kobulnicky2014, moe2017mind}. For
instance, for random pairings from a Salpeter mass function $P(m) \propto
m^{-p}$, $P(q) \propto q^{-p}$ as well \citep[see also
Appendix~\ref{s:ratios}]{moe2017mind}. This suggests that the initial
distribution of $q$ in wide BH binaries is unlikely to prefer large $q$
\citep[for reasonable assumptions on BH remnant masses after supernovae, see
e.g.][]{farmer_bhmass}. As our results suggest the comparable-mass
tertiary-induced merger channel further enhances merger rates for small $q$, we
see that this channel yields a distribution of $q$ than is in tension with the
$P(q)$ found by LIGO/VIRGO\@. However, these studies generally show that the
mass ratio distribution for merging binaries is more weighted towards smaller
mass ratios than the initial distribution \citep{silsbee2017lidov,
fragione2019}, consistent with our findings. It is nevertheless possible that
the initial mass ratio distribution in BH binaries strongly prefers large $q$,
or that they all have small $\epsilon_{\rm oct}$ such that the $q$ enhancement
is weak.

Note that the double-averaged (DA) equations of motion are used throughout this
work. However, most of the qualitative behavior we observe is expected to
persist when using either single-averaged equations or N-body simulations, e.g.\
\citet{LL19} find that $f_{\rm merger}$ is larger for the single-averaged and
N-body regimes, but the qualitative trends are unchanged. Furthermore, the
phenomenon of one-shot mergers can restore the validity of the DA equations:
recall that the DA approximation reqires the orbital period $P_{\rm out}$ of the
outer binary be much longer than all dynamical timescales of the inner binary,
i.e.
\begin{equation}
    t_{\rm ZLK}j\p{e_{\max}} \lesssim P_{\rm out},
\end{equation}
For our fiducial parameter regime, this is not satisfied when $e_{\max} =
e_{\lim}$, but it is satisfied when $e_{\max} = e_{\rm os} < e_{\lim}$. Thus,
in our simulations, the inner binary will begin to decay significantly as a
one-shot merger before the DA approximation breaks down, and use of the DA
equations of motion is justified.

\section{Acknowledgements}\label{s:ack}

We thank Jiseon Min for useful discussions. YS is supported by the NASA FINESST
grant 19-ASTRO19-0041.%chktex 8

\section{Data Availability}

The data referenced in this article will be shared upon reasonable request to
the corresponding author.

\bibliographystyle{mnras}
\bibliography{Su_EZLK}

\clearpage
\onecolumn

\appendix

\section{Mass Ratio Distribution: Random Drawings from Initial Mass
Function}\label{s:ratios}

We seek the probability distribution of $q$, the mass ratio, of a binary,
assuming its two components are drawn from distributions $P(m) \propto m^{-p}$.
First, we assume that the two masses $m_1$ and $m_2$ are uncorrelated. Without
loss of generality, assume that $m_2 \leq m_1$, then
\begin{align}
    P(q) &= \int\limits_{m_{\min}}^{m_{\max}}\int\limits_{m_{\min}}^{m_1}
            \delta\p{\frac{m_2}{m_1} - q}P(m_1) P(m_2)
            \;\mathrm{d}m_2\;\mathrm{d}m_1,\\
        &= \int\limits_{m_{\min}}^{m_{\max}} m_1 P(m_1) P(qm_1)
            \;\mathrm{d}m_1,\\
        &\propto q^{-p}.
\end{align}
This result is known \citep[e.g.][]{moe2017mind}.

Observationally, we are more interested in the distribution of $q$ for a given
$m_{12}$. Without loss of generality, assume again that $m_2 \geq m_1$, then
\begin{align}
    P(q) &= \int\limits_{m_{\min}}^{m_{\max}}
            \int\limits_{m_{\min}}^{m_1}
            \delta\p{\frac{m_2}{m_1} - q}
            \delta\p{\frac{m_1 + m_2}{m_{12}} - 1}
            P(m_1)
            P\p{m_2}\;\mathrm{d}m_2\;\mathrm{d}m_1,\\
        &= \int\limits_{m_{\min}}^{m_{\max}}
            m_{12}\delta\p{\frac{m_{12} - m_1}{m_1} - q}
            P(m_1)
            P\p{m_{12} - m_1}\;\mathrm{d}m_1,\\
        &= P\p{\frac{m_{12}}{1 + q}}P\p{m_{12} - \frac{m_{12}}{1 + q}}
            \p{\frac{m_{12}}{1 + q}}^2,\\
        &\propto \frac{q^{-p}}{\p{1 + q}^{-2p + 2}}.
\end{align}
While the specific distribution changes slightly, this distribution is still
sharply skewed towards smaller $q$.

\section{Origin of Octupole-Inactive Gap}\label{app:gap}

In this section, we investigate the origin of the ``octupole-inactive gap'', an
inclination range $I_0 \approx 90^\circ$ for which $e_{\max}$ does not attain
$e_{\lim}$ despite being in between two octupole-active windows. This gap was
first identified in Section~\ref{ss:oct_gen}, and exists even without GW
radiation. We take the top panel Fig.~\ref{fig:composite_1p2} to be our primary
example, and reproduce it as the top panel of Fig.~\ref{fig:kdist}.

\begin{figure}
    \centering
    \includegraphics[width=\colummwidth]{1p2dist.png}
    \caption{Plot illustrating the octupole-inactive gap. The top panel shows
    $e_{\max}$ (blue dots), $e_{\lim}$ (Eq.~\ref{eq:def_elim}, horizontal
    solid line), and the quadrupole-level result for $e_{\max}$
    (Eq.~\ref{eq:emax_quad}, dashed line) from the top panel of
    Fig.~\ref{fig:composite_1p2}. The vertical black line is the
    empirically-determined center of the gap, denoted $I_{\rm 0, gap}$. Here,
    $I_{\rm 0, gap} \approx 88.32^\circ$. The bottom panel shows the range of
    oscillation in $K$ (Eq.~\ref{eq:def_K}), denoted by $K_{\min}$ and
    $K_{\max}$, for the same parameters, as well as $K_0$, the initial $K$ for a
    given $I_0$, in the black dashed line. The critical $K_{\rm c} = -\eta / 2$
    for orbit flipping is shown by the horizontal red line. It can be seen that
    when $e_{\max}$ attains $e_{\lim}$ in the top panel correspond to the ranges
    where $K_{\min} < K_{\rm c} < K_{\max}$. It is therefore clear that the
    octupole-inactive gap is due to a suppression of oscillations in $K$, and
    indeed, the center of the octupole-inactive gap corresponds to the
    inclination for which $K$ oscillations are completely suppressed.
    }\label{fig:kdist}
\end{figure}

At the octupole order, \citet{katz2011long} showed in the test particle limit
that $K = j(e) \cos I$ oscillates over long timescales when $\omega$, the
argument of pericenter of the inner orbit, is circulating. This then leads to
orbit flips between prograde and retrograde inclinations when $K$ changes signs:
since $j(e)$ is nonnegative, the sign of $K$ determines the sign of $\cos I$.
They find that the evolution of $K$ is coupled with that of $\Omega_{\rm e}$,
the azimuthal angle of the inner eccentricity vector in the inertial reference
frame. The amplitude of oscillation of $K$ can then be analytically computed,
and the octupole-active window (the range of $I_0$ over which orbit flips occur)
is the region for which the $K$ oscillation amplitude exceeds $\abs{K_0}$, the
initial $K$ \citep{katz2011long}. Finally, when $\omega$ is circulating,
$\Omega_{\rm e}$ also varies gradually over long timescales, but when $\omega$
is librating, $\Omega_{\rm e}$ jumps by $\sim 180^\circ$ every ZLK cycle.

As discussed in Section~\ref{ss:oct_gen}, we find that the relation between $K$
oscillations and orbit flipping can be generalized even when $\eta$ is nonzero:
$K$, given by Eq.~\eqref{eq:def_K}, is again oscillatory when $\omega$ is
circulating, and when it crosses $K_{\rm c} \equiv -\eta / 2$, the inner orbit
flips. To b precise, orbit flips entail the range of inclination oscillations
changing from $\p{\cos I_0}_- < \cos I < \cos I_{0, \lim}$ to $\cos I_{0, \lim}
< \cos I < \p{\cos I_0}_+$ or vice versa, where $\p{\cos I_0}_{\pm}$ are given
by Eq.~\eqref{eq:I0bounds} and $I_{0, \lim}$ satisfies Eq.~\eqref{eq:def_I0lim}.

However, we find novel behavior when examining the oscillation amplitude of $K$.
The bottom panel of Fig.~\ref{fig:kdist} illusrtates the minimum and maximum $K$
attained for a given $I_0$. If we call the center of the gap $I_{\rm 0, gap}$
(shown as the vertical black line), then we find that $K$ oscillates about
$K(I_{\rm 0, gap})$, which is \emph{positive}, and the oscillation amplitude
goes to zero at $I_{\rm 0, gap}$. Since orbit flips occur when the oscillation
amplitude crosses $K_{\rm c} < 0$, there will then always be a range of $I_0$
about $I_{\rm 0, gap}$ for which the oscillation amplitude is smaller than
$K\p{I_{\rm 0, gap}} - K_{\rm c}$, and orbit flips are impossible in this range.
This range is then precisely the octupole-inactive gap.

This analysis has simply transplanted our lack of understanding into a new
quantity: why are $K$ oscillations suppressed in the neighborhood of $I_{\rm
0, gap}$? A quantitative answer to this question is beyond the scope of this
paper, but for a qualitative understanding, we can examine the evolution of a
system in the octupole-inactive gap. The left panel of Fig.~\ref{fig:nogw_circ}
shows the same simulation as Fig.~\ref{fig:nogw_fiducial} but with an additional
panel showing $\Omega_{\rm e}$, while the right panel shows a simulation with
all the same parameters except $I_0 = 88^\circ$, which is near $I_{\rm 0, gap}$
(see Fig.~\ref{fig:kdist}). The oscillations in $K$ (third panels) are much
smaller for $I_0 = 88^\circ$ than for $I_0 = 93.5^\circ$, and no orbit flips
occur. Most interestingly, the fourth panel shows that the evolution of
$\Omega_{\rm e}$ is much less smooth than in Fig.~\ref{fig:nogw_fiducial},
jumping at almost every other eccentricity maximum. \citet{katz2011long} have
already pointed out that jumps in $\Omega_{\rm e}$ occur when $\omega$ is
\emph{librating}, rather than circulating.

\begin{figure}
    \centering
    \includegraphics[width=\colummwidth]{1nogw_vec_q02.png}
    \includegraphics[width=\colummwidth]{1nogw_vec88.png}
    \caption{Left is the same as Fig.~\ref{fig:nogw_fiducial} but including the
    evolution of the azimuthal angle of the eccentricity vector, $\Omega_{\rm
    e}$, while right is the same but for $I_0 = 88^\circ$. For both of
    these examples, we have used $\omega_0 = 0$, but the evolution is very
    similar for all other $\omega_0$.}\label{fig:nogw_circ}
\end{figure}

When the octupole-order terms are neglected, the circulation-libration
boundary is a boundary in $\omega$: as long as the ZLK separatrix exists in the
$e$-$\omega$ plane and $e_0 > 0$, then an initial $\omega_0 = 0$ causes $\omega$
to circulate, while an initial $\omega_0 = \pi/2$ causes $\omega$ to librate
\citep[e.g.,][]{kinoshita, shevchenko2016lidov}. However, when including
octupole-order terms, this picture breaks down. To illustrate this, for a range
of $I_0$ and both $\omega_0 = 0$ and $\omega_0 = \pi$, we evolve the fiducial
system parameters for a single ZLK cycle, using $q = 0.2$ as is used for
Figs.~\ref{fig:kdist} and~\ref{fig:nogw_circ}, and considered both the dynamics
with and without the octupole-order terms. Figure~\ref{fig:dW} gives the
resulting changes in $\Omega_{\rm e}$ over a single ZLK period when the
octupole-order effects are neglected (top) and when they are not (bottom).
Two observations can be made: (i) $I_{\rm 0, gap}$ is approximately where
$\Delta \Omega_{\rm e} = 0$ for circulating initial conditions, and (ii) the
inclusion of the octupole-order terms seem to cause $\Omega_{\rm e}$ to
exclusively vary slowly except for $I_{\rm 0, gap} < I_0 < I_{0, \lim}$. The
former is intuitive, as any equilibrium in $K$-$\Omega_{\rm e}$ space must have
$\Delta \Omega_{\rm e} = 0$. The latter suggests that the assumption of
circulation in \citet{katz2011long} may largely hold for most intiial conditions
except for those in the octupole-inactive gap. Finally, we can write down the
width of the octupole-inactive inclination gap as
\begin{equation}
    \text{Gap Width} = 2\p{I_{0, \lim} - I_{\rm 0, gap}}.
\end{equation}
This also explains why the gap does not exist in the test-particle regime, as
$I_{0, \lim} = I_{\rm 0, gap} = 90^\circ$ by symmetry of the equations of
motion.

It is clear from the preceeding discussion and Fig.~\ref{fig:dW} that the
octupole-order, finite-$\eta$ dynamics are complex, and our preceeding
discussion can only be considered heuristic. Nevertheless, in the absence of a
closed form solution to the octupole-order ZLK equations of motion or a full
generalization of the work of \citet{katz2011long}, they provide an accurate
characterization of the octupole-inactive gap.

\begin{figure}
    \centering
    \includegraphics[width=\colummwidth]{2_dWsweeps6_2_dual.png}
    \caption{Plot of $\Delta \Omega_{\rm e}$, the change in $\Omega_{\rm e}$
    over a single ZLK cycle, for $q = 0.2$ and the fiducial parameters using
    different initial conditions. In the top panel, octupole-order terms are
    neglected, while in the lower panel, they are not. The solid and dashed
    vertical black lines denote $I_{\rm 0, gap}$ and $I_{0, \lim}$
    respectively.}\label{fig:dW}
\end{figure}

\label{lastpage} % chktex 24
\end{document} % chktex 17
