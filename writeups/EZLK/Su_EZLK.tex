    \documentclass[
        fleqn,
        usenatbib,
        % referee,
    ]{mnras}
    \usepackage{
        amsmath,
        amssymb,
        newtxtext,
        newtxmath,
        ae, aecompl,
        graphicx,
        booktabs,
        xcolor,
    }
    \usepackage[T1]{fontenc}

    \newcommand*{\scinot}[2]{#1\times10^{#2}}
    \newcommand*{\rd}[2]{\frac{\mathrm{d}#1}{\mathrm{d}#2}}
    \newcommand*{\rtd}[2]{\frac{\mathrm{d}^2#1}{\mathrm{d}#2^2}}
    \newcommand*{\pd}[2]{\frac{\partial#1}{\partial#2}}
    \newcommand*{\ptd}[2]{\frac{\partial^2#1}{\partial#2^2}}
    % inline
    \newcommand*{\mdil}[2]{\mathrm{D}#1/\mathrm{D}#2}
    \newcommand*{\pdil}[2]{\partial#1/\partial#2}
    \newcommand*{\rdil}[2]{\mathrm{d}#1/\mathrm{d}#2}
    \newcommand*{\md}[2]{\frac{\mathrm{D}#1}{\mathrm{D}#2}}
    \newcommand*{\at}[1]{\left.#1\right|}
    \newcommand*{\abs}[1]{\left|#1\right|}
    \newcommand*{\ev}[1]{\left\langle#1\right\rangle}
    \newcommand*{\p}[1]{\left(#1\right)}
    \newcommand*{\s}[1]{\left[#1\right]}
    \newcommand*{\z}[1]{\left\{#1\right\}}
    \newcommand*{\bm}[1]{\mathbf{#1}}
    \newcommand*{\uv}[1]{\hat{\mathbf{#1}}}
    \DeclareMathOperator*{\med}{med}
    \DeclareMathOperator*{\erf}{erf}

    \colorlet{Corr}{red}

    \newlength{\colummwidth}
    \setlength{\colummwidth}{246.0pt} % columnwidth for reprint

\title[Mass Ratio Distribution]{The Mass Ratio Distribution of
Black Hole Mergers Induced by a Comparable Mass Tertiary}
\author[Y. Su et\ al.]{
Yubo Su,$^1$,
Bin Liu,$^{1,2}$,
Dong Lai$^1$
\\
$^1$ Cornell Center for Astrophysics and Planetary Science, Department of
Astronomy, Cornell University, Ithaca, NY 14853, USA\\
$^2$ Niels Bohr International Academy, Niels Bohr Institute, Blegdamsvej 17,
2100 Copenhagen, Denmark.
}

\date{Accepted XXX\@. Received YYY\@; in original form ZZZ}

\pubyear{2021}

\begin{document}\label{firstpage}
\pagerange{\pageref{firstpage}--\pageref{lastpage}}
\maketitle

\begin{abstract}
    Abstract
\end{abstract}

\begin{keywords}
binaries:close -- stars:black holes % chktex 8
\end{keywords}

\section{Introduction}\label{s:intro}

We study the von Zeipel-Lidov-Kozai effect (ZLK) for eccentric perturbers to
octupole order, also sometimes known as the eccentric Kozai mechanism
\citep[e.g.][]{lithwick2011eccentric}.

The mass ratio distribution among ZLK-induced BH binary mergers has already been
noted \citep[see Fig.~10 of][]{silsbee2017lidov}, but the origin of the effect
has not been carefully studied.

\section{Dynamics Without Gravitational Wave Radiation}\label{s:background}

Consider two BHs orbiting each other with masses $m_1$ and $m_2$ on a orbit with
semimajor axis $a$, eccentricity $e$, and angular momentum vector $\bm{L} \equiv
L\uv{L}$. Also consider a third BH of mass $m_3$ orbiting this binary with
semimajor axis $a_{\rm out}$ and eccentricity $e_{\rm out}$, with angular
momentum vector $\bm{L}_{\rm out} \equiv L_{\rm out} \uv{L}_{\rm out}$. Define
$q \equiv m_2 / m_1$ the mass ratio and $m_{12} = m_1 + m_2$ the total mass of
the inner binary. Our fiducial parameters are: $a = 100\;\mathrm{AU}$, $a_{\rm
out, eff} = 4500\;\mathrm{AU}$, $e_{\rm out} = 0.6$, $m_{12} = 50M_{\odot}$,
$m_3 = 30M_{\odot}$, and we use consistent initial inner binary eccentricity
$e_{\rm 0} = 10^{-3}$.

To define the coordinate system, we choose the $\uv{z}$ axis pointing along the
total angular momentum axis $\bm{L}_{\rm tot} = \bm{L} + \bm{L}_{\rm out}$.
Then, we denote the longitude of the ascending node and argument
of pericenter of the inner and outer orbits by $\Omega$, $\omega$, $\Omega_{\rm
out}$, and $\omega_{\rm out}$ respectively. Note that, by conservation of
angular momentum, $\Omega_{\rm out} = \Omega + \pi$.

We study this system using the octupole-order, double-averaged vectorial ZLK
equations from \citet{LML15}, including general relativistic apsidal precession,
a first post-Newtonian order (1PN) effect. For the remainder of this section, we
review a few results in the absence of gravitational wave (GW) radiation, a
2.5PN effect (to be considered in Section~\ref{s:with_gw}). We group these
results by increasing order of approximation, starting by ignoring the
octupole-order effects entirely.

\subsection{Quadrupole-Order ZLK}

The primary feature of the quadrupole-order dynamics are cycles of the inner
binary's eccentricity from $e \ll 1$ to a consistent maximum $e_{\max} \approx
1$ with period $\sim t_{\rm ZLK}$, where
\begin{equation}
    t_{\rm ZLK} = \frac{1}{n}\frac{m_{12}}{m_3}
            \p{\frac{a_{\rm out, eff}}{a}}^3,\label{eq:def_tzlk}
\end{equation}
where $n \equiv \sqrt{Gm_{12} / a^3}$ is the mean motion of the inner binary,
and $a_{\rm out, eff} \equiv a_{\rm out}\sqrt{1 - e_{\rm out}^2}$. During these
eccentricity cycles, there are two conserved quantities, the total energy and
the quantity \citep{LML15}
\begin{equation}
    K \equiv j(e) \cos I - \eta e^2 / 2. \label{eq:def_K}
\end{equation}
Here, $j(e) \equiv \sqrt{1 - e^2}$, $I \equiv \cos^{-1}(\uv{L} \cdot
\uv{L}_{\rm out})$ is the mutual inclination, and $\eta$ is the ratio of the
magnitudes of the angular momenta:
\begin{equation}
    \eta \equiv \at{\frac{L}{L_{\rm out}}}_{e = 0}
        = \frac{\mu}{\mu_{\rm out}}\s{\frac{m_{12}a}
            {m_{123}a_{\rm out}\p{1 - e_{\rm out}^2}}}^{1/2},\label{eq:def_eta}
\end{equation}
where $\mu \equiv m_1m_2 / m_{12}$, $m_{123} = m_{12} + m_3$, and $\mu_{\rm out}
\equiv m_{12} m_3 / m_{123}$. Note that when $\eta = 0$, $K$ reduces to the
classical ``Kozai constant''.

During these eccentricity cycles, the binary spends a fraction $\sim
j(e_{\max})$ of each ZLK eccentricity cycle near $e_{\max}$
\citep{anderson2016formation}. The exact value of $e_{\max}$ depends on the
general relativistic apsidal precession, which induces precession of the
eccentricity vector $\bm{e}$ of the inner binary as
\begin{equation}
    \at{\rd{\bm{e}}{t}}_{\rm GR} = \Omega_{\rm GR}\uv{L} \times \bm{e}
        = \frac{3Gnm_{12}}{c^2aj^2(e)}\uv{L} \times \bm{e}.
\end{equation}
It is useful to quantify the relative strength of this term via the parameter
\begin{equation}
    \epsilon_{\rm GR} \equiv \frac{3Gm_{12}}{c^2}
        \frac{m_{12}}{m_3}\frac{a_{\rm out, eff}^3}{a^4}.
\end{equation}
It can then be shown that $e_{\max}$ depends on the initial mutual inclination
$I_{\rm 0}$ via \citep{LML15, anderson2016formation}:
\begin{align}
    \frac{3}{8}\frac{j^2(e_{\max}) - 1}{j^2(e_{\max})}\Big[&
        5\p{\cos I_0 + \frac{\eta}{2}}^2
        - \p{3 + 4\eta \cos I_0 + \frac{9}{4}\eta^2}j(e_{\max})^2
            \nonumber\\
        &+ \eta^2 j^4(e_{\max})
    \Big] + \epsilon_{\rm GR}\p{1 - 1 / j(e_{\max})} = 0.\label{eq:emax_quad}
\end{align}
This only has solutions when $\cos I_{\rm 0} \in [(\cos I_{\rm 0})_-, (\cos
I_{\rm 0})_+]$ where \citep{anderson2016formation}
\begin{equation}
    \p{\cos I_{\rm 0}}_{\pm} = \frac{1}{10}\p{-\eta \pm \sqrt{\eta^2 + 60 -
        20\epsilon_{\rm GR}}}.\label{eq:I0bounds}
\end{equation}
For $I_{\rm 0}$ outside of this range, there are no ZLK oscillations; this
reduces to the well-known $\cos^2 I_{\rm 0} \leq 3/5$ when $\eta = \epsilon_{\rm
GR} = 0$. Additionally, the maximum value of $e_{\max}$, denoted $e_{\lim}$,
occurs when $I_{\rm 0} = I_{\rm 0, \lim}$, where
\begin{equation}
    \cos I_{\rm 0, \lim} = \frac{\eta}{2}\s{\frac{4}{5}j^2(e_{\lim}) -
        1}.\label{eq:I0lim}
\end{equation}
Note that $I_{\rm 0, \lim} \geq 90^\circ$ with equality only when $\eta
= 0$. Substituting Eq.~\eqref{eq:I0lim} into Eq.~\eqref{eq:emax_quad}
gives
\begin{align}
    \frac{3}{8}\p{j^2(e_{\max}) - 1}&\s{-3 + \frac{\eta^2}{4}
        \p{\frac{4}{5}j^2(e_{\max}) - 1}}\nonumber\\
        &+ \epsilon_{\rm GR}\p{1 - 1 / j(e_{\max})} = 0.
        \label{eq:def_elim}
\end{align}

\subsection{Octupole-Order ZLK, Test-Particle Limit}

\begin{figure}
    \centering
    \includegraphics[width=\colummwidth]{../../scripts/octlk/1sweepbin/composite_tp.png}
    \caption{Plot of the maximum eccentricity achieved for a binary with
    starting mutual inclination $I_{\rm 0} \in [50^\circ, 130^\circ]$ [spanning
    most of the ZLK-active region, Eq.~\eqref{eq:I0bounds}] for the fiducial
    parameters ($a = 100\;\mathrm{AU}$, $a_{\rm out, eff} = 3600\;\mathrm{AU}$,
    $m_{12} = 50M_{\odot}$, $m_3 = 30M_{\odot}$) where $e_{\rm out} = 0.6$ and
    $q \equiv m_2 / m_1 = 0.01$, i.e.\ in the test-particle regime $\eta \ll 1$
    [Eq.~\eqref{eq:def_eta}]. Simulations are run for $500t_{\rm ZLK}$
    [Eq.~\eqref{eq:def_tzlk}] including octupole-order terms, for $1000$ initial
    inclinations. Each $I_{\rm 0}$ is further simulated five times with the
    initial orbital elements $\omega$, $\omega_{\rm out}$, and $\Omega =
    \Omega_{\rm out} - \pi$ chosen randomly $\in [0, 2\pi)$% chktex 9
    for each simulation. The dotted black line shows $e_{\max}$ given by
    Eq.~\eqref{eq:emax_quad}, and $e_{\lim}$ [Eq.~\eqref{eq:def_elim}] is shown
    as the horizontal red line. The vertical purple lines denote the boundary of
    the predicted octupole-active inclinations in the test-particle regime using
    the fitting formula from \citet{MLL16} [Eq.~\eqref{eq:I_oct_MLL}].
    }\label{fig:composite_tp}
\end{figure}

To begin to understand the octupole-order effects, we first restrict our
discussion to the well-studied test-particle limit, where $m_2 = \eta = 0$.
In this limit, the dynamics are symmetric about $I_{\rm 0} = 90^\circ$, and
$I_{\rm 0, \lim} = 90^\circ$ notably. The strength of the octupole-order effects
is quantified by the dimensionless parameter
\begin{equation}
    \epsilon_{\rm oct}^{\rm (tp)} = \frac{a}{a_{\rm
        out}} \frac{e_{\rm out}}{1 - e_{\rm out}^2}.\label{eq:eps_oct_tp}
\end{equation}

When $\epsilon_{\rm oct} > 0$, $K$ is no longer conserved, and the system
becomes non-integrable and chaotic \citep{ford2000secular, katz2011long,
lithwick2011eccentric}. During this chaotic behavior, orbit flips become
possible, where $I$ flips from prograde ($I < 90^\circ$) to retrograde ($I >
90^\circ$) and back. During these orbit flips, $e$ generally attains the
limiting value $e_{\lim}$ even when $I_{\rm 0} \neq I_{\rm 0, \lim}$
\citep{lithwick2011eccentric, LML15}. The origin of these orbit flips is due to
oscillations in $K$ over long timescales $\gg t_{\rm ZLK}$ whose amplitude can
be estimated analytically assuming $\omega$, the argument of pericenter of the
inner orbit, is circulating \citep{katz2011long}. \citet{katz2011long} use this
oscillation amplitude to predict a a range of inclinations $I_0 \in \s{I_{\rm
flip, -}, I_{\rm flip, +}}$ over which orbit flips occur, as if $K$ changes
signs, the sign of $\cos I$ must also change, corresponding to an orbit flip.
However, their calculation is only accurate for $\epsilon_{\rm oct} \ll 1$; a
more general fitting formula is given by \citep{MLL16}
\begin{equation}
    \cos^2 I_{\rm flip, \pm} = \begin{cases}
        0.26\p{\frac{\epsilon_{\rm oct}}{0.1}}
            - 0.536\p{\frac{\epsilon_{\rm oct}}{0.1}}^2\\
            \quad + 12.05\p{\frac{\epsilon_{\rm oct}}{0.1}}^3
            - 16.78\p{\frac{\epsilon_{\rm oct}}{0.1}}^4.
            & \epsilon_{\rm oct} \lesssim 0.05,\\
        \cos^2 50^\circ & \epsilon_{\rm oct} \gtrsim 0.05.
    \end{cases} \label{eq:I_oct_MLL}
\end{equation}
Finally, the characteristic timescale of these orbit flips is approximately
\citep{antognini2015timescales}
\begin{equation}
    t_{\rm ZLK, oct} = t_{\rm ZLK}\frac{128\sqrt{10}}{
        15\pi\sqrt{\epsilon_{\rm oct}}}.\label{eq:def_tzlkoct}
\end{equation}

\subsection{Octupole-Order ZLK, General Masses}\label{ss:oct_gen}

\begin{figure}
    \centering
    \includegraphics[width=\colummwidth]{../../scripts/octlk/1nogw_sims/1nogw_vec.png}
    \caption{Example octupole-order, finite-mass simulation showing orbit
    flipping. We use the same fiducial parameters as in
    Fig.~\ref{fig:composite_tp} but have $e_{\rm out} = 0.6$, $q = 0.2$, and
    $I_0 = 93.5^\circ$. The four panels show the inner orbit eccentricity, the
    mutual inclination, $K$ [Eq.~\eqref{eq:def_K}], and the azimuthal angle of
    the inner eccentricity vector $\Omega_{\rm e} \equiv \tan^{-1}(e_y / e_x)$.
    Orbit flips occur when $K$ crosses the dotted line $K = K_{\rm c} \equiv
    -\eta / 2$. Furthermore, the times when the angle $\Omega_{\rm e}$ changes
    slowly are when $\omega$ is circulating, and correspond to smooth,
    sinusoidal oscillations in $K$; this is in agreement with the test-particle
    results of \citet{katz2011long}.
    }\label{fig:nogw_fiducial}
\end{figure}
When $m_1, m_2$ are comparable, Eq.~\eqref{eq:eps_oct_tp} generalizes to
\begin{equation}
    \epsilon_{\rm oct} = \frac{m_1 - m_2}{m_{12}} \frac{a}{a_{\rm out}}
        \frac{e_{\rm out}}{1 - e_{\rm out}^2}.\label{eq:eps_oct}
\end{equation}
Though the comparable-mass regime is qualitatively different from the
test-particle regime (e.g.\ Rodet et al, 2021?), many results discussed in the
previous section still hold. In particular, $K$ still smoothly oscillates when
$\omega$ is circulating, and orbit flips occur when $K$ crosses $K = K_{\rm c}
\equiv -\eta / 2$ [we see from Eq.~\eqref{eq:def_K} that this is where $\cos I$
changes signs]; see Fig.~\ref{fig:nogw_fiducial} for an example simulation when
$\eta > 0$ showing this behavior.

However, in this regime, the dynamics are no longer symmetric about $I_0 =
90^\circ$. This significantly complicates the simple octupole-active inclination
window shown in Fig.~\ref{fig:composite_tp}, which can be contrasted with the
analogous top panels of
Figs.~\ref{fig:composite_dist}--\ref{fig:composite_bindist} that illustrate the
behavior of the maximum inner binary eccentricity for a variety of system
parameters. In general, the system will always have an $e_{\lim}$-attaining
window of inclinations near $I_{\rm 0, \lim} > 90^\circ$, but only sometimes is
able to attain $e_{\lim}$ for some range of prograde inclinations $I_{\rm 0} <
90^\circ$. Notably, there is a persistent, small range of inclinations $I_{\rm
0} \approx 90^\circ$ for which $e_{\max}$ is well described by
Eq.~\eqref{eq:emax_quad} in spite of a substantial $\epsilon_{\rm oct}$. We
refer to this as the ``octupole-inactive gap'' and further investigate its
origin in Appendix~\ref{app:gap}.

\section{Dynamics With Gravitational Wave Radiation}\label{s:with_gw}

Emission of gravitational waves (GWs) further affects the evolution of $\bm{L}$
and $\bm{e}$ \citep{peters1964, LL18}. The associated orbital and eccentricity
decay rates are
\begin{align}
    \at{\frac{1}{a}\rd{a}{t}}_{\rm GW} &\equiv \frac{1}{t_{\rm GW}}\nonumber\\
        &= -\frac{64}{5}\frac{G^3 \mu m_{12}^2}{c^5a^4j^7(e)}
            \p{1 + \frac{73}{24}e^2 + \frac{37}{96}e^4}\label{eq:def_tgw},\\
    \at{\rd{e}{t}}_{\rm GW} &= -\frac{304}{15}\frac{G^3 \mu m_{12}^2}{c^5a^4}
        \frac{1}{j^{5}(e)}\p{1 + \frac{121}{304}e^2}\label{eq:dedt_gw}.
\end{align}

\subsection{Merger Fraction}

If the eccentricity maxima of the ZLK cycles are sufficiently large, GW
radiation at $e \simeq e_{\max}$ can be sufficiently enhanced that even very
wide binaries $a = 100\;\mathrm{AU}$ can be induced to merge within a Hubble
time, $10\;\mathrm{Gyr}$ \citep{LL18, LL19}. When including the chaotic
octupole-order ZLK terms, the system's starting inclination alone may not be
sufficient to determine whether it merges within a Hubble time. Instead,
multiple simulations using different $\omega$, $\omega_{\rm out}$, and $\Omega$
can be used to associate a merger fraction $f_{\rm merge}$ with each $I_{\rm
0}$. For the same range of initial conditions considered in
Section~\ref{s:background}, we simulate the octupole-order ZLK equations
and include GW radiation to understand what systems merge in under a Hubble
time. To be precise, a successful merger is defined as the inner binary reaching
$a = 0.5\;\mathrm{AU}$---since the inner binary is very eccentric at this point,
the additional time required to physically merge is negligible. The resulting
merger times $T_{\rm m}$ are shown in the bottom panels of
Figs.~\ref{fig:composite_dist}--\ref{fig:composite_bindist}. For the range of
inclinations that have nonzero $f_{\rm merge}$, we sample the range of
inclinations much more finely and run $20$ simulations for each $I_0$ ($5$ for
the more computationally expensive $a = 10\;\mathrm{AU}$ case in
Fig.~\ref{fig:composite_bindist}). The resulting merger fractions are shown in
the middle panels of
Figs.~\ref{fig:composite_dist}--\ref{fig:composite_bindist}. The
octupole-inactive gaps first identified in Section~\ref{ss:oct_gen} are also
present as gaps in $f_{\rm merge}$.

\subsection{Semianalytic Merger Criteria}\label{ss:nogw_merger}

It is clear that features in the $e_{\max}$ plots are correlated with behavior
in the $f_{\rm merge}$ and $T_{\rm m}$ plots. Here, we further develop this
connection and show that the GW-free simulations can indeed be used to predict
the outcomes of simulations with GW dissipation rather reliably. This is
advantageous as GW-free simulations are faster to run. This advantage is
essentially because they are run for $500 t_{\rm ZLK} \ll 10\;\mathrm{Gyr}$ for
our parameter regime of interest.

There are two general categories of BH mergers in our simulations: (i)
the binary merges after a single large burst of GW radiation during a
high-eccentricity ZLK cycle (``one-shot merger'') or (ii) the binary merges
gradually by emitting a varying amount of GW radiation at every eccentricity
maximum (``smooth merger'').

Towards understanding the one-shot mergers, we first define $e_{\rm os}$ to be the
$e_{\max}$ required to dissipate an order-unity fraction of the binary's orbital
energy via GW emission in a single ZLK cycle. Recalling that a binary spends a
fraction $\sim j(e_{\max})$ of each ZLK cycle near $e_{\max}$, we write
\begin{align}
    j\p{e_{\rm os}}\at{\rd{\ln a}{t}}_{e = e_{\rm os}} &\sim
        \frac{1}{t_{\rm ZLK}},\\
    j^6(e_{\rm os})
        &\equiv \frac{842}{15}
            \frac{G^3 \mu m_{12}^3}{m_3c^5a^4n}
            \p{\frac{a_{\rm out, eff}}{a}}^3.
            \label{eq:def_e_os}
\end{align}
We have approximated $e_{\rm os} \approx 1$, so $\p{1 + 73e_{\rm os}^2/24 + 37
e_{\rm os}^4/96} \approx 421 / 96$. Then, if a system ever attains $e_{\max} >
e_{\rm os}$, it is expected to undergo a one-shot merger. In particular, if
$e_{\lim} \gtrsim e_{\rm os}$ for particular parameters, then all systems that
exhibit orbit flips in the absence of GW radiation will execute one-shot mergers
when GW is considered. When $e_{\lim} \approx 1$, Eq.~\eqref{eq:def_elim}
reduces to $j(e_{\lim}) \approx 4\epsilon_{\rm GR} / 9$, which lets us rewrite
the constraint $e_{\lim} \gtrsim e_{\rm os}$ as
\begin{align}
    \p{\frac{a}{a_{\rm out, eff}}} \gtrsim{}&
        0.0149
        \p{\frac{a_{\rm out, eff}}{3600\;\mathrm{AU}}}^{-7/37}
        \p{\frac{m_{12}}{50M_{\odot}}}^{17/37}\nonumber\\
        &\times\p{\frac{30M_{\odot}}{m_3}}^{10/37}
        \p{\frac{q / (1 + q)^2}{1/4}}^{-2/37}.\label{eq:q_237}
\end{align}
In Figs.~\ref{fig:composite_dist}--\ref{fig:composite_e91p5}, this constraint is
indeed satisfied, and we see indeed that wherever the top panel suggests orbit
flipping ($e_{\max} = e_{\lim}$), the bottom panel shows $f_{\rm merge} \approx
1$. In particular, this gives us some more insight as to why some prograde
inclinations near $90^\circ$ in Fig.~\ref{fig:composite_1p2} have $0 < f_{\rm
merge} < 1$: the top panel of Fig.~\ref{fig:composite_1p2} shows that there is a
distinct sub-population of systems whose $e_{\max} < e_{\rm os}$. Such systems
are not expected to merge based on their $e_{\max}$ alone. In fact, we attribute
this behavior to the presence of initial conditions for which $\omega$ is
librating. \citet{katz2011long} point out that $K$ oscillations are attenuated
when $\omega$ is librating, which produces smaller $e_{\max}$.

Towards understanding smooth mergers, we require a characteristic eccentricity
that captures GW emission over timescales $\simeq t_{\rm ZLK, oct}$, i.e.\ over
many ZLK cycles. We define $e_{\rm eff}$ as an effective \emph{maximum}
eccentricity, such that the average GW emission is the same as a system that
undergoes ZLK eccentricity cycles to a consistent maximum eccentricity $e_{\rm
eff}$. In other words,
\begin{align}
    \ev{\rd{\ln a}{t}} &\approx -\frac{1}{t_{\rm GW, 0}}
            \ev{\frac{1 + 73e_{\max}^2/24 + 37e_{\max}^4/96}
                {j^6(e_{\max})}}\nonumber\\
        &\equiv -\frac{421/96}{t_{\rm GW, 0}}\frac{1}{j^6(e_{\rm eff})},
        \label{eq:def_e_eff}
\end{align}
where $t_{\rm GW, 0} = \p{t_{\rm GW}}_{\rm e = 0}$. Here, the angle brackets
denote averaging over $\sim t_{\rm ZLK, oct}$, i.e.\ over many ZLK cycles.
Define next the critical effective eccentricity $e_{\rm eff, c}$ such that the
inspiral time is a Hubble time, or
\begin{equation}
    \ev{\rd{\ln a}{t}} = \frac{421/96}{t_{\rm GW, 0}j^6(e_{\rm eff, c})}
        = \p{10\;\mathrm{Gyr}}^{-1}. \label{eq:def_e_eff_c}
\end{equation}
Thus, if a system is evolved using the GW-free equations of motion and satisfies
$e_{\rm eff} > e_{\rm eff, c}$, then it is expected to successfully undergo a
smooth merger within a Hubble time. In
Figs.~\ref{fig:composite_dist}--\ref{fig:composite_e91p5}, this criterion is
satisfied only if $e_{\max} > e_{\rm os}$, so the two criteria give the same
prediction for which inclinations yield successful mergers. However, in
Fig.~\ref{fig:composite_bindist}, none of the prograde inclinations satisfy
$e_{\max} > e_{\rm os}$, but there are successful mergers for $I_0 < 90^\circ$.
On the other hand, the inclination range over which $e_{\rm eff} > e_{\rm eff,
c}$ predicts these mergers quite well. This illustrates a general point about
these two semi-analytic criteria: the averaging in Eq.~\eqref{eq:def_e_eff} is
heavily weighted towards larger eccentricities, so $e_{\rm eff}$ is typically
not too different from $e_{\max}$. At the same time, for wider binaries, $t_{\rm
ZLK}$ is longer, so $e_{\rm os}$ is closer to $e_{\rm eff, c}$. Thus, in wider
binaries, the two criteria often give the same prediction, but in harder
binaries, it is more likely to satisfy $e_{\rm eff} > e_{\rm eff, c}$ without
satisfying $e_{\max} > e_{\rm os}$.

\begin{figure}
    \centering
    \includegraphics[width=\colummwidth]{../../scripts/octlk/1sweepbin/composite_1p5dist.png}
    \caption{A plot of the system dynamics for the fiducial parameter regime ($a
    = 100\;\mathrm{AU}$, $a_{\rm out, eff} = 3600\;\mathrm{AU}$, $m_{12} =
    50M_{\odot}$, $m_3 = 30M_{\odot}$) with $q = 0.5$ and $e_{\rm out} = 0.6$.
    Here, $\eta \approx 0.087$ is nonnegligible, and $\epsilon_{\rm oct} \approx
    0.007$. In the top panel, for each of $1000$ initial inclinations, we choose
    $5$ different random $\omega$, $\omega_{\rm out}$, and $\Omega$ as initial
    conditions, then run for $500t_{\rm ZLK}$ without gravitational wave
    radiation. The maximum eccentricity $e_{\max}$ (blue dots) as well as the
    effective eccentricity [Eq.~\eqref{eq:def_e_eff}; green dots] over this
    period are displayed. For comparison, $e_{\rm eff, c}$
    [Eq.~\eqref{eq:def_e_eff_c}] is given by the horizontal green dashed line,
    $e_{\rm os}$ [Eq.~\eqref{eq:def_e_os}] is shown in the horizontal blue line,
    and $e_{\lim}$ [Eq.~\eqref{eq:def_elim}] is shown in the horizontal red
    dashed line. The vertical purple lines are the fitting formula of
    \citet{MLL16} and no longer accurately describe the $e_{\lim}$-attaining
    inclination window. The black dashed line is given by
    Eq.~\eqref{eq:emax_quad}. In the bottom panel, we simulate the same range of
    initial conditions while including gravitational wave radiation and record
    the outcome, which is either a successful merger ($T_{\rm m} <
    10\;\mathrm{Gyr}$; green dots) or not ($T_{\rm m} > 10\;\mathrm{Gyr}$; blue
    triangles). The horizontal black dashed line denotes $t_{\rm ZLK}$
    [Eq.~\eqref{eq:def_tzlk}] while the horizontal blue dashed line indicates
    $t_{\rm ZLK, oct}$ [Eq.~\eqref{eq:def_tzlkoct}]. Here, each $I_{\rm 0}$ is
    run $20$ times, so we can estimate a merger fraction for each $I_{\rm
    0}$, which is depicted in the middle panel. As described in
    Section~\ref{ss:nogw_merger}, systems can be predicted to merge using the
    results shown in the top panel if $e_{\max} > e_{\rm os}$ (blue dots below
    blue line) or if $e_{\rm eff} > e_{\rm eff, c}$ (green dots below green
    line). This criterion accurately captures the regions for which the
    dissipative simulations predict nonzero merger fraction (middle panel).
    }\label{fig:composite_dist}
\end{figure}
\begin{figure}
    \centering
    \includegraphics[width=\colummwidth]{../../scripts/octlk/1sweepbin/composite_1p2dist.png}
    \caption{Same as Fig.~\ref{fig:composite_dist} but for $q = 0.2$, $\eta
    \approx 0.054$, and $\epsilon_{\rm oct} \approx 0.014$. Note that prograde
    perturbers ($I_0 < 90^\circ$) are able to induce mergers, unlike in
    Fig.~\ref{fig:composite_dist}. A ``gap'' in merger fractions for $I_0
    \approx 90^\circ$ is evident, which is discussed further in
    Appendix~\ref{app:gap}. There is a substantial range of prograde
    inclinations for which the $e_{\max}$ behavior is double-valued and for
    which the merger fraction $0 < f_{\rm merge} < 1$. We attribute this
    subpopulation to initial conditions where $\omega$ is librating, for which
    octupole-induced oscillations in $K$ are suppressed \citep{katz2011long} and
    $e_{\max}$ does not attain sufficiently large values to induce merger.
    }\label{fig:composite_1p2}
\end{figure}
\begin{figure}
    \centering
    \includegraphics[width=\colummwidth]{../../scripts/octlk/1sweepbin/composite_e91p5dist.png}
    \caption{Same as Fig.~\ref{fig:composite_dist} but $e_{\rm out} = 0.9$
    while holding $a_{\rm out, eff} = 3600\;\mathrm{AU}$ constant. Here, $\eta
    \approx 0.12$ while $\epsilon_{\rm oct} \approx 0.019$. Note that even
    though $\epsilon_{\rm oct}$ is larger here than in
    Fig.~\ref{fig:composite_1p2}, no mergers are possible here with a prograde
    perturber. This is because $\eta$ is more than twice as large for these
    parameters.
    }\label{fig:composite_e91p5}
\end{figure}
\begin{figure}
    \centering
    \includegraphics[width=\colummwidth]{../../scripts/octlk/1sweepbin/bindist.png}
    \caption{Same as Figs.~\ref{fig:composite_tp}--\ref{fig:composite_1p2}
    but for a compact inner binary; the parameters are $a = 10\;\mathrm{AU}$,
    $a_{\rm out, eff} = 700\;\mathrm{AU}$, $m_{12} = 50M_{\odot}$, $m_3 =
    30M_{\odot}$, and $e_{\rm out} = 0.9$, $q = 0.4$. Note that when the
    perturber is prograde ($I_0 < 90^\circ$), $e_{\max} <
    e_{\rm os}$ but $e_{\rm eff} > e_{\rm eff, c}$, which predicts that systems
    are able to merge. This prediction is borne out by simulations with GW
    radiation (middle and bottom panels). }\label{fig:composite_bindist}
\end{figure}

\section{Mass Ratio Distribution of Merging Black Hole Binaries}

A physically representative ensemble is beyond the scope of this paper, and we
instead focus on simple, illustrative populations of BBH to demonstrate the
effect.

\subsection{Fixed Tertiary Eccentricity}

We first consider the simplified case where $e_{\rm out}$ is fixed at a few
specific values and compute the merger fraction as a function of the mass ratio
$q$. For this study, we let $\cos I_0$ be drawn uniformly from the range $[-1,
1]$, let $\omega$, $\omega_{\rm out}$, and $\Omega$ be drawn from the range $[0,
2\pi)$% chktex 9
and adopt the fiducial parameters. The resulting merger fraction distribution is
shown as the solid dots and lines in Fig.~\ref{fig:total_merger_fracs}, both as
a function of $q$ and as a function of $\epsilon_{\rm oct}$. For each
simulation, we also use GW-free criteria introduced in
Section~\ref{ss:nogw_merger} to predict the outcome and generate a merger
fraction, shown as the crosses and dashed lines in
Fig.~\ref{fig:total_merger_fracs}. It is clear both that smaller $q$
significantly enhances the merger fraction and that the GW-free criteria do an
adequate job of tracking the merger fraction.

In the right panel of Fig.~\ref{fig:total_merger_fracs}, we see that the merger
fractions for the three $e_{\rm out}$ values overlap for sufficiently small
$\epsilon_{\rm oct}$. This implies that $f_{\rm merge}$ depends only on
$\epsilon_{\rm oct}$ for these values, and not the values of $q$ and $e_{\rm
out}$ independently. However, for larger $\epsilon_{\rm oct}$, the three curves
diverge. This is attributed to their different $\eta$ values: for sufficiently
small $\epsilon_{\rm oct}$, no prograde initial inclinations not successfully
merge (e.g.\ Fig.~\ref{fig:composite_dist}). However, the size of the
octupole-inactive gap is different for different $\eta$, so prograde
inclinations begin to merge successfully at different $e_{\rm out}$ thresholds,
breaking the degeneracy. This again illustrates the importance of understanding
the octupole-inactive gap, which we comment on in Appendix~\ref{app:gap}.

\textcolor{Corr}{Rewrite if not desire $a / a_{\rm out, eff}$ constant,
probably, right?} Finally, for comparison, we also show a version of
Fig.~\ref{fig:total_merger_fracs} for $a = 50\;\mathrm{AU}$ and $a_{\rm out,
eff} = 1800\;\mathrm{AU}$ in Fig.~\ref{fig:sweepbin_simple}. According to
Eq.~\eqref{eq:q_237}, this new parameter regime also satisfies $e_{\lim} >
e_{\rm os}$, so all systems that undergo orbit flips will also successfully
merge. As such, we expect their merger fraction behaviors to be quite similar,
and this is indeed the case.
\begin{figure*}
    \centering
    \includegraphics[width=0.8\textwidth]{../../scripts/octlk/1sweepbin/total_merger_fracs.png}
    \caption{From Figs.~\ref{fig:composite_dist}--\ref{fig:composite_e91p5},
    we can compute the total merger fraction in the presence of GW radiation
    assuming $\cos I_0$ is uniformly distributed $\in [-1, 1]$. We do this for
    three values of $e_{\rm out}$ and six values of $q$ and show the results
    with solid dots. The X's show the results when GW-free simulations are
    instead used to predict the total merger fraction, i.e.\ using the
    criteria shown in the top panels of
    Figs.~\ref{fig:composite_dist}--\ref{fig:composite_e91p5}; good agreement
    is observed. }\label{fig:total_merger_fracs}
\end{figure*}
\begin{figure*}
    \centering
    \includegraphics[width=0.8\textwidth]{../../scripts/octlk/1sweepbin_simple/simple.png}
    \caption{Same as Fig.~\ref{fig:total_merger_fracs} but for $a =
    50\;\mathrm{AU}$ and $a_{\rm out, eff} = 1800\;\mathrm{AU}$. Since this
    configuration has the same $a / a_{\rm out, eff}$ ratio, as well as $\eta$
    and $\epsilon_{\rm oct}$, as Fig.~\ref{fig:total_merger_fracs}, the two
    plots are very similar. \textcolor{Corr}{TODO\@: try instead with $\p{50,
    3600}$ again? I think the percentages are all very small there
    though.}}\label{fig:sweepbin_simple}
\end{figure*}

\subsection{Distribution of Tertiary Eccentricities}

We can also draw $e_{\rm out} \in [0, 0.9]$ with both a uniform probability
distribution and a thermal one $P(e_{\rm out}) \propto e_{\rm out}$, and examine
$f_{\rm merge}$ as a function of $q$ (the true $e_{\rm out}$ distribution is
likely to be between these two cases). The results are shown in
Fig.~\ref{fig:popsynth}, where each $q$ has $1000$ realizations using random
$e_{\rm out}$, $\cos I_0$, $\omega$, $\omega_{\rm out}$, and $\Omega$. If
$e_{\rm out}$ is thermally distributed (red), binaries with smaller $q$ are even
more likely to merge, since $\epsilon_{\rm oct}$ tends to be larger. To
understand the impact of our random sampling, we also compute a merger fraction
using the GW-free criteria of Section~\ref{ss:nogw_merger} over a much denser,
uniform grid of $(\cos I_0, e_{\rm out}, q)$, shown as the blue dotted line.
While the resolution is still limited, it is clear that the true $f_{\rm
merge}$ should be a monotonic function of $q$. The distributions and medians of
the merger time and eccentricity in the LISA and LIGO bands are also shown. For
the LISA and LIGO band eccentricities, the inner binaries are evolved from when
they reach $0.5\;\mathrm{AU}$ to physical merger using
Eqs.~(\ref{eq:def_tgw}--\ref{eq:dedt_gw}). While all of these eccentricities are
quite small, this has to do with our using the double-averaged equations of
motion. Both the single-averaged and the full n-body equations of motion produce
larger eccentricities in the LISA and LIGO bands \citep{LL19}.

For comparison, we also show the results when $a_{\rm out, eff} =
5500\;\mathrm{AU}$, with all other parameters unchanged, in
Fig.~\ref{fig:popsynth5500}. While $f_{\rm merge}$ is lower than for $a_{\rm
out, eff} = 3600\;\mathrm{AU}$, there is still a large increase between large
and small $q$. This is expected, since $e_{\lim} > e_{\rm os}$ by
Eq.~\eqref{eq:q_237}, and the $e_{\lim}$-achieving range of inclinations changes
dramatically for the range of $\epsilon_{\rm oct}$ values in question.
\begin{figure}
    \centering
    \includegraphics[width=\columnwidth]{../../scripts/octlk/1popsynth/a2eff3600.png}
    \caption{Merger fractions with the fiducial parameters obtained by randomly
    drawing $\cos I_0$ uniformly distributed $\in [-1, 1]$ and drawing $e_{\rm
    out}$ from either a uniform distribution ($e_{\rm out} \in [0, 0.9]$; black)
    or a thermal distribution ($P(e_{\rm out}) \propto e_{\rm out}$, $e_{\rm
    out} \in [0, 0.9]$; red). The blue dotted line instead samples a dense,
    uniform grid in $\cos I_0$, $e_{\rm out}$, and $q \in [0.01, 1]$ using the
    GW-free criteria, confirming that the non-monotonicity of the $f_{\rm
    merge}$ calculated from simulations including GW radiation is due to the
    random sampling involved. The middle panel shows the merger time for
    successful mergers (the median is denoted with the large black dot). The
    bottom panel shows the binary eccentricity in the LISA band
    ($0.1\;\mathrm{Hz}$; blue) and in the LIGO band ($10 \;\mathrm{Hz}$; red).
    }\label{fig:popsynth}
\end{figure}
\begin{figure}
    \centering
    \includegraphics[width=\columnwidth]{../../scripts/octlk/1popsynth/a2eff5500.png}
    \caption{Same as Fig.~\ref{fig:popsynth} but for $a_{\rm out, eff} =
    5500\;\mathrm{AU}$. }\label{fig:popsynth5500}
\end{figure}

\subsection{Effect of Smaller Mass Ratios}

The above results may seem somewhat counterintuitive at first glance, since
$t_{\rm GW} \propto \mu$ which should grow as $q$ is decreased. This effect
should increase the merger times and decrease the merger fractions, but the data
illustrate otherwise. Why is this?

The answer is that, if $e_{\lim} > e_{\rm os}$, then many binaries execute
one-shot mergers when undergoing an orbit flip. Since $t_{\rm oct,
ZLK} \ll 10\;\mathrm{Gyr}$ typically, this implies that octupole-ZLK-induced
binary merger fractions determined by what initial conditions execute orbit
flips, and not by the detailed GW radiation rate. Additionally,
Eq.~\eqref{eq:q_237} shows that, while $e_{\lim} > e_{\rm os}$ is indeed
violated if $q$ is decreased sufficiently with all else held constant, the
dependence is extremely weak. As such, $f_{\rm merge}$ is expected to be very
nearly constant in $q$ for all physical values of $q$, as shown in
\begin{figure}
    \centering
    \includegraphics[width=\colummwidth]{../../scripts/octlk/1popsynth/a2eff_nogw_lowq3600.png}
    \caption{Same as blue dashed line of the top panel of
    Fig.~\ref{fig:popsynth} but extended to very small $q$. Due to the very weak
    $q$ dependence in Eq.~\eqref{eq:q_237}, $f_{\rm merge}$ is expected to
    depend very weakly on $q$ when $q \ll 1$ (such that $\epsilon_{\rm oct}$ is
    approximately constant), which agrees with the simulation
    results.}\label{fig:popsynth_lowq}
\end{figure}

\section{Conclusion and Discussion}

We have considered the dynamics of a comparable-mass tertiary-induced binary BH
merger by studying the ZLK effect at octupole order. We showed for sufficiently
heirarchical binaries [Eq.~\eqref{eq:q_237} is satisfied, and the
double-averaged equations of motion are valid] that the merger fraction is
dramatically larger for binaries with smaller mass ratios (see
e.g.\ Fig.~\ref{fig:popsynth}). We showed that, due to the nonneglible angular
momentum ratio $\eta$ in such systems, the inclinations for which extreme
eccentricities can be attained exhibit much more complex behavior than in the
test-particle case \citep[as studied by][]{katz2011long, lithwick2011eccentric,
LML15, MLL16}. Notably, the outcomes are no longer symmetric about $I_0 =
90^\circ$, and there is a gap near $I_0 \approx 90^\circ$ for which extreme
eccentricity excitation is always forbidden
(Figs.~\ref{fig:composite_dist}--\ref{fig:composite_bindist}).

We discuss the implications of our results in context of the observed binary BH
mergers from LIGO/VIRGO\@. When including the latest data from the O3a observing
run, the observed distribution in $q$ significantly prefers larger mass ratios.
In particular, if $P(q) = q^{\beta_q}$, then $\beta_q > 0$ at $89\%$ or more
\citep{LIGOO3a}. At the same time, the primordial $q$ distribution in massive
stellar binaries is generally expected to prefer smaller $q$ to varying extents
\citep[e.g.][]{sana2012binary, moe2017mind}. In conjunction with our results
in this paper, these results appear to be in tension with observation. However,
such a simple analysis omits many possible confounding effects during the
transition from stellar binary to isolated BH triple (e.g.\ supernova kicks,
stellar fly-bys), and a detailed analysis beyond the scope of this paper is
required to understand the cumulative effect of these complications.

Most of our analysis above assumes $e_{\lim} > e_{\rm os}$
[Eq.~\eqref{eq:q_237}]. However, if this is not satisfied, the dependency of
$f_{\rm merge}$ on $q$ becomes much weaker. \textcolor{Corr}{I will add a plot
for this if we prefer and move it into S4.1}.

Finally, while this work was conducted using the double-averaged (DA) equations
of motion, most of the qualitative behavior is expected to be robust to
higher-order approximations \citep[e.g.][]{LL19}. Of particular interest is how
one-shot mergers restore the validity of the DA equations. In particular, the DA
approximation reqires the orbital period $P_{\rm out}$ of the outer binary be
much longer than all dynamical timescales of the inner binary, i.e.
\begin{equation}
    t_{\rm ZLK}j\p{e_{\max}} \lesssim P_{\rm out},
\end{equation}
For our fiducial parameter regime, this is not strictly satisfied when
$e_{\max}$ reaches $e_{\lim}$. However, it is satisfied when $e_{\max}$ reaches
$e_{\rm os} < e_{\lim}$, so the inner binary will begin to decay significantly,
as a one-shot merger, before the DA approximation breaks down.

\section{Acknowledgements}\label{s:ack}

YS is supported by the NASA FINESST grant 19-ASTRO19-0041.%chktex 8

\bibliographystyle{mnras}
\bibliography{Su_EZLK}

\clearpage
\onecolumn

\appendix

\section{Origin of Octupole-Inactive Gap}\label{app:gap}

\begin{figure}
    \centering
    \includegraphics[width=\colummwidth]{../../scripts/octlk/1sweepbin_emax/1p2dist.png}
    \caption{Top panel is the same as the top panel of
    Fig.~\ref{fig:composite_1p2}. The bottom panel shows the range of
    oscillation in $K$, denoted by $K_{\min}$ and $K_{\max}$, for the same
    parameters. The critical $K = \eta / 2$ for orbit flipping is shown with the
    horizontal red line. It can be seen that when $K_{\min} < \eta / 2 <
    K_{\max}$ that $e_{\max} \simeq e_{\lim}$ in the top panel. It is therefore
    clear that the gap in $e_{\max}$ excitation is due to a limited range of
    oscillation in $K$.
    }\label{fig:kdist}
\end{figure}

\begin{itemize}
    \item Show the $K_{\min}$ and $K_{\max}$ plot, Fig.~\ref{fig:kdist}.

    \item Is because librating! example simulation where librating,
        Fig.~\ref{fig:nogw_circ}.

    \item Point out that the character of the circulation-libration transition
        changes when $\epsilon_{\rm oct}$ is substantial, to not only depend on
        $\omega$, Fig.~\ref{fig:dW}.
\end{itemize}

\begin{figure}
    \centering
    \includegraphics[width=\colummwidth]{../../scripts/octlk/1nogw_sims/1nogw_vec88.png}
    \caption{Example simulation where $\Omega_{\rm e}$ is primarily circulating,
    which suppresses the amplitude of oscillation of $K$. As a result, the orbit
    does not flip.}\label{fig:nogw_circ}
\end{figure}
\begin{figure}
    \centering
    \includegraphics[width=\colummwidth]{../../scripts/octlk/2dW_sweeps/2_dWsweeps6_2_dual.png}
    \caption{Plot of $\Delta \Omega_{\rm e}$, the change in the co-longitude of
    the inner eccentricity vector $\Omega_{\rm e}$ over a ZLK cycle, for
    different initial conditions and when octupole terms are off/on. In the top
    panel, the octupole terms are neglected, and $\omega_{\rm 0} = 0$ results in
    circulation ($\Delta \omega = 0$ and $\Delta \Omega_{\rm e} = 180^\circ$)
    while $\omega_{\rm 0} = \pi / 2$ results in libration, as expected. In the
    bottom panel, the octupole terms are included, and it is seen that the
    circulation-libration transition no longer depends on $\omega_{\rm 0}$ but
    instead on $I_0$. }\label{fig:dW}
\end{figure}

\label{lastpage} % chktex 24
\end{document} % chktex 17
