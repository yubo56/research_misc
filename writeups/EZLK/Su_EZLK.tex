    \documentclass[
        fleqn,
        usenatbib,
        % referee,
    ]{mnras}
    \usepackage{
        amsmath,
        amssymb,
        newtxtext,
        newtxmath,
        ae, aecompl,
        graphicx,
        booktabs,
        xcolor,
    }
    \usepackage[T1]{fontenc}

    \newcommand*{\scinot}[2]{#1\times10^{#2}}
    \newcommand*{\rd}[2]{\frac{\mathrm{d}#1}{\mathrm{d}#2}}
    \newcommand*{\rtd}[2]{\frac{\mathrm{d}^2#1}{\mathrm{d}#2^2}}
    \newcommand*{\pd}[2]{\frac{\partial#1}{\partial#2}}
    \newcommand*{\ptd}[2]{\frac{\partial^2#1}{\partial#2^2}}
    % inline
    \newcommand*{\mdil}[2]{\mathrm{D}#1/\mathrm{D}#2}
    \newcommand*{\pdil}[2]{\partial#1/\partial#2}
    \newcommand*{\rdil}[2]{\mathrm{d}#1/\mathrm{d}#2}
    \newcommand*{\md}[2]{\frac{\mathrm{D}#1}{\mathrm{D}#2}}
    \newcommand*{\at}[1]{\left.#1\right|}
    \newcommand*{\abs}[1]{\left|#1\right|}
    \newcommand*{\ev}[1]{\left\langle#1\right\rangle}
    \newcommand*{\p}[1]{\left(#1\right)}
    \newcommand*{\s}[1]{\left[#1\right]}
    \newcommand*{\z}[1]{\left\{#1\right\}}
    \newcommand*{\bm}[1]{\mathbf{#1}}
    \newcommand*{\uv}[1]{\hat{\mathbf{#1}}}
    \DeclareMathOperator*{\med}{med}
    \DeclareMathOperator*{\erf}{erf}

    \colorlet{Corr}{red}

    \newlength{\colummwidth}
    \setlength{\colummwidth}{246.0pt} % columnwidth for reprint

\title[Mass Ratio Distribution]{The Mass Ratio Distribution of
Black Hole Mergers Induced by a Comparable Mass Tertiary}
\author[Y. Su et\ al.]{
Yubo Su,$^1$,
Bin Liu,$^{1,2}$,
Dong Lai$^1$
\\
$^1$ Cornell Center for Astrophysics and Planetary Science, Department of
Astronomy, Cornell University, Ithaca, NY 14853, USA\\
$^2$ Niels Bohr International Academy, Niels Bohr Institute, Blegdamsvej 17,
2100 Copenhagen, Denmark.
}

\date{Accepted XXX\@. Received YYY\@; in original form ZZZ}

\pubyear{2021}

\begin{document}\label{firstpage}
\pagerange{\pageref{firstpage}--\pageref{lastpage}}
\maketitle

\begin{abstract}
    Abstract
\end{abstract}

\begin{keywords}
binaries:close -- stars:black holes % chktex 8
\end{keywords}

\section{Introduction}\label{s:intro}

We study the von Zeipel-Lidov-Kozai effect (ZLK) for eccentric perturbers to
octupole order, also sometimes known as the eccentric Kozai mechanism
\citep[e.g.][]{lithwick2011eccentric}.

LIGO O3a results are in Fig.~\ref{fig:qhist}.
\begin{figure}
    \centering
    \includegraphics[width=\colummwidth]{../../scripts/octlk/3qhist.png}
    \caption{Histogram of the mass ratio $q \equiv m_2 / m_1$ in the LIGO O3a
    dataset, excluding the two NS-NS mergers but including GW190814
    \citep{LIGOO3a}.}\label{fig:qhist}
\end{figure}

The mass ratio distribution among ZLK-induced BH binary mergers has already been
noted \citep[see Fig.~10 of][]{silsbee2017lidov}, but the origin of the effect
has not been carefully studied.

\section{Dynamics Without Gravitational Wave Radiation}\label{s:background}

Consider two BHs orbiting each other with masses $m_1$ and $m_2$ on a orbit with
semimajor axis $a$, eccentricity $e$, and angular momentum vector $\bm{L} \equiv
L\uv{L}$. Additionally, consider a third BH of mass $m_3$ orbiting this binary
with semimajor axis $a_{\rm out}$ and eccentricity $e_{\rm out}$, with angular
momentum vector $\bm{L}_{\rm out} \equiv L_{\rm out} \uv{L}_{\rm out}$. We
define $q \equiv m_2 / m_1$, the mass ratio, and $m_{12} = m_1 + m_2$, the total
mass of the inner binary. Our fiducial parameters are: $a = 100\;\mathrm{AU}$,
$a_{\rm out, eff} = 4500\;\mathrm{AU}$, $e_{\rm out} = 0.6$, $m_{12} =
50M_{\odot}$, $m_3 = 30M_{\odot}$. We take the initial inner binary eccentricity
to be $e_0 = 10^{-3}$.

To define the coordinate system, we choose the $\uv{z}$ axis pointing along the
total angular momentum, $\bm{L}_{\rm tot} = \bm{L} + \bm{L}_{\rm out}$. In this
coordinate system, we denote the longitude of the ascending node and argument of
pericenter of the inner and outer orbits by $\Omega$, $\omega$, $\Omega_{\rm
out}$, and $\omega_{\rm out}$ respectively. Note that, by conservation of
angular momentum, $\Omega_{\rm out} = \Omega + \pi$.

We study this system using the octupole-order, double-averaged vectorial ZLK
equations from \citet{LML15}, including general relativistic apsidal precession,
a first post-Newtonian order (1PN) effect. For the remainder of this section, we
review a few results in the absence of gravitational wave (GW) radiation, a
2.5PN effect (to be considered in Section~\ref{s:with_gw}). We group these
results by increasing order of approximation, starting by ignoring the
octupole-order effects entirely.

\subsection{Quadrupole-Order ZLK}

The primary feature of the quadrupole-order dynamics are cycles of the inner
binary's eccentricity from $e \ll 1$ to a consistent maximum $e_{\max} \approx
1$ with period $\sim t_{\rm ZLK}$, where
\begin{equation}
    t_{\rm ZLK} = \frac{1}{n}\frac{m_{12}}{m_3}
            \p{\frac{a_{\rm out, eff}}{a}}^3,\label{eq:def_tzlk}
\end{equation}
where $n \equiv \sqrt{Gm_{12} / a^3}$ is the mean motion of the inner binary,
and $a_{\rm out, eff} \equiv a_{\rm out}\sqrt{1 - e_{\rm out}^2}$. During these
eccentricity cycles, there are two conserved quantities, the total energy and
the quantity \citep{LML15}
\begin{equation}
    K \equiv j(e) \cos I - \eta e^2 / 2. \label{eq:def_K}
\end{equation}
Here, $j(e) \equiv \sqrt{1 - e^2}$, $I \equiv \cos^{-1}(\uv{L} \cdot
\uv{L}_{\rm out})$ is the mutual inclination, and $\eta$ is the ratio of the
magnitudes of the angular momenta:
\begin{equation}
    \eta \equiv \at{\frac{L}{L_{\rm out}}}_{e = 0}
        = \frac{\mu}{\mu_{\rm out}}\s{\frac{m_{12}a}
            {m_{123}a_{\rm out}\p{1 - e_{\rm out}^2}}}^{1/2},\label{eq:def_eta}
\end{equation}
where $\mu \equiv m_1m_2 / m_{12}$, $m_{123} = m_{12} + m_3$, and $\mu_{\rm out}
\equiv m_{12} m_3 / m_{123}$. Note that when $\eta = 0$, $K$ reduces to the
classical ``Kozai constant''.

During these eccentricity cycles, the binary spends a fraction $\sim
j(e_{\max})$ of each ZLK eccentricity cycle near $e_{\max}$
\citep{anderson2016formation}. The exact value of $e_{\max}$ depends on the
general relativistic apsidal precession, which induces precession of the
eccentricity vector $\bm{e}$ of the inner binary as
\begin{equation}
    \at{\rd{\bm{e}}{t}}_{\rm GR} = \Omega_{\rm GR}\uv{L} \times \bm{e}
        = \frac{3Gnm_{12}}{c^2aj^2(e)}\uv{L} \times \bm{e}.
\end{equation}
It is useful to quantify the relative strength of this precession via the
parameter
\begin{equation}
    \epsilon_{\rm GR} \equiv \p{\Omega_{\rm GR} t_{\rm ZLK}}_{e = 0}
        = \frac{3Gm_{12}}{c^2} \frac{m_{12}}{m_3}\frac{a_{\rm out, eff}^3}{a^4}.
\end{equation}
It can then be shown that, when $e_0 \ll 1$, $e_{\max}$ depends on the initial
mutual inclination $I_{\rm 0}$ via \citep{LML15, anderson2016formation}:
\begin{align}
    \frac{3}{8}\frac{j^2(e_{\max}) - 1}{j^2(e_{\max})}\Big[&
        5\p{\cos I_0 + \frac{\eta}{2}}^2
        - \p{3 + 4\eta \cos I_0 + \frac{9}{4}\eta^2}j^2(e_{\max})
            \nonumber\\
        &+ \eta^2 j^4(e_{\max})
    \Big] + \epsilon_{\rm GR}\s{1 - 1 / j(e_{\max})} = 0.\label{eq:emax_quad}
\end{align}
This only has solutions when $(\cos I_{\rm 0})_- \leq \cos I_{\rm 0} \leq (\cos
I_{\rm 0})_+$ where \citep{anderson2016formation}
\begin{equation}
    \p{\cos I_{\rm 0}}_{\pm} = \frac{1}{10}\p{-\eta \pm \sqrt{\eta^2 + 60 -
        20\epsilon_{\rm GR}}}.\label{eq:I0bounds}
\end{equation}
For $I_{\rm 0}$ outside of this range, there are no ZLK oscillations; this
reduces to the well-known $\cos^2 I_{\rm 0} \leq 3/5$ when $\eta = \epsilon_{\rm
GR} = 0$.

Additionally, the maximum value of $e_{\max}$, denoted $e_{\lim}$, occurs when
$I_{\rm 0} = I_{\rm 0, \lim}$, where
\begin{equation}
    \cos I_{\rm 0, \lim} = \frac{\eta}{2}\s{\frac{4}{5}j^2(e_{\lim}) -
        1}.\label{eq:I0lim}
\end{equation}
Note that $I_{\rm 0, \lim} \geq 90^\circ$ with equality only when $\eta
= 0$. Substituting Eq.~\eqref{eq:I0lim} into Eq.~\eqref{eq:emax_quad}
gives
\begin{align}
    \frac{3}{8}\p{j^2(e_{\lim}) - 1}&\s{-3 + \frac{\eta^2}{4}
        \p{\frac{4}{5}j^2(e_{\lim}) - 1}}\nonumber\\
        &+ \epsilon_{\rm GR}\s{1 - 1 / j(e_{\lim})} = 0.
        \label{eq:def_elim}
\end{align}

\subsection{Octupole-Order ZLK, Test-Particle Limit}

\begin{figure}
    \centering
    \includegraphics[width=\colummwidth]{../../scripts/octlk/1sweepbin/composite_tp.png}
    \caption{Plot of the maximum eccentricity achieved for a binary with
    starting mutual inclination $50^\circ < I_{\rm 0} < 130^\circ$ [spanning
    most of the ZLK-active region, Eq.~\eqref{eq:I0bounds}] for the fiducial
    parameters ($a = 100\;\mathrm{AU}$, $a_{\rm out, eff} = 3600\;\mathrm{AU}$,
    $m_{12} = 50M_{\odot}$, $m_3 = 30M_{\odot}$) where $e_{\rm out} = 0.6$ and
    $q \equiv m_2 / m_1 = 0.01$, i.e.\ in the test-particle regime $\eta \ll 1$
    [Eq.~\eqref{eq:def_eta}]. Simulations are run for $500t_{\rm ZLK}$
    [Eq.~\eqref{eq:def_tzlk}] including octupole-order terms, for $1000$ initial
    inclinations. Each $I_{\rm 0}$ is further simulated five times with the
    initial orbital elements $\omega$, $\omega_{\rm out}$, and $\Omega =
    \Omega_{\rm out} - \pi$ chosen randomly $\in [0, 2\pi)$% chktex 9 for each
    simulation. The dotted black line shows $e_{\max}$ given by
    Eq.~\eqref{eq:emax_quad}, and $e_{\lim}$ [Eq.~\eqref{eq:def_elim}] is shown
    as the horizontal red line. The vertical purple lines denote the boundary of
    the predicted octupole-active inclinations in the test-particle regime using
    the fitting formula from \citet{MLL16} [Eq.~\eqref{eq:I_oct_MLL}].
    }\label{fig:composite_tp}
\end{figure}

To begin to understand the ZLK dynamics at octupole order, we first restrict our
discussion to the well-studied test-particle limit, where $m_2 = \eta = 0$. In
this limit, the equations of motion are symmetric about $I_{\rm 0} = 90^\circ$,
and $I_{\rm 0, \lim} = 90^\circ$. The strength of the octupole-order effects is
quantified by the dimensionless parameter \citep{katz2011long,
lithwick2011eccentric, naoz2012formation}
\begin{equation}
    \epsilon_{\rm oct}^{\rm (tp)} = \frac{a}{a_{\rm
        out}} \frac{e_{\rm out}}{1 - e_{\rm out}^2}.\label{eq:eps_oct_tp}
\end{equation}

When $\epsilon_{\rm oct} > 0$, $K$ is no longer conserved, and the system
becomes non-integrable and chaotic \citep{ford2000secular, katz2011long,
lithwick2011eccentric}. As a result, orbit flips become possible, where $I$
flips from prograde ($I < 90^\circ$) to retrograde ($I > 90^\circ$) and back.
During these orbit flips, $e$ generally attains the limiting value $e_{\lim}$
even when $I_{\rm 0} \neq I_{\rm 0, \lim}$ \citep{lithwick2011eccentric, LML15}.
The origin of these orbit flips is due to oscillations in $K$ over long
timescales $\gg t_{\rm ZLK}$: if $K$ changes signs, the sign of $\cos I$ must
also change, yielding an orbit flip. \citet{katz2011long} estimate the amplitude
of these oscillations analytically, assuming $\omega$ is circulating, and use
their estimate to predict a a range of inclinations $I_{\rm flip, -} \lesssim
I_0 \lesssim I_{\rm flip, +}$ over which orbit flips occur; we refer to this as
the \emph{octupole-active window}. However, their calculation is only accurate
for $\epsilon_{\rm oct} \ll 1$; a more general fitting formula is given by
\citep{MLL16}
\begin{equation}
    \cos^2 I_{\rm flip, \pm} = \begin{cases}
        0.26\p{\frac{\epsilon_{\rm oct}^{\rm(tp)}}{0.1}}
            - 0.536\p{\frac{\epsilon_{\rm oct}^{\rm(tp)}}{0.1}}^2\\
            \quad + 12.05\p{\frac{\epsilon_{\rm oct}^{\rm(tp)}}{0.1}}^3
            - 16.78\p{\frac{\epsilon_{\rm oct}^{\rm(tp)}}{0.1}}^4.
            & \epsilon_{\rm oct}^{\rm(tp)} \lesssim 0.05,\\
        0.45 & \epsilon_{\rm oct}^{\rm(tp)} \gtrsim 0.05.
    \end{cases} \label{eq:I_oct_MLL}
\end{equation}
The accuracy of this fitting formula is displayed in
Fig.~\ref{fig:composite_tp}, where we evolve a system for which $m_2 \ll m_1$
for $500t_{\rm ZLK}$ using a range of $I_0$. We record the maximum eccentricity
$e_{\max}$ obtained by the inner binary, and find that it attains $e_{\lim}$ if
and only if $I_0$ is within the octupole-active window as givenc by
Eq.~\eqref{eq:I_oct_MLL}.

Finally, the characteristic timescale of these orbit flips is approximately
\citep{antognini2015timescales}
\begin{equation}
    t_{\rm ZLK, oct} = t_{\rm ZLK}\frac{128\sqrt{10}}{
        15\pi\sqrt{\epsilon_{\rm oct}^{\rm(tp)}}}.\label{eq:def_tzlkoct}
\end{equation}

\subsection{Octupole-Order ZLK, General Masses}\label{ss:oct_gen}

\begin{figure}
    \centering
    \includegraphics[width=\colummwidth]{../../scripts/octlk/1nogw_sims/1nogw_vec.png}
    \caption{Example octupole-order, finite-mass simulation showing orbit
    flipping. We use the same fiducial parameters as in
    Fig.~\ref{fig:composite_tp} but have $e_{\rm out} = 0.6$, $q = 0.2$, and
    $I_0 = 93.5^\circ$. The four panels show the inner orbit eccentricity, the
    mutual inclination, $K$ [Eq.~\eqref{eq:def_K}], and the azimuthal angle of
    the inner eccentricity vector $\Omega_{\rm e} \equiv \tan^{-1}(e_y / e_x)$.
    Orbit flips occur when $K$ crosses the dotted line $K = K_{\rm c} \equiv
    -\eta / 2$. Furthermore, the times when the angle $\Omega_{\rm e}$ changes
    slowly are when $\omega$ is circulating, and correspond to smooth,
    sinusoidal oscillations in $K$; this is in agreement with the test-particle
    results of \citet{katz2011long}.
    }\label{fig:nogw_fiducial}
\end{figure}
When $m_1$ and $m_2$ are comparable, Eq.~\eqref{eq:eps_oct_tp} generalizes to
\citep{LML15, anderson2016formation, LL18}
\begin{equation}
    \epsilon_{\rm oct} = \frac{m_1 - m_2}{m_{12}} \frac{a}{a_{\rm out}}
        \frac{e_{\rm out}}{1 - e_{\rm out}^2}.\label{eq:eps_oct}
\end{equation}
As we will see, the comparable-mass regime exhibits much new behavior compared
to the test-particle regime \citep[see also][]{rodet_inprep}, but other results
from the previous section still hold. In particular, $K$ still slowly fluctuates
when $\omega$ is circulating, and orbit flips occur when $K$ crosses $K = K_{\rm
c} \equiv -\eta / 2$ [we see from Eq.~\eqref{eq:def_K} that this is where $\cos
I$ changes signs]; see Fig.~\ref{fig:nogw_fiducial} for an example simulation
when $\eta > 0$ showing this behavior.

However, in this regime, the dynamics are no longer symmetric about $I_0 =
90^\circ$. This significantly complicates the simple octupole-active inclination
window shown in Fig.~\ref{fig:composite_tp}, which can be contrasted with the
analogous top panels of
Figs.~\ref{fig:composite_dist}--\ref{fig:composite_bindist} that illustrate the
behavior of the maximum inner binary eccentricity for a variety of system
parameters. In general, the system will always have an $e_{\lim}$-attaining
window of inclinations near $I_{\rm 0, \lim} > 90^\circ$, but only sometimes is
able to attain $e_{\lim}$ for some range of prograde inclinations $I_{\rm 0} <
90^\circ$. Notably, there is a persistent, small range of inclinations $I_{\rm
0} \approx 90^\circ$ for which $e_{\max}$ is well described by
Eq.~\eqref{eq:emax_quad} in spite of a substantial $\epsilon_{\rm oct}$. We
refer to this as the ``octupole-inactive gap'' and further investigate its
origin in Appendix~\ref{app:gap}.

\section{Dynamics With Gravitational Wave Radiation}\label{s:with_gw}

Emission of gravitational waves (GWs) also affects the evolution of $\bm{L}$ and
$\bm{e}$ \citep{peters1964, LL18}. The associated orbital and eccentricity decay
rates are
\begin{align}
    \at{\frac{1}{a}\rd{a}{t}}_{\rm GW} &\equiv \frac{1}{t_{\rm GW}}\nonumber\\
        &= -\frac{64}{5}\frac{G^3 \mu m_{12}^2}{c^5a^4j^7(e)}
            \p{1 + \frac{73}{24}e^2 + \frac{37}{96}e^4}\label{eq:def_tgw},\\
    \at{\rd{e}{t}}_{\rm GW} &= -\frac{304}{15}\frac{G^3 \mu m_{12}^2}{c^5a^4}
        \frac{1}{j^{5}(e)}\p{1 + \frac{121}{304}e^2}\label{eq:dedt_gw}.
\end{align}

\subsection{Merger Windows and Probability}

If the eccentricity maxima of the ZLK cycles are sufficiently large, GW
radiation at $e \simeq e_{\max}$ can be sufficiently enhanced that even very
wide binaries $a = 100\;\mathrm{AU}$ can be induced to merge within a Hubble
time, $10\;\mathrm{Gyr}$ \citep{LL18, LL19}. When including the chaotic
octupole-order ZLK terms, the system's starting inclination alone may not be
sufficient to determine whether it merges within a Hubble time. Instead,
multiple simulations using different $\omega$, $\omega_{\rm out}$, and $\Omega$
can be used to compute a merger probability $P_{\rm merge}\p{I_{\rm 0}; q,
e_{\rm out}}$. For the same range of initial conditions considered in
Section~\ref{s:background}, we simulate the octupole-order ZLK equations and
include GW radiation to understand what systems merge in under a Hubble time. To
be precise, a successful merger is defined as the inner binary reaching $a =
0.5\;\mathrm{AU}$---since the inner binary is very eccentric at this point, the
additional time required to physically merge is negligible. The resulting merger
times $T_{\rm m}$ are shown in the middle panels of
Figs.~\ref{fig:composite_dist}--\ref{fig:composite_bindist}. For the range of
inclinations that have nonzero $f_{\rm merge}$, we sample the range of
inclinations much more finely and run $20$ simulations for each $I_0$ ($5$ for
the more computationally expensive $a = 10\;\mathrm{AU}$ case in
Fig.~\ref{fig:composite_bindist}). The resulting merger fractions are shown in
the bottom panels of
Figs.~\ref{fig:composite_dist}--\ref{fig:composite_bindist}. The
octupole-inactive gaps first identified in Section~\ref{ss:oct_gen} are also
present as gaps in $f_{\rm merge}$.

Note that there are two general categories of BH mergers in our simulations: (i)
the binary merges after a single large burst of GW radiation during a
high-eccentricity ZLK cycle (``one-shot merger'') or (ii) the binary merges
gradually by emitting a varying amount of GW radiation at every eccentricity
maximum (``smooth merger''). One-shot mergers merge upon reaching $e_{\lim}$,
and so merge with characteristic $T_{\rm m} \sim t_{\rm ZLK, oct}$, while smooth
mergers will typically exhibit a broad range of merger times. For retrograde
perturbers in Figs.~\ref{fig:composite_dist}--\ref{fig:composite_e91p5}, there
are comparatively few mergers with $T_{\rm m} \gtrsim 1\;\mathrm{Gyr}$, so the
fiducial parameter regime tends to merge via one-shot mergers. On the other
hand, prograde inclinations in Figs.~\ref{fig:composite_1p3dist}
and~\ref{fig:composite_bindist} produce a broad range of $T_{\rm m}$ and
represent an example of smooth mergers.

\subsection{Semi-Analytic Merger Criteria}\label{ss:nogw_merger}

It is clear that features in the $e_{\max}$ plots are correlated with behavior
in the $f_{\rm merge}$ and $T_{\rm m}$ plots. Here, we further develop this
connection and show that the non-dissipative simulations can indeed be used to predict
the outcomes of simulations with GW dissipation rather reliably.

Towards understanding the one-shot mergers, we first define $e_{\rm os}$ to be
the $e_{\max}$ required to dissipate an order-unity fraction of the binary's
orbital energy via GW emission in a single ZLK cycle. Recalling that a binary
spends a fraction $\sim j(e_{\max})$ of each ZLK cycle near $e_{\max}$, we write
\begin{align}
    j\p{e_{\rm os}}\at{\rd{\ln a}{t}}_{e = e_{\rm os}} &\sim
        \frac{1}{t_{\rm ZLK}},\\
    j^6(e_{\rm os})
        &\equiv \frac{842}{15}
            \frac{G^3 \mu m_{12}^3}{m_3c^5a^4n}
            \p{\frac{a_{\rm out, eff}}{a}}^3.
            \label{eq:def_e_os}
\end{align}
We have approximated $e_{\rm os} \approx 1$, so $\p{1 + 73e_{\rm os}^2/24 + 37
e_{\rm os}^4/96} \approx 421 / 96$. Then, if a system ever attains $e_{\max} >
e_{\rm os}$, it is expected to undergo a one-shot merger. In particular, if
$e_{\lim} \gtrsim e_{\rm os}$ for particular parameters, then all systems that
exhibit orbit flips in the absence of GW radiation will execute one-shot mergers
when GW is considered. When $e_{\lim} \approx 1$, Eq.~\eqref{eq:def_elim}
reduces to $j(e_{\lim}) \approx 4\epsilon_{\rm GR} / 9$, which lets us rewrite
the constraint $e_{\lim} \gtrsim e_{\rm os}$ as
\begin{align}
    \p{\frac{a}{a_{\rm out, eff}}} \gtrsim{}&
        0.0149
        \p{\frac{a_{\rm out, eff}}{3600\;\mathrm{AU}}}^{-7/37}
        \p{\frac{m_{12}}{50M_{\odot}}}^{17/37}\nonumber\\
        &\times\p{\frac{30M_{\odot}}{m_3}}^{10/37}
        \p{\frac{q / (1 + q)^2}{1/4}}^{-2/37}.\label{eq:q_237}
\end{align}
In Figs.~\ref{fig:composite_dist}--\ref{fig:composite_e91p5}, this constraint is
indeed satisfied, and we see indeed that wherever the top panel suggests orbit
flipping ($e_{\max} = e_{\lim}$), the bottom panel shows $f_{\rm merge} \approx
1$. In particular, this gives us some more insight as to why some prograde
inclinations near $90^\circ$ in Fig.~\ref{fig:composite_1p2} have $0 < f_{\rm
merge} < 1$: the top panel of Fig.~\ref{fig:composite_1p2} shows that there is a
distinct sub-population of systems whose $e_{\max} < e_{\rm os}$. Such systems
are not expected to merge based on their $e_{\max}$ alone. In fact, we attribute
this behavior to the presence of initial conditions for which $\omega$ is
librating. \citet{katz2011long} point out that $K$ oscillations are attenuated
when $\omega$ is librating, which produces smaller $e_{\max}$.

Towards understanding smooth mergers, we require a characteristic eccentricity
that captures GW emission over timescales $\simeq t_{\rm ZLK, oct}$, i.e.\ over
many ZLK cycles. We define $e_{\rm eff}$ as an effective \emph{maximum}
eccentricity, such that the average GW emission is the same as a system that
undergoes ZLK eccentricity cycles to a consistent maximum eccentricity $e_{\rm
eff}$. In other words,
\begin{align}
    \ev{\rd{\ln a}{t}} &\approx -\frac{1}{t_{\rm GW, 0}}
            \ev{\frac{1 + 73e_{\max}^2/24 + 37e_{\max}^4/96}
                {j^6(e_{\max})}}\nonumber\\
        &\equiv -\frac{421/96}{t_{\rm GW, 0}}\frac{1}{j^6(e_{\rm eff})},
        \label{eq:def_e_eff}
\end{align}
where $t_{\rm GW, 0} = \p{t_{\rm GW}}_{\rm e = 0}$. Here, the angle brackets
denote averaging over $\sim t_{\rm ZLK, oct}$, i.e.\ over many ZLK cycles.
Define next the critical effective eccentricity $e_{\rm eff, c}$ such that the
inspiral time is a Hubble time:
\begin{equation}
    \ev{\rd{\ln a}{t}} \equiv \frac{421/96}{t_{\rm GW, 0}j^6(e_{\rm eff, c})}
        = \p{10\;\mathrm{Gyr}}^{-1}. \label{eq:def_e_eff_c}
\end{equation}
Thus, if a system is evolved using the non-dissipative equations of motion and
satisfies $e_{\rm eff} > e_{\rm eff, c}$, then it is expected to successfully
undergo a smooth merger within a Hubble time. In summary, we obtain
\begin{equation}
    P_{\rm merge} = P\p{e_{\rm eff} > e_{\rm eff, c} \;\;\text{or}\;\;
        e_{\max} > e_{\rm os}}.\label{eq:def_sa_criteria}
\end{equation}
In practice, $e_{\rm eff} > e_{\rm eff, c}$ often ensures $e_{\max} > e_{\rm
os}$, as the averaging in Eq.~\eqref{eq:def_e_eff} is heavily weighted towards
extreme eccentricities.

In Figs.~\ref{fig:composite_dist}--\ref{fig:composite_e91p5},
Eq.~\eqref{eq:def_sa_criteria} is satisfied as long as $e_{\max} > e_{\rm os}$.
However, for the more compact architecture, Fig.~\ref{fig:composite_bindist},
none of the successfully-merging, prograde inclinations satisfy $e_{\max} >
e_{\rm os}$, but they yield $e_{\rm eff} > e_{\rm eff, c}$. This illustrates a
general point about these two semi-analytic criteria: for wider binaries,
$t_{\rm ZLK}$ is longer, so $e_{\rm os}$ is closer to $e_{\rm eff, c}$. Thus, in
wider binaries, the two criteria often give the same prediction, but in more
compact binaries, it is increasing probable that a system satisfies $e_{\rm eff}
> e_{\rm eff, c}$ without satisfying $e_{\max} > e_{\rm os}$.

\begin{figure}
    \centering
    \includegraphics[width=\colummwidth]{../../scripts/octlk/1sweepbin/composite_1p5dist.png}
    \caption{A plot of the system dynamics for the fiducial parameter regime ($a
    = 100\;\mathrm{AU}$, $a_{\rm out, eff} = 3600\;\mathrm{AU}$, $m_{12} =
    50M_{\odot}$, $m_3 = 30M_{\odot}$) with $q = 0.5$ and $e_{\rm out} = 0.6$.
    Here, $\eta \approx 0.087$ is nonnegligible, and $\epsilon_{\rm oct} \approx
    0.007$. In the top panel, for each of $1000$ initial inclinations, we choose
    $5$ different random $\omega$, $\omega_{\rm out}$, and $\Omega$ as initial
    conditions, then run for $500t_{\rm ZLK}$ without gravitational wave
    radiation. The maximum eccentricity $e_{\max}$ (blue dots) as well as the
    effective eccentricity [Eq.~\eqref{eq:def_e_eff}; green dots] over this
    period are displayed. For comparison, $e_{\rm eff, c}$
    [Eq.~\eqref{eq:def_e_eff_c}] is given by the horizontal green dashed line,
    $e_{\rm os}$ [Eq.~\eqref{eq:def_e_os}] is shown in the horizontal blue line,
    and $e_{\lim}$ [Eq.~\eqref{eq:def_elim}] is shown in the horizontal red
    dashed line. The vertical purple lines are the fitting formula of
    \citet{MLL16} (we evaluate for $\epsilon_{\rm oct}^{\rm (tp)} =
    \epsilon_{\rm oct}$) and no longer accurately describe the
    $e_{\lim}$-attaining inclination window. The black dashed line is given by
    Eq.~\eqref{eq:emax_quad}. In the middle panel, we show the binary merger
    times when including gravitational wave radiation and using the same range
    of initial conditions. Simulations are terminated when $T_{\rm m} >
    10\;\mathrm{Gyr}$ and marked as unsuccessful mergers. The horizontal dashed
    line denotes $t_{\rm ZLK}$ [Eq.~\eqref{eq:def_tzlk}] while the horizontal
    dash-dotted line indicates $t_{\rm ZLK, oct}$ [Eq.~\eqref{eq:def_tzlkoct};
    we evaluate for $\epsilon_{\rm oct}^{\rm (tp)} = \epsilon_{\rm oct}$]. Here,
    each $I_{\rm 0}$ is run $20$ times, so we can estimate a merger probability
    $P_{\rm merge}$ for each $I_{\rm 0}$, which is shown as the black line in
    the bottom panel. As described in Section~\ref{ss:nogw_merger}, systems can
    be predicted to merge using the results shown in the top panel if $e_{\max}
    > e_{\rm os}$ or if $e_{\rm eff} > e_{\rm eff, c}$. These give the
    semi-analytic prediction for the merger probability shown in the green line
    of the bottom panel, which is in reasonable agreement with the full
    simulations. }\label{fig:composite_dist}
\end{figure}
\begin{figure}
    \centering
    \includegraphics[width=\colummwidth]{../../scripts/octlk/1sweepbin/composite_1p3dist.png}
    \caption{Same as Fig.~\ref{fig:composite_dist} but for $q = 0.3$, $\eta
    \approx 0.07$, and $\epsilon_{\rm oct} \approx 0.011$. Note that prograde
    perturbers ($I_0 < 90^\circ$) only sometimes give successful mergers,
    resulting in a $P_{\rm merge}$ that displays very complex structure, unlike
    in Fig.~\ref{fig:composite_dist}. We attribute this to the fact that $e_{\rm
    eff} \sim e_{\rm eff, c}$ for this broad inclination range, such that
    whether a particular system successfully merges depends heavily on the
    detailed initial conditions ($\omega$, $\omega_{\rm out}$, and $\Omega$).
    }\label{fig:composite_1p2}
\end{figure}
\begin{figure}
    \centering
    \includegraphics[width=\colummwidth]{../../scripts/octlk/1sweepbin/composite_1p2dist.png}
    \caption{Same as Fig.~\ref{fig:composite_dist} but for $q = 0.2$, $\eta
    \approx 0.054$, and $\epsilon_{\rm oct} \approx 0.014$. Note that a large
    range of prograde $I_0$ are able to guarantee mergers, unlike in
    Figs.~\ref{fig:composite_dist} and~\ref{fig:composite_1p3}. A ``gap'' in
    merger fractions for $I_0 \approx 90^\circ$ is evident, which is discussed
    further in Appendix~\ref{app:gap}. There is a substantial range of prograde
    inclinations for which the $e_{\max}$ behavior is double-valued and for
    which the merger fraction $0 < f_{\rm merge} < 1$. We attribute this
    subpopulation to initial conditions where $\omega$ is librating, for which
    octupole-induced oscillations in $K$ are suppressed \citep{katz2011long} and
    $e_{\max}$ does not attain sufficiently large values to induce merger.
    }\label{fig:composite_1p3}
\end{figure}
\begin{figure}
    \centering
    \includegraphics[width=\colummwidth]{../../scripts/octlk/1sweepbin/composite_e91p5dist.png}
    \caption{Same as Fig.~\ref{fig:composite_dist} but with $q = 0.5$ and
    instead $e_{\rm out} = 0.9$ while holding $a_{\rm out, eff} =
    3600\;\mathrm{AU}$ constant. Here, $\eta \approx 0.12$ while $\epsilon_{\rm
    oct} \approx 0.019$. Note that even though $\epsilon_{\rm oct}$ is larger
    here than in Fig.~\ref{fig:composite_1p2}, no mergers are possible here with
    a prograde perturber. This is because $\eta$ is more than twice as large for
    these parameters. }\label{fig:composite_e91p5}
\end{figure}
\begin{figure}
    \centering
    \includegraphics[width=\colummwidth]{../../scripts/octlk/1sweepbin/bindist.png}
    \caption{Same as Figs.~\ref{fig:composite_dist}--\ref{fig:composite_1p2}
    but for a compact inner binary; the parameters are $a = 10\;\mathrm{AU}$,
    $a_{\rm out, eff} = 700\;\mathrm{AU}$, $m_{12} = 50M_{\odot}$, $m_3 =
    30M_{\odot}$, and $e_{\rm out} = 0.9$, $q = 0.4$. Note that when the
    perturber is prograde ($I_0 < 90^\circ$), $e_{\max} <
    e_{\rm os}$ but $e_{\rm eff} > e_{\rm eff, c}$, which predicts that systems
    are able to merge. This prediction is borne out by simulations with GW
    radiation (middle and bottom panels). }\label{fig:composite_bindist}
\end{figure}

\section{Merger Fraction as a Function of Mass Ratio}

Using the merger probability $P_{\rm merge}\p{I_0; q, e_{\rm out}}$, it is
natural to next seek the mass ratio distribution of merging BH binaries via the
comparable-mass tertiary-induced channel. However, this relies on many
uncertainties, such as the primordial $q$ distribution in BH binaries, that are
beyond the scope of this paper. Instead, we aim to characterize the effect
described in the preceeding sections by studying the fraction of BH binaries
that successfully merge under various conditions.

\subsection{Fixed Tertiary Eccentricity}

We first consider the simplified case where $e_{\rm out}$ is fixed at a few
specific values and compute the merger fraction as a function of the mass ratio
$q$. We let $\cos I_0$ be drawn uniformly from the range $[-1, 1]$, let
$\omega$, $\omega_{\rm out}$, and $\Omega$ be drawn from the range $[0,
2\pi)$ as before% chktex 9
and adopt the fiducial parameters. The merger fraction is defined as:
\begin{equation}
    f_{\rm merge}\p{q, e_{\rm out}} \equiv
        \frac{1}{2}\int\limits_{-1}^1P_{\rm merge}\p{I_0; q, e_{\rm out}}
            \;\mathrm{d}\cos I_0.
\end{equation}
This is plotted for various choices of $\p{q, e_{\rm out}}$ as the solid dots in
Fig.~\ref{fig:total_merger_fracs}. For each simulation, we also use
semi-analytic criteria introduced in Section~\ref{ss:nogw_merger} to predict the
outcome and generate a merger fraction, shown as the crosses and dashed lines in
Fig.~\ref{fig:total_merger_fracs}. It is clear both that smaller $q$
significantly enhances the merger fraction and that the semi-analytic criteria
do an adequate job of tracking the merger fraction.

In the right panel of Fig.~\ref{fig:total_merger_fracs}, we see that the merger
fractions for the three $e_{\rm out}$ values overlap for sufficiently small
$\epsilon_{\rm oct}$. This implies that $f_{\rm merge}$ depends only on
$\epsilon_{\rm oct}$ for these values, and not the values of $q$ and $e_{\rm
out}$ independently. However, for larger $\epsilon_{\rm oct}$, the three curves
diverge. This is attributed to their different $\eta$ values: for sufficiently
small $\epsilon_{\rm oct}$, no prograde initial inclinations not successfully
merge (e.g.\ Fig.~\ref{fig:composite_dist}). However, the size of the
octupole-inactive gap is different for different $\eta$, so prograde
inclinations begin to merge successfully at different $e_{\rm out}$ thresholds,
breaking the degeneracy. This again illustrates the importance of understanding
the octupole-inactive gap, which we comment on in Appendix~\ref{app:gap}.
Choosing instead $a = 50\;\mathrm{AU}$ and $a_{\rm out, eff} =
1800\;\mathrm{AU}$ does not significantly change
Fig.~\ref{fig:total_merger_fracs}. This is because $\epsilon_{\rm oct}$ and
$\eta$ are unchanged, while Eq.~\eqref{eq:q_237} only changes very weakly.

If instead, we choose $a = 50\;\mathrm{AU}$ but keep $a_{\rm out, eff} =
3600\;\mathrm{AU}$, we obtain Fig.~\ref{fig:sweepbin_simpleouter}. According to
Eq.~\eqref{eq:q_237}, this new parameter regime no longer satisfies $e_{\lim}
\gtrsim e_{\rm os}$, so the merger fraction is expected to diminish strongly and
vary much more weakly with $q$, which is indeed observed.
\begin{figure*}
    \centering
    \includegraphics[width=0.8\textwidth]{../../scripts/octlk/1sweepbin/total_merger_fracs.png}
    \caption{From Figs.~\ref{fig:composite_dist}--\ref{fig:composite_e91p5},
    we can compute the total merger fraction in the presence of GW radiation
    assuming $\cos I_0$ is uniformly distributed $\in [-1, 1]$. We do this for
    three values of $e_{\rm out}$ and six values of $q$ and show the results
    with solid dots. The X's show the results when using the semi-analytic
    merger criteria of Section~\ref{ss:nogw_merger}; good agreement
    is observed. }\label{fig:total_merger_fracs}
\end{figure*}
\begin{figure*}
    \centering
    \includegraphics[width=0.8\textwidth]{../../scripts/octlk/1sweepbin_simple/simpleouter.png}
    \caption{Same as Fig.~\ref{fig:total_merger_fracs} but for $a =
    50\;\mathrm{AU}$. Note that the $f_{\rm merge}$ enhancement for smaller $q$
    is smaller, as the condition Eq.~\eqref{eq:q_237} is no longer satisfied.
    }\label{fig:sweepbin_simpleouter}
\end{figure*}

\subsection{Distribution of Tertiary Eccentricities}

We can also draw $e_{\rm out} \in [0, 0.9]$ with both a uniform probability
distribution and a thermal one $P(e_{\rm out}) \propto e_{\rm out}$, and examine
$f_{\rm merge}$ as a function of $q$ (the true $e_{\rm out}$ distribution is
likely to be between these two cases). The results are shown in
Fig.~\ref{fig:popsynth}, where each $q$ has $1000$ realizations using random
$e_{\rm out}$, $\cos I_0$, $\omega$, $\omega_{\rm out}$, and $\Omega$. If
$e_{\rm out}$ is thermally distributed (red), binaries with smaller $q$ are even
more likely to merge, since $\epsilon_{\rm oct}$ tends to be larger. To
understand the impact of our random sampling, we also compute a merger fraction
using the semi-analytic merger criteria of Section~\ref{ss:nogw_merger} over a
much denser, uniform grid of $(\cos I_0, e_{\rm out}, q)$, shown as the blue
dotted line. While the resolution is still limited, it is clear that the true
$f_{\rm merge}$ should be a monotonic function of $q$. The distributions and
medians of the merger time and eccentricity in the LISA and LIGO bands are also
shown. For the LISA and LIGO band eccentricities, the inner binaries are evolved
from when they reach $0.5\;\mathrm{AU}$ to physical merger using
Eqs.~(\ref{eq:def_tgw}--\ref{eq:dedt_gw}). While all of these eccentricities are
quite small, this has to do with our using the double-averaged equations of
motion. Both the single-averaged and the full n-body equations of motion produce
larger eccentricities in the LISA and LIGO bands \citep{LL19}.

For comparison, we also show the results when $a_{\rm out, eff} =
5500\;\mathrm{AU}$, with all other parameters unchanged, in
Fig.~\ref{fig:popsynth5500}. While $f_{\rm merge}$ is lower than for $a_{\rm
out, eff} = 3600\;\mathrm{AU}$, there is still a large increase between large
and small $q$. This is expected, since $e_{\lim} > e_{\rm os}$ by
Eq.~\eqref{eq:q_237}, and the $e_{\lim}$-achieving range of inclinations changes
dramatically for the range of $\epsilon_{\rm oct}$ values in question.
\begin{figure}
    \centering
    \includegraphics[width=\columnwidth]{../../scripts/octlk/1popsynth/a2eff3600.png}
    \caption{Merger fractions with the fiducial parameters obtained by randomly
    drawing $\cos I_0$ uniformly distributed $\in [-1, 1]$ and drawing $e_{\rm
    out}$ from either a uniform distribution ($e_{\rm out} \in [0, 0.9]$; black)
    or a thermal distribution ($P(e_{\rm out}) \propto e_{\rm out}$, $e_{\rm
    out} \in [0, 0.9]$; red). The blue dotted line instead samples a dense,
    uniform grid in $\cos I_0$, $e_{\rm out}$, and $q \in [0.01, 1]$ using the
    semi-analytic merger criteria (Section~\ref{ss:nogw_merger}), confirming
    that the non-monotonicity of the $f_{\rm merge}$ calculated from simulations
    including GW radiation is due to the random sampling involved. The middle
    panel shows the merger time for successful mergers (the median is denoted
    with the large black dot). The bottom panel shows the binary eccentricity in
    the LISA band ($0.1\;\mathrm{Hz}$; blue) and in the LIGO band ($10
    \;\mathrm{Hz}$; red).
    }\label{fig:popsynth}
\end{figure}
\begin{figure}
    \centering
    \includegraphics[width=\columnwidth]{../../scripts/octlk/1popsynth/a2eff5500.png}
    \caption{Same as Fig.~\ref{fig:popsynth} but for $a_{\rm out, eff} =
    5500\;\mathrm{AU}$. }\label{fig:popsynth5500}
\end{figure}

\subsection{Semi-Analytic Merger Fraction Completeness}\label{ss:completeness}

It is evident from Figs.~\ref{fig:total_merger_fracs}--\ref{fig:popsynth5500}
that the semi-analytic merger fractions are systematically lower than the values
obtained from the fullly dissipative simulations. This can also be seen in the
bottom panels of Figs.~\ref{fig:composite_dist}--\ref{fig:composite_bindist},
where the semi-analytic merger criteria are inaccurate, particularly at the
edges of the merger windows.

The reason this arises is because the non-dissipative simulations used to
compute $e_{\rm eff}$ and $e_{\max}$ are only run for $500 t_{\rm LK} \approx
0.7\;\mathrm{Gyr}$, while the simulations including GW dissipation are run for
$10\;\mathrm{Gyr}$. Owing to the chaotic nature of the octupole-order ZLK
effect, this means that, if an initial condition only reaches more extreme
eccentricities after a few Gyr, the $e_{\rm eff}$ and $e_{\max}$ are
underpredicted by the non-dissipative simulations. Additionally, phases where
$\omega$ is librating can last an unpredictable amount of time, during which
orbit flips are much rarer \citep{katz2011long}. This suggests that the
semi-analytic merger criteria becomes more complete as the integration
time is increased.

To quantify this, we examine the completeness of the semi-analytic merger
fraction via the ratio $f_{\rm merge, SA}\p{t_{\rm f}} / f_{\rm merge}$, where
$f_{\rm merge, SA}\p{t_{\rm f}}$ is the predicted merger fraction using the
semi-analytic merger criteria in Section~\ref{ss:nogw_merger} when calculated
using non-dissipative simulations run for duration $t_{\rm f}$ (i.e.\ in the
previous sections, $t_{\rm f} = 500t_{\rm ZLK}$). This is given in
\begin{figure}
    \centering
    \includegraphics[width=\colummwidth]{../../scripts/octlk/1sweepbin/completeness.png}
    \caption{Completeness of the semi-analytic merger fraction, defined as
    $f_{\rm merge, SA} / f_{\rm merge}$, as a function of the integration time
    $t_{\rm f}$ used for the non-dissipative simulations, in the fiducial
    parameter regime while $e_{\rm out}$ is fixed at a few values. The thin grey
    lines indicate the completeness for particular combinations of $(q, e_{\rm
    out})$, and the thick black line denotes their average. We see that
    completeness still has not converged for $t_{\rm f}$ as large as $2000
    t_{\rm ZLK} = 3.5\;\mathrm{Gyr}$; $t_{\rm f} = 500t_{\rm ZLK}$ is used in
    this paper. }\label{fig:completeness}
\end{figure}

\subsection{Effect of Smaller Mass Ratios}

The above results may seem somewhat counterintuitive at first glance, since
$t_{\rm GW} \propto \mu$ which should grow as $q$ is decreased. This effect
should increase the merger times and decrease the merger fractions, but the data
illustrate otherwise.

The key insight is that, if $e_{\lim} > e_{\rm os}$, then many binaries execute
one-shot mergers when undergoing an orbit flip. Since $t_{\rm oct,
ZLK} \ll 10\;\mathrm{Gyr}$ typically, this implies that octupole-ZLK-induced
binary merger fractions determined by what initial conditions execute orbit
flips, and not by the detailed GW radiation rate. Additionally,
Eq.~\eqref{eq:q_237} shows that, while $e_{\lim} > e_{\rm os}$ is indeed
violated if $q$ is decreased sufficiently with all else held constant, the
dependence is extremely weak. As such, $f_{\rm merge}$ is expected to be very
nearly constant in $q$ for all physical values of $q$, as shown in
Fig.~\ref{fig:popsynth_lowq}.
\begin{figure}
    \centering
    \includegraphics[width=\colummwidth]{../../scripts/octlk/1popsynth/a2eff_nogw_lowq3600.png}
    \caption{Same as blue dashed line of the top panel of
    Fig.~\ref{fig:popsynth} but extended to very small $q$. Due to the very weak
    $q$ dependence in Eq.~\eqref{eq:q_237}, $f_{\rm merge}$ is expected to
    depend very weakly on $q$ when $q \ll 1$ (such that $\epsilon_{\rm oct}$ is
    approximately constant), which agrees with the simulation
    results.}\label{fig:popsynth_lowq}
\end{figure}

\section{Conclusion and Discussion}

We have considered the dynamics of a comparable-mass tertiary-induced binary BH
merger by studying the ZLK effect at octupole order. We showed for sufficiently
heirarchical binaries [Eq.~\eqref{eq:q_237} is satisfied, and the
double-averaged equations of motion are valid] that the merger fraction is
dramatically larger for binaries with smaller mass ratios (see
e.g.\ Fig.~\ref{fig:popsynth}). We showed that, due to the nonneglible angular
momentum ratio $\eta$ in such systems, the inclinations for which extreme
eccentricities can be attained exhibit much more complex behavior than in the
test-particle case \citep[as studied by][]{katz2011long, lithwick2011eccentric,
LML15, MLL16}. Notably, the outcomes are no longer symmetric about $I_0 =
90^\circ$, and there is a gap near $I_0 \approx 90^\circ$ for which extreme
eccentricity excitation is always forbidden
(Figs.~\ref{fig:composite_dist}--\ref{fig:composite_bindist}).

We discuss the implications of our results in context of the observed binary BH
mergers from LIGO/VIRGO\@. When including the latest data from the O3a observing
run, the observed distribution in $q$ significantly prefers larger mass ratios.
In particular, if $P(q) = q^{\beta_q}$, then $\beta_q > 0$ at $89\%$ or more
\citep{LIGOO3a}. At the same time, the primordial $q$ distribution in massive
stellar binaries is generally expected to roughly uniform or prefer smaller $q$
\citep[e.g.][]{sana2012binary, moe2017mind}. In conjunction with our results in
this paper, these results appear to be in tension with observation. However,
such a simple analysis omits many possible confounding effects during the
transition from stellar binary to isolated BH triple (e.g.\ supernova kicks,
stellar fly-bys), and a detailed analysis beyond the scope of this paper is
required to understand the cumulative effect of these complications.

Most of our analysis above assumes $e_{\lim} > e_{\rm os}$
[Eq.~\eqref{eq:q_237}]. However, if this is not satisfied, the dependency of
$f_{\rm merge}$ on $q$ becomes much weaker, e.g.\
Fig.~\ref{fig:sweepbin_simpleouter}.

Finally, while this work was conducted using the double-averaged (DA) equations
of motion, most of the qualitative behavior is expected to be robust to
higher-order approximations \citep[e.g.][]{LL19}. Of particular interest is how
one-shot mergers restore the validity of the DA equations. In particular, the DA
approximation reqires the orbital period $P_{\rm out}$ of the outer binary be
much longer than all dynamical timescales of the inner binary, i.e.
\begin{equation}
    t_{\rm ZLK}j\p{e_{\max}} \lesssim P_{\rm out},
\end{equation}
For our fiducial parameter regime, this is not strictly satisfied when
$e_{\max}$ reaches $e_{\lim}$. However, it is satisfied when $e_{\max}$ reaches
$e_{\rm os} < e_{\lim}$, so the inner binary will begin to decay significantly,
as a one-shot merger, before the DA approximation breaks down.

\section{Acknowledgements}\label{s:ack}

YS is supported by the NASA FINESST grant 19-ASTRO19-0041.%chktex 8

\bibliographystyle{mnras}
\bibliography{Su_EZLK}

\clearpage
\onecolumn

\appendix

\section{Origin of Octupole-Inactive Gap}\label{app:gap}

\begin{figure}
    \centering
    \includegraphics[width=\colummwidth]{../../scripts/octlk/1sweepbin_emax/1p2dist.png}
    \caption{Top panel is the same as the top panel of
    Fig.~\ref{fig:composite_1p2}. The bottom panel shows the range of
    oscillation in $K$, denoted by $K_{\min}$ and $K_{\max}$, for the same
    parameters. The critical $K = \eta / 2$ for orbit flipping is shown with the
    horizontal red line. It can be seen that when $K_{\min} < \eta / 2 <
    K_{\max}$ that $e_{\max} \simeq e_{\lim}$ in the top panel. It is therefore
    clear that the gap in $e_{\max}$ excitation is due to a limited range of
    oscillation in $K$.
    }\label{fig:kdist}
\end{figure}

\begin{itemize}
    \item Show the $K_{\min}$ and $K_{\max}$ plot, Fig.~\ref{fig:kdist}.

    \item Is because librating! example simulation where librating,
        Fig.~\ref{fig:nogw_circ}.

    \item Point out that the character of the circulation-libration transition
        changes when $\epsilon_{\rm oct}$ is substantial, to not only depend on
        $\omega$, Fig.~\ref{fig:dW}.
\end{itemize}

\begin{figure}
    \centering
    \includegraphics[width=\colummwidth]{../../scripts/octlk/1nogw_sims/1nogw_vec88.png}
    \caption{Example simulation where $\Omega_{\rm e}$ is primarily circulating,
    which suppresses the amplitude of oscillation of $K$. As a result, the orbit
    does not flip.}\label{fig:nogw_circ}
\end{figure}
\begin{figure}
    \centering
    \includegraphics[width=\colummwidth]{../../scripts/octlk/2dW_sweeps/2_dWsweeps6_2_dual.png}
    \caption{Plot of $\Delta \Omega_{\rm e}$, the change in the co-longitude of
    the inner eccentricity vector $\Omega_{\rm e}$ over a ZLK cycle, for
    different initial conditions and when octupole terms are off/on. In the top
    panel, the octupole terms are neglected, and $\omega_{\rm 0} = 0$ results in
    circulation ($\Delta \omega = 0$ and $\Delta \Omega_{\rm e} = 180^\circ$)
    while $\omega_{\rm 0} = \pi / 2$ results in libration, as expected. In the
    bottom panel, the octupole terms are included, and it is seen that the
    circulation-libration transition no longer depends on $\omega_{\rm 0}$ but
    instead on $I_0$. }\label{fig:dW}
\end{figure}

\label{lastpage} % chktex 24
\end{document} % chktex 17
