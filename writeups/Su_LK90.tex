% rm -f LK90_plots/*.png; for i in M1_M2_thetasl_thetae_70 M1_M2_thetasl_thetae_88 5_dWs_inner 5_dWs 6bin_comp_zoom 4qslscan/qslscan 4sims_scan/deviations_one 4sims/4sim_90_200_zoom 4sims/4sim_90_350_zoom 4sims/4sim_90_350 7_3vec_cropped 4sims/good_quants; do cp ../scripts/lk90/$i.png LK90_plots; done
    \documentclass[
        twocolumn,
        twocolappendix
    ]{aastex63}
    \usepackage{
        amsmath,
        amssymb,
        newtxtext,
        newtxmath,
        graphicx,
        ae,
        aecompl,
        booktabs,
        wasysym
    }
    \usepackage[T1]{fontenc}

    \newcommand*{\rd}[2]{\frac{\mathrm{d}#1}{\mathrm{d}#2}}
    \newcommand*{\rtd}[2]{\frac{\mathrm{d}^2#1}{\mathrm{d}#2^2}}
    \newcommand*{\pd}[2]{\frac{\partial#1}{\partial#2}}
    \newcommand*{\md}[2]{\frac{\mathrm{D}#1}{\mathrm{D}#2}}
    \newcommand*{\at}[1]{\left.#1\right|}
    \newcommand*{\abs}[1]{\left|#1\right|}
    \newcommand*{\ev}[1]{\langle#1\rangle}
    \renewcommand*{\bm}[1]{\boldsymbol{\mathbf{#1}}}
    \newcommand*{\uv}[1]{\hat{\bm{#1}}}
    \newcommand*{\p}[1]{\left(#1\right)}
    \newcommand*{\s}[1]{\left[#1\right]}
    \newcommand*{\z}[1]{\left\{#1\right\}}
    \DeclareMathOperator*{\argmin}{argmin}
    \DeclareMathOperator*{\argmax}{argmax}
    \DeclareMathOperator*{\med}{med}
    \DeclareMathOperator*{\sgn}{sgn}
    \let\Re\undefined
    \let\Im\undefined
    \DeclareMathOperator{\Re}{Re}
    \DeclareMathOperator{\Im}{Im}

    \newcommand\aastex{AAS\TeX}
    \newcommand\latex{La\TeX}

    \received{XXXX}
    \revised{XXXX}
    \accepted{XXXX}
    \submitjournal{ApJL}

\shorttitle{BH Triple Spin-Orbit Dynamics}
\shortauthors{Y.\ Su et.\ al.}

\begin{document}

\title{Spin Dynamics in Hierarchical Black Hole Triples: Predicting Final
Spin-Orbit Misalignment Angle From Initial Conditions}

\correspondingauthor{Yubo Su}
\email{yubosu@astro.cornell.edu}

\author[0000-0001-8283-3425]{Yubo Su}% chktex 8
\affiliation{Cornell Center for Astrophysics and Planetary Science, Department
of Astronomy, Cornell University, Ithaca, NY 14853, USA}

\author[0000-0002-1934-6250]{Dong Lai}% chktex 8
\affiliation{Cornell Center for Astrophysics and Planetary Science, Department
of Astronomy, Cornell University, Ithaca, NY 14853, USA}

\author[0000-0002-0643-8295]{Bin Liu}% chktex 8
\affiliation{Cornell Center for Astrophysics and Planetary Science, Department
of Astronomy, Cornell University, Ithaca, NY 14853, USA}

\begin{abstract}
    Abstract
\end{abstract}

\keywords{keywords}

\section{Introduction---WORK IN PROGRESS DO NOT READ}\label{s:intro}

Para 1: (see papers Liu-Lai, can copy/adapt) The standard scenario for the
formation merging BH binaries involve isolated binary evolution, in which mass
transfer and frcition in the common envelope phase bring the binar closer (TODO)
The recent detections of (TODO) by LIGO have motivated the studies of various
alternative, dynamical formation channels (TODO) One of the dynamical channels
is "tertiary-induced mergers", in which (TODO)

Para 2: One of the ways to distiguish different channels is spin (TODO).
Introduce $\chi_eff$ (TODO)

Para 3: In a recent paper, Liu et al found that LK-induced mergers can often
lead to a "$90^\circ$ spin attractor": Starting for aligned spi-orbit, the inner
binary eventually evolves toward (TODO)  This "attractor" requires the
LK-induced orbital evolution/decay to be slow and regular, and octupole effect
not important. Figure 1 (your current 3) gives an exmaple (TODO) Figure 2 (your
current Fig.2) shows (TODO).

Para 4: The physical origin of this "90 spin attractor" is not understood. Liu
et al proposed an explanation based on analogy with an adiabatic invariant in
circular systems where the tertiary make the inner binary process but does not
induce eccentricity oscillation (Liu-Lai 2017). But this adiabatic invariant
cannot be applied to LK-induced mergers, where e-excitation lays a crucial role
(TODO)   In addition, Liu-Lai (2017) considered found that for "mild LK-enhanced
mergers", the attractor does not exist (TODO).despite the fcat that the orbital
evolution is slow and regular (TODO)

Para 5: In this paper we examine the theory that explain the attractor, as well
as the regime of validity (TODO).

\begin{figure*}
    \centering
    \includegraphics[width=\textwidth]{LK90_plots/4sim_90_350.png}
    \caption{Orbital and spin evolution in a system for which the total change
    in the adiabatic invariant $\theta_{\rm e}$ is $\lesssim 0.01^\circ$. The
    unit of time $t_{\rm LK, 0}$ is the LK timescale [Eq.~\eqref{eq:t_lk}]
    evaluated for the initial conditions. The inner binary is taken to have $a =
    100\;\mathrm{AU}$, $m_1 = 30M_{\odot}$, $m_2 = 20M_{\odot}$, $I_0 =
    90.35^\circ$, and $e_0 = 0.001$, while the tertiary SMBH has $\tilde{a}_3 =
    2.2\;\mathrm{pc}$, $m_3 = 3 \times 10^7 M_{\odot}$. We take $\theta_{\rm
    sl}^{\rm i} = 0$. The top three panels $a$; $e$; and the inclination of the
    inner binary, both instantaneous ($I$) and appropriately averaged following
    Eq.~\eqref{eq:barI} ($\bar{I}$). The bottom three panels show the
    instantaneous spin-orbit misalignment angle $\theta_{\rm sl}$; the angle
    between $\bm{\overline{\Omega}}_{\rm e}$ [Eq.~\eqref{eq:weff_def}] and both
    the instantaneous spin vector (light grey) and the LK-averaged spin vector
    [red dots, denoted $\theta_{\rm e}$, Eq.~\eqref{eq:q_eff}]; and four
    characteristic frequencies of the system [Eqs.~\ref{eq:weff_def}
    and~\eqref{eq:Wldef}].}\label{fig:4sim_90_350}
\end{figure*}
\begin{figure}
    \centering
    \includegraphics[width=\columnwidth]{LK90_plots/qslscan.png}
    \caption{Plot of merger time of the inner binary and $\theta_{\rm sl}^{\rm
    f}$ for $m_1 = 30M_{\odot}$, $m_2 = 20M_{\odot}$, $m_3 = 3 \times 10^7
    M_{\odot}$, $a_{\rm in} = 100\;\mathrm{AU}$, $\tilde{a}_3 =
    2.2\;\mathrm{pc}$, $e_3 = 0$, where $e_0 = 0.001$ and $\theta_{\rm sl}^{\rm
    i} = 0$ over a restricted range of $I_0$ \citep[analogous to the bottom-most
    panel in Fig.~3 of][]{bin2}, where the blue dots are taken from numerical
    simulations. It is clear that for $I_0$ sufficiently far from $90^\circ$,
    the resulting $\theta_{\rm sl}^{\rm f}$ are quite similar and are near
    $90^\circ$\citep{bin2}.}\label{fig:qslscan}
\end{figure}

In Section~\ref{s:setup}, we set up the relevant equations of motion for the
orbital and spin evolution of the three BHs, and we argue for the primary result
of the paper, conservation of the angle $\theta_{\rm e}$. In
Sections~\ref{s:fast_merger} and~\ref{s:harmonic}, we consider two scenarios
under which conservation of $\theta_{\rm e}$ can be violated. We discuss and
conclude in Section~\ref{s:discussion}.

\section{LK-Induced Mergers: Orbital Evolution}\label{s:setup}

We study Lidov-Kozai (LK) oscillations due to an external perturber to
quadrupole order and include apsidal precession and gravitational wave
radiation due to general relativity. Consider an black hole (BH) binary
with masses $m_1$ and $m_2$ having total mass $m_{12}$ and reduced mass $\mu$
orbiting a supermassive black hole (SMBH) with mass $m_3 \gg m_1$ and $m_2$.
Call $a_3$ and $e_3$ the semimajor axis and eccentricity of the orbit of the
inner binary around the SMBH, and define effective semimajor axis
\begin{equation}
    \tilde{a}_3 \equiv a_3\sqrt{1 - e_3^2}.
\end{equation}
Finally, call $\bm{L}_{\rm out} \equiv L_{\rm out} \uv{L}_{\rm out}$ the angular
momentum of the SMBH relative to the center of mass of the inner BH binary, and
call $\bm{L} \equiv L \uv{L}$ the orbital angular momentum of the inner BH
binary. We take $\bm{L}_{\rm out}$ to be fixed.

We then consider the motion of the inner binary, described by the Keplerian
orbital elements $(a, e, \ascnode, I, \omega)$ (respectively: semimajor axis,
eccentricity, longitude of the ascending node, inclination, and argument of
periapsis). The equations describing the motion of these orbital elements are
\citep{peters1964,storch,bin2}
\begin{align}
    \rd{a}{t} &= \p{\rd{a}{t}}_{\rm GW},\\
    \rd{e}{t} &= \frac{15}{8t_{\rm LK}} e\sqrt{1 - e^2}\sin 2\omega
        \sin^2 I + \p{\rd{e}{t}}_{\rm GW}\label{eq:dedt},\\
    \rd{\ascnode}{t} &= \frac{3}{4t_{\rm LK}}
        \frac{\cos I\p{5e^2 \cos^2\omega - 4e^2 - 1}}{\sqrt{1 - e^2}},\\
    \rd{I}{t} &= \frac{15}{16}\frac{e^2\sin 2\omega \sin 2I}{
        \sqrt{1 - e^2}},\\
    \rd{\omega}{t} &= \frac{3}{4t_{\rm LK}}
        \frac{2\p{1 - e^2} + 5\sin^2\omega
            (e^2 - \sin^2 I)}{\sqrt{1 - e^2}}
        + \Omega_{\rm GR},\label{eq:dwdt}
\end{align}
where we define
\begin{align}
    t_{\rm LK}^{-1} &= n\p{\frac{m_3}{m_{12}}}\p{\frac{a}{\tilde{a}_3}}^3,
        \label{eq:t_lk}\\
    \p{\rd{a}{t}}_{\rm GW} &= -\frac{a}{t_{\rm GW}(e)},\\
    t_{\rm GW}^{-1}(e) &= \frac{64}{5}\frac{G^3 \mu m_{12}^2}{c^5a^4}
            \frac{1}{\p{1 - e^2}^{7/2}}\p{1 + \frac{73}{24}e^2
                + \frac{37}{96}e^4},\label{eq:dadt_gw}\\
    \p{\rd{e}{t}}_{\rm GW} &= -\frac{304}{15}\frac{G^3 \mu m_{12}^2}{c^5a^4}
        \frac{1}{\p{1 - e^2}^{5/2}}\p{1 + \frac{121}{304}e^2}\label{eq:dedt_gw}
            ,\\
    \Omega_{\rm GR} &= \frac{3Gnm_{12}}{c^2a\p{1 - e^2}},
\end{align}
and $n = \sqrt{Gm_{12}/a^3}$ is the mean motion of the inner binary. We will
often refer to $e_{\min}$ and $e_{\max}$ the minimum/maximum eccentricity in a
single LK cycle, and we will sometimes notate $j(e) = \sqrt{1 - e^2}$ and
$j(e_{\max}) = \sqrt{1 - e_{\max}^2}$.

Finally, for concreteness, we adopt fiducial parameters similar to those from
\citet{bin2}: the inner binary has $a = 100\;\mathrm{AU}$, $m_1 = 30M_{\odot}$,
$m_2 = 20M_{\odot}$, and initial $e_0 = 0.001$ with varying $I_0$. We take the
SMBH tertiary companion to have $m_3 = 3 \times 10^{7} M_{\odot}$ and
$\tilde{a}_3 = 4.5 \times 10^5\;\mathrm{AU} = 2.2\;\mathrm{pc}$. This gives the
same $t_{\rm LK}$ as in \citet{bin2}.

We summarize a few key analytical properties of this orbital evolution below:
\begin{itemize}
    \item Neglecting GR effects (GW radiation and apsidal precession), the
        above equations describe periodic oscillations in $e$ and $I$. There are
        two conserved quantities, the total angular momentum $\bm{L}_{\rm out} +
        \bm{L}$ and the ``Kozai constant'' \citep{lidov,kozai}:
        \begin{equation}
            K \equiv j(e) \cos I.\label{eq:K_def}
        \end{equation}
        This implies that $e$ is a function of $\omega$ only.

        Note that an eccentricity maxiumum occurs every half-period of $\omega$,
        at $\omega = \pi/2$ and $\omega = 3\pi/2$ \citep{anderson2016formation}.
        It is most convenient to define the LK period $P_{\rm LK}$ as the
        \emph{half}-period of $\omega$. Then we define characteristic LK
        frequency
        \begin{equation}
            \Omega \equiv \frac{2\pi}{P_{\rm LK}}.\label{eq:WLK_def}
        \end{equation}

        The conservation laws above can be combined to calculate the maximum
        eccentricity as a function of $I$. If the minimum eccentricity is
        negligible, then
        \begin{equation}
            e_{\max} \equiv \sqrt{1 - \frac{5}{3}\cos^2 I_0}.\label{eq:emax}
        \end{equation}
        Note that these results assume $L_{\rm out} \gg L$ and that both
        octupole and GR effects are negligible; see \citet{bin2} for generalized
        forms of the above results.

        Finally, when $e_{\min} \ll e_{\max}$, the binary spends a fraction
        $\sim j(e_{\max})$ of the LK cycle near $e \simeq e_{\max}$
        \citep{anderson2016formation}.

    \item When GR effects are considered, the system gradually coalesces due
        to GW radiation, primarily near eccentricity maxima. This process is
        governed by three competing timescales: (i) LK oscillations ($\sim
        j(e_{\max}) t_{\rm LK}$), (ii) apsidal precession due to post-Newtonian
        effects ($\sim \Omega_{\rm GR}^{-1}(e_{\max})$), and (iii) orbital decay
        due to GW radiation ($\sim t_{\rm GW}(e_{\max})$). At early times, (i)
        is the shortest, but as $a$ decreases, (ii) and (iii) become dominant
        and eccentricity oscillations become suppressed, which we term
        ``eccentricity freezing''. We consider the two conditions under which
        this happens.

        Apsidal precession becomes important when $\Omega_{\rm GR}$ is the
        dominant contribution in Eq.~\eqref{eq:dwdt}. Quantitatively, this
        occurs when
        \begin{equation}
            \frac{\epsilon_{\rm GR}}{j(e_{\max})} \gg 1,
        \end{equation}
        where we define
        \begin{equation}
            \epsilon_{\rm GR} \equiv \frac{3Gm_{12}^2 \tilde{a}_3^3}{
                c^2m_3a^4}.
        \end{equation}

        GW radiation can also inhibit eccentricity oscillations when $t_{\rm
        LK}j(e) \gtrsim t_{\rm GW}(e)$ [see Eq.~\eqref{eq:dedt}]. In order for
        GW radiation to be the dominant source of eccentricity freezing, we
        require
        \begin{align}
            1 \lesssim \frac{t_{\rm GW}^{-1}(e)}{\Omega_{\rm GR}}
                &\sim \p{\frac{Gm_{12}}{a(1 - e^2)c^2}}^{3/2} \frac{1}{1 - e^2}
                    ,\\
                &\sim \p{\frac{v_p}{c}}^3\frac{1}{1 - e^2},
        \end{align}
        where $v_p$ is the pericenter velocity of the inner binary. Since $v_p/c
        \ll 1$, we find that apsidal precession is usually more effective at
        freezing the eccentricity.
\end{itemize}
The behavior of $a$, $e$, and $I$ for a characteristic LK-induced merger can be
seen in Fig.~\ref{fig:4sim_90_350}. At early times, $e$ and $I$ have large
oscillations while $a$ slowly decreases. At later times, $e$ and $I$ stop
oscillating as apsidal precession freezes the eccentricity, and both $a$ and $e$
decrease under GW radiation. Since $P_{\rm LK}$ is defined as the half-period in
$\omega$, it asymptotes to $\Omega_{\rm GR}$ at late times even though the
eccentricity is frozen (bottom right panel of Fig.~\ref{fig:4sim_90_350}).

\section{Spin Dynamics: Equations}

We are ultimately interested in the spin orientations of the inner BHs at merger
as a function of initial conditions. Since they evolve independently to leading
post-Newtonian order, we focus on the dynamics of a single BH spin $\uv{S}$.
Since the spin magnitude does not enter into the dynamics, we write $\bm{S}
\equiv \uv{S}$ for brevity (i.e.\ $\bm{S}$ is a unit vector). Neglecting
spin-spin interactions, $\bm{S}$ undergoes de Sitter precession about $\bm{L}$
as
\begin{align}
    \rd{\bm{S}}{t} &= \Omega_{\rm SL}\hat{\bm{L}} \times \bm{S}
            \label{eq:dsdt},\\
        \Omega_{\rm SL} &= \frac{3Gn\p{m_2 + \mu/3}}{2c^2a\p{1 - e^2}}.
\end{align}

To analyze the dynamics of the spin vector, we go to co-rotating frame with
$\uv{L}$ about $\uv{L}_{\rm out}$, where Eq.~\eqref{eq:dsdt} becomes
\begin{align}
    \p{\rd{\bm{S}}{t}}_{\rm rot}
        &= \p{\Omega_{\rm L}\uv{L}_{\rm out}
            + \Omega_{\rm SL}\uv{L}} \times \bm{S},\\
        &= \bm{\Omega}_{\rm e} \times \bm{S}\label{eq:dsdt_weff},
\end{align}
where we define
\begin{align}
    \bm{\Omega}_{\rm e} &\equiv \Omega_{\rm L}\uv{L}_{\rm out} + \Omega_{\rm SL}
            \uv{L},\label{eq:weff_def}\\
    \Omega_{\rm L} &\equiv -\rd{\ascnode}{t}.\label{eq:Wldef}
\end{align}

In general, Eq.~\eqref{eq:dsdt_weff} is difficult to analyze, since $\Omega_{\rm
L}$, $\Omega_{\rm SL}$ and $I$ all vary significantly within each LK cycle and
change secularly over timescale $\sim t_{\rm GW}(e)$. However, at times when
both $t_{\rm GW}(e) \gg \Omega_{\rm GR}^{-1}$ and $t_{\rm GW}(e) \ll
j\p{e_{\max}}t_{\rm LK}$ are satisfied, the system can be treated as nearly
periodic within each LK cycle\footnote{Note that apsidal precession is a
non-dissipative effect, so the LK cycle is nearly periodic even when $\Omega_{\rm
GR} \gtrsim \p{j(e)t_{\rm LK}}^{-1}$.}. We can then rewrite
Eq.~\eqref{eq:dsdt_weff} in Fourier components
\begin{equation}
    \p{\rd{\bm{S}}{t}}_{\rm rot}
        = \s{\bm{\overline{\Omega}}_{\rm e} + \sum\limits_{N = 1}^\infty
            \bm{\Omega}_{\rm eN}\cos \p{\frac{2\pi N t}{P_{\rm LK}}}}
            \times \bm{S}.\label{eq:dsdt_fullft}
\end{equation}
The bar denotes an average over an LK cycle, and we write
$\bm{\overline{\Omega}}_{\rm e} \equiv \bm{\Omega}_{\rm e0}$ for convenience. We
adopt convention where $t = 0$ is the maximum eccentricity phase of the LK
cycle. The Fourier components $\bm{\Omega}_{\rm eN}$ then vary over timescales
$\sim t_{\rm GW}(e)$.

\subsection{Component Form}

For later analysis, it is also useful to write Eq.~\eqref{eq:dsdt_fullft} in
components. To do so, it is useful to define inclination angle $\bar{I}_{\rm e}$
as the angle between $\bm{\overline{\Omega}}_{\rm e}$ and $\bm{L}_{\rm out}$ as
shown in Fig.~\ref{fig:3vec}. To express $\bar{I}_{\rm e}$ algebraically, we
first define LK-averaged quantities
\begin{figure}
    \centering
    \includegraphics[width=0.5\columnwidth]{LK90_plots/7_3vec_cropped.png}
    \caption{Definition of angles, shown in plane of the two angular momenta
    $\bm{L}_{\rm out}$ and $\bm{L}$. Note that for $I_0 > 90^\circ$, implying
    $\bar{I} > 90^\circ$, we have $\bar{I}_{\rm e} < 0$ since $\Omega_{\rm L} <
    0$. The bottom right shows our choice of coordinate axes.}\label{fig:3vec}
\end{figure}
\begin{align}
    \overline{\Omega_{\rm SL} \sin I} &\equiv
            \overline{\Omega}_{\rm SL} \sin \bar{I},\\
    \overline{\Omega_{\rm SL} \cos I} &\equiv
            \overline{\Omega}_{\rm SL} \cos \bar{I}.\label{eq:barI}
\end{align}
It then follows that
\begin{equation}
    \tan \bar{I}_{\rm e} = \frac{\mathcal{A}\sin \bar{I}}{
        1 + \mathcal{A}\cos \bar{I}},\label{eq:ie_def}
\end{equation}
where
\begin{equation}
    \mathcal{A} \equiv \frac{\overline{\Omega}_{\rm SL}}{
        \overline{\Omega}_{\rm L}}.
\end{equation}

To do so, we choose non-inertial coordinate system where $\uv{z}
\propto \bm{\overline{\Omega}}_{\rm e}$ and $\uv{x}$ lies in the plane of
$\bm{L}_{\rm out}$ and $\bm{L}$ with positive component along $\bm{L}$ (see
Fig.~\ref{fig:3vec}). In this reference frame, the polar coordinate is just
$\theta_{\rm e}$ as defined above in Eq.~\eqref{eq:q_eff}, and the equation of
motion becomes
\begin{align}
    \rd{\bm{S}}{t} &= \s{\overline{\Omega}_{\rm e}\uv{z}
         + \sum\limits_{N = 1}^\infty
            \bm{\Omega}_{\rm eN}\cos \p{\frac{2\pi N t}{P_{\rm LK}}}}
        \times \bm{S}
        - \dot{I}_{\rm e} \uv{y} \times \bm{S}\label{eq:eom_prime}.
\end{align}
One further simplification lets us cast this vector equation of motion into
scalar form. Break $\bm{S}$ into components $\bm{S} = S_x\uv{x} + S_y \uv{y} +
\cos \theta_{\rm e} \uv{z}$ and define complex variable
\begin{equation}
    S_\perp \equiv S_x + iS_y.
\end{equation}
Then, we can rewrite Eq.~\ref{eq:eom_prime} as
\begin{align}
    \rd{S_{\perp}}{t} ={}& i\overline{\Omega}_{\rm e}S_\perp
            - \dot{I}_{\rm e} \cos \theta_{\rm e}
        + \sum\limits_{N = 1}^\infty\left[
            \cos \p{\Delta I_{\rm eN}}S_\perp\right. \nonumber\\
        &- \left.i\cos \theta \sin \p{\Delta I_{\rm eN}}\right]
            \Omega_{\rm eN}\cos N\Omega t.\label{eq:formal_eom_allgen}
\end{align}
Here, for each $\bm{\Omega}_{\rm eN}$ Fourier harmonic, we denote its magnitude
$\Omega_{\rm eN}$ and its inclination angle relative to $\bm{L}_{\rm out}$ as
$I_{\rm eN}$ using the same convention as Fig.~\ref{fig:3vec} (where
$\bar{I}_{\rm e} \equiv I_{\rm e0}$), and $\Delta I_{\rm eN} = \bar{I}_{\rm e} -
I_{\rm eN}$.

\section{Analysis: Deviation from Adiabaticity}\label{s:fast_merger}

In general, Eq.~\eqref{eq:dsdt_fullft} is difficult to study analytically. Two
possible simplifying approximations can be made: (i) the effect of the $N \geq
1$ harmonic terms is negligible, or (ii) the $\bm{\Omega}_{\rm eN}$ Fourier
coefficients all evolve very slowly compared to other dynamical timescales
(i.e.\ $\dot{I}_{\rm eN} \approx 0$). In this section, we analyze the former
approximation and provide accurate analytic descriptions of dynamics. The latter
approximation is studied in Section~\ref{s:harmonic}.

\subsection{The Adiabatic Invariant}

When neglecting the $N \geq 1$ harmonic terms, the equation of motion is
modified to
\begin{equation}
    \p{\rd{\overline{\bm{S}}}{t}}_{\rm rot}
        = \bm{\overline{\Omega}}_{\rm e}
            \times \overline{\bm{S}}.\label{eq:dsdt_0only}
\end{equation}

It is not obvious in what capacity analysis of Eq.~\eqref{eq:dsdt_0only} is
applicable to Eq.~\eqref{eq:dsdt_fullft}. Empirically, we find that the
LK-average of $\bm{S}$ (which no longer has unit norm) often evolves following
Eq.~\eqref{eq:dsdt_0only}, which is why we have used the notation
$\overline{\bm{S}}$. This is expected, as over timescales $\lesssim P_{\rm LK}$
the evolution of $\bm{S}$ should be dominated by the $N \geq 1$ harmonics we
have neglected. An intuitive interpretation of this result is that the $N \geq
1$ harmonics vanish when taking the LK-average of Eq.~\eqref{eq:dsdt_fullft}.

Eq.~\eqref{eq:dsdt_0only} has one desirable property: $\theta_{\rm e}$, given by
\begin{equation}
    \cos \theta_{\rm e} \equiv
        \overline{\bm{S}} \cdot \uv{\overline{\Omega}}_{\rm e},
        \label{eq:q_eff}
\end{equation}
is an adiabatic invariant. The adiabaticity condition requires the precession
axis evolve slowly compared to the precession frequency at all times:
\begin{equation}
    \rd{\bar{I}_{\rm e}}{t} \ll \overline{\Omega}_{\rm e}.\label{eq:ad_constr}
\end{equation}
Here, $\bar{I}_{\rm e}$ changes on timescale $t_{\rm GW}(e)$ while
\begin{equation}
    \overline{\Omega}_{\rm e} \sim \max\p{\Omega_{\rm GR}(e_{\max}),
        j(e_{\max})t_{\rm LK}}j(e_{\max}).
\end{equation}
We see that the adiabatic assumption is more stringent than the condition for
the Fourier decomposition in Eq.~\eqref{eq:dsdt_fullft} to be valid by a factor
$\sim j(e_{\max})$.

\subsection{Calculating Deviation from Adiabaticity}\label{ss:eom_0}

In real systems, the particular extent to which $\theta_{\rm e}$ is conserved
depends on how well Eq.~\eqref{eq:ad_constr} is satisfied. In this subsection,
we derive a loose bound on the total non-conservation of $\theta_{\rm e}$, then
in the next subsection we show this bound can be estimated from initial
conditions.

When neglecting harmonic terms, the scalar equation of motion
Eq.~\eqref{eq:formal_eom_allgen} becomes
\begin{align}
    \rd{S_{\perp}}{t} &= i\overline{\Omega}_{\rm e}S_\perp
            - \dot{I}_{\rm e} \cos \theta_{\rm e}.
\end{align}
This can be solved in closed form using an integrating factor. Defining
\begin{equation}
    \Phi(t) \equiv \int\limits^t \overline{\Omega}_{\rm e}\;\mathrm{d}t,
\end{equation}
we obtain solution up until some final time $t_{\rm f}$
\begin{equation}
    e^{-i\Phi}S_{\perp}\bigg|_{-\infty}^{t_{\rm f}}
        = -\int\limits_{-\infty}^{t_{\rm f}}
            e^{-i\Phi(\tau)}\dot{I}_{\rm e} \cos \theta\;\mathrm{d}\tau.
            \label{eq:formal_sol_0}
\end{equation}
It can be seen that, in the adiabatic limit [Eq.~\eqref{eq:ad_constr}],
$\abs{S_\perp} = \sin \theta_{\rm e}$ is indeed conserved, as the phase of the
integrand in the right hand side varies much faster than the magnitude.

\begin{figure*}
    \centering
    \includegraphics[width=0.9\textwidth]{LK90_plots/4sim_90_350_zoom.png}
    \caption{The same simulation as Fig.~\ref{fig:4sim_90_350} but shown
    focusing on the region where $\mathcal{A} \simeq 1$. The top three
    panels depict $a$, $e$, $I$ and $\bar{I}$ as before, but in addition $I_{\rm
    e}$ [Eq.~\eqref{eq:ie_def}] and $I_{\rm e1}$ are shown in the third panel. The
    bottom three panels depict the frequency ratios between the zeroth and first
    Fourier components of $\bm{\Omega}_{\rm e}$ to the LK frequency $\Omega =
    2\pi / P_{\rm LK}$; the magnitude of oscillation of $\theta_{\rm e}$ away
    from its initial value (red dots) as well as amplitude estimates due to
    non-adiabatic effects [green, Eq.~\eqref{eq:nonad_dqeff}] and due to
    resonances with harmonic terms [blue, Eq.~\eqref{eq:harmonic_dqeff}]; and
    the same characteristic frequencies as before. In the bottom middle panel,
    it is clear that oscillations in $\theta_{\rm e}$ are dominantly driven by
    interactions with the $N = 1$ harmonic.}\label{fig:4sim_90_350_zoom}
\end{figure*}

\begin{figure*}
    \centering
    \includegraphics[width=0.9\textwidth]{LK90_plots/4sim_90_200_zoom.png}
    \caption{Same as Fig.~\ref{fig:4sim_90_350_zoom} except for $I_0 =
    90.2^\circ$, corresponding to a faster merger and a total change in
    $\theta_{\rm e}$ of $\approx 2^\circ$. In the bottom middle panel, the
    nonadiabatic contribution is more significant and causes much poorer
    conservation of $\theta_{\rm e}$.}\label{fig:4sim_90_200_zoom}
\end{figure*}

Recalling $\abs{S_{\perp}} = \sin \theta_{\rm e}$ and analyzing
Eq.~\eqref{eq:formal_sol_0}, we see that $\sin \theta_{\rm e} \approx
\theta_{\rm e}$ oscillates about its value at $t = -\infty$ with amplitude
\begin{equation}
    \abs{\Delta \theta_{\rm e}} \sim
        \frac{\dot{I}_{\rm e}}{\overline{\Omega}_{\rm e}}.\label{eq:nonad_dqeff}
\end{equation}
This is compared to the $\Delta \theta_{\rm e}$ from simulations in the bottom
center panels of Figs.~\ref{fig:4sim_90_350_zoom} and~\ref{fig:4sim_90_200_zoom}
as the green line. We see that in the latter simulation, the faster merger, the
order of magnitude of $\abs{\Delta \theta_{\rm e}}$ is somewhat well predicted,
while in the slower merger a second contribution dominates $\Delta \theta_{\rm
e}$ oscillations, discussed in Section~\ref{s:harmonic}.

Furthermore, if we denote $\abs{\Delta \theta_{\rm e}}^{\rm f}$ to be the total
change in $\theta_{\rm e}$ from $t = -\infty$ to merger, we can give loose
bound\footnote{Given the complicated evolution of $\overline{\Omega}_{\rm e}$
and $\dot{I}_{\rm e}$, it is difficult to give a more exact bound on the
deviation from adiabaticity. In practice, deviations $\lesssim 1^\circ$ are
observationally indistinguishable, so the exact scaling in this regime is
negligible.}
\begin{equation}
    \abs{\Delta \theta_{\rm e}}^{\rm f} \lesssim
        \abs{\frac{\dot{I}_{\rm e}}{\overline{\Omega}_{\rm e}}}_{\max}.
        \label{eq:nonad_dqeff_tot}
\end{equation}

\subsection{Estimate of Deviation from Adiabaticity from Initial Conditions}

To estimate $\abs{\dot{I}_{\rm e} / \overline{\Omega}_{\rm e}}_{\max}$ from
initial conditions, we first differentiate Eq.~\eqref{eq:ie_def},
\begin{equation}
    \dot{I}_{\rm e} = \p{\frac{\dot{\mathcal{A}}}{
            \mathcal{A}}}
        \frac{\mathcal{A} \sin \bar{I}}{
            1 + 2\mathcal{A}\cos \bar{I}
                + \mathcal{A}^2}.
\end{equation}
It also follows from Eq.~\eqref{eq:weff_def} that
\begin{equation}
    \overline{\Omega}_{\rm e} = \overline{\Omega}_{\rm L}
        \p{1 + 2\mathcal{A}\cos \bar{I}
            + \overline{\mathcal{A}}^2}^{1/2},
\end{equation}
from which we obtain
\begin{equation}
    \abs{\frac{\dot{I}_{\rm e}}{\overline{\Omega}_{\rm e}}}
        = \abs{\frac{\dot{\mathcal{A}}}{
            \mathcal{A}}}
        \frac{1}{\abs{\overline{\Omega}_{\rm L}}}
        \frac{\mathcal{A} \sin \bar{I}}{
            \p{1 + 2\mathcal{A}\cos \bar{I}
                + \mathcal{A}}^{3/2}}.
\end{equation}
This is maximized when $\mathcal{A} \simeq 1$, and so we obtain that
the maximum deviation should be bounded by
\begin{equation}
    \abs{\frac{\dot{I}_{\rm e}}{\overline{\Omega}_{\rm e}}}_{\max}
        \simeq \abs{\frac{\dot{\mathcal{A}}}{\mathcal{A}}}
            \frac{1}{\abs{\overline{\Omega}_{\rm L}}}
            \frac{\sin \bar{I}}{\p{2 + 2\cos \bar{I}}^{3/2}}.
            \label{eq:idot_over_W}
\end{equation}

To evaluate this, we make two assumptions: (i) $\bar{I}$ is approximately
constant, and (ii) $j(e)$ evaluated at $\mathcal{A} \simeq 1$ can be
approximated as a constant multiple of the initial $j(e_{\max})$, so that
\begin{equation}
    j_{\star} \equiv j(e_{\star}) = f
        \sqrt{\frac{5}{3}\cos^2 I_0},\label{eq:jstar_ansatz}
\end{equation}
for some unknown factor $f > 1$; we use star subscripts to denote evaluation at
$\mathcal{A} \simeq 1$. $f$ turns out to be relatively insensitive to $I_0$.
This can be as systems with lower $e_{\max}$ values taking more cycles to attain
$\mathcal{A} \simeq 1$, resulting in all systems experiencing a similar amount
of decay due to GW radiation.

For simplicity, let's first assume $\mathcal{A} \simeq 1$ is
satisfied when the LK oscillations are mostly suppressed, and $e_\star \approx
1$ throughout the LK cycle (we will later see that the scalings are the same in
the LK-oscillating regime). Then we can write
\begin{align}
    \mathcal{A} &\simeq \frac{3Gn\p{m_2 + \mu/3}}{
        2c^2a j^2}
            \s{\frac{3\cos \bar{I}}{
                4t_{\rm LK}} \frac{1 + 3e^2/2}{j}}^{-1},\\
        &\simeq \frac{G(m_2 + \mu/3) m_{12}\tilde{a}_3^3}{
            c^2m_3a^4 j \cos \bar{I}},\label{eq:abar_eq1}\\
        &\propto \frac{1}{a^4j},\\
    \frac{\dot{\mathcal{A}}}{\mathcal{A}}
        &= -4\p{\frac{\dot{a}}{a}}_{\rm GW}
            + \frac{e}{j^2}\p{\rd{e}{t}}_{\rm GW}.
\end{align}
Approximating $e_\star \approx 1$ in Eqs.~\eqref{eq:dadt_gw} and~\eqref{eq:dedt_gw}
gives
\begin{align}
    \s{\frac{\dot{\mathcal{A}}}{\mathcal{A}}}_{
        \mathcal{A} = 1}
        &\simeq \frac{64G^3 \mu m_{12}^2}{5c^5a_\star^4j_\star^7} \times 15,\\
    \overline{\Omega}_{\rm L, \star}
        &\approx \frac{3\cos \bar{I}}{2t_{\rm LK}j_\star},\\
    \abs{\frac{\dot{I}_{\rm e}}{\overline{\Omega}_{\rm e}}}_{\max}
        &\approx \frac{128G^3 \mu m_{12}^2}{c^5 a_\star^4j_\star^6}
            \frac{t_{\rm LK}}{\cos \bar{I}}
            \frac{\sin \bar{I}}{\p{2 + 2\cos \bar{I}}^{3/2}}.
\end{align}
With the ansatz for $j_\star$ given by Eq.~\eqref{eq:jstar_ansatz} and requiring
Eq.~\eqref{eq:abar_eq1} equal $1$ for a given $j_\star$ and $a_\star$ gives us
the final expression
\begin{align}
    \abs{\frac{\dot{I}_{\rm e}}{\overline{\Omega}_{\rm e}}}_{\max}
        \approx{}& \frac{128 G^3 \mu m_{12}^3 \tilde{a}_3^3}{c^5
        \sqrt{Gm_{12}} m_3}
            \p{\frac{c^2 m_3 \cos \bar{I}}{G(m_2 + \mu / 3) m_{12}
                \tilde{a}_3^3}}^{11/8}\nonumber\\
        &\times \p{j_\star}^{-37/8}
            \frac{\tan \bar{I}}{
            \p{2 + 2 \cos \bar{I}}^{3/2}}.\label{eq:prediction}
\end{align}
% note this is equal to YS's I/\Omega_e to some small factor, a few percent?
The agreement of Eq.~\eqref{eq:prediction} with numerical simulation is
remarkable, as shown in Fig.~\ref{fig:good_quants}.
\begin{figure}
    \centering
    \includegraphics[width=\columnwidth]{LK90_plots/good_quants.png}
    \caption{Comparison of $\abs{\dot{I}_{\rm e} /
    \overline{\Omega}_{\rm e}}_{\max}$ extracted from simulations and using
    Eq.~\eqref{eq:prediction}, where we take $f = 2.6$ in
    Eq.~\eqref{eq:jstar_ansatz}. The merger time $P_{\rm m}$ is shown along the
    top axis of the plot in units of the characteristic LK timescale at the
    start of inspiral $t_{\rm LK, 0}$; the LK period is initially of order a few
    $t_{\rm LK, 0}$. The agreement is remarkable for mergers that are more
    adiabatic (towards the right).}\label{fig:good_quants}
\end{figure}

Above, we assumed that $\mathcal{A} \simeq 1$ is satisfied when the eccentricity
is mostly constant (see Fig.~\ref{fig:4sim_90_350} for an indication of how
accurate this is for the parameter space explored in
Fig.~\ref{fig:good_quants}). It is also possible that $\mathcal{A} \simeq 1$
occurs when the eccentricity is still undergoing substantial oscillations. In
fact, Eq.~\eqref{eq:prediction} is still accurate in this regime when replacing
$e$ with $e_{\max}$, due to the following analysis. Recall that when $e_{\min}
\ll e_{\max}$, the binary spends a fraction $\sim j(e_{\max})$ of the LK cycle
near $e \simeq e_{\max}$ \citep{anderson2016formation}. This fraction of the LK
cycle dominates both GW dissipation and $\overline{\Omega}_{\rm e}$ precession.
Thus, both terms in the expression $\dot{I}_{\rm e} / \overline{\Omega}_{\rm e}$
are evaluated near the maximum eccentricity and are suppressed by a factor
of $j(e_{\max})$, which cancels out.

The accuracy of Eq.~\eqref{eq:prediction} in bounding the total change in
$\Delta \theta_{\rm e}^{\rm f}$ over inspiral is shown in
Fig.~\ref{fig:deviations}. Note that conservation of $\theta_{\rm e}$ is
generally much better than Eq.~\eqref{eq:prediction} predicts; cancellation of
phases in Eq.~\eqref{eq:formal_sol_0} is generally more efficient than
Eq.~\eqref{eq:prediction} assumes.
\begin{figure}
    \centering
    \includegraphics[width=\columnwidth]{LK90_plots/deviations_one.png}
    \caption{Total change in $\theta_{\rm e}$ over inspiral as a function of
    initial inclination $I_0$, where the initial $\bm{\overline{\Omega}}_{\rm
    e}$ is computed without GW dissipation. For each $I_0$, $100$ simulations
    are run for $\bm{S}$ on a uniform, isotropic grid. Plotted for comparison is
    the bound $\abs{\Delta \theta_{\rm e}}^{\rm f} \lesssim \abs{\dot{I}_{\rm e} /
    \overline{\Omega}_{\rm e}}_{\max}$, using the analytical scaling given by
    Eq.~\eqref{eq:prediction}. It is clear that the given bound is not tight but
    provides an upper bound for non-conservation of $\theta_{\rm e}$ due to
    nonadiabatic effects. At the right of the plot, the accuracy saturates: this
    is because neglecting GW dissipation causes inaccuracies when computing the
    average $\bm{\overline{\Omega}}_{\rm e}$.}\label{fig:deviations}
\end{figure}

\section{Analysis: Resonances and Breakdown of $\theta_{\rm e}$ Conservation
}\label{s:harmonic}

In the previous section, we neglected the $N \geq 1$ Fourier harmonics in
Eq.~\eqref{eq:dsdt_fullft}. However, this assumption breaks down near certain
resonances described below. For simplicity, we ignore the effects of GW
dissipation in this section and assume the system is exactly periodic (and
$\dot{I}_{\rm e} = 0$). With these assumptions, the system resembles that
studied in \citet{storch}, but the dynamics turn out to be vastly different.
Below, we present an analysis sufficient for the key results of this paper, and
give a more formal treatment in Appendix~\ref{app:nondisp}.

When $\dot{I}_{\rm e} = 0$, the scalar equation of motion
Eq.~\eqref{eq:formal_eom_allgen} is then:
\begin{align}
    \rd{S_{\perp}}{t} ={}& i\overline{\Omega}_{\rm e}S_\perp
        + \sum\limits_{N = 1}^\infty[
            \cos \p{\Delta I_{\rm eN}}S_\perp \nonumber\\
        &- i\cos \theta \sin \p{\Delta I_{\rm eN}}]
            \Omega_{\rm eN}\cos N\Omega t.\label{eq:formal_sol_gen}
\end{align}
Resonances can occur when $\overline{\Omega}_{\rm e} = N\Omega$. Since
$\overline{\Omega}_{\rm e} \lesssim \Omega$ for most regions of parameter space
(see Fig.~\ref{fig:dWs}), we restrict our analysis to resonances with the $N =
1$ component. The two possible resonant behaviors are a parametric resonance
[modulation of the oscillation frequency in Eq.~\eqref{eq:formal_sol_gen}] and
resonant forcing by the second term. Parametric resonances are typically very
narrow and are therefore hard to excite as the system's frequencies change under
GW dissipation. As such, we consider only the effect of the second term in
Eq.~\eqref{eq:formal_sol_gen}.

Restricting our attention to $N = 1$ and neglecting the parametric term, the
equation of motion reduces to
\begin{align}
    \rd{S_{\perp}}{t} &\approx i\overline{\Omega}_{\rm e}S_\perp
        - i\cos \theta_{\rm e} \sin \p{\Delta I_{\rm eN}} \Omega_{\rm eN}
            \cos \p{N\Omega t}.
\end{align}
We can approximate $\cos \p{N\Omega t} \approx e^{iN\Omega t} / 2$, as the
$e^{-iN\Omega t}$ component is far from resonance. Then we can write down
solution
\begin{align}
    \Phi(t) &= \int\limits^t \overline{\Omega}_{\rm e}\;\mathrm{d}\tau,\\
    e^{-i\Phi}S_{\perp}\bigg|_{-\infty}^\infty
        &= -\int\limits_{-\infty}^\infty
            \frac{i\sin\p{\Delta I_{\rm e1}} \Omega_{\rm e1}}{2}
                e^{-i\Phi(\tau) + i\Omega \tau} \cos \theta_{\rm e}
            \;\mathrm{d}\tau.\label{eq:harmonic_dS}
\end{align}
Thus, similarly to Section~\ref{ss:eom_0}, $\abs{\Delta \theta_{\rm e}}$ can be
bound by
\begin{equation}
    \abs{\Delta \theta_{\rm e}} \sim \frac{1}{2}\frac{\sin \p{\Delta I_{\rm e1}}
        \Omega_{\rm e1}}{\abs{\Omega - \overline{\Omega}_{\rm e}}}.
        \label{eq:harmonic_dqeff}
\end{equation}
This is shown as the blue line in the bottom center panels of
Fig.~\ref{fig:4sim_90_350_zoom} and~\ref{fig:4sim_90_200_zoom}. We see that the
amplitude of oscillations in $\theta_{\rm e}$ are well described by
Eq.~\eqref{eq:harmonic_dqeff}, particularly in the former case (where the
non-adiabatic contribution is weaker).

Again analogously to Section~\ref{ss:eom_0}, we obtain loose bound for total
nonconservation of $\theta_{\rm e}$
\begin{equation}
    \abs{\Delta \theta_{\rm e}}^{\rm f} \lesssim \frac{1}{2}
        \abs{\frac{\sin \p{\Delta I_{\rm e1}}
        \Omega_{\rm e1}}{ \Omega - \overline{\Omega}_{\rm e}}}_{\max}.
        \label{eq:harmonic_dqeff_tot}
\end{equation}
If we assume the right hand side in Eq.~\eqref{eq:harmonic_dqeff_tot} is
maximized initially, we can predict $\Delta \theta_{\rm e}^{\rm f}$ for mildly
LK-enhanced mergers \citep{bin1} rather well, as shown in
Fig.~\ref{fig:bin_comp_zoom}.
\begin{figure}
    \centering
    \includegraphics[width=\columnwidth]{LK90_plots/6bin_comp_zoom.png}
    \caption{Plot of $\theta_{\rm sl}^{\rm f}$ for $m_1 = m_2 = m_3 =
    30M_{\odot}$, $a_{\rm in} = 0.1\;\mathrm{AU}$, $\tilde{a}_3 =
    3\;\mathrm{AU}$, $e_3 = 0$, where $e_0 = 0.001$ and $\theta_{\rm sl}^{\rm i}
    = 0$ over a restricted range of $I_0$ \citep[analogous to the top panel of
    Fig.~4 in][]{bin1}. The blue dots denote $\theta_{\rm sl}^{\rm f}$ obtained
    from numerical simulation. The green line gives $\theta_{\rm sl}^{\rm f}$
    assuming perfect conservation of $\theta_{\rm e}$, and the green shaded area
    shows the expected deviation following
    Eq.~\eqref{eq:harmonic_dqeff_tot}.}\label{fig:bin_comp_zoom}
\end{figure}

While Eq.~\eqref{eq:harmonic_dqeff} depends on the properties of the $N = 1$
Fourier component, the conditions for substantial $\theta_{\rm e}$
non-conservation can be understood in terms of physical quantities:
\begin{itemize}
    \item $\sin \p{\Delta I_{\rm e1}}$ is small unless $\mathcal{A} \simeq
        1$. Otherwise, $\bm{\Omega}_{\rm e}$ does not nutate appreciably within
        an LK cycle, and all the $\bm{\Omega}_{\rm eN}$ are aligned with
        $\bm{\overline{\Omega}}_{\rm e}$, implying all the $\Delta I_{\rm eN} \approx
        0$.

    \item Smaller values of both $e_{\min}$ and $e_{\max}$ increase
        $\overline{\Omega}_{\rm e} / \Omega$, as shown in Fig.~\ref{fig:dWs},
        strengthening the interaction with the $N = 1$ resonance.
\end{itemize}
\begin{figure}
    \centering
    \includegraphics[width=\columnwidth]{LK90_plots/5_dWs.png}
    \includegraphics[width=\columnwidth]{LK90_plots/5_dWs_inner.png}
    \caption{For the fiducial parameters (top two plots) and for the compact
    parameter regime studied in Fig.~\ref{fig:bin_comp_zoom} and \citet{bin1}
    (bottom two plots), $e_{\max}$ and $\overline{\Omega}_{\rm e} / \Omega$ as a
    function of $I_0$ for varying values of $e_{\min}$. Both smaller $e_{\min}$
    and $e_{\max}$ values more easily satisfy the resonant condition
    $\overline{\Omega}_{\rm e} / \Omega \approx 1$.}\label{fig:dWs}
\end{figure}

LK-driven coalescence causes $\mathcal{A}$ to increase on a similar
timescale to that of $e_{\min}$ increase (see Fig.~\ref{fig:4sim_90_350}). As
such, we conclude that the effect of harmonic terms generally only affects
$\theta_{\rm e}$ conservation when $\mathcal{A} \approx 1$ initially.

\section{Conclusion and Discussion}\label{s:discussion}

Relation between $\theta_{\rm e}$ and $\theta_{\rm sl}^{\rm f}$, as a function
of $I$.

The ``chaotic'' behavior in Paper I is because it satisfies the heuristic
provided at the end of the harmonics section well.

Interestingly, harmonic terms begin to dominate $\Delta \theta_{\rm e}$ at quite
small inclinations $I_0 \lesssim 90.35^\circ$, but the non-adiabatic
contribution to nonconservation obviously dominates out to $I_0 \approx
90.4^\circ$.

\bibliography{Su_LK90}
\bibliographystyle{aasjournal}

\appendix

\section{Nondissipative Spin Dynamics}\label{app:nondisp}

\subsection{Floquet Analysis}

In this appendix, we consider the dynamics of the equation of motion
\begin{equation}
    \rd{\bm{S}}{t} = \bm{\Omega}_{\rm e} \times \bm{S},\label{eq:app_dsdt}
\end{equation}
where $\bm{\Omega}_{\rm e}$ is exactly periodic with period $P_{\rm LK}$. This
corresponds to taking the $t_{\rm GW} \to \infty$ limit of the systems
considered in the main text.

When analyzing a system bearing superficial resemblance to
Eq.~\eqref{eq:app_dsdt}, \citet{storch} (hereafter SL15) found chaotic dynamics
due to resonance overlap. The hallmark of chaos in their study is the presence
of fine structure in a bifurcation diagram (Fig.~1 of SL15) that shows the
oscillation amplitude of a misalignment angle $\theta_{\rm sl}$ when varying
system parameters. To generate an analogous bifurcation diagram for our problem,
we perform the following procedure: begin with $\bm{S} \parallel \bm{L}$, then
evolve Eq.~\eqref{eq:app_dsdt} while sampling both $\theta_{\rm sl}$ and
$\theta_{\rm e}$ at eccentricity maxima. We repeat this procedure while varying
the mass ratio $m_1 / m_{12}$ of the inner binary, which only changes
$\Omega_{\rm SL}$ without changing the orbital evolution. $\theta_{\rm e}$ turns
out to be the best analog to the angle $\theta_{\rm sl}$ in SL15. We show the
results of this exercise in Figs.~\ref{fig:bifurcation_70}
and~\ref{fig:bifurcation_88}.
\begin{figure}
    \centering
    \includegraphics[width=\columnwidth]{LK90_plots/M1_M2_thetasl_thetae_70.png}
    \caption{$I_0 = 70^\circ$}\label{fig:bifurcation_70}
\end{figure}
\begin{figure}
    \centering
    \includegraphics[width=\columnwidth]{LK90_plots/M1_M2_thetasl_thetae_88.png}
    \caption{$I_0 = 88^\circ$}\label{fig:bifurcation_88}
\end{figure}

It can be observed that the bifurcation diagram is smooth and does not exhibit
small scale structure, suggesting that Eq.~\eqref{eq:app_dsdt} does not produce
chaotic behavior. We can understand this using Floquet theory, as
Eq.~\eqref{eq:app_dsdt} is a linear system with periodic coefficients (the
system studied in SL15 is nonlinear, and so generates resonances that can
overlap). Floquet's theorem says that when a linear system with periodic
coefficients is integrated over a period, the evolution can be described by a
linear transformation, the \emph{monodromy matrix} $\bm{M}$ (TODO this needs to
be better worded).

In our problem, $\bm{M}$ must be a proper orthogonal matrix, as it represents
the effect of many infinitesimal rotations about $\bm{\Omega}_{\rm e}$. Since
$\bm{M}$ is a $3 \times 3$ matrix, its eigenvalues must be $1$ and two other
roots of unity that are complex conjugates of one another. Denote the
eigenvector with eigenvalue $1$ by $\bm{R}$, then we have proven that
Eq.~\eqref{eq:app_dsdt} consists of rotation about $\bm{R}$ every $P_{\rm LK}$.
Chiefly, it cannot be chaotic.

Numerically, we find that $\bm{R} \approx \overline{\bm{\Omega}}_{\rm e}$ except
near the sorts of resonances described in Section~\ref{s:harmonic}.

Toy model?

\end{document}
