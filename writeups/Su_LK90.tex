    \documentclass[
        fleqn,
        usenatbib,
        % referee,
    ]{mnras}
    \usepackage{
        amsmath,
        amssymb,
        newtxtext,
        newtxmath,
        graphicx,
        ae, aecompl,
        booktabs,
        caption,
        subcaption,
    }
    \usepackage[T1]{fontenc}
    \captionsetup{compatibility=false}

    \newcommand*{\rd}[2]{\frac{\mathrm{d}#1}{\mathrm{d}#2}}
    \newcommand*{\rtd}[2]{\frac{\mathrm{d}^2#1}{\mathrm{d}#2^2}}
    \newcommand*{\pd}[2]{\frac{\partial#1}{\partial#2}}
    \newcommand*{\md}[2]{\frac{\mathrm{D}#1}{\mathrm{D}#2}}
    \newcommand*{\at}[1]{\left.#1\right|}
    \newcommand*{\abs}[1]{\left|#1\right|}
    \newcommand*{\ev}[1]{\langle#1\rangle}
    \newcommand*{\bm}[1]{\boldsymbol{\mathbf{#1}}}
    \newcommand*{\uv}[1]{\hat{\bm{#1}}}
    \newcommand*{\p}[1]{\left(#1\right)}
    \newcommand*{\s}[1]{\left[#1\right]}
    \newcommand*{\z}[1]{\left\{#1\right\}}
    \DeclareMathOperator*{\argmin}{argmin}
    \DeclareMathOperator*{\argmax}{argmax}
    \DeclareMathOperator*{\med}{med}
    \DeclareMathOperator*{\sgn}{sgn}
    \let\Re\undefined
    \let\Im\undefined
    \DeclareMathOperator{\Re}{Re}
    \DeclareMathOperator{\Im}{Im}

\title[BH Triple Spin-Orbit Dynamics]{Spin-Orbit Dynamics in Hierarchical Black
Hole Triples: Analytical Theory}
\author[Authors]{
Yubo Su$^1$
Dong Lai$^1$
Bin Liu$^1$
\\
$^1$ Cornell Center for Astrophysics and Planetary Science, Department of
Astronomy, Cornell University, Ithaca, NY 14853, USA
}

\date{Accepted XXX\@. Received YYY\@; in original form ZZZ}

\pubyear{2020}

\begin{document}\label{firstpage}
\pagerange{\pageref{firstpage}--\pageref{lastpage}}
\maketitle

\begin{abstract}
    Abstract
\end{abstract}

\begin{keywords}
keywords % chktex 8
\end{keywords}

\section{Introduction}

This problem is important.

\section{Analytical Setup}

\subsection{Equations of Motion}

We study Lidov-Kozai (LK) oscillations due to an external perturber to
quadrupole order and include precession of pericenter and gravitational wave
radiation due to general relativity. Consider an inner black hole (BH) binary
with masses $m_1$ and $m_2$ having total mass $m_{12}$ and reduced mass $\mu$
orbited by a third BH with mass $m_3$. Call $a_3$ the orbital semimajor axis of
the third BH from the center of mass of the inner binary, and $e_3$ the
eccentricity of its orbit, and define effective semimajor axis
\begin{equation}
    \bar{a}_3 \equiv a_3\sqrt{1 - e_3^2}.
\end{equation}
We adopt the test particle approximation such that the orbit of the third mass
is fixed. Finally, call $\bm{L}_{\rm out} \equiv L_{\rm out} \uv{L}_{\rm out}$
the fixed angular momentum of the outer BH relative to the center of mass of the
inner BH binary, and call $\bm{L} \equiv L \uv{L}$ the orbital angular momentum
of the inner BH binary.

We then consider the motion of the inner binary, described by orbital elements
Keplerian orbital elements $(a, e, \Omega, I, \omega)$. The equations describing
the motion of these orbital elements is then \citep{peters1964,storch,bin2}
\begin{align}
    \rd{a}{t} &= -\frac{a}{t_{\rm GW}},\\
    \rd{e}{t} &= \frac{15}{8t_{\rm LK}} e\sqrt{1 - e^2}\sin 2\omega
        \sin^2 I,\\
    \rd{\Omega}{t} &= \frac{3}{4t_{\rm LK}}
        \frac{\cos I\p{5e^2 \cos^2\omega - 4e^2 - 1}}{\sqrt{1 - e^2}}
        + \Omega_{\rm GR},\\
    \rd{I}{t} &= \frac{15}{16}\frac{e^2\sin 2\omega \sin 2I}{
        \sqrt{1 - e^2}},\\
    \rd{\omega}{t} &= \frac{3}{4t_{\rm LK}}
        \frac{2\p{1 - e^2} + 5\sin^2\omega
            (e^2 - \sin^2 I)}{\sqrt{1 - e^2}}.
\end{align}
Here, we have defined
\begin{align}
    t_{\rm LK}^{-1} &= n\p{\frac{m_3}{m_{12}}}\p{\frac{a}{\bar{a}_3}}^3,\\
    t_{\rm GW}^{-1} &= \frac{64}{5}\frac{G^3 \mu m_{12}^2}{c^5a^4}
        \frac{1}{\p{1 - e^2}^{7/2}}\p{1 + \frac{73}{24}e^2
            + \frac{37}{96}e^4},\\
    \Omega_{\rm GR} &= \frac{3Gnm_{12}}{c^2a\p{1 - e^2}},
\end{align}
and $n = \sqrt{Gm_{12}/a^3}$ is the mean motion of the inner binary.

We are ultimately interested in the evolution of the spin angular momentum of
the inner BHs. Since they evolve independently to leading post-Newtonian order,
we focus on the dynamics of a single BH spin vector $\bm{S} = S\uv{S}$
Neglecting spin-spin interactions, $\uv{S}$ undergoes de Sitter precession about
$\bm{L}$ as
\begin{align}
    \rd{\hat{\bm{S}}}{t} &= \Omega_{\rm SL}\hat{\bm{L}} \times \hat{\bm{S}}
            \label{eq:dsdt},\\
        \Omega_{\rm SL} &= \frac{3Gn\p{m_2 + \mu/3}}{2c^2a\p{1 - e^2}}.
\end{align}

We next go to the co-rotating frame with $\uv{L}$ about $\uv{L}_{\rm out}$. Choose
$\uv{L}_{\rm out} = \uv{z}$, and choose the $\uv{x}$ axis such that $\uv{L}$
lies in the $x$-$z$ plane. In this coordinate system, Eq.~\eqref{eq:dsdt}
becomes
\begin{align}
    \p{\rd{\bm{S}}{t}}_{\rm rot}
        &= \p{-\rd{\Omega}{t}\uv{z} + \Omega_{\rm SL}\uv{L}} \times \uv{S},\\
        &= \bm{\Omega}_{\rm e} \times \uv{S}\label{eq:dsdt_weff},\\
    \bm{\Omega}_{\rm e} &\equiv \Omega_{\rm PL}\uv{z} + \Omega_{\rm SL}
            \p{\cos I \uv{z} + \sin I \uv{x}},\\
    \Omega_{\rm PL} &\equiv -\rd{\Omega}{t}.
\end{align}

In general, Eq.~\eqref{eq:dsdt_weff} is difficult to analyze, since $\Omega_{\rm
PL}$, $\Omega_{\rm SL}$ and $I$ all vary significantly within every LK cycle.
However, since these quantities are all periodic with period $T_{\rm LK}$, the
LK period, we can rewrite Eq.~\eqref{eq:dsdt_weff} in Fourier components
\begin{equation}
    \p{\rd{\uv{S}}{t}}_{\rm rot}
        = \s{\ev{\bm{\Omega}_{\rm e}} + \sum\limits_{N = 1}^\infty
            \bm{\Omega}_{\rm e, N}\cos \p{\frac{2\pi N t}{T_{\rm LK}}}}
            \times \uv{S}.\label{eq:dsdt_fullft}
\end{equation}
The angle brackets denote an average over an LK cycle. We adopt convention where
$t = 0$ is the maximum eccentricity phase of the LK cycle.

\subsection{Analysis}

Assuming the $N \geq 1$ terms can be dropped (see Appendix~\ref{s:harmonic}),
our objective becomes the analysis of the equation
\begin{equation}
    \p{\rd{\uv{S}}{t}}_{\rm rot}
        = \ev{\bm{\Omega}_{\rm e}} \times \uv{S}.\label{eq:dsdt_0only}
\end{equation}
For the remainder of the main text of the paper, we will omit the angle brackets
denoting averaging over an LK cycle.

\bibliographystyle{mnras}
\bibliography{Su_LK90}

\clearpage
\onecolumn
\appendix

\section{Effect of Harmonic Terms on Evolution: Averaging Revisited
}\label{s:harmonic}

In order for the harmonic terms in Eq.~\eqref{eq:dsdt_fullft} to be nonzero, the
system must be in the LK-oscillating regime. Consider the Hamiltonian
corresponding to Eq.~\eqref{eq:dsdt_fullft} \citep[see
e.g.][]{kinoshita,storch}:
\begin{equation}
    H = \s{\ev{\bm{\Omega}_{\rm e}} + \sum\limits_{N = 1}^\infty
            \bm{\Omega}_{\rm e, N}\cos \p{\frac{2\pi N t}{T_{\rm LK}}}}
            \cdot \uv{S}.
\end{equation}
Assume for simplicity that the system is not evolving, such that all Fourier
components $\ev{\bm{\Omega}_{\rm e}}, \bm{\Omega}_{\rm e, N}$ are constant in
time.

The objective is now to average $H$ over a suitable interval of time. Assume
that $\uv{S}$ is also periodic with some period $T_{\rm S}$, such that
\begin{equation}
    \uv{S} = \s{\ev{\uv{S}} + \sum\limits_{M = 1}^\infty
            \bm{S}_{\rm M}\exp \p{i\frac{2\pi M pt}{T}}}.
\end{equation}
Note that $\bm{S}_{\rm M}$ must generally be complex, to ensure that $\uv{S}
\cdot \uv{S}$ does not vary in time. Note also that $T_{\rm S} / T_{\rm LK}$ is
generally \emph{irrational}. Nevertheless, consider averaging over some interval
of time $T$ that is a near-integer multiple of both $T_{\rm S}$ and $T_{\rm
LK}$:
\begin{align}
    T &\approx pT_{\rm S} \approx qT_{\rm LK}\label{eq:T_def},\\
    \frac{1}{T}\int\limits_0^T H\;\mathrm{d}t
        &= \frac{1}{T}\int\limits_0^T
            \s{\ev{\bm{\Omega}_{\rm e}} + \sum\limits_{N = 1}^\infty
            \bm{\Omega}_{\rm e, N}\cos \p{\frac{2\pi N qt}{T}}}
            \cdot \s{\ev{\uv{S}} + \sum\limits_{M = 1}^\infty
            \bm{S}_{\rm M}\exp \p{i\frac{2\pi M pt}{T}}}\;\mathrm{d}t,\\
    \ev{H} &= \ev{\bm{\Omega}_{\rm e}} \cdot \ev{\bm{\uv{S}}}
            + \frac{1}{2}\sum\limits_{j = 1}^\infty
                \bm{\Omega}_{\rm e, jp} \cdot \left(\Re\bm{S}_{jq}\right).
                \label{eq:exp_h}
\end{align}

Note that $\ev{H}$ is a conserved quantity of the evolution of the system,
absent GW dissipation. When the summation in this equation can be neglected,
this implies $\theta_{\rm e}$ is a conserved quantity, as claimed in the main
text. To understand when the summation can be neglected, we make the following
observations:
\begin{itemize}
    \item The magnitudes of the coefficients $\bm{\Omega}_{\rm e, N}$ fall
        off exponentially with characteristic scale $N_{\rm scale} \sim
        \frac{\Delta t}{T_{\rm LK}}$, where $\Delta t$ is the width of the LK
        eccentricity peak. When $e_{\min} \approx 0$ and $e_{\max} \to 1$, this
        is $N_\Omega \sim j_{\min}^{1/2}$ \citep{anderson2016formation}. At
        later times, $N_\Omega$ decreases as $e_{\min}$ is excited and the LK
        cycle becomes less violent.

    \item In general, $\uv{S}$ tends to precess around $\ev{\bm{\Omega}_{\rm
        e}}$, albeit not uniformly in time. Therefore, $\ev{\uv{S}}$ is expected
        to be approximately aligned with $\ev{\bm{\Omega}_{\rm e}}$, while
        the $\bm{S}_{\rm M}$ are expected to be \emph{perpendicular} to
        $\ev{\bm{\Omega}_{\rm e}}$.

    \item The degree of misalignment of each of the $\bm{\Omega}_{\rm e, N}$
        from $\ev{\bm{\Omega}_{\rm e}}$ is related to the amplitude of nutation
        of $\bm{\Omega}_{\rm e}$ over an LK cycle. In particular, the $x$ and
        $z$ components of $\bm{\Omega}_{\rm e, N}$ will exhibit order-unity
        variations in cutoff harmonic $N_{\rm scale}$ when $I_{\rm e}$ is
        strongly nutating.

    \item Finally, since $\uv{S}$ is driven by $\bm{\Omega}_{\rm e}$, it must
        have a similar frequency spectrum, implying $M_{\rm scale} \sim N_{\rm
        scale}q / p$.
\end{itemize}

Thus, non-negligible contributions from the summation in Eq.~\eqref{eq:exp_h}
arise when: (i) $\bm{\Omega}_{\rm e}$ is nutating substantially, and (ii) $q, p$
are sufficiently small that a substantial number of terms $\bm{S}_{jq}$ dot
against the non-aligned $\bm{\Omega}_{\rm e, jp}$.

Of course, the only astrophysically relevant triples are those that merge within
a Hubble time, and the systems that best satisfy the assumptions laid out in
this study are those with well-separated inner binaries $a_{\rm in} \gtrsim
10\;\mathrm{AU}$. Such systems, which merge within just a few thousands of
$T_{\rm LK}$, evolve very quickly relative to $T_{\rm LK}$ at early times, and
exhibit very little $I_{\rm e}$ nutation at later times, and so the contribution
from such terms in Eq.~\eqref{eq:exp_h} can be neglected.

For systems with more compact binaries, such as those studied in \citep{bin1},
the contributions from such higher harmonics cannot be neglected. This explains
the breakdown of conservation of $\theta_{\rm e}$ in these systems.

\bsp
\label{lastpage} % chktex 24
\end{document}
