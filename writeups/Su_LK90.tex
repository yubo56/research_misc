    \documentclass[
        fleqn,
        usenatbib,
        % referee,
    ]{mnras}
    \usepackage{
        amsmath,
        amssymb,
        newtxtext,
        newtxmath,
        graphicx,
        ae, aecompl,
        booktabs,
        caption,
        subcaption,
    }
    \usepackage[T1]{fontenc}
    \captionsetup{compatibility=false}

    \newcommand*{\rd}[2]{\frac{\mathrm{d}#1}{\mathrm{d}#2}}
    \newcommand*{\rtd}[2]{\frac{\mathrm{d}^2#1}{\mathrm{d}#2^2}}
    \newcommand*{\pd}[2]{\frac{\partial#1}{\partial#2}}
    \newcommand*{\md}[2]{\frac{\mathrm{D}#1}{\mathrm{D}#2}}
    \newcommand*{\at}[1]{\left.#1\right|}
    \newcommand*{\abs}[1]{\left|#1\right|}
    \newcommand*{\ev}[1]{\langle#1\rangle}
    \newcommand*{\bm}[1]{\boldsymbol{\mathbf{#1}}}
    \newcommand*{\uv}[1]{\hat{\bm{#1}}}
    \newcommand*{\p}[1]{\left(#1\right)}
    \newcommand*{\s}[1]{\left[#1\right]}
    \newcommand*{\z}[1]{\left\{#1\right\}}
    \DeclareMathOperator*{\argmin}{argmin}
    \DeclareMathOperator*{\argmax}{argmax}
    \DeclareMathOperator*{\med}{med}
    \DeclareMathOperator*{\sgn}{sgn}
    \let\Re\undefined
    \let\Im\undefined
    \DeclareMathOperator{\Re}{Re}
    \DeclareMathOperator{\Im}{Im}

\title[BH Triple Spin-Orbit Dynamics]{Spin-Orbit Dynamics in Hierarchical Black
Hole Triples: Analytical Theory}
\author[Authors]{
Yubo Su$^1$
Dong Lai$^1$
Bin Liu$^1$
\\
$^1$ Cornell Center for Astrophysics and Planetary Science, Department of
Astronomy, Cornell University, Ithaca, NY 14853, USA
}

\date{Accepted XXX\@. Received YYY\@; in original form ZZZ}

\pubyear{2020}

\begin{document}\label{firstpage}
\pagerange{\pageref{firstpage}--\pageref{lastpage}}
\maketitle

\begin{abstract}
    Abstract
\end{abstract}

\begin{keywords}
keywords % chktex 8
\end{keywords}

\section{Introduction}

This problem is important.

\section{Analytical Setup}

\subsection{Orbital Evolution}

We study Lidov-Kozai (LK) oscillations due to an external perturber to
quadrupole order and include precession of pericenter and gravitational wave
radiation due to general relativity. Consider an inner black hole (BH) binary
with masses $m_1$ and $m_2$ having total mass $m_{12}$ and reduced mass $\mu$
orbited by a third BH with mass $m_3$. Call $a_3$ the orbital semimajor axis of
the third BH from the center of mass of the inner binary, and $e_3$ the
eccentricity of its orbit, and define effective semimajor axis
\begin{equation}
    \tilde{a}_3 \equiv a_3\sqrt{1 - e_3^2}.
\end{equation}
We adopt the test particle approximation such that the orbit of the third mass
is fixed. Finally, call $\bm{L}_{\rm out} \equiv L_{\rm out} \uv{L}_{\rm out}$
the fixed angular momentum of the outer BH relative to the center of mass of the
inner BH binary, and call $\bm{L} \equiv L \uv{L}$ the orbital angular momentum
of the inner BH binary.

We then consider the motion of the inner binary, described by orbital elements
Keplerian orbital elements $(a, e, \Omega, I, \omega)$. The equations describing
the motion of these orbital elements are then \citep{peters1964,storch,bin2}
\begin{align}
    \rd{a}{t} &= \p{\rd{a}{t}}_{\rm GW},\\
    \rd{e}{t} &= \frac{15}{8t_{\rm LK}} e\sqrt{1 - e^2}\sin 2\omega
        \sin^2 I + \p{\rd{e}{t}}_{\rm GW},\\
    \rd{\Omega}{t} &= \frac{3}{4t_{\rm LK}}
        \frac{\cos I\p{5e^2 \cos^2\omega - 4e^2 - 1}}{\sqrt{1 - e^2}}
        + \Omega_{\rm GR},\\
    \rd{I}{t} &= \frac{15}{16}\frac{e^2\sin 2\omega \sin 2I}{
        \sqrt{1 - e^2}},\\
    \rd{\omega}{t} &= \frac{3}{4t_{\rm LK}}
        \frac{2\p{1 - e^2} + 5\sin^2\omega
            (e^2 - \sin^2 I)}{\sqrt{1 - e^2}},
\end{align}
where we define
\begin{align}
    t_{\rm LK}^{-1} &= n\p{\frac{m_3}{m_{12}}}\p{\frac{a}{\tilde{a}_3}}^3,\\
    \p{\rd{a}{t}}_{\rm GW} &= -\frac{a}{t_{\rm GW}},\\
         &= \frac{64}{5}\frac{G^3 \mu m_{12}^2}{c^5a^3}
            \frac{1}{\p{1 - e^2}^{7/2}}\p{1 + \frac{73}{24}e^2
                + \frac{37}{96}e^4},\label{eq:dadt_gw}\\
    \p{\rd{e}{t}}_{\rm GW} &= -\frac{304}{15}\frac{G^3 \mu m_{12}^2}{c^5a^4}
        \frac{1}{\p{1 - e^2}^{5/2}}\p{1 + \frac{121}{304}e^2}\label{eq:dedt_gw}
            ,\\
    \Omega_{\rm GR} &= \frac{3Gnm_{12}}{c^2a\p{1 - e^2}},
\end{align}
and $n = \sqrt{Gm_{12}/a^3}$ is the mean motion of the inner binary. We will
also sometimes notate $j \equiv \sqrt{1 - e^2}$.

The evolution of these orbital elements has been well characterized in previous
studies \citep{anderson2016formation,bin1}. We focus on the evolution with LK
oscillations, in which case the evolution is approximately broken into two
phases [see panels (a) and (b) on both plots in Fig.~\ref{fig:4sim_90_350} for
reference]:
\begin{itemize}
    \item In the first phase, $e$ and $I$ vary significantly within each LK
        period. If $e_{\min} \ll e_{\max}$, where $e_{\min}$ and $e_{\max}$
        refer to the minimum and maximum $e$ within a LK period, then the system
        spends a fraction $\sqrt{1 - e_{\max}^{2}}$ of the LK period at $e
        \simeq e_{\max}$ \citep{anderson2016formation}.

        During this phase, $e_{\min}$ is excited to larger values under the dual
        GR effects of gravitational wave radiation and pericenter advance, while
        $a$ and $e_{\max}$ evolve comparatively little .

    \item In the second phase, $e_{\min} \approx e_{\max}$, and the system
        coalesces under gravitational wave radiation with little variation over
        each LK period.
\end{itemize}

\subsection{Spin Dynamics}

We are ultimately interested in the spin orientations of the inner BHs at merger
as a function of initial conditions. Since they evolve independently to leading
post-Newtonian order, we focus on the dynamics of a single BH spin vector
$\bm{S} = S\uv{S}$ Neglecting spin-spin interactions, $\uv{S}$ undergoes de
Sitter precession about $\bm{L}$ as
\begin{align}
    \rd{\hat{\bm{S}}}{t} &= \Omega_{\rm SL}\hat{\bm{L}} \times \hat{\bm{S}}
            \label{eq:dsdt},\\
        \Omega_{\rm SL} &= \frac{3Gn\p{m_2 + \mu/3}}{2c^2a\p{1 - e^2}}.
\end{align}

To analyze the dynamics of the spin vector, we go to co-rotating frame with
$\uv{L}$ about $\uv{L}_{\rm out}$. Choose $\uv{L}_{\rm out} = \uv{z}$, and
choose the $\uv{x}$ axis such that $\uv{L}$ lies in the $x$-$z$ plane. In this
0oordinate system, Eq.~\eqref{eq:dsdt} becomes
\begin{align}
    \p{\rd{\bm{S}}{t}}_{\rm rot}
        &= \p{-\rd{\Omega}{t}\uv{z} + \Omega_{\rm SL}\uv{L}} \times \uv{S},\\
        &= \bm{\Omega}_{\rm e} \times \uv{S}\label{eq:dsdt_weff},\\
    \bm{\Omega}_{\rm e} &\equiv \Omega_{\rm L}\uv{z} + \Omega_{\rm SL}
            \p{\cos I \uv{z} + \sin I \uv{x}},\label{eq:weff_def}\\
    \Omega_{\rm L} &\equiv -\rd{\Omega}{t}.
\end{align}

In general, Eq.~\eqref{eq:dsdt_weff} is difficult to analyze, since $\Omega_{\rm
L}$, $\Omega_{\rm SL}$ and $I$ all vary significantly within each LK period,
and we are interested in the final outcome after many LK periods. However, if we
assume $t_{\rm GW} \gg t_{\rm LK}$, then the system can be treated as nearly
periodic within each LK cycle. We can then rewrite Eq.~\eqref{eq:dsdt_weff} in
Fourier components
\begin{equation}
    \p{\rd{\uv{S}}{t}}_{\rm rot}
        = \s{\bm{\overline{\Omega}}_{\rm e} + \sum\limits_{N = 1}^\infty
            \bm{\Omega}_{\rm e, N}\cos \p{\frac{2\pi N t}{T_{\rm LK}}}}
            \times \uv{S}.\label{eq:dsdt_fullft}
\end{equation}
The bar denotes an average over an LK cycle. We adopt convention where $t = 0$
is the maximum eccentricity phase of the LK cycle.

We next assume that the $N \geq 1$ harmonics vanish when the equation of motion
is averaged over an LK cycle\footnote{While this is not strictly accurate and
gives incorrect results for certain parameters, it is an intuitively clear
picture and makes correct predictions for many physically relevant
configurations. A more rigorous discussion is provided in
Appendix~\ref{s:harmonic}.}, which gives
\begin{equation}
    \p{\rd{\uv{\overline{S}}}{t}}_{\rm rot}
        = \bm{\overline{\Omega}}_{\rm e}
            \times \uv{\overline{S}}.\label{eq:dsdt_0only}
\end{equation}
We accordingly define angle
\begin{equation}
    \cos \theta_{\rm e} \equiv
        \uv{\overline{S}} \cdot \uv{\overline{\Omega}}_{\rm e}.
        \label{eq:q_eff}
\end{equation}
Eq.~\eqref{eq:dsdt_0only} suggests that $\theta_{\rm e}$ should be a conserved
quantity when $\bm{\Omega}_{\rm e}$ varies adiabatically. The adiabaticity
condition requires the precession axis evolve slowly compared to the precession
frequency at all times:
\begin{equation}
    \abs{\rd{\uv{\overline{\Omega}}_{\rm e}}{t}} \ll
        \abs{\bm{\overline{\Omega}_{\rm e}}}. \label{eq:ad_constr}
\end{equation}
Since the orientation of $\bm{\overline{\Omega}}_{\rm e}$ changes on timescale
$t_{\rm GW}$, we see that the adiabatic assumption is roughly equivalent to
assuming the Fourier decomposition [Eq.~\eqref{eq:dsdt_fullft}] within each LK
period is valid.

\begin{figure}
    \centering
    \includegraphics[width=\columnwidth]{LK90_plots/4sim_90_350.png}
    \includegraphics[width=\columnwidth]{LK90_plots/4sim_90_350_zoom.png}
    \caption{Plot.}\label{fig:4sim_90_350}
\end{figure}

To be more precise, we define an inclination angle $I_{\rm e}$ for
$\bm{\overline{\Omega}}_{\rm e}$ as shown in Fig.~\ref{fig:3vec}.
\begin{figure}
    \centering
    \includegraphics[width=0.6\columnwidth]{LK90_plots/7_3vec_cropped.png}
    \caption{Definition of angles, shown in plane of the two angular momenta
    $\bm{L}_{\rm out}$ and $\bm{L}$, or the $\uv{x}$--$\uv{z}$ plane in the
    corotating frame. Note that for $I > 90^\circ$, $I_{\rm e} <
    0$.}\label{fig:3vec}
\end{figure}
Denoting also $\overline{\Omega}_{\rm e} \equiv \abs{\bm{\overline{\Omega}}_{\rm
e}}$, the adiabaticity condition can be expressed as
\begin{equation}
    \rd{I_{\rm e}}{t} \ll \overline{\Omega}_{\rm e}.\label{eq:ad_constr_idot}
\end{equation}

Next, we express $I_{\rm e}$ can be expressed in closed form. When the LK
oscillations are strong (``phase one''), we define averaged quantities
\begin{align}
    \overline{\Omega_{\rm SL} \sin I} &\equiv
            \overline{\Omega}_{\rm SL} \sin \bar{I},\\
    \overline{\Omega_{\rm SL} \cos I} &\equiv
            \overline{\Omega}_{\rm SL} \cos \bar{I}.
\end{align}
Then, using Eq.\eqref{eq:weff_def}, we can see that
\begin{equation}
    \tan I_{\rm e} = \frac{\overline{\mathcal{A}}\sin \bar{I}}{
        1 + \overline{\mathcal{A}}\cos \bar{I}},\label{eq:ie_def}
\end{equation}
where
\begin{equation}
    \overline{\mathcal{A}} \equiv \frac{\overline{\Omega}_{\rm SL}}{
        \overline{\Omega}_{\rm L}}.
\end{equation}
When LK oscillations are suppressed (``phase two''), we just have
$\overline{\Omega}_{\rm SL} = \Omega_{\rm SL}$, $\overline{\Omega}_{\rm L} =
\Omega_{\rm L}$, and $\bar{I} = I$.

\section{Analysis: Deviation from Adiabaticity}

In real systems, the particular extent to which $\overline{\theta}_{\rm e}$ is
conserved depends on how well Eq.~\eqref{eq:ad_constr} is satisfied. We will
first present equations of motion for $\overline{\theta}_{\rm e}$. We will then
derive accurate estimates for important quantities in these equations of motion,
and use these estimates to derive upper bounds on $\Delta \overline{\theta}_{\rm
e}$, the change in $\overline{\theta}_{\rm e}$ over the entire inspiral. Taken
together, this calculation estimates the deviation from adiabaticity as a
function of initial conditions.

\subsection{Equations of Motion}

From the corotating frame [Eq.~\eqref{eq:dsdt_0only}], consider going to the
reference frame where $\uv{z} = \uv{\overline{\Omega}}_{\rm e} $ by rotation
$-\dot{I}_{\rm e}\uv{y}$. In this reference frame, the polar coordinate is just
$\theta_{\rm e}$ as defined above in Eq.~\eqref{eq:q_eff}, and we call the
azimuthal coordinate $\phi_{\rm e}$. In this reference frame, the equation of
motion becomes
\begin{align}
    \p{\rd{\uv{S}}{t}}' &= \uv{\overline{\Omega}}_{\rm e} \times \uv{S}
        - \dot{I}_{\rm e} \uv{y} \times \uv{S}\label{eq:eom_prime}.
\end{align}
If we break $\uv{S}$ into components $\uv{S} = S_x\uv{x} + S_y \uv{y} + \cos
\theta_{\rm e} \uv{z}$ and define complex variable
\begin{equation}
    S_\perp \equiv S_x + iS_y = \sin\theta e^{i\phi_{\rm e}},
\end{equation}
we can rewrite Eq.~\ref{eq:eom_prime} as
\begin{equation}
    \rd{S_\perp}{t} = i\p{\overline{\Omega}_{\rm e}} S_\perp
        - \dot{I}_{\rm e} \cos \theta_{\rm e}.
\end{equation}
Defining
\begin{equation}
    \Phi(t) = \int\limits^t \overline{\Omega}_{\rm e}\;\mathrm{d}t,
\end{equation}
we obtain formal solution
\begin{equation}
    e^{-i\Phi}\s{S_{\perp}(t = \infty) - S_{\perp}(t = -\infty)}
        = -\int\limits_{-\infty}^\infty
            e^{-i\Phi(\tau)}\dot{I}_{\rm e} \cos \theta\;\mathrm{d}\tau.
\end{equation}
It can be seen that, in the adiabatic limit [Eq.~\eqref{eq:ad_constr_idot}],
$\abs{S_\perp}$ (and therefore $\theta_{\rm e}$) is conserved, as the phase of
the integrand in the right hand side varies much faster than the magnitude.
Furthermore, the deviation from exact conservation of $\abs{S_{\perp}}$ cannot
exceed $\dot{I}_{\rm e} / \overline{\Omega}_{\rm
e}$ so long as $\dot{I}_{\rm e} \lesssim \overline{\Omega}_{\rm
e}$\footnote{Given the complicated evolution of $\overline{\Omega}_{\rm e}$ and
$\dot{I}_{\rm e}$, it is difficult to give a more exact bound on the deviation
from adiabaticity. In practice, deviations $\lesssim 1^\circ$ are
astrophysically negligible, so the exact scaling in this regime is of little
consequence.}. In the following section, we show that this maximum value can be
calculated to good accuracy from initial conditions.

\subsection{Estimate of Deviation from Adiabaticity}

Towards estimating $\max \dot{I}_{\rm e} / \overline{\Omega}_{\rm e}$, we first
differentiate Eq.~\eqref{eq:ie_def},
\begin{equation}
    \dot{\overline{I}}_{\rm e} = \p{\frac{\dot{\overline{\mathcal{A}}}}{
            \overline{\mathcal{A}}}}
        \frac{\overline{\mathcal{A}} \sin \overline{I}}{
            1 + 2\overline{\mathcal{A}}\cos \overline{I}
                + \overline{\mathcal{A}}^2}.
\end{equation}
It can be easily hown from Eq.~\eqref{eq:weff_def} that
\begin{equation}
    \overline{\Omega}_{\rm e} = \overline{\Omega}_{\rm L}
        \p{1 + 2\overline{\mathcal{A}}\cos \overline{I}
            + \overline{A}^2}^{1/2},
\end{equation}
from which we obtain
\begin{equation}
    \abs{\frac{\dot{\overline{I}}_{\rm e}}{\overline{\Omega}_{\rm e}}}
        = \abs{\frac{\dot{\overline{\mathcal{A}}}}{
            \overline{\mathcal{A}}}}
        \frac{1}{\abs{\overline{\Omega}_{\rm L}}}
        \frac{\overline{\mathcal{A}} \sin \overline{I}}{
            \p{1 + 2\overline{\mathcal{A}}\cos \overline{I}
                + \overline{\mathcal{A}}}^{3/2}}.
\end{equation}
This is maximized when $\overline{\mathcal{A}} \simeq 1$, and so we obtain that
the maximum deviation should be bounded by
\begin{equation}
    \abs{\frac{\dot{I}_{\rm e}}{\overline{\Omega}_{\rm e}}}_{\max}
        \simeq \abs{\frac{\dot{\overline{\mathcal{A}}}}{\overline{\mathcal{A}}}}
            \frac{1}{\abs{\overline{\Omega}_{\rm L}}}
            \frac{\sin \overline{I}}{\p{2 + 2\cos \overline{I}}^{3/2}}.
            \label{eq:idot_over_W}
\end{equation}

To evaluate this, we make two assumptions: (i) $\overline{I}$ is approximately
constant, and (ii) $j_{\min} = \sqrt{1 - e_{\max}^2}$ evaluated at
$\overline{\mathcal{A}} \simeq 1$ is some constant multiple of the initial
$j_{\min}$, so that
\begin{equation}
    \p{j_{\min}}_{\overline{\mathcal{A}} \simeq 1} = f \sqrt{\frac{5}{3}\cos^2
        I_0},
\end{equation}
for some unknown factor $f > 1$. $f$ turns out to be relatively insensitive to
$I_0$, which is unsurprising, as systems with lower $e_{\max}$ values take more
cycles to attain $\overline{\mathcal{A}} \simeq 1$ and thus experience a similar
amount of decay due to GW radiation.

For simplicity, let's first assume $\overline{\mathcal{A}} \simeq 1$ is
satisfied when the LK oscillations are mostly suppressed, and $e \approx
e_{\max}$ throughout the LK cycle (we will later see that the scalings are the
same in the LK-oscillating regime). Then we can write
\begin{align}
    \overline{\mathcal{A}} &= \frac{\overline{\Omega}_{\rm SL}}{
        \overline{\Omega}_{\rm L}}\\
        &\simeq \frac{3n\p{m_2 + \mu/3}}{2aj^2}\s{\frac{3\cos \overline{I}}{
            4t_{\rm LK}} \frac{1 + 3e^2/2}{j}}^{-1},\\
        &\propto \frac{a^{-4}}{j},\\
    \frac{\dot{\overline{\mathcal{A}}}}{\overline{\mathcal{A}}}
        &= -4\p{\frac{\dot{a}}{a}}_{\rm GW}
            + \frac{e}{j^2}\p{\rd{e}{t}}_{\rm GW}.
\end{align}
Approximating $e \lesssim 1$ in Eqs.~\eqref{eq:dadt_gw} and~\eqref{eq:dedt_gw}
gives
\begin{align}
    \frac{\dot{\overline{\mathcal{A}}}}{\overline{\mathcal{A}}}
        &\simeq \frac{64\mu_{12}^2}{5a^4j^7}15,\\
    \overline{\Omega}_{\rm L}
        &\approx \frac{3\cos \overline{I}}{2t_{\rm LK}j},\\
    \abs{\frac{\dot{\overline{I}}_{\rm e}}{\overline{\Omega}_{\rm e}}}
        &\approx \frac{128\mu_{12}^2}{a^4j^7}
            \frac{t_{\rm LK}}{\cos \overline{I}}
            \frac{\sin \overline{I}}{\p{2 + 2\cos \overline{I}}^{3/2}}.
\end{align}
$a$ and $j$ are evaluated when $\overline{\mathcal{A}} \simeq 1$, providing a
further constraint:

\bibliographystyle{mnras}
\bibliography{Su_LK90}

\clearpage
\onecolumn
\appendix

\section{Effect of Harmonic Terms on Evolution: Averaging Revisited
}\label{s:harmonic}

Floquet!

\section{Deviation from Adiabaticity: Simplified Models}\label{s:ad_dev_toy}

\bsp
\label{lastpage} % chktex 24
\end{document}
