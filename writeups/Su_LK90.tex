% for i in ../scripts/lk90/5_dWs.png ../scripts/lk90/4sims_scan/deviations_one.png ../scripts/lk90/4sims/4sim_90_200_zoom.png ../scripts/lk90/4sims/4sim_90_350_zoom.png ../scripts/lk90/4sims/4sim_90_350.png ../scripts/lk90/7_3vec_cropped.png ../scripts/lk90/4sims/good_quants.png; do cp $i LK90_plots; done
    \documentclass[
        twocolumn,
        twocolappendix
    ]{aastex63}
    \usepackage{
        amsmath,
        amssymb,
        newtxtext,
        newtxmath,
        graphicx,
        ae, aecompl,
        booktabs
    }
    \usepackage[T1]{fontenc}

    \newcommand*{\rd}[2]{\frac{\mathrm{d}#1}{\mathrm{d}#2}}
    \newcommand*{\rtd}[2]{\frac{\mathrm{d}^2#1}{\mathrm{d}#2^2}}
    \newcommand*{\pd}[2]{\frac{\partial#1}{\partial#2}}
    \newcommand*{\md}[2]{\frac{\mathrm{D}#1}{\mathrm{D}#2}}
    \newcommand*{\at}[1]{\left.#1\right|}
    \newcommand*{\abs}[1]{\left|#1\right|}
    \newcommand*{\ev}[1]{\langle#1\rangle}
    \renewcommand*{\bm}[1]{\boldsymbol{\mathbf{#1}}}
    \newcommand*{\uv}[1]{\hat{\bm{#1}}}
    \newcommand*{\p}[1]{\left(#1\right)}
    \newcommand*{\s}[1]{\left[#1\right]}
    \newcommand*{\z}[1]{\left\{#1\right\}}
    \DeclareMathOperator*{\argmin}{argmin}
    \DeclareMathOperator*{\argmax}{argmax}
    \DeclareMathOperator*{\med}{med}
    \DeclareMathOperator*{\sgn}{sgn}
    \let\Re\undefined
    \let\Im\undefined
    \DeclareMathOperator{\Re}{Re}
    \DeclareMathOperator{\Im}{Im}

    \newcommand\aastex{AAS\TeX}
    \newcommand\latex{La\TeX}

    \received{XXXX}
    \revised{XXXX}
    \accepted{XXXX}
    \submitjournal{ApJL}

\shorttitle{BH Triple Spin-Orbit Dynamics}
\shortauthors{Y.\ Su et.\ al.}

\begin{document}

\title{Spin Dynamics in Hierarchical Black Hole Triples: Predicting Final
Spin-Orbit Misalignment Angle From Initial Conditions}

\correspondingauthor{Yubo Su}
\email{yubosu@astro.cornell.edu}

\author[0000-0001-8283-3425]{Yubo Su}% chktex 8
\affiliation{Cornell Center for Astrophysics and Planetary Science, Department
of Astronomy, Cornell University, Ithaca, NY 14853, USA}

\author[0000-0002-1934-6250]{Dong Lai}% chktex 8
\affiliation{Cornell Center for Astrophysics and Planetary Science, Department
of Astronomy, Cornell University, Ithaca, NY 14853, USA}

\author{Bin Liu}

\begin{abstract}
    Abstract
\end{abstract}

\keywords{keywords}

\section{Introduction}\label{s:intro}

This problem is important.

In Section~\ref{s:setup}, we set up the relevant equations of motion for the
orbital and spin evolution of the three BHs, and we argue for the primary result
of the paper, conservation of the angle $\theta_{\rm e}$. In
Sections~\ref{s:fast_merger} and~\ref{s:harmonic}, we consider two scenarios
under which conservation of $\theta_{\rm e}$ can be violated. We discuss and
conclude in Section~\ref{s:discussion}.

\section{Analytical Setup}\label{s:setup}

\subsection{Orbital Evolution}

We study Lidov-Kozai (LK) oscillations due to an external perturber to
quadrupole order and include precession of pericenter and gravitational wave
radiation due to general relativity. Consider an inner black hole (BH) binary
with masses $m_1$ and $m_2$ having total mass $m_{12}$ and reduced mass $\mu$
orbited by a third BH with mass $m_3$. Call $a_3$ the orbital semimajor axis of
the third BH from the center of mass of the inner binary, and $e_3$ the
eccentricity of its orbit, and define effective semimajor axis
\begin{equation}
    \tilde{a}_3 \equiv a_3\sqrt{1 - e_3^2}.
\end{equation}
For simplicity, we adopt the test particle approximation such that the orbit of
the third mass is fixed\footnote{At quadrupole order, including the
back-reaction terms is equivalent to considering LK oscillations of the inner
binary about a fixed total angular momentum axis rather than $\bm{L}_{\rm out}$
\citep[see e.g.][]{bin1,bin2}.}. Call $\bm{L}_{\rm out} \equiv L_{\rm out}
\uv{L}_{\rm out}$ the fixed angular momentum of the outer BH relative to the
center of mass of the inner BH binary, and call $\bm{L} \equiv L \uv{L}$ the
orbital angular momentum of the inner BH binary.

We then consider the motion of the inner binary, described by orbital elements
Keplerian orbital elements $(a, e, \Omega, I, \omega)$. The equations describing
the motion of these orbital elements are \citep{peters1964,storch,bin2}
\begin{align}
    \rd{a}{t} &= \p{\rd{a}{t}}_{\rm GW},\\
    \rd{e}{t} &= \frac{15}{8t_{\rm LK}} e\sqrt{1 - e^2}\sin 2\omega
        \sin^2 I + \p{\rd{e}{t}}_{\rm GW},\\
    \rd{\Omega}{t} &= \frac{3}{4t_{\rm LK}}
        \frac{\cos I\p{5e^2 \cos^2\omega - 4e^2 - 1}}{\sqrt{1 - e^2}}
        + \Omega_{\rm GR},\\
    \rd{I}{t} &= \frac{15}{16}\frac{e^2\sin 2\omega \sin 2I}{
        \sqrt{1 - e^2}},\\
    \rd{\omega}{t} &= \frac{3}{4t_{\rm LK}}
        \frac{2\p{1 - e^2} + 5\sin^2\omega
            (e^2 - \sin^2 I)}{\sqrt{1 - e^2}},
\end{align}
where we define
\begin{align}
    t_{\rm LK}^{-1} &= n\p{\frac{m_3}{m_{12}}}\p{\frac{a}{\tilde{a}_3}}^3,
        \label{eq:t_lk}\\
    \p{\rd{a}{t}}_{\rm GW} &= -\frac{a}{t_{\rm GW}},\\
         &= \frac{64}{5}\frac{G^3 \mu m_{12}^2}{c^5a^3}
            \frac{1}{\p{1 - e^2}^{7/2}}\p{1 + \frac{73}{24}e^2
                + \frac{37}{96}e^4},\label{eq:dadt_gw}\\
    \p{\rd{e}{t}}_{\rm GW} &= -\frac{304}{15}\frac{G^3 \mu m_{12}^2}{c^5a^4}
        \frac{1}{\p{1 - e^2}^{5/2}}\p{1 + \frac{121}{304}e^2}\label{eq:dedt_gw}
            ,\\
    \Omega_{\rm GR} &= \frac{3Gnm_{12}}{c^2a\p{1 - e^2}},
\end{align}
and $n = \sqrt{Gm_{12}/a^3}$ is the mean motion of the inner binary. We will
often refer to $e_{\min}$ and $e_{\max}$ the minimum/maximum eccentricity in a
single LK cycle, and we will sometimes notate $j = \sqrt{1 - e^2}$ and $j_{\min}
= \sqrt{1 - e_{\max}^2}$.

Finally, for concreteness, we adopt fiducial parameters mirroring those from
\citet{bin2}: the inner binary has $a = 100\;\mathrm{AU}$, $m_1 = 30M_{\odot}$,
$m_2 = 20M_{\odot}$, and initial $e_0 = 0.001$ with varying $I_0$, while the
tertiary companion has $\tilde{a}_3 = 4500\;\mathrm{AU}$, $m_3 = 30M_{\odot}$.

\subsection{Spin Dynamics: An Approximate Adiabatic Invariant}

We are ultimately interested in the spin orientations of the inner BHs at merger
as a function of initial conditions. Since they evolve independently to leading
post-Newtonian order, we focus on the dynamics of a single BH spin vector
$\bm{S} = S\uv{S}$. Neglecting spin-spin interactions, $\uv{S}$ undergoes de
Sitter precession about $\bm{L}$ as
\begin{align}
    \rd{\hat{\bm{S}}}{t} &= \Omega_{\rm SL}\hat{\bm{L}} \times \hat{\bm{S}}
            \label{eq:dsdt},\\
        \Omega_{\rm SL} &= \frac{3Gn\p{m_2 + \mu/3}}{2c^2a\p{1 - e^2}}.
\end{align}

To analyze the dynamics of the spin vector, we go to co-rotating frame with
$\uv{L}$ about $\uv{L}_{\rm out}$. Choose $\uv{L}_{\rm out} = \uv{z}$, and
choose the $\uv{x}$ axis such that $\uv{L}$ lies in the $x$-$z$ plane. In this
coordinate system, Eq.~\eqref{eq:dsdt} becomes
\begin{align}
    \p{\rd{\bm{S}}{t}}_{\rm rot}
        &= \p{-\rd{\Omega}{t}\uv{z} + \Omega_{\rm SL}\uv{L}} \times \uv{S},\\
        &= \bm{\Omega}_{\rm e} \times \uv{S}\label{eq:dsdt_weff},\\
    \bm{\Omega}_{\rm e} &\equiv \Omega_{\rm L}\uv{z} + \Omega_{\rm SL}
            \p{\cos I \uv{z} + \sin I \uv{x}},\label{eq:weff_def}\\
    \Omega_{\rm L} &\equiv -\rd{\Omega}{t}.\label{eq:Wldef}
\end{align}

In general, Eq.~\eqref{eq:dsdt_weff} is difficult to analyze, since $\Omega_{\rm
L}$, $\Omega_{\rm SL}$ and $I$ all vary significantly within each LK period,
and we are interested in the final outcome after many LK periods. However, if we
assume $t_{\rm GW} \gg t_{\rm LK}$, then the system can be treated as nearly
periodic within each LK cycle. We can then rewrite Eq.~\eqref{eq:dsdt_weff} in
Fourier components
\begin{equation}
    \p{\rd{\uv{S}}{t}}_{\rm rot}
        = \s{\bm{\overline{\Omega}}_{\rm e} + \sum\limits_{N = 1}^\infty
            \bm{\Omega}_{\rm eN}\cos \p{\frac{2\pi N t}{P_{\rm LK}}}}
            \times \uv{S}.\label{eq:dsdt_fullft}
\end{equation}
The bar denotes an average over an LK cycle. We adopt convention where $t = 0$
is the maximum eccentricity phase of the LK cycle.

We next assume that the $N \geq 1$ harmonics vanish exactly when the equation of
motion is averaged over an LK cycle. This gives
\begin{equation}
    \p{\rd{\uv{\overline{S}}}{t}}_{\rm rot}
        = \bm{\overline{\Omega}}_{\rm e}
            \times \uv{\overline{S}}.\label{eq:dsdt_0only}
\end{equation}
Eq.~\eqref{eq:dsdt_0only} suggests that $\theta_{\rm e}$, given by
\begin{equation}
    \cos \theta_{\rm e} \equiv
        \uv{\overline{S}} \cdot \uv{\overline{\Omega}}_{\rm e},
        \label{eq:q_eff}
\end{equation}
is an adiabatic invariant. The adiabaticity condition requires the precession
axis evolve slowly compared to the precession frequency at all times:
\begin{equation}
    \abs{\rd{\uv{\overline{\Omega}}_{\rm e}}{t}} \ll
        \abs{\bm{\overline{\Omega}}_{\rm e}}. \label{eq:ad_constr}
\end{equation}
Since the orientation of $\bm{\overline{\Omega}}_{\rm e}$ changes on timescale
$t_{\rm GW}$, we see that the adiabatic assumption is roughly equivalent to
assuming each LK period can be Fourier decomposed [Eq.~\eqref{eq:dsdt_fullft}].

\begin{figure*}
    \centering
    \includegraphics[width=0.9\textwidth]{LK90_plots/4sim_90_350.png}
    \caption{Orbital and spin evolution in a system for which the total change
    in the adiabatic invariant $\theta_{\rm e}$ is $\lesssim 0.01^\circ$. The
    inner binary is taken to have $a = 100\;\mathrm{AU}$, $m_1 = 30M_{\odot}$,
    $m_2 = 20M_{\odot}$, $I_0 = 90.35^\circ$, and $e_0 = 0.001$, while the
    tertiary companion has $\tilde{a}_3 = 4500\;\mathrm{AU}$, $m_3 =
    30M_{\odot}$. The top three panels $a$; $e$; and the inclination of the
    inner binary, both instantaneous ($I$) and appropriately averaged following
    Eq.~\eqref{eq:barI} ($\bar{I}$). The bottom three panels show the
    instantaneous spin-orbit misalignment angle $\theta_{\rm sl}$; the angle
    between $\overline{\bm{\Omega}}_{\rm e}$ [Eq.~\eqref{eq:weff_def}] and both
    the instantaneous spin vector (light grey) and the LK-averaged spin vector
    [red dots, denoted $\theta_{\rm e}$, Eq.~\eqref{eq:q_eff}]; and four
    characteristic frequencies of the system [Eqs.~\ref{eq:weff_def}
    and~\eqref{eq:Wldef}]. The unit of time is the LK timescale
    [Eq.~\eqref{eq:t_lk}] evaluated for the initial conditions $t_{\rm LK,
    0}$.}\label{fig:4sim_90_350}
\end{figure*}

\begin{figure}
    \centering
    \includegraphics[width=0.5\columnwidth]{LK90_plots/7_3vec_cropped.png}
    \caption{Definition of angles, shown in plane of the two angular momenta
    $\bm{L}_{\rm out}$ and $\bm{L}$, or the $\uv{x}$--$\uv{z}$ plane in the
    corotating frame. Note that for $I > 90^\circ$, $I_{\rm e} <
    0$.}\label{fig:3vec}
\end{figure}
To be more precise, we define the inclination angle $I_{\rm e}$ as the
angle between $\bm{\overline{\Omega}}_{\rm e}$ and $\bm{L}_{\rm out}$ as shown in
Fig.~\ref{fig:3vec}. Denoting also $\overline{\Omega}_{\rm e} \equiv
\abs{\bm{\overline{\Omega}}_{\rm e}}$, the adiabaticity condition can be
expressed as
\begin{equation}
    \rd{I_{\rm e}}{t} \ll \overline{\Omega}_{\rm e}.\label{eq:ad_constr_idot}
\end{equation}

Next, we express $I_{\rm e}$ in closed form. When the eccentricity is strongly
oscillatory within each LK cycle (early in the evolution, see
Fig.~\ref{fig:4sim_90_350}), we define averaged quantities
\begin{align}
    \overline{\Omega_{\rm SL} \sin I} &\equiv
            \overline{\Omega}_{\rm SL} \sin \bar{I},\\
    \overline{\Omega_{\rm SL} \cos I} &\equiv
            \overline{\Omega}_{\rm SL} \cos \bar{I}.\label{eq:barI}
\end{align}
Then, using Eq.\eqref{eq:weff_def}, we can see that
\begin{equation}
    \tan I_{\rm e} = \frac{\bar{\mathcal{A}}\sin \bar{I}}{
        1 + \bar{\mathcal{A}}\cos \bar{I}},\label{eq:ie_def}
\end{equation}
where
\begin{equation}
    \bar{\mathcal{A}} \equiv \frac{\overline{\Omega}_{\rm SL}}{
        \overline{\Omega}_{\rm L}}.
\end{equation}
When eccentricity oscillations are suppressed at later times, we just have
$\overline{\Omega}_{\rm SL} = \Omega_{\rm SL}$, $\overline{\Omega}_{\rm L} =
\Omega_{\rm L}$, and $\bar{I} = I$.

\section{Analysis: Deviation from Adiabaticity}\label{s:fast_merger}

In real systems, the particular extent to which $\theta_{\rm e}$ is conserved
depends on how well Eq.~\eqref{eq:ad_constr} is satisfied. We will first present
equations of motion for $\theta_{\rm e}$. We will then derive accurate estimates
for important quantities in these equations of motion, and use these estimates
to derive upper bounds on $\Delta \theta_{\rm e}$, the change in $\theta_{\rm
e}$ over the entire inspiral. Taken together, this calculation estimates the
deviation from adiabaticity as a function of initial conditions.

\subsection{Equation of Motion}\label{ss:eom_0}

From the corotating frame [Eq.~\eqref{eq:dsdt_0only}], consider going to the
reference frame where $\uv{z}' = \uv{\overline{\Omega}}_{\rm e} $ by rotation
$-\dot{I}_{\rm e}\uv{y}$. In this reference frame, the polar coordinate is just
$\theta_{\rm e}$ as defined above in Eq.~\eqref{eq:q_eff}, and the equation of
motion becomes
\begin{align}
    \p{\rd{\uv{S}}{t}}' &= \overline{\Omega}_{\rm e}\uv{z}' \times \uv{S}'
        - \dot{I}_{\rm e} \uv{y}' \times \uv{S}'\label{eq:eom_prime}.
\end{align}
If we break $\uv{S}'$ into components $\uv{S}' = S_x'\uv{x}' + S_y' \uv{y}' +
\cos \theta_{\rm e} \uv{z}'$ and define complex variable
\begin{equation}
    S_\perp \equiv S_x' + iS_y',
\end{equation}
we can rewrite Eq.~\ref{eq:eom_prime} as
\begin{equation}
    \rd{S_\perp}{t} = i\p{\overline{\Omega}_{\rm e}} S_\perp
        - \dot{I}_{\rm e} \cos \theta_{\rm e}.
\end{equation}
This can be solved in closed form using an integrating factor. Defining
\begin{equation}
    \Phi(t) \equiv \int\limits^t \overline{\Omega}_{\rm e}\;\mathrm{d}t,
\end{equation}
we obtain solution until final time $t_{\rm f}$
\begin{equation}
    e^{-i\Phi}S_{\perp}\bigg|_{-\infty}^{t_{\rm f}}
        = -\int\limits_{-\infty}^{t_{\rm f}}
            e^{-i\Phi(\tau)}\dot{I}_{\rm e} \cos \theta\;\mathrm{d}\tau.
            \label{eq:formal_sol_0}
\end{equation}
It can be seen that, in the adiabatic limit [Eq.~\eqref{eq:ad_constr_idot}],
$\abs{S_\perp} = \sin \theta_{\rm e}$ is conserved over timescales
$\gg 1 / \overline{\Omega}_{\rm e}$, as the phase of the integrand in the right
hand side varies much faster than the magnitude.

\begin{figure*}
    \centering
    \includegraphics[width=0.9\textwidth]{LK90_plots/4sim_90_350_zoom.png}
    \caption{The same simulation as Fig.~\ref{fig:4sim_90_350} but shown
    focusing on the region where $\bar{\mathcal{A}} \simeq 1$. The top three
    panels depict $a$, $e$, $I$ and $\bar{I}$ as before, but in addition $I_{\rm
    e}$ [Eq.~\eqref{eq:ie_def}] and $I_1$ [Eq.~\eqref{eq:eom_harmonic}]
    are shown in the third panel. The bottom three panels depict the frequency
    ratios between the zeroth and first Fourier components of $\bm{\Omega}_{\rm
    e}$ to the LK frequency $\Omega = 2\pi / P_{\rm LK}$; the magnitude of
    oscillation of $\theta_{\rm e}$ away from its initial value (red dots) as
    well as amplitude estimates due to non-adiabatic effects [green,
    Eq.~\eqref{eq:nonad_dqeff}] and due to resonances with harmonic terms [blue,
    Eq.~\eqref{eq:harmonic_dqeff}]; and the same characteristic frequencies as
    before. In the bottom middle panel, it is clear that oscillations in
    $\theta_{\rm e}$ are dominantly driven by interactions with the $N = 1$
    harmonic.}\label{fig:4sim_90_350_zoom}
\end{figure*}

\begin{figure*}
    \centering
    \includegraphics[width=0.9\textwidth]{LK90_plots/4sim_90_200_zoom.png}
    \caption{Same as Fig.~\ref{fig:4sim_90_350_zoom} except for $I_0 =
    90.2^\circ$, corresponding to a faster merger and a total change in
    $\theta_{\rm e}$ of $\approx 2^\circ$. In the bottom middle panel, the
    nonadiabatic contribution is more significant and causes much poorer
    conservation of $\theta_{\rm e}$.}\label{fig:4sim_90_200_zoom}
\end{figure*}

Recalling $\abs{S_{\perp}} = \sin \theta_{\rm e}$ and analyzing
Eq.~\eqref{eq:formal_sol_0}, we see that $\sin \theta_{\rm e} \approx
\theta_{\rm e}$ oscillates about its value at $t = -\infty$ with amplitude
\begin{equation}
    \abs{\Delta \theta_{\rm e}} \sim
        \frac{\dot{I}_{\rm e}}{\overline{\Omega}_{\rm e}}.\label{eq:nonad_dqeff}
\end{equation}
This is compared to the $\Delta \theta_{\rm e}$ from simulations in the bottom
center panels of Figs.~\ref{fig:4sim_90_350_zoom} and~\ref{fig:4sim_90_200_zoom}
as the green line. We see that in the latter simulation, the faster merger, the
order of magnitude of $\abs{\Delta \theta_{\rm e}}$ is somewhat well predicted,
while in the slower merger a second contribution dominates $\Delta \theta_{\rm
e}$ oscillations, discussed in Section~\ref{s:harmonic}.

Furthermore, if we denote $\Delta \theta_{\rm e}^{(f)}$ to be the total change
in $\theta_{\rm e}$ from $t = -\infty$ to merger, we can give loose
bound\footnote{Given the complicated evolution of $\overline{\Omega}_{\rm e}$
and $\dot{I}_{\rm e}$, it is difficult to give a more exact bound on the
deviation from adiabaticity. In practice, deviations $\lesssim 1^\circ$ are
observationally indistinguishable, so the exact scaling in this regime is
negligible.}
\begin{equation}
    \abs{\Delta \theta_{\rm e}^{(f)}} \lesssim
        \abs{\frac{\dot{I}_{\rm e}}{\overline{\Omega}_{\rm e}}}_{\max}.
        \label{eq:nonad_dqeff_tot}
\end{equation}
In the following section, we show that this value can be calculated to good
accuracy from initial conditions.

\subsection{Estimate of Deviation from Adiabaticity}

Towards estimating $\s{\dot{I}_{\rm e} / \overline{\Omega}_{\rm e}}_{\max}$, we
first differentiate Eq.~\eqref{eq:ie_def},
\begin{equation}
    \dot{I}_{\rm e} = \p{\frac{\dot{\bar{\mathcal{A}}}}{
            \bar{\mathcal{A}}}}
        \frac{\bar{\mathcal{A}} \sin \bar{I}}{
            1 + 2\bar{\mathcal{A}}\cos \bar{I}
                + \bar{\mathcal{A}}^2}.
\end{equation}
It also follows from Eq.~\eqref{eq:weff_def} that
\begin{equation}
    \overline{\Omega}_{\rm e} = \overline{\Omega}_{\rm L}
        \p{1 + 2\bar{\mathcal{A}}\cos \bar{I}
            + \overline{\mathcal{A}}^2}^{1/2},
\end{equation}
from which we obtain
\begin{equation}
    \abs{\frac{\dot{\bar{I}}_{\rm e}}{\overline{\Omega}_{\rm e}}}
        = \abs{\frac{\dot{\bar{\mathcal{A}}}}{
            \bar{\mathcal{A}}}}
        \frac{1}{\abs{\overline{\Omega}_{\rm L}}}
        \frac{\bar{\mathcal{A}} \sin \bar{I}}{
            \p{1 + 2\bar{\mathcal{A}}\cos \bar{I}
                + \bar{\mathcal{A}}}^{3/2}}.
\end{equation}
This is maximized when $\bar{\mathcal{A}} \simeq 1$, and so we obtain that
the maximum deviation should be bounded by
\begin{equation}
    \abs{\frac{\dot{I}_{\rm e}}{\overline{\Omega}_{\rm e}}}_{\max}
        \simeq \abs{\frac{\dot{\bar{\mathcal{A}}}}{\bar{\mathcal{A}}}}
            \frac{1}{\abs{\overline{\Omega}_{\rm L}}}
            \frac{\sin \bar{I}}{\p{2 + 2\cos \bar{I}}^{3/2}}.
            \label{eq:idot_over_W}
\end{equation}

To evaluate this, we make two assumptions: (i) $\bar{I}$ is approximately
constant, and (ii) $j_{\min} = \sqrt{1 - e_{\max}^2}$ evaluated at
$\bar{\mathcal{A}} \simeq 1$ is some constant multiple of the initial
$j_{\min}$, so that
\begin{equation}
    j_{\star} \equiv (j)_{\bar{\mathcal{A}} \simeq 1} = f
        \sqrt{\frac{5}{3}\cos^2 I_0},\label{eq:jstar_ansatz}
\end{equation}
for some unknown factor $f > 1$; we use star subscripts to denote evaluation at
$\bar{\mathcal{A}} \simeq 1$. $f$ turns out to be relatively insensitive to
$I_0$. This can be as systems with lower $e_{\max}$ values take more
cycles to attain $\bar{\mathcal{A}} \simeq 1$ and thus experience a similar
amount of decay due to GW radiation.

For simplicity, let's first assume $\bar{\mathcal{A}} \simeq 1$ is
satisfied when the LK oscillations are mostly suppressed, and $e_\star \approx
1$ throughout the LK cycle (we will later see that the scalings are the same in
the LK-oscillating regime). Then we can write
\begin{align}
    \bar{\mathcal{A}} &\simeq \frac{3Gn\p{m_2 + \mu/3}}{
        2c^2a j^2}
            \s{\frac{3\cos \bar{I}}{
                4t_{\rm LK}} \frac{1 + 3e^2/2}{j}}^{-1},\\
        &\simeq \frac{G(m_2 + \mu/3) m_{12}\tilde{a}_3^3}{
            c^2m_3a^4 j \cos \bar{I}},\label{eq:abar_eq1}\\
        &\propto \frac{1}{a^4j},\\
    \frac{\dot{\bar{\mathcal{A}}}}{\bar{\mathcal{A}}}
        &= -4\p{\frac{\dot{a}}{a}}_{\rm GW}
            + \frac{e}{j^2}\p{\rd{e}{t}}_{\rm GW}.
\end{align}
Approximating $e_\star \approx 1$ in Eqs.~\eqref{eq:dadt_gw} and~\eqref{eq:dedt_gw}
gives
\begin{align}
    \s{\frac{\dot{\bar{\mathcal{A}}}}{\bar{\mathcal{A}}}}_{
        \bar{\mathcal{A}} = 1}
        &\simeq \frac{64G^3 \mu m_{12}^2}{5c^5a_\star^4j_\star^7} \times 15,\\
    \overline{\Omega}_{\rm L, \star}
        &\approx \frac{3\cos \bar{I}}{2t_{\rm LK}j_\star},\\
    \abs{\frac{\dot{\bar{I}}_{\rm e}}{\overline{\Omega}_{\rm e}}}_{\max}
        &\approx \frac{128G^3 \mu m_{12}^2}{c^5 a_\star^4j_\star^6}
            \frac{t_{\rm LK}}{\cos \bar{I}}
            \frac{\sin \bar{I}}{\p{2 + 2\cos \bar{I}}^{3/2}}.
\end{align}
With the ansatz for $j_\star$ given by Eq.~\eqref{eq:jstar_ansatz} and requiring
Eq.~\eqref{eq:abar_eq1} equal $1$ for a given $j_\star$ and $a_\star$ gives us
the final expression
\begin{align}
    \abs{\frac{\dot{\bar{I}}_{\rm e}}{\overline{\Omega}_{\rm e}}}_{\max}
        \approx{}& \frac{128 G^3 \mu m_{12}^3 \tilde{a}_3^3}{c^5
        \sqrt{Gm_{12}} m_3}
            \p{\frac{c^2 m_3 \cos \bar{I}}{G(m_2 + \mu / 3) m_{12}
                \tilde{a}_3^3}}^{11/8}\nonumber\\
        &\times \p{j_\star}^{-37/8}
            \frac{\tan \bar{I}}{
            \p{2 + 2 \cos \bar{I}}^{3/2}}.\label{eq:prediction}
\end{align}
% note this is equal to YS's I/\Omega_e to some small factor, a few percent?
The agreement of Eq.~\eqref{eq:prediction} with numerical simulation is
remarkable, as shown in Fig.~\ref{fig:good_quants}.
\begin{figure}
    \centering
    \includegraphics[width=\columnwidth]{LK90_plots/good_quants.png}
    \caption{Comparison of $\abs{\dot{\bar{I}}_{\rm e} /
    \overline{\Omega}_{\rm e}}_{\max}$ extracted from simulations and using
    Eq.~\eqref{eq:prediction}, where we take $f = 2.6$ in
    Eq.~\eqref{eq:jstar_ansatz}. The merger time $P_{\rm m}$ is shown along the
    top axis of the plot in units of the characteristic LK timescale at the
    start of inspiral $t_{\rm LK, 0}$; the LK period is initially of order a few
    $t_{\rm LK, 0}$. The agreement is remarkable for mergers that are more
    adiabatic (towards the right).}\label{fig:good_quants}
\end{figure}

Above, we assumed that $\bar{\mathcal{A}} \simeq 1$ is satisfied when the
eccentricity is mostly constant (see Fig.~\ref{fig:4sim_90_350} for an
indication of how accurate this is for the parameter space explored in
Fig.~\ref{fig:good_quants}). It is also possible that $\bar{\mathcal{A}} \simeq
1$ occurs when the eccentricity is still undergoing substantial oscillations. In
fact, Eq.~\eqref{eq:prediction} is still accurate in this regime when replacing
$e$ with $e_{\max}$, due to the following analysis. When $e_{\min} \ll
e_{\max}$, the binary spends a fraction $\sim j_{\min}$ of the LK cycle near $e
\simeq e_{\max}$ \citep{anderson2016formation}. This fraction of the LK cycle
dominates both GW dissipation and $\overline{\Omega}_{\rm L}$ precession. Thus,
both $\dot{\bar{\mathcal{A}}}$ and $\overline{\Omega}_{\rm L}$ in
Eq.~\eqref{eq:idot_over_W} are evaluated at $e \approx e_{\max}$ and are
suppressed by a factor of $j_{\min}$. However, $\dot{\bar{\mathcal{A}}}$ and
$\overline{\Omega}_{\rm L}$ appear in the numerator and denominator of
Eq.~\eqref{eq:prediction} respectively, and so the $j_{\min}$ factors cancel. In
conclusion, when the eccentricity is still substantially oscillating,
Eq.~\eqref{eq:prediction} remains accurate when $e$ is replaced with $e_{\max}$.

The accuracy of Eq.~\eqref{eq:prediction} in bounding the total change in
$\Delta \theta_{\rm e}^{(f)}$ over inspiral is shown in Fig.~\ref{fig:deviations}.
Note that conservation of $\theta_{\rm e}$ is generally much better than
Eq.~\eqref{eq:prediction} predicts; cancellation of phases in
Eq.~\eqref{eq:formal_sol_0} is generally more
efficient than Eq.~\eqref{eq:prediction} assumes.
\begin{figure}
    \centering
    \includegraphics[width=\columnwidth]{LK90_plots/deviations_one.png}
    \caption{Change in $\theta_{\rm e}$ over inspiral as a function of initial
    inclination $I_0$. Plotted for comparison is the bound $\abs{\Delta
    \theta_{\rm e}^{(f)}} \lesssim \abs{\dot{I}_{\rm e} / \overline{\Omega}_{\rm
    e}}_{\max}$, given by Eq.~\eqref{eq:prediction}. It is clear that the given
    bound is not tight but provides an upper bound for non-conservation of
    $\theta_{\rm e}$ due to nonadiabatic effects. The leftmost portion of this
    plot is less reliable as the quasi-periodic assumption within each LK cycle
    breaks down as GW dissipation within each LK cycle is
    substantial.}\label{fig:deviations}
\end{figure}

\section{Analysis: Resonances and Breakdown of $\theta_{\rm e}$ Conservation
}\label{s:harmonic}

In the previous section, we assumed the $N \geq 1$ Fourier harmonics in
Eq.~\eqref{eq:dsdt_fullft} are negligible when averaging over an LK period.
However, this assumption breaks down near resonances. For simplicity, we ignore
the effects of GW dissipation in this section and assume the system is exactly
periodic. For each $\bm{\Omega}_{\rm eN}$ Fourier harmonic [defined in
Eq.~\eqref{eq:dsdt_fullft}], denote its magnitude $\Omega_{\rm eN}$ and its
inclination angle relative to $\bm{L}_{\rm out}$ as $I_N$, using the same
convention as Fig.~\ref{fig:3vec}. We then rewrite Eq.~\eqref{eq:dsdt_fullft} in
component form in the reference frame where $\uv{z}' \propto
\bm{\overline{\Omega}}_{\rm e}$:
\begin{align}
    \p{\rd{\uv{S}}{t}}'
        &= \overline{\Omega}_{\rm e}\uv{z}' \times \uv{S}'\nonumber\\
        &+ \s{\sum\limits_{N = 1}^\infty
            \Omega_{\rm e N}\cos\p{N\Omega t}
                \p{\cos (\Delta I_N) \uv{z}' + \sin (\Delta I_N) \uv{x}'}}
        \times \uv{S}'.\label{eq:eom_harmonic}
\end{align}
Here, $\Delta I_N \equiv I_{\rm e} - I_N$ and $\Omega = 2\pi / P_{\rm LK}$ is
the LK angular frequency. Following a similar procedure to
Section~\ref{ss:eom_0}, we obtain an equation analogous to
Eq.~\eqref{eq:formal_sol_0}:
\begin{align}
    \rd{S_{\perp}}{t} ={}& i\p{\overline{\Omega}_{\rm e} + \Omega_{\rm eN}
        \cos \p{\Delta I_N}\cos N \Omega t} S_\perp\nonumber\\
        &- i\cos \theta \sin \p{\Delta I_N} \Omega_{\rm eN} \cos N\Omega
        t.\label{eq:formal_sol_gen}
\end{align}
In this equation, there are potentially resonances when $\overline{\Omega}_{\rm
e} = N\Omega$. Since, $\overline{\Omega}_{\rm e} \lesssim \Omega$ for most
regions of parameter space (see Fig.~\ref{fig:dWs}), we restrict our analysis to
resonances with the $N = 1$ component. The two possible resonant behaviors are a
parametric resonance [modulation of the oscillation frequency in
Eq.~\eqref{eq:formal_sol_gen}] and resonant forcing by the second term.
Parametric resonances are typically very narrow and are therefore hard to excite
as the system's frequencies change under GW dissipation. As such, we consider
only the effect of the second term in Eq.~\eqref{eq:formal_sol_gen}.

Restricting our attention to $N = 1$ and neglecting the parametric term, the
equation of motion reduces to
\begin{align}
    \rd{S_{\perp}}{t} &\approx i\overline{\Omega}_{\rm e}S_\perp
        - i\cos \theta_{\rm e} \sin \p{\Delta I_N} \Omega_{\rm eN}
            \cos \p{N\Omega t}.
\end{align}
We can approximate $\cos \p{N\Omega t} \approx e^{iN\Omega t} / 2$, as the
$e^{-iN\Omega t}$ component is far from resonance. Then we can write down
solution
\begin{align}
    \Phi(t) &= \int\limits^t \overline{\Omega}_{\rm e}\;\mathrm{d}\tau,\\
    e^{-i\Phi}S_{\perp}\bigg|_{-\infty}^\infty
        &= -\int\limits_{-\infty}^\infty
            \frac{i\sin\p{\Delta I_1} \Omega_{\rm e1}}{2}
                e^{-i\Phi(\tau) + i\Omega \tau} \cos \theta_{\rm e}
            \;\mathrm{d}\tau.\label{eq:harmonic_dS}
\end{align}
Thus, similarly to Section~\ref{ss:eom_0}, $\abs{\Delta \theta_{\rm e}}$ can be
bound by
\begin{equation}
    \abs{\Delta \theta_{\rm e}} \sim \frac{1}{2}\frac{\sin \p{\Delta I_1}
        \Omega_{\rm e1}}{ \Omega - \overline{\Omega}_{\rm e}}.
        \label{eq:harmonic_dqeff}
\end{equation}
This is shown as the blue line in the bottom center panels of
Fig.~\ref{fig:4sim_90_350_zoom} and~\ref{fig:4sim_90_200_zoom}. We see that the
amplitude of oscillations in $\theta_{\rm e}$ are well described by
Eq.~\eqref{eq:harmonic_dqeff}, particularly in the former case (where the
non-adiabatic contribution is weaker).

Again analogously to Section~\ref{ss:eom_0}, we obtain loose bound for total
nonconservation of $\theta_{\rm e}$
\begin{equation}
    \abs{\Delta \theta_{\rm e}^{(f)}} \lesssim \s{\frac{\sin \p{\Delta I_1}
        \Omega_{\rm e1}}{ \Omega - \overline{\Omega}_{\rm e}}}_{\max}.
        \label{eq:harmonic_dqeff_tot}
\end{equation}

While Eq.~\eqref{eq:harmonic_dqeff} depends on the properties of the $N = 1$
Fourier component, the condition for substantial $\theta_{\rm e}$
non-conservation can be understood in terms of physical quantities:
\begin{itemize}
    \item $\sin \p{\Delta I_1}$ is small unless $\bar{\mathcal{A}} \simeq
        1$. Otherwise, $\bm{\Omega}_{\rm e}$ does not nutate appreciably within
        an LK cycle, and all the $\bm{\Omega}_{\rm eN}$ are aligned with
        $\overline{\bm{\Omega}}_{\rm e}$, implying all the $\Delta I_N \approx
        0$.

    \item Smaller values of both $e_{\min}$ and $e_{\max}$ increase
        $\overline{\Omega}_{\rm e} / \Omega$, as shown in Fig.~\ref{fig:dWs},
        strengthening the interaction with the $N = 1$ resonance.
\end{itemize}
\begin{figure}
    \centering
    \includegraphics[width=\columnwidth]{LK90_plots/5_dWs.png}
    \caption{For the fiducial set of parameters, $\overline{\Omega}_{\rm e} /
    \Omega$ as a function of $I_0$ (corresponding to $e_{\max}$ labeled on top
    axis) for varying values of $e_{\min}$. Both smaller $e_{\min}$ and
    $e_{\max}$ values more easily satisfy the resonant condition
    $\overline{\Omega}_{\rm e} / \Omega \approx 1$. Data are only shown when
    $e_{\min} < e_{\max}$.}\label{fig:dWs}
\end{figure}

LK-driven coalescence causes $\bar{\mathcal{A}}$ to increase on a similar
timescale to that of $e_{\min}$ increase (see Fig.~\ref{fig:4sim_90_350}). As
such, we conclude that the effect of harmonic terms generally only affects
$\theta_{\rm e}$ conservation when $\bar{\mathcal{A}} \approx 1$ initially.

\section{Conclusion and Discussion}\label{s:discussion}

Relation between $\theta_{\rm e}$ and $\theta_{\rm sl}^{\rm (f)}$, as a function
of $I$.

The ``chaotic'' behavior in Paper I is because it satisfies the heuristic
provided at the end of the harmonics section well.

Interestingly, harmonic terms begin to dominate $\Delta \theta_{\rm e}$ at quite
small inclinations $I_0 \lesssim 90.35^\circ$, but the non-adiabatic
contribution to nonconservation obviously dominates out to $I_0 \approx
90.4^\circ$.

\bibliography{Su_LK90}
\bibliographystyle{aasjournal}

\end{document}
