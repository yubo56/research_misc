    \documentclass[
        fleqn,
        usenatbib,
        % referee,
    ]{mnras}
    \usepackage{
        amsmath,
        amssymb,
        newtxtext,
        newtxmath,
        graphicx,
        ae, aecompl,
        booktabs,
        caption,
        subcaption,
    }
    \usepackage[T1]{fontenc}
    \captionsetup{compatibility=false}

    \newcommand*{\rd}[2]{\frac{\mathrm{d}#1}{\mathrm{d}#2}}
    \newcommand*{\rtd}[2]{\frac{\mathrm{d}^2#1}{\mathrm{d}#2^2}}
    \newcommand*{\pd}[2]{\frac{\partial#1}{\partial#2}}
    \newcommand*{\md}[2]{\frac{\mathrm{D}#1}{\mathrm{D}#2}}
    \newcommand*{\at}[1]{\left.#1\right|}
    \newcommand*{\abs}[1]{\left|#1\right|}
    \newcommand*{\ev}[1]{\langle#1\rangle}
    \newcommand*{\bm}[1]{\boldsymbol{\mathbf{#1}}}
    \newcommand*{\uv}[1]{\hat{\bm{#1}}}
    \newcommand*{\p}[1]{\left(#1\right)}
    \newcommand*{\s}[1]{\left[#1\right]}
    \newcommand*{\z}[1]{\left\{#1\right\}}
    \DeclareMathOperator*{\argmin}{argmin}
    \DeclareMathOperator*{\argmax}{argmax}
    \DeclareMathOperator*{\med}{med}
    \DeclareMathOperator*{\sgn}{sgn}

\title[$90^\circ$ LK]{$90^\circ$ Attractor}
\author[Authors]{
Millie Vick$^1$
\\
$^1$ Cornell Center for Astrophysics and Planetary Science, Department of
Astronomy, Cornell University, Ithaca, NY 14853, USA
}

\date{Accepted XXX\@. Received YYY\@; in original form ZZZ}

\pubyear{2020}

\begin{document}\label{firstpage}
\pagerange{\pageref{firstpage}--\pageref{lastpage}}
\maketitle

\begin{abstract}
    Abstract
\end{abstract}

\begin{keywords}
keywords % chktex 8
\end{keywords}

\section{Introduction}

This problem is important as it will be important for self-driving cars, curing
cancer, and the search for extraterrestrial intelligence.

\section{Analytical Results}

\subsection{Equations of Motion}

We study Lidov-Kozai oscillations due to an external perturber to quadrupole
order and include precession of pericenter and gravitational wave radiation due
to general relativity. Consider an inner BH binary with masses $m_1$ and $m_2$
having total mass $m_{12}$ and reduced mass $\mu$ orbited by a third BH with
mass $m_3$. Call $a_3$ the orbital semimajor axis of the third BH from the
center of mass of the inner binary, and $e_3$ the eccentricity of its orbit, and
define effective semimajor axis
\begin{equation}
    \bar{a}_3 \equiv a_3\sqrt{1 - e_3^2}.
\end{equation}
We adopt the test particle approximation such that the orbit of the third mass
is fixed in space.

We then consider the motion of the inner binary, described by orbital elements
$(a, e, \Omega, I, \omega)$. The equations describing the motion of these
orbital elements is then \citep{storch,bin2,peters1964}
\begin{align}
    \rd{a}{t} &= -\frac{a}{t_{GW}},\\
    \rd{e}{t} &= \frac{15}{8t_{LK}} e\sqrt{1 - e^2}\sin 2\omega
        \sin^2 I,\\
    \rd{\Omega}{t} &= \frac{3}{4t_{LK}}
        \frac{\cos I\p{5e^2 \cos^2\omega - 4e^2 - 1}}{\sqrt{1 - e^2}}
        + \Omega_{GR},\\
    \rd{I}{t} &= \frac{15}{16}\frac{e^2\sin 2\omega \sin 2I}{
        \sqrt{1 - e^2}},\\
    \rd{\omega}{t} &= \frac{3}{4t_{LK}}
        \frac{2\p{1 - e^2} + 5\sin^2\omega
            (e^2 - \sin^2 I)}{\sqrt{1 - e^2}}.
\end{align}
Here, we have defined
\begin{align}
    t_{LK}^{-1} &= n\p{\frac{m_3}{m_{12}}}\p{\frac{a}{\bar{a}_3}}^3,\\
    t_{GW}^{-1} &= \frac{64}{5}\frac{G^3 \mu m_{12}^2}{c^5a^4}
        \frac{1}{\p{1 - e^2}^{7/2}}\p{1 + \frac{73}{24}e^2
            + \frac{37}{96}e^4},\\
    \Omega_{GR} &= \frac{3Gnm_{12}}{c^2a\p{1 - e^2}},
\end{align}
and $n = \sqrt{Gm_{12}/a^3}$ is the mean motion of the inner binary.

We are ultimately interested in the evolution of the spin vector of the inner
BHs. Neglecting spin-spin interactions, they evolve independently as
\begin{align}
    \rd{\hat{\bm{S}}}{t} &= \Omega_{SL}\hat{\bm{L}} \times \hat{\bm{S}},\\
    \Omega_{SL} &= \frac{3Gn\p{m_2 + \mu/3}}{2c^2a\p{1 - e^2}}.
\end{align}

\appendix

\section{Equations of Motion}

We follow \citet{kinoshita,storch} and write down Hamiltonian in the frame where

\section{Difference from Chaotic Stellar Spin Dynamics}

\begin{itemize}
    \item LK is not a perturbation for us (compared to $\uv{S} \cdot \uv{L}$
        dynamics), it is significantly dominant. This corresponds to the
        $\mathcal{A} \ll 1$ regime of SL15. They obtain a neat bifurcation due
        to separatrix crossing, which is not observed in our LK simulations, so
        this cannot in spirit be a similar mechanism.

    \item SL15 focuses on adiabatically changing $\mathcal{A}$ and seeing how it
        encounters resonances. In our problem, nothing nontrivial can occur if
        $\mathcal{A}$ changes slowly.

    \item Our Hamiltonian takes on form $H = \p*{\bm{\Omega}_{SL} - \bm{R}}
        \cdot \uv{S}$. This will never have any resonances since it's perfectly
        linear; anything that looks nonlinear is a pure consequence of
        coordinates (e.g.\ multiplication of $\theta$ and $\phi$ terms).
\end{itemize}

\bibliographystyle{mnras}
\bibliography{Su_LK90}

% \clearpage
% \onecolumn

\bsp
\label{lastpage} % chktex 24
\end{document}
