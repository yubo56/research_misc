    \documentclass[dvipsnames]{beamer}
    \usetheme{Madrid}
    \usefonttheme{professionalfonts}
    \usepackage{
        amsmath,
        amssymb,
        fouriernc, % fourier font w/ new century book
        fancyhdr, % page styling
        lastpage, % footer fanciness
        hyperref, % various links
        setspace, % line spacing
        amsthm, % newtheorem and proof environment
        mathtools, % \Aboxed for boxing inside aligns, among others
        float, % Allow [H] figure env alignment
        enumerate, % Allow custom enumerate numbering
        graphicx, % allow includegraphics with more filetypes
        wasysym, % \smiley!
        upgreek, % \upmu for \mum macro
        listings, % writing TrueType fonts and including code prettily
        tikz, % drawing things
        booktabs, % \bottomrule instead of hline apparently
        cancel % can cancel things out!
    }
    \usepackage[
        labelfont=bf, % caption names are labeled in bold
        font=scriptsize % smaller font for captions
    ]{caption}
    \usepackage[font=scriptsize]{subcaption} % subfigures

    \newcommand*{\scinot}[2]{#1\times10^{#2}}
    \newcommand*{\dotp}[2]{\left<#1\,\middle|\,#2\right>}
    \newcommand*{\rd}[2]{\frac{\mathrm{d}#1}{\mathrm{d}#2}}
    \newcommand*{\pd}[2]{\frac{\partial#1}{\partial#2}}
    \newcommand*{\rtd}[2]{\frac{\mathrm{d}^2#1}{\mathrm{d}#2^2}}
    \newcommand*{\ptd}[2]{\frac{\partial^2 #1}{\partial#2^2}}
    \newcommand*{\md}[2]{\frac{\mathrm{D}#1}{\mathrm{D}#2}}
    \newcommand*{\pvec}[1]{\vec{#1}^{\,\prime}}
    \newcommand*{\svec}[1]{\vec{#1}\;\!}
    \newcommand*{\bm}[1]{\boldsymbol{\mathbf{#1}}}
    \newcommand*{\ang}[0]{\;\text{\AA}}
    \newcommand*{\mum}[0]{\;\upmu \mathrm{m}}
    \newcommand*{\at}[1]{\left.#1\right|}

    \let\Re\undefined
    \let\Im\undefined
    \DeclareMathOperator{\Res}{Res}
    \DeclareMathOperator{\Re}{Re}
    \DeclareMathOperator{\Im}{Im}
    \DeclareMathOperator{\Log}{Log}
    \DeclareMathOperator{\Arg}{Arg}
    \DeclareMathOperator{\Tr}{Tr}
    \DeclareMathOperator{\E}{E}
    \DeclareMathOperator{\Var}{Var}
    \DeclareMathOperator*{\argmin}{argmin}
    \DeclareMathOperator*{\argmax}{argmax}
    \DeclareMathOperator{\sgn}{sgn}
    \DeclareMathOperator{\diag}{diag\;}

    \DeclarePairedDelimiter\bra{\langle}{\rvert}
    \DeclarePairedDelimiter\ket{\lvert}{\rangle}
    \DeclarePairedDelimiter\abs{\lvert}{\rvert}
    \DeclarePairedDelimiter\ev{\langle}{\rangle}
    \DeclarePairedDelimiter\p{\lparen}{\rparen}
    \DeclarePairedDelimiter\s{\lbrack}{\rbrack}
    \DeclarePairedDelimiter\z{\lbrace}{\rbrace}

    % \everymath{\displaystyle} % biggify limits of inline sums and integrals
    \tikzstyle{circ} % usage: \node[circ, placement] (label) {text};
        = [draw, circle, fill=white, node distance=3cm, minimum height=2em]
    \definecolor{commentgreen}{rgb}{0,0.6,0}
    \lstset{
        basicstyle=\ttfamily\footnotesize,
        frame=single,
        numbers=left,
        showstringspaces=false,
        keywordstyle=\color{blue},
        stringstyle=\color{purple},
        commentstyle=\color{commentgreen},
        morecomment=[l][\color{magenta}]{\#}
    }

\begin{document}

\title{Dynamical Tides in Eccentric Massive Stellar Binaries}
\subtitle{Group Meeting}
\author{Yubo Su}
\date{Jan 23, 2020}

\maketitle

\begin{frame}
    \frametitle{Setup}
    \framesubtitle{Problem Description}

    \begin{itemize}
        \item Massive star with eccentric binary companion inducing dynamical
            tides.

        \item Primary difficulty: dynamical tides is typically messy, sum over
            many modes, hard to gain analytical intuition.

        \item Question: can we obtain a \emph{simple closed form for dynamical
            tides} in this system?
    \end{itemize}
\end{frame}

\begin{frame}
    \frametitle{Setup}
    \framesubtitle{Previous Work}
    {\small
    \begin{itemize}
        \item Dynamical tides in massive stars due to \emph{circular
            companion} (Kushnir et.\ al.\ 2017).
            \begin{equation}
                \tau(\omega; r_c) = \beta_2\frac{GM_2^2r_c^5}{a^6}
                    \p*{\frac{\omega}{\sqrt{GM_c/r_c^3}}}^{8/3}
                        \frac{\rho_c}{\bar{\rho}_c} \p*{1 -
                        \frac{\rho_c}{\bar{\rho}_c}}^2.
            \end{equation}

        \item Eccentric forcing is just sum of many circular forcings
            (Fourier transform, e.g.\ Vick et.\ al.\ 2017)
            \begin{equation}
                \tau = T_0 \sum\limits_{N = -\infty}^\infty
                    F_{N2}^2 \sgn\p*{N\Omega - 2\Omega_s} \tau
                        \p*{\omega = \abs*{N\Omega - 2\Omega_s}},
            \end{equation}
            where $F_{Nm}$ are the \emph{Hansen coefficients}
            \begin{equation}
                F_{Nm} = \frac{1}{\pi}\int\limits_{0}^{\pi}
                    \frac{\cos\s*{N\p*{E - e\sin E} - mf(E)}}
                        {\p*{1 - e\cos E}^2}\;\mathrm{d}E.
            \end{equation}
    \end{itemize}
    }
\end{frame}

\begin{frame}
    \frametitle{Solution}
    \framesubtitle{Key Insight}

    \begin{itemize}
        \item Key insight: $e \to 1$, the Hansen coefficients can be well
            approximated in closed form, and their sum approximated as an
            integral.
    \end{itemize}
\end{frame}

\end{document}

