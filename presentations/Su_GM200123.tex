    \documentclass[dvipsnames,8pt]{beamer}
    \usetheme{Madrid}
    \usefonttheme{professionalfonts}
    \usepackage{
        amsmath,
        amssymb,
        % fouriernc, % fourier font w/ new century book
        fancyhdr, % page styling
        lastpage, % footer fanciness
        hyperref, % various links
        setspace, % line spacing
        amsthm, % newtheorem and proof environment
        mathtools, % \Aboxed for boxing inside aligns, among others
        float, % Allow [H] figure env alignment
        enumerate, % Allow custom enumerate numbering
        graphicx, % allow includegraphics with more filetypes
        wasysym, % \smiley!
        upgreek, % \upmu for \mum macro
        listings, % writing TrueType fonts and including code prettily
        tikz, % drawing things
        booktabs, % \bottomrule instead of hline apparently
        cancel % can cancel things out!
    }
    \usepackage[
        labelfont=bf, % caption names are labeled in bold
        font=scriptsize % smaller font for captions
    ]{caption}
    \usepackage[font=scriptsize]{subcaption} % subfigures

    \newcommand*{\scinot}[2]{#1\times10^{#2}}
    \newcommand*{\dotp}[2]{\left<#1\,\middle|\,#2\right>}
    \newcommand*{\rd}[2]{\frac{\mathrm{d}#1}{\mathrm{d}#2}}
    \newcommand*{\pd}[2]{\frac{\partial#1}{\partial#2}}
    \newcommand*{\rtd}[2]{\frac{\mathrm{d}^2#1}{\mathrm{d}#2^2}}
    \newcommand*{\ptd}[2]{\frac{\partial^2 #1}{\partial#2^2}}
    \newcommand*{\md}[2]{\frac{\mathrm{D}#1}{\mathrm{D}#2}}
    \newcommand*{\pvec}[1]{\vec{#1}^{\,\prime}}
    \newcommand*{\svec}[1]{\vec{#1}\;\!}
    \newcommand*{\bm}[1]{\boldsymbol{\mathbf{#1}}}
    \newcommand*{\ang}[0]{\;\text{\AA}}
    \newcommand*{\mum}[0]{\;\upmu \mathrm{m}}
    \newcommand*{\at}[1]{\left.#1\right|}

    \let\Re\undefined
    \let\Im\undefined
    \DeclareMathOperator{\Res}{Res}
    \DeclareMathOperator{\Re}{Re}
    \DeclareMathOperator{\Im}{Im}
    \DeclareMathOperator{\Log}{Log}
    \DeclareMathOperator{\Arg}{Arg}
    \DeclareMathOperator{\Tr}{Tr}
    \DeclareMathOperator{\E}{E}
    \DeclareMathOperator{\Var}{Var}
    \DeclareMathOperator*{\argmin}{argmin}
    \DeclareMathOperator*{\argmax}{argmax}
    \DeclareMathOperator{\sgn}{sgn}
    \DeclareMathOperator{\diag}{diag\;}

    \DeclarePairedDelimiter\bra{\langle}{\rvert}
    \DeclarePairedDelimiter\ket{\lvert}{\rangle}
    \DeclarePairedDelimiter\abs{\lvert}{\rvert}
    \DeclarePairedDelimiter\ev{\langle}{\rangle}
    \DeclarePairedDelimiter\p{\lparen}{\rparen}
    \DeclarePairedDelimiter\s{\lbrack}{\rbrack}
    \DeclarePairedDelimiter\z{\lbrace}{\rbrace}

    % \everymath{\displaystyle} % biggify limits of inline sums and integrals
    \tikzstyle{circ} % usage: \node[circ, placement] (label) {text};
        = [draw, circle, fill=white, node distance=3cm, minimum height=2em]
    \definecolor{commentgreen}{rgb}{0,0.6,0}
    \lstset{
        basicstyle=\ttfamily\footnotesize,
        frame=single,
        numbers=left,
        showstringspaces=false,
        keywordstyle=\color{blue},
        stringstyle=\color{purple},
        commentstyle=\color{commentgreen},
        morecomment=[l][\color{magenta}]{\#}
    }

\begin{document}

\title{Dynamical Tides in Eccentric Massive Stellar Binaries}
\subtitle{Group Meeting}
\author{Yubo Su \& Dong Lai}
\date{Jan 23, 2020}

\maketitle

\begin{frame}
    \frametitle{Setup}
    \framesubtitle{Problem Description \& Previous Work}

    \begin{columns}
        \begin{column}{0.5\textwidth}
            \begin{itemize}
                \item Massive star with eccentric binary companion inducing dynamical
                    tides.

                \item Primary difficulty: dynamical tides is typically messy, sum over
                    many modes, hard to gain analytical intuition.

                \item Question: can we obtain a \emph{simple closed form for dynamical
                    tides} in this system?

                \item Dynamical tide in massive stars due to companion on \emph{circular
                    orbit} (Kushnir et.\ al.\ 2017).
                    \begin{align*}
                        \tau(\omega; r_c) ={}& \beta_2\frac{GM_2^2r_c^5}{a^6}
                                \frac{\rho_c}{\bar{\rho}_c} \p*{1 -
                                \frac{\rho_c}{\bar{\rho}_c}}^2\times\\
                            &\p*{\frac{\omega}{\sqrt{GM_c/r_c^3}}}^{8/3}.
                    \end{align*}

            \end{itemize}
        \end{column}
        \begin{column}{0.5\textwidth}
            \begin{itemize}
                \item Eccentric forcing is just sum of many circular forcings
                    (Fourier transform, e.g.\ Vick et.\ al.\ 2017)
                    \begin{equation*}
                        \tau_{\rm tot} = T_0 \sum\limits_{N = -\infty}^\infty
                            F_{N2}^2 \sgn\p*{\sigma} \tau
                                \p*{\omega = \abs*{\sigma}},
                    \end{equation*}
                    where $\sigma \equiv N\Omega - 2\Omega_s$ and $F_{Nm}$ are
                    the \emph{Hansen coefficients}
                    \begin{equation*}
                        F_{Nm} = \frac{1}{\pi}\int\limits_{0}^{\pi}
                            \frac{\cos\s*{N\mathcal{M}(E) - mf(E)}}
                                {\p*{1 - e\cos E}^2}\;\mathrm{d}E,
                    \end{equation*}
                    where $f$, $\mathcal{M}$, and $E$ are the true, mean, and
                    eccentric anomalies.
            \end{itemize}
        \end{column}
    \end{columns}
\end{frame}

\begin{frame}
    \frametitle{Setup}
    \framesubtitle{Hansen Coefficients}

    \begin{itemize}
        \item Thus, want to evaluate something of form
            \begin{equation*}
                \tau_{\rm tot} = \hat{T}(r_c, \Omega)
                    \sum\limits_{N = -\infty}^\infty F_{N2}^2(e)
                        \sgn\p*{N - 2\frac{\Omega_s}{\Omega}}
                        \abs*{N - 2\frac{\Omega_s}{\Omega}}^{8/3}.
            \end{equation*}
        \item The $F_{N2}$ look like (note: FT is fastest to compute
            coefficients, $e = 0.99$ took $\sim 2\;\mathrm{s}$)
            \begin{figure}[h]
                \centering
                \includegraphics[width=0.5\textwidth]{../scripts/eccentric_tides/hansens_plain.png}
            \end{figure}
        \item Key insight: only one important hump ($\sim N_{\rm peri}$), seek
            inspired fit.
    \end{itemize}
\end{frame}

\begin{frame}
    \frametitle{Solution}
    \framesubtitle{Key Insight}
    \begin{equation*}
        \tau_{\rm tot} = \hat{T}(r_c, \Omega)
            \sum\limits_{N = -\infty}^\infty F_{N2}^2(e)
                \sgn\p*{N - 2\frac{\Omega_s}{\Omega}}
                \abs*{N - 2\frac{\Omega_s}{\Omega}}^{8/3}.
    \end{equation*}

    \begin{itemize}
        \item Criteria for approximate $F_{N2}(e)$:
        \begin{itemize}
            \item Should only have one scale, $N_{\rm peri}$
            \item Should exponentially fall off for large $N$ (smoothness)
            \item $F_{02}(e) \approx 0$.
        \end{itemize}

    \end{itemize}

    \begin{columns}
        \begin{column}{0.5\textwidth}
            \begin{itemize}
                \item Guess: maybe
                    \begin{equation}
                        F_{N2}(e) \simeq C(e) N^{p(e)} e^{-N / \eta(e)}.
                    \end{equation}
                    Turns out $p \approx 2$.

                \item Furthermore, $\argmax\limits_N F_{N2}(e) =
                    p\eta(e)$, so $\eta \simeq N_{\rm peri} / 2$.

                \item $C$ is fixed by normalization (Parseval's).

                \item Finally, $\int\limits_0^\infty x^pe^{-x}\;\mathrm{d}x
                    \equiv \Gamma(p - 1)$.
            \end{itemize}
        \end{column}
        \begin{column}{0.5\textwidth}
            \begin{figure}[h]
                \centering
                \includegraphics[width=0.8\columnwidth]{../scripts/eccentric_tides/hansens99.png}
            \end{figure}
        \end{column}
    \end{columns}
\end{frame}

\begin{frame}
    \frametitle{Solution}
    \framesubtitle{Tidal Torque}

    \begin{itemize}
        \item Use resulting $\tau_{\rm tot}$ in closed form (piecewise for
            $\Omega_s \gg N_{\rm peri}\Omega / 2$ or $\Omega_s \ll N_{\rm
            peri}\Omega / 2$), some small fudge factors, compare with explicit
            sum:
            \begin{figure}[h]
                \centering
                \includegraphics[width=0.4\textwidth]{../scripts/eccentric_tides/totals_ecc_0.png}
                \includegraphics[width=0.4\textwidth]{../scripts/eccentric_tides/totals_ecc_400.png}
                \includegraphics[width=0.4\textwidth]{../scripts/eccentric_tides/totals_s_0_9.png}
            \end{figure}
    \end{itemize}
\end{frame}

\begin{frame}
    \frametitle{Solution}
    \framesubtitle{Heating}

    \begin{itemize}
        \item For heating, same treatment for $F_{N2}$, need $m = 0$ Hansen
            coefficients $F_{N0}$ too:
        \begin{align}
             \dot{E}_{\rm in} = \frac{1}{2}\hat{T}\p*{r_c, \Omega}
                 \sum\limits_{N = -\infty}^\infty\s*{
                    N\Omega F_{N2}^2 \sgn \p*{\sigma} \abs*{\sigma}^{8/3}
                    + \p*{\frac{W_{20}}{W_{22}}}^2\Omega F_{N0}^2 \abs*{N}^{11/3}}.
                    \label{eq:e_sum}
        \end{align}
        \begin{figure}[h]
            \centering
            \includegraphics[width=0.5\textwidth]{../scripts/eccentric_tides/hansens/hansens0_90.png}
        \end{figure}

        \item Only characteristic scale is still $N_{\rm peri}$.
    \end{itemize}
\end{frame}

\begin{frame}
    \frametitle{Solution}
    \framesubtitle{Heating}

    \begin{itemize}
        \item Guess $F_{N0} \simeq Ae^{-\frac{\abs{N}}{N_{\rm peri}}}$.
            Empirically, find
            \begin{equation}
                F_{N0} \approx Ae^{-\frac{\abs{N}}{N_{\rm peri}/\sqrt{2}}},
            \end{equation}

        \begin{figure}[t]
            \centering
            \includegraphics[width=0.4\textwidth]{../scripts/eccentric_tides/totals_e_0.png}
            \includegraphics[width=0.4\textwidth]{../scripts/eccentric_tides/totals_e_400.png}
        \end{figure}
    \end{itemize}
\end{frame}

\begin{frame}
    \frametitle{Upcoming Work}

    \begin{itemize}
        \item Application to J0045+7319?
        \begin{itemize}
            \item Since $e$ is known, $\tau_{\rm tot}$ is very easy to evaluate
                without approximation.

            \item Using earliest works (Lai 1996, Kumar \& Quaetert 1997), find
                $\frac{\Omega_s}{\Omega} \approx -0.37$.

            \item Since these works, stellar mass $\sim 10 \pm 1 M_{\odot}$.
                Working on recalculating using MESA, numbers seem wrong (K\&Q
                get $M_{c} = 3M_{\odot}$, $R_{\star} = 6R_{\odot}$, $R_c =
                0.23R_{\star}$, while i get $M_c = 2.75M_{\odot}, R_\star =
                3.5R_{\odot}, R_c = 0.2R_{\star}$). Debugging.
        \end{itemize}

        \item Tidal synchronization timescale as a function of $e$?
    \end{itemize}
\end{frame}

\end{document}

