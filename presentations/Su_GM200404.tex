    \documentclass[dvipsnames, 8pt]{beamer}
    \usetheme{Madrid}
    \usefonttheme{professionalfonts}
    \usepackage{
        amsmath,
        amssymb,
        fouriernc, % fourier font w/ new century book
        fancyhdr, % page styling
        lastpage, % footer fanciness
        hyperref, % various links
        setspace, % line spacing
        amsthm, % newtheorem and proof environment
        mathtools, % \Aboxed for boxing inside aligns, among others
        float, % Allow [H] figure env alignment
        enumerate, % Allow custom enumerate numbering
        graphicx, % allow includegraphics with more filetypes
        wasysym, % \smiley!
        upgreek, % \upmu for \mum macro
        listings, % writing TrueType fonts and including code prettily
        tikz, % drawing things
        booktabs, % \bottomrule instead of hline apparently
        cancel % can cancel things out!
    }
    \usepackage[
        labelfont=bf, % caption names are labeled in bold
        font=scriptsize % smaller font for captions
    ]{caption}
    \usepackage[font=scriptsize]{subcaption} % subfigures

    \newcommand*{\scinot}[2]{#1\times10^{#2}}
    \newcommand*{\dotp}[2]{\left<#1\,\middle|\,#2\right>}
    \newcommand*{\rd}[2]{\frac{\mathrm{d}#1}{\mathrm{d}#2}}
    \newcommand*{\pd}[2]{\frac{\partial#1}{\partial#2}}
    \newcommand*{\rtd}[2]{\frac{\mathrm{d}^2#1}{\mathrm{d}#2^2}}
    \newcommand*{\ptd}[2]{\frac{\partial^2 #1}{\partial#2^2}}
    \newcommand*{\md}[2]{\frac{\mathrm{D}#1}{\mathrm{D}#2}}
    \newcommand*{\pvec}[1]{\vec{#1}^{\,\prime}}
    \newcommand*{\svec}[1]{\vec{#1}\;\!}
    \newcommand*{\bm}[1]{\boldsymbol{\mathbf{#1}}}
    \newcommand*{\uv}[1]{\hat{\bm{#1}}}
    \newcommand*{\ang}[0]{\;\text{\AA}}
    \newcommand*{\mum}[0]{\;\upmu \mathrm{m}}
    \newcommand*{\at}[1]{\left.#1\right|}
    \newcommand*{\bra}[1]{\left<#1\right|}
    \newcommand*{\ket}[1]{\left|#1\right>}
    \newcommand*{\abs}[1]{\left|#1\right|}
    \newcommand*{\ev}[1]{\langle#1\rangle}
    \newcommand*{\p}[1]{\left(#1\right)}
    \newcommand*{\s}[1]{\left[#1\right]}
    \newcommand*{\z}[1]{\left\{#1\right\}}

    \let\Re\undefined
    \let\Im\undefined
    \DeclareMathOperator{\Res}{Res}
    \DeclareMathOperator{\Re}{Re}
    \DeclareMathOperator{\Im}{Im}
    \DeclareMathOperator{\Log}{Log}
    \DeclareMathOperator{\Arg}{Arg}
    \DeclareMathOperator{\Tr}{Tr}
    \DeclareMathOperator{\E}{E}
    \DeclareMathOperator{\Var}{Var}
    \DeclareMathOperator*{\argmin}{argmin}
    \DeclareMathOperator*{\argmax}{argmax}
    \DeclareMathOperator{\sgn}{sgn}
    \DeclareMathOperator{\diag}{diag\;}

    % \everymath{\displaystyle} % biggify limits of inline sums and integrals
    \tikzstyle{circ} % usage: \node[circ, placement] (label) {text};
        = [draw, circle, fill=white, node distance=3cm, minimum height=2em]
    \definecolor{commentgreen}{rgb}{0,0.6,0}
    \lstset{
        basicstyle=\ttfamily\footnotesize,
        frame=single,
        numbers=left,
        showstringspaces=false,
        keywordstyle=\color{blue},
        stringstyle=\color{purple},
        commentstyle=\color{commentgreen},
        morecomment=[l][\color{magenta}]{\#}
    }

\begin{document}

\title{Spin-Orbit Misalignment Dynamics in Black Hole Triples}
\subtitle{Group Meeting?}
\author{Yubo Su}
\date{Apr. 4, 2020?}

\maketitle

\begin{frame}
    \frametitle{Background}
    \framesubtitle{Equations of Motion}

    \begin{columns}
        \begin{column}{0.5\textwidth}
            Equations:
            \begin{align}
                \rd{\uv{S}}{t} &= \Omega_{SL} \uv{L} \times \uv{S},\\
                \Omega_{SL} &= \frac{3Gn\p{m_2 + \mu/3}}{2c^2a\p{1 - e^2}},\\
                \rd{I}{t} &= -\frac{15}{16t_{LK}}\frac{e^2\sin\p{2\omega}
                    \sin \p{2I}}{\sqrt{1 - e^2}},\\
                \rd{\Omega}{t} &= \frac{3}{4t_{LK}}
                    \frac{\cos i\p{5e^2\cos^2\omega - 4e^2 - 1}}{\sqrt{1 - e^2}}
                    ,\\
                \frac{1}{t_{LK}} &= n\p{\frac{m_3}{m_1 + m_2}}
                    \p{\frac{a}{\bar{a}_{3, \rm eff}}}^3.
            \end{align}
            GW radiation narrows range of $e$ oscillations. $\theta_{\rm sl}^f =
            \cos^{-1} \p{\uv{L} \cdot \uv{S}}$?
        \end{column}
        \begin{column}{0.5\textwidth}
            \begin{figure}
                \centering
                \begin{tikzpicture}[scale=2]
                    \draw[black, ->] (0, 0) -- (0, 1);
                    \draw[red, ->] (0, 0) -- (-0.9, -0.15);
                    \draw[blue, ->] (0, 0) -- (-0.85, -0.22);
                    \node[right] at (0, 1) {$\uv{L}_3$};
                    \node[above] at (-0.9, -0.15) {$\uv{L}$};
                    \node[below] at (-0.85, -0.22) {$\uv{S}$};
                \end{tikzpicture}
            \end{figure}
        \end{column}
    \end{columns}
\end{frame}

\begin{frame}
    \frametitle{Averaging and Resonances}
    \framesubtitle{Toy Model}

    \begin{columns}
        \begin{column}{0.5\textwidth}
            \begin{itemize}
                \item Go to corotating frame with $\dot{\Omega}$, so that
                    $\uv{L}$ nutates and $\uv{L}_3 = \uv{z}$:
                    \begin{equation}
                        \rd{\uv{S}}{t} = \p{\Omega_{SL} \uv{L}
                            - \dot{\Omega} \uv{L}_3} \times \uv{S}
                            \equiv \uv{\Omega}_{\rm eff} \times \uv{S}.
                    \end{equation}

                \item Intuition: If no LK, $\Omega_{SL}$, $\dot{\Omega}$ slowly
                    vary, $\theta_{\rm sl}^f = \theta_{\rm s3}^i$.

                \item $\Omega_{SL}\uv{L}$ and $\dot{\Omega}\uv{L}_3$ periodic
                    with $T_{LK}$ (varying), so decompose about the mean value:
                    \begin{align}
                        \rd{\uv{S}}{t} ={}&
                            \ev{\uv{\Omega}_{\rm eff}}_{LK}
                                \times \uv{S}\nonumber\\
                            &- \bm{S} \times \sum\limits_{N = 1}^\infty \uv{A}_N
                                \cos \p{\frac{2\pi N t}{T_{LK}}}.
                    \end{align}

                \item Formally, can average (or WKBJ) \textbf{unless}
                    $\Omega_{\rm eff} = \frac{2\pi \p{N' / 2}}{T_{LK}}$.
            \end{itemize}
        \end{column}
        \begin{column}{0.5\textwidth}
            \begin{itemize}
                \item What are these \textbf{linear} resonances? Consider toy
                    model, $\epsilon \to 0$:
                    \begin{align}
                        \rd{\uv{S}}{t} &= \s{\omega_0 \uv{z}
                            + \epsilon\p{\cos \omega t \uv{x} + \sin
                            \omega t \uv{y}}} \times \uv{S},\\
                        \p{\rd{\uv{S}}{t}}_{\rm rot} &=
                            \s{\p{\omega_0 - \omega}\uv{z} + \epsilon
                                \uv{x}} \times \uv{S}.
                    \end{align}

                \item In resonances, the precession axis tilts away from
                    $\uv{z}$ plane when $\omega_0 - \omega \sim \epsilon$. Then
                    $\dot{\theta} \simeq \epsilon$.

                \item Thus, when crossing:
                    \begin{itemize}
                        \item If slow, adiabatic invariance, $\theta^f =
                            \theta^i$.
                        \item If fast, no time to rotate, $\theta^f \approx
                            \theta^i$.
                        \item Only when $\rd{\ln \omega}{t} \simeq \epsilon$
                            will $\theta^f \neq \theta^i$.
                    \end{itemize}
            \end{itemize}
        \end{column}
    \end{columns}
\end{frame}

\begin{frame}
    \frametitle{Averaged Scenario}
    \framesubtitle{Resonances in the Slow-Merger Regime}

    % N sums with (N + N'), gives cos((2N + N') / 2) cos(N' / 2)
    \begin{columns}
        \begin{column}{0.5\textwidth}
            \begin{itemize}
                \item If resonance $\Omega_{\rm eff} = \pi N'/T_{LK}$:
            \end{itemize}
            \begin{align}
                \rd{\uv{S}}{t} ={}&
                    \ev{\uv{\Omega}_{\rm eff}}_{LK}
                        \times \uv{S}\nonumber\\
                    &-\frac{\uv{S}}{2} \times \sum\limits_{N = 1}^{N'} \uv{A}_N
                        \cos \p{\frac{2\pi N t}{T_{LK}}}\nonumber\\
                    &- \frac{\uv{S}}{2} \times \sum\limits_{N = 1}^\infty
                        \uv{B}_N \cos\p{\Omega_{\rm eff}t}
                        - \uv{C}_N \sin \p{\Omega_{\rm eff}t},\\
                \uv{B}_N(t) &= \p{\uv{A}_N + \uv{A}_{N + N'}}\cos \p{\frac{\pi
                    \p{N + N'}t}{T_{LK}}},\\
                \uv{C}_N(t) &= \p{\uv{A}_N - \uv{A}_{N + N'}}\sin \p{\frac{\pi
                    \p{N + N'}t}{T_{LK}}}.
            \end{align}
        \end{column}
        \begin{column}{0.5\textwidth}
            \begin{itemize}
                \item Equivalently, consider using Hamiltonian and going to
                    coordinates (sum every $N$ and $N + N'$ term to generate
                    multiples of $N'/2$). Note $\sin\p{\phi - \frac{\pi N'
                    t}{T_{LK}}}$ dependencies.
            \end{itemize}
            \begin{align}
                H ={}&
                    \ev{\uv{\Omega}_{\rm eff}}_{LK}
                        \cdot \uv{S}\nonumber\\
                    &+ \sum\limits_{N = 1}^\infty \p{A_{N,z}\cos\theta
                        + A_{N,x}\sin\theta \cos \phi}
                        \cos \p{\frac{2\pi N t}{T_{LK}}},\\
                    ={}& \ev{\uv{\Omega}_{\rm eff}}_{LK}
                        \cdot \uv{S}\nonumber\\
                    &+\frac{\uv{S}}{2} \cdot \sum\limits_{N = 1}^{N'} \uv{A}_N
                        \cos \p{\frac{2\pi N t}{T_{LK}}}\nonumber\\
                    &+ \frac{\uv{S}}{2} \cdot \sum\limits_{N = 1}^\infty
                        \p{B_{N,z} \cos\theta + B_{N,x}\sin\theta\cos \phi}
                            \cos\p{\frac{\pi N' t}{T_{LK}}}\nonumber\\
                    &- \p{C_{N,z}\cos\theta + C_{N,x}\sin\theta \cos \phi }
                        \sin \p{\frac{\pi N' t}{T_{LK}}}.
            \end{align}
        \end{column}
    \end{columns}
\end{frame}
\begin{frame}
    \frametitle{Averaged Scenario}
    \framesubtitle{When are Resonances Hit?}
    \begin{itemize}
        \item Note: If $\Omega_{SL} \gtrsim \dot{\Omega}$, LK
            suppressed. Thus, consider primarily
            $\ev{\dot{\Omega}}_{LK}\uv{z}$ dominating $\uv{\Omega}_{\rm eff}$.

        \item We want $\ev{\dot{\Omega}} = \pi N'/T_{LK}$, which is
            equivalent to:
            \begin{equation}
                \pi N' = \Delta \Omega = \int\limits_0^{T_{LK}}
                    \dot{\Omega}\;\mathrm{d}t.
            \end{equation}

        \item It turns out that for a wide range of parameters,
            $\abs{\Delta \Omega} < \pi$, thus, no resonance can be hit.
    \end{itemize}
\end{frame}

\begin{frame}
    \frametitle{Averaged Scenario}
    \framesubtitle{When are Resonances Hit?}

    \begin{itemize}
        \item As GW radiates, $e_0$ increases, \emph{decreases} $\Delta
            \Omega$ (right; numerical). Never resonance crossing ($\Delta \Omega
            < -\pi$, black line) when $e_0 = 10^{-3}$ for $I_0 \in [75, 105]$.
            Seems about right! Fig.~4 from Liu \& Lai 2017.
    \end{itemize}

    \begin{figure}[h]
        \centering
        \includegraphics[width=0.4\textwidth]{bin_2017_f4.png}
        \includegraphics[width=0.55\textwidth]{../scripts/lk90/5_dWs.png}
    \end{figure}
\end{frame}

\begin{frame}
    \frametitle{Averaged Scenario}
    \framesubtitle{What is the Effect of Hitting a Resonance?}

    \begin{itemize}
        \item Consider again toy problem
            \begin{equation}
                \p{\rd{\uv{S}}{t}}_{\rm rot} =
                    \s{\p{\omega_0 - \omega}\uv{z} + \epsilon
                        \uv{x}} \times \uv{S}.
            \end{equation}

        \item Allow $\omega = \omega_0 + \dot{\omega}t$. What is $\theta_f -
            \theta_i$? Exact:
            \begin{equation}
                \Delta \theta = \epsilon \sqrt{\frac{2\pi}{\dot{\omega}}}
                    \sin \p{\phi_0 + \frac{\pi}{4}}.
            \end{equation}

        \item Here, $\dot{\omega} \sim 1 / T_{GW}$, while $\epsilon \propto 1 /
            T_{LK}$, $\Rightarrow \Delta \theta \gg 1$?

        \item Caveat: $\epsilon$ is Fourier coefficient of $\dot{\Omega}$, has
            $1/T_{LK}$ scaling but \emph{is probably substantially smaller}. So
            can probably predict width ($\sim \epsilon$)?

        \begin{itemize}
            \item Might explain why deviations from the blue curve in Bin's
                figure seem to appear closer to $I_0 = 90^\circ$ than expected
                ($100^\circ$ vs $105^\circ$).

            \item Might also predict the amplitude of the deviations (which
                appear bounded closer to $90^\circ$).
        \end{itemize}
    \end{itemize}
\end{frame}

\begin{frame}
    \frametitle{Resonance-Free Dynamics}
    \framesubtitle{Few-Shot Merger Solution}

    \begin{columns}
        \begin{column}{0.5\textwidth}
            \begin{itemize}
                \item When no resonances, just
                    \begin{align}
                        \rd{\uv{S}}{t} &=
                            \ev{\Omega_{SL}\uv{L} - \dot{\Omega} \uv{L_3}}_{LK}
                                \times \uv{S},\\
                        \frac{1}{\dot{\Omega}} \rd{\uv{S}}{t}
                            &\simeq \s{\mathcal{A}_0\p{\sin I \uv{x}
                                + \cos I \uv{z}}
                                \mp \uv{z}} \times \uv{S}.
                    \end{align}

                \item Here, $I$ is the inclination of $\ev{\Omega_{SL}\uv{L}}$,
                    shouldn't change much, $\sim 120^\circ$?

                \item Crosses resonance at $\mathcal{A} \simeq 1 / \cos I \simeq
                    1$, and the jump
                    in angle is
                    \begin{equation}
                        \Delta \theta \simeq \sin I
                            \sqrt{\frac{2\pi}{\dot{\mathcal{A}_0}}}
                                \sin \p{\phi_0 + \pi/4}.
                    \end{equation}

                \item Explains why for fixed $\phi$, $\Delta \theta$ overlaps.
            \end{itemize}
        \end{column}
        \begin{column}{0.5\textwidth}
            \begin{itemize}
                \item Sweeped some stuff under the rug, but does this make
                    qualitative sense?

                \item $T_{\rm m} \sim \cos^6 I_0$, so if $\dot{\mathcal{A}}_0
                    \propto 1 / T_{\rm m}$ then $\Delta \theta \propto 1 /
                    \cos^3 I_0$. Seems to be the right scaling (black is fit by
                    eye, correct power law index):

                \begin{figure}[h]
                    \centering
                    \includegraphics[width=0.7\textwidth]{../scripts/lk90/4sims_ensemble/deviations.png}
                \end{figure}

                \item {\small (Ignore the nonzero tail at the right, I made a
                    lazy approximation in my code)}
            \end{itemize}
        \end{column}
    \end{columns}
\end{frame}

\end{document}

