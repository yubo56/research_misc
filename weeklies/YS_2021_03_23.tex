    \documentclass[11pt,
        usenames, % allows access to some tikz colors
        dvipsnames % more colors: https://en.wikibooks.org/wiki/LaTeX/Colors
    ]{article}
    \usepackage{
        amsmath,
        amssymb,
        fouriernc, % fourier font w/ new century book
        fancyhdr, % page styling
        lastpage, % footer fanciness
        hyperref, % various links
        setspace, % line spacing
        amsthm, % newtheorem and proof environment
        mathtools, % \Aboxed for boxing inside aligns, among others
        float, % Allow [H] figure env alignment
        enumerate, % Allow custom enumerate numbering
        graphicx, % allow includegraphics with more filetypes
        wasysym, % \smiley!
        upgreek, % \upmu for \mum macro
        listings, % writing TrueType fonts and including code prettily
        tikz, % drawing things
        booktabs, % \bottomrule instead of hline apparently
        xcolor, % colored text
        cancel % can cancel things out!
    }
    \usepackage[margin=1in]{geometry} % page geometry
    \usepackage[
        labelfont=bf, % caption names are labeled in bold
        font=scriptsize % smaller font for captions
    ]{caption}
    \usepackage[font=scriptsize]{subcaption} % subfigures

    \newcommand*{\scinot}[2]{#1\times10^{#2}}
    \newcommand*{\dotp}[2]{\left<#1\,\middle|\,#2\right>}
    \newcommand*{\rd}[2]{\frac{\mathrm{d}#1}{\mathrm{d}#2}}
    \newcommand*{\pd}[2]{\frac{\partial#1}{\partial#2}}
    \newcommand*{\rdil}[2]{\mathrm{d}#1 / \mathrm{d}#2}
    \newcommand*{\pdil}[2]{\partial#1 / \partial#2}
    \newcommand*{\rtd}[2]{\frac{\mathrm{d}^2#1}{\mathrm{d}#2^2}}
    \newcommand*{\ptd}[2]{\frac{\partial^2 #1}{\partial#2^2}}
    \newcommand*{\md}[2]{\frac{\mathrm{D}#1}{\mathrm{D}#2}}
    \newcommand*{\pvec}[1]{\vec{#1}^{\,\prime}}
    \newcommand*{\svec}[1]{\vec{#1}\;\!}
    \newcommand*{\bm}[1]{\boldsymbol{\mathbf{#1}}}
    \newcommand*{\uv}[1]{\hat{\bm{#1}}}
    \newcommand*{\ang}[0]{\;\text{\AA}}
    \newcommand*{\mum}[0]{\;\upmu \mathrm{m}}
    \newcommand*{\at}[1]{\left.#1\right|}
    \newcommand*{\bra}[1]{\left<#1\right|}
    \newcommand*{\ket}[1]{\left|#1\right>}
    \newcommand*{\abs}[1]{\left|#1\right|}
    \newcommand*{\ev}[1]{\left\langle#1\right\rangle}
    \newcommand*{\p}[1]{\left(#1\right)}
    \newcommand*{\s}[1]{\left[#1\right]}
    \newcommand*{\z}[1]{\left\{#1\right\}}

    \newtheorem{theorem}{Theorem}[section]

    \let\Re\undefined
    \let\Im\undefined
    \DeclareMathOperator{\Res}{Res}
    \DeclareMathOperator{\Re}{Re}
    \DeclareMathOperator{\Im}{Im}
    \DeclareMathOperator{\Log}{Log}
    \DeclareMathOperator{\Arg}{Arg}
    \DeclareMathOperator{\Tr}{Tr}
    \DeclareMathOperator{\E}{E}
    \DeclareMathOperator{\Var}{Var}
    \DeclareMathOperator*{\argmin}{argmin}
    \DeclareMathOperator*{\argmax}{argmax}
    \DeclareMathOperator{\sgn}{sgn}
    \DeclareMathOperator{\diag}{diag\;}

    \colorlet{Corr}{red}

    % \everymath{\displaystyle} % biggify limits of inline sums and integrals
    \tikzstyle{circ} % usage: \node[circ, placement] (label) {text};
        = [draw, circle, fill=white, node distance=3cm, minimum height=2em]
    \definecolor{commentgreen}{rgb}{0,0.6,0}
    \lstset{
        basicstyle=\ttfamily\footnotesize,
        frame=single,
        numbers=left,
        showstringspaces=false,
        keywordstyle=\color{blue},
        stringstyle=\color{purple},
        commentstyle=\color{commentgreen},
        morecomment=[l][\color{magenta}]{\#}
    }

\begin{document}

The characteristic parameters I choose are: $I = 0$, $e_{0} = 10^{-3}$, $a_{\rm
out} = 2.38\;\mathrm{AU}$, $a \approx 0.002\;\mathrm{AU}$ ($f =
10^{-3}\;\mathrm{Hz}$), $m_1 = m_2 = m_3 = M_{\odot}$.

\section{Circular}

\subsection{Analytic}

I was able to rederive Wenrui's Hamiltonian by averaging the SA Hamiltonian,
then I used sympy to check that:
\begin{align}
    H\p{\Gamma,\phi} &\approx \Gamma P - \Gamma^2 Q + R \Gamma \cos \phi,&&\\
        P &\approx 2\s{1 - \frac{\Omega_{\rm out}}{\Omega_{\rm GR, 0}} -
            \frac{\epsilon\p{12 J_3 - 3}}{4}},&
        Q &\approx 4 - \frac{3\epsilon}{2} \approx 4,\nonumber\\
        R &\approx \frac{15\epsilon}{2},\\
        \epsilon &= \frac{m_3a^4 c^2}{3Gm_{12}^2a_{\rm out}^3},&
        \frac{\Omega_{\rm out}}{\Omega_{\rm GR, 0}}
            &= \frac{\p{a / a_{\rm out}}^{3/2}\p{m_{123} / m_{12}}^{1/2}}{
                3Gm_{12} / (c^2a)}.
\end{align}
Here, my notation is: $\Omega_{\rm out} = \sqrt{Gm_{123}/a_{\rm out}^3}$, $\phi
= 2\p{\varpi - \lambda_{\rm out}}$, $\Gamma = -\p{1 - \sqrt{1 - e^2}} / 2
\approx -e^2/4$ is its conjugate variable, $J_3 = \sqrt{1 - e^2}\p{1 - \cos I}$
is a constant, $\Omega_{\rm GR, 0} = 3Gnm_{12} / (c^2a)$, and $\epsilon =
\Phi_{\rm out} / \Phi_{\rm GR, 0}$, or
\begin{equation}
    \epsilon = \frac{m_3a^4 c^2}{3Gm_{12}^2a_{\rm out}^3} \approx 10^{-5}.
\end{equation}

\subsection{Numerical}

The location of the resonance can be found by scanning
\begin{equation}
    \Delta e \equiv e_{\max} - e_{\min}
\end{equation}
when integrating the SA equations. This is shown in
Fig~\ref{fig:composite_circ}.
\begin{figure}[h]
    \centering
    \includegraphics[width=0.5\columnwidth]{../scripts/misc/1evection/composite.png}
    \caption{Circular resonance.}\label{fig:composite_circ}
\end{figure}

The Hamiltonian phase portrait is shown in Fig.~\ref{fig:H}. For comparison,
numerical simulations are shown in Fig.~\ref{fig:sim}. Note that the actual
eccentricity maxima are slightly larger, as $H$ is not conserved too well due to
the change in $a_{\rm out}$.
\begin{figure}[h]
    \centering
    \includegraphics[width=0.5\columnwidth]{../scripts/misc/sim38_H.png}
    \caption{Hamiltonian for fiducial params.}\label{fig:H}
\end{figure}
\begin{figure}[h]
    \centering
    \includegraphics[width=0.44\columnwidth]{../scripts/misc/1evection/sim38long.png}
    \includegraphics[width=0.44\columnwidth]{../scripts/misc/1evection/sim38long1.png}
    \includegraphics[width=0.44\columnwidth]{../scripts/misc/1evection/sim38long2.png}
    \includegraphics[width=0.44\columnwidth]{../scripts/misc/1evection/sim38long3.png}
    \includegraphics[width=0.44\columnwidth]{../scripts/misc/1evection/sim38long4.png}
    \caption{Sims for fiducial, circular.}\label{fig:sim}
\end{figure}

\section{Eccentric}

\subsection{Analytic}

I'm not sure if I'm averaging correctly, see my calculation below. The SA H is:
\begin{align}
    \tilde{H}_{\rm SA, out}
        = \frac{1}{4}\p{\frac{a_{\rm out}}{r_{\rm out}}}^3 \s{
            -1 + 6e^2 + 3\p{1 - e^2}\p{\uv{n} \cdot \uv{r}_{\rm out}}^2
            - 15 e^2\p{\uv{e} \cdot \uv{r}_{\rm out}}^2}.
\end{align}
We use coordinate system
\begin{align}
    r_{\rm out} &= \frac{a_{\rm out}\p{1 - e_{\rm out}^2}}{1 + e_{\rm out} \cos
        f_{\rm out}} &
    \uv{r}_{\rm out} &= \begin{pmatrix}
        \cos v_{\rm out}\\
        \sin v_{\rm out} \cos I\\
        \sin v_{\rm out} \sin I
    \end{pmatrix}.
\end{align}
Here, $v_{\rm out} = \ascnode_{\rm out} + \omega_{\rm out} + f_{\rm out} =
\omega_{\rm out} + f_{\rm out}$ is the \emph{true longitude}. When averaging,
everything non-resonant can be averaged via the usual identities (note that
averaging over $f_{\rm out}$ is the same as averaging over $v_{\rm out}$ since
$\omega_{\rm out}$ is approximately constant)
\begin{align}
    \ev{\frac{\cos^2 f_{\rm out}}{r_{\rm out}^3}} &=
        \ev{\frac{\sin^2 f_{\rm out}}{r_{\rm out}^3}}
        = \frac{1}{2a_{\rm out}^3\p{1 - e_{\rm out}^2}^{3/2}},\\
    \ev{\frac{1}{r_{\rm out}^3}} &= \frac{1}{a_{\rm out}^3\p{1 - e_{\rm
        out}^2}^{3/2}},\\
    \ev{\frac{\cos f_{\rm out} \sin f_{\rm out}}{r_{\rm out}^3}} &= 0.
\end{align}
The only resonant term is the $\p{\uv{e} \cdot \uv{r}_{\rm out}}^2$ term. We
handle this via:
\begin{align}
    \uv{e}(t) &= \cos \p{\Omega_{\rm GR}t + \varpi_0} \uv{x} +
        \sin\p{\Omega_{\rm GR}t + \varpi_0}\uv{y},\\
    \frac{\uv{r}_{\rm out}(t)_{\perp}}{r_{\rm out}^{3/2}} &=
        \frac{1}{r_{\rm out}^{3/2}}\s{\cos v_{\rm out}
            \uv{x} + \sin v_{\rm out}\cos I \uv{y}},\\
        &= \sum\limits_{N = 1}^\infty
            \frac{c_N}{a_{\rm out}^{3/2}}\z{\cos\p{N \Omega_{\rm out}t} \uv{x}
            + \sin\p{N \Omega_{\rm out}t} \cos I\uv{y}},\\
    \uv{e} \cdot \frac{\uv{r}_{\rm out}}{r_{\rm out}^{3/2}}
        &= \sum\limits_{N = 1}^\infty
            \frac{c_N}{a_{\rm out}^{3/2}}
                \z{\cos\p{N \Omega_{\rm out}t} \cos\p{\Omega_{\rm GR}t +
                    \varpi_0}
            + \sin\p{N \Omega_{\rm out}t} \cos I \sin\p{\Omega_{\rm GR}t +
                \varpi_0}},\\
        &= \sum\limits_{N = 1}^\infty
            \frac{c_N}{a_{\rm out}^{3/2}}
                \z{
                    \cos\p{\p{N\Omega_{\rm out} - \Omega_{\rm GR}}t - \varpi_0}
                        \p{\frac{1 + \cos I}{2}}
                    + \cos\p{\p{N\Omega_{\rm out} + \Omega_{\rm GR}}t + \varpi_0}
                        \p{\frac{1 - \cos I}{2}}
                    },\\
    \ev{\frac{\p{\uv{e} \cdot \uv{r}_{\rm out}}^2}{r_{\rm out}^3}}
        &= \sum\limits_{M = 0}^{N - 1}
            \p{2 - \delta_{M(N/2)}}
            \frac{c_{N_{\rm GR} + M}c_{N_{\rm GR} - M}}{2a_{\rm out}^3}
            \p{\frac{1 + \cos I}{2}}^2 \cos\p{\p{2N_{\rm GR}\Omega_{\rm out}
                - 2\Omega_{\rm GR}}t - 2\varpi_0}.
\end{align}
Here, $N_{\rm GR} \equiv \lfloor \Omega_{\rm GR} / \Omega_{\rm out}$. Since the
$c_N$ should fall off for $N \gtrsim N_{\rm p}$ where
\begin{equation}
    N_{\rm p} \equiv \frac{\sqrt{1 + e}}{\p{1 - e_{\rm out}}^{3/2}},
\end{equation}
we see that there will generally be resonances for all $N\Omega_{\rm out} \sim
\Omega_{\rm GR}$ as long as $N \lesssim N_{\rm p}$.

\subsection{Numeric}

We consider the same fiducial parameters as above except for $e_{\rm out} =
0.6$, for which $N_{\rm p} = 5$.
\begin{figure}[h]
    \centering
    \includegraphics[width=0.6\columnwidth]{../scripts/misc/1evection_eccscan/composite.png}
    \includegraphics[width=0.6\columnwidth]{../scripts/misc/1evection_eccscan/composite_delta.png}
    \caption{Evection resonance, eccentric. Bottom: classified by the resonance
    order $N\Omega_{\rm out} = \Omega_{\rm GR, 0}$.}\label{fig:eccentric_composite}
\end{figure}

\end{document}

