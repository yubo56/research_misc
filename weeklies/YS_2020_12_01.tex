    \documentclass[11pt,
        usenames, % allows access to some tikz colors
        dvipsnames % more colors: https://en.wikibooks.org/wiki/LaTeX/Colors
    ]{article}
    \usepackage{
        amsmath,
        amssymb,
        fouriernc, % fourier font w/ new century book
        fancyhdr, % page styling
        lastpage, % footer fanciness
        hyperref, % various links
        setspace, % line spacing
        amsthm, % newtheorem and proof environment
        mathtools, % \Aboxed for boxing inside aligns, among others
        float, % Allow [H] figure env alignment
        enumerate, % Allow custom enumerate numbering
        graphicx, % allow includegraphics with more filetypes
        wasysym, % \smiley!
        upgreek, % \upmu for \mum macro
        listings, % writing TrueType fonts and including code prettily
        tikz, % drawing things
        booktabs, % \bottomrule instead of hline apparently
        xcolor, % colored text
        cancel % can cancel things out!
    }
    \usepackage[margin=1in]{geometry} % page geometry
    \usepackage[
        labelfont=bf, % caption names are labeled in bold
        font=scriptsize % smaller font for captions
    ]{caption}
    \usepackage[font=scriptsize]{subcaption} % subfigures

    \newcommand*{\scinot}[2]{#1\times10^{#2}}
    \newcommand*{\dotp}[2]{\left<#1\,\middle|\,#2\right>}
    \newcommand*{\rd}[2]{\frac{\mathrm{d}#1}{\mathrm{d}#2}}
    \newcommand*{\pd}[2]{\frac{\partial#1}{\partial#2}}
    \newcommand*{\rdil}[2]{\mathrm{d}#1 / \mathrm{d}#2}
    \newcommand*{\pdil}[2]{\partial#1 / \partial#2}
    \newcommand*{\rtd}[2]{\frac{\mathrm{d}^2#1}{\mathrm{d}#2^2}}
    \newcommand*{\ptd}[2]{\frac{\partial^2 #1}{\partial#2^2}}
    \newcommand*{\md}[2]{\frac{\mathrm{D}#1}{\mathrm{D}#2}}
    \newcommand*{\pvec}[1]{\vec{#1}^{\,\prime}}
    \newcommand*{\svec}[1]{\vec{#1}\;\!}
    \newcommand*{\bm}[1]{\boldsymbol{\mathbf{#1}}}
    \newcommand*{\uv}[1]{\hat{\bm{#1}}}
    \newcommand*{\ang}[0]{\;\text{\AA}}
    \newcommand*{\mum}[0]{\;\upmu \mathrm{m}}
    \newcommand*{\at}[1]{\left.#1\right|}
    \newcommand*{\bra}[1]{\left<#1\right|}
    \newcommand*{\ket}[1]{\left|#1\right>}
    \newcommand*{\abs}[1]{\left|#1\right|}
    \newcommand*{\ev}[1]{\left\langle#1\right\rangle}
    \newcommand*{\p}[1]{\left(#1\right)}
    \newcommand*{\s}[1]{\left[#1\right]}
    \newcommand*{\z}[1]{\left\{#1\right\}}

    \newtheorem{theorem}{Theorem}[section]

    \let\Re\undefined
    \let\Im\undefined
    \DeclareMathOperator{\Res}{Res}
    \DeclareMathOperator{\Re}{Re}
    \DeclareMathOperator{\Im}{Im}
    \DeclareMathOperator{\Log}{Log}
    \DeclareMathOperator{\Arg}{Arg}
    \DeclareMathOperator{\Tr}{Tr}
    \DeclareMathOperator{\E}{E}
    \DeclareMathOperator{\Var}{Var}
    \DeclareMathOperator*{\argmin}{argmin}
    \DeclareMathOperator*{\argmax}{argmax}
    \DeclareMathOperator{\sgn}{sgn}
    \DeclareMathOperator{\diag}{diag\;}

    \colorlet{Corr}{red}

    % \everymath{\displaystyle} % biggify limits of inline sums and integrals
    \tikzstyle{circ} % usage: \node[circ, placement] (label) {text};
        = [draw, circle, fill=white, node distance=3cm, minimum height=2em]
    \definecolor{commentgreen}{rgb}{0,0.6,0}
    \lstset{
        basicstyle=\ttfamily\footnotesize,
        frame=single,
        numbers=left,
        showstringspaces=false,
        keywordstyle=\color{blue},
        stringstyle=\color{purple},
        commentstyle=\color{commentgreen},
        morecomment=[l][\color{magenta}]{\#}
    }

\begin{document}

\section{Mass Ratio Distribution}

I retried using $19$ values of fixed $q$, running each $1000$ times, instead of
sampling. I only finished two $a_{\rm out, eff}$ values. As a reminder:
\begin{align*}
    m_{12} &= 50M_{\odot}, &
    m_3 &= 30M_{\odot}, &
    a_0 &= 100\;\mathrm{AU}, \\
    e_0 &= 10^{-3},&
    e_{\rm out, 0} &\in [0, 0.9], &
    \cos I_0 &\in [\cos 50^\circ, \cos 130^\circ].
\end{align*}
The plots are shown in Fig.~\ref{fig:popsynth}. Recall that $e_{\rm os}$ is
defined such that
\begin{align}
    \ev{\rd{\ln a}{t}}_{\rm LK} &\sim \frac{1}{t_{\rm GW, 0}j^6(e_{\max})},\\
    j^6\p{e_{\rm os}} \equiv j_{\rm os}
        &= \frac{ t_{\rm LK}}{t_{\rm GW, 0}},\\
        &= \frac{256}{5}\frac{G^3 \mu m_{12}^3}{m_3c^5a^4n}
            \p{\frac{a_{\rm out, eff}}{a}}^3.
\end{align}
\begin{figure}[h]
    \centering
    \includegraphics[width=0.45\columnwidth]{../scripts/octlk/1popsynth/a2eff3600.png}
    \includegraphics[width=0.45\columnwidth]{../scripts/octlk/1popsynth/a2eff5500.png}
    \caption{Merger fraction distribution for $a_{\rm out, eff} = 3600,
    5500\;\mathrm{AU}$ respectively, where each $q$ has $1000$ trials. Error
    bars are just $\sqrt{N}$ for $N$ counts. Green dots denote predicted merger
    fractions if $j_{\min}$ ever dips below $3j(e_{\rm os})$ (I added a small
    fudge factor, it yields a small but not significant improvement) the
    one-shot merger criterion. Underprediction of merger rates is expected: once
    large eccentricities are reached, future coalescence is
    accelerated.}\label{fig:popsynth}
\end{figure}

\section{Example of Octupole-Enhanced Mergers}

For Bin's example, where $a_0 = 10\;\mathrm{AU}$ and $a_{\rm out, eff} =
300\;\mathrm{AU}$, I ran the $e_{\rm out} = 0.4$ case a long time ago. It shows
the octupole-enhanced case, see Fig.~\ref{fig:bindist}.
\begin{figure}
    \centering
    \includegraphics[width=0.5\columnwidth]{../scripts/octlk/1sweepbin/bindist.png}
    \caption{Example of octupole-enhanced mergers. Prograde orientations do not
    reach $e_{\rm os}$ (i.e.\ blue dots remain above horizontal blue line) but
    are still able to merge, as their $e_{\rm eff}$ (green
    dots go below green line).}\label{fig:bindist}
\end{figure}

\section{SRF-free $e_{\max}$ Plot}

We discussed the possible interest of this, turning off apsidal precession and
plotting $e_{\max}$. Very little changes, as seen in Fig.~\ref{fig:srf}.
\begin{figure}[h]
    \centering
    \includegraphics[width=0.45\columnwidth]{../scripts/octlk/1sweepbin_emax/1p4dist.png}
    \includegraphics[width=0.45\columnwidth]{../scripts/octlk/1sweepbin_emax/1p4dist_gr0.png}
    \caption{Two $e_{\max}$ distributions for $a_0 = 100\;\mathrm{AU}$, $a_{\rm
    out, eff} = 3600\;\mathrm{AU}$, $q = 0.4$ and $e_{\rm out} = 0.6$. The one
    on the left has SRF turned off. Very little changes}\label{fig:srf}
\end{figure}

\section{A Signature for the Gap}

I have some tentative evidence for why the gap exists. In brief, when
$\bm{L}_{\rm in}$ is librating, octupole-induced eccentricity cycles are
expected to be heavily suppressed, and only long periods of circulation generate
substantial eccentricity cycles. Integrating the octupole equations for just a
single period compared to their quadrupole counterparts, I found that when
moving to the gapped region, most ICs librate, and outside of the gapped region,
most circulate, e.g.\ see Fig.~\ref{fig:gap_maybe}. This is a general feature,
though it breaks down somewhat for $q = 0.2$. I may try with a few more
$\omega_1$ values at a later date to make this story more robust.
\begin{figure}
    \centering
    \includegraphics[width=0.5\columnwidth]{../scripts/octlk/2dW_sweeps/2_dWsweeps6_5_dual.png}
    \caption{Plot of $\Delta \Omega_{\rm e} \sim \Delta \varpi$, where $\Delta$
    indicates change over a period, so libration means $\Delta \Omega_{\rm e}
    \approx 180^\circ$. Top is quadrupole-only, bottom is with octupole.
    Vertical black line is the center of the ``gap''. The detailed parameters
    used are $a_0 = 100\;\mathrm{AU}$, $a_2 = 3600\;\mathrm{AU}$, $e_2 = 0.6$,
    and $q = 0.5$.}\label{fig:gap_maybe}
\end{figure}

Note that in the test mass, quadrupole limit, $\Delta \Omega = 180^\circ$ only
at $I_0 = 90^\circ$, so $\Delta \Omega_{\rm e} = 180^\circ$ at $I_0 = 90^\circ$
for the librating case. Thus, by integrating the quadrupole, finite-$\eta$
equations, it may be possible to predict where the gap is (or even obtain a
leading order expression).

\end{document}

