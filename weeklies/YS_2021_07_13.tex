    \documentclass[11pt,
        usenames, % allows access to some tikz colors
        dvipsnames % more colors: https://en.wikibooks.org/wiki/LaTeX/Colors
    ]{article}
    \usepackage{
        amsmath,
        amssymb,
        fouriernc, % fourier font w/ new century book
        fancyhdr, % page styling
        lastpage, % footer fanciness
        hyperref, % various links
        setspace, % line spacing
        amsthm, % newtheorem and proof environment
        mathtools, % \Aboxed for boxing inside aligns, among others
        float, % Allow [H] figure env alignment
        enumerate, % Allow custom enumerate numbering
        graphicx, % allow includegraphics with more filetypes
        wasysym, % \smiley!
        upgreek, % \upmu for \mum macro
        listings, % writing TrueType fonts and including code prettily
        tikz, % drawing things
        booktabs, % \bottomrule instead of hline apparently
        xcolor, % colored text
        cancel % can cancel things out!
    }
    \usepackage[margin=1in]{geometry} % page geometry
    \usepackage[
        labelfont=bf, % caption names are labeled in bold
        font=scriptsize % smaller font for captions
    ]{caption}
    \usepackage[font=scriptsize]{subcaption} % subfigures

    \newcommand*{\scinot}[2]{#1\times10^{#2}}
    \newcommand*{\dotp}[2]{\left<#1\,\middle|\,#2\right>}
    \newcommand*{\rd}[2]{\frac{\mathrm{d}#1}{\mathrm{d}#2}}
    \newcommand*{\pd}[2]{\frac{\partial#1}{\partial#2}}
    \newcommand*{\rdil}[2]{\mathrm{d}#1 / \mathrm{d}#2}
    \newcommand*{\pdil}[2]{\partial#1 / \partial#2}
    \newcommand*{\rtd}[2]{\frac{\mathrm{d}^2#1}{\mathrm{d}#2^2}}
    \newcommand*{\ptd}[2]{\frac{\partial^2 #1}{\partial#2^2}}
    \newcommand*{\md}[2]{\frac{\mathrm{D}#1}{\mathrm{D}#2}}
    \newcommand*{\pvec}[1]{\vec{#1}^{\,\prime}}
    \newcommand*{\svec}[1]{\vec{#1}\;\!}
    \newcommand*{\bm}[1]{\boldsymbol{\mathbf{#1}}}
    \newcommand*{\uv}[1]{\hat{\bm{#1}}}
    \newcommand*{\ang}[0]{\;\text{\AA}}
    \newcommand*{\mum}[0]{\;\upmu \mathrm{m}}
    \newcommand*{\at}[1]{\left.#1\right|}
    \newcommand*{\bra}[1]{\left<#1\right|}
    \newcommand*{\ket}[1]{\left|#1\right>}
    \newcommand*{\abs}[1]{\left|#1\right|}
    \newcommand*{\ev}[1]{\left\langle#1\right\rangle}
    \newcommand*{\p}[1]{\left(#1\right)}
    \newcommand*{\s}[1]{\left[#1\right]}
    \newcommand*{\z}[1]{\left\{#1\right\}}

    \newtheorem{theorem}{Theorem}[section]

    \let\Re\undefined
    \let\Im\undefined
    \DeclareMathOperator{\Res}{Res}
    \DeclareMathOperator{\Re}{Re}
    \DeclareMathOperator{\Im}{Im}
    \DeclareMathOperator{\Log}{Log}
    \DeclareMathOperator{\Arg}{Arg}
    \DeclareMathOperator{\Tr}{Tr}
    \DeclareMathOperator{\E}{E}
    \DeclareMathOperator{\Var}{Var}
    \DeclareMathOperator*{\argmin}{argmin}
    \DeclareMathOperator*{\argmax}{argmax}
    \DeclareMathOperator{\sgn}{sgn}
    \DeclareMathOperator{\diag}{diag\;}

    \colorlet{Corr}{red}

    % \everymath{\displaystyle} % biggify limits of inline sums and integrals
    \tikzstyle{circ} % usage: \node[circ, placement] (label) {text};
        = [draw, circle, fill=white, node distance=3cm, minimum height=2em]
    \definecolor{commentgreen}{rgb}{0,0.6,0}
    \lstset{
        basicstyle=\ttfamily\footnotesize,
        frame=single,
        numbers=left,
        showstringspaces=false,
        keywordstyle=\color{blue},
        stringstyle=\color{purple},
        commentstyle=\color{commentgreen},
        morecomment=[l][\color{magenta}]{\#}
    }

\begin{document}

\section{Analytical Results}

We consider the EOM in the ``co-rotating'' frame where $\uv{l} = \uv{z}$ is
stationary and $\uv{J}$ is nutating, fixed in the $\uv{x}$-$\uv{z}$ plane. Then:
\begin{equation}
    \p{\rd{\uv{s}}{t}}_{\rm rot} = \alpha\p{\uv{s} \cdot \uv{l}} \p{\uv{s}
        \times \uv{l}} - \uv{s} \times \p{\dot{\Omega} \uv{J} + \dot{I}\uv{y}},
\end{equation}
where
\begin{align}
    \uv{J} &= \cos I \uv{z} - \sin I \uv{x},\\
    \dot{I} &\approx -I_2 \Delta g \sin\p{\Delta g t},\\
    \dot{\Omega} &\approx g_1 + \Delta g \frac{I_2}{I_1}\cos\p{\Delta g t}.
\end{align}

However, we seek an equilibrium with the resonant angle $\phi_{\rm res} =
\phi_{\rm sl} + \p{\Delta g / 2}t$. Thus, we perform another rotation into the
$\uv{x}'$-$\uv{y}'$-$\uv{z}$ frame (where $\uv{x} = \cos\p{\Delta g t /
2}\uv{x}' - \sin\p{\Delta g t / 2}\uv{y}'$ and $\uv{y} = \sin\p{\Delta g t /
2}\uv{x}' + \cos\p{\Delta g t / 2}\uv{y}'$), then
{\small
\begin{align}
    \p{\rd{\uv{s}}{t}}_{\rm res} &= \alpha\p{\uv{s} \cdot \uv{l}} \p{\uv{s}
            \times \uv{l}} - \uv{s} \times \p{\dot{\Omega} \uv{J} +
            \dot{I}\uv{y}} - \uv{s} \times \frac{\Delta g}{2}\uv{l},\\
        &\approx \alpha\p{\uv{s} \cdot \uv{l}} \p{\uv{s} \times \uv{l}}
            - \uv{s} \times \s{
                g_1\p{\uv{z} - \sin I_1 \uv{x}}
                + \Delta g \frac{I_2}{I_1}\cos\p{\Delta g t}\uv{z}
                - \Delta g I_2\cos\p{\Delta g t}\uv{x}
                - I_2 \Delta g \sin \p{\Delta g t}\uv{y}
                + \frac{\Delta g}{2}\uv{z}}.
\end{align}}

Now, we consider expansion in successive orders of $I_2$, so we decompose
$\uv{s} = \bm{s}_0 + \bm{s}_1 + \dots$ where $\bm{s}_k \sim \mathcal{O}(I_2^k)$.
Then, suppressing the ``res'' subscript:
\begin{align}
    \rd{\bm{s}_0}{t} ={}&
        \alpha\p{\bm{s}_0 \cdot \uv{z}}\p{\bm{s}_0 \times \uv{z}}
        - \bm{s}_0\times \p{\frac{g_1 + g_2}{2}}\uv{z}\label{eq:one},\\
    \rd{\bm{s}_1}{t} ={}&
            \alpha\p{\bm{s}_1 \cdot \uv{z}}\p{\bm{s}_0 \times \uv{z}}
            + \cancel{\alpha\p{\bm{s}_0 \cdot \uv{z}}\p{\bm{s}_1 \times \uv{z}}
            - \bm{s}_1 \times \p{\frac{g_1 + g_2}{2}\uv{z}}}\nonumber\\
        &- \bm{s}_0 \times \Bigg[-\cancelto{?}{g_1 \sin I_1 \uv{x}}
            + \Delta g \frac{I_2}{I_1}\cos\p{\Delta g t}\uv{z}
            - I_2 \Delta g\s{\sin \p{\Delta g t}\uv{y}
                + \cos\p{\Delta g t}\uv{x}}
        \Bigg],\label{eq:2}\\
        ={}&
            \alpha\p{\bm{s}_1 \cdot \uv{z}}\s{\bm{s}_0 \times \uv{z}}
        - \bm{s}_0 \times \Bigg\{\cancel{-g_1 \sin I_1 \p{
            \cos\p{\frac{\Delta g t}{2}}\uv{x}'
            + \sin\p{\frac{\Delta g t}{2}}\uv{y}'}}\nonumber\\
        &+ \Delta g \frac{I_2}{I_1}\cos\p{\Delta g t}\uv{z}
            + I_2 \Delta g\s{-\sin \p{\frac{\Delta g t}{2}}\uv{y}'
                + \cos\p{\frac{\Delta g t}{2}}\uv{x}'}\Bigg\}.\label{eq:3}
\end{align}
Eq.~\eqref{eq:one} reduces to the expression we already know,
\begin{equation}
    \bm{s}_0 \cdot \uv{z} = \frac{g_1 + g_2}{2\alpha}.\label{eq:q_res}
\end{equation}
At the next order, we can cancel one of the terms in Eq.~\eqref{eq:2} using
Eq.~\eqref{eq:one} (shown), arriving at Eq.~\eqref{eq:3}. If we ignore the two
$\bm{s}_0 \times \uv{z}$ terms (they will act purely in the $\phi$ direction and
are at the $\Delta g$ frequency, so may be responsible for the $\Delta g$
harmonic in the $\phi(t)$ graph), then we arrive at
\begin{align}
    \rd{\bm{s}_1}{t} &=
        \bm{s}_0 \times \Bigg\{I_2 \Delta g\s{\sin \p{\frac{\Delta g
            t}{2}}\uv{y}' - \cos\p{\frac{\Delta g t}{2}}\uv{x}'}\Bigg\}.
\end{align}
Thus, if there is no transient behavior (due to the alignment torque), we find
that the oscillation amplitude in $\theta$ should be given by
\begin{align}
    \Delta \theta &\sim 2I_0 \sin \theta_0. \label{eq:dq_amp}
\end{align}
\textbf{NB:} I had to ignore the $-g_1$ term to make this work; perhaps it is
because it isn't an $I_2$-dependent term? It may be luck, or maybe this is the
correct calculation.

How well does this do? We consult Fig.~\ref{fig:zooms}. The agreement seems
qualitatively correct, though there is yet still work to be done.
\begin{figure}
    \centering
    \includegraphics[width=0.4\columnwidth]{../../attractors/initial/4nplanet/3paramtide/disp_7_reszoom.png}
    \includegraphics[width=0.4\columnwidth]{../../attractors/initial/4nplanet/3paramtide/disp_10_reszoom.png}
    \includegraphics[width=0.4\columnwidth]{../../attractors/initial/4nplanet/3paramtide/disp_15_reszoom.png}
    \includegraphics[width=0.4\columnwidth]{../../attractors/initial/4nplanet/3paramtide/disp_10_res2zoom.png}
    \includegraphics[width=0.4\columnwidth]{../../attractors/initial/4nplanet/3paramtide/disp_10_res4zoom.png}
    \caption{Some examples of zoomed-in evolution for (unless otherwise noted,
    $\alpha = 10g_1$ and $I = 1^\circ$): (i) $g_2 = 7g_1$, (ii)
    $g_2 = 10g_1$, (iii) $g_2 = 15g_1$, (iv) $\alpha = 15g_1$, (v) $I_2 =
    1.5^\circ$. Black line is Eq.~\eqref{eq:q_res} and red line is
    Eq.~\eqref{eq:dq_amp}.}\label{fig:zooms}
\end{figure}

\section{Numerics}

Numerical outcomes for $I = 3^\circ$ and a variety of $g_2$ values are shown in
Fig.~\ref{fig:outcomes}. Visible are: (i) the decrease in M1-CS2 (mode 1, CS2)
as $\eta_2$ is increased from zero, since chaos begins to eject trajectories
near the edge of the separatrix, and (ii) the appearance of a substantial mixed
mode for $\eta_2 > 1.0$.

Perhaps further simulations would be interesting to explore the regime $\eta_2
\lesssim 1$.
\begin{figure}
    \centering
    \includegraphics[width=0.8\columnwidth]{../../attractors/initial/4nplanet/3outcomes.png}
    \caption{Probability of reaching various CSs for each mode as a function of
    $\eta_2$, with inclinations labeled. $\alpha = 10g_1$.}\label{fig:outcomes}
\end{figure}

\end{document}

