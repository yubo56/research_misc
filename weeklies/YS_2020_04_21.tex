% single document keeping track of all weekly notes sent to Dong
    \documentclass[11pt,
        usenames, % allows access to some tikz colors
        dvipsnames, % more colors: https://en.wikibooks.org/wiki/LaTeX/Colors
        twocolumn,
    ]{article}
    \usepackage{
        amsmath,
        amssymb,
        fouriernc, % fourier font w/ new century book
        fancyhdr, % page styling
        lastpage, % footer fanciness
        hyperref, % various links
        setspace, % line spacing
        amsthm, % newtheorem and proof environment
        mathtools, % \Aboxed for boxing inside aligns, among others
        float, % Allow [H] figure env alignment
        enumerate, % Allow custom enumerate numbering
        graphicx, % allow includegraphics with more filetypes
        wasysym, % \smiley!
        upgreek, % \upmu for \mum macro
        listings, % writing TrueType fonts and including code prettily
        tikz, % drawing things
        booktabs, % \bottomrule instead of hline apparently
        cancel % can cancel things out!
    }
    \usepackage[margin=0.5in]{geometry} % page geometry
    \usepackage[
        labelfont=bf, % caption names are labeled in bold
        font=scriptsize % smaller font for captions
    ]{caption}
    \usepackage[font=scriptsize]{subcaption} % subfigures

    \newcommand*{\scinot}[2]{#1\times10^{#2}}
    \newcommand*{\dotp}[2]{\left<#1\,\middle|\,#2\right>}
    \newcommand*{\rd}[2]{\frac{\mathrm{d}#1}{\mathrm{d}#2}}
    \newcommand*{\pd}[2]{\frac{\partial#1}{\partial#2}}
    \newcommand*{\rtd}[2]{\frac{\mathrm{d}^2#1}{\mathrm{d}#2^2}}
    \newcommand*{\ptd}[2]{\frac{\partial^2 #1}{\partial#2^2}}
    \newcommand*{\md}[2]{\frac{\mathrm{D}#1}{\mathrm{D}#2}}
    \newcommand*{\pvec}[1]{\vec{#1}^{\,\prime}}
    \newcommand*{\svec}[1]{\vec{#1}\;\!}
    \newcommand*{\bm}[1]{\boldsymbol{\mathbf{#1}}}
    \newcommand*{\uv}[1]{\hat{\bm{#1}}}
    \newcommand*{\ang}[0]{\;\text{\AA}}
    \newcommand*{\mum}[0]{\;\upmu \mathrm{m}}
    \newcommand*{\at}[1]{\left.#1\right|}
    \newcommand*{\bra}[1]{\left<#1\right|}
    \newcommand*{\ket}[1]{\left|#1\right>}
    \newcommand*{\abs}[1]{\left|#1\right|}
    \newcommand*{\ev}[1]{\langle#1\rangle}
    \newcommand*{\p}[1]{\left(#1\right)}
    \newcommand*{\s}[1]{\left[#1\right]}
    \newcommand*{\z}[1]{\left\{#1\right\}}

    \newtheorem{theorem}{Theorem}[section]

    \let\Re\undefined
    \let\Im\undefined
    \DeclareMathOperator{\Res}{Res}
    \DeclareMathOperator{\Re}{Re}
    \DeclareMathOperator{\Im}{Im}
    \DeclareMathOperator{\Log}{Log}
    \DeclareMathOperator{\Arg}{Arg}
    \DeclareMathOperator{\Tr}{Tr}
    \DeclareMathOperator{\E}{E}
    \DeclareMathOperator{\Var}{Var}
    \DeclareMathOperator*{\argmin}{argmin}
    \DeclareMathOperator*{\argmax}{argmax}
    \DeclareMathOperator{\sgn}{sgn}
    \DeclareMathOperator{\diag}{diag\;}

    % \everymath{\displaystyle} % biggify limits of inline sums and integrals
    \tikzstyle{circ} %u sage: \cnode[irc, placement] (label) {text};
        = [draw, circle, fill=white, node distance=3cm, minimum height=2em]
    \definecolor{commentgreen}{rgb}{0,0.6,0}
    \lstset{
        basicstyle=\ttfamily\footnotesize,
        frame=single,
        numbers=left,
        showstringspaces=false,
        keywordstyle=\color{blue},
        stringstyle=\color{purple},
        commentstyle=\color{commentgreen},
        morecomment=[l][\color{magenta}]{\#}
    }

\begin{document}

\def\Snospace~{\S{}} % hack to remove the space left after autorefs
\renewcommand*{\sectionautorefname}{\Snospace}
\renewcommand*{\appendixautorefname}{\Snospace}
\renewcommand*{\figureautorefname}{Fig.}
\renewcommand*{\equationautorefname}{Eq.}
\renewcommand*{\tableautorefname}{Tab.}

\singlespacing

\section{Apr 21, 2020}

For the $90^\circ$ attractor in the LK problem, investigated the $N = 0$
dynamics. Recall EOM
\begin{align}
    \rd{\uv{S}}{t} ={}& \ev{-\dot{\Omega}\uv{z} + \Omega_{\rm SL}\uv{L}}_{\rm LK}
        \times \uv{S}\nonumber\\
        &+ \s{
            \sum\limits_{N = 1}^\infty \uv{\Omega}_{\rm eff, N}
            \exp\p{2\pi i N t / t_{\rm LK}}} \times \uv{S}.
\end{align}

We ignore $N \geq 1$ for now, assuming resonances are not hit, so we examine
\begin{equation}
    \rd{\uv{S}}{t} = \ev{-\dot{\Omega}\uv{z} + \Omega_{\rm SL}\uv{L}}_{\rm LK}
        \times \uv{S}.
\end{equation}

Consider rotation by $I_{\rm out}$ given by Figure~\ref{fig:geometry}.
\begin{figure}[h]
    \centering
    \begin{tikzpicture}
        \draw[->] (0, 0)--(1, 1);
        \draw[->] (0, 0)--(-1.3, 0.3);
        \draw[->] (0, 0) -- (0, 1.414);
        \node[right] at (1, 1) {$\uv{L}_{\rm out}$};
        \node[left] at (-1.3, 0.3) {$\uv{L}$};
        \node[above] at (0, 1.414) {$\Omega_{\rm eff}$};
        \node[right] at (0, 1.0) {$I_{\rm out}$};
    \end{tikzpicture}
    \caption{Geometry. $I$ is angle between $\uv{L}, \uv{L}_{\rm out}$ while
    $I_{\rm out}$ is angle between $\uv{L}_{\rm out}, \Omega_{\rm eff}$.
    }\label{fig:geometry}
\end{figure}
In equations, this requires
\begin{equation}
    -\dot{\Omega} \sin I_{\rm out} + \Omega_{SL}\sin \p{I + I_{\rm out}} = 0.
\end{equation}

We obtain EOM (note that LK-averaged $I$ is almost constant):
\begin{align}
    \rd{\uv{S}}{t} ={}& \s{-\dot{\Omega}\cos I_{\rm out}
        + \Omega_{\rm SL}\cos \p{I + I_{\rm out}}}\uv{z} \times
        \uv{S}\nonumber\\
        &- \dot{I}_{\rm out} \uv{y} \times \uv{S},\\
    \Delta \phi(t) ={}& \int\limits_0^t \s{-\dot{\Omega}\cos I_{\rm out}(\tau)
        + \Omega_{\rm SL}\cos \p{I + I_{\rm out}(\tau)}}
            \;\mathrm{d}\tau,\\
    \Delta \theta ={}& \int\limits_0^{T_{\rm m}}
        \dot{I}_{\rm out} \sin \phi\;\mathrm{d}t.
\end{align}
Intuition: $\dot{I}_{\rm out} \sim 1 / T_{\rm GW}$ while $\dot{\phi} \sim
\min\p{\dot{\Omega}, \Omega_{\rm SL}}$. Thus, when sufficiently slow, phases
cancel, $\Delta \theta \to 0$. \textbf{NB:} in full numerical simulations,
$\ev{\theta}_{\rm LK}$ is the approximately conserved quantity, $\theta$ varies
within LK cycles.

How does this do? Figure~\ref{fig:plots}.
\begin{figure}[h]
    \centering
    \includegraphics[width=\columnwidth]{plots/4sim_90_300_qN0.png}

    \includegraphics[width=\columnwidth]{plots/4sim_90_300_Iout.png}

    \includegraphics[width=\columnwidth]{plots/4sim_90_300_phidots.png}
    \caption{$I = 90.3^\circ$. Top: Comparison of $\theta$ (black) vs
    $\theta_{\rm eff}$ from Bin's paper (red). Middle: $I_{\rm out}$ is
    red, $\dot{I}_{\rm out}$ is light black. Bottom: Comparison of LK-averaged
    $\dot{\Omega}$ (red), $\Omega_{\rm SL}$ (green), and $\ev{\dot{\Omega} +
    \Omega_{\rm SL}}_{\rm LK}$ (black).}\label{fig:plots}
\end{figure}

\clearpage

How well does this do for our ensemble? See Figure~\ref{fig:deviations}.
Blue is a $\propto \cos^{-9}(I_0)$ slope, dashed green shows the semimajor axis
at the end of the Kozai cycle we average over (recall we have to average over a
Kozai cycle to determine the initial $\theta$).
\begin{figure}[h]
    \centering
    \includegraphics[width=\columnwidth]{plots/deviations_one.png}
    \caption{Deviations from exact conservation of
    $\theta$.}\label{fig:deviations}
\end{figure}

This seems to confirm with expectation: we have calculations that show
\begin{equation}
    \dot{I}_{\rm out}\p{-\cot I_{\rm out} + \cot \p{I + I_{\rm out}}}
        = \rd{}{t}\ln \p{-\dot{\Omega} / \Omega_{\rm SL}}.
\end{equation}
The RHS of the above $\propto 1 / T_{\rm m} \propto \cos^{-6}I_0$, which is the
peak of $\dot{I}_{\rm out}$. Therefore, the width of $\dot{I}_{\rm out}$ is
$\propto I T_{\rm m}$. Phases cancel better then when $I_{\rm out}$ is broader,
so a scaling stronger than $-6$ seems understandable.

\end{document}

