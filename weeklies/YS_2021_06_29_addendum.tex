    \documentclass[11pt,
        usenames, % allows access to some tikz colors
        dvipsnames % more colors: https://en.wikibooks.org/wiki/LaTeX/Colors
    ]{article}
    \usepackage{
        amsmath,
        amssymb,
        fouriernc, % fourier font w/ new century book
        fancyhdr, % page styling
        lastpage, % footer fanciness
        hyperref, % various links
        setspace, % line spacing
        amsthm, % newtheorem and proof environment
        mathtools, % \Aboxed for boxing inside aligns, among others
        float, % Allow [H] figure env alignment
        enumerate, % Allow custom enumerate numbering
        graphicx, % allow includegraphics with more filetypes
        wasysym, % \smiley!
        upgreek, % \upmu for \mum macro
        listings, % writing TrueType fonts and including code prettily
        tikz, % drawing things
        booktabs, % \bottomrule instead of hline apparently
        xcolor, % colored text
        cancel % can cancel things out!
    }
    \usepackage[margin=1in]{geometry} % page geometry
    \usepackage[
        labelfont=bf, % caption names are labeled in bold
        font=scriptsize % smaller font for captions
    ]{caption}
    \usepackage[font=scriptsize]{subcaption} % subfigures

    \newcommand*{\scinot}[2]{#1\times10^{#2}}
    \newcommand*{\dotp}[2]{\left<#1\,\middle|\,#2\right>}
    \newcommand*{\rd}[2]{\frac{\mathrm{d}#1}{\mathrm{d}#2}}
    \newcommand*{\pd}[2]{\frac{\partial#1}{\partial#2}}
    \newcommand*{\rdil}[2]{\mathrm{d}#1 / \mathrm{d}#2}
    \newcommand*{\pdil}[2]{\partial#1 / \partial#2}
    \newcommand*{\rtd}[2]{\frac{\mathrm{d}^2#1}{\mathrm{d}#2^2}}
    \newcommand*{\ptd}[2]{\frac{\partial^2 #1}{\partial#2^2}}
    \newcommand*{\md}[2]{\frac{\mathrm{D}#1}{\mathrm{D}#2}}
    \newcommand*{\pvec}[1]{\vec{#1}^{\,\prime}}
    \newcommand*{\svec}[1]{\vec{#1}\;\!}
    \newcommand*{\bm}[1]{\boldsymbol{\mathbf{#1}}}
    \newcommand*{\uv}[1]{\hat{\bm{#1}}}
    \newcommand*{\ang}[0]{\;\text{\AA}}
    \newcommand*{\mum}[0]{\;\upmu \mathrm{m}}
    \newcommand*{\at}[1]{\left.#1\right|}
    \newcommand*{\bra}[1]{\left<#1\right|}
    \newcommand*{\ket}[1]{\left|#1\right>}
    \newcommand*{\abs}[1]{\left|#1\right|}
    \newcommand*{\ev}[1]{\left\langle#1\right\rangle}
    \newcommand*{\p}[1]{\left(#1\right)}
    \newcommand*{\s}[1]{\left[#1\right]}
    \newcommand*{\z}[1]{\left\{#1\right\}}

    \newtheorem{theorem}{Theorem}[section]

    \let\Re\undefined
    \let\Im\undefined
    \DeclareMathOperator{\Res}{Res}
    \DeclareMathOperator{\Re}{Re}
    \DeclareMathOperator{\Im}{Im}
    \DeclareMathOperator{\Log}{Log}
    \DeclareMathOperator{\Arg}{Arg}
    \DeclareMathOperator{\Tr}{Tr}
    \DeclareMathOperator{\E}{E}
    \DeclareMathOperator{\Var}{Var}
    \DeclareMathOperator*{\argmin}{argmin}
    \DeclareMathOperator*{\argmax}{argmax}
    \DeclareMathOperator{\sgn}{sgn}
    \DeclareMathOperator{\diag}{diag\;}

    \colorlet{Corr}{red}

    % \everymath{\displaystyle} % biggify limits of inline sums and integrals
    \tikzstyle{circ} % usage: \node[circ, placement] (label) {text};
        = [draw, circle, fill=white, node distance=3cm, minimum height=2em]
    \definecolor{commentgreen}{rgb}{0,0.6,0}
    \lstset{
        basicstyle=\ttfamily\footnotesize,
        frame=single,
        numbers=left,
        showstringspaces=false,
        keywordstyle=\color{blue},
        stringstyle=\color{purple},
        commentstyle=\color{commentgreen},
        morecomment=[l][\color{magenta}]{\#}
    }

\begin{document}

I show four simulations in phase space coordinates ($\theta_{\rm sl}$,
$\phi_{\rm sl}$) showing the resonant angles that we see, with an alignment
timescale $g_1t_{\rm al} = 2500$. Here, $\cos \theta_{\rm sl} = \uv{s} \cdot
\uv{l}$ while $\phi_{\rm sl}$ is defined relative to the $\uv{l}$--$\uv{l}_{\rm
p}$ plane. For all of these, $I_1 = 10^\circ$, $I_2 = 1^\circ$, and $g_2 =
10g_1$.
\begin{itemize}
    \item In Fig~\ref{fig:cs1}, we show a case where $\alpha = 10g_1$ (so
        $\eta_1 = 0.1$ and $\eta_1 = 1$). We start with $\theta_{\rm sl, i} =
        10^\circ$. The system librates with resonant angle $\phi_{\rm sl} +
        \Delta g t$, where $\Delta g = g_2 - g_1 = 9g_1$. The obliquity
        oscillates about the mode II CS2 obliquity in the limit where $I_1 =
        0^\circ$, the horizontal line (i.e.\ the CS2 obliquity for $\eta =
        \eta_2 = 1$).

    \item In Fig~\ref{fig:cs2}, we use the same parameters as for
        Fig.~\ref{fig:cs1} but starting with $\theta_{\rm sl, i} = 90^\circ$.
        The system librates with resonant angle $\phi_{\rm sl}$. The obliquity
        oscillates about the mode I CS2 obliquity in the limit where $I_2 =
        0^\circ$, the horizontal line (i.e.\ the CS2 obliquity for $\eta =
        \eta_1 = 0.1$).

    \item In Fig~\ref{fig:res}, we again use the same parameters but start with
        $\theta_{\rm sl, i} = 55^\circ$. The system librates with resonant angle
        $\phi_{\rm sl} + \Delta g t / 2$. Furthermore, the oscillation frequency
        in $\theta_{\rm sl}$ is exactly twice that of the oscillation frequency
        in $I$ (and we showed that the oscillation frequency of $I$ is $\Delta
        g$). The obliquity oscillates about
        \begin{equation}
            \theta_{\rm sl, res} \equiv \frac{g_1 + g_2}{2\alpha}\label{eq:qslres},
        \end{equation}
        shown in the horizontal line.

        Recall that the argument for this was very simplistic: assuming $\uv{l}
        \approx \uv{l}_{\rm p}$ (i.e.\ to zeroth order in $I_1$ and $I_2$), then
        we can work in the inertial frame where $\uv{l} = \uv{l}_{\rm p} =
        \uv{z}$, and $\phi_{\rm sl} \approx \phi_{\rm inertial} + g_1t$. Then,
        if we go to the co-rotating frame where $\phi_{\rm sl} + \Delta g t / 2
        = \phi_{\rm inertial} + \p{g_1 + g_2}t / 2$ is librating, then:
        \begin{align}
            \p{\rd{\uv{s}}{t}}_{\rm rot}
                = 0 &\approx \alpha\p{\uv{s} \cdot \uv{z}}\p{\uv{s} \times \uv{z}}
                    - \frac{g_1 + g_2}{2}\p{\uv{s} \times \uv{z}},\\
                &= \s{\alpha\p{\uv{s} \cdot \uv{z}}
                    - \frac{g_1 + g_2}{2}}\p{\uv{s} \times \uv{z}}.
        \end{align}
        Thus, $\uv{s} \cdot \uv{z} = (g_1 + g_2) / (2\alpha)$ is an equilibrium.
        Again, this is accurate to zeroth order in $I_1$ and $I_2$.

    \item In Fig~\ref{fig:res2}, we show a case where $\alpha = 15g_1$ and
        starting with $\theta_{\rm sl, i} = 68^\circ$. The system again librates
        with resonant angle $\phi_{\rm sl} + \Delta g t / 2$. The obliquity
        still oscillates about $\theta_{\rm sl, res} \equiv \p{g_1 + g_2} /
        (2\alpha)$ (horizontal line).
\end{itemize}

According to my suspicion/intuition, this new equilibrium is a parametric-type
resonance and is unlikely to have the same phase space structure as a Cassini
State (so Natalia's formalism may not work, since the character of the resonance
is different). Some analytical insight can be obtained by considering the
equation that Dong wrote down, in the ``co-rotating'' frame where $\uv{l} =
\uv{z}$ is fixed and $\uv{l}_{\rm p}$ is only allowed to nutate:
\begin{equation}
    \p{\rd{\uv{s}}{t}}_{\rm rot} =
        \alpha \p{\uv{s} \cdot \uv{z}}\p{\uv{s} \times \uv{z}}
            - \uv{R} \times \uv{s},
\end{equation}
where $\uv{R}$ is the ``rotation'' matrix with that has a time dependency with
frequency $\Delta g$. This is clearly a parametrically driven, nonlinear system.
But maybe if the periodic component of $\uv{R}$ is treated as a small
perturbation, the leading order time-varying component of $\uv{s}$ may be able
to be solved for, like:
\begin{align}
    \uv{s}(t) &= \bm{s}_0 + \bm{s}_1\cos\p{\frac{\Delta g}{2}t} + \dots,\\
    \uv{R} &= \bm{R}_0 + \bm{R}_1\cos \p{\Delta gt} + \dots\\
    \text{At the frequency} \cos\p{ \frac{\Delta g}{2}t}:\quad 0 &=
        \alpha \p{\uv{s}_1 \cdot \uv{z}}\p{\uv{s}_0 \times \uv{z}}
        + \alpha \p{\uv{s}_0 \cdot \uv{z}}\p{\uv{s}_1 \times \uv{z}}
            - \frac{\uv{R}_1}{2} \times \uv{s}_1.
\end{align}

\begin{figure}
    \centering
    \includegraphics[width=0.4\columnwidth]{../../attractors/initial/4nplanet/3paramtide/disp_10_cs1.png}
    \includegraphics[width=0.4\columnwidth]{../../attractors/initial/4nplanet/3paramtide/disp_10_cs1zoom.png}
    \caption{Plot of the evolution of an initially low-obliquity
    system. Horizontal line is CS2 of mode 2.}\label{fig:cs1}
\end{figure}
\begin{figure}
    \centering
    \includegraphics[width=0.4\columnwidth]{../../attractors/initial/4nplanet/3paramtide/disp_10_cs2.png}
    \includegraphics[width=0.4\columnwidth]{../../attractors/initial/4nplanet/3paramtide/disp_10_cs2zoom.png}
    \caption{Plot of the evolution of an initially high-obliquity
    system. Horizontal line is CS2 of mode 1.}\label{fig:cs2}
\end{figure}
\begin{figure}
    \centering
    \includegraphics[width=0.4\columnwidth]{../../attractors/initial/4nplanet/3paramtide/disp_10_res.png}
    \includegraphics[width=0.4\columnwidth]{../../attractors/initial/4nplanet/3paramtide/disp_10_reszoom.png}
    \caption{Plot of the evolution of an initially intermediate-obliquity
    system. Horizontal line is Eq.~\eqref{eq:qslres}.}\label{fig:res}
\end{figure}
\begin{figure}
    \centering
    \includegraphics[width=0.4\columnwidth]{../../attractors/initial/4nplanet/3paramtide/disp_10_res2.png}
    \includegraphics[width=0.4\columnwidth]{../../attractors/initial/4nplanet/3paramtide/disp_10_res2zoom.png}
    \caption{Same as Fig.~\ref{fig:res} but for $\alpha =
    15g_1$. Horizontal line is Eq.~\eqref{eq:qslres}.}\label{fig:res2}
\end{figure}

\end{document}

