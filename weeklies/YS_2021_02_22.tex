    \documentclass[11pt,
        usenames, % allows access to some tikz colors
        dvipsnames % more colors: https://en.wikibooks.org/wiki/LaTeX/Colors
    ]{article}
    \usepackage{
        amsmath,
        amssymb,
        fouriernc, % fourier font w/ new century book
        fancyhdr, % page styling
        lastpage, % footer fanciness
        hyperref, % various links
        setspace, % line spacing
        amsthm, % newtheorem and proof environment
        mathtools, % \Aboxed for boxing inside aligns, among others
        float, % Allow [H] figure env alignment
        enumerate, % Allow custom enumerate numbering
        graphicx, % allow includegraphics with more filetypes
        wasysym, % \smiley!
        upgreek, % \upmu for \mum macro
        listings, % writing TrueType fonts and including code prettily
        tikz, % drawing things
        booktabs, % \bottomrule instead of hline apparently
        xcolor, % colored text
        cancel % can cancel things out!
    }
    \usepackage[margin=1in]{geometry} % page geometry
    \usepackage[
        labelfont=bf, % caption names are labeled in bold
        font=scriptsize % smaller font for captions
    ]{caption}
    \usepackage[font=scriptsize]{subcaption} % subfigures

    \newcommand*{\scinot}[2]{#1\times10^{#2}}
    \newcommand*{\dotp}[2]{\left<#1\,\middle|\,#2\right>}
    \newcommand*{\rd}[2]{\frac{\mathrm{d}#1}{\mathrm{d}#2}}
    \newcommand*{\pd}[2]{\frac{\partial#1}{\partial#2}}
    \newcommand*{\rdil}[2]{\mathrm{d}#1 / \mathrm{d}#2}
    \newcommand*{\pdil}[2]{\partial#1 / \partial#2}
    \newcommand*{\rtd}[2]{\frac{\mathrm{d}^2#1}{\mathrm{d}#2^2}}
    \newcommand*{\ptd}[2]{\frac{\partial^2 #1}{\partial#2^2}}
    \newcommand*{\md}[2]{\frac{\mathrm{D}#1}{\mathrm{D}#2}}
    \newcommand*{\pvec}[1]{\vec{#1}^{\,\prime}}
    \newcommand*{\svec}[1]{\vec{#1}\;\!}
    \newcommand*{\bm}[1]{\boldsymbol{\mathbf{#1}}}
    \newcommand*{\uv}[1]{\hat{\bm{#1}}}
    \newcommand*{\ang}[0]{\;\text{\AA}}
    \newcommand*{\mum}[0]{\;\upmu \mathrm{m}}
    \newcommand*{\at}[1]{\left.#1\right|}
    \newcommand*{\bra}[1]{\left<#1\right|}
    \newcommand*{\ket}[1]{\left|#1\right>}
    \newcommand*{\abs}[1]{\left|#1\right|}
    \newcommand*{\ev}[1]{\left\langle#1\right\rangle}
    \newcommand*{\p}[1]{\left(#1\right)}
    \newcommand*{\s}[1]{\left[#1\right]}
    \newcommand*{\z}[1]{\left\{#1\right\}}

    \newtheorem{theorem}{Theorem}[section]

    \let\Re\undefined
    \let\Im\undefined
    \DeclareMathOperator{\Res}{Res}
    \DeclareMathOperator{\Re}{Re}
    \DeclareMathOperator{\Im}{Im}
    \DeclareMathOperator{\Log}{Log}
    \DeclareMathOperator{\Arg}{Arg}
    \DeclareMathOperator{\Tr}{Tr}
    \DeclareMathOperator{\E}{E}
    \DeclareMathOperator{\Var}{Var}
    \DeclareMathOperator*{\argmin}{argmin}
    \DeclareMathOperator*{\argmax}{argmax}
    \DeclareMathOperator{\sgn}{sgn}
    \DeclareMathOperator{\diag}{diag\;}

    \colorlet{Corr}{red}

    % \everymath{\displaystyle} % biggify limits of inline sums and integrals
    \tikzstyle{circ} % usage: \node[circ, placement] (label) {text};
        = [draw, circle, fill=white, node distance=3cm, minimum height=2em]
    \definecolor{commentgreen}{rgb}{0,0.6,0}
    \lstset{
        basicstyle=\ttfamily\footnotesize,
        frame=single,
        numbers=left,
        showstringspaces=false,
        keywordstyle=\color{blue},
        stringstyle=\color{purple},
        commentstyle=\color{commentgreen},
        morecomment=[l][\color{magenta}]{\#}
    }

\begin{document}

We want to answer what the primordial BH $q$ distribution is in a few simplified
cases if:
\begin{itemize}
    \item The ZAMS masses are randomly drawn Salpeter IMF $P(M) \propto
        M^{-2.35}$, then go supernova following
        \url{https://ui.adsabs.harvard.edu/abs/2017MNRAS.470.4739S/abstract}
        (bounded by large/small Z)

    \item The ZAMS mass ratio is uniform.

    \item The ZAMS mass ratio is uniform in $\log q$.
\end{itemize}

For reference, the supernova mass transfer function is shown in
Fig.~\ref{fig:sne}
\begin{figure}[t]
    \centering
    \includegraphics[width=0.5\columnwidth]{../scripts/octlk/3plots/sne.png}
    \caption{SNe mass transfer function}\label{fig:sne}
\end{figure}

\section{Corrections to Appendix A}

I found Appendix A is wrong: $P(q) \propto q^{-p}$ using the convention $q \geq
1$, but not in our convention! See Fig.~\ref{fig:1}. To draw the distributions,
I use either
\begin{equation}
    q = \min\p{\frac{m_2}{m_1}, \frac{m_1}{m_2}} \leq 1,
\end{equation}
or $\max$ and $\geq 1$, where $m_{1,2}$ are drawn from $P(m) \propto m^{-2.35}$.
I double checked the Moe \& di Stefano paper, and under their (2) they really
assume that $P(q \leq 1) \propto q^{-p}$ as well, so I think this might be a
misconception in their paper as well?
\begin{figure}
    \centering
    \includegraphics[width=0.45\columnwidth]{../scripts/octlk/3plots/qdist_salpeter.png}
    \includegraphics[width=0.45\columnwidth]{../scripts/octlk/3plots/qdist_salpeter2.png}
    \includegraphics[width=0.45\columnwidth]{../scripts/octlk/3plots/masses.png}
    \caption{(i) Histogram of $q \leq 1$ with random pairings from Salpeter IMF,
    (ii) histogram of $q \geq 1$ with random pairings from Salpeter IMF, with $q
    ^{ -2.35}$ overlaid, and (iii) histogram of masses, with $M^{-2.35}$ power
    law overlaid, as a sanity check.}\label{fig:1}
\end{figure}

Note that in my Appendix, the calculation doesn't change if we take $m_2 \geq
m_1$, i.e.\ originally
\begin{align}
    P\p{\frac{m_{\min}}{m_{\max}} \leq q \leq 1} &=
        \int\limits_{m_{\min}}^{m_{\max}}\mathrm{d}m_1
        \int\limits_{m_{\min}}^{m_1}\mathrm{d}m_2\;
                \delta\p{\frac{m_2}{m_1} - q}P(m_1) P(m_2),\\
        &= \int\limits_{m_{\min}}^{m_{\max}}\mathrm{d}m_1 m_1 P(m_1) P(qm_1),\\
        &\propto q^{-p},
\end{align}
but also
\begin{align}
    P\p{1 \leq q \leq \frac{m_{\max}}{m_{\min}}} &=
        \int\limits_{m_{\min}}^{m_{\max}}\mathrm{d}m_1
        \int\limits_{m_1}^{m_{\max}}\mathrm{d}m_2\;
                \delta\p{\frac{m_2}{m_1} - q}P(m_1) P(m_2),\\
        &= \int\limits_{m_{\min}}^{m_{\max}}\mathrm{d}m_1 m_1 P(m_1) P(qm_1),\\
        &\propto q^{-p}.
\end{align}
Note the different bounds of integration on the second integral. Clearly the
first of these two derivations is wrong (the one that is in the paper), but I am
not sure why yet. I will hopefully have an answer by the time of the meeting.

\subsection{Resolution}

In fact, Tout 1991 \url{https://doi.org/10.1093/mnras/250.4.701} has solved this
problem, and in particular, ``notice that it is sharply peaked at $q = 1$ and
does not have the form $n(q) \propto q^{-\alpha}$ when $q < 1$ as many authors,
following Warner (1961), have assumed''. Both integrals I wrote above are
incorrect, since the bounds of integration on the first integral should change
based on the value of $q$ ($m_{\min} \Rightarrow m_{\min} / q$ in the former,
and $m_{\max} \Rightarrow m_{\max} / q$ in the latter). However, the error on
the second one is nearly negligible, while the error on the former is
significant and changes the asymptotic behavior! The correct distributions are
then:
\begin{align}
    P\p{q_{\min} \leq q \leq 1} &= \int\limits_{m_{\min} / q}^{m_{\max}}
            \mathrm{d}m_1\; (-m_1) \p{m_1}^{-2p}q^{-p},\\
        &\propto q^{-p}\s{-m_1^{-2p + 2}}_{m_{\min} / q}^{m_{\max}},\\
        &\propto \p{\frac{m_{\min}^{2 - 2p}}{q^{2 - p}}}
            - m_{\max}^{2 - 2p}q^{-p},\\
        &\propto q^{p - 2}\s{1 - \p{\frac{q_{\min}}{q}}^{2p -
        2}},\label{eq:salpeter_1}
\end{align}
and
\begin{align}
    P\p{1 \leq q \leq q_{\max}} &= \int\limits_{m_{\min}}^{m_{\max} / q}
            \mathrm{d}m_1\; (-m_1) \p{m_1}^{-2p}q^{-p},\\
        &= q^{-p}\s{-m_1^{-2p + 2}}_{m_{\min}}^{m_{\max} / q},\\
        &= m_{\min}^{2 - 2p}q^{-p} - m_{\max}^{2 - 2p}q^{p - 2},\\
        &\propto q^{-p}\s{1 - \p{\frac{q}{q_{\max}}}^{2p -
        2}}.\label{eq:salpeter_2}
\end{align}

\section{Histograms}

The three requested plots are shown in Fig.~\ref{fig:2}. For (i), I just took
the masses from the previous section and sent them through the SNe transfer
function (Fig.~\ref{fig:sne}). For (ii) and (iii), the procedure is somewhat
more complicated; for each value of $q$ at ZAMS\@: choose $m_2 \in [M_{\min},
qM_{\max}]$ and $m_1 = m_2 / q$. Compute the BH value of $q_{\rm BH}$ by sending
it through the SNe transfer function, and weight it by $P(m_2)$ alone. Repeat
for a grid of $q$ and $m_2$, and histogram it all.
\begin{figure}
    \centering
    \includegraphics[width=0.45\columnwidth]{../scripts/octlk/3plots/qdist_salpeter_sne.png}
    \includegraphics[width=0.45\columnwidth]{../scripts/octlk/3plots/qdist_uniform.png}
    \includegraphics[width=0.45\columnwidth]{../scripts/octlk/3plots/qdist_logu.png}
    \caption{Distribution of $q$ after (i) random pairings Salpeter IMF +
    supernovae, (ii) uniform $q_{\rm ZAMS}$, and (iii) uniform $\log\p{q_{\rm
    ZAMS}}$. Black dashed lines are
    Eqs.~(\ref{eq:salpeter_1},~\ref{eq:salpeter_2}), and $P(m) \propto
    m^{-2.35}$ in the three plots respectively. }\label{fig:2}
\end{figure}

\section{Checking Formula from Paper}

I double checked Equations~(20-21) in the new draft. I actually had the correct
$j(e_{\lim})$ expression that Dong left in comments in the paper, but saw LML15
Eq.~(52) and convinced myself that I had made an algebra mistake. It appears I
must have misread LML15.

The derivation of the formula in the paper was omitted for its ugliness, but I
give it below for verification. The equations we have are:
\begin{align}
    j^6\p{e_{\rm os}} &= \frac{842}{15}\frac{G^3 \mu m_{12}^3}{m_3c^5a^4n}
        \p{\frac{a_{\rm out, eff}}{a}}^3,\\
    j(e_{\lim}) &\approx \frac{8\epsilon_{\rm GR}}{9 + 3\eta^2/4}.
\end{align}
We re-express the condition $j\p{e_{\rm os}} \gtrsim j(e_{\lim})$ (i.e.\ the
limiting eccentricity is sufficiently extreme to execute one-shot mergers), so
\begin{align}
    842 \frac{G^{5/2}a_{\rm out, eff}^3 m_{12}^{5/2}\mu}{
        15a^{11/2}c^5 m_3}
        &\gtrsim \p{\frac{8}{9 + 3\eta^2/4}
            \frac{3Gm_{12}^2a_{\rm out, eff}^3}{c^2a^4m_3}}^6,\\
    a^{37/2} &\gtrsim \frac{2^{18} \cdot 15}{842} \frac{G^{7/2}a_{\rm
        out, eff}^{15} m_{12}^{19/2}}{c^7 m_3^5 \mu \p{3 + \eta^2/4}^6}
        ,\\
    \p{\frac{a}{a_{\rm out, eff}}}^{37/2} &\gtrsim \frac{2^{18} \cdot 15}{842}
            \frac{G^{7/2} m_{12}^{17/2}}{c^7 m_3^5 a_{\rm out, eff}^{7/2} [q /
            (1 + q)^2] \p{3 + \eta^2/4}^6},\\
        &\gtrsim
            0.0186
            \p{\frac{a_{\rm out, eff}}{3600\;\mathrm{AU}}}^{-7/37}
            \p{\frac{m_{12}}{50M_{\odot}}}^{17/37}
            \p{\frac{30M_{\odot}}{m_3}}^{10/37}
            \p{\frac{q / (1 + q)^2}{1/4}}^{-2/37}.
\end{align}
We have used that $\mu = m_{12}\s{q / (1 + q)^2}$. The final numerical
evaluation was done using WolframAlpha, and the URL linking to the evaluation is
provided
\href{https://www.wolframalpha.com/input/?i=%282%5E%2818%29+*+15+%2F+842+*+G%5E%287%2F2%29+*+%283600+AU%29%5E%28-7%2F2%29+*+%2850+solar+mass%29%5E%2817%2F2%29+%2F+%28c%5E7+*+%2830+solar+mass%29%5E5+*+%281%2F4%29+*+%283%5E6%29%29%29%5E%282%2F37%29}{\texttt{here}}.

\end{document}

