    \documentclass[10pt,
        usenames, % allows access to some tikz colors
        dvipsnames, % more colors: https://en.wikibooks.org/wiki/LaTeX/Colors
    ]{article}
    \usepackage{
        amsmath,
        amssymb,
        fouriernc, % fourier font w/ new century book
        fancyhdr, % page styling
        lastpage, % footer fanciness
        hyperref, % various links
        setspace, % line spacing
        amsthm, % newtheorem and proof environment
        mathtools, % \Aboxed for boxing inside aligns, among others
        float, % Allow [H] figure env alignment
        enumerate, % Allow custom enumerate numbering
        graphicx, % allow includegraphics with more filetypes
        wasysym, % \smiley!
        upgreek, % \upmu for \mum macro
        listings, % writing TrueType fonts and including code prettily
        tikz, % drawing things
        booktabs, % \bottomrule instead of hline apparently
        cancel % can cancel things out!
    }
    \usepackage[margin=0.5in]{geometry} % page geometry
    \usepackage[
        labelfont=bf, % caption names are labeled in bold
        font=scriptsize % smaller font for captions
    ]{caption}
    \usepackage[font=scriptsize]{subcaption} % subfigures

    \newcommand*{\scinot}[2]{#1\times10^{#2}}
    \newcommand*{\dotp}[2]{\left<#1\,\middle|\,#2\right>}
    \newcommand*{\rd}[2]{\frac{\mathrm{d}#1}{\mathrm{d}#2}}
    \newcommand*{\pd}[2]{\frac{\partial#1}{\partial#2}}
    \newcommand*{\rtd}[2]{\frac{\mathrm{d}^2#1}{\mathrm{d}#2^2}}
    \newcommand*{\ptd}[2]{\frac{\partial^2 #1}{\partial#2^2}}
    \newcommand*{\md}[2]{\frac{\mathrm{D}#1}{\mathrm{D}#2}}
    \newcommand*{\pvec}[1]{\vec{#1}^{\,\prime}}
    \newcommand*{\svec}[1]{\vec{#1}\;\!}
    \newcommand*{\bm}[1]{\boldsymbol{\mathbf{#1}}}
    \newcommand*{\uv}[1]{\hat{\bm{#1}}}
    \newcommand*{\ang}[0]{\;\text{\AA}}
    \newcommand*{\mum}[0]{\;\upmu \mathrm{m}}
    \newcommand*{\at}[1]{\left.#1\right|}
    \newcommand*{\bra}[1]{\left<#1\right|}
    \newcommand*{\ket}[1]{\left|#1\right>}
    \newcommand*{\abs}[1]{\left|#1\right|}
    \newcommand*{\ev}[1]{\langle#1\rangle}
    \newcommand*{\p}[1]{\left(#1\right)}
    \newcommand*{\s}[1]{\left[#1\right]}
    \newcommand*{\z}[1]{\left\{#1\right\}}

    \newtheorem{theorem}{Theorem}[section]

    \let\Re\undefined
    \let\Im\undefined
    \DeclareMathOperator{\Res}{Res}
    \DeclareMathOperator{\Re}{Re}
    \DeclareMathOperator{\Im}{Im}
    \DeclareMathOperator{\Log}{Log}
    \DeclareMathOperator{\Arg}{Arg}
    \DeclareMathOperator{\Tr}{Tr}
    \DeclareMathOperator{\E}{E}
    \DeclareMathOperator{\Var}{Var}
    \DeclareMathOperator*{\argmin}{argmin}
    \DeclareMathOperator*{\argmax}{argmax}
    \DeclareMathOperator{\sgn}{sgn}
    \DeclareMathOperator{\diag}{diag\;}

    % \everymath{\displaystyle} % biggify limits of inline sums and integrals
    \tikzstyle{circ} %u sage: \cnode[irc, placement] (label) {text};
        = [draw, circle, fill=white, node distance=3cm, minimum height=2em]
    \definecolor{commentgreen}{rgb}{0,0.6,0}
    \lstset{
        basicstyle=\ttfamily\footnotesize,
        frame=single,
        numbers=left,
        showstringspaces=false,
        keywordstyle=\color{blue},
        stringstyle=\color{purple},
        commentstyle=\color{commentgreen},
        morecomment=[l][\color{magenta}]{\#}
    }

\begin{document}

\def\Snospace~{\S{}} % hack to remove the space left after autorefs
\renewcommand*{\sectionautorefname}{\Snospace}
\renewcommand*{\appendixautorefname}{\Snospace}
\renewcommand*{\figureautorefname}{Fig.}
\renewcommand*{\equationautorefname}{Eq.}
\renewcommand*{\tableautorefname}{Tab.}

\singlespacing

\section{04/28/20}

\subsection{Adiabatic Invariance, LL17 Reproduction}

In panel 1 of LL17 Fig.~4, the result of applying adiabatic invariance assuming
$e = 0$ is presented. We have an updated prescription for adiabatic invariance
though, so I have regenerated the figure using the original data.

In the original paper, the adiabatic invariant is (all precession frequencies
evaluated at $e = 0$)
\begin{align}
    \cos \theta_{\rm eff, S1} &= \uv{S} \cdot \uv{\Omega}_{\rm
        eff},\label{eq:bin}\\
    \bm{\Omega}_{\rm eff} &= \Omega_{\rm SL} \uv{L}
        + \Omega_{\rm L}\frac{J}{L_{\rm out}} \uv{J},\\
    \bm{J} &= \bm{L} + \bm{L}_{\rm out}.
\end{align}
The updated adiabatic invariant is (angle brackets denote period averages for a
given $I$, allowing $e$ to vary)
\begin{align}
    \cos \theta_{\rm eff, YS} &= \uv{S} \cdot \uv{\Omega}_{\rm eff},
        \label{eq:YS}\\
    \bm{\Omega}_{\rm eff} &= \ev{\Omega_{\rm SL}\uv{L}}
        - \ev{\dot{\Omega}\uv{J}}.
\end{align}
Reusing the data from LL17, the agreement can be seen to improve, see
Fig.~\ref{fig:bin_comp}. \textbf{NB:} The agreement near $I = 90^\circ$ can be
seen to be slightly off-center, since $\eta \neq 0$ (neither my
numerically-averaged prediction nor the $e=0$ prediction can capture this).
Rerunning the simulations for $\eta = 0$ would produce a minimum at $I_0 =
90^\circ$ as expected. In lieu of this, I have temporarily manually offset the
data to be centered at $90^\circ$ to investigate the scaling near $I_{0,\lim}$,
and my prediction is very good.
\begin{figure}
    \centering
    \includegraphics[width=0.6\textwidth]{plots/6bin_comp.png}
    \includegraphics[width=0.6\textwidth]{plots/6bin_comp_zoom.png}
    \caption{Top: comparison of predicted final spin-orbit misalignment angles
    between Eqs.~\eqref{eq:bin} and~\eqref{eq:YS}. Bottom: Zoomed in on $I_0 =
    90^\circ$ with an artificial offset introduced to illustrate
    comparison.}\label{fig:bin_comp}
\end{figure}

Vertical dotted lines denote where $\ev{\Omega_{\rm eff}} = \pi / P_{\rm LK}$.
It was suspected this was the cutoff where a resonance could be hit (precession
period is \emph{half} the Kozai period). We investigate this below

\subsection{Resonances: Toy Problem}

Recall EOM
\begin{align}
    \rd{\uv{S}}{t} ={}& \ev{\Omega_{\rm SL}\uv{L} - \dot{\Omega}\uv{z}}
        \times \uv{S}\nonumber\\
        &+ \s{
            \sum\limits_{N = 1}^\infty \uv{\Omega}_{\rm eff, N}
            \exp\p{2\pi i N t / t_{\rm LK}}} \times \uv{S}.
\end{align}
Ignore GR, such that $\bm{\Omega}_{\rm eff}$ is a constant, and orient it along
$\uv{z}$. Further consider including only the $N = 1$ term, then we can write
down the fundamental toy problem ($\bm{\Omega}_{0} \propto \uv{z}$)
\begin{equation}
    \rd{\uv{S}}{t} = \bm{\Omega}_0 \times \uv{S}
        + \epsilon \sin \p{\omega t} \uv{\Omega}_1 \times \uv{S}.
        \label{eq:eom}
\end{equation}

\subsubsection{Simple Resonance}

When $\omega \approx \Omega_0$ and $\uv{\Omega}_1 = \uv{x}$ (for simplicity), we
have seen how this is solved, transforming to the frame rotating as $\omega
\uv{z}$ gives the following EOM (in rotating frame)
\begin{subequations}
    \begin{align}
        \rd{\theta}{t} &= -\epsilon \s{
            \sin \phi - \sin \omega t \cos\p{\phi - \omega t}},\\
        \rd{\phi}{t} &= \Omega_0 - \omega + \epsilon \cos \omega t \cot \theta
            \cos\p{\phi - \omega t}.
    \end{align}
\end{subequations}
If $\Omega_0 \approx \omega$, then assume $\omega t$ terms can be
dropped/averaged out, this equates to EOM in rotating frame of form (after some work)
\begin{equation}
    \rd{\uv{S}}{t} = \p{\Omega_0 - \omega}\uv{z} \times \uv{S}
        + \epsilon \uv{x} \times \uv{S}.\label{eq:simple_eom}
\end{equation}
Given initial $\theta, \phi$, we can easily compute the range of $\theta$
(precession about a fixed axis). This can be compared to simulations and yields
plausible agreement (Fig.~\ref{fig:5_resonance_heights}).
\begin{figure}
    \centering
    \includegraphics[width=0.6\textwidth]{plots/5_resonance_heights.png}
    \caption{Range of $\cos \theta$ excited when simulating Eq.~\eqref{eq:eom},
    starting with $\theta = 20^\circ$ (here, $\uv{\Omega}_1$ is pointed at
    $60^\circ$). Legend shows different values of $\epsilon$. Dotted lines are
    analytical predictions from analysis of
    Eq.~\eqref{eq:simple_eom}.}\label{fig:5_resonance_heights}
\end{figure}

Of particular interest in the figure is the behavior of the peak at $\Omega_0 /
\omega = 0.5$, and I have discovered the following properties (cited without
evidence):
\begin{itemize}
    \item Amplitude \& inverse period scales with $\epsilon$.
    \item Depends on the angle of $\uv{\Omega}_1$.
    \item Smaller peaks seem to exist for $\Omega_0 / \omega = 1/N$.
    \item Requires amplitude modulation of perturbation (i.e.\ not excited for
        perturbation $\uv{x}\sin(\omega t) + \uv{y}\cos\p{\omega t}$).
\end{itemize}
The first two points suggest some sort of parametric instability, but I have not
solved it yet.

\end{document}

