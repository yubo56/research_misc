    \documentclass[11pt,
        usenames, % allows access to some tikz colors
        dvipsnames % more colors: https://en.wikibooks.org/wiki/LaTeX/Colors
    ]{article}
    \usepackage{
        amsmath,
        amssymb,
        fouriernc, % fourier font w/ new century book
        fancyhdr, % page styling
        lastpage, % footer fanciness
        hyperref, % various links
        setspace, % line spacing
        amsthm, % newtheorem and proof environment
        mathtools, % \Aboxed for boxing inside aligns, among others
        float, % Allow [H] figure env alignment
        enumerate, % Allow custom enumerate numbering
        graphicx, % allow includegraphics with more filetypes
        wasysym, % \smiley!
        upgreek, % \upmu for \mum macro
        listings, % writing TrueType fonts and including code prettily
        tikz, % drawing things
        booktabs, % \bottomrule instead of hline apparently
        xcolor, % colored text
        cancel % can cancel things out!
    }
    \usepackage[margin=1in]{geometry} % page geometry
    \usepackage[
        labelfont=bf, % caption names are labeled in bold
        font=scriptsize % smaller font for captions
    ]{caption}
    \usepackage[font=scriptsize]{subcaption} % subfigures

    \newcommand*{\scinot}[2]{#1\times10^{#2}}
    \newcommand*{\dotp}[2]{\left<#1\,\middle|\,#2\right>}
    \newcommand*{\rd}[2]{\frac{\mathrm{d}#1}{\mathrm{d}#2}}
    \newcommand*{\pd}[2]{\frac{\partial#1}{\partial#2}}
    \newcommand*{\rdil}[2]{\mathrm{d}#1 / \mathrm{d}#2}
    \newcommand*{\pdil}[2]{\partial#1 / \partial#2}
    \newcommand*{\rtd}[2]{\frac{\mathrm{d}^2#1}{\mathrm{d}#2^2}}
    \newcommand*{\ptd}[2]{\frac{\partial^2 #1}{\partial#2^2}}
    \newcommand*{\md}[2]{\frac{\mathrm{D}#1}{\mathrm{D}#2}}
    \newcommand*{\pvec}[1]{\vec{#1}^{\,\prime}}
    \newcommand*{\svec}[1]{\vec{#1}\;\!}
    \newcommand*{\bm}[1]{\boldsymbol{\mathbf{#1}}}
    \newcommand*{\uv}[1]{\hat{\bm{#1}}}
    \newcommand*{\ang}[0]{\;\text{\AA}}
    \newcommand*{\mum}[0]{\;\upmu \mathrm{m}}
    \newcommand*{\at}[1]{\left.#1\right|}
    \newcommand*{\bra}[1]{\left<#1\right|}
    \newcommand*{\ket}[1]{\left|#1\right>}
    \newcommand*{\abs}[1]{\left|#1\right|}
    \newcommand*{\ev}[1]{\langle#1\rangle}
    \newcommand*{\p}[1]{\left(#1\right)}
    \newcommand*{\s}[1]{\left[#1\right]}
    \newcommand*{\z}[1]{\left\{#1\right\}}

    \newtheorem{theorem}{Theorem}[section]

    \let\Re\undefined
    \let\Im\undefined
    \DeclareMathOperator{\Res}{Res}
    \DeclareMathOperator{\Re}{Re}
    \DeclareMathOperator{\Im}{Im}
    \DeclareMathOperator{\Log}{Log}
    \DeclareMathOperator{\Arg}{Arg}
    \DeclareMathOperator{\Tr}{Tr}
    \DeclareMathOperator{\E}{E}
    \DeclareMathOperator{\Var}{Var}
    \DeclareMathOperator*{\argmin}{argmin}
    \DeclareMathOperator*{\argmax}{argmax}
    \DeclareMathOperator{\sgn}{sgn}
    \DeclareMathOperator{\diag}{diag\;}

    \colorlet{Corr}{red}

    % \everymath{\displaystyle} % biggify limits of inline sums and integrals
    \tikzstyle{circ} % usage: \node[circ, placement] (label) {text};
        = [draw, circle, fill=white, node distance=3cm, minimum height=2em]
    \definecolor{commentgreen}{rgb}{0,0.6,0}
    \lstset{
        basicstyle=\ttfamily\footnotesize,
        frame=single,
        numbers=left,
        showstringspaces=false,
        keywordstyle=\color{blue},
        stringstyle=\color{purple},
        commentstyle=\color{commentgreen},
        morecomment=[l][\color{magenta}]{\#}
    }

\begin{document}

\def\Snospace~{\S{}} % hack to remove the space left after autorefs
\renewcommand*{\sectionautorefname}{\Snospace}
\renewcommand*{\appendixautorefname}{\Snospace}
\renewcommand*{\figureautorefname}{Fig.}
\renewcommand*{\equationautorefname}{Eq.}
\renewcommand*{\tableautorefname}{Tab.}

\section{Recap}

Recall EOM in corotating frame with $\uv{L}$
\begin{equation}
    \p{\rd{\uv{S}}{t}}_{\rm rot}
        = \p{\Omega_{\rm SL}\uv{L} - \dot{\Omega} \uv{z}} \cdot \uv{S}
        = \ev{\Omega_{\rm SL}\uv{L} - \dot{\Omega}\uv{z}} \times \uv{S}
        + \s{
            \sum\limits_{N = 1}^\infty \bm{\Omega}_{\rm eff, N}
            \sin\p{2\pi N t / t_{\rm LK}}} \times \uv{S}.\label{eq:eom0}
\end{equation}
We have a good understanding of the non-adiabatic (fast-merger) regime. In the
adiabatic regime, we expect good conservation of $\cos \theta_{\rm eff} =
\uv{S} \cdot \bm{\Omega}_{\rm eff}$, as the fast-varying terms should generally
be able to be averaged over. We proposed the below toy model to understand the
contribution of the leading-order $N = 1$ term, where we have rotated
$\bm{\Omega}_{\rm eff} \propto \uv{z}$:
\begin{equation}
    \frac{1}{\Omega_{\rm eff}}\rd{\uv{S}}{t} = \uv{z} \times \uv{S}
        + \sin \p{\omega t} \frac{\uv{\Omega}_{\rm eff, 1}}{\Omega_{\rm eff}}
            \times \uv{S}.\label{eq:eom}
\end{equation}

Under the na\"\i{}ve expectation, where $\omega \gg \Omega_{\rm eff}, \Omega_{\rm
eff, 1}$, the second term should average away, and $\uv{S}$ simply precesses
around $\uv{z}$, conserving $\theta_{\rm eff}$. Note that $\omega
\Leftrightarrow 2\pi / P_{\rm LK}$ while $\Omega_{\rm eff} \simeq \dot{\Omega}
\lesssim \omega$ is our regime of interest.

To estimate the deviation from exact conservation of $\theta_{\rm eff}$, we
simulate $\rdil{\uv{S}}{t}$ for a long time and compute the total amplitude of
oscillation of $\cos \theta_{\rm eff}$, and we found behavior as described in
Fig.~\ref{fig:resonance_heights}.
\begin{figure}
    \centering
    \includegraphics[width=0.7\textwidth]{plots/5_resonance_heights.png}
    \caption{Plot of $\max \cos \theta_{\rm eff} - \min \cos \theta_{\rm eff}$
    when solving Eq.~\eqref{eq:eom}. Legend denotes $\Omega_{\rm eff, 1} /
    \Omega_{\rm eff}$. Orientation is $\arccos \p{\uv{z} \cdot \uv{\Omega}_{\rm
    eff, 1}} = 60^\circ$.}\label{fig:resonance_heights}
\end{figure}

Per Dong's suggestion, the next step is to evaluate Eq.~\eqref{eq:eom} with a
real numerical LK spectrum, not restricting to the first harmonic.

\section{This Week's Work}

\subsection{Change to $\theta_{\rm eff}$: Effect of $I_0$}\label{ss:sims}

Following the suggestion, I did not continue analytical work towards
Eq.~\eqref{eq:eom}, but instead used a real LK spectrum. To be precise, for
physical parameters from both the mild LK scenario (Paper I) and the distance LK
scenario (Paper II), I solved the GW-free (but including de Sitter precession)
LK oscillation equations and obtained $\dot{\Omega}$, $\Omega_{\rm SL}$, and
$\uv{L}$ over a single LK period. I then evolve $\uv{S}$ using the first EOM in
Eq.~\eqref{eq:eom0} over 500 LK periods, using the numerical LK solution over
each period. I then measure $\theta_{\rm eff}$ at the end of every LK period,
and plot $\theta_{\rm eff, \max} - \theta_{\rm eff, \min}$\footnote{This
amplitude is of interest since it provides an upper bound on $\theta_{\rm eff}$
excitations when hitting a resonance; the actual excitation versus this upper
bound will be considered another week, but must depend on the adiabaticity of
resonance crossing.}. I start at a variety of initial $I_0$, but use a uniform
initial eccentricity $e_0 = 0.003$ (slightly larger than the values used in
Paper I, Paper II, to account for some small initial GW dissipation). I choose
the initial $\uv{S}$ to lie along $\uv{L}$ for simplicity; I explore the effect
of varying initial conditions later. Three sample trajectories for the Paper I
scenario have similar $I$ ($117.543^\circ$, $120^\circ$, $125.15^\circ$) but
significantly different amplitudes of $\theta_{\rm eff}$ excursions are shown in
Fig.~\ref{fig:inners}. As can be seen, the two large-amplitude oscillations have
notably different characters, owing to their belonging to two different types of
resonances; we explain this later.
\begin{figure}
    \centering
    \includegraphics[width=0.32\textwidth]{plots/6_devs_inner117.png}
    \includegraphics[width=0.32\textwidth]{plots/6_devs_inner120.png}
    \includegraphics[width=0.32\textwidth]{plots/6_devs_inner125.png}
    \caption{Plot of $\theta_{\rm eff}$ (y-axis) measured after each LK cycle
    using numerically integrated LK solution (x-axis is time in units of $t_{\rm
    LK}$). Physical parameters are from Paper I, $e_0 = 0.003$ and $I_0 =
    117.543^\circ$, $120^\circ$, and $125.15^\circ$
    respectively.}\label{fig:inners}
\end{figure}

I tried this for both Paper I and Paper II\@; Paper I's scenario is useful to
calibrate against because we know what range of $I_0$ experience $\theta_{\rm
eff}$ excursions, even though Paper II's parameters are more physically
relevant. Both are presented in Fig.~\ref{fig:Iscan}. It is evident that Paper
II's scenario conserves $\theta_{\rm eff}$ quite well. Comparison to the results
of Paper I shows that substantial $\theta_{\rm eff}$ excitation is indeed
possible, and occurs at particular values of $I_0$ given a particular $e_0$.
Including GW dissipation, which changes both $I_0$ and $e_0$ (initial
inclinations/eccentricities at the start of each LK cycle), this implies a
large range of initial $(I, e=0.001)$ can experience $\theta_{\rm eff}$
excitation over the course of GW radiation (the bounds are believably
consistent).
\begin{figure}
    \centering
    \includegraphics[width=0.45\textwidth]{plots/6_Iscan.png}
    \includegraphics[width=0.45\textwidth]{plots/6_Iscan_out.png}
    \caption{$\theta_{\rm eff, \max} - \theta_{\rm eff, \min}$ (in degrees)
    using numerical LK solutions to evolve an initial spin $\uv{S} = \uv{L}$ by
    500 LK cycles, for $e_0 = 0.003$ in the Paper I and Paper II regimes
    respectively. Inclinations are sampled $I_0 \in [95^\circ,
    135^\circ]$.}\label{fig:Iscan}
\end{figure}

Above, we said we fixed $\uv{S}_{\rm i} = \uv{L}_0$. I also explored the
variation of $\max \theta_{\rm eff} - \min \theta_{\rm eff}$ as a function of
$(\theta, \phi)$ (in coordinates where $\uv{L}_{\rm out} \propto \uv{z}$). I am
currently regenerating these plots for particularly interesting $I_0$, but one
for the Paper II regime and $I_0 = 125^\circ$ is given in
Fig.~\ref{fig:ampgrid_frag_large}.
\begin{figure}
    \centering
    \includegraphics[width=0.5\textwidth]{plots/6_ampgrid_frac_large.png}
    \caption{$\theta_{\rm eff}$ for various possible $\uv{S}$ initial
    conditions. The angular variation is relatively
    smooth.}\label{fig:ampgrid_frag_large}
\end{figure}

\subsection{Contributing Factors to Resonance}

Based on the toy model from the previous week
(Fig.~\ref{fig:resonance_heights}), we know the two primary contributing factors
to the amplitude of the $\theta_{\rm eff}$ oscillations are: (i) the strength of
the $N = 1$ term ($\sim \Omega_{\rm eff, 1} / \Omega_{\rm eff}$ up to a
projection factor), which sets the width of the resonance(s); and (ii) the ratio
$\ev{\Omega_{\rm eff}} P_{\rm LK} / (2\pi)$, which sets whether we are near a
resonance. (iii) A third one is the alignment between the $N = 0$
and $N = 1$ component of $\bm{\Omega}_{\rm eff}$; this proves not particularly
useful as all Fourier components are generally quite well aligned (the vectors
they approximate are close to delta functions at the eccentricity maxima).

We make plots of all three of these in Fig.~\ref{fig:IscanN1}.
\begin{figure}
    \centering
    \includegraphics[width=0.45\textwidth]{plots/6_IscanN1inner.png}
    \includegraphics[width=0.45\textwidth]{plots/6_IscanN1outer.png}
    \caption{Plot of the ratio of $N = 1$ to $N = 0$ (in magnitude).}\label{fig:IscanN1}
\end{figure}

This paints an encouraging picture. If, for each $I_0$, we take the ratio in the
third panel, and look at Fig.~\ref{fig:resonance_heights}, this seems to
coarsely predict the correct behavior in Fig.~\ref{fig:Iscan}. Further work is
needed to confirm this.

\subsection{Qualitative Picture So Far}

We have restricted $e_0 = 0.003$ in Section~\ref{ss:sims}, so we will focus on
this case, and we have restricted $I_0 \in [95^\circ, 135^\circ]$.

Based on Fig.~\ref{fig:IscanN1}, in the Paper I scenario, we expect a
$\Omega_0/\omega = 1$ resonance at around $117^\circ$ and a $\Omega_0/\omega =
0.5$ resoonance at around $127^\circ$. In the Paper II scenario, the
$\Delta \Omega = \pi$ resonance is missed by our $I_0$ range (I will fix), and
instead deviations from adiabaticity are generated by the $\Delta \Omega = 2\pi$
resonance, which is very weak since it occurs for $I \gtrsim 130^\circ$ where
the $N = 1$ Fourier coefficient is very weak (Fig.~\ref{fig:IscanN1}) and LK
oscillations disappear. Both of these match quite well with
Fig.~\ref{fig:Iscan}.

The necessary/possible next objectives are:
\begin{itemize}
    \item Examine $I_0 \approx 92.5^\circ$ for the Paper II regime, where there
        is likely another resonance, much stronger than the effect observed in
        Fig.~\ref{fig:Iscan}.
    \item Understand how the resonance sweeps through trajectories under GW
        decay (increase of $e_0$ and decrease of $I_0$ from initial conditions)
    \item Analytical understanding of $\Delta \Omega = \pi$ resonance.
\end{itemize}

\end{document}

