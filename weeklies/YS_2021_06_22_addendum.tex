    \documentclass[11pt,
        usenames, % allows access to some tikz colors
        dvipsnames % more colors: https://en.wikibooks.org/wiki/LaTeX/Colors
    ]{article}
    \usepackage{
        amsmath,
        amssymb,
        fouriernc, % fourier font w/ new century book
        fancyhdr, % page styling
        lastpage, % footer fanciness
        hyperref, % various links
        setspace, % line spacing
        amsthm, % newtheorem and proof environment
        mathtools, % \Aboxed for boxing inside aligns, among others
        float, % Allow [H] figure env alignment
        enumerate, % Allow custom enumerate numbering
        graphicx, % allow includegraphics with more filetypes
        wasysym, % \smiley!
        upgreek, % \upmu for \mum macro
        listings, % writing TrueType fonts and including code prettily
        tikz, % drawing things
        booktabs, % \bottomrule instead of hline apparently
        xcolor, % colored text
        cancel % can cancel things out!
    }
    \usepackage[margin=1in]{geometry} % page geometry
    \usepackage[
        labelfont=bf, % caption names are labeled in bold
        font=scriptsize % smaller font for captions
    ]{caption}
    \usepackage[font=scriptsize]{subcaption} % subfigures

    \newcommand*{\scinot}[2]{#1\times10^{#2}}
    \newcommand*{\dotp}[2]{\left<#1\,\middle|\,#2\right>}
    \newcommand*{\rd}[2]{\frac{\mathrm{d}#1}{\mathrm{d}#2}}
    \newcommand*{\pd}[2]{\frac{\partial#1}{\partial#2}}
    \newcommand*{\rdil}[2]{\mathrm{d}#1 / \mathrm{d}#2}
    \newcommand*{\pdil}[2]{\partial#1 / \partial#2}
    \newcommand*{\rtd}[2]{\frac{\mathrm{d}^2#1}{\mathrm{d}#2^2}}
    \newcommand*{\ptd}[2]{\frac{\partial^2 #1}{\partial#2^2}}
    \newcommand*{\md}[2]{\frac{\mathrm{D}#1}{\mathrm{D}#2}}
    \newcommand*{\pvec}[1]{\vec{#1}^{\,\prime}}
    \newcommand*{\svec}[1]{\vec{#1}\;\!}
    \newcommand*{\bm}[1]{\boldsymbol{\mathbf{#1}}}
    \newcommand*{\uv}[1]{\hat{\bm{#1}}}
    \newcommand*{\ang}[0]{\;\text{\AA}}
    \newcommand*{\mum}[0]{\;\upmu \mathrm{m}}
    \newcommand*{\at}[1]{\left.#1\right|}
    \newcommand*{\bra}[1]{\left<#1\right|}
    \newcommand*{\ket}[1]{\left|#1\right>}
    \newcommand*{\abs}[1]{\left|#1\right|}
    \newcommand*{\ev}[1]{\left\langle#1\right\rangle}
    \newcommand*{\p}[1]{\left(#1\right)}
    \newcommand*{\s}[1]{\left[#1\right]}
    \newcommand*{\z}[1]{\left\{#1\right\}}

    \newtheorem{theorem}{Theorem}[section]

    \let\Re\undefined
    \let\Im\undefined
    \DeclareMathOperator{\Res}{Res}
    \DeclareMathOperator{\Re}{Re}
    \DeclareMathOperator{\Im}{Im}
    \DeclareMathOperator{\Log}{Log}
    \DeclareMathOperator{\Arg}{Arg}
    \DeclareMathOperator{\Tr}{Tr}
    \DeclareMathOperator{\E}{E}
    \DeclareMathOperator{\Var}{Var}
    \DeclareMathOperator*{\argmin}{argmin}
    \DeclareMathOperator*{\argmax}{argmax}
    \DeclareMathOperator{\sgn}{sgn}
    \DeclareMathOperator{\diag}{diag\;}

    \colorlet{Corr}{red}

    % \everymath{\displaystyle} % biggify limits of inline sums and integrals
    \tikzstyle{circ} % usage: \node[circ, placement] (label) {text};
        = [draw, circle, fill=white, node distance=3cm, minimum height=2em]
    \definecolor{commentgreen}{rgb}{0,0.6,0}
    \lstset{
        basicstyle=\ttfamily\footnotesize,
        frame=single,
        numbers=left,
        showstringspaces=false,
        keywordstyle=\color{blue},
        stringstyle=\color{purple},
        commentstyle=\color{commentgreen},
        morecomment=[l][\color{magenta}]{\#}
    }

\begin{document}

\section{$\bar{g}$ Equilibrium}

Recall that we showed that, in the co-rotating frame with frequency $\bar{g} =
\p{g_1 + g_2} / 2$:
\begin{align}
    \p{\rd{\uv{s}}{t}}_{\rm rot}
        &= \alpha\p{\uv{s} \cdot \uv{l}}\p{\uv{s} \times \uv{l}}
            - \bar{g}\p{\uv{s} \times \uv{z}}\label{eq:eom_rot},\\
    \uv{l}(t) &= \begin{bmatrix}
        \p{I_1 + I_2}\cos \p{\frac{\Delta gt - \phi_0}{2}}\\
        \p{I_1 - I_2}\sin \p{\frac{\Delta gt - \phi_0}{2}}\\
        1
    \end{bmatrix} + \mathcal{O}\s{\p{I_1 + I_2}^2}.
\end{align}
We will try to seek a set of conditions for an $\bm{s}$ to be exactly stationary
under these equations. We will find a leading-order prediction of the location
of an equilibrium, though an exact equilibrium is impossible.

We first examine the $\uv{z}$ component of the equation of motion for $\uv{s}$ ,
suppressing the ``rot'' subscript. Denote $\bm{s}_{\perp}$ and $\bm{l}_{\perp}$
to be the $\uv{x}$-$\uv{y}$ plane components of the two vectors, then we obtain
\begin{equation}
    \rd{s_z}{t} = \alpha\p{\uv{s} \cdot \uv{l}}\p{\uv{s}_{\perp} \times
            \uv{l}_{\perp}}.
\end{equation}
An easy way for this to vanish is if
\begin{equation}
    \uv{s}_\perp \parallel \uv{l}_\perp\label{eq:parallel_cons}.
\end{equation}

We next examine the in-plane component of the equation of motion:
\begin{align}
    \rd{\bm{s}_\perp}{t}
        &= \alpha\p{\uv{s} \cdot \uv{l}}
                \p{\bm{s}_\perp \times \uv{z} + s_z\uv{z} \times \bm{l}_\perp}
            - \bar{g}\p{\bm{s}_\perp \times \uv{z}}.
\end{align}
Since $\bm{s}_\perp \parallel \bm{l}_\perp$, we can express everything in terms
of $\bm{l}_\perp$ and the two magnitudes $s_\perp$ and $l_\perp$. This gives the
following manipulation:
\begin{align}
    \rd{\bm{s}_\perp}{t}
        &= \alpha\p{\uv{s} \cdot \uv{l}}
                \p{\bm{s}_\perp \times \uv{z} - s_z\frac{l_\perp}{s_\perp}
                    \bm{s}_\perp \times \uv{z}}
            - \bar{g}\p{\bm{s}_\perp \times \uv{z}},\\
        &= \s{\alpha\p{\uv{s} \cdot \uv{l}}
            \p{1 - s_z\frac{l_\perp}{s_\perp}} - \bar{g}}
            \p{\bm{s}_\perp \times \uv{z}},\\
        &= \z{\s{\alpha s_z - \bar{g}}
            + \alpha \s{s_\perp l_\perp - s_z^2\frac{l_\perp}{s_\perp}}
            - s_z l_\perp^2}
            \p{\bm{s}_\perp \times \uv{z}}\label{eq:perp_const}.
\end{align}
I've grouped the terms in order of $\mathcal{O}\p{l_\perp}$, since $l_\perp \ll
1$. If an exact equilibrium exists, the expression in the curly brackets must
vanish, as well as Eq.~\eqref{eq:parallel_cons} be satisfied exactly. In
particular, if $l_\perp \ll 1$, we recover the relation I claimed in the writeup
earlier:
\begin{equation}
    s_z \approx \frac{\bar{g}}{\alpha} + \mathcal{O}\p{l_\perp}.
\end{equation}
However, since $l_\perp$ is in general \emph{not} constant in time (ranging from
$I_1 + I_2$ to $I_1 - I_2$), an exact equilibrium does not exist: $l_\perp(t)$
has a $\Delta g / 2$ harmonic, which means $s_\perp(t)$ will also have a
$\Delta g / 2$ harmonic.

\subsection{Perturbation Theory}

We try to do perturbation theory in $I_2$, since nonzero $I_2$ gives rise to the
harmonic terms.

\end{document}

