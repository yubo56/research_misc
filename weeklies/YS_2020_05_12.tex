    \documentclass[11pt,
        usenames, % allows access to some tikz colors
        dvipsnames % more colors: https://en.wikibooks.org/wiki/LaTeX/Colors
    ]{article}
    \usepackage{
        amsmath,
        amssymb,
        fouriernc, % fourier font w/ new century book
        fancyhdr, % page styling
        lastpage, % footer fanciness
        hyperref, % various links
        setspace, % line spacing
        amsthm, % newtheorem and proof environment
        mathtools, % \Aboxed for boxing inside aligns, among others
        float, % Allow [H] figure env alignment
        enumerate, % Allow custom enumerate numbering
        graphicx, % allow includegraphics with more filetypes
        wasysym, % \smiley!
        upgreek, % \upmu for \mum macro
        listings, % writing TrueType fonts and including code prettily
        tikz, % drawing things
        booktabs, % \bottomrule instead of hline apparently
        xcolor, % colored text
        cancel % can cancel things out!
    }
    \usepackage[margin=1in]{geometry} % page geometry
    \usepackage[
        labelfont=bf, % caption names are labeled in bold
        font=scriptsize % smaller font for captions
    ]{caption}
    \usepackage[font=scriptsize]{subcaption} % subfigures

    \newcommand*{\scinot}[2]{#1\times10^{#2}}
    \newcommand*{\dotp}[2]{\left<#1\,\middle|\,#2\right>}
    \newcommand*{\rd}[2]{\frac{\mathrm{d}#1}{\mathrm{d}#2}}
    \newcommand*{\pd}[2]{\frac{\partial#1}{\partial#2}}
    \newcommand*{\rdil}[2]{\mathrm{d}#1 / \mathrm{d}#2}
    \newcommand*{\pdil}[2]{\partial#1 / \partial#2}
    \newcommand*{\rtd}[2]{\frac{\mathrm{d}^2#1}{\mathrm{d}#2^2}}
    \newcommand*{\ptd}[2]{\frac{\partial^2 #1}{\partial#2^2}}
    \newcommand*{\md}[2]{\frac{\mathrm{D}#1}{\mathrm{D}#2}}
    \newcommand*{\pvec}[1]{\vec{#1}^{\,\prime}}
    \newcommand*{\svec}[1]{\vec{#1}\;\!}
    \newcommand*{\bm}[1]{\boldsymbol{\mathbf{#1}}}
    \newcommand*{\uv}[1]{\hat{\bm{#1}}}
    \newcommand*{\ang}[0]{\;\text{\AA}}
    \newcommand*{\mum}[0]{\;\upmu \mathrm{m}}
    \newcommand*{\at}[1]{\left.#1\right|}
    \newcommand*{\bra}[1]{\left<#1\right|}
    \newcommand*{\ket}[1]{\left|#1\right>}
    \newcommand*{\abs}[1]{\left|#1\right|}
    \newcommand*{\ev}[1]{\langle#1\rangle}
    \newcommand*{\p}[1]{\left(#1\right)}
    \newcommand*{\s}[1]{\left[#1\right]}
    \newcommand*{\z}[1]{\left\{#1\right\}}

    \newtheorem{theorem}{Theorem}[section]

    \let\Re\undefined
    \let\Im\undefined
    \DeclareMathOperator{\Res}{Res}
    \DeclareMathOperator{\Re}{Re}
    \DeclareMathOperator{\Im}{Im}
    \DeclareMathOperator{\Log}{Log}
    \DeclareMathOperator{\Arg}{Arg}
    \DeclareMathOperator{\Tr}{Tr}
    \DeclareMathOperator{\E}{E}
    \DeclareMathOperator{\Var}{Var}
    \DeclareMathOperator*{\argmin}{argmin}
    \DeclareMathOperator*{\argmax}{argmax}
    \DeclareMathOperator{\sgn}{sgn}
    \DeclareMathOperator{\diag}{diag\;}

    \colorlet{Corr}{red}

    % \everymath{\displaystyle} % biggify limits of inline sums and integrals
    \tikzstyle{circ} % usage: \node[circ, placement] (label) {text};
        = [draw, circle, fill=white, node distance=3cm, minimum height=2em]
    \definecolor{commentgreen}{rgb}{0,0.6,0}
    \lstset{
        basicstyle=\ttfamily\footnotesize,
        frame=single,
        numbers=left,
        showstringspaces=false,
        keywordstyle=\color{blue},
        stringstyle=\color{purple},
        commentstyle=\color{commentgreen},
        morecomment=[l][\color{magenta}]{\#}
    }

\begin{document}

\def\Snospace~{\S{}} % hack to remove the space left after autorefs
\renewcommand*{\sectionautorefname}{\Snospace}
\renewcommand*{\appendixautorefname}{\Snospace}
\renewcommand*{\figureautorefname}{Fig.}
\renewcommand*{\equationautorefname}{Eq.}
\renewcommand*{\tableautorefname}{Tab.}

Very brief update since last week.

\section{Recap}

Recall we found that, when setting $e_0 = 3 \times 10^{-3}$ (the eccentricity at
the start of the LK cycle), we can numerically obtain the amplitude of
oscillations of $\theta_{\rm eff}$ as a function of $I_0$ the inclination at the
start of the LK cycle, where
\begin{equation}
    \Delta \theta_{\rm eff} \equiv \theta_{\rm eff, \max} - \theta_{\rm eff,
        \min},
\end{equation}
and we obtained as shown in Fig.~\ref{fig:Iscan}.
\begin{figure}
    \centering
    \includegraphics[width=0.45\textwidth]{plots/6_Iscan.png}
    \includegraphics[width=0.45\textwidth]{plots/6_Iscan_out.png}
    \caption{Last week plot. $\Delta \theta_{\rm eff}$ (in degrees) using
    numerical LK solutions to evolve an initial spin $\uv{S} = \uv{L}$ by 500 LK
    cycles, for $e_0 = 0.003$ in the Paper I and Paper II regimes respectively.
    Inclinations are sampled $I_0 \in [95^\circ, 135^\circ]$.}\label{fig:Iscan}
\end{figure}
We found evidence consistent with the hypothesis that the toy spin model,
coupled with numerical integrations of LK + pericenter advance, generates these
oscillations in $\theta_{\rm eff}$.

We then asked: for a real LK + GW system, $e_0, I_0$ will vary over the course
of the inspiral. What is the behavior of $\Delta \theta_{\rm eff}$ for this
realistic sequence of orbital parameters $e_0(t)$ and $I_0(t)$\footnote{This is
formally a discrete sequence of eccentricities and inclinations, but it's easier
to think about time-indexed.}? For clarity, I call these parameters
$e_{\min}(t)$ and $I_{\min}(t)$, so that $e_0, I_0$ can refer to the initial
value at the start of the inspiral.

\section{This Week}

I first sampled $\Delta \theta_{\rm eff}$ over a grid of $\p{e_{\min},
I_{\min}}$ (uniform in $I \in [90.5, 135]$ and log-uniform in $e \in [10^{-3},
0.9]$). I computed this for both the Paper I and Paper II regimes. This is given
in Fig.~\ref{fig:scangrid}.

\begin{figure}
    \centering
    \includegraphics[width=0.45\textwidth]{plots/6_Iscangrid_inner.png}
    \includegraphics[width=0.45\textwidth]{plots/6_Iscangrid.png}
    \caption{$\Delta \theta_{\rm eff}$ (in degrees, colorbar) plotted over a
    scan of $e_{\min}$ and $I_{\min}$, the ``osculating'' orbital elements relevant
    for the LK cycle, for Paper I and Paper II parameters respectively.
    Overplotted are lines of constant Kozai constant (dotted red) and the
    trajectory swept through using the paper II parameters for $e_0 = 10^{-3}$
    and $I_0 = 90.5^\circ$. The real LK + GW simulation shown follows the
    constant-$K$ curves somewhat, since $K$ is slowly varying under GW
    dissipation.}\label{fig:scangrid}
\end{figure}

Focusing on the right plot (paper II regime), it appears $\Delta \theta_{\rm
eff}$ never exceeds a few degrees in the Paper II regime, which is of primary
interest. To better quantify this, we can make a plot of $\Delta \theta_{\rm
eff}\p{e_{\min}(t), I_{\min}(t)}$ using the $e_{\min}(t)$ and $I_{\min}(t)$ from
the LK + GW inspiral simulation using the fiducial paper II parameters (and $e_0
= 10^{-3}$, $I_0 = 90.5^\circ$). This is given in Fig.~\ref{fig:905}.
\begin{figure}
    \centering
    \includegraphics[width=0.6\textwidth]{plots/6_905.png}
    \caption{Plot of $\Delta \theta_{\rm eff}\p{e_{\min}(t),
    I_{\min}(t)}$ (degrees) using the orbital histories from a LK + GW inspiral
    simulation.}\label{fig:905}
\end{figure}

It bears noting that, in our actual LK + GW inspiral simulations at $I_0 =
90.5^\circ$, the actual deviation of $\theta_{\rm eff}$ from the beginning to
end of the simulation is $\ll$ 1 degree, much smaller than the $\Delta
\theta_{\rm eff}$ given in these plots. However, indeed, plots show that
$\theta_{\rm eff}$ varies on a scale of $\sim 5$--$10^\circ$ right before
merger, as shown in Fig.~\ref{fig:905qn0}. I haven't had time to check the
relevant timescales, but I suspect this implies the system adiabatically enters
and exits the resonance, thus $\theta_{\rm eff}$ returns to its initial value
despite an intermediate period of oscillation. So this seems to suggest that
$\Delta \theta_{\rm eff}$ being small is not sufficient to guarantee extremely
good ($\ll$ 1 degree) conservation of $\Delta \theta_{\rm eff}$, only that it
won't vary by any more than a few degrees.
\begin{figure}
    \centering
    \includegraphics[width=0.6\textwidth]{plots/4sim_90_500_qN0.png}
    \caption{Plot of $\Delta \theta_{\rm eff}$ (black; ignore red; called
    $\theta_{N = 0}$ in this old plot) for $I_0 = 90.5^\circ$ LK + GW inspiral
    simulation. Important to see is that, near merger, $\theta_{\rm eff}$ has
    quite large amplitude oscillations, but narrows into its final
    value.}\label{fig:905qn0}
\end{figure}

The adiabaticity explanation may also be able to explain the Paper I results. In
this regime, LK is much milder and so we expect transitions between states to be
much more gradual. One competing trade-off is that the significantly milder
oscillation gives a much narrower resonance width, so transitions may not
necessarily be ``more adiabatic'' in the paper I regime. This may be necessary
to explain why systems in the Paper I regime deviate from exact conservation of
$\theta_{\rm eff}$ for $I \gtrsim 100^\circ$.

\end{document}

