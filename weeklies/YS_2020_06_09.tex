    \documentclass[11pt,
        usenames, % allows access to some tikz colors
        dvipsnames % more colors: https://en.wikibooks.org/wiki/LaTeX/Colors
    ]{article}
    \usepackage{
        amsmath,
        amssymb,
        fouriernc, % fourier font w/ new century book
        fancyhdr, % page styling
        lastpage, % footer fanciness
        hyperref, % various links
        setspace, % line spacing
        amsthm, % newtheorem and proof environment
        mathtools, % \Aboxed for boxing inside aligns, among others
        float, % Allow [H] figure env alignment
        enumerate, % Allow custom enumerate numbering
        graphicx, % allow includegraphics with more filetypes
        wasysym, % \smiley!
        upgreek, % \upmu for \mum macro
        listings, % writing TrueType fonts and including code prettily
        tikz, % drawing things
        booktabs, % \bottomrule instead of hline apparently
        xcolor, % colored text
        cancel % can cancel things out!
    }
    \usepackage[margin=1in]{geometry} % page geometry
    \usepackage[
        labelfont=bf, % caption names are labeled in bold
        font=scriptsize % smaller font for captions
    ]{caption}
    \usepackage[font=scriptsize]{subcaption} % subfigures

    \newcommand*{\scinot}[2]{#1\times10^{#2}}
    \newcommand*{\dotp}[2]{\left<#1\,\middle|\,#2\right>}
    \newcommand*{\rd}[2]{\frac{\mathrm{d}#1}{\mathrm{d}#2}}
    \newcommand*{\pd}[2]{\frac{\partial#1}{\partial#2}}
    \newcommand*{\rdil}[2]{\mathrm{d}#1 / \mathrm{d}#2}
    \newcommand*{\pdil}[2]{\partial#1 / \partial#2}
    \newcommand*{\rtd}[2]{\frac{\mathrm{d}^2#1}{\mathrm{d}#2^2}}
    \newcommand*{\ptd}[2]{\frac{\partial^2 #1}{\partial#2^2}}
    \newcommand*{\md}[2]{\frac{\mathrm{D}#1}{\mathrm{D}#2}}
    \newcommand*{\pvec}[1]{\vec{#1}^{\,\prime}}
    \newcommand*{\svec}[1]{\vec{#1}\;\!}
    \newcommand*{\bm}[1]{\boldsymbol{\mathbf{#1}}}
    \newcommand*{\uv}[1]{\hat{\bm{#1}}}
    \newcommand*{\ang}[0]{\;\text{\AA}}
    \newcommand*{\mum}[0]{\;\upmu \mathrm{m}}
    \newcommand*{\at}[1]{\left.#1\right|}
    \newcommand*{\bra}[1]{\left<#1\right|}
    \newcommand*{\ket}[1]{\left|#1\right>}
    \newcommand*{\abs}[1]{\left|#1\right|}
    \newcommand*{\ev}[1]{\langle#1\rangle}
    \newcommand*{\p}[1]{\left(#1\right)}
    \newcommand*{\s}[1]{\left[#1\right]}
    \newcommand*{\z}[1]{\left\{#1\right\}}

    \newtheorem{theorem}{Theorem}[section]

    \let\Re\undefined
    \let\Im\undefined
    \DeclareMathOperator{\Res}{Res}
    \DeclareMathOperator{\Re}{Re}
    \DeclareMathOperator{\Im}{Im}
    \DeclareMathOperator{\Log}{Log}
    \DeclareMathOperator{\Arg}{Arg}
    \DeclareMathOperator{\Tr}{Tr}
    \DeclareMathOperator{\E}{E}
    \DeclareMathOperator{\Var}{Var}
    \DeclareMathOperator*{\argmin}{argmin}
    \DeclareMathOperator*{\argmax}{argmax}
    \DeclareMathOperator{\sgn}{sgn}
    \DeclareMathOperator{\diag}{diag\;}

    \colorlet{Corr}{red}

    % \everymath{\displaystyle} % biggify limits of inline sums and integrals
    \tikzstyle{circ} % usage: \node[circ, placement] (label) {text};
        = [draw, circle, fill=white, node distance=3cm, minimum height=2em]
    \definecolor{commentgreen}{rgb}{0,0.6,0}
    \lstset{
        basicstyle=\ttfamily\footnotesize,
        frame=single,
        numbers=left,
        showstringspaces=false,
        keywordstyle=\color{blue},
        stringstyle=\color{purple},
        commentstyle=\color{commentgreen},
        morecomment=[l][\color{magenta}]{\#}
    }

\begin{document}

\def\Snospace~{\S{}} % hack to remove the space left after autorefs
\renewcommand*{\sectionautorefname}{\Snospace}
\renewcommand*{\appendixautorefname}{\Snospace}
\renewcommand*{\figureautorefname}{Fig.}
\renewcommand*{\equationautorefname}{Eq.}
\renewcommand*{\tableautorefname}{Tab.}

\section{Rapid Mergers}

Last week, Dong sent out some notes describing how to estimate the important
frequencies in understanding the adiabatic criterion:
\begin{equation}
    \abs{\rd{\ev{\uv{\Omega}_{\rm e}}}{t}}_{\star}
        \ll \abs{\ev{\Omega_{\rm e}}}_{\star}.\label{eq:ad_constr}
\end{equation}
The $\star$ subscript denotes evaluation at $\mathcal{A} \simeq 1$, where
\begin{equation}
    \mathcal{A} \equiv \frac{\Omega_{\rm SL}}{\abs{\dot{\Omega}}}.
\end{equation}
I use $\ev{\uv{\Omega}_{\rm e}}$ to denote the LK-averaged ($N = 0$) component
of the effective spin angular frequency.

A useful insight is that the eccentricity during the LK-driven inspiral has two
phases of evolution:
\begin{description}
    \item[Phase I] The eccentricity undergoes significant excursions over the
        course of an LK cycle. During this phase, $e_{\max}$ and $a$ decrease very
        slowly under GW, while $e_{\min}$ increases relatively quickly due to
        combined GR effects.

    \item[Phase II] The eccentricity $e$ is \emph{frozen} at a fixed value over
        the course of an LK cycle (``suppressed by GR''), and both $e$ and $a$
        decrease under GW, while $I$ is fixe.
\end{description}

In the Paper II regime, $\mathcal{A} \simeq 1$ occurs in Phase II, where the
eccentricity is already frozen and only undergoes GW-driven evolution. In this
limit, we can make very accurate predictions about the relevant frequencies
towards evaluating Eq.~\eqref{eq:ad_constr}.

\subsection{Estimating Frequencies assuming Phase II Inspiral}

We adopt the same notation as Dong but derive much tighter scaling relations and
bounds. Define $j \equiv \sqrt{1 - e^2}$, and $j_\star$ is evaluated at
$\mathcal{A} \simeq 1$. All quantities in this section are LK-averaged, so we
omit brackets.

Note that a simple and symmetric form may be derived for $I_{\rm e}$ as follows:
\begin{align}
    \bm{\Omega}_{\rm e} &= -\dot{\Omega} \uv{z} + \Omega_{\rm SL}
            L_{\rm in},\\
        &= -\dot{\Omega} \uv{z} + \Omega_{\rm SL}\cos I \uv{z}
            + \Omega_{\rm SL} \sin I \uv{x},\\
    \uv{\Omega}_{\rm e} &\equiv \cos I_{\rm e} \uv{z}
        + \sin I_{\rm e} \uv{x},\\
    0 &= \abs{\bm{\Omega}_{\rm e} \times \uv{\Omega}_{\rm e}},\\
        &= -\dot{\Omega} \sin I_{\rm e} + \Omega_{\rm SL}\sin
            \p{I + I_{\rm e}}.
\end{align}
Assuming $\dot{I} = 0$ in Phase II (numerically confirmed), this can be
differentiated:
\begin{align}
    \dot{I}_{\rm e}\p{-\dot{\Omega} \cos I_{\rm e}
            + \Omega_{\rm SL} \cos \p{I + I_{\rm e}}}
        &= \ddot{\Omega} \sin I_{\rm e} - \dot{\Omega}_{\rm SL} \sin
            \p{I + I_{\rm e}},\\
    \dot{I}_{\rm e}\p{-\cot I_{\rm e} + \cot \p{I + I_{\rm e}}}
        &= \rd{\p{\ln \abs{\dot{\Omega}}}}{t}
            - \rd{\p{\ln \Omega_{\rm SL}}}{t},\\
        &= -\rd{}{t}\p{\ln \mathcal{A}} < 0.
\end{align}
The sign convention on $I_{\rm e}$ depends on whether $I < 90^\circ$ or $I >
90^\circ$, but it is clear that $\dot{I}_{\rm e}$ is almost always zero except
when $\mathcal{A} = \mathcal{A}_0 / j_{\star} = 1$ as anticipated.

For instance, specialize to $\dot{\Omega} > 0$, which corresponds to $I >
90^\circ$. Then, $I_{\rm e}$ evolves from $0$\footnote{$I_{\rm e}$ is not
exactly zero originally, as for any nonzero $\Omega_{\rm SL}$, $\bm{\Omega}_{\rm
e}$ is not exactly coincident with $\uv{z}$.} to $180^\circ - I < 90^\circ$,
$\cot I_{\rm e} > 0$ while $\cot (I + I_{\rm e}) < 0$, and indeed $\dot{I}_{\rm
e} > 0$. Finally, evaluating at $\mathcal{A} \simeq 1$ just amounts to
evaluating at $I_{\rm e} \simeq (180^\circ - I) / 2$, and we can say
\begin{equation}
    \dot{I}_{\rm e, \max} \simeq \frac{1}{2 \cot \p{I_{\rm e} / 2}}
        \p{\rd{\ln \mathcal{A}}{t}}_{\mathcal{A} \simeq 1}.
\end{equation}
Since, during Phase II, only $a$ and $e$ change under GW radiation, and
$\dot{I}_{\rm e}$ only scales with the \emph{log} of these quantities, we can
make quite a robust prediction for $\dot{I}_{\rm e}$. In particular, taking the
logarithm of $\Omega_{\rm SL} / \abs{\dot{\Omega}}$ and only keeping scalings
with $a$ and $j = \sqrt{1 - e^2}$ [units of time are $t_{\rm LK, 0}$, the
initial LK time, and $a$ is normalized to its initial value ($a_0 = 1$)]:
\begin{align}
    \p{\rd{\p{\ln a}}{t}}_{\rm GW} &= -\epsilon_{\rm GW}
        \frac{64}{5}\frac{1 + 73e^2/24 + 37e^4/96}{a^4\p{1 - e^2}^{7/2}},\\
        &\approx -\epsilon_{\rm GW} \frac{64}{5}\frac{4}{a^4j^{7/2}},\\
    \p{\rd{\p{\ln (1 - e^2)}}{t}}_{\rm GW}
        &= \epsilon_{\rm GW}\frac{608}{15} \frac{e^2\p{1 + 121e^2/304}}{
            a^4(1 - e^2)^{7/2}},\\
        &\approx \epsilon_{\rm GW}\frac{608}{15}\frac{1}{a^4j^{7/2}},\\
    \Omega_{\rm SL} &= \frac{\epsilon_{\rm SL}}{a^{5/2}j^2},\\
    \dot{\Omega} &= \frac{3a^{3/2}}{4}\frac{\cos I\p{5e^2 \cos^2\omega
            - 4e^2 - 1}}{\sqrt{1 - e^2}},\\
        &\approx \frac{3a^{3/2}}{2j},\\
    \dot{I}_{\rm e, \max} &\simeq \frac{1}{2\cot \p{I_{\rm e} / 2}}
            \p{-4\rd{\p{\ln a}}{t} - \frac{1}{2}\rd{\p{\ln (1 - e^2)}}{t}}_{
            \mathcal{A} = 1},\\
        &\approx 200\frac{\epsilon_{\rm GW}}{a_\star^4j_\star^7}.
\end{align}

To simplify this, we next impose two constraints: $\Omega_{\rm SL}(a_\star,
j_\star) = \abs{\dot{\Omega}\p{a_\star, j_\star}}$, and $j_{\star} \propto
j_{\min} \equiv f j_{\min}$, some scaling ansatz. These two constraints together
give
\begin{align}
    \mathcal{A} = 1 &\approx \frac{3}{2}\frac{a_\star^4 j_\star}{
            \epsilon_{\rm SL}},\\
    a_\star &\approx \p{\frac{2\epsilon_{\rm SL}/3}{fj_{\min}}}^{1/4}.
\end{align}
This gives us the two primary predictions for this calculation:
\begin{align}
    \dot{I}_{\rm e, \max} &= 200\frac{1}{2\cot \p{I_{\rm e} / 2}}
        \frac{\epsilon_{\rm GW}}{\epsilon_{\rm SL}}
        \frac{1}{(f j_{\min})^6}\label{eq:result1},\\
    \Omega_{\rm e,\star} = \abs{\dot{\Omega}} \cos (I/2)
        + \Omega_{\rm SL}\cos (I/2)
        &= \frac{\epsilon_{\rm SL}^{3/8}}{(f j_{\min})^{11/8}}
            2\cos (I/2).\label{eq:result2}
\end{align}
Note that $j_{\min} = \sqrt{\frac{5 \cos^2 I_0}{3}}$. Numerically, we find that
$f \approx 2.2$ fits the data well. While there is a very steep dependence on
$f$, note that $f \geq 1$, so it is a reasonably well constrained parameter. We
present the agreement below in Fig.~\ref{fig:good_quants}, for which $I_0$ is
varied and $I \approx 125^\circ$ is used throughout.
\begin{figure}
    \centering
    \includegraphics[width=0.7\textwidth]{plots/good_quants}
    \caption{Plot of $\Omega_{\rm e}$ and $\dot{I}_{\rm e}$ at $\mathcal{A}
    \simeq 1$, with Eqs.~\eqref{eq:result1} and~\eqref{eq:result2} overlaid.
    While the constant prefactors are not a perfect match, the scaling is very
    good. Top axis labels merger time, for reference.}\label{fig:good_quants}
\end{figure}

\subsection{Attempt at Predicting $\Delta \theta_{\rm e}$}

Since we know $\dot{I}_{\rm e, \max}$, as well as the total range over which
$I_{\rm e}$ evolves ($0 \to 180 - I$, or, for our particular simulations, $10
\to 55$), we can model $\dot{I}_{\rm e}(t)$ as a Gaussian with width
\begin{equation}
    \sigma_{\rm e} = \frac{\Delta I_{\rm e}}{\dot{I}_{\rm e, \max} \sqrt{2\pi}},
\end{equation}
where $\Delta I_{\rm e}$ is the total change in $I_{\rm e}$.

Following the calculation given in the last week, this can yield an approximate
estimate for $\Delta \theta_{\rm e}$ for some arbitrary initial phase $\phi_0$
as:
\begin{align}
    \Delta \theta_{\rm e} &\approx \int\limits_{-\infty}^\infty
        \dot{I}_{\rm e, \max} \exp\s{
            -\frac{(t - t_\star)^2}{2\sigma_{\rm e}^2}} \cos \p{\Omega_{\rm e}t
            + \phi_0}\;\mathrm{d}t.
\end{align}
If we assume $\Omega_{\rm e} \approx \Omega_{\rm e, \star}$, this can easily be
evaluated
\begin{equation}
    \Delta \theta_{\rm e} \approx \Delta I_{\rm e}
        e^{-\Omega_{\rm e}^2 \sigma_{\rm e}^2 / 2} \cos
        \phi_0.\label{eq:exp_law}
\end{equation}
Using Eqs.~\eqref{eq:result1} and~\eqref{eq:result2}, the scaling with $I_0$ can
be evaluated; it's important to see that the deviation from adiabaticity decays
exponentially, in agreement with Landau \& Lifshitz (Vol 1, Eq.~51.6).

This does \emph{not} reproduce the simulation results, however, as the data
suggest a power law decay, see Fig.~\ref{fig:deviations}. Blue dots are explicit
integrations of the Gaussian approximation to $\dot{I}_{\rm e}$ against the real
$\phi_{\rm e}$, i.e.
\begin{equation}
    \Delta \theta_{\rm e, blue} = \int\limits_0^{t_{\rm f}}
        \dot{I}_{\rm e, \max} \exp\s{
            -\frac{(t - t_\star)^2}{2\sigma_{\rm e}^2}}
            \cos \p{\phi_{\rm e}}\;\mathrm{d}t.\label{eq:blue}
\end{equation}
It is clear that this semi-analytic intergation yields the correct scalings,
albeit off by a constant factor, and so treating $\Omega_{\rm e}$ as a constant
predicts much better conservation than expected (though the deviation is
$\lesssim 1^\circ$). Note that I use the real $\phi_{\rm e}$, as any sort of
smoothing seems to introduce too much numerical noise, somehow.
\begin{figure}
    \centering
    \includegraphics[width=0.8\textwidth]{plots/deviations_one_agree.png}
    \caption{As a function of $I_0 - 90^\circ$: (red/black dots) $\Delta
    \theta_{\rm e}$ for different spin realizations at a fixed $I_0$ (spin
    initial condition is varied), where red denotes negative. Blue line is the
    maximum permitted $\Delta \theta_{\rm e}$ permitted by
    Eq.~\eqref{eq:exp_law}, which can be clearly seen to be a poor fit to the
    data. Blue dots are integrations using the true phases from the data, as
    given by Eq.~\eqref{eq:blue}. Agreement is poor for $I_0 \lesssim
    90.2^\circ$ as the system evolves significantly within LK cycles. Note that
    the right end of the plot is dominated by numerical noise, as simulations
    towards the right merge more slowly and accumulate more numerical error; a
    low precision was used for these simulations.}\label{fig:deviations}
\end{figure}

A second attempt can be made by allowing a $\dot{\Omega}_{\rm e}$ term; this
produces estimate
\begin{align}
    \Delta \theta_{\rm e} &\approx \Re \int\limits_{-\infty}^\infty
        \dot{I}_{\rm e, \max} \exp\s{
            -\frac{(t - t_\star)^2}{2\sigma_{\rm e}^2} + i
            \p{\Omega_{\rm e,\star} + \dot{\Omega}_{\rm e}(t - t_\star)} t
            + \phi_0}\;\mathrm{d}t,\\
        &\equiv \Re \int\limits_{-\infty}^\infty
        \dot{I}_{\rm e, \max} \exp\s{
            - a\tau^2 + i\Omega_{\rm e,\star}
            + \phi_0}\;\mathrm{d}\tau,\\
        &\approx \frac{\Delta I}{\sqrt{2\sigma^2 a}}
            \exp\s{-\p{8\sigma^2\rd{\ln \Omega_{\rm e}}{t}}^{-1}},\\
        &\sim \frac{\Delta I}{\sqrt{2 \sigma_{\rm e} {\Omega}_{\rm e,\star}f'}}
            \exp\s{-\p{8\p{1 - 1/f'}}
            ^{-1}}.
\end{align}
The last approximation is in the limit $\dot{\Omega}\sigma^2 \gg 1$, and
approximating $\sigma_{\rm e}\rdil{\ln \Omega_{\rm e}}{t} \simeq f'$, where $f'
\sim 1/8$ according to our estimates. It's clear $f'$ scales as a constant,
since
\begin{align}
    \sigma \rd{\ln \Omega_{\rm e}}{t} &\simeq \frac{\Delta I}{\sqrt{2\pi}}
        \p{\frac{1}{2\cot I_{\rm e}}\rd{\p{\ln (\Omega_{\rm
            SL}/\abs{\dot{\Omega}})}}{t}}^{-1} \rd{\ln \Omega_{\rm e}}{t},\\
        &\sim \frac{\Delta I (2\cot I_{\rm e})}{\sqrt{2\pi}}
            \s{\frac{\rdil{\ln \Omega_{\rm e}}{t}}{
            \rdil{\p{\ln (\Omega_{\rm SL}/\abs{\dot{\Omega}})}}{t}}}.
\end{align}
The first term is a constant, and the second term is a quotient of a bunch of
GW-decay terms with identical scalings.

Obviously, the result is a highly sensitive function of $f'$, but the
qualitative behavior may be in agreement with the tail of the distribution? See
Fig.~\ref{fig:deviations}. This model correctly predicts a power law, though
there is an intermediate regime with a clear power law dependence that falls
under neither of the two models considered here. The integration using the exact
$\phi_{\rm e}$ seems to capture the correct power law scope, surprisingly,
despite its very jumpy nature, so maybe that is responsible for the actual decay
law. In any case, it is clear that it is easy to break exponential convergence
to zero, so we shouldn't be troubled by the ``unexpectedly poor'' conservation
of adiabatic invariant.

\end{document}

