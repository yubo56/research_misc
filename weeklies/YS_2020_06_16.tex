    \documentclass[11pt,
        usenames, % allows access to some tikz colors
        dvipsnames % more colors: https://en.wikibooks.org/wiki/LaTeX/Colors
    ]{article}
    \usepackage{
        amsmath,
        amssymb,
        fouriernc, % fourier font w/ new century book
        fancyhdr, % page styling
        lastpage, % footer fanciness
        hyperref, % various links
        setspace, % line spacing
        amsthm, % newtheorem and proof environment
        mathtools, % \Aboxed for boxing inside aligns, among others
        float, % Allow [H] figure env alignment
        enumerate, % Allow custom enumerate numbering
        graphicx, % allow includegraphics with more filetypes
        wasysym, % \smiley!
        upgreek, % \upmu for \mum macro
        listings, % writing TrueType fonts and including code prettily
        tikz, % drawing things
        booktabs, % \bottomrule instead of hline apparently
        xcolor, % colored text
        cancel % can cancel things out!
    }
    \usepackage[margin=1in]{geometry} % page geometry
    \usepackage[
        labelfont=bf, % caption names are labeled in bold
        font=scriptsize % smaller font for captions
    ]{caption}
    \usepackage[font=scriptsize]{subcaption} % subfigures

    \newcommand*{\scinot}[2]{#1\times10^{#2}}
    \newcommand*{\dotp}[2]{\left<#1\,\middle|\,#2\right>}
    \newcommand*{\rd}[2]{\frac{\mathrm{d}#1}{\mathrm{d}#2}}
    \newcommand*{\pd}[2]{\frac{\partial#1}{\partial#2}}
    \newcommand*{\rdil}[2]{\mathrm{d}#1 / \mathrm{d}#2}
    \newcommand*{\pdil}[2]{\partial#1 / \partial#2}
    \newcommand*{\rtd}[2]{\frac{\mathrm{d}^2#1}{\mathrm{d}#2^2}}
    \newcommand*{\ptd}[2]{\frac{\partial^2 #1}{\partial#2^2}}
    \newcommand*{\md}[2]{\frac{\mathrm{D}#1}{\mathrm{D}#2}}
    \newcommand*{\pvec}[1]{\vec{#1}^{\,\prime}}
    \newcommand*{\svec}[1]{\vec{#1}\;\!}
    \newcommand*{\bm}[1]{\boldsymbol{\mathbf{#1}}}
    \newcommand*{\uv}[1]{\hat{\bm{#1}}}
    \newcommand*{\ang}[0]{\;\text{\AA}}
    \newcommand*{\mum}[0]{\;\upmu \mathrm{m}}
    \newcommand*{\at}[1]{\left.#1\right|}
    \newcommand*{\bra}[1]{\left<#1\right|}
    \newcommand*{\ket}[1]{\left|#1\right>}
    \newcommand*{\abs}[1]{\left|#1\right|}
    \newcommand*{\ev}[1]{\langle#1\rangle}
    \newcommand*{\p}[1]{\left(#1\right)}
    \newcommand*{\s}[1]{\left[#1\right]}
    \newcommand*{\z}[1]{\left\{#1\right\}}

    \newtheorem{theorem}{Theorem}[section]

    \let\Re\undefined
    \let\Im\undefined
    \DeclareMathOperator{\Res}{Res}
    \DeclareMathOperator{\Re}{Re}
    \DeclareMathOperator{\Im}{Im}
    \DeclareMathOperator{\Log}{Log}
    \DeclareMathOperator{\Arg}{Arg}
    \DeclareMathOperator{\Tr}{Tr}
    \DeclareMathOperator{\E}{E}
    \DeclareMathOperator{\Var}{Var}
    \DeclareMathOperator*{\argmin}{argmin}
    \DeclareMathOperator*{\argmax}{argmax}
    \DeclareMathOperator{\sgn}{sgn}
    \DeclareMathOperator{\diag}{diag\;}

    \colorlet{Corr}{red}

    % \everymath{\displaystyle} % biggify limits of inline sums and integrals
    \tikzstyle{circ} % usage: \node[circ, placement] (label) {text};
        = [draw, circle, fill=white, node distance=3cm, minimum height=2em]
    \definecolor{commentgreen}{rgb}{0,0.6,0}
    \lstset{
        basicstyle=\ttfamily\footnotesize,
        frame=single,
        numbers=left,
        showstringspaces=false,
        keywordstyle=\color{blue},
        stringstyle=\color{purple},
        commentstyle=\color{commentgreen},
        morecomment=[l][\color{magenta}]{\#}
    }

\begin{document}

\def\Snospace~{\S{}} % hack to remove the space left after autorefs
\renewcommand*{\sectionautorefname}{\Snospace}
\renewcommand*{\appendixautorefname}{\Snospace}
\renewcommand*{\figureautorefname}{Fig.}
\renewcommand*{\equationautorefname}{Eq.}
\renewcommand*{\tableautorefname}{Tab.}

I checked DL's notes, and our scalings for $\dot{I}_{\rm e} / \Omega_{\rm e}$ as
well as $\Omega_{\rm e}$ at $\bar{A} \simeq 1$ agree. I will use DL notation
when writing it up.

In this particular parameter regime, $A \simeq 1$ is not in the completely
frozen regime, but it is also not in the oscillating regime as defined by DL's
notes, as $e_{\max} - e_{\min} \ll 1$. I don't think the distinction ends up
mattering for scaling purposes though.

\section{Requested Plots}

I have made many of these, but I attach just the ones for $I_0 = 90.5^\circ$
below, see Fig.~\ref{fig:plots}. Of note:
\begin{itemize}
    \item $\dot{\bar{I}}_{\rm e}$ is very smooth.

    \item $I_{\rm e}$ still nutates rather significantly at $A \simeq 1$.

    \item $T_{\rm LK}$ is defined even in the $e$-frozen regime as $\pi / T_{\rm
        \omega}$, where $T_{\rm \omega}$ is the period of the $\omega$ orbital
        element.

    \item In panel 6, it is clear $\max \dot{\bar{I}}_{\rm e} / \Omega_{\rm e}$
        greatly overpredicts the final $\Delta \theta_{\rm e}$ as expected, and
        that there is significant damping of fluctuations psat the maximum
        deviation (``narrowing'' as described before, and ``cancellations'' in
        the DL explicit solution).

        Interestingly, for $I_0 \gtrsim 90.35^\circ$, the shape of $\Delta
        \theta_{\rm e}$ does not change any more (red dots) even as
        $\dot{\bar{I}}_{\rm e} / \Omega_{\rm e}$ continues to decrease with
        increasing $I_0$. This suggests some other mechanism is sustaining these
        oscillations. Note though that this does not affect the \emph{final}
        $\Delta \theta_{\rm e}$, which decreases with increasing $I_0$.

    \item Following the results of the next section, the averaging in Panel 6
        should be done over multiple LK periods. We average $\theta_{\rm e}$
        over $4T_{\rm LK}$, following the approximate ratio in Panel 4.
\end{itemize}
\begin{figure}
    \centering
    \includegraphics[width=\textwidth]{plots/200616/4sim_90_500.png}
    \includegraphics[width=\textwidth]{plots/200616/4sim_90_500_zoom.png}
    \caption{Plot of interesting quantities, top 6 panels are for entire
    simulation while bottom 8 are zoomed in near $\bar{A} \simeq 1$. For full
    description, see text.}\label{fig:plots}
\end{figure}

\section{Comment on Averaging Procedure}

Consider the full form of the Hamiltonian
\begin{equation}
    H = \bm{\Omega}_{\rm e} \cdot \uv{S}.
\end{equation}
Here, $\bm{\Omega}_{\rm e}$ is periodic with period $T_{\rm LK}$. Assume
$\uv{S}$ is also periodic with some period $T_{\rm S}$ (e.g.\ $\sim 2\pi /
\bar{\Omega}_{\rm e}$).

In general, these two periods are irrational, but for sufficiently large
integers $p, q$, there will be a period $T$ satisfying
\begin{equation}
    T \approx pT_{\rm LK} \approx q T_{\rm S}.
\end{equation}
Consider averaging the Hamiltonian over interval $T$. Writing (note that
$\uv{S}_M$ must be complex, as $\uv{S}$ is precessing; while $\bm{\Omega}_{\rm
e, N}$ can be made real by choice of $t$)
\begin{align}
    \bm{\Omega}_{\rm e} &= \bar{\bm{\Omega}}_{\rm e} + \sum\limits_{N = 1}^\infty
            \bm{\Omega}_{\rm e, N}\cos \p{\frac{2\pi N t}{T_{\rm LK}}},\\
    \uv{S} &= \s{\ev{\uv{S}} + \sum\limits_{M = 1}^\infty
            \bm{S}_{\rm M}\exp \p{i\frac{2\pi M t}{T_{\rm S}}}},\\
    \frac{1}{T}\int\limits_0^T H\;\mathrm{d}t
        &= \frac{1}{T}\int\limits_0^T
            \s{\bar{\bm{\Omega}}_{\rm e} + \sum\limits_{N = 1}^\infty
            \bm{\Omega}_{\rm e, N}\cos \p{\frac{2\pi N qt}{T}}}
            \cdot \s{\ev{\uv{S}} + \sum\limits_{M = 1}^\infty
            \bm{S}_{\rm M}\exp \p{i\frac{2\pi M pt}{T}}}\;\mathrm{d}t,\\
    \ev{H} &= \bar{\bm{\Omega}}_{\rm e} \cdot \ev{\bm{\uv{S}}}
            + \frac{1}{2}\sum\limits_{j = 1}^\infty
                \bm{\Omega}_{\rm e, jp} \cdot \left(\Re\bm{S}_{jq}\right).
                \label{eq:exp_h}
\end{align}
When the terms in the summation can be neglected, this reduces to the claim we
have made: that $\ev{\bar{\bm{\Omega}}_{\rm e} \cdot \uv{S}}$ is an adiabatic
invariant, since
\begin{equation}
    A \equiv \oint \cos \theta_{\rm e} \;\mathrm{d}\phi_{\rm e}
        \approx \Omega_{\rm e} \ev{\cos \theta_{\rm e}}.
\end{equation}

This argument suggests that the correct timescale to average over is $T$, a
near-integer multiple of both $T_{\rm LK}$ and $T_{\rm S} \simeq 2\pi /
\bar{\bm{\Omega}}_{\rm e}$.

Indeed, when using a grid of high-precision $I_0 = 90.5^\circ$ simulations,
the maximum $\Delta \theta_{\rm e}$ goes down by a factor of three when using $T
= 2T_{\rm LK}$ (see Fig.~\ref{fig:devsgrid}). Initially, $T_{\rm LK} \approx
0.4706\p{2\pi / \bar{\Omega}_{\rm e}}$.
\begin{figure}
    \centering
    \includegraphics[width=0.45\textwidth]{plots/200616/devsgrid_avg1.png}
    \includegraphics[width=0.45\textwidth]{plots/200616/devsgrid.png}
    \caption{Averaging over $T_{\rm LK}$ and $2T_{\rm LK}$
    respectively.}\label{fig:devsgrid}
\end{figure}

I suspect there is a good reason the summed terms in Eq.~\eqref{eq:exp_h} can be
neglected: $\ev{\uv{S}} \parallel \bar{\bm{\Omega}}_{\rm e}$, $\bm{S}_{M} \perp
\ev{\uv{S}}$ while $\bm{\Omega}_{\rm e, N} \parallel \bar{\bm{\Omega}}_{\rm e}$
(only when the nutation of $\bm{\Omega}_{\rm e}$ is negligible), naively. I
haven't been able to check whether this works yet. If the above claim is true,
then conservation of $\theta_{\rm e}$ depends on how much $\bar{I}_{\rm e}$ is
nutating when $A \simeq 1$.

\end{document}

