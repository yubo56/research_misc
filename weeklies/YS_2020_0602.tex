    \documentclass[11pt,
        usenames, % allows access to some tikz colors
        dvipsnames % more colors: https://en.wikibooks.org/wiki/LaTeX/Colors
    ]{article}
    \usepackage{
        amsmath,
        amssymb,
        fouriernc, % fourier font w/ new century book
        fancyhdr, % page styling
        lastpage, % footer fanciness
        hyperref, % various links
        setspace, % line spacing
        amsthm, % newtheorem and proof environment
        mathtools, % \Aboxed for boxing inside aligns, among others
        float, % Allow [H] figure env alignment
        enumerate, % Allow custom enumerate numbering
        graphicx, % allow includegraphics with more filetypes
        wasysym, % \smiley!
        upgreek, % \upmu for \mum macro
        listings, % writing TrueType fonts and including code prettily
        tikz, % drawing things
        booktabs, % \bottomrule instead of hline apparently
        xcolor, % colored text
        cancel % can cancel things out!
    }
    \usepackage[margin=1in]{geometry} % page geometry
    \usepackage[
        labelfont=bf, % caption names are labeled in bold
        font=scriptsize % smaller font for captions
    ]{caption}
    \usepackage[font=scriptsize]{subcaption} % subfigures

    \newcommand*{\scinot}[2]{#1\times10^{#2}}
    \newcommand*{\dotp}[2]{\left<#1\,\middle|\,#2\right>}
    \newcommand*{\rd}[2]{\frac{\mathrm{d}#1}{\mathrm{d}#2}}
    \newcommand*{\pd}[2]{\frac{\partial#1}{\partial#2}}
    \newcommand*{\rdil}[2]{\mathrm{d}#1 / \mathrm{d}#2}
    \newcommand*{\pdil}[2]{\partial#1 / \partial#2}
    \newcommand*{\rtd}[2]{\frac{\mathrm{d}^2#1}{\mathrm{d}#2^2}}
    \newcommand*{\ptd}[2]{\frac{\partial^2 #1}{\partial#2^2}}
    \newcommand*{\md}[2]{\frac{\mathrm{D}#1}{\mathrm{D}#2}}
    \newcommand*{\pvec}[1]{\vec{#1}^{\,\prime}}
    \newcommand*{\svec}[1]{\vec{#1}\;\!}
    \newcommand*{\bm}[1]{\boldsymbol{\mathbf{#1}}}
    \newcommand*{\uv}[1]{\hat{\bm{#1}}}
    \newcommand*{\ang}[0]{\;\text{\AA}}
    \newcommand*{\mum}[0]{\;\upmu \mathrm{m}}
    \newcommand*{\at}[1]{\left.#1\right|}
    \newcommand*{\bra}[1]{\left<#1\right|}
    \newcommand*{\ket}[1]{\left|#1\right>}
    \newcommand*{\abs}[1]{\left|#1\right|}
    \newcommand*{\ev}[1]{\langle#1\rangle}
    \newcommand*{\p}[1]{\left(#1\right)}
    \newcommand*{\s}[1]{\left[#1\right]}
    \newcommand*{\z}[1]{\left\{#1\right\}}

    \newtheorem{theorem}{Theorem}[section]

    \let\Re\undefined
    \let\Im\undefined
    \DeclareMathOperator{\Res}{Res}
    \DeclareMathOperator{\Re}{Re}
    \DeclareMathOperator{\Im}{Im}
    \DeclareMathOperator{\Log}{Log}
    \DeclareMathOperator{\Arg}{Arg}
    \DeclareMathOperator{\Tr}{Tr}
    \DeclareMathOperator{\E}{E}
    \DeclareMathOperator{\Var}{Var}
    \DeclareMathOperator*{\argmin}{argmin}
    \DeclareMathOperator*{\argmax}{argmax}
    \DeclareMathOperator{\sgn}{sgn}
    \DeclareMathOperator{\diag}{diag\;}

    \colorlet{Corr}{red}

    % \everymath{\displaystyle} % biggify limits of inline sums and integrals
    \tikzstyle{circ} % usage: \node[circ, placement] (label) {text};
        = [draw, circle, fill=white, node distance=3cm, minimum height=2em]
    \definecolor{commentgreen}{rgb}{0,0.6,0}
    \lstset{
        basicstyle=\ttfamily\footnotesize,
        frame=single,
        numbers=left,
        showstringspaces=false,
        keywordstyle=\color{blue},
        stringstyle=\color{purple},
        commentstyle=\color{commentgreen},
        morecomment=[l][\color{magenta}]{\#}
    }

\begin{document}

\def\Snospace~{\S{}} % hack to remove the space left after autorefs
\renewcommand*{\sectionautorefname}{\Snospace}
\renewcommand*{\appendixautorefname}{\Snospace}
\renewcommand*{\figureautorefname}{Fig.}
\renewcommand*{\equationautorefname}{Eq.}
\renewcommand*{\tableautorefname}{Tab.}

\section{Theory}

As Dong correctly pointed out last week, I left out a term (and flipped a sign),
and the correct equations of motion are
\begin{align}
    \rd{\theta}{t} &= -\dot{I}_{\rm e} \cos \phi\label{eq:dq},\\
    \rd{\phi}{t} &= \Omega_{\rm e} + \dot{I}_{\rm e} \cot \theta \sin
        \phi\label{eq:dphi}.
\end{align}
This system of equations is useful because $\dot{I}_{\rm e}$ obeys the following
ODE (note that $I$ is the LK-averaged $I$ and turns out to be very nearly
constant, so its derivative is dropped):
\begin{align}
    -\dot{\Omega} \sin I_{\rm e} + \Omega_{SL}\sin \p{I + I_{\rm e}}
        &= 0\label{eq:I_out_def},\\
    \dot{I}_{\rm e}\p{-\dot{\Omega}\cos I_{\rm e}
        + \Omega_{\rm SL}\cos\p{I + I_{\rm e}}}
        &= \ddot{\Omega} \sin I_{\rm e} - \dot{\Omega}_{\rm SL}
            \sin\p{I + I_{\rm e}},\\
        &= \ddot{\Omega}\frac{\Omega_{\rm SL}\sin\p{I + I_{\rm
            e}}}{\dot{\Omega}} - \dot{\Omega}_{\rm SL}\sin\p{I + I_{\rm e}},\\
    \dot{I}_{\rm e}\p{-\cot I_{\rm e} + \cot \p{I + I_{\rm e}}}
        &= \rd{}{t}\p{\ln \p{-\dot{\Omega}} - \ln
        \Omega_{\rm SL}}.\label{eq:idot_out_sol}
\end{align}
The coefficient of $\dot{I}_{\rm e}$ is almost always very large except when
$\pi - I_{\rm e} \approx I + I_{\rm e}$, where it is $\sim -2$ (negative when $I
> 90^\circ$), and $-\dot{\Omega} \simeq \Omega_{\rm SL}$. Thus, we find
\begin{equation}
    \max \dot{I}_{\rm e} \simeq
        \frac{1}{2}\rd{}{t}\p{\frac{\Omega_{\rm
        SL}}{-\dot{\Omega}}} \equiv
        \frac{1}{2}\p{\dot{\mathcal{A}}}_{\mathcal{A} = 1}.\label{eq:ie_max}
\end{equation}
Here, $\mathcal{A} \equiv \Omega_{\rm SL} / (-\dot{\Omega})$ is the adiabaticity
parameter from LL18. We omit the subscript going forwards.

Furthermore, since $\dot{I}_{\rm e}$ must integrate to $\min(I, \pi/2 - I)$,
this provides a constraint on the width of $\dot{I}_{\rm e}(t)$, and we can
write down
\begin{equation}
    \dot{I}_{\rm e} \simeq \frac{\dot{\mathcal{A}}}{2}
        \exp\s{-\frac{t^2}{(2 I / \dot{\mathcal{A}})^2 /\pi}}
\end{equation}
As can be seen in the figure shown in the previous writeup, $\dot{I}_{\rm e}$ is
indeed peaked, though somewhat asymmetric (Fig.~\ref{fig:phidots}).
\begin{figure}
    \centering
    \includegraphics[width=0.45\textwidth]{plots/4sim_90_500_phidots.png}
    \caption{Plot of precession frequencies for $I_0 = 90.5^\circ$, ``slow
    merger''. $\dot{I}_{\rm e} \equiv \abs{\rdil{\uv{\Omega}_{\rm e,0}}{t}}$
    is shown in blue.}\label{fig:phidots}
\end{figure}
If we drop the $\dot{I}_{\rm e}$ contribution in Eq.~\eqref{eq:dphi} and assume
$\Omega_{\rm e}$ is approximately constant, the integral that determines $\Delta
\theta_{\rm e}$ is then
\begin{align}
    \Delta \theta_{\rm e} &= \int\limits_{-\infty}^\infty
            -\dot{I}_{\rm e}e^{i(\phi_0 + \Omega_{\rm e}t)}\;\mathrm{d}t,\\
        &= -I e^{i\phi_0}
        % e^{-(1/s^2) * (t^2 - iWeff ts^2 - Weff^2s^4/)}
            \exp\s{-\frac{\Omega_{\rm e}^2I^2}{\pi \dot{\mathcal{A}}^2}}.
\end{align}
Note the dependence on $\phi_0 = \phi_{\rm sb, 0}$ and the lack of dependence on
$\theta_{\rm sb, 0}$: this are both results that agree with simulations (see
e.g. Fig.~\ref{fig:ensemble_phase}).
\begin{figure}
    \centering
    \includegraphics[width=0.6\textwidth]{plots/ensemble_phase.png}
    \caption{Old plot of the difference between the final spin orbit
    misalignment angle and the predicted one, as a function of inclination
    $I_0$. Legend labels are $(\theta_{\rm sl}^i, \phi_{\rm sb}^i)$; the weird
    coordinates are an artifact of being an old plot, but the dependence on
    $\phi_{\rm sb}^i \approx \phi_{\rm e}^i$ and not on $\theta_{\rm sl}^i
    \approx I - \theta_{\rm e}^i$ is clear.}\label{fig:ensemble_phase}
\end{figure}

Caveats:
\begin{itemize}
    \item Of course, a treatment with non-constant $\Omega_{\rm e}$ is possible,
        and a first order correction $\phi = \phi_0 + \Omega_{\rm e}t +
        \dot{\Omega}_{\rm e}t^2/2$ simply changes the width of the Gaussian a
        bit.

    \item The approximation of dropping $\dot{I}_{\rm e}$ in Eq.~\eqref{eq:dphi}
        is less legal. However, if $\phi$ is fast-varying (requiring
        $\Omega_{\rm e} \geq \dot{I}_{\rm e}$), then dropping $\dot{I}_{\rm e}$
        should be a very good approximation, as $\sin \phi$ should significantly
        suppress the contribution. The $\cot \theta$ enhancement however
        illustrates that, regardless of how much $\Omega_{\rm e} \gg
        \dot{I}_{\rm e}$, there will always be some IC for which $\theta_{\rm
        e}$ can evolve significantly.
\end{itemize}

\section{Comparison to Simulations}

This section is incomplete, I plan to finish it tomorrow morning (numerically
exploring the effective quantities I defined above). I will at least discuss the
three plots below.
\begin{figure}
    \centering
    \includegraphics[width=0.6\textwidth]{plots/devsgrid.png}
    \includegraphics[width=0.6\textwidth]{plots/devshist.png}
    \caption{Top: Plot of the deviations $\Delta \theta_{\rm e}$ as a function
    of initial spin vector orientation, $\theta_{\rm e}$ and $\phi_{\rm e}$.
    This plot is corrected from last week. Two notable outliers were found,
    though they've been omitted from the colorbar for clarity (shown later). The
    distribution of the $400$ values is shown in the bottom
    panel.}\label{fig:devsgrid}
\end{figure}

\begin{figure}
    \centering
    \includegraphics[width=\textwidth]{plots/4sim_qsl87_phi_sb333_90_500_qN0.png}
    \caption{Possible resonant case similar to that observed in LL17, but using
    the parameters from the Paper II regime. The total $\Delta \theta_{\rm e}$
    is $2^\circ$, owing to an instantaneous kick at around $t = 125$. I intend
    to investigate this further next week, but the behavior is similar to that
    seen in LL17.}\label{fig:kick}
\end{figure}

In Fig.~\ref{fig:devsgrid}, we display a histogram of the $\Delta \theta_{\rm
e}$ values, each for an isotropically sampled initial spin condition.
Alternatively, we can numerically compute the integral
\begin{equation}
    \Delta \theta_{\rm guess}(\phi_{\rm e}^i) = \int\limits_0^{t_{\rm f}}
        -\dot{I}_{\rm e} \cos \p{\Omega_{\rm e}t + \phi_{\rm
            e}^i}\;\mathrm{d}t,
\end{equation}
and take the maximum over a few choices of $\phi_{\rm e}$. This does not require
any spin evolutions, only the orbital evolution, and should provide an upper
bound on the maximum $\Delta \theta_{\rm e}$ observed. Indeed, this procedure
estimates $\max \Delta \theta_{\rm e} \approx 0.09$, very consistent with the
histogram, but \emph{failing} to capture the outlier in Fig.~\ref{fig:kick}.
This suggests that the EOM written down for $\dot{\theta}_{\rm e}$ is correct,
though it does not validate the analysis of the previous section.

\end{document}

