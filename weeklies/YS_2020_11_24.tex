    \documentclass[11pt,
        usenames, % allows access to some tikz colors
        dvipsnames % more colors: https://en.wikibooks.org/wiki/LaTeX/Colors
    ]{article}
    \usepackage{
        amsmath,
        amssymb,
        fouriernc, % fourier font w/ new century book
        fancyhdr, % page styling
        lastpage, % footer fanciness
        hyperref, % various links
        setspace, % line spacing
        amsthm, % newtheorem and proof environment
        mathtools, % \Aboxed for boxing inside aligns, among others
        float, % Allow [H] figure env alignment
        enumerate, % Allow custom enumerate numbering
        graphicx, % allow includegraphics with more filetypes
        wasysym, % \smiley!
        upgreek, % \upmu for \mum macro
        listings, % writing TrueType fonts and including code prettily
        tikz, % drawing things
        booktabs, % \bottomrule instead of hline apparently
        xcolor, % colored text
        cancel % can cancel things out!
    }
    \usepackage[margin=1in]{geometry} % page geometry
    \usepackage[
        labelfont=bf, % caption names are labeled in bold
        font=scriptsize % smaller font for captions
    ]{caption}
    \usepackage[font=scriptsize]{subcaption} % subfigures

    \newcommand*{\scinot}[2]{#1\times10^{#2}}
    \newcommand*{\dotp}[2]{\left<#1\,\middle|\,#2\right>}
    \newcommand*{\rd}[2]{\frac{\mathrm{d}#1}{\mathrm{d}#2}}
    \newcommand*{\pd}[2]{\frac{\partial#1}{\partial#2}}
    \newcommand*{\rdil}[2]{\mathrm{d}#1 / \mathrm{d}#2}
    \newcommand*{\pdil}[2]{\partial#1 / \partial#2}
    \newcommand*{\rtd}[2]{\frac{\mathrm{d}^2#1}{\mathrm{d}#2^2}}
    \newcommand*{\ptd}[2]{\frac{\partial^2 #1}{\partial#2^2}}
    \newcommand*{\md}[2]{\frac{\mathrm{D}#1}{\mathrm{D}#2}}
    \newcommand*{\pvec}[1]{\vec{#1}^{\,\prime}}
    \newcommand*{\svec}[1]{\vec{#1}\;\!}
    \newcommand*{\bm}[1]{\boldsymbol{\mathbf{#1}}}
    \newcommand*{\uv}[1]{\hat{\bm{#1}}}
    \newcommand*{\ang}[0]{\;\text{\AA}}
    \newcommand*{\mum}[0]{\;\upmu \mathrm{m}}
    \newcommand*{\at}[1]{\left.#1\right|}
    \newcommand*{\bra}[1]{\left<#1\right|}
    \newcommand*{\ket}[1]{\left|#1\right>}
    \newcommand*{\abs}[1]{\left|#1\right|}
    \newcommand*{\ev}[1]{\left\langle#1\right\rangle}
    \newcommand*{\p}[1]{\left(#1\right)}
    \newcommand*{\s}[1]{\left[#1\right]}
    \newcommand*{\z}[1]{\left\{#1\right\}}

    \newtheorem{theorem}{Theorem}[section]

    \let\Re\undefined
    \let\Im\undefined
    \DeclareMathOperator{\Res}{Res}
    \DeclareMathOperator{\Re}{Re}
    \DeclareMathOperator{\Im}{Im}
    \DeclareMathOperator{\Log}{Log}
    \DeclareMathOperator{\Arg}{Arg}
    \DeclareMathOperator{\Tr}{Tr}
    \DeclareMathOperator{\E}{E}
    \DeclareMathOperator{\Var}{Var}
    \DeclareMathOperator*{\argmin}{argmin}
    \DeclareMathOperator*{\argmax}{argmax}
    \DeclareMathOperator{\sgn}{sgn}
    \DeclareMathOperator{\diag}{diag\;}

    \colorlet{Corr}{red}

    % \everymath{\displaystyle} % biggify limits of inline sums and integrals
    \tikzstyle{circ} % usage: \node[circ, placement] (label) {text};
        = [draw, circle, fill=white, node distance=3cm, minimum height=2em]
    \definecolor{commentgreen}{rgb}{0,0.6,0}
    \lstset{
        basicstyle=\ttfamily\footnotesize,
        frame=single,
        numbers=left,
        showstringspaces=false,
        keywordstyle=\color{blue},
        stringstyle=\color{purple},
        commentstyle=\color{commentgreen},
        morecomment=[l][\color{magenta}]{\#}
    }

\begin{document}

\onehalfspacing

\section{Tertiary-Induced BBH Merger Fraction Simulations}

I completed $a_{\rm out, eff} = [2800, 3600, 4500, 5500, 7000]$, and also am
starting calculations for $a_{\rm out, eff} = [1200, 2000]$ (not yet
completed). The other parameters are:
\begin{align*}[h]
    m_{12} &= 50M_{\odot}, &
    m_3 &= 30M_{\odot}, &
    a_0 &= 100\;\mathrm{AU}, &
    e_0 &= 10^{-3},\\
    q &\in [0.2, 1], &
    e_{\rm out, 0} &\in [0, 0.9], &
    \cos I_0 &\in [\cos 50^\circ, \cos 130^\circ].
\end{align*}
All distributions are uniform. The restricted range in $I_0$ is because the
other regions are never octupole-active (at least not at $a_{\rm out, eff} =
3600$). The resulting merger probabilities as a
function of $q$, defined as
\begin{equation}
    f_{\rm merge} = 100 \times \frac{\cos 50^\circ - \cos 130^\circ}{2}
        \times \frac{\textrm{\# merged}}{\textrm{\# run}},
\end{equation}
are shown in Fig.~\ref{fig:all}
\begin{figure}[h]
    \centering
    \includegraphics[width=0.32\columnwidth]{../scripts/octlk/1popsynth/a2eff2800.png}
    \includegraphics[width=0.32\columnwidth]{../scripts/octlk/1popsynth/a2eff3600.png}
    \includegraphics[width=0.32\columnwidth]{../scripts/octlk/1popsynth/a2eff4500.png}
    \includegraphics[width=0.32\columnwidth]{../scripts/octlk/1popsynth/a2eff5500.png}
    \includegraphics[width=0.32\columnwidth]{../scripts/octlk/1popsynth/a2eff7000.png}
    \caption{Merger probabilities as a function of $q$.}\label{fig:all}
\end{figure}

The cumulative merger probabilities as a function of $a_{\rm out, eff}$ are
shown in Fig.~\ref{fig:merge}
\begin{figure}
    \centering
    \includegraphics[width=0.6\columnwidth]{../scripts/octlk/1popsynth/total.png}
    \caption{Total merger probabilities as a function of $a_{\rm out,
    eff}$.}\label{fig:merge}
\end{figure}

\clearpage

\section{Planet Octupole Simulations}

I ran simulations to determine $e_{\max}(I_0)$, where $I_0$ is the initial
mutual inclination between the planets' orbits. The parameters I used were:
\begin{align*}
    m_1 &= 1M_{\odot}, &
    m_2 = m_3 &= 1M_J, &
    a &= 5\;\mathrm{AU}, &
    a_{\rm out} &= 50\;\mathrm{AU},\\
    e_0 &= 10^{-3}, &
    k_2 &= 0.37, &
    R_2 &= R_J.
\end{align*}
I sampled $I_0 \in [40^\circ, 140^\circ]$, though the upper inclination range
doesn't seem always to be sufficient. I tried among $e_2 = [0.1, 0.3, 0.5, 0.6,
0.8, 0.9]$. I also retried all my simulations with
\begin{align*}
    m_3 &= M_{\odot}, &
    a_{\rm out} &= 500\;\mathrm{AU}.
\end{align*}
This ensures the same quadrupole strength ($\bar{a}_{\rm out, eff}$) but causes
$\eta$ to decrease by a factor of $\approx 3000$, satisfying the test
particle approximation. This is for verification against the MLL16 fitting
formula. The results are shown in Fig.~\ref{fig:p1} and~\ref{fig:p2}. Each
inclination is run for $3$ random choices of $\omega_i, \Omega_i$ angles, while
a total of up to $2000$ inclinations are sampled uniformly. The value of
$e_{\max}$ in these simulations is $1 - e_{\max} \approx 10^{-3}$, in line with
the analytical estimate, suggesting my $\dot{\omega}_{\rm tide}$ is correct.
\begin{figure}
    \centering
    \includegraphics[width=0.37\columnwidth]{../scripts/octlk/1laetitia/e2_1.png}
    \includegraphics[width=0.37\columnwidth]{../scripts/octlk/1laetitia/e2_1tp.png}
    \includegraphics[width=0.37\columnwidth]{../scripts/octlk/1laetitia/e2_3.png}
    \includegraphics[width=0.37\columnwidth]{../scripts/octlk/1laetitia/e2_3tp.png}
    \includegraphics[width=0.37\columnwidth]{../scripts/octlk/1laetitia/e2_5.png}
    \includegraphics[width=0.37\columnwidth]{../scripts/octlk/1laetitia/e2_5tp.png}
    \caption{Part 1. Rows are $e_{\rm out} = [0.1, 0.3, 0.5]$, while columns are
    for the fiducial parameters and for $m_3 = M_{\odot}$. Dots indicate
    $e_{\max}$ when run over $500 t_{\rm LK}$ with apsidal precession (due to
    both GR and tides) but with no dissipation. Vertical lines are fits from
    MLL16. Note that $\epsilon_{\rm oct}$ is larger on the left column by a
    factor of $10$.}\label{fig:p1}
\end{figure}
\begin{figure}
    \centering
    \includegraphics[width=0.37\columnwidth]{../scripts/octlk/1laetitia/e2_6.png}
    \includegraphics[width=0.37\columnwidth]{../scripts/octlk/1laetitia/e2_6tp.png}
    \includegraphics[width=0.37\columnwidth]{../scripts/octlk/1laetitia/e2_8.png}
    \includegraphics[width=0.37\columnwidth]{../scripts/octlk/1laetitia/e2_8tp.png}
    \includegraphics[width=0.37\columnwidth]{../scripts/octlk/1laetitia/e2_9.png}
    \includegraphics[width=0.37\columnwidth]{../scripts/octlk/1laetitia/e2_9tp.png}
    \caption{Part 2. Rows are $e_{\rm out} = [0.6, 0.8, 0.9]$, while columns are
    for the fiducial parameters and for $m_3 = M_{\odot}$. No simulations are
    available for $e_{\rm out} = 0.9$ and $m_3 = M_J$ yet.}\label{fig:p2}
\end{figure}

\end{document}

