    \documentclass[11pt,
        usenames, % allows access to some tikz colors
        dvipsnames % more colors: https://en.wikibooks.org/wiki/LaTeX/Colors
    ]{article}
    \usepackage{
        amsmath,
        amssymb,
        fouriernc, % fourier font w/ new century book
        fancyhdr, % page styling
        lastpage, % footer fanciness
        hyperref, % various links
        setspace, % line spacing
        amsthm, % newtheorem and proof environment
        mathtools, % \Aboxed for boxing inside aligns, among others
        float, % Allow [H] figure env alignment
        enumerate, % Allow custom enumerate numbering
        graphicx, % allow includegraphics with more filetypes
        wasysym, % \smiley!
        upgreek, % \upmu for \mum macro
        listings, % writing TrueType fonts and including code prettily
        tikz, % drawing things
        booktabs, % \bottomrule instead of hline apparently
        xcolor, % colored text
        cancel % can cancel things out!
    }
    \usepackage[margin=1in]{geometry} % page geometry
    \usepackage[
        labelfont=bf, % caption names are labeled in bold
        font=scriptsize % smaller font for captions
    ]{caption}
    \usepackage[font=scriptsize]{subcaption} % subfigures

    \newcommand*{\scinot}[2]{#1\times10^{#2}}
    \newcommand*{\dotp}[2]{\left<#1\,\middle|\,#2\right>}
    \newcommand*{\rd}[2]{\frac{\mathrm{d}#1}{\mathrm{d}#2}}
    \newcommand*{\pd}[2]{\frac{\partial#1}{\partial#2}}
    \newcommand*{\rdil}[2]{\mathrm{d}#1 / \mathrm{d}#2}
    \newcommand*{\pdil}[2]{\partial#1 / \partial#2}
    \newcommand*{\rtd}[2]{\frac{\mathrm{d}^2#1}{\mathrm{d}#2^2}}
    \newcommand*{\ptd}[2]{\frac{\partial^2 #1}{\partial#2^2}}
    \newcommand*{\md}[2]{\frac{\mathrm{D}#1}{\mathrm{D}#2}}
    \newcommand*{\pvec}[1]{\vec{#1}^{\,\prime}}
    \newcommand*{\svec}[1]{\vec{#1}\;\!}
    \newcommand*{\bm}[1]{\boldsymbol{\mathbf{#1}}}
    \newcommand*{\uv}[1]{\hat{\bm{#1}}}
    \newcommand*{\ang}[0]{\;\text{\AA}}
    \newcommand*{\mum}[0]{\;\upmu \mathrm{m}}
    \newcommand*{\at}[1]{\left.#1\right|}
    \newcommand*{\bra}[1]{\left<#1\right|}
    \newcommand*{\ket}[1]{\left|#1\right>}
    \newcommand*{\abs}[1]{\left|#1\right|}
    \newcommand*{\ev}[1]{\left\langle#1\right\rangle}
    \newcommand*{\p}[1]{\left(#1\right)}
    \newcommand*{\s}[1]{\left[#1\right]}
    \newcommand*{\z}[1]{\left\{#1\right\}}

    \newtheorem{theorem}{Theorem}[section]

    \let\Re\undefined
    \let\Im\undefined
    \DeclareMathOperator{\Res}{Res}
    \DeclareMathOperator{\Re}{Re}
    \DeclareMathOperator{\Im}{Im}
    \DeclareMathOperator{\Log}{Log}
    \DeclareMathOperator{\Arg}{Arg}
    \DeclareMathOperator{\Tr}{Tr}
    \DeclareMathOperator{\E}{E}
    \DeclareMathOperator{\Var}{Var}
    \DeclareMathOperator*{\argmin}{argmin}
    \DeclareMathOperator*{\argmax}{argmax}
    \DeclareMathOperator{\sgn}{sgn}
    \DeclareMathOperator{\diag}{diag\;}

    \colorlet{Corr}{red}

    % \everymath{\displaystyle} % biggify limits of inline sums and integrals
    \tikzstyle{circ} % usage: \node[circ, placement] (label) {text};
        = [draw, circle, fill=white, node distance=3cm, minimum height=2em]
    \definecolor{commentgreen}{rgb}{0,0.6,0}
    \lstset{
        basicstyle=\ttfamily\footnotesize,
        frame=single,
        numbers=left,
        showstringspaces=false,
        keywordstyle=\color{blue},
        stringstyle=\color{purple},
        commentstyle=\color{commentgreen},
        morecomment=[l][\color{magenta}]{\#}
    }

\begin{document}

\section{Executive Summary}

\begin{itemize}
    \item In the regime where $e_{\lim}$ is sufficient to induce one-shot
        mergers, the inclinations $I_0$ where $e_{\lim}$ is attainable
        correspond very well with the regions where the octupole-induced merger
        probability is nonzero.

    \item Furthermore, there are regions where ``octupole-enhanced'' mergers
        occur. This is because octupole effects excite the suitably-averaged
        eccentricity enough that mergers occur within a Hubble time. This effect
        is much more substantial for smaller $a$, where one-shot mergers are
        difficult.

    \item For $I_0$ sufficiently close to $90^\circ$, finite $\eta$ suppresses
        octupole-LK excitation to $e_{\lim}$.
\end{itemize}

\section{Simulations}

\begin{figure}[h]
    \centering
    \includegraphics[width=0.6\columnwidth]{../scripts/octlk/1sweepbin/total_merger_fracs.png}
    \caption{Total merger fractions as a function of $q$ (left) and
    $\epsilon_{\rm oct}$ (right). Different symbols denote different $e_{\rm
    out}$ (legend), and different colors denote different $q \in [0.2, 0.3, 0.4,
    0.5, 0.7, 1.0]$ (darker colors correspond to smaller $q$). Only $q = 0.4,
    e_{\rm out} = 0.9$ is missing.}\label{fig:merger_fracs}
\end{figure}

$10\;\mathrm{AU}$ is hard to run, RAM intensive and takes many cycles to merger.
I tried to run too dense of a grid of simulations and also had some RAM issues
on accident, didn't finish. But most of the behavior can be seen in the
$100\;\mathrm{AU}$ simulations that I ran, which are much faster ($t_{\rm LK}
\propto a^{3/2}$).

Tried to run wide range of simulations for $\z{q \in [0.2, 0.3, 0.4, 0.5, 0.7,
1.0]} \otimes \z{e_{\rm out} \in [0.6, 0.8, 0.9]}$ while holding $a_{\rm out,
eff} = 3600\;\mathrm{AU}$ ($e_{\rm out} = 0.6$, $a_{\rm out} =
4500\;\mathrm{AU}$) constant. The other parameters are, as before:
\begin{align*}
    m_{12} &= 50M_{\odot}, &
    m_3 &= 30 M_{\odot}, &
    a_{1,0} &= 100\;\mathrm{AU}&
    e_{1, 0} &= 10^{-3}.
\end{align*}
Of the $18$ parameter regimes targeted, $17$ have complete data. The current
merger fraction plot is in Fig.~\ref{fig:merger_fracs}.

However, I do reproduce the ``gap'' that Bin saw. A characteristic plots are
shown in Figs.~\ref{fig:composite1}--\ref{fig:composite4}.
\begin{figure}
    \centering
    \includegraphics[width=0.7\columnwidth]{../scripts/octlk/1sweepbin/composite_1p5dist.png}
    \caption{Example 1. Top panel is merger probability, middle panel is merger
    times (green dots are mergers, blue triangles are non-merging systems), and
    the bottom panel is the plot of $1 - e$ without GW run over $500t_{\rm LK}$.
    On the bottom panel, the black dashed curve represents the quadrupole
    $e_{\max}$ values [analytic, LL18 Eq.~(42)], the blue dots represent the
    $e_{\max}$ over the GW-less run, and the green dots represent the average
    $e_{\rm eff}$ [Eq.~\eqref{eq:e_eff}] over the same interval; the horizontal
    lines correspond to $e_{\lim}$ (black), the one-shot merger eccentricity
    $e_{\rm os}$ (blue), the necessary effective eccentricity to merge $e_{\rm
    eff, c}$ (green); and the pink vertical lines represent $I_{\lim}$ given by
    MLL16 for reference. In broad summary, systems with $e_{\max}$ below the
    blue line are expected to merge, as are systems with $\ev{e_{\rm eff}}$
    below the green line.}\label{fig:composite1}
\end{figure}
\begin{figure}
    \centering
    \includegraphics[width=0.7\columnwidth]{../scripts/octlk/1sweepbin/composite_1p2dist.png}
    \caption{Example 2}\label{fig:composite2}
\end{figure}
\begin{figure}
    \centering
    \includegraphics[width=0.7\columnwidth]{../scripts/octlk/1sweepbin/composite_e81p5dist.png}
    \caption{Example 3}\label{fig:composite3}
\end{figure}
\begin{figure}
    \centering
    \includegraphics[width=0.7\columnwidth]{../scripts/octlk/1sweepbin/composite_e81p2dist.png}
    \caption{Example 4}\label{fig:composite4}
\end{figure}

\section{Theory}

\subsection{One-Shot Merger Eccentricity}

We seek a critical $e_{\rm os}$ for which the system can merge in ``one-shot'',
i.e.\ in one LK cycle. This can be computed:
\begin{align}
    \rd{\ln a}{t} &= -\frac{64}{5j^7(e)}\p{1 + \frac{73}{24}e^2 +
        \frac{37}{96}e^4}\frac{G^3 \mu m_{12}^2}{c^5a^4},\\
    \ev{\rd{\ln a}{t}}_{\rm LK} &\sim -j\p{e_{\max}}\frac{64}{5j^7(e_{\max})}
            (4) \frac{G^3\mu m_{12}^2}{c^5a^4}
            = \frac{1}{t_{\rm GW, 0}j^6(e_{\max})},\\
    j^6\p{e_{\rm os}} \equiv j_{\rm os}
        &= \frac{ t_{\rm LK}}{t_{\rm GW, 0}},\\
        &= \frac{256}{5}\frac{G^3 \mu m_{12}^3}{m_3c^5a^4n}
            \p{\frac{a_{\rm out, eff}}{a}}^3,
\end{align}
where $t_{\rm GW, 0}$ denotes the $e = 0$ evaluation of the GW inspiral
timescale, and angle brackets denote averaging over a \emph{single} LK cycle.
Thus, when systems satisfy $e_{\lim} \gtrsim e_{\rm os}$, all orbit flips will
immediately lead to mergers. We can estimate
\begin{align}
    j_{\lim} &\approx \frac{8\epsilon_{\rm GR}}{9 + 3\eta^2/4},\\
    \p{\frac{a}{a_{\rm out, eff}}} &\gtrsim
        0.0118
        \p{\frac{a_{\rm out, eff}}{3600\;\mathrm{AU}}}^{-7/37}
        \p{\frac{m_{12}}{50M_{\odot}}}^{17/37}
        \p{\frac{30M_{\odot}}{m_3}}^{10/37}
        \p{\frac{q / (1 + q)^2}{1/4}}^{-2/37}.
\end{align}
This is the regime in which ELK-induced mergers is easiest to
understand.

\subsection{Effective Merging Eccentricity}

Some systems can merge even without undergoing orbit flips. These correspond to
the probabilistic regions. We define an ``effective'' eccentricity, over the
GW-less simulations, such that the total GW emission is comparable:
\begin{align}
    \ev{\rd{\ln a}{t}}_{\rm LK} &= -\ev{\frac{a}{t_{\rm GW}}}_{\rm LK},\\
        &\approx -\frac{a}{t_{\rm GW, 0}}
            \ev{\frac{1 + 73e_{\max}^2/24 + 37e_{\max}^4/96}
                {j^6(e_{\max})}},\\
    \ev{\rd{\ln a}{t}}_{\text{Many LK Cycles}} &\equiv -\frac{a}{t_{\rm GW,0}}
            \underbrace{\p{\frac{1 + 73e_{\rm eff}^2/24 + 37e_{\rm
            eff}^4/96}{j^6(e_{\rm eff})}}}_{f\p{e_{\rm eff}}}
            \approx \frac{-4a}{t_{\rm GW, 0}j^6(e_{\rm eff})}\label{eq:e_eff}
\end{align}
where the second average is taken over many LK cycles, as denoted.

\textbf{Comment after meeting:} Note that formulating $e_{\rm eff}$ by averaging
over just eccentricity \emph{maxima}, rather than the full $e(t)$, may have
advantages if we are able to estimate the octupole-induced oscillation amplitude
of $K$ [Eq.~\eqref{eq:K}, see discussion at the end of Section~\ref{ss:gap}].

We can then ask what level of $e_{\rm eff}$ is required to induce
merger within a Hubble time $t_{\rm Hubb}$. This can also be estimated
\begin{align}
    \frac{t_{\rm GW, 0}}{t_{\rm Hubb}}\frac{1}{f\p{e_{\rm eff}}}
        &\lesssim 1,\\
    \p{\frac{4t_{\rm Hubb}}{t_{\rm GW, 0}}}^{1/6}
        &\gtrsim j\p{e_{\rm eff}},\\
    0.01461\p{\frac{100\;\mathrm{AU}}{a}}^{2/3}
        \p{\frac{1/4}{q(1 + q)^2}}^{1/6}
        &\gtrsim j\p{e_{\rm eff}}.
\end{align}

We can see that the probabilistic regime in the earlier figures is where $e_{\rm
eff}$ spans a few orders of magnitude. This suggests that $e_{\rm eff}$
stochastically attains these large values over timescales $\gg 500 t_{\rm LK}$
(i.e.\ over time, some fraction of systems will enter a very high
maximum-eccentricity state).

\subsection{Origin of the Gap}\label{ss:gap}

We have already reproduced the gap in our simulations. For reference, we can
also look at $e_{\max}$ distribution for Bin's case from last week $a_{\rm out}
= 700$, $e_{\rm out} = 0.9$, $a_{\rm in} = 10\;\mathrm{AU}$, and $q = 0.4$ in
Fig.~\ref{fig:bindist_emax} (NB\@: I messed up and used $m_{12} = 50M_{\odot}$
while Bin used $m_{12} = 60M_{\odot}$).
\begin{figure}
    \centering
    \includegraphics[width=0.7\columnwidth]{../scripts/octlk/1sweepbin_emax/bindist.png}
    \caption{$e_{\max}$ distribution for Bin's gapped case last week. Note that
    $e_{\lim} \approx e_{\rm os}$ for this parameter regime.}\label{fig:bindist_emax}
\end{figure}

The origin of the gap is because $e_{\max}$ oscillations are suppressed near $I
= 90^\circ$. I think this happens because Katz et.\ al.\ 2011 show that ELK
oscillations happen due to a feedback between $j_z = j\cos I$ (conserved to
quadrupole order in the test mass limit) and $\Omega_e$ the co-longitude
(azimuthal angle relative to $\uv{z}$) of the inner eccentricity vector.
However, when $\eta > 0$, we know that the conserved quantity is (LL18)
\begin{equation}
    K \equiv j\cos I - \eta \frac{e^2}{2}.\label{eq:K}
\end{equation}
This suggests that when $\abs{j \cos I} \lesssim \eta / 2$, that $\eta$
suppresses oscillations in $j \cos I$ and the oscillations in the eccentricity
maxima. I think this is the right mechanism, but I don't have the analytical
solution.

I add two plots I showed last week, to illustrate the oscillations in $K$ and
their effect on the behavior of $e_{\max}$, shown in Fig.~\ref{fig:1nogw}. We
see that $K$ oscillates (albeit somewhat irregularly) over timescales much
longer than a LK cycle. This may allow us to estimate the $e_{\rm eff}$
enhancement expected due to octupole effects when orbital flips are not
observed.
\begin{figure}[h]
    \centering
    \includegraphics[width=0.7\columnwidth]{../scripts/octlk/1nogw_vec.png}
    \includegraphics[width=0.7\columnwidth]{../scripts/octlk/1nogw_vec95.png}
    \caption{Two fiducial simulations showing the behavior of $K$ in the absence
    of GW radiation. Here, $q = 2/3$. In the fourth panels of both simulations,
    the horizontal black dotted line gives $K = -\eta_0/2$, where orbital flips
    are expected.}\label{fig:1nogw}
\end{figure}

\end{document}

