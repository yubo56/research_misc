    \documentclass[11pt,
        usenames, % allows access to some tikz colors
        dvipsnames % more colors: https://en.wikibooks.org/wiki/LaTeX/Colors
    ]{article}
    \usepackage{
        amsmath,
        amssymb,
        fouriernc, % fourier font w/ new century book
        fancyhdr, % page styling
        lastpage, % footer fanciness
        hyperref, % various links
        setspace, % line spacing
        amsthm, % newtheorem and proof environment
        mathtools, % \Aboxed for boxing inside aligns, among others
        float, % Allow [H] figure env alignment
        enumerate, % Allow custom enumerate numbering
        graphicx, % allow includegraphics with more filetypes
        wasysym, % \smiley!
        upgreek, % \upmu for \mum macro
        listings, % writing TrueType fonts and including code prettily
        tikz, % drawing things
        booktabs, % \bottomrule instead of hline apparently
        xcolor, % colored text
        cancel % can cancel things out!
    }
    \usepackage[margin=1in]{geometry} % page geometry
    \usepackage[
        labelfont=bf, % caption names are labeled in bold
        font=scriptsize % smaller font for captions
    ]{caption}
    \usepackage[font=scriptsize]{subcaption} % subfigures

    \newcommand*{\scinot}[2]{#1\times10^{#2}}
    \newcommand*{\dotp}[2]{\left<#1\,\middle|\,#2\right>}
    \newcommand*{\rd}[2]{\frac{\mathrm{d}#1}{\mathrm{d}#2}}
    \newcommand*{\pd}[2]{\frac{\partial#1}{\partial#2}}
    \newcommand*{\rdil}[2]{\mathrm{d}#1 / \mathrm{d}#2}
    \newcommand*{\pdil}[2]{\partial#1 / \partial#2}
    \newcommand*{\rtd}[2]{\frac{\mathrm{d}^2#1}{\mathrm{d}#2^2}}
    \newcommand*{\ptd}[2]{\frac{\partial^2 #1}{\partial#2^2}}
    \newcommand*{\md}[2]{\frac{\mathrm{D}#1}{\mathrm{D}#2}}
    \newcommand*{\pvec}[1]{\vec{#1}^{\,\prime}}
    \newcommand*{\svec}[1]{\vec{#1}\;\!}
    \newcommand*{\bm}[1]{\boldsymbol{\mathbf{#1}}}
    \newcommand*{\uv}[1]{\hat{\bm{#1}}}
    \newcommand*{\ang}[0]{\;\text{\AA}}
    \newcommand*{\mum}[0]{\;\upmu \mathrm{m}}
    \newcommand*{\at}[1]{\left.#1\right|}
    \newcommand*{\bra}[1]{\left<#1\right|}
    \newcommand*{\ket}[1]{\left|#1\right>}
    \newcommand*{\abs}[1]{\left|#1\right|}
    \newcommand*{\ev}[1]{\left\langle#1\right\rangle}
    \newcommand*{\p}[1]{\left(#1\right)}
    \newcommand*{\s}[1]{\left[#1\right]}
    \newcommand*{\z}[1]{\left\{#1\right\}}

    \newtheorem{theorem}{Theorem}[section]

    \let\Re\undefined
    \let\Im\undefined
    \DeclareMathOperator{\Res}{Res}
    \DeclareMathOperator{\Re}{Re}
    \DeclareMathOperator{\Im}{Im}
    \DeclareMathOperator{\Log}{Log}
    \DeclareMathOperator{\Arg}{Arg}
    \DeclareMathOperator{\Tr}{Tr}
    \DeclareMathOperator{\E}{E}
    \DeclareMathOperator{\Var}{Var}
    \DeclareMathOperator*{\argmin}{argmin}
    \DeclareMathOperator*{\argmax}{argmax}
    \DeclareMathOperator{\sgn}{sgn}
    \DeclareMathOperator{\diag}{diag\;}

    \colorlet{Corr}{red}

    % \everymath{\displaystyle} % biggify limits of inline sums and integrals
    \tikzstyle{circ} % usage: \node[circ, placement] (label) {text};
        = [draw, circle, fill=white, node distance=3cm, minimum height=2em]
    \definecolor{commentgreen}{rgb}{0,0.6,0}
    \lstset{
        basicstyle=\ttfamily\footnotesize,
        frame=single,
        numbers=left,
        showstringspaces=false,
        keywordstyle=\color{blue},
        stringstyle=\color{purple},
        commentstyle=\color{commentgreen},
        morecomment=[l][\color{magenta}]{\#}
    }

\begin{document}

\onehalfspacing

\pagestyle{fancy}
\rfoot{Yubo Su}
\rhead{}
\cfoot{\thepage/\pageref{LastPage}}

Call $P(q; a_{\rm out, eff})$ the merger probability as a function of $q$.
Questions to consider at this point:
\begin{enumerate}[(1)]
    \item For what $q$ is the $P(q)$ maximized? ($P(0) = 0$)

    \item What does $P(q)$ look like with fewer bins (less noise)?

    \item What does $P(q)$ look like if $e_{\rm out}$ is thermal?

    \item Does $P(a_{\rm out, eff})$ vanish at the same place as LL18's
        analytic expression (in the quadrupole limit)?

    \item LIGO/LISA band eccentricities?

    \item What is the primordial $q$ distribution in high-mass binaries?
\end{enumerate}

These are all relatively short, I address them in kind:

\section{Maximum $P(q)$}

This happens at very small $q$ because $e_{\rm os}$ is a very insensitive
function of $q$, (i.e..\ $j\p{e_{\rm os}}^6 \sim t_{\rm LK} / t_{\rm GW, 0}
\propto 1/\mu$). I derived before (probably with some mistakes)
\begin{align}
    j^6\p{e_{\rm os}} \equiv j_{\rm os}
        &= \frac{256}{5}\frac{G^3 \mu m_{12}^2}{c^5a^4}
            \frac{1}{n}\frac{m_{12}}{m_3}\p{\frac{a_{\rm out,
            eff}}{a}}^3,\\
        &= \frac{256}{5}\frac{G^3 \mu m_{12}^3}{m_3c^5a^4n}
            \p{\frac{a_{\rm out, eff}}{a}}^3.
\end{align}
Approximating $j_{\lim} \approx \frac{8\epsilon_{\rm GR}}{9 + 3\eta^2/4}$, we
obtain the $a$ criterion for one-shot mergers
\begin{align}
    256 \frac{G^{5/2}a_{\rm out, eff}^3 m_{12}^{5/2}\mu}{
        5a^{11/2}c^5 m_3}
        &\gtrsim \p{\frac{8}{9 + 3\eta^2/4}
            \frac{3Gm_{12}^2a_{\rm out, eff}^3}{c^2a^4m_3}}^6,\\
    a^{37/2} &\gtrsim 5 \cdot 1024 \frac{G^{7/2}a_{\rm out, eff}^{15}
        m_{12}^{19/2}}{c^7 m_3^5 \mu \p{3 + \eta^2/4}^6},\\
    \p{\frac{a}{a_{\rm out, eff}}} &\gtrsim
        0.0118
        \p{\frac{a_{\rm out, eff}}{3600\;\mathrm{AU}}}^{-7/37}
        \p{\frac{m_{12}}{50M_{\odot}}}^{17/37}
        \p{\frac{30M_{\odot}}{m_3}}^{10/37}
        \p{\frac{q / (1 + q)^2}{1/4}}^{-2/37}.
\end{align}
Note that $\mu / m_{12} = m_1m_2 / m_{12}^2 = q / (1 + q)^2$. Thus, if $a$ is
fixed, $q$ must decrease dramatically to cause the system to transition from
$e_{\lim} \gtrsim e_{\rm os}$ to $e_{\lim} \lesssim e_{\rm os}$.

% TODO add plot?

\section{$P(q)$ with fewer bins \& Thermal Distribution}

There was some computer downtime, but I was able to get one $P(q)$ plot with
fewer bins, see Fig.~\ref{fig:popsynths}. I also include the distributions if
the distribution of $e_2$ is thermal in red dots. I also tried using the GW-only
simulations to uniformly scan a grid of $(I_0, e_{\rm out})$ for each $q$, using
a much denser grid of $q$, shown in the blue line.
\begin{figure}
    \centering
    \includegraphics[width=0.45\columnwidth]{../scripts/octlk/1popsynth/a2eff3600.png}
    \includegraphics[width=0.45\columnwidth]{../scripts/octlk/1popsynth/a2eff2800.png}
    \caption{$P(q; a_{\rm out} = 3600\;\mathrm{AU})$ with $19$ bins, and $P\p{q;
    a_{\rm out}} = 2800\;\mathrm{AU}$ with $7$ bins.}\label{fig:popsynths}
\end{figure}

\section{Comparison of Merger Fraction to LL18}

Last meeting, we mentioned that a figure like Fig.~8 of LL18 is not useful
because it requires assuming distributions on $q$ and $e_{\rm out}$. However, if
we compare the LL18 criterion to our ``effective eccentricity'' GW-less merger
criterion, we notice that their criterion, given by
\begin{equation}
    T_{\rm m, 0}j^6\p{e_{\rm m}} = T_{\rm crit},
\end{equation}
is equivalent to requiring $j\p{e_{\rm eff}} \lesssim j_{\rm eff, crit}$, where
$j_{\rm eff, crit}$ is the effective maximum eccentricity of the LK cycle (equal
to the quadrupole $e_{\max}$ when octupole effects are small). I haven't checked
this against data.

\section{LISA/LIGO Band Eccentricities}

I will generate these plots in time for the meeting. %TODO

\section{Primordial $q$ distribution in HM binaries}


See \url{https://arxiv.org/pdf/1606.05347.pdf}, in particular Figs.~2,~8, and~11
(NB\@: Salpeter has $gamma = -2.35$). Notably,
- Abstract
- The intrinsic population of binaries with longer orbital periods is even
  further skewed toward smaller mass ratios. Our result is consistent with the
  conclusions of Abt et al. (1990), who also find that early B spectroscopic
  binaries become weighted toward smaller mass ratios with increasing separation
\end{document}

