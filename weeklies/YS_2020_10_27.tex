    \documentclass[11pt,
        usenames, % allows access to some tikz colors
        dvipsnames % more colors: https://en.wikibooks.org/wiki/LaTeX/Colors
    ]{article}
    \usepackage{
        amsmath,
        amssymb,
        fouriernc, % fourier font w/ new century book
        fancyhdr, % page styling
        lastpage, % footer fanciness
        hyperref, % various links
        setspace, % line spacing
        amsthm, % newtheorem and proof environment
        mathtools, % \Aboxed for boxing inside aligns, among others
        float, % Allow [H] figure env alignment
        enumerate, % Allow custom enumerate numbering
        graphicx, % allow includegraphics with more filetypes
        wasysym, % \smiley!
        upgreek, % \upmu for \mum macro
        listings, % writing TrueType fonts and including code prettily
        tikz, % drawing things
        booktabs, % \bottomrule instead of hline apparently
        xcolor, % colored text
        cancel % can cancel things out!
    }
    \usepackage[margin=1in]{geometry} % page geometry
    \usepackage[
        labelfont=bf, % caption names are labeled in bold
        font=scriptsize % smaller font for captions
    ]{caption}
    \usepackage[font=scriptsize]{subcaption} % subfigures

    \newcommand*{\scinot}[2]{#1\times10^{#2}}
    \newcommand*{\dotp}[2]{\left<#1\,\middle|\,#2\right>}
    \newcommand*{\rd}[2]{\frac{\mathrm{d}#1}{\mathrm{d}#2}}
    \newcommand*{\pd}[2]{\frac{\partial#1}{\partial#2}}
    \newcommand*{\rdil}[2]{\mathrm{d}#1 / \mathrm{d}#2}
    \newcommand*{\pdil}[2]{\partial#1 / \partial#2}
    \newcommand*{\rtd}[2]{\frac{\mathrm{d}^2#1}{\mathrm{d}#2^2}}
    \newcommand*{\ptd}[2]{\frac{\partial^2 #1}{\partial#2^2}}
    \newcommand*{\md}[2]{\frac{\mathrm{D}#1}{\mathrm{D}#2}}
    \newcommand*{\pvec}[1]{\vec{#1}^{\,\prime}}
    \newcommand*{\svec}[1]{\vec{#1}\;\!}
    \newcommand*{\bm}[1]{\boldsymbol{\mathbf{#1}}}
    \newcommand*{\uv}[1]{\hat{\bm{#1}}}
    \newcommand*{\ang}[0]{\;\text{\AA}}
    \newcommand*{\mum}[0]{\;\upmu \mathrm{m}}
    \newcommand*{\at}[1]{\left.#1\right|}
    \newcommand*{\bra}[1]{\left<#1\right|}
    \newcommand*{\ket}[1]{\left|#1\right>}
    \newcommand*{\abs}[1]{\left|#1\right|}
    \newcommand*{\ev}[1]{\langle#1\rangle}
    \newcommand*{\p}[1]{\left(#1\right)}
    \newcommand*{\s}[1]{\left[#1\right]}
    \newcommand*{\z}[1]{\left\{#1\right\}}

    \newtheorem{theorem}{Theorem}[section]

    \let\Re\undefined
    \let\Im\undefined
    \DeclareMathOperator{\Res}{Res}
    \DeclareMathOperator{\Re}{Re}
    \DeclareMathOperator{\Im}{Im}
    \DeclareMathOperator{\Log}{Log}
    \DeclareMathOperator{\Arg}{Arg}
    \DeclareMathOperator{\Tr}{Tr}
    \DeclareMathOperator{\E}{E}
    \DeclareMathOperator{\Var}{Var}
    \DeclareMathOperator*{\argmin}{argmin}
    \DeclareMathOperator*{\argmax}{argmax}
    \DeclareMathOperator{\sgn}{sgn}
    \DeclareMathOperator{\diag}{diag\;}

    \colorlet{Corr}{red}

    % \everymath{\displaystyle} % biggify limits of inline sums and integrals
    \tikzstyle{circ} % usage: \node[circ, placement] (label) {text};
        = [draw, circle, fill=white, node distance=3cm, minimum height=2em]
    \definecolor{commentgreen}{rgb}{0,0.6,0}
    \lstset{
        basicstyle=\ttfamily\footnotesize,
        frame=single,
        numbers=left,
        showstringspaces=false,
        keywordstyle=\color{blue},
        stringstyle=\color{purple},
        commentstyle=\color{commentgreen},
        morecomment=[l][\color{magenta}]{\#}
    }

\begin{document}

\onehalfspacing

\title{Octupole Merger Window Update}
\author{Yubo Su}

\maketitle

\section{Simulation}

I implemented the equations from Appendix~A of Liu et.\ al.\ 2015, LK
oscillations in orbital elements, and added GW dissipation to $a_1$, $e_1$, and
apsidal precession in $\omega_1$. The fiducial parameters are:
\begin{align*}
    m_{12} &= 50M_{\odot}, &
    m_3 &= 30 M_{\odot}, &
    a_{1,0} &= 100\;\mathrm{AU}, \\
    a_2 &= 6000\;\mathrm{AU},&
    e_{1, 0} &= 10^{-3}, &
    e_{2, 0} &= 0.6.
\end{align*}
I ran an example simulation using $I_{\rm tot, 0} = I_{1, 0} + I_{2, 0} =
93.5^\circ$, initial angles $\Omega_{1, 0} = \Omega_{2, 0} + \pi = \omega_{1, 0}
= 0$, $\omega_{2, 0} = 0.7\;\mathrm{rad}$, and masses $m_1 = 30M_{\odot}$, $m_2
= 20M_{\odot}$, so resembling Fig.~10 of LL18. The resulting evolution of the
orbit is shown in Fig.~\ref{fig:fiducial}.
\begin{figure}
    \centering
    \includegraphics[width=0.5\columnwidth]{../scripts/octlk/1fiducial.png}
    \caption{Fiducial simulation using same params \& ICs as Fig.~10 of LL18,
    but with completely different results (failing to merge in
    $10^{10}\;\mathrm{yr}$). However, it bears noting that for $q = 0.7, I_{\rm
    tot, 0} = 93.5^\circ$ (this simulation has $q = 2/3$), I got many
    simulations merging in a few $10^8\;\mathrm{yr}$, see Fig.~\ref{fig:sweep},
    so it is possible this is just an abnormally long lived
    IC.}\label{fig:fiducial}
\end{figure}

\section{Population}

I then swept over $I_{\rm tot, 0} \in [91^\circ, 95^\circ]$ for mass ratios $q =
1.0, 0.7, 0.5, 0.4, 0.3, 0.2$. I used 60 different initial inclinations, and for
each initial inclination, I randomly chose $\Omega_{1, 0}$, $\omega_{1, 0}$, and
$\omega_{2, 0}$ five times. The resulting merger times are shown below in
Fig.~\ref{fig:sweep}.
\begin{figure*}
    \centering
    \includegraphics[width=0.45\columnwidth]{../scripts/octlk/1sweep/1equaldist.png}
    \includegraphics[width=0.45\columnwidth]{../scripts/octlk/1sweep/1p7dist.png}

    \includegraphics[width=0.45\columnwidth]{../scripts/octlk/1sweep/1p5dist.png}
    \includegraphics[width=0.45\columnwidth]{../scripts/octlk/1sweep/1p4dist.png}

    \includegraphics[width=0.45\columnwidth]{../scripts/octlk/1sweep/1p3dist.png}
    \includegraphics[width=0.45\columnwidth]{../scripts/octlk/1sweep/1p2dist.png}
    \caption{Merger times with varying $q$ using the fidicual parameters, where
    every initial mutual inclination is retried $5$ times with different
    $\Omega, \omega$. In order: $q = 1.0, 0.7, 0.5, 0.4, 0.3, 0.2$. Blue points
    denote systems that do not merger within a Hubble time $10\;\mathrm{Gyr}$,
    while green points denote systems that do. The qualitative trend seems to
    agree with Fig.~9 of LL18, where as $\epsilon_{\rm oct}$ is increased, the
    merger window grows towards larger inclinations first.}\label{fig:sweep}
\end{figure*}

\end{document}

