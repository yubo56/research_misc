    \documentclass[11pt,
        usenames, % allows access to some tikz colors
        dvipsnames % more colors: https://en.wikibooks.org/wiki/LaTeX/Colors
    ]{article}
    \usepackage{
        amsmath,
        amssymb,
        fouriernc, % fourier font w/ new century book
        fancyhdr, % page styling
        lastpage, % footer fanciness
        hyperref, % various links
        setspace, % line spacing
        amsthm, % newtheorem and proof environment
        mathtools, % \Aboxed for boxing inside aligns, among others
        float, % Allow [H] figure env alignment
        enumerate, % Allow custom enumerate numbering
        graphicx, % allow includegraphics with more filetypes
        wasysym, % \smiley!
        upgreek, % \upmu for \mum macro
        listings, % writing TrueType fonts and including code prettily
        tikz, % drawing things
        booktabs, % \bottomrule instead of hline apparently
        xcolor, % colored text
        cancel % can cancel things out!
    }
    \usepackage[margin=1in]{geometry} % page geometry
    \usepackage[
        labelfont=bf, % caption names are labeled in bold
        font=scriptsize % smaller font for captions
    ]{caption}
    \usepackage[font=scriptsize]{subcaption} % subfigures

    \newcommand*{\scinot}[2]{#1\times10^{#2}}
    \newcommand*{\dotp}[2]{\left<#1\,\middle|\,#2\right>}
    \newcommand*{\rd}[2]{\frac{\mathrm{d}#1}{\mathrm{d}#2}}
    \newcommand*{\pd}[2]{\frac{\partial#1}{\partial#2}}
    \newcommand*{\rdil}[2]{\mathrm{d}#1 / \mathrm{d}#2}
    \newcommand*{\pdil}[2]{\partial#1 / \partial#2}
    \newcommand*{\rtd}[2]{\frac{\mathrm{d}^2#1}{\mathrm{d}#2^2}}
    \newcommand*{\ptd}[2]{\frac{\partial^2 #1}{\partial#2^2}}
    \newcommand*{\md}[2]{\frac{\mathrm{D}#1}{\mathrm{D}#2}}
    \newcommand*{\pvec}[1]{\vec{#1}^{\,\prime}}
    \newcommand*{\svec}[1]{\vec{#1}\;\!}
    \newcommand*{\bm}[1]{\boldsymbol{\mathbf{#1}}}
    \newcommand*{\uv}[1]{\hat{\bm{#1}}}
    \newcommand*{\ang}[0]{\;\text{\AA}}
    \newcommand*{\mum}[0]{\;\upmu \mathrm{m}}
    \newcommand*{\at}[1]{\left.#1\right|}
    \newcommand*{\bra}[1]{\left<#1\right|}
    \newcommand*{\ket}[1]{\left|#1\right>}
    \newcommand*{\abs}[1]{\left|#1\right|}
    \newcommand*{\ev}[1]{\left\langle#1\right\rangle}
    \newcommand*{\p}[1]{\left(#1\right)}
    \newcommand*{\s}[1]{\left[#1\right]}
    \newcommand*{\z}[1]{\left\{#1\right\}}

    \newtheorem{theorem}{Theorem}[section]

    \let\Re\undefined
    \let\Im\undefined
    \DeclareMathOperator{\Res}{Res}
    \DeclareMathOperator{\Re}{Re}
    \DeclareMathOperator{\Im}{Im}
    \DeclareMathOperator{\Log}{Log}
    \DeclareMathOperator{\Arg}{Arg}
    \DeclareMathOperator{\Tr}{Tr}
    \DeclareMathOperator{\E}{E}
    \DeclareMathOperator{\Var}{Var}
    \DeclareMathOperator*{\argmin}{argmin}
    \DeclareMathOperator*{\argmax}{argmax}
    \DeclareMathOperator{\sgn}{sgn}
    \DeclareMathOperator{\diag}{diag\;}

    \colorlet{Corr}{red}

    % \everymath{\displaystyle} % biggify limits of inline sums and integrals
    \tikzstyle{circ} % usage: \node[circ, placement] (label) {text};
        = [draw, circle, fill=white, node distance=3cm, minimum height=2em]
    \definecolor{commentgreen}{rgb}{0,0.6,0}
    \lstset{
        basicstyle=\ttfamily\footnotesize,
        frame=single,
        numbers=left,
        showstringspaces=false,
        keywordstyle=\color{blue},
        stringstyle=\color{purple},
        commentstyle=\color{commentgreen},
        morecomment=[l][\color{magenta}]{\#}
    }

\begin{document}

\section{Evection Resonance Maximum Growth}

If we require the evection resonance condition $\dot{\varpi} \sim \dot{f}_{\rm
out}$, or
\begin{equation}
    \frac{3Gm_{12}}{c^2a}\sqrt{\frac{m_{12}a_{\rm out}^3}{m_{123}a^3}} \sim 1,
\end{equation}
then this can be rewritten as
\begin{equation}
    \frac{a^5}{a_{\rm out}^3} \sim \frac{9G^2m_{12}^3}{c^2m_{123}}.
\end{equation}
The $\epsilon$ associated with the system can then be rewritten as:
\begin{align}
    \epsilon &= \frac{m_3 a^4c^2}{3Gm_{12}^2a_{\rm out}^3},\\
        &= \frac{3m_3}{m_{123}}\p{\frac{v}{c}}^2.
\end{align}
Here, $v = \sqrt{Gm_{12}/a}$ is the orbital velocity of the inner binary. Using
the above scalings, we find that $v \propto a^{-1/2} \propto a_{\rm
out}^{-3/10}$, and thus $\Delta e \propto a_{\rm out}^{-3/10}$. Is this
observed? Well, my $a_{\rm out} = 2.38\;\mathrm{AU}$ has $\Delta e = 0.006$ and
my $a_{\rm out} = 238\;\mathrm{AU}$ has $\Delta e \approx 0.0015$, which is in
rough agreement.

% simtest.png but no
% The result might be because the resonance has no time to grow? It grows on
% timescales $\sim t_{\rm ZLK} \propto a^{3/2} / a_{\rm out}^3 \propto a_{\rm
% out}^{-21/10}$, which is way steeper than observed, though some points follow
% along this slope.

\subsection{With Eccentricity}

We showed in our notes that the evection Hamiltonian looks something like
\begin{align}
    H\p{\Gamma, \phi} &= P\Gamma - 4\Gamma^2 + R\cos \phi,\nonumber\\
    P &= 2\s{1 - \Omega_{\rm out} / \Omega_{\rm GR, 0} + 3\epsilon/4},\nonumber\\
    R &= \frac{15\epsilon}{2}\p{1 + F_{N2}}.
\end{align}
where $\Gamma \approx -e^2/4$, and $F_{N2}$ is the Hansen coefficient. The
equilibrium of the Hamiltonian, when it exists is located at
\begin{equation}
    \Gamma_{\rm eq} = \frac{P - R}{8} \sim \mathcal{O}(\epsilon).
\end{equation}
Note that even if $e_{\rm out} = 0.9$, $F_{N2}$ only maximizes at $\sim 20$, so
the evection eccentricity cannot be enhanced by more than a factor of $4$--$5$
realistically except for extremely strong eccentricities: to leading order,
$F_{N2}$ is maximized at $N \simeq \p{1 - e_{\rm out}^2}^{-3/2}$ at a value of
$(1 - e_{\rm out}^2)^{-3/2}$, so we expect an enhancement of the evection
resonance eccentricity excitation by $\sim \p{1 - e_{\rm out}^2}^{-3/4}$, not a
lot.

So, basically, without some sort of exotic $2+1+1$ system, we're probably out of
luck.

The problem here is that $\epsilon \sim \Phi_{\rm ZLK} / \Phi_{\rm GR}$ is too
small when $\Omega_{\rm out} \sim \dot{\varpi}_{\rm GR}$, i.e.\ the hierarchy of
scales between the quadrupole ZLK coupling and the simple Keplerian coupling is
too large.

\end{document}

