    \documentclass[11pt,
        usenames, % allows access to some tikz colors
        dvipsnames % more colors: https://en.wikibooks.org/wiki/LaTeX/Colors
    ]{article}
    \usepackage{
        amsmath,
        amssymb,
        fouriernc, % fourier font w/ new century book
        fancyhdr, % page styling
        lastpage, % footer fanciness
        hyperref, % various links
        setspace, % line spacing
        amsthm, % newtheorem and proof environment
        mathtools, % \Aboxed for boxing inside aligns, among others
        float, % Allow [H] figure env alignment
        enumerate, % Allow custom enumerate numbering
        graphicx, % allow includegraphics with more filetypes
        wasysym, % \smiley!
        upgreek, % \upmu for \mum macro
        listings, % writing TrueType fonts and including code prettily
        tikz, % drawing things
        booktabs, % \bottomrule instead of hline apparently
        xcolor, % colored text
        cancel % can cancel things out!
    }
    \usepackage[margin=1in]{geometry} % page geometry
    \usepackage[
        labelfont=bf, % caption names are labeled in bold
        font=scriptsize % smaller font for captions
    ]{caption}
    \usepackage[font=scriptsize]{subcaption} % subfigures

    \newcommand*{\scinot}[2]{#1\times10^{#2}}
    \newcommand*{\dotp}[2]{\left<#1\,\middle|\,#2\right>}
    \newcommand*{\rd}[2]{\frac{\mathrm{d}#1}{\mathrm{d}#2}}
    \newcommand*{\pd}[2]{\frac{\partial#1}{\partial#2}}
    \newcommand*{\rdil}[2]{\mathrm{d}#1 / \mathrm{d}#2}
    \newcommand*{\pdil}[2]{\partial#1 / \partial#2}
    \newcommand*{\rtd}[2]{\frac{\mathrm{d}^2#1}{\mathrm{d}#2^2}}
    \newcommand*{\ptd}[2]{\frac{\partial^2 #1}{\partial#2^2}}
    \newcommand*{\md}[2]{\frac{\mathrm{D}#1}{\mathrm{D}#2}}
    \newcommand*{\pvec}[1]{\vec{#1}^{\,\prime}}
    \newcommand*{\svec}[1]{\vec{#1}\;\!}
    \newcommand*{\bm}[1]{\boldsymbol{\mathbf{#1}}}
    \newcommand*{\uv}[1]{\hat{\bm{#1}}}
    \newcommand*{\ang}[0]{\;\text{\AA}}
    \newcommand*{\mum}[0]{\;\upmu \mathrm{m}}
    \newcommand*{\at}[1]{\left.#1\right|}
    \newcommand*{\bra}[1]{\left<#1\right|}
    \newcommand*{\ket}[1]{\left|#1\right>}
    \newcommand*{\abs}[1]{\left|#1\right|}
    \newcommand*{\ev}[1]{\langle#1\rangle}
    \newcommand*{\p}[1]{\left(#1\right)}
    \newcommand*{\s}[1]{\left[#1\right]}
    \newcommand*{\z}[1]{\left\{#1\right\}}

    \newtheorem{theorem}{Theorem}[section]

    \let\Re\undefined
    \let\Im\undefined
    \DeclareMathOperator{\Res}{Res}
    \DeclareMathOperator{\Re}{Re}
    \DeclareMathOperator{\Im}{Im}
    \DeclareMathOperator{\Log}{Log}
    \DeclareMathOperator{\Arg}{Arg}
    \DeclareMathOperator{\Tr}{Tr}
    \DeclareMathOperator{\E}{E}
    \DeclareMathOperator{\Var}{Var}
    \DeclareMathOperator*{\argmin}{argmin}
    \DeclareMathOperator*{\argmax}{argmax}
    \DeclareMathOperator{\sgn}{sgn}
    \DeclareMathOperator{\diag}{diag\;}

    \colorlet{Corr}{red}

    % \everymath{\displaystyle} % biggify limits of inline sums and integrals
    \tikzstyle{circ} % usage: \node[circ, placement] (label) {text};
        = [draw, circle, fill=white, node distance=3cm, minimum height=2em]
    \definecolor{commentgreen}{rgb}{0,0.6,0}
    \lstset{
        basicstyle=\ttfamily\footnotesize,
        frame=single,
        numbers=left,
        showstringspaces=false,
        keywordstyle=\color{blue},
        stringstyle=\color{purple},
        commentstyle=\color{commentgreen},
        morecomment=[l][\color{magenta}]{\#}
    }

\begin{document}

\def\Snospace~{\S{}} % hack to remove the space left after autorefs
\renewcommand*{\sectionautorefname}{\Snospace}
\renewcommand*{\appendixautorefname}{\Snospace}
\renewcommand*{\figureautorefname}{Fig.}
\renewcommand*{\equationautorefname}{Eq.}
\renewcommand*{\tableautorefname}{Tab.}

\section{Recap}

We analyze in the co-rotating frame
\begin{align}
    \p{\rd{\uv{S}}{t}}_{\rm rot}
        &= \underbrace{\p{\Omega_{\rm SL}\p{\sin I \uv{x} + \cos I \uv{z}}
                - \dot{\Omega} \uv{z}}}
            _{\bm{\Omega}_{\rm eff}} \cdot \uv{S},\label{eq:eom0}\\
        &= \bm{\Omega}_{\rm eff, 0} \times \uv{S}
        + \s{
            \sum\limits_{N = 1}^\infty \bm{\Omega}_{\rm eff, N}
            \sin\p{2\pi N t / t_{\rm LK}}} \times \uv{S}.\label{eq:eom1}
\end{align}
where $\bm{\Omega}_{\rm eff, N}$ is the $N$-th component of the vector Fourier
transform of $\bm{\Omega}_{\rm eff}$.

\begin{itemize}
    \item In the Paper II regime, and in the Paper I regime near $I_0 =
        90^\circ$, we found good conservation of $\theta_{\rm eff, 0}$ where
        \begin{equation}
            \cos \theta_{\rm eff, 0} = \uv{S} \cdot \uv{\Omega}_{\rm eff, 0}.
        \end{equation}
        Note that to estimate the initial $\theta_{\rm eff, 0}$, it is necessary
        to average over an LK cycle, as the angle is fast-varying.

        We justified this analytically by ignoring the $N \geq 1$ terms in
        Eq.~\eqref{eq:eom1} and assuming the merger is very gentle. However,
        there are two observed regimes in which this conservation principle
        breaks down.

    \item The easiest deviation to understand from conservation of $\theta_{\rm
        eff, 0}$ is when the merger is fast (Paper II regime, $I_0 - 90^\circ
        \lesssim 0.4^\circ$); we developed a theory for this in a prior week.

    \item A trickier one is in the Paper I regime, where, even though the merger
        is peaceful, when $\abs{I_0 - 90^\circ} \gtrsim 15^\circ$, we
        numerically find poor conservation of $\theta_{\rm eff, 0}$ (I
        reproduced this in a single simulation at $I_0 = 70^\circ$).

        This is contrasted with the Paper II regime where a peaceful merger is a
        sufficient condition for conservation of $\theta_{\rm eff}$.
\end{itemize}

\textbf{The question I spent the past two weeks investigating is thus:} Why is
the peaceful merger condition sometimes but not always sufficient to guarantee
conservation of $\theta_{\rm eff, 0}$? I have performed many numerical
explorations, and while they shed some more insight on the problem, I do not yet
have a precise answer.

\section{Ongoing Work}

\subsection{Angular Dependence}

% TODO
First, we consider whether conservation of $\theta_{\rm eff, 0}$ has any
angular dependence. For the Paper II regime (I will try this for the Paper I
regime this coming week) and fiducial parameters ($I_0 = 90.5^\circ$), we can
sample the initial $\uv{S}$ uniformly and plot the difference between
$\theta_{\rm eff, 0}^{\rm i}$ and the final $\theta_{\rm sl}^{\rm f} =
\theta_{\rm eff, 0}^{\rm f}$. This shows no angular dependence as shown in
Fig.~\ref{fig:devsgrid}. NB\@: I made a coordinate mistake and only covered half
the unit sphere of possible initial conditions.
\begin{figure}
    \centering
    \includegraphics[width=0.5\textwidth]{plots/devsgrid.png}
    \caption{$\theta_{\rm eff, 0}^{\rm i} - \theta_{\rm sl}^{\rm f}$ for
    uniformly sampled $\uv{S}$. No angular dependence is observed,
    uniform conservation is observed. Note that the $y$-axis is actually
    $\theta_{\rm sl}^{\rm i}$, which is not a very physically meaningful angle,
    but is okay since convergence is so uniform. NB\@: I made a coordinate mitsake
    and only half of the hemisphere is covered; nevertheless, the conclusion
    seems plausibly robust.}\label{fig:devsgrid}
\end{figure}

For reference, the behavior of $\theta_{\rm eff, 0}^{\rm i}$ at late times is
given as the black line in Fig.~\ref{fig:905qN0}.
\begin{figure}
    \centering
    \includegraphics[width=0.5\textwidth]{plots/4sim_90_500_qN0.png}
    \caption{Zoomed in behavior of $\theta_{\rm eff, 0}$ at later times in the
    fiducial Paper II/$I_0 = 90.5^\circ$ simulation.}\label{fig:905qN0}
\end{figure}

Finally, we can track the evolution of each of these initial conditions over
time, as shown in Fig.~\ref{fig:griddist_corot} in the corotating frame.
\begin{figure}
    \centering
    \includegraphics[width=0.5\textwidth]{plots/6_905_griddist.png}
    \caption{Distribution of spin vector orientations as a function of time;
    each blue dot is a realization of the Paper II/$I_0 = 90.5^\circ$ simulation
    for a different initial spin vector. Uniform precession about an effective
    spin axis is observed (fixed orientation in the corotating
    frame).}\label{fig:griddist_corot}
\end{figure}

\subsection{Locally Nondissipative System}

One idea that was developed to attempt to understand whether particular
resonances could kick $\theta_{\rm eff, 0}$ was to consider the locally
nondissipative system. Here, for a given $a$, $e_{\min}$, and $I_{\min}$ (for
$I_0 > 90^\circ$, $I$ is minimized when $e$ is minimized) during inspiral, we
solve Eq.\eqref{eq:eom0} for some initial $\uv{S}$ over $100$--$500$ LK cycles
using $\uv{L}(t)$, $\Omega_{\rm SL}(t)$, and $\dot{\Omega}(t)$ for a
\emph{single} LK cycle, ignoring gravitational radiation.

One useful quantity then to measure is $\Delta \theta_{\rm eff}$ for such a
locally nondissipative system, the difference between the maximum and minimum
$\theta_{\rm eff}$ (angle is to $\bm{\Omega}_{\rm eff}$, not $\bm{ \Omega}_{\rm
eff, 0}$, note; maybe this is bad?) attained. We choose to only measure at each
LK cycle, so
\begin{equation}
    \Delta \theta_{\rm eff} = \max\limits_i \theta_{\rm eff}(\tau_i)
        - \min\limits_j \theta_{\rm eff}(\tau_j),
\end{equation}
where $\tau_i$ are the times of \emph{minimum} eccentricity in each LK cycle.
If $\Delta \theta_{\rm eff}$ is small for the entirety of the fiducial Paper
II simulation, then conservation of $\theta_{\rm eff}$ can easily be
understood.

In reality, it turns out not to be so simple. Note that $\Delta \theta_{\rm eff,
0}$ is in general a function of the $\uv{S}^{\rm i}$. Parameterize $\uv{S}^{\rm
i}$ by $(\theta, \phi)$ in the coordinate system of Eq.~\eqref{eq:eom0}, then a
plot of $\Delta \theta_{\rm eff}(\theta, \phi)$ is given in
Fig.~\ref{fig:ampgridt1692}.
\begin{figure}
    \centering
    \includegraphics[width=0.5\textwidth]{plots/6_ampgrid_905_t1692_txtind.png}
    \caption{Plot of $\Delta \theta_{\rm eff}(\theta, \phi)$ for the locally
    nondissipative system of the fiducial Paper II simulation at $t = 1692$
    (Fig.~\ref{fig:905qN0}). While much of parameter space has $\Delta
    \theta_{\rm eff} = 0$, a clear resonant zone exists. The width of the
    zone decreases at later times.}\label{fig:ampgridt1692}
\end{figure}

Comparing to Fig.~\ref{fig:905qN0}, it is clear that the amplitude of
oscillation of $\theta_{\rm eff}$ from the GW simulation is not consistent
with the prediction of $\Delta \theta_{\rm eff}$. But Fig.~\ref{fig:905qN0}
is not fine-tuned, choosing a different initial spin for the inspiral simulation
shows a similar behavior of $\theta_{\rm eff}$. Evaluating $\Delta
\theta_{\rm eff}$ along the inspiral points seems to underpredict variations
in $\theta_{\rm eff}$, as shown in Fig.~\ref{fig:6_905real}.
\begin{figure}
    \centering
    \includegraphics[width=0.5\textwidth]{plots/6_905_real.png}
    \caption{$\Delta \theta_{\rm eff}$ evaluated for one realization of the
    fiducial simulation, using the $\uv{S}$ at the beginning of each LK cycle as
    initial conditions for the locally nondissipative
    simulation. Bottom plot shows $\abs{\bm{\Omega}_{\rm eff} }$ in time
    units (LK period is implied by horizontal spacing).}\label{fig:6_905real}
\end{figure}

This contrasts with the simulation in the Paper I regime, where the amplitude of
oscillation in $\theta_{\rm eff, 0}$ matches quite well with $\Delta \theta_{\rm
eff}$, see Figs.~\ref{fig:4_70_qN0} and~\ref{fig:ampgrid_569}.
\begin{figure}
    \centering
    \includegraphics[width=0.5\textwidth]{plots/4sim_70_000_qN0.png}
    \caption{Plot of $\theta_{\rm eff, 0}$ near a possible resonant kick for a
    Paper I/$I_0 = 70^\circ$ realisation. This seems similar in nature to the
    kicks seen in Fig.~1 of LL17.}\label{fig:4_70_qN0}
\end{figure}
\begin{figure}
    \centering
    \includegraphics[width=0.5\textwidth]{plots/6_ampgrid_700_t569_txtind.png}
    \caption{Same as Fig.~\ref{fig:ampgridt1692} but for $t = 569$ in
    Fig.~\ref{fig:4_70_qN0}.}\label{fig:ampgrid_569}
\end{figure}

This suggests that the resonant kick behavior relies on variations in
$\theta_{\rm eff, 0}$ being generated by the locally nondissipative dynamics,
rather than GW radiation. If this is the case (pending further investigation;
the results here are very scattered and are somewhat apples-to-oranges, we
should resolve inconsistencies before drawing concrete conclusions), then a
simple comparison of timescales over which the locally nondissipative dynamics
generate kicks to $\theta_{\rm eff, 0}$ to the GW radiation timescale gives the
answer to our proposed question.

\section{Bin's Response to Line in Hang Yu's Paper}

Hang Yu's paper contains a line where they are not sure whether $\theta_{\rm
sl}^{\rm f}$ ends up being $\theta_{\rm sb}^{\rm i}$ or $180^\circ -
\theta_{\rm sb}^{\rm i}$. With our basic theory, in the corotating
frame, this is a very simple insight that we discussed early in this project.
Take the limit where $\bm{\Omega}_{\rm eff}^{\rm i} = -\dot{\Omega}\uv{L}_{\rm
out}$ for simplicity, and perform analysis in the corotating frame (where
$\bm{S}$ must always precess in the positive direction about $\bm{\Omega}_{\rm
eff}$):
\begin{itemize}
    \item At the end of the dynamics, $\uv{S}$ precesses around $\bm{L}_{\rm
        in}$ always in the positive direction, and so $\bm{\Omega}_{\rm
        eff}^{\rm f} = \bm{L}_{\rm in}$

    \item Initially, $\uv{S}$ precesses about $\bm{L}_{\rm out}$ in either the
        positive ($I_0 < 90^\circ$) or negative ($I_0 > 90^\circ$) direction,
        due to the sign of $\dot{\Omega}$ in the LK EOM\@. Requiring that
        $\uv{S}$ precess around $\bm{\Omega}_{\rm eff}^{\rm i}$ in the positive
        direction shows that $\bm{\Omega}_{\rm eff} = \bm{L}_{\rm out}$ when
        $I_0 < 90^\circ$ and $\bm{\Omega}_{\rm eff} = -\bm{L}_{\rm out}$ when
        $I_0 > 90^\circ$.

        Thus, when $I_0 < 90^\circ$, we have conservation of $\theta_{\rm eff} =
        \theta_{\rm sb, i} = \theta_{\rm sl, f}$, while when $I_0 > 90^\circ$ we
        have conservation of $\theta_{\rm eff} = 180 - \theta_{\rm sb, i} =
        \theta_{\rm sl, f}$.
\end{itemize}
This generalizes easily to $I_0 < I_{0,\lim}$ and $I_0 > I_{0, \lim}$ when
$\eta$ is nonzero.

\end{document}

