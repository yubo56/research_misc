    \documentclass[11pt,
        usenames, % allows access to some tikz colors
        dvipsnames % more colors: https://en.wikibooks.org/wiki/LaTeX/Colors
    ]{article}
    \usepackage{
        amsmath,
        amssymb,
        fouriernc, % fourier font w/ new century book
        fancyhdr, % page styling
        lastpage, % footer fanciness
        hyperref, % various links
        setspace, % line spacing
        amsthm, % newtheorem and proof environment
        mathtools, % \Aboxed for boxing inside aligns, among others
        float, % Allow [H] figure env alignment
        enumerate, % Allow custom enumerate numbering
        graphicx, % allow includegraphics with more filetypes
        wasysym, % \smiley!
        upgreek, % \upmu for \mum macro
        listings, % writing TrueType fonts and including code prettily
        tikz, % drawing things
        booktabs, % \bottomrule instead of hline apparently
        xcolor, % colored text
        cancel % can cancel things out!
    }
    \usepackage[margin=1in]{geometry} % page geometry
    \usepackage[
        labelfont=bf, % caption names are labeled in bold
        font=scriptsize % smaller font for captions
    ]{caption}
    \usepackage[font=scriptsize]{subcaption} % subfigures

    \newcommand*{\scinot}[2]{#1\times10^{#2}}
    \newcommand*{\dotp}[2]{\left<#1\,\middle|\,#2\right>}
    \newcommand*{\rd}[2]{\frac{\mathrm{d}#1}{\mathrm{d}#2}}
    \newcommand*{\pd}[2]{\frac{\partial#1}{\partial#2}}
    \newcommand*{\rdil}[2]{\mathrm{d}#1 / \mathrm{d}#2}
    \newcommand*{\pdil}[2]{\partial#1 / \partial#2}
    \newcommand*{\rtd}[2]{\frac{\mathrm{d}^2#1}{\mathrm{d}#2^2}}
    \newcommand*{\ptd}[2]{\frac{\partial^2 #1}{\partial#2^2}}
    \newcommand*{\md}[2]{\frac{\mathrm{D}#1}{\mathrm{D}#2}}
    \newcommand*{\pvec}[1]{\vec{#1}^{\,\prime}}
    \newcommand*{\svec}[1]{\vec{#1}\;\!}
    \newcommand*{\bm}[1]{\boldsymbol{\mathbf{#1}}}
    \newcommand*{\uv}[1]{\hat{\bm{#1}}}
    \newcommand*{\ang}[0]{\;\text{\AA}}
    \newcommand*{\mum}[0]{\;\upmu \mathrm{m}}
    \newcommand*{\at}[1]{\left.#1\right|}
    \newcommand*{\bra}[1]{\left<#1\right|}
    \newcommand*{\ket}[1]{\left|#1\right>}
    \newcommand*{\abs}[1]{\left|#1\right|}
    \newcommand*{\ev}[1]{\langle#1\rangle}
    \newcommand*{\p}[1]{\left(#1\right)}
    \newcommand*{\s}[1]{\left[#1\right]}
    \newcommand*{\z}[1]{\left\{#1\right\}}

    \newtheorem{theorem}{Theorem}[section]

    \let\Re\undefined
    \let\Im\undefined
    \DeclareMathOperator{\Res}{Res}
    \DeclareMathOperator{\Re}{Re}
    \DeclareMathOperator{\Im}{Im}
    \DeclareMathOperator{\Log}{Log}
    \DeclareMathOperator{\Arg}{Arg}
    \DeclareMathOperator{\Tr}{Tr}
    \DeclareMathOperator{\E}{E}
    \DeclareMathOperator{\Var}{Var}
    \DeclareMathOperator*{\argmin}{argmin}
    \DeclareMathOperator*{\argmax}{argmax}
    \DeclareMathOperator{\sgn}{sgn}
    \DeclareMathOperator{\diag}{diag\;}

    \colorlet{Corr}{red}

    % \everymath{\displaystyle} % biggify limits of inline sums and integrals
    \tikzstyle{circ} % usage: \node[circ, placement] (label) {text};
        = [draw, circle, fill=white, node distance=3cm, minimum height=2em]
    \definecolor{commentgreen}{rgb}{0,0.6,0}
    \lstset{
        basicstyle=\ttfamily\footnotesize,
        frame=single,
        numbers=left,
        showstringspaces=false,
        keywordstyle=\color{blue},
        stringstyle=\color{purple},
        commentstyle=\color{commentgreen},
        morecomment=[l][\color{magenta}]{\#}
    }

\begin{document}

\def\Snospace~{\S{}} % hack to remove the space left after autorefs
\renewcommand*{\sectionautorefname}{\Snospace}
\renewcommand*{\appendixautorefname}{\Snospace}
\renewcommand*{\figureautorefname}{Fig.}
\renewcommand*{\equationautorefname}{Eq.}
\renewcommand*{\tableautorefname}{Tab.}

Spent most of the week writing the paper draft, so just a few plots below.

\section{Bifurcation Diagram}

We take $\bm{\Omega}_{\rm e}$ from a full LK simulation, and go to the
co-rotating frame where $\bm{\Omega}_{\rm e} \cdot \uv{y} = 0$. We then evolve
\begin{equation}
    \rd{\uv{S}}{t} = \bm{\Omega}_{\rm e} \times \uv{S},\label{eq:eom}
\end{equation}
using the periodic solution for $\bm{\Omega}_{\rm e}$ over $200 T_{\rm LK}$ (I
ran for $500$ but with a bug, so I ran for $200$ to have it done in time for the
meeting). Then, at each eccentricity maximum, we measure
\begin{equation}
    \cos \theta_{\rm e} \equiv \uv{S} \cdot \uv{\Omega}_{\rm e}.
\end{equation}
We vary the mass ratio $m_1 / m_{12}$, which changes $\Omega_{\rm SL}$ and thus
$\bm{\Omega}_{\rm e}$. We run this for three parameter sets: the Paper I regime
for $I_0 = 88^\circ$ and $e_0 = 10^{-3}$, the Paper I regime for $I_0 =
70^\circ$ and $e_0 = 10^{-3}$, and the Paper II regime for $I_0 = 90.5^\circ$
and $e_0 = 10^{-3}$. These are shown in Fig.~\ref{fig:poincare}. We start from
initial alignment, for convenience, so $\uv{S} = \uv{L}_{\rm in}$.

\begin{figure}
    \centering
    \includegraphics[width=0.45\textwidth]{plots/6_poincare_inner88.png}
    \includegraphics[width=0.45\textwidth]{plots/6_poincare_inner.png}
    \includegraphics[width=0.45\textwidth]{plots/6_poincarescan.png}
    \caption{Bifurcation diagrams for the indicated parameters, where
    $\theta_{\rm e}$ is sampled at every eccentricity maximum for $200$ LK
    cycles.}\label{fig:poincare}
\end{figure}

\section{Individual Trajectories and Comparison to Floquet Theory}

Consider a single trajectory, evaluated when $m_1 / m_{12} = 0.5$. We consider
plotting $\uv{S}$ in the $\uv{x}$--$\uv{z}$ plane. We can evaluate $\uv{S}$ at
all times (that are returned by the integrator) and at eccentricity maxima.
Three such trajectories are presented in Fig.~\ref{fig:poincare_singles}.

\begin{figure}
    \centering
    \includegraphics[width=0.45\textwidth]{plots/single_6_poincare_inner88.png}
    \includegraphics[width=0.45\textwidth]{plots/single_6_poincare_inner.png}
    \includegraphics[width=0.45\textwidth]{plots/single_6_poincare_outer.png}
    \caption{Single $\uv{S}$ trajectories plotted in the $\uv{x}$--$\uv{z}$
    plane. Red dots are evaluated at all times returned by the integrator, black
    dots are those at eccentricity maxima, blue is $\bm{\Omega}_{\rm e}$, green
    is the eigenvector of the monodromy matrix, and the black dashed line is the
    numerically computed axis of rotation of the black dots. Good agreement is
    observed between the numerical axis of rotation and the monodromy matrix
    eigenvector.}\label{fig:poincare_singles}
\end{figure}

In Fig.~\ref{fig:poincare_singles}, we have included three axes: (blue) is the
proposed $\bm{\Omega}_{\rm e}$, which is our proposed axis of rotation, (green)
is the axis of the so-called \emph{monodromy matrix}, described later, and
(black dotted line) is the numerically calculated axis of rotation for the
Poincar\'e section. Note that the aspect ratio isn't 1:1, so perpendicular lines
do not appear perpendicular. It can be seen that the green line and black dotted
line are in excellent agreement.

\subsection{Floquet Theory}

The basis of Floquet theory is that, for any linear system with periodic
coefficients, the \emph{monodromy matrix} gives the evolution over integer
multiples of the period. In our problem, the monodromy matrix is constructed as
follows:
\begin{itemize}
    \item Evolve Eq.~\eqref{eq:eom} using the linearly independent solutions
        $\uv{x}$, $\uv{y}$, and $\uv{z}$. Call these solutions $\phi_1(t)$, $
        \phi_2(t)$, and $\phi_3(t)$.

    \item The monodromy matrix is defined ($T = T_{\rm LK}$ period):
        \begin{equation}
            \bm{\tilde{M}} = \begin{bmatrix}
                \phi_1(0) & \phi_2(0) & \phi_3(0)
            \end{bmatrix}^T \begin{bmatrix}
                \phi_1(T) & \phi_2(T) & \phi_3(T)
            \end{bmatrix}.
        \end{equation}
        Note that $\bm{\tilde{M}}$ must be a proper orthogonal matrix, so it is
        a rotation matrix about some rotation axis $\uv{R}$ by some angle
        $\psi$. $\uv{R}$ is the eigenvalue of $\bm{\tilde{M}}$ with eigenvalue
        $1$.
\end{itemize}
The general solution can then be written
\begin{equation}
    \uv{S}(NT) = \bm{\tilde{M}}^N \uv{S}(0).
\end{equation}
Thus, every $T$, $\uv{S}$ must rotate about $\uv{R}$. $\uv{R}$ is the green line
in Fig.~\ref{fig:poincare_singles}.

Thus, deviation from exact conservation of $\theta_{\rm e}$ occur when $\uv{R}$
differs significantly from $\bm{\Omega}_{\rm e}$. At late times, it's obvious
that $\uv{R} = \uv{\Omega}_{\rm e} = \uv{L}_{\rm in}$.

I still haven't checked whether the deviation from $\theta_{\rm e}$ is due to:
\begin{itemize}
    \item The true conserved angle should be $\uv{S} \cdot \uv{R}$, evaluated at
        each eccentricity maximum.

    \item Nonadiabatic passage through large changes in $\uv{R}$ could kick
        $\theta_{\rm e}$.
\end{itemize}

Finally, note that we generally expect $\uv{R} = \uv{\Omega}_{\rm e}$ when the
$N \geq 1$ components are neglected in Eq.~\eqref{eq:eom}, so this is consistent
with our earlier picture.

\end{document}

