    \documentclass[11pt,
        usenames, % allows access to some tikz colors
        dvipsnames % more colors: https://en.wikibooks.org/wiki/LaTeX/Colors
    ]{article}
    \usepackage{
        amsmath,
        amssymb,
        fouriernc, % fourier font w/ new century book
        fancyhdr, % page styling
        lastpage, % footer fanciness
        hyperref, % various links
        setspace, % line spacing
        amsthm, % newtheorem and proof environment
        mathtools, % \Aboxed for boxing inside aligns, among others
        float, % Allow [H] figure env alignment
        enumerate, % Allow custom enumerate numbering
        graphicx, % allow includegraphics with more filetypes
        wasysym, % \smiley!
        upgreek, % \upmu for \mum macro
        listings, % writing TrueType fonts and including code prettily
        tikz, % drawing things
        booktabs, % \bottomrule instead of hline apparently
        xcolor, % colored text
        cancel % can cancel things out!
    }
    \usepackage[margin=1in]{geometry} % page geometry
    \usepackage[
        labelfont=bf, % caption names are labeled in bold
        font=scriptsize % smaller font for captions
    ]{caption}
    \usepackage[font=scriptsize]{subcaption} % subfigures

    \newcommand*{\scinot}[2]{#1\times10^{#2}}
    \newcommand*{\dotp}[2]{\left<#1\,\middle|\,#2\right>}
    \newcommand*{\rd}[2]{\frac{\mathrm{d}#1}{\mathrm{d}#2}}
    \newcommand*{\pd}[2]{\frac{\partial#1}{\partial#2}}
    \newcommand*{\rdil}[2]{\mathrm{d}#1 / \mathrm{d}#2}
    \newcommand*{\pdil}[2]{\partial#1 / \partial#2}
    \newcommand*{\rtd}[2]{\frac{\mathrm{d}^2#1}{\mathrm{d}#2^2}}
    \newcommand*{\ptd}[2]{\frac{\partial^2 #1}{\partial#2^2}}
    \newcommand*{\md}[2]{\frac{\mathrm{D}#1}{\mathrm{D}#2}}
    \newcommand*{\pvec}[1]{\vec{#1}^{\,\prime}}
    \newcommand*{\svec}[1]{\vec{#1}\;\!}
    \newcommand*{\bm}[1]{\boldsymbol{\mathbf{#1}}}
    \newcommand*{\uv}[1]{\hat{\bm{#1}}}
    \newcommand*{\ang}[0]{\;\text{\AA}}
    \newcommand*{\mum}[0]{\;\upmu \mathrm{m}}
    \newcommand*{\at}[1]{\left.#1\right|}
    \newcommand*{\bra}[1]{\left<#1\right|}
    \newcommand*{\ket}[1]{\left|#1\right>}
    \newcommand*{\abs}[1]{\left|#1\right|}
    \newcommand*{\ev}[1]{\langle#1\rangle}
    \newcommand*{\p}[1]{\left(#1\right)}
    \newcommand*{\s}[1]{\left[#1\right]}
    \newcommand*{\z}[1]{\left\{#1\right\}}

    \newtheorem{theorem}{Theorem}[section]

    \let\Re\undefined
    \let\Im\undefined
    \DeclareMathOperator{\Res}{Res}
    \DeclareMathOperator{\Re}{Re}
    \DeclareMathOperator{\Im}{Im}
    \DeclareMathOperator{\Log}{Log}
    \DeclareMathOperator{\Arg}{Arg}
    \DeclareMathOperator{\Tr}{Tr}
    \DeclareMathOperator{\E}{E}
    \DeclareMathOperator{\Var}{Var}
    \DeclareMathOperator*{\argmin}{argmin}
    \DeclareMathOperator*{\argmax}{argmax}
    \DeclareMathOperator{\sgn}{sgn}
    \DeclareMathOperator{\diag}{diag\;}

    \colorlet{Corr}{red}

    % \everymath{\displaystyle} % biggify limits of inline sums and integrals
    \tikzstyle{circ} % usage: \node[circ, placement] (label) {text};
        = [draw, circle, fill=white, node distance=3cm, minimum height=2em]
    \definecolor{commentgreen}{rgb}{0,0.6,0}
    \lstset{
        basicstyle=\ttfamily\footnotesize,
        frame=single,
        numbers=left,
        showstringspaces=false,
        keywordstyle=\color{blue},
        stringstyle=\color{purple},
        commentstyle=\color{commentgreen},
        morecomment=[l][\color{magenta}]{\#}
    }

\begin{document}

\def\Snospace~{\S{}} % hack to remove the space left after autorefs
\renewcommand*{\sectionautorefname}{\Snospace}
\renewcommand*{\appendixautorefname}{\Snospace}
\renewcommand*{\figureautorefname}{Fig.}
\renewcommand*{\equationautorefname}{Eq.}
\renewcommand*{\tableautorefname}{Tab.}

Throughout this writeup, we will only consider the $N = 0$ harmonic of
$\bm{\Omega}_{\rm eff}$, such that
\begin{align}
    \p{\rd{\uv{S}}{t}}_{\rm rot}
            &= \ev{-\dot{\Omega}\uv{z} + \Omega_{\rm SL} \uv{L}_{\rm in}}
                \times \uv{S}\label{eq:initial},\\
        &\equiv \bm{\Omega}_{\rm eff,0} \times \uv{S},
\end{align}
in the corotating frame (where $\uv{L}_{\rm out} = \uv{z}$ and $\uv{L}_{\rm in}$
lies in the $x$-$z$ plane),

\section{Adiabaticity Criterion (DL) and Plots}

\textbf{Adiabaticity Criterion (DL):} Adiabaticity requires
\begin{equation}
    \abs{\rdil{\hat{\Omega}_{\rm eff,0}}{t}} \ll \Omega_{\rm eff,0} \equiv
        \abs{\bm{\Omega}_{\rm eff,0}}.\label{eq:dlad}
\end{equation}
To parameterize $\abs{\rdil{\hat{\Omega}_{\rm eff,0}}{t}}$, call $I_{\rm out}$
the angle between $\uv{\Omega}_{\rm eff,0}$ and $\uv{L}_{\rm out}$, such that
\begin{equation}
    \uv{\Omega}_{\rm eff,0} = \cos I_{\rm out} \uv{z} + \sin I_{\rm out}\uv{x},
\end{equation}
as shown in Fig.~\ref{fig:iout}. Thus,
\begin{equation}
    \abs{\rdil{\hat{\Omega}_{\rm eff,0}}{t}} = \rd{I_{\rm out}}{t},
\end{equation}
and the two rates of change to compare in Eq.~\eqref{eq:dlad} are $\dot{I}_{\rm
out}$ and $\Omega_{\rm eff,0}$.
\begin{figure}[t]
    \centering
    \begin{tikzpicture}[scale=1.5]
        \draw[->] (0, 0)--(1, 1);
        \draw[->] (0, 0)--(-1.3, -0.3);
        \draw[->] (0, 0) -- (0, 1.414);
        \node[right] at (1, 1) {$\uv{\Omega}_{\rm eff,0}$};
        \node[left] at (-1.3, -0.3) {$\left\langle\uv{L}\right\rangle$};
        \node[above] at (0, 1.414) {$\uv{L}_{\rm out}$};
        \node[right] at (0, 1.0) {$+I_{\rm out}$};
    \end{tikzpicture}
    \caption{Definition of the angle $I_{\rm out}$. It turns out that for $I_0 >
    90^\circ$, $\Omega_{\rm eff, 0}$ eventually aligns with $-\ev{\uv{L}}$,
    hence the sign convention for $I_{\rm out}$.}\label{fig:iout}
\end{figure}

To examine how well this works, let's examine two simulations in
Fig.~\ref{fig:phidots}. Explanation in caption.
\begin{figure}
    \centering
    \includegraphics[width=0.45\textwidth]{plots/4sim_90_250_phidots.png}
    \includegraphics[width=0.45\textwidth]{plots/4sim_90_500_phidots.png}
    \caption{Plot of precession frequencies in two simulations (left: $I_0 =
    90.25^\circ$, ``fast merger'', right: $I_0 = 90.5^\circ$, ``slow merger'').
    The relevant precession frequencies are $\Omega_{\rm eff,0}$ (black) and
    $\dot{I}_{\rm out} \equiv \abs{\rdil{\uv{\Omega}_{\rm eff,0}}{t}}$ (blue). The
    change in $\theta_{\rm eff, 0}$ is shown in the title.
    Note that in the left plot, $\dot{I}_{\rm out}$ exceeds $\Omega_{\rm eff,0}$,
    and as a result conservation of $\theta_{\rm eff,0}$ is comparatively poor.
    In the right plot, $\dot{I}_{\rm out}$ is always much smaller than
    $\Omega_{\rm eff,0}$, causing conservation of $\theta_{\rm eff, 0}$ to
    improve by three orders of magnitude.}\label{fig:phidots}
\end{figure}

\section{Yubo's Old Result}\label{s:old_result}

I've continually referred to a set of equations of motion without ever
explaining them correctly, but they are approximately equivalent to the results
of the above section and provide more precise quantitative insights.

Construct a new coordinate system such that $\uv{z}' = \uv{\Omega}_{\rm eff,
0}$, and $\uv{x}'$ lies in the plane of $\uv{L}_{\rm in}$ and $\uv{L}_{\rm out}$
(this corresponds to rotating Fig.~\ref{fig:iout} counter-clockwise by $I_{\rm
out}$). Define a spherical coordinate system $\p{\theta_{\rm eff, 0}, \phi_{\rm
eff, 0}}$ for spin vector $\uv{S}$ in this new coordinate system, then they can
be shown to obey equations of motion:
\begin{align}
    \rd{\phi_{\rm eff,0}}{t} &= \p{-\dot{\Omega}\cos I_{\rm out}
            + \Omega_{SL}\p{\cos \p{I + I_{\rm out}}}}\label{eq:dphi},\\
        &\simeq \Omega_{\rm eff, 0}\label{eq:approx},\\
    \rd{\theta_{\rm eff,0}}{t} &= \dot{I}_{\rm out}\cos \phi_{\rm eff,
            0}\label{eq:dqreal},\\
        &\approx \abs{\rd{\uv{\Omega}_{\rm eff,0}}{t}}
            \cos \p{\Omega_{\rm eff, 0}t}.\label{eq:dq}
\end{align}
The approximate scaling in Eq.~\eqref{eq:approx} can be seen as follows:
$\Omega_{\rm eff, 0} \simeq \max\p{\abs{\dot{\Omega}}, \Omega_{\rm SL}}$, while
when $\abs{\dot{\Omega}} \gg\; \textcolor{Corr}{[\ll]} \Omega_{\rm SL}$, $\cos I
\; \textcolor{Corr}{[\cos (I + I_{\rm out})]}\approx 1$, so the integrand also
satisfies $\rd{\phi_{\rm eff,0}}{t} \simeq \max\p{\abs{\dot{\Omega}},
\Omega_{\rm SL}}$. As such, Eq.~\eqref{eq:dphi} can indeed be approximated by
Eq.~\eqref{eq:approx}.

Now, if we require the usual adiabaticity condition given by
Eq.~\eqref{eq:dlad}, then it is clear why Eq.~\eqref{eq:dq} predicts $\Delta
\theta_{\rm eff, 0} \to 0$: we are integrating a small quantity multiplied by a
fast-varying phase. As such, Eq.~\eqref{eq:dq} \emph{can be thought to be a
quantitative prediction of the deviation from adiabaticity.}

\subsection{Derivation of Equations of Motion}

We provide a very brief derivation of Eqs.~\eqref{eq:dphi}
and~\eqref{eq:dqreal}. We start with Eq.~\eqref{eq:initial} and rotate about the
$\uv{y}$ axis (pointing into the page) such that $\uv{\Omega}_{\rm eff,0}$
points upwards. This requires rotation by $I_{\rm out}$ satisfying
\begin{equation}
    -\dot{\Omega} \sin I_{\rm out} + \Omega_{SL}\sin \p{I + I_{\rm out}}
        = 0.\label{eq:I_out_def}
\end{equation}
Here, $I$ is the angle between $\uv{L}_{\rm in}$ and $\uv{L}_{\rm out}$; this
equation is general for $I > 90^\circ$ and $I < 90^\circ$. The equation of
motion in this frame is then
\begin{equation}
    \rd{\uv{S}}{t} = \s{-\dot{\Omega} \cos I_{\rm out}\uv{z} + \Omega_{SL}
        \cos \p{I + I_{\rm out}}\uv{z} - \dot{I}_{\rm out}\uv{y}} \times
        \uv{S}.\label{eq:shat_eom}
\end{equation}
The components of this equation then directly give the equations of motion I
used.

\section{Today's plot}\label{s:today}

Today, I showed an old plot with some new information, shown in
Fig.~\ref{fig:newplot}. The plot depicts, for a full inspiral
simulation\footnote{ In all quantities shown in this plot, $\Omega_{\rm eff, 0}$
(an LK-averaged
quantity) is linearly interpolated within each LK cycle.}:
\begin{description}
    \item[Grey Line] This shows $\cos^{-1}\p{\uv{S} \cdot \Omega_{\rm eff, 0}}
        (t)$ at all times. Fluctuations are expected since $\uv{S}$ fluctuates
        within each Lidov-Kozai (LK) period.

    \item[Blue Dots] This shows $\cos^{-1}\p{\uv{S} \cdot \Omega_{\rm eff,
        0}}(T_i)$ where each $T_i$ is the middle of each LK period (maximum
        eccentricity). Changing to sample the start of each period does not
        change the plot significantly.

    \item[Red Dots] This shows the LK-average of the grey line. Specifically,
        the $i$-th red dot denotes (for $T_i$ the $i$-th LK cycle)
        \begin{equation}
            \theta_{\rm red, i} \equiv \frac{1}{T_{i + 1} - T_i}
                \int\limits_{T_i}^{T_{i + 1}}
                    \cos^{-1}\p{\uv{S} \cdot \Omega_{\rm eff, 0}}\;\mathrm{d}t.
        \end{equation}

    \item[Green Line] This is new, and possibly not very useful. A function such
        as $g(t) = \int^t A\cos(\omega t)\;\mathrm{dt}$ oscillates with
        amplitude $A/\omega$. Thus, if we know $\dot{I}$ and $\Omega_{\rm eff,
        0}$ at all times, we can make a prediction about the amplitude of
        oscillation of $\theta_{\rm red,i}$ if the results of
        Section~\ref{s:old_result} are a complete description of $\theta_{\rm
        red, i}$ (i.e. \emph{if the only thing driving
        oscillations is nonadiabaticity due to a finite-time merger}).

        The green line visibly underpredicts the oscillations of the red dots at
        $t \sim 1650$-$1700$, so the equations given in
        Section~\ref{s:old_result} fails to capture the dynamics of $\theta_{\rm
        eff, 0}$ at all times, even though it predicts the final deviations
        well.
\end{description}
\begin{figure}
    \centering
    \includegraphics[width=\textwidth]{plots/4sim_90_500_qN0.png}
    \caption{New plot shown today (May 29, 2020) during group meeting. Owing to
    complexity, the description is given in the text (Section~\ref{s:today}).
    }\label{fig:newplot}
\end{figure}

\end{document}

