    \documentclass[11pt,
        usenames, % allows access to some tikz colors
        dvipsnames % more colors: https://en.wikibooks.org/wiki/LaTeX/Colors
    ]{article}
    \usepackage{
        amsmath,
        amssymb,
        fouriernc, % fourier font w/ new century book
        fancyhdr, % page styling
        lastpage, % footer fanciness
        hyperref, % various links
        setspace, % line spacing
        amsthm, % newtheorem and proof environment
        mathtools, % \Aboxed for boxing inside aligns, among others
        float, % Allow [H] figure env alignment
        enumerate, % Allow custom enumerate numbering
        graphicx, % allow includegraphics with more filetypes
        wasysym, % \smiley!
        upgreek, % \upmu for \mum macro
        listings, % writing TrueType fonts and including code prettily
        tikz, % drawing things
        booktabs, % \bottomrule instead of hline apparently
        xcolor, % colored text
        cancel % can cancel things out!
    }
    \usepackage[margin=1in]{geometry} % page geometry
    \usepackage[
        labelfont=bf, % caption names are labeled in bold
        font=scriptsize % smaller font for captions
    ]{caption}
    \usepackage[font=scriptsize]{subcaption} % subfigures

    \newcommand*{\scinot}[2]{#1\times10^{#2}}
    \newcommand*{\dotp}[2]{\left<#1\,\middle|\,#2\right>}
    \newcommand*{\rd}[2]{\frac{\mathrm{d}#1}{\mathrm{d}#2}}
    \newcommand*{\pd}[2]{\frac{\partial#1}{\partial#2}}
    \newcommand*{\rdil}[2]{\mathrm{d}#1 / \mathrm{d}#2}
    \newcommand*{\pdil}[2]{\partial#1 / \partial#2}
    \newcommand*{\rtd}[2]{\frac{\mathrm{d}^2#1}{\mathrm{d}#2^2}}
    \newcommand*{\ptd}[2]{\frac{\partial^2 #1}{\partial#2^2}}
    \newcommand*{\md}[2]{\frac{\mathrm{D}#1}{\mathrm{D}#2}}
    \newcommand*{\pvec}[1]{\vec{#1}^{\,\prime}}
    \newcommand*{\svec}[1]{\vec{#1}\;\!}
    \newcommand*{\bm}[1]{\boldsymbol{\mathbf{#1}}}
    \newcommand*{\uv}[1]{\hat{\bm{#1}}}
    \newcommand*{\ang}[0]{\;\text{\AA}}
    \newcommand*{\mum}[0]{\;\upmu \mathrm{m}}
    \newcommand*{\at}[1]{\left.#1\right|}
    \newcommand*{\bra}[1]{\left<#1\right|}
    \newcommand*{\ket}[1]{\left|#1\right>}
    \newcommand*{\abs}[1]{\left|#1\right|}
    \newcommand*{\ev}[1]{\langle#1\rangle}
    \newcommand*{\p}[1]{\left(#1\right)}
    \newcommand*{\s}[1]{\left[#1\right]}
    \newcommand*{\z}[1]{\left\{#1\right\}}

    \newtheorem{theorem}{Theorem}[section]

    \let\Re\undefined
    \let\Im\undefined
    \DeclareMathOperator{\Res}{Res}
    \DeclareMathOperator{\Re}{Re}
    \DeclareMathOperator{\Im}{Im}
    \DeclareMathOperator{\Log}{Log}
    \DeclareMathOperator{\Arg}{Arg}
    \DeclareMathOperator{\Tr}{Tr}
    \DeclareMathOperator{\E}{E}
    \DeclareMathOperator{\Var}{Var}
    \DeclareMathOperator*{\argmin}{argmin}
    \DeclareMathOperator*{\argmax}{argmax}
    \DeclareMathOperator{\sgn}{sgn}
    \DeclareMathOperator{\diag}{diag\;}

    \colorlet{Corr}{red}

    % \everymath{\displaystyle} % biggify limits of inline sums and integrals
    \tikzstyle{circ} % usage: \node[circ, placement] (label) {text};
        = [draw, circle, fill=white, node distance=3cm, minimum height=2em]
    \definecolor{commentgreen}{rgb}{0,0.6,0}
    \lstset{
        basicstyle=\ttfamily\footnotesize,
        frame=single,
        numbers=left,
        showstringspaces=false,
        keywordstyle=\color{blue},
        stringstyle=\color{purple},
        commentstyle=\color{commentgreen},
        morecomment=[l][\color{magenta}]{\#}
    }

\begin{document}

\def\Snospace~{\S{}} % hack to remove the space left after autorefs
\renewcommand*{\sectionautorefname}{\Snospace}
\renewcommand*{\appendixautorefname}{\Snospace}
\renewcommand*{\figureautorefname}{Fig.}
\renewcommand*{\equationautorefname}{Eq.}
\renewcommand*{\tableautorefname}{Tab.}

\onehalfspacing

\pagestyle{fancy}
\rfoot{Yubo Su}
\rhead{\today}
\cfoot{\thepage/\pageref{LastPage}}

\section{Examination of Resonant Terms}

We study the equation of motion
\begin{align}
    \rd{S_{\perp}}{t} ={}& i\overline{\Omega}_{\rm e}S_\perp
        + \sum\limits_{N = 1}^\infty\s{
            \cos \p{\Delta I_{\rm eN}}S_\perp
        - i\cos \bar{\theta}_{\rm e} \sin \p{\Delta I_{\rm eN}}}
            \Omega_{\rm eN}\cos (N\Omega_{\rm LK} t).\label{eq:full}
\end{align}

\subsection{Summary of Existing Results in My Paper}

For simplicity, we consider only a single $N$. In the paper, we obtained the
result
\begin{align}
    S_\perp(t) &= S_\perp(t_{\rm i})
        \exp\s{i\overline{\Omega}_{\rm e}\p{t_{\rm f} - t_{\rm i}}
            + \frac{\cos\p{\Delta I_{\rm eN}}
            \Omega_{\rm eN}}{N\Omega_{\rm LK}}\s{\sin\p{N\Omega_{\rm LK}
            t_{\rm f}} - \sin \p{N\Omega_{\rm LK} t_{\rm i}}}}.
            \label{eq:paper_parametric}
\end{align}
when neglecting the driving term, and the result
\begin{align}
    e^{-i\overline{\Omega}_{\rm e}t}S_{\perp}\bigg|_{t_{\rm i}}^{t_{\rm f}}
        &= -\int\limits_{t_{\rm i}}^{t_{\rm f}}
            \frac{i\sin\p{\Delta I_{\rm eN}} \Omega_{\rm eN}}{2}
                e^{-i\overline{\Omega}_{\rm e}t + iN\Omega_{\rm LK} t} \cos
                \bar{\theta}_{\rm e}
            \;\mathrm{d}t,\label{eq:paper_driving}
\end{align}
when neglecting the parametric term.

\subsection{General Result for a Single Resonance}

In full generality, if we define
\begin{align}
    \Phi &\equiv \int\limits^t
        i\overline{\Omega}_{\rm e} + \cos (N\Omega t)
            \cos \p{\Delta I_{\rm N}} \Omega_{\rm eN}\;\mathrm{d}t,\nonumber\\
        &= i\overline{\Omega}_{\rm e}t - \frac{\cos \Delta I_{\rm N}
            \Omega_{\rm eN}}{N\Omega}\sin (N\Omega t),\nonumber\\
        &\equiv i\overline{\Omega}_{\rm e}t + \eta \sin\p{N\Omega t},
\end{align}
where $\eta \equiv \p{\cos \Delta I_{\rm N} \Omega_{\rm eN}}/\p{N\Omega}$, then it is
easy to obtain solution
\begin{align}
    e^{-\Phi(t)} S_\perp(t) - e^{-\Phi\p{t_{\rm i}}}S_{\perp}(t_{\rm i}) &=
        \int\limits_{t_{\rm i}}^{t_{\rm f}}
            -e^{-\Phi(\tau)}
            i\cos \bar{\theta}_{\rm e} \cos (N\Omega \tau)\p{\sin \Delta I_{\rm
            N}} \Omega_{\rm eN}\;\mathrm{d}\tau,\nonumber\\
        &\equiv A\int\limits_{t_{\rm i}}^{t_{\rm f}}
            \cos\p{N \Omega \tau}e^{-\Phi(\tau)} \;\mathrm{d}\tau,
\end{align}
where $A = -i \cos \bar{\theta}_{\rm e} \sin \Delta I_{\rm N} \Omega_{\rm eN}$.
We note in passing that if $A = 0$, we recover Eq.~\eqref{eq:paper_parametric}.
We further expand
\begin{equation}
    \int\limits_{t_{\rm i}}^{t_{\rm f}} \cos\p{N \Omega \tau}e^{-\Phi(\tau)}
            \;\mathrm{d}\tau
        = \frac{1}{2}\int\limits_{t_{\rm i}}^{t_{\rm f}}
            \p{e^{iN\Omega \tau} + e^{-iN\Omega \tau}}
            \exp\p{i\overline{\Omega}_{\rm e}\tau + \eta \sin\p{N\Omega \tau}}
            \;\mathrm{d}\tau.
\end{equation}
If we define $\omega_{\pm} = \overline{\Omega}_{\rm e} \pm N\Omega$,
understanding the behavior of the term above requires we understand the integral
\begin{align}
    \int\limits_{t_{\rm i}}^{t_{\rm f}}
        \exp\p{i\omega_{\pm}\tau + \eta \sin\p{N\Omega \tau}}\;\mathrm{d}\tau
        &= \frac{1}{\omega_{\pm}}
            \int\limits_{x_{\rm i}}^{x_{\rm f}}
                \exp\s{-ix' - \eta \sin \beta x}\;\mathrm{d}x',
                \label{eq:key}
\end{align}
where $x' \equiv \omega_{\pm} \tau$ and $x_{\rm i, f} = \omega_{\pm} t_{\rm i,
f}$, and we've defined $\beta \equiv N\Omega / \omega_{\pm}$. We also note at
this point that if $\eta = 0$, the above integral oscillates between $\pm
\frac{1}{\omega_{\pm}}$, so the total amplitude of oscillation of $S_{\perp}$ is
$A / \min\p{\omega_+, \omega_-} = A / \p{\overline{\Omega}_{\rm e} - N\Omega}$,
since $\overline{\Omega}_{\rm e} > 0$, and we recover
Eq.~\eqref{eq:paper_driving}.

At this point, we have recovered both limits considered in the paper, but the
$\eta$-dependent term in the integrand of Eq.~\eqref{eq:key} is new to this
treatment. The effect of this term can easily be calculated analytically,
however. We take $\omega_{\pm} \neq 0$, since this is the resonance already
well understood in the paper, then
\begin{align}
    I(x_{\rm f}) = \int\limits_{x_{\rm i}}^{x_{\rm f}}
            \exp\s{-ix' - \eta \sin \beta x}\;\mathrm{d}x'
        &= \int\limits_{x_{\rm i}}^{x_{\rm f}}
            \p{\cos x' - i\sin x'}
                \sum\limits_{k = 0}^\infty
                \frac{\p{-\eta \sin (\beta x')}^k}{k!}
            \;\mathrm{d}x'.\label{eq:integral}
\end{align}
We next examine the general power-reduction trigonometric
identities\footnote{Zwillinger, Daniel, ed. CRC standard mathematical tables and
formulae. CRC press, 2002.}:
\begin{align}
    \sin^{2n}y
        &= \frac{1}{2^{2n}} \binom{2n}{n}
            + \frac{\p{-1}^n}{2^{2n - 1}}\sum\limits_{k = 0}^{n - 1}
                \p{-1}^k\binom{2n}{k}
                \cos\s{2\p{n - k}y},\label{eq:identity1}\\
    \sin^{2n + 1}y &= \frac{\p{-1}^n}{4^n}\sum\limits_{k = 0}^n
        \p{-1}^k \binom{2n + 1}{k}\sin\s{\p{2n + 1 - 2k}y}.\label{eq:identity2}
\end{align}
We consider three cases for $\beta$:
\begin{itemize}
    \item If $\beta$ is irrational, $\sin^k\p{\beta x'}$ decomposes into
        trigonometric functions with irrational frequency, which when
        integrated by $\cos x'$ or $\sin x'$ will always be bounded.

    \item If $\beta = 1/q$ for some integer $q$, then let's evaluate
        Eq.~\eqref{eq:integral} over interval $x_{\rm f} - x_{\rm i} = 2\pi q$.
        Then many terms will vanish since the trigonometric functions satisfy
        orthogonality conditions:
        \begin{align}
            \int\limits_0^{2\pi} \cos mx \cos nx\;\mathrm{d}x &= \delta_{mn},\\
            \int\limits_0^{2\pi} \sin mx \sin nx\;\mathrm{d}x &= \delta_{mn},\\
            \int\limits_0^{2\pi} \sin mx \cos nx\;\mathrm{d}x &= 0,
        \end{align}
        where $\delta_{mn}$ is the Kronecker delta. However, $\sin^{q}\p{x'/q}$
        will contain either a $\sin\p{x'}$ or $\cos\p{x'}$ term if $q$ is odd or
        even respectively. The coefficient of this term is given by
        either Eq.~\eqref{eq:identity1} or~\eqref{eq:identity2}, and we conclude
        \begin{equation}
            \abs{I\p{x_i + 2\pi mq}}
                \approx 2\pi mq\frac{\eta^q}{q!}\frac{1}{2^q}.
        \end{equation}
        Note that this formula is approximate, because the in
        Eq.~\eqref{eq:integral} of form $\sin^{q + 2k}\p{x' / q}$ for positive
        integer $k$ will also contain factors of $\sin\p{x'}$ or $\cos\p{x'}$,
        but these are higher order corrections. We conclude that when $\beta =
        1/q$, $S_{\perp}(t)$ grows without bound, and the growth rate is
        estimated by
        \begin{equation}
            \rd{\abs{I(x)}}{x} \approx \frac{\eta^q}{q!2^q}.
                \label{eq:growth}
        \end{equation}

    \item If $\beta = p/q$ for integers $p, q$, we only get unbounded growth for
        $I(x)$ if either $2\p{n - k}p/q = 1$ or $(2\p{n - k} + 1)p/q = 1$ for
        any integers $n, k$. Since both $2\p{n - k}$ and $(2\p{n - k} + 1)$ are
        integers, their product with $p / q$ can only equal $1$ if $p = 1$,
        which is the case studied above. Otherwise, there is also no unbounded
        growth.
\end{itemize}
In conclusion, the resonance condition is, for nonzero integer $q$,
\begin{equation}
    \beta = \frac{1}{q} = \frac{1}{\overline{\Omega}_{\rm e} / N\Omega \pm 1}.
\end{equation}
This is satisfied when $\overline{\Omega}_{\rm e} / N\Omega$ is an integer,
except when $\overline{\Omega}_{\rm e} = N\Omega$ as we already have studied in
the paper. However, the growth rate of this instability falls off very quickly
for large $q$, see Eq.~\eqref{eq:growth}.

\subsection{Numerical Simulations}

We numerically compute the two integrals
\begin{align}
    F(t) &= \int\limits_0^t \cos x' e^{-\eta \sin \beta x'}\;\mathrm{d}x',
        \label{eq:cos}\\
    G(t) &= \int\limits_0^t \sin x' e^{-\eta \sin \beta x'}\;\mathrm{d}x'.
        \label{eq:sin}
\end{align}
We choose $\eta = 1$ for simplicity. We expect $F(t)$ to have resonant growth
when $\beta = 1 / 2n$, and $G(t)$ to have resonant growth when $\beta = 1 / (2n
+ 1)$ for $n \in \mathbb{Z}_+$. These are plotted in
Fig.~\ref{fig:resonant_check}.
\begin{figure}
    \centering
    \includegraphics[width=0.45\columnwidth]{../scripts/lk90/exact_resonance_check.png}
    \includegraphics[width=0.45\columnwidth]{../scripts/lk90/exact_resonance_sin.png}
    \caption{Plot of $F(t)$ [Eq.~\eqref{eq:cos}, left] and plot of $G(t)$
    [Eq.~\eqref{eq:sin}, right]. Our analysis suggests resonant growth when
    $\beta = 1 / 2n$ for $F(t)$ and when $\beta = 1 / (2n + 1)$ for $G(t)$,
    where $n \in \mathbb{Z}_+$, which agrees with the simulation. The thick
    grey lines are the analytic growth rates predicted by
    Eq.~\eqref{eq:growth}, illustrating good
    agreement.}\label{fig:resonant_check}
\end{figure}

\end{document}

