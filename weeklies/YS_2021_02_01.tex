% chktex-file 8
    \documentclass[11pt,
        usenames, % allows access to some tikz colors
        dvipsnames % more colors: https://en.wikibooks.org/wiki/LaTeX/Colors
    ]{article}
    \usepackage{
        amsmath,
        amssymb,
        fouriernc, % fourier font w/ new century book
        fancyhdr, % page styling
        lastpage, % footer fanciness
        hyperref, % various links
        setspace, % line spacing
        amsthm, % newtheorem and proof environment
        mathtools, % \Aboxed for boxing inside aligns, among others
        float, % Allow [H] figure env alignment
        enumerate, % Allow custom enumerate numbering
        graphicx, % allow includegraphics with more filetypes
        wasysym, % \smiley!
        upgreek, % \upmu for \mum macro
        listings, % writing TrueType fonts and including code prettily
        tikz, % drawing things
        booktabs, % \bottomrule instead of hline apparently
        xcolor, % colored text
        cancel % can cancel things out!
    }
    \usepackage[margin=1in]{geometry} % page geometry
    \usepackage[
        labelfont=bf, % caption names are labeled in bold
        font=scriptsize % smaller font for captions
    ]{caption}
    \usepackage[font=scriptsize]{subcaption} % subfigures

    \newcommand*{\scinot}[2]{#1\times10^{#2}}
    \newcommand*{\dotp}[2]{\left<#1\,\middle|\,#2\right>}
    \newcommand*{\rd}[2]{\frac{\mathrm{d}#1}{\mathrm{d}#2}}
    \newcommand*{\pd}[2]{\frac{\partial#1}{\partial#2}}
    \newcommand*{\rdil}[2]{\mathrm{d}#1 / \mathrm{d}#2}
    \newcommand*{\pdil}[2]{\partial#1 / \partial#2}
    \newcommand*{\rtd}[2]{\frac{\mathrm{d}^2#1}{\mathrm{d}#2^2}}
    \newcommand*{\ptd}[2]{\frac{\partial^2 #1}{\partial#2^2}}
    \newcommand*{\md}[2]{\frac{\mathrm{D}#1}{\mathrm{D}#2}}
    \newcommand*{\pvec}[1]{\vec{#1}^{\,\prime}}
    \newcommand*{\svec}[1]{\vec{#1}\;\!}
    \newcommand*{\bm}[1]{\boldsymbol{\mathbf{#1}}}
    \newcommand*{\uv}[1]{\hat{\bm{#1}}}
    \newcommand*{\ang}[0]{\;\text{\AA}}
    \newcommand*{\mum}[0]{\;\upmu \mathrm{m}}
    \newcommand*{\at}[1]{\left.#1\right|}
    \newcommand*{\bra}[1]{\left<#1\right|}
    \newcommand*{\ket}[1]{\left|#1\right>}
    \newcommand*{\abs}[1]{\left|#1\right|}
    \newcommand*{\ev}[1]{\left\langle#1\right\rangle}
    \newcommand*{\p}[1]{\left(#1\right)}
    \newcommand*{\s}[1]{\left[#1\right]}
    \newcommand*{\z}[1]{\left\{#1\right\}}

    \newtheorem{theorem}{Theorem}[section]

    \let\Re\undefined
    \let\Im\undefined
    \DeclareMathOperator{\Res}{Res}
    \DeclareMathOperator{\Re}{Re}
    \DeclareMathOperator{\Im}{Im}
    \DeclareMathOperator{\Log}{Log}
    \DeclareMathOperator{\Arg}{Arg}
    \DeclareMathOperator{\Tr}{Tr}
    \DeclareMathOperator{\E}{E}
    \DeclareMathOperator{\Var}{Var}
    \DeclareMathOperator*{\argmin}{argmin}
    \DeclareMathOperator*{\argmax}{argmax}
    \DeclareMathOperator{\sgn}{sgn}
    \DeclareMathOperator{\diag}{diag\;}

    \colorlet{Corr}{red}

    % \everymath{\displaystyle} % biggify limits of inline sums and integrals
    \tikzstyle{circ} % usage: \node[circ, placement] (label) {text};
        = [draw, circle, fill=white, node distance=3cm, minimum height=2em]
    \definecolor{commentgreen}{rgb}{0,0.6,0}
    \lstset{
        basicstyle=\ttfamily\footnotesize,
        frame=single,
        numbers=left,
        showstringspaces=false,
        keywordstyle=\color{blue},
        stringstyle=\color{purple},
        commentstyle=\color{commentgreen},
        morecomment=[l][\color{magenta}]{\#}
    }

\begin{document}

\section{Canonical Parameters}

We use the canonical values: $M_2 = 1.4M_{\odot}$, $q = 6.3$, $e = 0.808$,
orbital period $P = 51.17\;\mathrm{days}$, and $\dot{P} = -3.03\times 10^{-7}$,
and an inferred $8.8M_{\odot}$ for the massive star (giving $a = 126R_{\odot}$).
The massive star is assumed to have core mass $3M_{\odot}$ and radius
$1.38R_{\odot}$. The form of energy dissipation can be written
\begin{equation}
    \dot{E}_{\rm in} = \hat{\tau} \Omega g\p{e, \Omega_{\rm s} / \Omega},
\end{equation}
where $\hat{\tau}$ is the torque assuming a circular orbit, $\Omega_{\rm s}$ is
the spin of the star, $\Omega$ is the orbital angular frequency of the orbit,
and $e$ is the eccentricity.

Then, since
\begin{equation}
    \dot{E}_{\rm g} = \frac{GqM_2^2}{3a}\frac{\dot{P}}{P},
\end{equation}
we can calculate $\dot{P}$ as a function of $\Omega_{\rm s}$. Figure~\ref{fig:1}
shows the result of this calculation and compares it to the observed $\dot{P}$.
\begin{figure}[h]
    \centering
    \includegraphics[width=0.7\columnwidth]{../scripts/eccentric_tides/1_7319_disps.png}
    \caption{$\dot{P}$ as a function of $\Omega_{\rm s}$ using the canonical
    parameters for PSR J0045-7319.}\label{fig:1}
\end{figure}

\section{MESA}

I present the results of my MESA simulations here. The observations for the star
that we must match are $L = 1.2 \times 10^4 L_{\odot}$ and $T = (2.4 \pm 0.1)
\times 10^4\;\mathrm{K}$. The three sets of simulations I present here are: (i)
low metallicity, non-rotating, (ii) rotating with large convective overshoot,
(iii) same as (ii) but with metallicity $Z = 0.004$. These are shown in
Fig.~\ref{fig:2}
\begin{figure}
    \centering
    \includegraphics[width=0.3\columnwidth]{../../../MESA/tutorials/4_J00457319/plots/lowm_nonrots_LT.png}
    \includegraphics[width=0.3\columnwidth]{../../../MESA/tutorials/4_J00457319/plots/lowm_nonrots_9_4_radii.png}
    \includegraphics[width=0.3\columnwidth]{../../../MESA/tutorials/4_J00457319/plots/lowm_nonrots_9_4_prop.png}

    \includegraphics[width=0.3\columnwidth]{../../../MESA/tutorials/4_J00457319/plots/highalpha_sims_LT.png}
    \includegraphics[width=0.3\columnwidth]{../../../MESA/tutorials/4_J00457319/plots/highalpha_sims_9_0_radii.png}
    \includegraphics[width=0.3\columnwidth]{../../../MESA/tutorials/4_J00457319/plots/highalpha_sims_9_0_prop.png}

    \includegraphics[width=0.3\columnwidth]{../../../MESA/tutorials/4_J00457319/plots/higha_004_sims_LT.png}
    \includegraphics[width=0.3\columnwidth]{../../../MESA/tutorials/4_J00457319/plots/higha_004_sims_9_8_radii.png}
    \includegraphics[width=0.3\columnwidth]{../../../MESA/tutorials/4_J00457319/plots/higha_004_sims_9_8_prop.png}
    \caption{The three sets of simulations I present here are shown in each row:
    (i) low metallicity, non-rotating, (ii) rotating with large convective
    overshoot, (iii) same as (ii) but with metallicity $Z = 0.004$. The first
    column shows the $L$-$T$ diagram, with which we try to match the observed
    parameters, the second column shows the evolution of some parameters of the
    best-fitting star as a function of age, and the third column shows the
    propagation diagrams at the time of best fit (dashed lines denote negative
    values). The best fitting masses are (i) $M = 9.6M_{\odot}$, (ii) $M =
    9.0M_{\odot}$, and (iii) $M = 9.8M_{\odot}$. }\label{fig:2}
\end{figure}

It is also worth noting that this implies Zahn's formula, given in the circular
case by
\begin{equation}
    \tau = \frac{3}{2}\frac{GM_2^2R^5}{a^6}E_2 \p{\frac{2\abs{\Omega -
        \Omega_{\rm s}}}{\Omega_{\rm s, c}}}^{8/3},
\end{equation}
(e.g.\ Kushnir et.\ al\. 2016) depends only on $R$ and not $r_{\rm c}$. Since
$E_2 = 1.592 \times 10^{-9}\p{M / M_{\odot}}^{2.84}$ (Hurley et.\ al.\ 2002;
Vigna-G\'omez et.\ al.\ 2020), we see that this dynamical tide formula predicts
that the torque increases as the star expands at constant mass. Our formula
predicts that the torque remains roughly constant.

\subsection{Constraint}

We note that $\Omega_{\rm s} \leq \Omega_{\rm s, c} \equiv \sqrt{GM_{\rm c} /
r_{\rm c}^3}$ the breakup frequency. This lets us put a constraint on the
maximum $\dot{P}$ obtainable from tides:
\begin{equation}
    \abs{\dot{P}} \leq
        -\frac{6\pi}{q}\beta_2 \left(\frac{r_{\rm c}}{a}\right)^5
        \frac{\rho_{\rm c}}{\bar{\rho}_{\rm c}} \left(1 -
        \frac{\rho_{\rm c}}{\bar{\rho}_{\rm c}}\right)^2 2^{8/3}\frac{f_2}{(1 -
        e^2)^6}.
\end{equation}
Here, $\beta_2 \approx 1$, $\rho_{\rm c}$ is the density at the RCB,
$\bar{\rho}_{\rm c}$ is the average density of the core, and $f_2 = 1 + 15e^2/2 +
45e^4/8 + 5e^6/16$.

If we require $-3.03 \times 10^{-7}$ to be within this range and take $\rho_{\rm
c}/ \bar{\rho}_{\rm c} \approx 0.74$, this requires $r_{\rm c} \gtrsim
1.19R_{\odot}$, which is indeed satisfied by the canonical parameters. However,
our simulations disagree with these parameters!

It is furthermore worth noting that $P$ is measured, not $a$; this implies that
$a$ is not a constant, but instead $a = \p{GM_2(1 + q) / \Omega^2}^{1/3}$, so
\begin{equation}
    \abs{\dot{P}} \leq -3.03 \times 10^{-7}\p{\frac{r_{\rm c}}{1.19R_{\odot}}}^5
        \p{\frac{qM_2}{8.8M_{\odot}}}^{-5/3}.
\end{equation}
In other words, if we use larger stellar masses (like our MESA models suggest),
the $\dot{P}$ constraint is even more strict!

\end{document}

