    \documentclass[11pt,
        usenames, % allows access to some tikz colors
        dvipsnames % more colors: https://en.wikibooks.org/wiki/LaTeX/Colors
    ]{article}
    \usepackage{
        amsmath,
        amssymb,
        fouriernc, % fourier font w/ new century book
        fancyhdr, % page styling
        lastpage, % footer fanciness
        hyperref, % various links
        setspace, % line spacing
        amsthm, % newtheorem and proof environment
        mathtools, % \Aboxed for boxing inside aligns, among others
        float, % Allow [H] figure env alignment
        enumerate, % Allow custom enumerate numbering
        graphicx, % allow includegraphics with more filetypes
        wasysym, % \smiley!
        upgreek, % \upmu for \mum macro
        listings, % writing TrueType fonts and including code prettily
        tikz, % drawing things
        booktabs, % \bottomrule instead of hline apparently
        xcolor, % colored text
        cancel % can cancel things out!
    }
    \usepackage[margin=1in]{geometry} % page geometry
    \usepackage[
        labelfont=bf, % caption names are labeled in bold
        font=scriptsize % smaller font for captions
    ]{caption}
    \usepackage[font=scriptsize]{subcaption} % subfigures

    \newcommand*{\scinot}[2]{#1\times10^{#2}}
    \newcommand*{\dotp}[2]{\left<#1\,\middle|\,#2\right>}
    \newcommand*{\rd}[2]{\frac{\mathrm{d}#1}{\mathrm{d}#2}}
    \newcommand*{\pd}[2]{\frac{\partial#1}{\partial#2}}
    \newcommand*{\rdil}[2]{\mathrm{d}#1 / \mathrm{d}#2}
    \newcommand*{\pdil}[2]{\partial#1 / \partial#2}
    \newcommand*{\rtd}[2]{\frac{\mathrm{d}^2#1}{\mathrm{d}#2^2}}
    \newcommand*{\ptd}[2]{\frac{\partial^2 #1}{\partial#2^2}}
    \newcommand*{\md}[2]{\frac{\mathrm{D}#1}{\mathrm{D}#2}}
    \newcommand*{\pvec}[1]{\vec{#1}^{\,\prime}}
    \newcommand*{\svec}[1]{\vec{#1}\;\!}
    \newcommand*{\bm}[1]{\boldsymbol{\mathbf{#1}}}
    \newcommand*{\uv}[1]{\hat{\bm{#1}}}
    \newcommand*{\ang}[0]{\;\text{\AA}}
    \newcommand*{\mum}[0]{\;\upmu \mathrm{m}}
    \newcommand*{\at}[1]{\left.#1\right|}
    \newcommand*{\bra}[1]{\left<#1\right|}
    \newcommand*{\ket}[1]{\left|#1\right>}
    \newcommand*{\abs}[1]{\left|#1\right|}
    \newcommand*{\ev}[1]{\langle#1\rangle}
    \newcommand*{\p}[1]{\left(#1\right)}
    \newcommand*{\s}[1]{\left[#1\right]}
    \newcommand*{\z}[1]{\left\{#1\right\}}

    \newtheorem{theorem}{Theorem}[section]

    \let\Re\undefined
    \let\Im\undefined
    \DeclareMathOperator{\Res}{Res}
    \DeclareMathOperator{\Re}{Re}
    \DeclareMathOperator{\Im}{Im}
    \DeclareMathOperator{\Log}{Log}
    \DeclareMathOperator{\Arg}{Arg}
    \DeclareMathOperator{\Tr}{Tr}
    \DeclareMathOperator{\E}{E}
    \DeclareMathOperator{\Var}{Var}
    \DeclareMathOperator*{\argmin}{argmin}
    \DeclareMathOperator*{\argmax}{argmax}
    \DeclareMathOperator{\sgn}{sgn}
    \DeclareMathOperator{\diag}{diag\;}

    \colorlet{Corr}{red}

    % \everymath{\displaystyle} % biggify limits of inline sums and integrals
    \tikzstyle{circ} % usage: \node[circ, placement] (label) {text};
        = [draw, circle, fill=white, node distance=3cm, minimum height=2em]
    \definecolor{commentgreen}{rgb}{0,0.6,0}
    \lstset{
        basicstyle=\ttfamily\footnotesize,
        frame=single,
        numbers=left,
        showstringspaces=false,
        keywordstyle=\color{blue},
        stringstyle=\color{purple},
        commentstyle=\color{commentgreen},
        morecomment=[l][\color{magenta}]{\#}
    }

\begin{document}

\onehalfspacing

\pagestyle{fancy}
\rfoot{Yubo Su}
\rhead{}
\cfoot{\thepage/\pageref{LastPage}}

I adapted Bin's equations from his Mathematica notebook for use in Python.

\textbf{Executive Summary:} For sufficiently wide initial binaries,
octupole-induced mergers occur due to the extremely high eccentricities
encountered during orbital flips (where $I$ changes from $> 90^\circ$ to
$<90^\circ$ or vice versa). A clean, though incomplete, analytical criterion can
be derived by considering oscillations in the generalized Kozai constant
\begin{equation}
    K = j_{\rm in}\cos I - \eta\frac{e_{\rm in}^2}{2},
        \label{eq:K}
\end{equation}
which is constant to quadrupolar order but not to octupolar order.

The resulting merger fraction plot that we obtain is shown in
Fig.~\ref{fig:merges}.
\begin{figure}
    \centering
    \includegraphics[width=0.6\columnwidth]{../scripts/octlk/1sweepbin/total_merger_fracs.png}
    \caption{Total merger fractions}\label{fig:merges}
\end{figure}

\section{Results of Simulations}

For each of the $6$ $q$ values I explored last week, I again show the merger
time distribution. The fiducial values used were:
\begin{align*}
    m_{12} &= 50M_{\odot}, &
    m_3 &= 30 M_{\odot}, &
    a_{1,0} &= 100\;\mathrm{AU}, \\
    a_2 &= 4500\;\mathrm{AU},&
    e_{1, 0} &= 10^{-3}, &
    e_{2, 0} &= 0.6.
\end{align*}
$q = m_1 / m_2$ and $I_0$, the mutual inclination between $\bm{L}_{\rm in}$ and
$\bm{L}_{\rm out}$ were varied, and $\omega_{\rm in}$, $\omega_{\rm out}$, and
$\Omega$ were randomly chosen. The new plots are shown in
Fig.~\ref{fig:mergers} and~\ref{fig:mergers2}.

\begin{figure}
    \centering
    \includegraphics[width=0.45\columnwidth]{../scripts/octlk/1sweepbin/1p2dist.png}
    \includegraphics[width=0.45\columnwidth]{../scripts/octlk/1sweepbin/1p3dist.png}
    \includegraphics[width=0.45\columnwidth]{../scripts/octlk/1sweepbin/1p4dist.png}
    \caption{Merger probabilities (top) and time distributions as a function of
    $I_0$. In each plot, $200$ different $I_0$ are considered, and $20$
    different realisations with random $\omega, \Omega$ are used. Note that
    there is actually not much probabilistic behavior, as there are distinct
    regions where mergers occur and do not occur. There is also curious
    resemblance among the plots, which is consistent with our further
    analysis.}\label{fig:mergers}
\end{figure}
\begin{figure}
    \centering
    \includegraphics[width=0.45\columnwidth]{../scripts/octlk/1sweepbin/1p5dist.png}
    \includegraphics[width=0.45\columnwidth]{../scripts/octlk/1sweepbin/1p7dist.png}
    \includegraphics[width=0.45\columnwidth]{../scripts/octlk/1sweepbin/1equaldist.png}
    \caption{Continuation of Fig.~\ref{fig:mergers}}\label{fig:mergers2}
\end{figure}

\section{Exploration}

There is clearly a lot of similar structure seen in Fig.~\ref{fig:mergers}. In
this section, we explore the origin of this behavior.

\subsection{Exploratory Plotting: $e_{\max}$ Distribution without GW}

We first consider the distribution of eccentricity maxima for the different $q$
when GW radiation is turned off. We pick a few representative $I_0$ for this
experiment: each $I_0$ has one fewer $q$ for which ELK-induced mergers are
possible (see Fig.~\ref{fig:mergers}).For each $q, I_0$, we pick $100$ random
different $\omega, \Omega$ and evolve to $3\;\mathrm{Gyr}$ without GW\@. We then
identify all eccentricity maxima. The distribution of these $e_{\max}$ is
given below in Figs.~\ref{fig:hists} and~\ref{fig:hists2}.
\begin{figure}
    \centering
    \includegraphics[width=0.55\columnwidth]{../scripts/octlk/1emax_q/q_sweep93hist.png}
    \includegraphics[width=0.55\columnwidth]{../scripts/octlk/1emax_q/q_sweep_935hist.png}
    \includegraphics[width=0.55\columnwidth]{../scripts/octlk/1emax_q/q_sweep_95hist.png}
    \caption{Histogram of eccentricity maxima under ELK, no GW\@. In each panel,
    the vertical blue line is the quadrupole $e_{\max}$, while the vertical
    black line is $e_{\lim}$. For each successive inclination, one fewer $q$
    value from Fig.~\ref{fig:mergers} exhibits mergers at that $q$ (i.e.\ for $I
    = 93^\circ$, all $q$ merge, while for $I = 95^\circ$, all but $2$ values of
    $q$ merge).}\label{fig:hists}
\end{figure}
\begin{figure}
    \centering
    \includegraphics[width=0.55\columnwidth]{../scripts/octlk/1emax_q/q_sweep_962hist.png}
    \includegraphics[width=0.55\columnwidth]{../scripts/octlk/1emax_q/q_sweep_97hist.png}
    \includegraphics[width=0.55\columnwidth]{../scripts/octlk/1emax_q/q_sweep_99hist.png}
    \caption{Histogram of eccentricity maxima under ELK, no GW,
    continued.}\label{fig:hists2}
\end{figure}

These plots show something very important: \emph{once the ELK effect is
strong beyond some threshold, the $e_{\max}$ distributions are insensitive to
even stronger $\epsilon_{\rm oct}$. Furthermore, this characteristic difference
in distributions corresponds exactly to the threshold required to induce merger
via the ELK effect.}

For completeness, we also provide the delay time distributions of these
$e_{\max}$ in Figs.~\ref{fig:delays} and~\ref{fig:delays2}, though it is not
essential to this story (see end).

\subsection{Origin of Dichotomy: Inclination Flipping}

In the previous section, we found a dichotomy of $e_{\max}$ distributions,
depending on $q, I$. We show two representative simulations over
$3\;\mathrm{Gyr}$ from the above, from either part of the dichotomy, in
Fig.~\ref{fig:fiducial}.
\begin{figure}
    \centering
    \includegraphics[width=0.7\columnwidth]{../scripts/octlk/1nogw_vec.png}
    \includegraphics[width=0.7\columnwidth]{../scripts/octlk/1nogw_vec95.png}
    \caption{Two fiducial simulations showing the origin of the $e_{\max}$
    dichotomy. Here, $q = 2/3$. In the fourth panels of both simulations, the
    horizontal black dotted line gives $K = -\eta_0/2$.}\label{fig:fiducial}
\end{figure}

It is clear that the $I_0 = 93.5^\circ$ exhibits orbit flipping, and it is
during these orbit flips that the eccentricity reaches large values (nearing
$e_{\lim}$).

This can also be seen in histograms of $I\p{e_{\rm in} = e_{\max}}$, shown in
Figs.~\ref{fig:histinc} and~\ref{fig:histinc2}.
\begin{figure}
    \centering
    \includegraphics[width=0.55\columnwidth]{../scripts/octlk/1emax_q/q_sweep93histinc.png}
    \includegraphics[width=0.55\columnwidth]{../scripts/octlk/1emax_q/q_sweep_935histinc.png}
    \includegraphics[width=0.55\columnwidth]{../scripts/octlk/1emax_q/q_sweep_95histinc.png}
    \caption{Same as Fig.~\ref{fig:hists} but for inclinations.}\label{fig:histinc}
\end{figure}
\begin{figure}
    \centering
    \includegraphics[width=0.55\columnwidth]{../scripts/octlk/1emax_q/q_sweep_962histinc.png}
    \includegraphics[width=0.55\columnwidth]{../scripts/octlk/1emax_q/q_sweep_97histinc.png}
    \includegraphics[width=0.55\columnwidth]{../scripts/octlk/1emax_q/q_sweep_99histinc.png}
    \caption{Same as Fig.~\ref{fig:hists2} but for inclinations.}\label{fig:histinc2}
\end{figure}

\subsection{Tentative Analytical Explanation}

As the evolution in Fig.~\ref{fig:fiducial} is not very chaotic (quasiperiodic
only?), we can attempt to to understand when orbit flips occur. In
Fig.~\ref{fig:fiducial}, we ploted in the bottom right panels the generalized
Kozai constant given by Eq.~\eqref{eq:K}. $K$ gives us information on when orbit
flips occur:
\begin{align}
    K &= j_{\rm in}\cos I - \eta \frac{e_{\rm in}^2}{2},\\
    \sgn\p{\cos I} &= \sgn\p{K - \eta \frac{e_{\rm in}^2}{2}},\\
    \at{\sgn\p{\cos I}}_{e_{\rm in} = e_{\max}}
        &\approx \sgn\p{K - \frac{\eta}{2}},\nonumber\\
        &\approx \sgn\p{K - \frac{\eta(e_{\rm out} = e_{\rm out, 0})}{2}}
        \equiv \sgn\p{K - \frac{\eta_0}{2}}
\end{align}
In the last line, we take advantage of the fact that $e_{\rm out}$ has only
small oscillations. Thus, orbit flips occur whenever $K - \eta_0/2$ changes
signs. This is thus a sufficient and necessary condition for ELK-induced
mergers.

\textbf{Update:} It seems like in the test mass limit ($m_2 = 0$), there is a
simple closed form for the critical $\epsilon_{\rm oct}$ required for flips, see
Katz et.\ al.\ 2011. The form obtained for their $C(\Omega_e)$ behaves quite
similarly to what we observe. Also in the test particle limit, Antognini 2015
shows that the timescale of this $K$ oscillation scales as $\propto
\epsilon_{\rm oct}^{-1/2}$. This probably agrees with the Katz result? As a
result, there's probably a complex but closed form generalization for
$\epsilon_{\rm oct, c}(I_0)$ that permits flips, and this sets the
octupole-induced merger window.

\begin{figure}
    \centering
    \includegraphics[width=0.55\columnwidth]{../scripts/octlk/1emax_q/q_sweep93delays.png}
    \includegraphics[width=0.55\columnwidth]{../scripts/octlk/1emax_q/q_sweep_935delays.png}
    \includegraphics[width=0.55\columnwidth]{../scripts/octlk/1emax_q/q_sweep_95delays.png}
    \caption{Delay time distribution for eccentricity maxima. Vertical blue line
    is quadrupolar $e_{\max}$ and vertical black line is $e_{\lim}$.}\label{fig:delays}
\end{figure}
\begin{figure}
    \centering
    \includegraphics[width=0.55\columnwidth]{../scripts/octlk/1emax_q/q_sweep_962delays.png}
    \includegraphics[width=0.55\columnwidth]{../scripts/octlk/1emax_q/q_sweep_97delays.png}
    \includegraphics[width=0.55\columnwidth]{../scripts/octlk/1emax_q/q_sweep_99delays.png}
    \caption{Delay time distribution for eccentricity maxima, pt 2.}\label{fig:delays2}
\end{figure}

\end{document}

