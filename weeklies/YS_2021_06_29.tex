    \documentclass[11pt,
        usenames, % allows access to some tikz colors
        dvipsnames % more colors: https://en.wikibooks.org/wiki/LaTeX/Colors
    ]{article}
    \usepackage{
        amsmath,
        amssymb,
        fouriernc, % fourier font w/ new century book
        fancyhdr, % page styling
        lastpage, % footer fanciness
        hyperref, % various links
        setspace, % line spacing
        amsthm, % newtheorem and proof environment
        mathtools, % \Aboxed for boxing inside aligns, among others
        float, % Allow [H] figure env alignment
        enumerate, % Allow custom enumerate numbering
        graphicx, % allow includegraphics with more filetypes
        wasysym, % \smiley!
        upgreek, % \upmu for \mum macro
        listings, % writing TrueType fonts and including code prettily
        tikz, % drawing things
        booktabs, % \bottomrule instead of hline apparently
        xcolor, % colored text
        cancel % can cancel things out!
    }
    \usepackage[margin=1in]{geometry} % page geometry
    \usepackage[
        labelfont=bf, % caption names are labeled in bold
        font=scriptsize % smaller font for captions
    ]{caption}
    \usepackage[font=scriptsize]{subcaption} % subfigures

    \newcommand*{\scinot}[2]{#1\times10^{#2}}
    \newcommand*{\dotp}[2]{\left<#1\,\middle|\,#2\right>}
    \newcommand*{\rd}[2]{\frac{\mathrm{d}#1}{\mathrm{d}#2}}
    \newcommand*{\pd}[2]{\frac{\partial#1}{\partial#2}}
    \newcommand*{\rdil}[2]{\mathrm{d}#1 / \mathrm{d}#2}
    \newcommand*{\pdil}[2]{\partial#1 / \partial#2}
    \newcommand*{\rtd}[2]{\frac{\mathrm{d}^2#1}{\mathrm{d}#2^2}}
    \newcommand*{\ptd}[2]{\frac{\partial^2 #1}{\partial#2^2}}
    \newcommand*{\md}[2]{\frac{\mathrm{D}#1}{\mathrm{D}#2}}
    \newcommand*{\pvec}[1]{\vec{#1}^{\,\prime}}
    \newcommand*{\svec}[1]{\vec{#1}\;\!}
    \newcommand*{\bm}[1]{\boldsymbol{\mathbf{#1}}}
    \newcommand*{\uv}[1]{\hat{\bm{#1}}}
    \newcommand*{\ang}[0]{\;\text{\AA}}
    \newcommand*{\mum}[0]{\;\upmu \mathrm{m}}
    \newcommand*{\at}[1]{\left.#1\right|}
    \newcommand*{\bra}[1]{\left<#1\right|}
    \newcommand*{\ket}[1]{\left|#1\right>}
    \newcommand*{\abs}[1]{\left|#1\right|}
    \newcommand*{\ev}[1]{\left\langle#1\right\rangle}
    \newcommand*{\p}[1]{\left(#1\right)}
    \newcommand*{\s}[1]{\left[#1\right]}
    \newcommand*{\z}[1]{\left\{#1\right\}}

    \newtheorem{theorem}{Theorem}[section]

    \let\Re\undefined
    \let\Im\undefined
    \DeclareMathOperator{\Res}{Res}
    \DeclareMathOperator{\Re}{Re}
    \DeclareMathOperator{\Im}{Im}
    \DeclareMathOperator{\Log}{Log}
    \DeclareMathOperator{\Arg}{Arg}
    \DeclareMathOperator{\Tr}{Tr}
    \DeclareMathOperator{\E}{E}
    \DeclareMathOperator{\Var}{Var}
    \DeclareMathOperator*{\argmin}{argmin}
    \DeclareMathOperator*{\argmax}{argmax}
    \DeclareMathOperator{\sgn}{sgn}
    \DeclareMathOperator{\diag}{diag\;}

    \colorlet{Corr}{red}

    % \everymath{\displaystyle} % biggify limits of inline sums and integrals
    \tikzstyle{circ} % usage: \node[circ, placement] (label) {text};
        = [draw, circle, fill=white, node distance=3cm, minimum height=2em]
    \definecolor{commentgreen}{rgb}{0,0.6,0}
    \lstset{
        basicstyle=\ttfamily\footnotesize,
        frame=single,
        numbers=left,
        showstringspaces=false,
        keywordstyle=\color{blue},
        stringstyle=\color{purple},
        commentstyle=\color{commentgreen},
        morecomment=[l][\color{magenta}]{\#}
    }

\begin{document}

\section{3-planet CS}

I have a few new plots since the group meeting presentation Friday. I primarily
just added a few other choices of resonant angles and changed the coordinate
system to have $\uv{l} \propto \uv{z}$. I focused on the case where $\alpha = 10
g_1$, $g_1 = 10g_2$, $I_1 = 10^\circ$, and $I_2 = 1^\circ$, for which the
outcomes and histogram are given in Fig.~\ref{fig:outcomes}. This results in
Fig.~\ref{fig:mm_tide}. I also have a few sample plots like this for different
$g_2$ values and different $I_2$ values.
\begin{figure}
    \centering
    \includegraphics[width=0.6\columnwidth]{../../attractors/initial/4nplanet/3paramtide/outcomes010.png}
    \includegraphics[width=0.6\columnwidth]{../../attractors/initial/4nplanet/3paramtide/outcomes010_hist.png}
    \caption{Outcomes and histogram.}\label{fig:outcomes}
\end{figure}
\begin{figure}
    \centering
    \includegraphics[width=0.77\columnwidth]{../../attractors/initial/4nplanet/3paramtide/mm_tide_mode0.png}

    \includegraphics[width=0.77\columnwidth]{../../attractors/initial/4nplanet/3paramtide/mm_tide_mode1.png}

    \includegraphics[width=0.77\columnwidth]{../../attractors/initial/4nplanet/3paramtide/mm_tide_mode2.png}
    \caption{Plot of ICs and final cycles for a variety of initial
    conditions. The histogram in Fig.~\ref{fig:outcomes} suggests that all ICs
    go to CS2 of mode 2, rather than CS1 of mode 1, and this is confirmed by
    these trajectories.}\label{fig:mm_tide}
\end{figure}

Regarding analytical work, I tried to do some work in the frame where $\uv{J}
\propto \uv{z}$, i.e.\
\begin{align}
    \p{\rd{\uv{s}}{t}}_{\rm rot}
        &= \alpha\p{\uv{s} \cdot \uv{l}}\p{\uv{s} \times \uv{l}}
            - \bar{g}\p{\uv{s} \times \uv{z}}\label{eq:eom_rot},\\
    \uv{l}(t) &= \begin{bmatrix}
        \p{I_1 + I_2}\cos \p{\frac{\Delta gt - \phi_0}{2}}\\
        \p{I_1 - I_2}\sin \p{\frac{\Delta gt - \phi_0}{2}}\\
        1
    \end{bmatrix} + \mathcal{O}\s{\p{I_1 + I_2}^2}.
\end{align}
but it's quite hard to enforce $\p{\uv{s} \cdot \uv{l}}$ to be fixed. I will
think about what you proposed as well\dots

% \subsection{Analytical Attempt}

% Towards analytical work, I actually began my analysis in a simpler co-rotating
% frame with frequency $\bar{g}$. Recall that we showed that, in the co-rotating
% frame with frequency $\bar{g} = \p{g_1 + g_2} / 2$:
% \begin{align}
%     \p{\rd{\uv{s}}{t}}_{\rm rot}
%         &= \alpha\p{\uv{s} \cdot \uv{l}}\p{\uv{s} \times \uv{l}}
%             - \bar{g}\p{\uv{s} \times \uv{z}}\label{eq:eom_rot},\\
%     \uv{l}(t) &= \begin{bmatrix}
%         \p{I_1 + I_2}\cos \p{\frac{\Delta gt - \phi_0}{2}}\\
%         \p{I_1 - I_2}\sin \p{\frac{\Delta gt - \phi_0}{2}}\\
%         1
%     \end{bmatrix} + \mathcal{O}\s{\p{I_1 + I_2}^2}.
% \end{align}
% We will first proceed for some arbitrary $\bar{g}$, then we will heuristically
% try to see whether $\bar{g}$ can be pinned down.

% We first examine the $\uv{z}$ component of the equation of motion for $\uv{s}$ ,
% suppressing the ``rot'' subscript. Denote $\bm{s}_{\perp}$ and $\bm{l}_{\perp}$
% to be the $\uv{x}$-$\uv{y}$ plane components of the two vectors, then we obtain
% \begin{equation}
%     \rd{s_z}{t} = \alpha\p{\uv{s} \cdot \uv{l}}\p{\uv{s}_{\perp} \times
%             \uv{l}_{\perp}}.
% \end{equation}
% An easy way for this to vanish is if
% \begin{equation}
%     \uv{s}_\perp \parallel \uv{l}_\perp\label{eq:parallel_cons}.
% \end{equation}

% We next examine the in-plane component of the equation of motion:
% \begin{align}
%     \rd{\bm{s}_\perp}{t}
%         &= \alpha\p{\uv{s} \cdot \uv{l}}
%                 \p{\bm{s}_\perp \times \uv{z} + s_z\uv{z} \times \bm{l}_\perp}
%             - \bar{g}\p{\bm{s}_\perp \times \uv{z}}.
% \end{align}
% Since $\bm{s}_\perp \parallel \bm{l}_\perp$, we can express everything in terms
% of $\bm{l}_\perp$ and the two magnitudes $s_\perp$ and $l_\perp$. This gives the
% following manipulation:
% \begin{align}
%     \rd{\bm{s}_\perp}{t}
%         &= \alpha\p{\uv{s} \cdot \uv{l}}
%                 \p{\bm{s}_\perp \times \uv{z} - s_z\frac{l_\perp}{s_\perp}
%                     \bm{s}_\perp \times \uv{z}}
%             - \bar{g}\p{\bm{s}_\perp \times \uv{z}},\\
%         &= \s{\alpha\p{\uv{s} \cdot \uv{l}}
%             \p{1 - s_z\frac{l_\perp}{s_\perp}} - \bar{g}}
%             \p{\bm{s}_\perp \times \uv{z}},\\
%         &= \z{\s{\alpha s_z - \bar{g}}
%             + \alpha \s{s_\perp l_\perp - s_z^2\frac{l_\perp}{s_\perp}}
%             - s_z l_\perp^2}
%             \p{\bm{s}_\perp \times \uv{z}}\label{eq:perp_const}.
% \end{align}
% I've grouped the terms in order of $\mathcal{O}\p{l_\perp}$, since $l_\perp \ll
% 1$. If an exact equilibrium exists, the expression in the curly brackets must
% vanish, as well as Eq.~\eqref{eq:parallel_cons} be satisfied exactly. In
% particular, if $l_\perp \ll 1$, we recover the relation I claimed in the writeup
% earlier:
% \begin{equation}
%     s_z \approx \frac{\bar{g}}{\alpha} + \mathcal{O}\p{l_\perp}.
% \end{equation}
% However, since $l_\perp$ is in general \emph{not} constant in time (ranging from
% $I_1 + I_2$ to $I_1 - I_2$), an exact equilibrium does not exist: $l_\perp(t)$
% has a $\Delta g / 2$ harmonic, which means $s_\perp(t)$ will also have a
% $\Delta g / 2$ harmonic.

% So far, we have not assumed anything special about $\bar{g}$. But we know from
% our phase portraits that the equilibria occur when $\uv{s} \cdot \uv{l}$ is
% constant.

\section{BH Spin Orbit Attractor}

I did some more investigation, and I realized that I wasn't computing the angle
correctly: the ``attractor'' is really defined relative to the changing $\uv{l}$
vector, and I forgot that $\uv{l}$ also precesses. When computing the updated
angle, using both the old (wrong) and new (Racine 2008) equations, I reproduce
the following behavior:
\begin{itemize}
    \item When $\theta_{s_1l} = 10^\circ$, $\bm{s}_1 \cdot \bm{s}_2 \to 1$,
        i.e.\ the spins align. This causes $\Delta \Phi = 0$. When
        $\theta_{s_2l} = 170^\circ$, $\bm{s}_1 \cdot \bm{s}_2 \to -1$, causing
        the spins to anti-align generally.

    \item When the masses are equal, $\bm{s}_1 \cdot \bm{s}_2$ is
        \emph{constant}. So I can't do any equal-mass analysis, unfortunately.
\end{itemize}
I show this in
\begin{figure}
    \centering
    \includegraphics[width=0.7\columnwidth]{../scripts/bh_sporb/2half_55_deltaphi.png}
    \includegraphics[width=0.7\columnwidth]{../scripts/bh_sporb/2half_55_retro_deltaphi.png}
    \caption{Initial (blue) and final (red) values of $\theta_{\rm 12} \equiv
    \arccos\p{\uv{s}_1 \cdot \uv{s}_2}$ and $\Delta \Phi$ (relative to
    $\uv{l}$) for the prograde and retrograde $\uv{s}_1$ case. Here, $m_1 =
    0.55M_{\odot}$ and $m_2 = 0.45M_{\odot}$.}\label{fig:2half55}
\end{figure}

This dot product obeys:
\begin{align}
    \rd{}{\psi}\p{\uv{s}_1 \cdot \uv{s}_2}
        &= \underbrace{\frac{3}{7 - 3\lambda/2}\s{
            \frac{m_2}{m_1} - \frac{m_1}{m_2}
            - \lambda\mu\p{\frac{1}{m_1} - \frac{1}{m_2}}}}_{\beta_-}
            \uv{J} \cdot \p{\uv{s}_1 \times \uv{s}_2}.
\end{align}
Note that this vanishes when the masses are equal, and indeed, numerically I can
verify that the $\Delta \Phi$ equilibria disappear! At first glance, this should
average out over a precession cycle, so the origin of this attracting behavior
is indeed somewhat mysterious. However, the signature and strength of the
attractor, the different phenomenology in the retrograde/prograde cases, are
both interesting.

\end{document}

