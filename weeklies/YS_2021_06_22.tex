    \documentclass[11pt,
        usenames, % allows access to some tikz colors
        dvipsnames % more colors: https://en.wikibooks.org/wiki/LaTeX/Colors
    ]{article}
    \usepackage{
        amsmath,
        amssymb,
        fancyhdr, % page styling
        lastpage, % footer fanciness
        hyperref, % various links
        setspace, % line spacing
        amsthm, % newtheorem and proof environment
        mathtools, % \Aboxed for boxing inside aligns, among others
        float, % Allow [H] figure env alignment
        enumerate, % Allow custom enumerate numbering
        graphicx, % allow includegraphics with more filetypes
        wasysym, % \smiley!
        upgreek, % \upmu for \mum macro
        listings, % writing TrueType fonts and including code prettily
        tikz, % drawing things
        booktabs, % \bottomrule instead of hline apparently
        xcolor, % colored text
        cancel % can cancel things out!
    }
    \usepackage[margin=1in]{geometry} % page geometry
    \usepackage[
        labelfont=bf, % caption names are labeled in bold
        font=scriptsize % smaller font for captions
    ]{caption}
    \usepackage[font=scriptsize]{subcaption} % subfigures

    \newcommand*{\scinot}[2]{#1\times10^{#2}}
    \newcommand*{\dotp}[2]{\left<#1\,\middle|\,#2\right>}
    \newcommand*{\rd}[2]{\frac{\mathrm{d}#1}{\mathrm{d}#2}}
    \newcommand*{\pd}[2]{\frac{\partial#1}{\partial#2}}
    \newcommand*{\rdil}[2]{\mathrm{d}#1 / \mathrm{d}#2}
    \newcommand*{\pdil}[2]{\partial#1 / \partial#2}
    \newcommand*{\rtd}[2]{\frac{\mathrm{d}^2#1}{\mathrm{d}#2^2}}
    \newcommand*{\ptd}[2]{\frac{\partial^2 #1}{\partial#2^2}}
    \newcommand*{\md}[2]{\frac{\mathrm{D}#1}{\mathrm{D}#2}}
    \newcommand*{\pvec}[1]{\vec{#1}^{\,\prime}}
    \newcommand*{\svec}[1]{\vec{#1}\;\!}
    \newcommand*{\bm}[1]{\boldsymbol{\mathbf{#1}}}
    \newcommand*{\uv}[1]{\hat{\bm{#1}}}
    \newcommand*{\ang}[0]{\;\text{\AA}}
    \newcommand*{\mum}[0]{\;\upmu \mathrm{m}}
    \newcommand*{\at}[1]{\left.#1\right|}
    \newcommand*{\bra}[1]{\left<#1\right|}
    \newcommand*{\ket}[1]{\left|#1\right>}
    \newcommand*{\abs}[1]{\left|#1\right|}
    \newcommand*{\ev}[1]{\left\langle#1\right\rangle}
    \newcommand*{\p}[1]{\left(#1\right)}
    \newcommand*{\s}[1]{\left[#1\right]}
    \newcommand*{\z}[1]{\left\{#1\right\}}

    \newtheorem{theorem}{Theorem}[section]

    \let\Re\undefined
    \let\Im\undefined
    \DeclareMathOperator{\Res}{Res}
    \DeclareMathOperator{\Re}{Re}
    \DeclareMathOperator{\Im}{Im}
    \DeclareMathOperator{\Log}{Log}
    \DeclareMathOperator{\Arg}{Arg}
    \DeclareMathOperator{\Tr}{Tr}
    \DeclareMathOperator{\E}{E}
    \DeclareMathOperator{\Var}{Var}
    \DeclareMathOperator*{\argmin}{argmin}
    \DeclareMathOperator*{\argmax}{argmax}
    \DeclareMathOperator{\sgn}{sgn}
    \DeclareMathOperator{\diag}{diag\;}

    \colorlet{Corr}{red}

    % \everymath{\displaystyle} % biggify limits of inline sums and integrals
    \tikzstyle{circ} % usage: \node[circ, placement] (label) {text};
        = [draw, circle, fill=white, node distance=3cm, minimum height=2em]
    \definecolor{commentgreen}{rgb}{0,0.6,0}
    \lstset{
        basicstyle=\ttfamily\footnotesize,
        frame=single,
        numbers=left,
        showstringspaces=false,
        keywordstyle=\color{blue},
        stringstyle=\color{purple},
        commentstyle=\color{commentgreen},
        morecomment=[l][\color{magenta}]{\#}
    }

\begin{document}

\section{2-Planet Cassini States}

I evolved the equations
\begin{align}
    \rd{\uv{s}}{t} &= \alpha\p{\uv{s} \cdot \uv{l}_1}\p{\uv{s} \times
        \uv{l}_1} + \epsilon_{\rm tide}\p{\uv{s}_i \times \p{\uv{l}_i \times
        \uv{s}_i} },\\
    \uv{l}_1 &= \begin{bmatrix}
            i_1 \cos (g_1t) + i_2 \cos\p{g_2 t + \phi_0}\\
            i_1 \sin (g_1t) + i_2 \sin\p{g_2 t + \phi_0}\\
            \cos\abs{\mathcal{I}_1}
        \end{bmatrix}.
\end{align}
I will refer to $i_2 \equiv i_{\rm 1f}$ the ``forced inclination'' from before
for convenient.

For $\epsilon_{\rm tide} = \scinot{2}{-3} g_1$ and a total integration time of
$t_{\rm f} = 5000 / g_1$, we can then calculate the final obliquity as a
function of the initial spin orientation $\uv{s}$. For simplicity, we take
$\alpha = 10g_1$. We do this for a variety of parameters below in
Figs.~\ref{fig:outcomes00}--\ref{fig:outcomes10}.
\begin{figure}
    \centering
    \includegraphics[width=0.45\columnwidth]{../../attractors/initial/4nplanet/3paramtide/outcomes00.png}
    \includegraphics[width=0.45\columnwidth]{../../attractors/initial/4nplanet/3paramtide/outcomes00_hist.png}
    \caption{Left: plot of the final obliquities (color coded) as a function of
    $\theta = \arccos\p{\uv{l}_1 \cdot \uv{s}}$ and the corresponding azimuthal
    angle $\phi$. The black [blue] dots denote CS2 for $g_1$ [$g_2$].. Here,
    $i_2 = 0$, so the behavior is exactly as expected for the single-planet
    CS\@. Right: histogram of final obliquities. The CS2 obliquity for both
    states is shown in the vertical dashed line, and the CS2 obliquity in the
    dash-dotted line.}\label{fig:outcomes00}
\end{figure}
\begin{figure}
    \centering
    \includegraphics[width=0.45\columnwidth]{../../attractors/initial/4nplanet/3paramtide/outcomes01.png}
    \includegraphics[width=0.45\columnwidth]{../../attractors/initial/4nplanet/3paramtide/outcomes01_hist.png}
    \caption{Same as Fig.~\ref{fig:outcomes00} but for $i_2 = 1^\circ$ and $g_2
    = 0.1g_1$. The separatricies are labeled.}\label{fig:outcomes01}
\end{figure}
\begin{figure}
    \centering
    \includegraphics[width=0.45\columnwidth]{../../attractors/initial/4nplanet/3paramtide/outcomes15.png}
    \includegraphics[width=0.45\columnwidth]{../../attractors/initial/4nplanet/3paramtide/outcomes15_hist.png}
    \caption{Same as Fig.~\ref{fig:outcomes00} but for $i_2 = 1^\circ$ and $g_2
    = 1.5g_1$.}\label{fig:outcomes15}
\end{figure}
\begin{figure}
    \centering
    \includegraphics[width=0.45\columnwidth]{../../attractors/initial/4nplanet/3paramtide/outcomes20.png}
    \includegraphics[width=0.45\columnwidth]{../../attractors/initial/4nplanet/3paramtide/outcomes20_hist.png}
    \caption{Same as Fig.~\ref{fig:outcomes00} but for $i_2 = 1^\circ$ and $g_2
    = 2g_1$.}\label{fig:outcomes20}
\end{figure}
\begin{figure}
    \centering
    \includegraphics[width=0.45\columnwidth]{../../attractors/initial/4nplanet/3paramtide/outcomes25.png}
    \includegraphics[width=0.45\columnwidth]{../../attractors/initial/4nplanet/3paramtide/outcomes25_hist.png}
    \caption{Same as Fig.~\ref{fig:outcomes00} but for $i_2 = 1^\circ$ and $g_2
    = 2.5g_1$.}\label{fig:outcomes25}
\end{figure}
\begin{figure}
    \centering
    \includegraphics[width=0.45\columnwidth]{../../attractors/initial/4nplanet/3paramtide/outcomes10.png}
    \includegraphics[width=0.45\columnwidth]{../../attractors/initial/4nplanet/3paramtide/outcomes10_hist.png}
    \caption{Same as Fig.~\ref{fig:outcomes00} but for $i_2 = 1^\circ$ and $g_2
    = 10g_1$. A distinct second stable state is visible at $\theta_{\rm f} = 60^
    \circ$ that corresponds to neither of the CSs.}\label{fig:outcomes10}
\end{figure}

The clear second equilibrium seen at $\theta_{\rm f} = 60^\circ$ is surprising,
since it doesn't correspond to either of the CS2s for either mode. In fact, the
resonant angle is $\phi = \p{g_1 + g_2}t$! Could this be a mixed mode?
Furthermore, it's clear that anything that makes it past this mixed mode is
trapped in the CS2 of $g_2$, rather than the CS1 of $g_1$.

\subsection{Analytical Theory of a new Equilibrium}

Recall that $\uv{l}(t)$ evolves as:
\begin{align}
    \uv{l}(t) &\approx \begin{bmatrix}
        I_1 \cos\p{g_1 t} + I_2 \cos\p{g_2 t + \phi_0}\\
        I_1 \sin\p{g_1 t} + I_2 \sin\p{g_2 t + \phi_0}\\
        1
    \end{bmatrix}.
\end{align}
Note that if we go to the co-rotating frame with frequency $\bar{g} \equiv
\p{g_1 + g_2}/2$, we can also choose the $\uv{y}$ direction such that the EOM
becomes:
\begin{align}
    \p{\rd{\uv{s}}{t}}_{\rm rot}
        &= \alpha\p{\uv{s} \cdot \uv{l}}\p{\uv{s} \times \uv{l}}
            - \bar{g}\p{\uv{s} \times \uv{z}},\\
    \uv{l}(t) &\approx \begin{bmatrix}
        I_1 \cos\p{(g_1 - \bar{g}) t
            - \phi_0 / 2} + I_2 \cos\p{\p{g_2 - \bar{g}} t + \phi_0 / 2}\\
        0\\
        1
    \end{bmatrix}\nonumber\\
        &= \underbrace{
            \p{I_1 + I_2} \cos \p{\frac{\p{g_2 - g_1}t - \phi_0}{2}}\uv{x}}
            _{\bm{l}_x} + \uv{z},\\
    \p{\rd{\uv{s}}{t}}_{\rm rot}
        &\approx \alpha \p{\uv{s} \cdot \uv{z}} \p{\uv{s} \times \uv{z}}
            - \bar{g}\p{\uv{s} \times \uv{z}}
            + \mathcal{O}\p{I_1 + I_2}.
\end{align}
An approximate equilibrium (neglecting the terms $\sim \mathcal{O}\p{I_1 +
I_2}$) must then satisfy
\begin{equation}
    \uv{s} \cdot \uv{z} = \frac{\bar{g}}{\alpha}.
\end{equation}
The $\uv{x}$ and $\uv{y}$ components will oscillate, but are determined by the
periodic motion at the $\sim \mathcal{O}\p{I_1 + I_2}$ order. The prediction of
the $z$ component of the spin is well predicted by the two dissipative
simulations shown in Fig.~\ref{fig:new}.
\begin{figure}
    \centering
    \includegraphics[width=0.45\columnwidth]{../../attractors/initial/4nplanet/3paramtide/disp_10_res.png}
    \includegraphics[width=0.45\columnwidth]{../../attractors/initial/4nplanet/3paramtide/disp_10_res2.png}
    \caption{Two cases for $g_2 = 10g_1$ and $\alpha = \z{10, 15}g_1$
    respectively, with weak tidal dissipation. The zeroth-order resonant
    obliquities would be $56.6^\circ$ and $68.5^\circ$ respectively, and the
    resonant angles are as predicted.}\label{fig:new}
\end{figure}

\section{BH Spin-Orbit}

\subsection{Co-rotating EOM\@: Equal Mass}

The conditions cited by Gerosa et.\ al\ 2013 are: (i) $0.4 \leq q < 1$, (ii)
$\xi_i \gtrsim 0.5$, and (iii) $\theta_{\rm s_1l} \neq \theta_{\rm s_2l}$.
However, I don't think these conditions are correct, and I don't see a citation
for these. I think that the ``attractor'' is in fact purely geometric. I focus
first on the case where the two BH have equal masses.

In this case, Racine 2008 shows that the sum of the spins $\bm{S} = \bm{s}_1 +
\bm{s}_2$ evolves following:
\begin{equation}
    \rd{\bm{S}}{\psi} = \uv{J} \times \bm{S},
\end{equation}
where $\psi$ is some dimensionless time-like variable
\begin{equation}
    \psi = \int\limits_0^t
        \frac{1}{2d^3}
            \s{7 - \frac{3}{2}\lambda}\abs{\bm{J}}\;\mathrm{d}t'.
\end{equation}
Then, in this case, the spins evolve following:
\begin{align}
    \rd{\bm{s}_1}{\psi} &= \p{\uv{J} - \alpha \bm{s}_2} \times \bm{s}_1,\\
    \rd{\bm{s}_2}{\psi} &= \p{\uv{J} - \alpha \bm{s}_1} \times \bm{s}_2.
\end{align}
It's easy to see that we can replace $\bm{s}_{2, 1}$ with $\bm{S}$ in the above
equations. Furthermore, we can go to the co-rotating frame where $\bm{S}$ is
fixed, so that:
\begin{align}
    \p{\rd{\bm{S}}{\psi}}_{\rm rot} &= 0,\\
    \p{\rd{\bm{s}_i}{\psi}}_{\rm rot} &= -\alpha \bm{S} \times \bm{s}_i,
\end{align}
where $i = 1, 2$. The physical picture is now clear: both spins $\bm{s}_i$
simply rotate around the fixed vector $\bm{S}$ in the co-rotating frame.

The two ``resonances'' are immediately obvious: when $\bm{s}_1 = \pm \bm{s}_2$,
then $\bm{S} \propto \bm{s}_i$, and there is no precession. These are the
equilibria of the system, and the two signs correspond to the cases where
$\Delta \Phi = 0, \pi$.

\subsection{``Attractor''}

Let's further specialize to the case where $\abs{\bm{s}_1} = \abs{\bm{s}_2}$,
and for simplicity, let's assume that initially $\abs{\bm{L}}_{\rm i} \gg
\abs{\bm{S}}_{\rm i}$, bu that at late times, $\abs{\bm{L}}_{\rm f} \ll
\abs{\bm{S}}_{\rm f}$ (we check how well this is satisfied later). Then:
\begin{itemize}
    \item Initially, $\bm{s}_i$ both precess independently about $\uv{L} \approx
        \uv{J}$, and $\Delta \Phi$ is approximately constant.

    \item At late times, $\bm{s}_i$ both precess \emph{directly opposed to each
        other} on opposite sides of $\bm{S}$.
\end{itemize}
If we consider the vectors as shown in Fig.~\ref{fig:fig}, then the physical
picture is immediately clear: If $\theta_{\rm s_2, S} = \theta_{\rm s_1, S} \leq
\theta_{\rm SL}$, then the two spins just precess around the total spin, and
both $\Phi$ angles are librating such that $-\pi \leq \Delta \Phi \leq \pi$. On
the other hand, if $\theta_{\rm s_1, S} \leq \theta_{\rm SL}$, then both angles
are circulating \emph{at the same rate}, so that $0 \leq \Delta \Phi \leq 2\pi$.
\begin{figure}
    \centering
    \begin{tikzpicture}
        \draw[->, ultra thick] (0, 0) -- (0, 5);
        \node[above] at (0, 5) {$\uv{L}$};
        \draw[->, ultra thick] (0, 0) -- (3, 4);
        \node[right] at (3, 4) {$\uv{S}$};
        \draw[->, ultra thick] (0, 0) -- (2.3, 1.8);
        \node[right] at (2, 2) {$\uv{s}_1$};
        \draw[->, ultra thick] (0, 0) -- (1, 2.6);
        \node[left] at (1, 2.7) {$\uv{s}_2$};
    \end{tikzpicture}
    \caption{Sample configuration of vectors (imagine that this is in the
    inertial frame, to facilitate consideration of $\Delta \phi$
    behavior).}\label{fig:fig}
\end{figure}

In the above, we assumed that $L_{\rm f} \ll S_{\rm f}$: is this valid? Well, at
merger, $a \simeq GM/c^2 = M$ (using GR units), so the total angular momentum $L
\simeq M$, while if both BH are maximally rotating and are aligned, $S \simeq
2M$. Thus, the assumption is indeed not well satisfied, but this suggests that
the ``strength'' of the attractor should improve if we artificially permit GW to
cause $L$ to decay to separations $\ll M$. This is because $\bm{J}$ is more
dominated by $\bm{S}$, and we are guaranteed that $\bm{s}_i$ are on opposite
sides of $\bm{S}$, constraining the range of $\Delta \Phi$ values better. We
show a few suites of simulations in Figs.~\ref{fig:2equal}--\ref{fig:2limit}.
\begin{figure}
    \centering
    \includegraphics[width=0.45\columnwidth]{../scripts/bh_sporb/2equal_dphis.png}
    \includegraphics[width=0.45\columnwidth]{../scripts/bh_sporb/2equal_qscat.png}
    \includegraphics[width=0.45\columnwidth]{../scripts/bh_sporb/2equal_qtotscat.png}
    % \includegraphics[width=0.45\columnwidth]{../scripts/bh_sporb/2equal_deltaphi.png}
    \caption{(i) Distribution of $\Delta \Phi$ initially ($a = 400 M$) and at
    coalescence ($a = M$). (ii) Distribution of final $\Delta \Phi$ as a
    function of $\theta_{\rm s_1L}$ and $\theta_{\rm s_2L}$, (iii) $\Delta
    \Phi_{\rm f}$ as a function of $\theta_{\rm SL, i}$, and (iv) the range of
    $\Delta \Phi$ over the last $1\%$ of points (not sure the detailed amount of
    time) of the simulation. Equal masses and equal spins, $\chi_i = 1$. An
    attractor can clearly be seen, but it can be seen to be a geometric
    effect (as can be seen from the $\Delta \Phi$ plots).}\label{fig:2equal}
\end{figure}
\begin{figure}
    \centering
    \includegraphics[width=0.45\columnwidth]{../scripts/bh_sporb/2half_dphis.png}
    \includegraphics[width=0.45\columnwidth]{../scripts/bh_sporb/2half_qscat.png}
    \includegraphics[width=0.45\columnwidth]{../scripts/bh_sporb/2half_qtotscat.png}
    % \includegraphics[width=0.45\columnwidth]{../scripts/bh_sporb/2half_deltaphi.png}
    \caption{Same as Fig.~\ref{fig:2equal} but with $M_1 = 2M_2$. The attractor
    is less visible.}\label{fig:2half}
\end{figure}
\begin{figure}
    \centering
    \includegraphics[width=0.45\columnwidth]{../scripts/bh_sporb/2lowspin_dphis.png}
    \includegraphics[width=0.45\columnwidth]{../scripts/bh_sporb/2lowspin_qscat.png}
    \includegraphics[width=0.45\columnwidth]{../scripts/bh_sporb/2lowspin_qtotscat.png}
    % \includegraphics[width=0.45\columnwidth]{../scripts/bh_sporb/2lowspin_deltaphi.png}
    \caption{Same as Fig.~\ref{fig:2equal} but with $\chi_i =
    0.1$. The attractor has mostly disappeared because $\uv{J}$ stays dominated
    by $\uv{L}$ throughout the inspiral.}\label{fig:2lowspin}
\end{figure}
\begin{figure}
    \centering
    \includegraphics[width=0.3\columnwidth]{../scripts/bh_sporb/2limit_dphis.png}
    \includegraphics[width=0.3\columnwidth]{../scripts/bh_sporb/2limit_qscat.png}
    \includegraphics[width=0.3\columnwidth]{../scripts/bh_sporb/2limit_qtotscat.png}
    \caption{Same as Fig.~\ref{fig:2equal} but allowing spindown to
    $0.01M$. The attractor is much more prominent because $S_{\rm f} \gg L_{\rm
    f}$ to much greater accuracy.}\label{fig:2limit}
\end{figure}

My hypothesis for the attractor behavior is very simple: if $\theta_{\rm SL}
\approx 0$, then $\bm{s}_1$ and $\bm{s}_2$ must be on opposite sides of
$\bm{L}$, since they are on opposite sides of $\bm{S}$. Their range can be much
larger when $\theta_{\rm SL}$ is substantial, but exactly $\Delta \Phi =
0^\circ$ is not very preferred since this would require $\bm{s}_1$ to be nearly
parallel to $\bm{s}_2$. Nevertheless, there can be a broad concentration near
$\Delta \Phi = 0^\circ$.

% We can quantify this algebraically in the equal-mass,
% equal-and-maximally-spinning case. Since $\theta_{\rm s_1S} = \theta_{\rm
% s_2S}$, we just need to ask: given two vectors $\bm{s}_1$ and $\bm{S}$ and the
% angle $\theta_{\rm SL}$ (WLOG, restrict $\theta_{\rm SL} \leq 90^\circ$), what
% is the probability that $\theta_{\rm s_1S} > \theta_{\rm SL}$? We should
% remember that, in the equal-spin case, $\theta_{\rm s_1S} \leq 90^\circ$. Thus,
% the probability is simply given by:
% \begin{equation}
%     P\p{\text{librating}} = 1 - \cos \theta_{\rm SL}.
% \end{equation}
% Here, librating means that both $\Phi_1$ and $\Phi_2$ are librating. These
% histograms are in the top panel, and seem to roughly reflect this trend, maybe?
% More controlled experiments are probably needed, e.g.\ GW radiation becomes
% extremely fast when we coalesce to $a \ll M$.

Relaxing either of the two assumptions, we can speak qualitatively:
\begin{itemize}
    \item When the spins are small, even at late times, the spins are just
        precessing with their own phases about $\bm{L}$, thus $\Delta \Phi$
        should be mostly uniformly distributed (unless $\bm{S}$ is very close to
        being aligned with $\bm{L}$, in which case this coincidentally implies
        that the $\bm{s}_i$ are on opposite sides of $\bm{L}$ and thus precess
        with fixed $\Delta \Phi$).

    \item When the spins are unequal, there is a possibility that one spin is
        circulating and one is librating (because they precess about $\bm{S}$ in
        different cones). This would result in another uniformly distributed
        $\Delta \Phi$ signature (not yet run).

    \item When the masses are unequal, it turns out that $\bm{S}$ does not
        cleanly precess around $\bm{J}$. However, looking at the equations, the
        overall phenomenology is mostly similar, but with a bit of a ``wobble''
        in the behavior due to the two precession frequencies being different.
\end{itemize}

\end{document}

