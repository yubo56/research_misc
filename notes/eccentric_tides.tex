    \documentclass[11pt,
        usenames, % allows access to some tikz colors
        dvipsnames % more colors: https://en.wikibooks.org/wiki/LaTeX/Colors
    ]{article}
    \usepackage{
        amsmath,
        amssymb,
        fouriernc, % fourier font w/ new century book
        fancyhdr, % page styling
        lastpage, % footer fanciness
        hyperref, % various links
        setspace, % line spacing
        amsthm, % newtheorem and proof environment
        mathtools, % \Aboxed for boxing inside aligns, among others
        float, % Allow [H] figure env alignment
        enumerate, % Allow custom enumerate numbering
        graphicx, % allow includegraphics with more filetypes
        wasysym, % \smiley!
        upgreek, % \upmu for \mum macro
        listings, % writing TrueType fonts and including code prettily
        tikz, % drawing things
        booktabs, % \bottomrule instead of hline apparently
        cancel % can cancel things out!
    }
    \usepackage[margin=1in]{geometry} % page geometry
    \usepackage[
        labelfont=bf, % caption names are labeled in bold
        font=scriptsize % smaller font for captions
    ]{caption}
    \usepackage[font=scriptsize]{subcaption} % subfigures

    \newcommand*{\scinot}[2]{#1\times10^{#2}}
    \newcommand*{\dotp}[2]{\left<#1\,\middle|\,#2\right>}
    \newcommand*{\rd}[2]{\frac{\mathrm{d}#1}{\mathrm{d}#2}}
    \newcommand*{\pd}[2]{\frac{\partial#1}{\partial#2}}
    \newcommand*{\rtd}[2]{\frac{\mathrm{d}^2#1}{\mathrm{d}#2^2}}
    \newcommand*{\ptd}[2]{\frac{\partial^2 #1}{\partial#2^2}}
    \newcommand*{\md}[2]{\frac{\mathrm{D}#1}{\mathrm{D}#2}}
    \newcommand*{\pvec}[1]{\vec{#1}^{\,\prime}}
    \newcommand*{\svec}[1]{\vec{#1}\;\!}
    \newcommand*{\bm}[1]{\boldsymbol{\mathbf{#1}}}
    \newcommand*{\ang}[0]{\;\text{\AA}}
    \newcommand*{\mum}[0]{\;\upmu \mathrm{m}}
    \newcommand*{\at}[1]{\left.#1\right|}

    \newtheorem{theorem}{Theorem}[section]

    \let\Re\undefined
    \let\Im\undefined
    \DeclareMathOperator{\Res}{Res}
    \DeclareMathOperator{\Re}{Re}
    \DeclareMathOperator{\Im}{Im}
    \DeclareMathOperator{\Log}{Log}
    \DeclareMathOperator{\Arg}{Arg}
    \DeclareMathOperator{\Tr}{Tr}
    \DeclareMathOperator{\E}{E}
    \DeclareMathOperator{\Var}{Var}
    \DeclareMathOperator*{\argmin}{argmin}
    \DeclareMathOperator*{\argmax}{argmax}
    \DeclareMathOperator{\sgn}{sgn}
    \DeclareMathOperator{\diag}{diag\;}

    \DeclarePairedDelimiter\bra{\langle}{\rvert}
    \DeclarePairedDelimiter\ket{\lvert}{\rangle}
    \DeclarePairedDelimiter\abs{\lvert}{\rvert}
    \DeclarePairedDelimiter\ev{\langle}{\rangle}
    \DeclarePairedDelimiter\p{\lparen}{\rparen}
    \DeclarePairedDelimiter\s{\lbrack}{\rbrack}
    \DeclarePairedDelimiter\z{\lbrace}{\rbrace}

    % \everymath{\displaystyle} % biggify limits of inline sums and integrals
    \tikzstyle{circ} % usage: \node[circ, placement] (label) {text};
        = [draw, circle, fill=white, node distance=3cm, minimum height=2em]
    \definecolor{commentgreen}{rgb}{0,0.6,0}
    \lstset{
        basicstyle=\ttfamily\footnotesize,
        frame=single,
        numbers=left,
        showstringspaces=false,
        keywordstyle=\color{blue},
        stringstyle=\color{purple},
        commentstyle=\color{commentgreen},
        morecomment=[l][\color{magenta}]{\#}
    }

\begin{document}

\def\Snospace~{\S{}} % hack to remove the space left after autorefs
\renewcommand*{\sectionautorefname}{\Snospace}
\renewcommand*{\appendixautorefname}{\Snospace}
\renewcommand*{\figureautorefname}{Fig.}
\renewcommand*{\equationautorefname}{Eq.}
\renewcommand*{\tableautorefname}{Tab.}

\onehalfspacing

\pagestyle{fancy}
\rfoot{Yubo Su}
\rhead{}
\cfoot{\thepage/\pageref{LastPage}}

\title{Eccentric Tides}
\author{Yubo Su}

\maketitle

As usual, this is kind of a scattered document. It isn't written linearly, so
notation evolves somewhat as it converges to the published version. Hopefully
things make sense if one jumps around a bit.

\section{Kushnir et.\ al., 2016}

We coarsely follow the derivation of Kushnir et.\ al., 2016 (KZ) to express the
traveling wave regime of dynamical tides in high-mass stars (convective core,
radiative envelope) in analytical form.

\subsection{Plane Parallel Case}

We will consider IGW in a plane parallel atmosphere in the Boussinesq
approximation. Consider buoyancy frequency
\begin{equation}
    N^2 = -g\p*{\rd{\ln \rho}{r} + \frac{g}{c_s^2}},
\end{equation}
where $c_s \to \infty$ is the sound speed in the fluid. Then the Boussinesq
equations can be written in terms of some \emph{buoyancy} variable $b$
\begin{subequations}\label{eq:bouss}
    \begin{align}
        \md{\vec{u}}{t} &= \frac{\vec{\nabla}P}{\rho_0} + b\hat{z},\\
        \md{b}{t} &= -N^2 u_z,\\
        \vec{\nabla} \cdot \vec{u} &= 0.
    \end{align}
\end{subequations}
Note that $b \equiv -\frac{\rho'}{\rho_0} g$, as can be verified via
direct substitution into the Euler equations:
\begin{subequations}\label{eq:bouss}
    \begin{align}
        0 &= \md{\rho'}{t} + \vec{u} \cdot \vec{\nabla}\rho_0 =
            \md{\rho'}{t} - u_z \frac{N^2}{g} \rho',\\
        \md{\vec{u}}{t} &= \frac{\vec{\nabla}P'}{\rho_0}
                - \frac{\rho'}{\rho_0^2} \vec{\nabla} P_0
            = \frac{\vec{\nabla}P'}{\rho_0} - \frac{\rho'}{\rho_0}g\hat{z}.
    \end{align}
\end{subequations}

These equations can be solved for $u_z$, or we can just recall the IGW
dispersion relation $\omega^2 k^2 = N^2k_{\perp}^2$ and write down PDE
\begin{equation}
    \ptd{}{t} \nabla^2 u_z = -N^2\nabla_{\perp}^2 u_z.\label{eq:igw_pde}
\end{equation}

Now, we might recall that in tidally-forced stars, $\omega$ the tidal forcing
frequency obeys $\omega \ll N$, or $k \gg k_\perp$. But the tidal potential, the
quadrupolar expansion of the gravitational perturbation from the companion, has
no quickly-varying directions, or can only excite $k \simeq k_\perp$ modes.
Thus, we intuit that waves must be excited where $N$ is much smaller than its
typical value, or near the \emph{radiative-convective boundary} (RCB). At the
RCB, $N^2 = 0$, and we are concerned with the turning point where $\omega^2 =
N^2$. We perform linear expansion about this turning point $z_c$, and for
convenience we set $z_c = 0$, then
\begin{equation}
    N^2 \approx \omega^2 + \rd{N^2}{z}z,
\end{equation}
where compared to KZ I've taken $N_0^2 = \omega^2$, there seems to be little
harm here. Making then general ansatz $u_z(z, \vec{r}_\perp, t) = \tilde{u}_z(z)
e^{i\p*{\vec{k}_{\perp} \cdot \vec{r}_{\perp} - \omega t}}$ we obtain
\begin{align}
    -\omega^2 \p*{-k_\perp^2 \tilde{u}_z + \tilde{u}_z''}
            &= N^2k_\perp^2 u_z,\\
        \tilde{u}_z'' + k_\perp^2\p*{\frac{N^2}{\omega^2} - 1}\tilde{u}_z &= 0
            ,\\
        \tilde{u}_z'' + k_\perp^2\rd{N^2}{z}\frac{z}{\omega^2}\tilde{u}_z &= 0.
\end{align}
It's easiest now to rescale $\tilde{k}_\perp^2 \equiv k_\perp^23
\rd{N^2}{z}\frac{1}{\omega^2}$ so that
\begin{equation}
    \tilde{u}_z'' + \tilde{k}_\perp z\tilde{u}_z = 0.
\end{equation}
The general solution to this ODE is written in terms of Airy functions for
arbitrary constants $a, b$
\begin{equation}
    \tilde{u}_z(z) = a\mathrm{Ai}\p*{-\frac{z}{\lambda}}
        + b\mathrm{Bi}\p*{-\frac{z}{\lambda}},\label{eq:gen_uz}
\end{equation}
where $\lambda = \tilde{k}_\perp^{-2/3}$. For large $-z$, it turns out that
\begin{align}
    \mathrm{Ai}\p*{-z} &\sim \frac{\sin\p*{\frac{2}{3}z^{3/2} +
        \frac{\pi}{4}}}{z^{1/4}} + \mathcal{O}\p*{z^{-7/4}},\\
    \mathrm{Bi}\p*{-z} &\sim \frac{\cos\p*{\frac{2}{3}z^{3/2} +
        \frac{\pi}{4}}}{z^{1/4}} + \mathcal{O}\p*{z^{-7/4}}.
\end{align}
In order for us to get traveling waves with \emph{group velocity} going outwards
(towards $z > 0$), we need $\tilde{u}_z(z) \sim e^{-ik_zz}$ such that $u(z, t)
\propto e^{i\p*{-k_zz - \omega t}}$ (phase velocity goes inwards, group velocity
goes outwards for IGW). Thus, $a = -ib$ in \autoref{eq:gen_uz}, and we obtain
\begin{equation}
    \tilde{u}_z(z) = b\p*{-i\mathrm{Ai}\p*{-\frac{z}{\lambda}}
        + \mathrm{Bi}\p*{-\frac{z}{\lambda}}},\label{eq:gen_uz}
\end{equation}

Now we just need to fix $b$. This is traditionally accomplished by mandating a
particular $\rd{\delta z}{z}$ the displacement of the mode at the turning point
$z = 0$. That the forcing results in a constraint on $\rd{\delta z}{z}$ is
similar to what I did in my IGW breaking forcing, where forcing induces a jump
in the $\rd{u_z}{z}$ above/below $z_c$ whose magnitude is fixed by the strength
of the forcing term. In the stellar problem, it appears the correct way to
obtain the $\delta z$ is to solve the inhomogeneous problem in the convective
zone where $N^2 = 0$ including the tidal potential, so it's not a perfect
analogy. But since $\mathrm{Ai}'(0) = -\frac{1}{3^{1/3}\Gamma(1/3)},
\mathrm{Bi}'(0) = \frac{3^{1/6}}{\Gamma(1/3)}$, this is not so difficult to
evaluate, and I cite the KZ result
\begin{equation}
    \rd{\delta z}{z} = -\frac{ib}{\lambda \omega}\frac{2}{3^{1/3}\Gamma(1/3)}
        \frac{3^{1/2} + i}{2}.
\end{equation}

Finally, we impose one more step: we will compute the luminosity or \emph{energy
flux} associated with the wave, since this is the easiest way to get the
resulting torque. We can easily write down the energy density of the wave
$\frac{\rho_0}{2}\p*{v^2 + \frac{b^2}{N^2}}$, for which the energy flux is
$\vec{F} = \vec{v}P$. Noting furthermore that $\pd{u_z}{z} = -ik_\perp u_x =
-\frac{ik_\perp P}{\rho_0}\frac{k_\perp }{\omega}$, we can explicitly express
$P$ in terms of $u_z'$, and so the energy flux density is then simply (I'm not
evaluating this, but KZ do)
\begin{align}
    \frac{\delta L}{\delta A} &= \frac{1}{2}\Re\p*{P u_z^*}
            = \frac{\rho_0 \omega}{2k_\perp^2} \Re\p*{iu_z'u_z^*},\\
        &= \frac{3^{2/3}\Gamma^2(1/3) \lambda \omega^3 \rho_0}{8\pi k_\perp^2}
            \p*{\rd{\delta z}{z}}^2.
\end{align}
We would then compute $L = \int \frac{\delta L}{\delta A}\;\mathrm{d}A$, which
for us is just $\frac{\delta L}{\delta A}A$ where $A$ is the surface area of the
wave.

Finally, we would compute the total torque from $L = \tau \omega$ (the same
as $E = \vec{F} \cdot \vec{v}$).

\subsection{Spherical Case}

To go to the spherical case, we simply replace $z \to r$ and $k_\perp^2 \to l(l
+ 1)/r^2$, which gives
\begin{equation}
    \lambda = \p*{\frac{l(l + 1)}{r^2\omega^2} \rd{N^2}{r}}^{-1/3}.
\end{equation}
Then to get $\rd{\delta z}{z} \to \rd{\delta r}{r}$, we use prescription
\begin{equation}
    \rd{\delta r}{r} = \alpha \frac{\Phi}{gr}
        \p*{1 - \frac{\rho(r)}{\bar{\rho}(r)}}.
\end{equation}
Here,
\begin{equation}
    \alpha = \p*{\frac{r_c}{R}}^{-5}\p*{\frac{M_c}{M}}\p*{1 -
        \frac{\rho}{\bar{\rho}}}^{-1}H_2,
\end{equation}
while $\bar{\rho}$ is the average density inside $r$. Finally, instead of
getting a clean $L = \frac{\delta L}{\delta A}A$, we have to actually do the
integral of $L = \int \frac{\delta L}{\delta A}\;\mathrm{d}A = \int (\dots)
\abs*{Y_{lm}}^2 r^2 \;\mathrm{d}\cos\theta \mathrm{d}\phi = r_c^2 L$ (note that
it's not $4\pi r_c^2$, thanks to the $Y_{lm}$ normalization). The $\ell = 2$
potential is taken to be
\begin{equation}
    \Phi_{\mathrm{ext}} = -\sqrt{\frac{6\pi}{5}}\frac{GM_2R_c^2}{D^3}.
\end{equation}
I guess the angular dependency is just dropped. With all these things together,
we obtain the final KZ result (I omit the derivation, this part is grungy and
not very physically interesting)
\begin{align}
    \tau = \dot{J}_z &= \frac{GM_2^2R_c^5}{D^6} \sigma_c^{8/3}
        \s*{\frac{r_c}{g_c}\p*{\rd{N^2}{\ln R}}_{r = r_c}}^{-1/3}
            \frac{\rho_c}{\bar{\rho}_c} \p*{1 - \frac{\rho_c}{\bar{\rho}_c}}^2
            \s*{\frac{3}{2}\frac{3^{2/3}\Gamma^2(1/3)}{5 \cdot
                6^{4/3}} \frac{3}{4\pi}\alpha^2},\\
        &= \frac{GM_2^2R_c^5}{D^6}2\hat{F}\p*{r_c, \sigma_c}.
            \label{eq:tau_fhat}
\end{align}
Note that $\hat{F}$ follows the convention from Equation 42 of Fuller \& Lai's
second paper (FL2) and Vick et.\ al's paper as well (VLF), while $\sigma_c =
2\abs*{\Omega - \Omega_s} / \sqrt{GM_c / r_c^3}$ is the ratio of the forcing
frequency to the breakup frequency \emph{of the core}. Finally, I've replaced
$M_2$ the mass of the companion, $R_c$ the radius of the core, and $D$ the
separation, while retaining $\Omega_s$ spin angular frequency and $\Omega$
orbital angular frequency.

\textbf{NB:} The exact definition of $\hat{F}$ for a given $m$ is given in
VLF.23 as
\begin{equation}
    \dot{J} = G\frac{M_2^2R^5}{a^3}\frac{\abs*{m}}{2}\hat{F}(\omega)
        = T_0 \frac{\abs*{m}}{2} \hat{F}(\omega).
\end{equation}
Since the total torque $\tau$ has already summed over $m = \pm 2$, we incur the
extra factor of $2$ above in \autoref{eq:tau_fhat}. That $m$ has already been
summed over is visible in the $\sqrt{6\pi/5}$ prefactor used in $\Phi_{ext}$,
compared to $W_{2\pm 2} = \sqrt{3\pi/10}$ as seen below.

\section{Vick et.\ al., 2016}

We now consider eccentric forcing. We will remove subscript compared to VLF and
just call $\vec{r}_i = \p*{r, \theta, \phi + \Omega_s t}$ the position
coordinate in the inertial frame. Then the $\ell = 2$ tidal forcing potential is
generally a sum over $m \in [-2, 2]$
\begin{align}
    U &= \sum\limits_m U_{2m} \p*{\vec{r}, t},\\
    U_{2m}\p*{\vec{r}} &= -\frac{GM_2 W_{2m} r^2}{D(t)^3}
        e^{-imf(t)} Y_{2m}(\theta, \phi).
\end{align}
Note that $f$ is the true anomaly here. Note that $W_{2m}$ is just a constant:
$W_{20} = \sqrt{\pi/5}$, $W_{2 \pm 1} = 0$, and $W_{2 \pm 2} = \sqrt{3\pi /
10}$.

This is complicated since $f(t)$ does not evolve uniformly, ando also since
$D(t)$ is time-varying! The easiest treatment is to decompose
\begin{equation}
    U_{2m} = -\frac{GM_2W_{2m}r^2}{a^3}Y_{2m}\p*{\theta, \phi}
        \sum\limits_{N = -\infty}^\infty F_{Nm}e^{-iN\Omega t}.
\end{equation}
Note that the $F_{Nm}$ here are \emph{Hansen coefficients} given by
\begin{equation}
    F_{Nm} = \frac{1}{\pi}\int\limits_{0}^{\pi}
        \frac{\cos\s*{N\p*{E - e\sin E} - mf(E)}}
            {\p*{1 - e\cos E}^2}\;\mathrm{d}E.
\end{equation}
Note $E$ is the eccentric anomaly. This differs from the VLF definition in a few
places but is in agreement with Natalia's paper w/ Dong (SD), such that $F_{Nm}
= \delta_{Nm}$ for $e = 0$. It bears noting that VLF's formula normalizes to
$F_{Nm} = 2\delta_{Nm}$, so we use the restricted domain of integration for
numerical speed (the integrand is symmetric since the argument of the cosine is
antisymmetric in $E$, so both the numerator/denominator are even in $E$).

Let's explicitly write out the $U_{2\pm 2}$, since they are the only ones that
contribute to the tidal torque
\begin{align}
    U_{22} &= -\frac{GM_2 \sqrt{\frac{3\pi}{10}}r^2}{a^3}
        \sum\limits_{N = 1}^\infty \s*{
            F_{N2}Y_{22}\p*{\theta, \phi}e^{-i\p*{N\Omega - 2\Omega_s}t}
            + F_{-N2}Y_{22}\p*{\theta, \phi}e^{i\p*{N\Omega + 2\Omega_s}t}},\\
    U_{2-2} &= -\frac{GM_2 \sqrt{\frac{3\pi}{10}} r^2}{a^3}
        \sum\limits_{N = 1}^\infty \s*{
            F_{-N2}Y_{2-2}\p*{\theta, \phi}e^{-i\p*{N\Omega - 2\Omega_s}t}
            + F_{N2}Y_{2-2}\p*{\theta, \phi}e^{i\p*{N\Omega + 2\Omega_s}t}},\\
    U_{22} + U_{2-2} &= -\frac{GM_2 \sqrt{\frac{3\pi}{10}} r^2}{a^3}
        \sum\limits_{N = 1}^\infty \s*{
            F_{N2}Y_{22}\p*{\theta, \phi}e^{-i\p*{N\Omega - 2\Omega_s}t}
            + c.c.}
\end{align}
We can verify that if the perturbing orbit is circular $e = 0$, then the Hansen
coefficient $F_{Nm} = \delta_{Nm}$, and we obtain
\begin{equation}
    U_{22} + U_{2-2} = -\frac{GM_2r^2}{a^3}\sqrt{\frac{6\pi}{5}}
        \Re \s*{Y_{22}\p*{\theta, \phi}e^{-2i\p*{\Omega - \Omega_s}t}}.
\end{equation}
This is the same torque used in KZ\@. Finally, this yields torque
\begin{equation}
    \dot{J} = \tau = T_0 \sum\limits_{N = -\infty}^\infty
        F_{N2}^2 \sgn\p*{N\Omega - 2\Omega_s} \hat{F}
            \p*{\omega = \abs*{N\Omega - 2\Omega_s}}.
\end{equation}

\subsection{Hansen Coefficients}

Maybe someday follow \url{https://arxiv.org/pdf/1308.0607.pdf} and get the
derivation of the Hansen coefficients? One fast way to calculate them is to take
an FFT of the $F^{lm} = \p*{\frac{r}{a}}^l e^{imf}$, per
\url{https://www.aanda.org/articles/aa/pdf/2014/11/aa24211-14.pdf}% chktex 8
(CBLR). Basically, the Hansen coefficients are just the FT of the disturbing
function. Consider that we want to make jump from
\begin{equation}
    U(r, t) = -GM_rr^2 \sum\limits_m \frac{W_{2m}}{D(t)^3}e^{-imf(t)}
        Y_{2m}\p*{\theta, \phi},
\end{equation}
to
\begin{equation}
    U(r, t) = -\frac{GM_2r^2}{a^3}\sum\limits_{m, N} W_{2m}
        F_{Nm}(e) Y_{2m}\p*{\theta, \phi} e^{-in\Omega t}.
\end{equation}
Thus, we seek coefficients such that
\begin{equation}
    \frac{a^3}{D(t)^3} e^{-imf} = \p*{\frac{1 + e\cos f}{1 - e^2}}^3
            e^{-imf}
        = \sum\limits_N F_{Nm} e^{-iN\Omega t}.
\end{equation}
Thus, it's clear the Hansen coefficients are defined by computing Fourier series
coefficients (NB\@: In hindsight, using $r = a\p*{1 - e\cos E}$ probably would
have been much faster/easier)
\begin{align}
    F_{Nm} &\equiv \frac{1}{T}\int\limits_{0}^T
        \frac{e^{-imf}}{\p*{1 - e^2}^3}\p*{1 + e\cos f}^3 e^{iN\Omega t}
            \;\mathrm{d}t,\\
        &= \frac{1}{2\pi} \int\limits_{0}^{2\pi}
            \frac{e^{-imf}}{\p*{1 - e^2}^3}\p*{1 + e\cos f}^3 e^{iN\Omega t}
                \;\mathrm{d}M
\end{align}
We have notated $T$ the period, and $M$ the mean anomaly. Then one just
evaluates using $\cos f = \frac{\cos E - e}{1 - e\cos E}$ and $M = E - e\sin E$
or more usefully $\mathrm{d}M = \p*{1 - e\cos E}\mathrm{d}E$ and obtains
\begin{align}
    F_{Nm} &= \frac{1}{2\pi}\int\limits_0^{2\pi}
            \p*{\frac{1 + e\cos f}{1 - e^2}}^3 e^{-imf + iN\Omega t}
                \;\mathrm{d}M,\\
        &= \frac{1}{2\pi}\int\limits_0^{2\pi}
            \p*{\frac{1}{1 - e\cos E}}^3 e^{-imf + iN M} \p*{1 - e\cos E}
                \;\mathrm{d}E,\\
        &= \frac{1}{2\pi}\int\limits_0^{2\pi}
            \frac{\exp\s*{i\p*{N\p*{E - e\sin E} - mf}}}{(1 - e \cos E)^2}
                \;\mathrm{d}E.
\end{align}
Now, as we observed above, the integrand is symmetric with respect to $E$, but
it had to be, since in an elliptical orbit the first half and second half are
obviously symmetric. Thus, we arrive at final expression as promised
\begin{equation}
    F_{Nm} = \frac{1}{\pi}\int\limits_{0}^{\pi}
        \frac{\cos\s*{N\p*{E - e\sin E} - mf(E)}}
            {\p*{1 - e\cos E}^2}\;\mathrm{d}E.
\end{equation}

\section{Combined Results}

We have been somewhat careful in checking the agreement between the VLF and KZ
forms. Note now that \autoref{eq:tau_fhat} has $\hat{F}$ for a single $m$
contribution, just as $\hat{F}$ is defined in VLF\@. Thus, we should be able to
simply plug in
\begin{align}
    \tau &= T_0 \sum\limits_{N = -\infty}^\infty
        F_{N2}^2 \frac{\sgn\p*{N\Omega - 2\Omega_s}}{2} \sigma_c^{8/3}
        \s*{\frac{r_c}{g_c}\p*{\rd{N^2}{\ln R}}_{r = r_c}}^{-1/3}
            \frac{\rho_c}{\bar{\rho}_c} \p*{1 - \frac{\rho_c}{\bar{\rho}_c}}^2
            \s*{\frac{3}{2}\frac{3^{2/3}\Gamma^2(1/3)}{5 \cdot
                6^{4/3}} \frac{3}{4\pi}\alpha^2},\\
        &= T_0 C(r_c) \sum\limits_{N = -\infty}^\infty
            F_{N2}^2 \sgn\p*{N\Omega - 2\Omega_s}
                \abs*{N\Omega - 2\Omega_s}^{8/3}.\label{eq:tau_sum}
\end{align}
Note that now $\sigma_c = \abs*{N\Omega - 2\Omega_s} / \sqrt{GM_c / r_c^3}$, and
I've defined $C(r_c)$ to be some (dimensional) constant defined at the RCB and
does not change with $N$.

Thus, the relative significance of each $N$ term is given by the summand of
\autoref{eq:tau_sum}, or
\begin{equation}
    \tau_N \equiv F_{N2}^2 \sgn\p*{N\Omega - 2\Omega_s} \sigma_c^{8/3},
\end{equation}
$F_{N2}$ is easiest to evaluate via an integral for now, but can probably be
done via a sampling + FFT when speed is necessary (CBLR).

One guess is that the sum is dominated by the contribution of the frequency at
pericenter. We can compute pericenter frequency as follows:.
\begin{align}
    r_p^2 \Omega_p &= \sqrt{GMa\p*{1 - e^2}},\\
    \Omega_p &= \sqrt{\frac{GMa\p*{1 + e}\p*{1 - e}}{a^4\p*{1 - e}^4}},\\
        &= \Omega \frac{\sqrt{1 + e}}{\p*{1 - e}^{3/2}}.
\end{align}
Thus, we should expect the dominant term to come at $N \sim
\frac{\Omega_p}{\Omega} = \frac{\sqrt{1 + e}}{\p*{1 - e}^{3/2}}$. This indeed
very nearly maximizes $F_{N2}$ but probably won't maximize $\tau_N$.

\textbf{NB:} It appears that the maximum $N$ to sum to is (according to
Michelle/Chris)
\begin{equation}
    N_{\max} = 10 \Omega_p.
\end{equation}

To understand exactly to what $N$ we should sum, we should recall $\abs*{\tau_N}
\propto \abs*{F_{N2}}^2 \abs*{N\Omega - 2\Omega_s}^{8/3}$. If $\Omega_s \gg
\Omega_p$, then the latter term $\sigma_c^2$ is roughly independent of $N$ and
indeed we get that $\tau_N$ turns over similarly to where $F_{N2}$ turns over.
On the other hand, if $\Omega_s \lesssim \Omega_p$, then $\tau_N \propto
\abs*{F_{N2}}^2 N^{8/3}$. Plots of these two are given in \autoref{fig:hansens}.
\begin{figure}[t]
    \centering
    \includegraphics[width=0.5\textwidth]{../scripts/eccentric_tides/hansens.png}
    \caption{$F_{N2}$ and $F_{N2} N^{8/3}$ as we integrate straightforwardly.
    Note the vertical blue line is $N = \p*{1 - e}^{-3/2}$ while the vertical
    black/green line are the actual $\argmax_N F_{N\pm 2}$ resrpectively.
    }\label{fig:hansens}
\end{figure}

A plot of the actual maximum $F_{N \pm 2}$ is provided below in
\autoref{fig:maxes}. Note that $F_{N2}$ seems to be a constant multiple of
$\p*{1 - e}^{-3/2}$ our prediction, while $F_{N-2}$ seems to be impacted by the
shallowness of the fit at smaller eccentricities. That $F_{N2}$ has its maximum
at a slight multiple of $\p*{1 - e}^{-3/2}$ should not be surprising, since the
actual pericenter passage time can be computed by conservation of angular
momentum
\begin{figure}[t]
    \centering
    \begin{subfigure}{0.45\textwidth}
        \centering
        \includegraphics[width=\textwidth]{../scripts/eccentric_tides/hansen_maxes.png}
    \end{subfigure}
    \begin{subfigure}{0.45\textwidth}
        \centering
        \includegraphics[width=\textwidth]{../scripts/eccentric_tides/hansen_maxes83.png}
    \end{subfigure}
    \caption{Maxima of $F_{N\pm 2}$ and $N^{8/3}F_{N\pm 2}$ with
    fits.}\label{fig:maxes}
\end{figure}

\subsection{$m = 2$ Hansen Coefficient Fit}\label{ss:F_n2}

To understand the behavior of how multiplying by $N^{8/3}$ changes the peak of
$F_{N2}$, let's consider the simplest model for the scaling of $F_{N2}$. Upon
examination, it decays as $F_{N2} \propto e^{-aN}$, where $a \approx
-\frac{1}{75}$. If we allow a simple model for $F_{N2} \propto N^q e^{-aN}$
(this conforms very coarsely with the plotted $F_{N2}$ and allows for a maximum
and an exponential tail), then we can identify that its maximum is at $N_{\max}
= q/a$. On the other hand, if we seek the maximum of $N^{p}F_{N2}$, we find its
maximum is instead at $N_{\max}^{(p)} = (p + q) / a$. Comparing the two, we
expect the maximum to be shifted by roughly factor $\frac{p + q}{q}$. Since the
actual shift is $\gtrsim 2$ for $p = 8/3$, we can guess $q \approx 2$, which is
plausible gauging from our loglog plot.

Empirically, we find that the $F_{N\pm 2} = CN^pe^{-N/a}$ is actually a
surprisingly good fit. This is not surprising: the coefficients must be small
for $N \ll N_{\max}$, but must fall off exponentially for $N \gg N_{\max}$. Note
that simple calculus shows us that $\argmax_N F_{N \pm 2} = p/a$, and so $a =
p / N_{\max} \simeq \frac{p\Omega_0}{\Omega_p}$. Furthermore, between $\Omega,
\Omega_p$, there is no preferred timescale, so a power law dependence between
$\Omega, \Omega_p$ is expected (and $\Omega \ll \Omega_p \approx 0$ for large
eccentricities).

\subsubsection{Parameter Scalings}

\begin{figure}[t]
    \centering
    \includegraphics[width=\textwidth]{../scripts/eccentric_tides/hansen_params.png}
    \caption{Params of the $P_{N \pm 2} = CN^pe^{-N/a}$ fit as function of
    eccentricity.}\label{fig:params}
\end{figure}
The next thing to check is how robust these parameters are, and whether they
have analytical forms. We present such a comparison in \autoref{fig:params}.
The good news is that $p$ is relatively independent of eccentricity! This gives
obvious scaling for
\begin{equation}
    a = \frac{N_{\max}}{p} = \frac{\sqrt{1 + e}}{p\p*{1 - e}^{3/2}},
\end{equation}
since I changed convention $P_{n \pm 2} \propto e^{-N/a}$ whoops.

The final difficulty is understanding how $C$ scales. To understand this, the
easiest thing to do seems to be to invoke Parseval's Theorem, which will let us
claim that
\begin{equation}
    \frac{1}{T}\int\limits_{0}^T \abs*{\frac{a^3}{D(t)^3}e^{-imf}}^2
            \;\mathrm{d}t =
        \sum\limits_{N = -\infty}^\infty \abs*{F_{Nm}}^2.
\end{equation}
Let's assume for now that $F_{N-2} \ll F_{N2}$\footnote{This is empirically
true, but also, we can examine the $F_{Nm}$ integral and observe that for large
$e$, $f(E)$ will only be nonzero if $\delta E \lesssim \sqrt{\delta e}$
($\delta e = 1 - e$; examine the arctan relation, but also, $f(E) = \pi$ at
apoapsis), then the argument of the cosine is $\p*{N \delta e - m/\sqrt{\delta
e}} \delta E$. Thus, it's quickly oscillating for most $N$ unless $N \sim
m/\delta e^{3/2}$, roughly our $N_{\max}$ criterion. Also, it is quickly
oscillating when $N, m$ have opposite signs.}. Let's further assume that $\Re
F_{N2} \gg \Im F_{N2}$ (no idea if this is the case yet), then the sum over
coefficients is just dominated the contribution from the $C_+ N^{p_+}e^{-N/a_+}$
parts, which we can approximate with an integral analytically.

What about the time-domain integral? Well, this seems to evaluate cleanly
\begin{align}
    \frac{1}{T}\int\limits_0^T \frac{a^6}{D^6}\;\mathrm{d}t &= \frac{1}{2\pi}
        \int\limits_{0}^{2\pi} \p*{\frac{1 + e\cos f}{1 - e^2}}^6
            \frac{\p*{1 - e^2}^{3/2}}{\p*{1 + e\cos f}^2}\;\mathrm{d}f,\\
        &= \frac{1}{2\pi\p*{1 - e^2}^{9/2}}\int\limits_0^{2\pi}
            \p*{1 + e\cos f}^4\;\mathrm{d}f,\\
        &= \frac{1}{2\pi\p*{1 - e^2}^{9/2}}\int\limits_0^{2\pi}
            1 + 4e\cos f + 6e^2\cos^2f + 4e^3\cos^3f + e^4\cos^4f\;\mathrm{d}f
                ,\\
        &= \frac{1}{\p*{1 - e^2}^{9/2}}\s*{
            1 + 3e^2 + \frac{3e^4}{8}}.
\end{align}
This is analytic and correct (I double checked numerically). The sum of
coefficients can be approximated (see later)
\begin{align}
    \sum\limits_{N = -\infty}^\infty \abs*{F_{Nm}}^2 &\approx
            \sum\limits_{N = 0}^\infty \p*{\Re F_{Nm}}^2,\\
        &\approx C_+^2 \p*{a_+ / 2}^{2p_+ + 1}\Gamma\p*{2 p_+ + 1}.
\end{align}
This agreement is remarkable (\lstinline{check_ft_parsevals.png}). Thus,
explicitly we may write down
\begin{equation}
    C_+ \approx \sqrt{\frac{ 1 + 3e^2 + \frac{3e^4}{8}}{\p*{1 - e^2}^{9/2}}
        \frac{1}{\p*{a_+/2}^{2p_+ + 1} \Gamma\p*{2 p_+ + 1}}}.
\end{equation}
Thus, if we simply take $p_+ = 2$, we have analytical predictions for each of
the params. It turns out that we want closer to $p_+ = 2, N_{\max} =
\sqrt{2}\Omega_p/\Omega$, but this gives us a very good fit, as seen in
\autoref{fig:params}.

\subsection{$m = 0$ Hansen Coefficients}

Consider the $m = 0$ Hansen coefficients, which are given by
\begin{equation}
    \frac{a^3}{D(t)^3} = \sum\limits_{N = -\infty}^\infty
        F_{N0} e^{-iN\Omega t}.
\end{equation}
Again, the tail must be exponential, but since there is obviously symmetry about
$N = 0$, the peak is at $N = 0$. We might have naively expected a Gaussian,
thanks to the tight peaking in the time domain about $t = 0$, but instead what
we observe is something like $F_{N0} \propto e^{-\sqrt{2}N/N_{\rm peri}}$ for
some reason.

Assuming this is the case, we can again invoke Parseval's and find
\begin{align}
    \frac{1}{\p*{1 - e^2}^{9/2}}\s*{1 + 3e^2 + \frac{3e^4}{8}}
        &\approx 2\int\limits_0^\infty C^2e^{-\frac{N2\sqrt{2}}{N_{\rm peri}}}
            \;\mathrm{d}N,\\
        &\approx 2C^2N_{\rm peri}\sqrt{2},\\
    C^2 &= \frac{1 + 3e^2 + \frac{3e^4}{8}}{\p*{1 - e^2}^{9/2}}
        \frac{1}{2N_{\rm peri}\sqrt{2}}.
\end{align}

\subsection{Effective Torque}

With this approximate closed form, it is very easy to sum over all $N$ by
approximating as an integral:
\begin{equation}
    \sum\limits_{N = 1}^\infty N^q F_{N \pm 2}
        \approx \int\limits_0^\infty CN^{p + q}e^{-N/a}\;\mathrm{d}N
\end{equation}
The RHS is almost a Gamma function though!! Thus
\begin{align}
    \sum\limits_{N = 1}^\infty N^q F_{N \pm 2} &\approx
        Ca^{p + q + 1}
        \int\limits_0^\infty (N / a)^{p + q}e^{-N/a}\;\mathrm{d}(N/a),\\
        &\approx Ca^{p + q + 1} \Gamma\p*{p + q + 1}.
\end{align}

Thus, we return to our total torque. Call $F_{N \pm 2} = C_{\pm}N^{p_{\pm}}
e^{-N/a_{\pm}} = F_{\pm N 2}$, then
\begin{align}
    \tau &= T_0 C(r_c) \sum\limits_{N = -\infty}^\infty
            F_{N2}^2 \sgn\p*{N\Omega - 2\Omega_s}
                \sigma_c^{8/3},\\
        &= T_0 C(r_c) \frac{\Omega^{8/3}}{\p*{GM_c/r_c^3}^{4/3}}
            \sum\limits_{N = -\infty}^\infty
                F_{N2}^2 \sgn\p*{N - 2\frac{\Omega_s}{\Omega}}
                    \abs*{N - 2\frac{\Omega_s}{\Omega}}^{8/3},\\
        &= T_0 \hat{C}(r_c) \sum\limits_{N = 0}^\infty
            \s*{F_{N2}^2 \sgn\p*{N - 2\frac{\Omega_s}{\Omega}}
                \abs*{N - 2\frac{\Omega_s}{\Omega}}^{8/3} -
                F_{-N2}^2 \abs*{-N - 2\frac{\Omega_s}{\Omega}}^{8/3}},\\
        &\approx T_0\hat{C}(r_c)\s*{
            \int\limits_0^\infty C_+^2N^{2p_{+}} e^{-2N / a_{+}}
                \sgn\p*{N - 2\frac{\Omega_s}{\Omega}}
                \abs*{N - 2\frac{\Omega_s}{\Omega}}^{8/3}\;\mathrm{d}N -
             \int\limits_0^\infty C_-^2N^{2p_{-}} e^{-2N / a_{-}}
                    \abs*{-N - 2\frac{\Omega_s}{\Omega}}^{8/3}\;\mathrm{d}N}.
\end{align}
Note that the summation should not double count the $F_{02}$ term; this is
resolved correctly in the integral, where the $N = 0$ contribution is not double
counted. It's convenient to call
\begin{equation}
    \hat{\tau}_N \equiv F_{N2}^2 \sgn\p*{N - 2\frac{\Omega_s}{\Omega}}
        \abs*{N - 2\frac{\Omega_s}{\Omega}}^{8/3}.
\end{equation}

At this point, let's specialize to two regimes:
\begin{itemize}
    \item Let $2\Omega_s \ll N_{\max} \Omega$, then the sign term is just the
        sign of $N$, and $\abs*{\pm N - \frac{2\Omega_s}{\Omega}} \sim
        \abs*{N}\p*{1 - \frac{\Omega_s/\Omega}{N_{peri}}}$. This gives
        (dropping the $C_-$ terms since we've seen they're unimportant and we
        don't have fits for them)
        \begin{align}
            \tau &\approx T_0 \hat{C}(r_c)
                \p*{1 - \frac{\Omega_s/\Omega}{N_{peri}}}^{8/3}
            \s*{
                C_+^2\int\limits_0^{\infty}
                    N^{2p_+ + 8/3}e^{-2N/a_+}\;\mathrm{d}N},\\
                &\approx T_0\hat{C}(r_c)\s*{
                    C_+^2 (a_+/2)^{2p_+ + 11/3} \Gamma\p*{2p_+ + 11/3}}.
        \end{align}

    \item Alternatively, let $2\Omega_s \gg N_{\max}\Omega$, then the sign is
        just always negative and $\abs*{N - \frac{2\Omega_s}{\Omega}} =
        \frac{2\Omega_s}{\Omega} - N_{\max}$, and we find
        \begin{align}
            \tau &\approx T_0 \hat{C}(r_c) \p*{\frac{2\Omega_s}{\Omega} -
                    N_{\max}}^{8/3}
                \s*{-C_+^2\int\limits_0^{\infty}
                    N^{2p_+}e^{-2N/a_+}\;\mathrm{d}N},\\
                &\approx T_0\hat{C}(r_c)
                    \p*{\frac{2\Omega_s}{\Omega} - N_{\max}}^{8/3}\s*{
                        -C_+^2 (a_+/2)^{2p_+ + 1} \Gamma\p*{2p_+ + 1}}.
        \end{align}
\end{itemize}
Indeed, for $2\Omega_s \ll N_{\max}\Omega$, we find $\tau > 0$, while for
$2\Omega_s \gg N_{\max} \Omega$, we obtain $\tau < 0$, which obeys intuition and
suggests some synchronization frequency around $2\Omega_s \simeq \Omega_p$.

We make some plots as in \autoref{fig:totals_s} and these agree. What remains is
to identify the eccentricity scaling, which will require understanding how the
fit coefficients depend on $e$, which we've also plotted but don't quite yet
understand so don't include.
\begin{figure}[t]
    \centering
    \includegraphics[width=\textwidth]{../scripts/eccentric_tides/totals_s_0_9.png}
    \caption{Plot of $\frac{\tau}{T_0 \hat{C}(r_c)}$. The predictions in the two
    regimes are the black horizontal line $\Omega_s = 0$ and red dashed line
    $2\Omega_s \gg N_{\max}\Omega$. Good agreement!}\label{fig:totals_s}
\end{figure}

\subsection{Heating}

Heating in the inertial frame is given (absorbed the second $N$ into the
absolute value exponent)
\begin{align}
     \dot{E}_{in} = \frac{1}{2}\hat{T}\p*{r_c, \Omega}\s*{
         \sum\limits_{N = -\infty}^\infty
            N\Omega F_{N2}^2 \sgn \p*{\sigma} \abs*{\sigma}^{8/3}
            + \p*{\frac{W_{20}}{W_{22}}}^2\Omega F_{N0}^2 \abs*{N}^{11/3}}.
\end{align}
I'm lazy so $\sigma = N - \frac{2\Omega_s}{\Omega}$, and $\p*{W_{20} /
W_{22}}^2 = 2/3$.

The former term follows our work above, for instance in the $\Omega_s \approx 0$
limit, we just have the first term (subscript)
\begin{equation}
    \dot{E}_1 = \frac{\hat{T}\Omega }{2} C^2
        \p*{\frac{\eta}{2}}^{26/3}\Gamma\p*{26/3},
\end{equation}
or in the $\Omega_s \gg 0$ case
\begin{equation}
    \dot{E}_1 = \frac{\hat{T}\Omega}{2} C^2
        \p*{\frac{2\Omega_s}{\Omega} - N_{\max}}^{8/3}
            (\eta/2)^{6} \Gamma\p*{6}.
\end{equation}

For the latter term, it is symmetric about $N$, so we can integrate about just
$N \geq 0$:
\begin{align}
    \dot{E}_2 &= \hat{T}\Omega \frac{2}{3} \int\limits_0^\infty
            C_0^2 e^{-\frac{N}{N_{\rm peri}}2\sqrt{2}}N^{11/3}\;\mathrm{d}N,\\
        &= \hat{T} \Omega \frac{2}{3} C_0^2
            \p*{\frac{N_{\rm peri}}{2\sqrt{2}}}^{14/3}
            \Gamma\p*{14/3}.
\end{align}

\section{Comparison with Existing Work, Moving Past Massive Stars}

Key results are Eq.~\eqref{eq:fn2_swiss_army_knife},
Eq.~\eqref{eq:traveling_torque}, and Eq.~\eqref{eq:total_heating}. Note that the
technique used to constrain the prefactor on the $\Omega_s \ll N_{\max}\Omega$
is general and necessary for any of these results where we need to expand
$\abs*{N\Omega - m\Omega_s}^p$ in two regimes.

\subsection{Weak Tides w/ Hut, Storch and Vick: Updating $F_{N2}$}

In the aforementioned papers, results of form
\begin{equation}
    \frac{1}{2}\sum\limits_{N}F_{N2}^2 \p*{N - m\frac{\Omega_s}{\Omega}}
        = \frac{1}{\p*{1 - e^2}^6}\s*{f_2 - \p*{1 - e^2}^{3/2}
            f_5\frac{\Omega_s}{\Omega}},
\end{equation}
are obtained, where $m = 2$, where $f_5 = 1 + 3e^2 + 3e^4/8$ and $f_2 = 1 +
15e^2/2 + 45e^4/8 + 5e^6/16$ are the relevant numbers. In the latter term, the
Parseval's result is simply reproduced exactly and is in agreement with above:
\begin{align}
    \sum\limits_N -F_{N2}^2\frac{\Omega_s}{\Omega}
        &= \p*{-\frac{\Omega_s}{\Omega}}\p*{\frac{1 + 3e^2 + 3e^4/8}{
            \p*{1 - e^2}^{9/2}}}.
\end{align}
The more interesting case is the linear term. An extended application of
Parseval's Theorem is in play, where the two functions are not the same. We then
must use
\begin{equation}
    \rd{}{\p*{-i \Omega t}}\frac{a^3}{D^3}e^{-imf}
        = \sum\limits_N F_{Nm} Ne^{-iN\Omega t},
\end{equation}
as then (star denotes conjugate)
\begin{align}
    \sum\limits_N F_{N2}^2 N &= \frac{1}{T}\int\limits
        \frac{a^3}{D^3}e^{-imf}
            \p*{\rd{}{\p*{-i \Omega t}}\p*{\frac{a^3}{D^3}
                e^{-imf}}}^*\;\mathrm{d}t,\\
        &= \frac{1}{2\pi} \int\limits_0^{2\pi}
            \frac{a^3}{D^3}e^{-imf}
            \p*{i\rd{}{f}\p*{\frac{a^3}{D^3}
                e^{-imf}}}^*\;\mathrm{d}f,\\
        &= \frac{1}{2\pi} \int\limits_0^{2\pi}
            \frac{a^6}{D^6}\p*{m + \frac{3e\sin f}{1 + e\cos f}}
                \;\mathrm{d}f.
\end{align}
Recall that $a/D = \p*{1 + e\cos f} / \p*{1 - e^2}$, so
\begin{align}
    \sum\limits_N F_{N2}^2 N
        &= \frac{1}{2\pi\p*{1 - e^2}^6} \int\limits_0^{2\pi}
            \p*{1 + e\cos f}^6\p*{m + i\frac{3e\sin f}{1 + e\cos f}}
                \;\mathrm{d}f.
\end{align}
The $\sin f$ term integrates to zero, and setting $m = 2$ cancels with the
prefactor, and we can explicitly integrate (only even powers survive)
\begin{align}
    \frac{1}{2}\sum\limits_N F_{N2}^2N
        &= \frac{1}{2\pi\p*{1 - e^2}^6}\int\limits_0^{2\pi}
            \p*{1 + 15e^2\cos^2f + 15e^4\cos^4f
                + e^6\cos^6 f}\;\mathrm{d}f,\\
        &= \frac{1}{\p*{1 - e^2}^6}\p*{1 + \frac{15e^2}{2}
            + \frac{45 e^4}{8} + \frac{5e^6}{16}}.
\end{align}
Note that we have used the following identities:
\begin{align}
    \frac{1}{2\pi}\int\limits_0^{2\pi}\cos^2x\;\mathrm{d}x
        &= \frac{1}{2\pi}
            \int\limits_0^{2\pi}\frac{1 + \cos 2x}{2}\;\mathrm{d}x
        = \frac{1}{2},\\
    \frac{1}{2\pi}\int\limits_0^{2\pi}\cos^4x\;\mathrm{d}x
        &= \frac{1}{2\pi}
            \int\limits_0^{2\pi}\p*{\frac{1 + \cos 2x}{2}}^2\;\mathrm{d}x
        = \frac{1}{2\pi}
            \int\limits_0^{2\pi}\frac{1 + \cos^2 2x}{4}
                + (\dots)\;\mathrm{d}x
        = \frac{3}{8},\\
    \frac{1}{2\pi}\int\limits_0^{2\pi}\cos^6x\;\mathrm{d}x
        &= \frac{1}{2\pi}
            \int\limits_0^{2\pi}\p*{\frac{1 + \cos 2x}{2}}^3\;\mathrm{d}x
        = \frac{1}{2\pi}
            \int\limits_0^{2\pi}\frac{1 + 3\cos^2 2x}{8}
                 + (\dots)
        = \frac{5}{16}.
\end{align}

Now, how does our fitted formula compare to this? Well, we really only have to
compare the $F_{N2}^2N$ term, so
\begin{align}
    \sum\limits_{N}F_{N2}^2N &\approx \int\limits_0^\infty
            C^2 N^4 e^{-2N/\eta}N\;\mathrm{d}N,\\
        &= \frac{f_5}{\p*{1 - e^2}^{9/2}}
                \frac{1}{\p*{\eta / 2}^5 4!}
            \int\limits_0^\infty N^5 e^{-2N/\eta}\;\mathrm{d}N,\\
        &= \frac{f_5}{\p*{1 - e^2}^{9/2}}
                \frac{1}{\p*{\eta / 2}^5 4!}
            \p*{\frac{\eta}{2}}^6
            \int\limits_0^\infty \p*{\frac{2N}{\eta}}^5
                e^{-2N/\eta}\;\mathrm{d}\frac{2N}{\eta},\\
        &= \frac{f_5}{\p*{1 - e^2}^{9/2}}
                \frac{1}{\p*{\eta / 2}^5 4!}
            \p*{\frac{\eta}{2}}^6
            \int\limits_0^\infty \p*{\frac{2N}{\eta}}^5
                e^{-2N/\eta}\;\mathrm{d}\frac{2N}{\eta},\\
        &= \frac{f_5}{\p*{1 - e^2}^{9/2}}
                \frac{1}{\p*{\eta / 2}^5 4!}
            \p*{\frac{\eta}{2}}^6 5!.
\end{align}
Recall that $\eta = \frac{N_{\max}}{2} = \frac{\sqrt{1 + e}}{2\p*{1 -
e^2}^{3/2}}\alpha$ per the results of \autoref{ss:F_n2}, where we guessed $\alpha
\approx \sqrt{2}$, and so everything all together evaluates to
\begin{equation}
    \sum\limits_{N}F_{N2}^2N \approx \frac{f_5}{\p*{1 - e^2}^{6}}
        \frac{5\sqrt{1 + e}}{4}\alpha.
\end{equation}

We thus want to compare
\begin{equation}
    2f_2 = 2\p*{1 + \frac{15 e^2}{2} + \frac{45e^4}{8} + \frac{5e^6}{16}}
        \approx \alpha\frac{5}{4}\sqrt{1 + e}
            \p*{1 + 3e^2 + \frac{3e^4}{8}}.
\end{equation}
Our approximation is valid when $e \to 1$, so let's first set $e = 1$ and
evaluate. This fixes $\alpha$, and indeed when we do so we find $\frac{231}{8}
\approx \frac{175\sqrt{2}}{32}\alpha$, so $\alpha = \frac{462\sqrt{2}}{175}
\approx 3.7335$. This is all verified numerically.

Now, what about when $\delta e \equiv 1 - e \ll 1$? Let's expand
\begin{align}
    2 \p*{1
            + \frac{15\p*{1 - 2\delta e}}{2}
            + \frac{45\p*{1 - 4\delta e}}{8}
            + \frac{5\p*{1 - 6\delta e}}{16}}
        &\approx
            \frac{5\alpha \sqrt{2}}{4} \p*{1 - \frac{\delta e}{4}} \p*{
                    1 + 3\p*{1 - 2\delta e} + \frac{3}{8}
                        \p*{1 - 4\delta e}
                },\\
    \frac{231}{8} - \frac{315}{4}\delta e
        &\approx \frac{5\alpha\sqrt{2}}{4}\p*{
            \frac{35}{8} - \frac{275}{32} \delta e}
            = \frac{231}{8} - \frac{1815}{32}\delta e
\end{align}
The leading order correction on the LHS evaluates to be $78.75$, while the LHS
is $56.71$. We could enforce exact agreement with this by relaxing $p = 2$, but
then we wouldn't be able to match $\sum\limits F_{N2}N^2$ anyway, so this seems
to be acceptable.

Of note: we only set $\alpha$ by setting the expressions equal at $e \to 1$, but
what about in general? Let's make a coarse plot, and of course it's linear and
not constant\dots see \autoref{fig:alpha}. It's approximately linear between
$\alpha(0) = 8/5$ and $\alpha(1) \approx 3.7335$.
\begin{figure}[h]
    \centering
    \includegraphics[width=0.6\textwidth]{../scripts/eccentric_tides/check_alpha.png}
    \caption{Plot of $\alpha(e)$, red dots is approximation}\label{fig:alpha}
\end{figure}
We should probably just generally use the exact value of $\alpha(e)$ then; it
only changes $\eta$ and not $C$ or $p = 2$, so it's not too bad.

Note that the power of our approximation comes in evaluating expressions as
follows (recall $\alpha \approx 2 (1 + e)$)
\begin{align}
    \sum\limits_{N = -\infty}^\infty F_{N2}^2 N^q
        &\approx \int\limits_0^\infty C^2 N^4 e^{-2 N / \eta}N^q
            \;\mathrm{d}N,\\
        &\approx C^2 \int\limits_{0}^\infty N^{4 + q}
            e^{-2N / \eta}\;\mathrm{d}N,\\
        &\approx C^2 \p*{\frac{\eta}{2}}^{5 + q}\Gamma\p*{5 + q},\\
        &\approx \frac{1 + 3e^2 + 3e^4/8}{\p*{1 - e^2}^{9/2}}
            \frac{\Gamma\p*{5 + q}}{4!}\p*{\frac{\eta}{2}}^q,\\
        &\approx \frac{1 + 3e^2 + 3e^4/8}{\p*{1 - e^2}^{9/2}}
            \frac{\Gamma\p*{5 + q}}{4!}\p*{\alpha\frac{\sqrt{1 + e}}{
                4\p*{1 - e^2}^{3/2}}}^q,\\
        &\approx \frac{f_5}{\p*{1 - e^2}^{(9 + 3q) / 2}}
            \frac{\Gamma\p*{5 + q}}{4!}\p*{\frac{\alpha}{4}}^q
            \p*{1 + e}^{q/2}.\label{eq:fn2_swiss_army_knife}
\end{align}

\subsection{Also Updating $F_{N0}$}

Before, we postulated that $F_{N0} \propto e^{-\sqrt{2}\abs*{N} / N_{peri}}$,
but we can do better now, with updated knowledge. Note that
\begin{equation}
    \sum\limits_N F_{Nm}^2 = \frac{1}{T}\int_0^T
        \frac{a^3}{D^3}e^{-imf}\frac{a^3}{D^3} e^{+imf}\;\mathrm{d}t.
\end{equation}
This makes it obvious that $\sum\limits_N F_{N2}^2 = \sum\limits_N F_{N0}^2 =
f_5 / \p*{1 - e^2}^{9/2}$. Taking our fitting formula to be of form $F_{N0} =
Ce^{-\abs*{N} / \eta}$, we find
\begin{equation}
    \int\limits_{-\infty}^\infty F_{N0}^2\;\mathrm{d}N
        = C^2\p*{\frac{\eta}{2}}.
\end{equation}

Furthermore, the first moment is also easier to write down
\begin{align}
    \sum\limits_N F_{N0}^2N
        &= \frac{1}{T} \int\limits \frac{a^3}{D^3}
            \p*{\rd{}{\p*{-i \Omega t}} \frac{a^3}{D^3}}\;\mathrm{d}t,\\
        &= \frac{1}{2\pi}\int\limits_0^{2\pi}
            \frac{a^6}{D^6} \frac{3e\sin f}{1 + e \cos f} \;\mathrm{d}f.
\end{align}
This vanishes, per our earlier analysis, and this is no suprise, since we found
that $F_{N0}$ are symmetric, so there can be no first moment. The second moment
is easiest to compute via
\begin{align}
    \sum\limits_N F_{N0}^2N^2
        &= \frac{1}{2\pi}\int\limits_0^{2\pi}
            \frac{a^6}{D^6} \p*{\frac{3e\sin f}{1 + e \cos f}}^2
                \rd{f}{t}\;\mathrm{d}f
                ,\\
        &= \frac{1}{2\pi\p*{1 - e^2}^6}\int\limits_0^{2\pi}
            \p*{1 + e\cos f}^4\p*{3e\sin f}^2
                \frac{\p*{1 + e \cos f}^2}{\p*{1 - e^2}^{3/2}}\;\mathrm{d}f,\\
        &= \frac{1}{2\pi\p*{1 - e^2}^{15/2}}\int\limits_0^{2\pi}
            \p*{1 + 15e^2\cos^2f + 15e^4\cos^4f
                + e^6\cos^6 f}\p*{3e\sin f}^2\;\mathrm{d}f,\\
        &= \frac{9e^2}{\p*{1 - e^2}^{15/2}}\p*{
            \frac{1}{2} + \frac{15e^2}{8} + \frac{15e^4}{16}
                + \frac{5e^6}{128}}
            \equiv \frac{9e^2}{2\p*{1 - e^2}^{15/2}}f_3.
\end{align}
I just Mathematica'd the trig identities, nothing magical going on. This is
numerically checked and in agreement w/ Michelle's result as well.

We can also look at the second moment of our fitting formula. Integrating twice
by parts lets us evaluate
\begin{align}
    \int\limits_{-\infty}^\infty C^2e^{-2\abs*{N} / \eta} N^2\;\mathrm{d}N
        &= 2C^2\int\limits_0^\infty e^{-2N/\eta}N^2\;\mathrm{d}N,\\
        &= 4C^2\p*{\frac{\eta}{2}}^3.
\end{align}
The latter follows from the gamma function definition of course (didn't see this
at first lol).

We now have two constraints (0th and 2nd moment of $F_{N0}$) for the two
variables $C, \eta$, and therefore we find that
\begin{align}
    C^2\eta &= f_5 / \p*{1 - e^2}^{9/2},\\
    C^2 \frac{\eta^3}{2} &= \frac{9e^2}{2\p*{1 - e^2}^{15/2}}f_3,\\
    \eta^2 &= \frac{9e^2f_3}{\p*{1 - e^2}^{3}f_5},\\
    C^2 &= \frac{f_5}{\eta\p*{1 - e^2}^{9/2}}.
\end{align}
We can verify that as $e \to 1$, we have the scaling $\eta \propto \p*{1 -
e^2}^{3/2}$, which probably is what told us $\eta \sim N_{peri}$.

In fact, in line with what we did above, let's assume $\eta \equiv \beta
N_{peri}$, then
\begin{equation}
    \beta = \frac{3e\sqrt{f_3/f_5}}{\sqrt{1 + e}}.
\end{equation}
The plot of $\beta$ is shown in \autoref{fig:check_beta}. It comes out to be
very nearly $\beta(e) \approx \sqrt{8}e$, which is much easier to remember +
use.
\begin{figure}[h]
    \centering
    \includegraphics[width=0.6\textwidth]{../scripts/eccentric_tides/check_beta.png}
    \caption{Plot of $\beta(e)$. Red dots is approximation.
    }\label{fig:check_beta}
\end{figure}

Thus, we may revise
\begin{equation}
    C^2 = \frac{2f_5}{\beta \sqrt{1 + e}\p*{1 - e^2}^3}.
\end{equation}

\subsection{New Closed Forms for Torque}

We can now update our torque formula. Recall we want to sum
\begin{equation}
    \tau = T_0 \hat{C}(r_c)
            \sum\limits_{N = -\infty}^\infty
                F_{N2}^2 \sgn\p*{N - 2\frac{\Omega_s}{\Omega}}
                    \abs*{N - 2\frac{\Omega_s}{\Omega}}^{8/3}.
\end{equation}
Note that $\hat{C}(r_c)$ was defined a while ago to be dimensionless quantity
\begin{equation}
    \hat{C}(r_c) \equiv \frac{1}{2}
        \s*{\frac{r_c}{g_c}\p*{\rd{N^2}{\ln R}}_{r = r_c}}^{-1/3}
            \frac{\rho_c}{\bar{\rho}_c} \p*{1 - \frac{\rho_c}{\bar{\rho}_c}}^2
            \s*{\frac{3}{2}\frac{3^{2/3}\Gamma^2(1/3)}{5 \cdot
                6^{4/3}} \frac{3}{4\pi}\alpha^2}.
\end{equation}
We can drop everything but the $1/2$ and the density terms under Kushnir's
suggestion.

Let's just consider the summation, which we can call $\hat{\tau}$
\begin{equation}
    \hat{\tau} \equiv \sum\limits_{N = -\infty}^\infty
        F_{N2}^2 \sgn\p*{N - 2\frac{\Omega_s}{\Omega}}
            \abs*{N - 2\frac{\Omega_s}{\Omega}}^{8/3}.
\end{equation}
Again, let's consider the two limits (recall $N_{\max} = \alpha \frac{\sqrt{1 +
e}}{\p*{1 - e}^{3/2}}$, while $f_5 = 1 + 3e^2 + 3e^4/8$):
\begin{itemize}
    \item Consider $2\Omega_s \gg N_{\max}\Omega$, then the sign is always
        negative and $\abs*{N - 2\Omega_s/\Omega} \approx
        \abs*{\frac{2\Omega_s}{\Omega} - N_{\max}}$ and
        \begin{align}
            \hat{\tau}
                &\approx -\abs*{\frac{2\Omega_s}{\Omega} - N_{\max}}^{8/3}
                    \sum\limits_{N = -\infty}^\infty F_{N2}^2,\\
                &\approx -\abs*{1 - \frac{2\Omega_s}{N_{\max}\Omega}}^{8/3}
                    \frac{f_5}{\p*{1 - e^2}^{9/2}}\p*{\alpha\p*{\frac{1 + e}{\p*{1 -
                    e^2}^3}}^{1/2}}^{8/3},\\
                &\approx -\abs*{1 - \frac{2\Omega_s}{N_{\max}\Omega}}^{8/3}
                    \frac{f_5\alpha^{8/3}\p*{1 + e}^{4/3}}{\p*{1 - e^2}^{17/2}}.
        \end{align}

    \item Consider $2\Omega_s \ll N_{\max} \Omega$, then the sign is the
        sign of $N_{\max}$ ($>0$). We choose a factorization for $\abs*{N -
        \frac{2\Omega_s}{\Omega}} \approx \abs*{N}\abs*{1 -
        \frac{2\Omega_s}{\beta N_{\max}\Omega}}$ such that we can express as a
        prefactor and a summation. We must choose it such that in the limit
        $\Omega_s \to -\infty$, the asymptotics are correct, that is:
        \begin{equation}
            \sum\limits_{N} F_{N2}^2 \abs*{2\Omega_s}^{8/3}
                = \abs*{\frac{2\Omega_s}{\beta N_{\max}\Omega}}^{8/3}
                    \sum\limits_{N}F_{N2}^2 N^{8/3}.
        \end{equation}

        This can be solved to find (I did this on scratch paper, see the
        calculation in heating for a solved example) that $\beta =
        \p*{\frac{\Gamma\p*{23/3}}{4!}}^{3/8}/4 \approx 1.447$, and so $2/\beta
        \approx 1.38$. Thus,
        \begin{align}
            \hat{\tau}
                &\approx \abs*{1 - 1.38\frac{\Omega_s}{N_{\max}\Omega}}^{8/3}
                    \sum\limits_{N = -\infty}^\infty F_{N2}^2 \abs*{N}^{8/3},\\
                &\approx \abs*{1 - 1.38\frac{\Omega_s}{N_{\max}\Omega}}^{8/3}
                    \frac{f_5}{\p*{1 - e^2}^{17/2}}
                    \frac{\Gamma(23/3)}{4!}\p*{\frac{\alpha}{4}}^{8/3}
                        \p*{1 + e}^{4/3}.
        \end{align}

        This approach is actually literally the same thing,
\end{itemize}
Thus, we arrive at final answer
\begin{equation}
    \hat{\tau} \approx \alpha^{8/3}
        \frac{f_5\p*{1 + e}^{4/3}}{\p*{1 - e^2}^{17/2}} \times
    \begin{cases}
        \abs*{1 - 1.38\frac{\Omega_s}{N_{\max}\Omega}}^{8/3}
            \frac{\Gamma(23/3)}{4!}\p*{\frac{1}{4}}^{8/3},
            & \Omega_s < N_{\max}\Omega / 2,\\[5pt]
        -\abs*{1 - 2\frac{\Omega_s}{N_{\max}\Omega}}^{8/3},
            & \Omega_s > N_{\max}\Omega / 2.
    \end{cases}\label{eq:traveling_torque}
\end{equation}

Note that the reason we have these two different forms is basically because I
didn't want three components to the piecewise function. But it should be obvious
that the two regimes $\abs*{\Omega_s} \gg \abs*{N_{\max}\Omega}$ should be
symmetric, and this is what fixes the prefactor on the $\Omega_s /
N_{\max}\Omega$ term inside the absolute value\dots no need to fully compute the
asymptotic behavior.

\subsection{And Heating}

In the case of the heating, we recall
\begin{align}
     \dot{E}_{in} = \frac{1}{2}\hat{T}\p*{r_c, \Omega}\s*{
         \sum\limits_{N = -\infty}^\infty
            N\Omega F_{N2}^2 \sgn \p*{\sigma} \abs*{\sigma}^{8/3}
            + \p*{\frac{W_{20}}{W_{22}}}^2\Omega F_{N0}^2 \abs*{N}^{11/3}},
\end{align}
where we write $\sigma = N - \frac{2\Omega_s}{\Omega}$ and know $\p*{W_{20} /
W_{22}}^2 = 2/3$.

Let's handle the second term first, since it's on our minds; subscript
\begin{align}
    \frac{\dot{E}_2}{\hat{T}\Omega}
        &= \frac{1}{3} \sum\limits_{N = -\infty}^\infty
            F_{N0}^2\abs*{N}^{11/3},\\
        &= \frac{2}{3}\int\limits_0^\infty C^2e^{-2N/\eta}N^{11/3}\;\mathrm{d}N,\\
        &= \frac{2C^2}{3}\p*{\frac{\eta}{2}}^{14/3}\Gamma(14/3),\\
        &= \frac{2}{3}\frac{f_5}{\beta\sqrt{1 + e}\p*{1 - e^2}^3}
            \p*{\frac{\beta}{2}}^{14/3}\p*{\frac{\sqrt{1 + e}}{\p*{1 -
                e^2}^{3/2}}}^{14/3} \Gamma(14/3),\\
        &= \frac{1}{2^{11/3}}
            \frac{f_5 \beta^{11/3}\p*{1 + e}^{11/6}}{3\p*{1 - e^2}^{10}}
            \Gamma\p*{14/3}.
\end{align}

The first term is the same procedure as for the torque, where we must expand and
match consistency. The basic form is
\begin{equation}
    \frac{\dot{E}_1}{\hat{T}\Omega} = \frac{1}{2}
        \sum\limits_{N = -\infty}^\infty NF_{N2}^2 \mathrm{sgn}(\sigma)
            \abs*{\sigma}^{8/3}.
\end{equation}
Recall $\sigma \equiv N - \frac{2\Omega_s}{\Omega}$. Thus, we calculate:
\begin{itemize}
    \item In the limit where $\Omega_s \gg N_{\max}\Omega$, then the sign is
        always negative and $\abs*{N - 2\Omega_s / \Omega} \approx
        \abs*{2\Omega_s / \Omega - N_{\max}}$ and we have straightforwardly
        \begin{align}
            \frac{\dot{E}_1}{\hat{T}\Omega}
                &= -\frac{1}{2}
                    \abs*{\frac{2\Omega_s}{\Omega} - N_{\max}}^{8/3}
                    \sum\limits_{N = -\infty}^\infty NF_{N2}^2,\\
                &= -\frac{1}{2}
                    \abs*{1 - \frac{2\Omega_s}{N_{\max}\Omega}}^{8/3}
                    N_{\max}^{8/3}
                    \frac{f_5}{\p*{1 - e^2}^6}\frac{5\sqrt{1 + e}}{4}\alpha,\\
                &= -\frac{1}{2}
                    \abs*{1 - \frac{2\Omega_s}{N_{\max}\Omega}}^{8/3}
                    \alpha^{11/3}
                    \frac{f_5}{\p*{1 - e^2}^{10}}\frac{5\p*{1 + e}^{11/6}}{4}.
        \end{align}
    \item Again, in the other limit where $2\Omega_s \ll N_{\max}\Omega$, we
        seek some factorization + summation $\abs*{N - \frac{2\Omega_s}{\Omega}}
        \approx \abs*{N}\abs*{1 - \frac{2\Omega_s}{\beta N_{\max}\Omega}}$ such
        that the asymptotics are correct:
        \begin{equation}
            \abs*{\frac{2\Omega_s}{\Omega}}^{8/3} \sum\limits_N F_{N2}^2 N
                = \abs*{\frac{2\Omega_s}{\beta N_{\max}\Omega}}^{8/3}
                    \sum\limits_N F_{N2}N^{11/3}.
        \end{equation}
        Using \autoref{eq:fn2_swiss_army_knife} and keeping only the $q$
        dependent parts, we obtain (recall $N_{\max} = \alpha \frac{\sqrt{1 +
        e}}{\p*{1 - e}^{3/2}}$)
        \begin{align}
            \abs*{\frac{2\Omega_s}{\Omega}}^{8/3} \frac{\Gamma(6)}{\p*{1 -
                    e^2}^{3/2}}
                    \p*{\frac{\alpha}{4}}\p*{1 + e}^{1/2}
                &= \abs*{\frac{2\Omega_s}{\beta N_{\max}\Omega}}^{8/3}
                    \frac{\Gamma(26/3)}{\p*{1 - e^2}^{11/2}}
                    \p*{\frac{\alpha}{4}}^{11/3}\p*{1 + e}^{11/6},\\
            \beta^{8/3}
                &= \frac{1}{N_{\max}}^{8/3}
                    \frac{\Gamma(26/3)}{5! \p*{1 - e^2}^{4}}
                    \p*{\frac{\alpha}{4}}^{8/3}\p*{1 + e}^{4/3},\\
            \beta
                &= \p*{\frac{\Gamma(26/3)}{5!}}^{3/8} \p*{\frac{1}{4}}
                    \approx 1.699.
        \end{align}
        Again, as was noted above, this number had to be obvious, but I'm
        stupid.

        Thus, we can now approximate the sum
        \begin{align}
            \frac{\dot{E}_1}{\hat{T}\Omega}
                &= \frac{1}{2}
                    \abs*{1 - 1.1772\frac{\Omega_s}{\Omega N_{\max}}}^{8/3}
                    \sum\limits_{N = -\infty}^\infty \abs*{N}^{11/3}F_{N2}^2,\\
                &= \frac{1}{2}
                    \abs*{1 - 1.1772\frac{\Omega_s}{N_{\max}\Omega}}^{8/3}
                    \frac{f_5}{\p*{1 - e^2}^{10}}\frac{5\p*{1 + e}^{11/6}}{4}
                    \p*{\frac{\alpha}{4}}^{11/3}
                    \frac{\Gamma(26/3)}{5!}.
        \end{align}
\end{itemize}
Thus, we have total heating rate (in the inertial frame, this indeed can be
signed)
\begin{align}
    \frac{\dot{E}_{in}}{\hat{T}\Omega} ={}&
            \frac{1}{2^{11/3}}
            \frac{f_5 \beta^{11/3}\p*{1 + e}^{11/6}}{3\p*{1 - e^2}^{10}}
            \Gamma\p*{14/3}\nonumber\\
        & + \frac{\alpha^{11/3}}{2}
                \frac{f_5\p*{1 + e}^{11/6}}{\p*{1 - e^2}^{10}}
                \times\nonumber\\
        &\begin{cases}
            \abs*{1 - 1.1772\frac{\Omega_s}{N_{\max}\Omega}}^{8/3}
                \frac{\Gamma(26/3)}{5!}\frac{1}{4^{8/3}},
                & \Omega_s < N_{\max}\Omega/2,\\[5pt]
            -\abs*{1 - 2\frac{\Omega_s}{N_{\max}\Omega}}^{8/3}\frac{5}{4},
                & \Omega_s > N_{\max}\Omega/2.
        \end{cases}\label{eq:total_heating}
\end{align}

\subsection{Being Brave and Useless: Expanding Non-integer Powers}

Let's just drop $\alpha \approx 4$ for the time being, so
\begin{equation}
    \sum\limits_{N = -\infty}^\infty F_{N2}^2 N^q
        \approx \frac{f_5}{\p*{1 - e^2}^{(9 + 3q) / 2}}
            \frac{\Gamma\p*{5 + q}}{4!} \p*{1 + e}^{q/2}.
\end{equation}
One fun question is: can we do better than the piecewise approximation that we
proposed above for the $N = 8/3$ case, via the generalized binomial theorem? For
reference, this is:
\begin{equation}
    \p*{a + b}^n = \sum\limits_{k = 0}^\infty \binom{n}{k}a^kb^{n - k},
\end{equation}
where $\binom{n}{k} \equiv \frac{n(n - 1)\dots (n - k + 1)}{k!}$ for integer $k$
and not necessarily integer $n$ (and thus, not necessarily positive $n - k +
1$).

Let's drop all the signs and absolute value nonsense, in which case we are
attempting to evaluate (where $m' \equiv m\Omega_s / \Omega$ for convenience)
\begin{align}
    \smiley
        &\equiv \sum\limits_{N = -\infty}^\infty F_{N2}^2 \p*{N - m'}^{8/3}
            ,\\
        &= \sum\limits_{N = -\infty}^\infty
            F_{N2}^2\p*{N^{8/3} + \frac{8}{3}N^{5/3}m'
                + \frac{(8/3)(5/3)}{2}N^{2/3}(m')^2 + \dots},\\
        &= \frac{f_5}{\p*{1 - e^2}^{9/2}4!}\s*{
            \frac{\Gamma\p*{5 + 8/3}\p*{1 + e}^{(8/3) / 2}}{\p*{1 - e^2}^{8/2}}
                + \frac{\frac{8}{3}\Gamma\p*{5 + 5/3}\p*{1 + e}^{(5/3) / 2}m'}{
                    \p*{1 - e^2}^{5/2}}
                + \frac{\frac{(8/3)(5/3)}{2}\Gamma\p*{5 + 2/3}\p*{1 + e}^{(2/3)
                    / 2} (m')^2}{\p*{1 - e^2}^{2/2}}
                + \dots}.
\end{align}
This distinction only makes sense obviously when $m' \approx N_{\max} \sim
\frac{2\sqrt{1 + e}}{\p*{1 - e^2}^{3/2}}$, so let's go ahead and set $m' =
N_{\max}\beta$; we want eventually to examine asymptotic behavior for $\beta
\to^{\pm} 1$ or some similarly critical value. This gives
\begin{align}
    \smiley
        &= \frac{f_5}{\p*{1 - e^2}^{9/2}4!}\sum\limits_{k = 0}^\infty
            \binom{8/3}{k}\Gamma\p*{\frac{23}{3} - k}
                \frac{\p*{1 + e}^{(8/3 - k) / 2}}{
                    \p*{1 - e^2}^{(8 - 3k)/2}}
                \p*{-2\beta}^k\frac{\p*{1 + e}^{k / 2}}{\p*{1 - e^2}^{3k / 2}},\\
        &= \frac{f_5\p*{1 + e}^{8/3}}{\p*{1 - e^2}^{17/2}4!}
            \sum\limits_{k = 0}^\infty
            \binom{8/3}{k}\Gamma\p*{\frac{23}{3} - k}\p*{-2\beta}^k.
\end{align}
The trick to proceed here is to handle negative arguments to the Gamma function
with the Euler reflection formula $\Gamma(z) \Gamma\p*{1 - z} = \pi/\sin\p*{\pi
z}$. Dropping the first few terms that don't have this problem (they're finite,
we can return to them), we then have
\begin{align}
    \smiley - \smiley_6
        &= \frac{f_5\p*{1 + e}^{8/3}}{\p*{1 - e^2}^{17/2}4!}
            \sum\limits_{k = 7}^\infty \binom{8/3}{k}
            \frac{\pi}{\sin \p*{\pi\p*{23/3 - k}}
                \Gamma\p*{k - 20/3}}\p*{-2\beta}^k,\\
        &= \frac{f_5\p*{1 + e}^{8/3}}{\p*{1 - e^2}^{17/2}4!}
            \sum\limits_{k = 7}^\infty \binom{8/3}{k}
            \frac{\pi}{-\sqrt{3}/2\p*{-1}^k
                \Gamma\p*{k - 20/3}}\p*{-2\beta}^k,\\
        &= -\frac{f_5\p*{1 + e}^{8/3}}{\p*{1 - e^2}^{17/2}4!}
            \sum\limits_{k = 7}^\infty \binom{8/3}{k}
            \frac{2\pi/\sqrt{3}}{\Gamma\p*{k - 20/3}}\p*{2\beta}^k.
\end{align}
Asymptotically, $\binom{8/3}{k}$ flips signs for every $k$, but each successive
term is being multiplied by a number that gets closer to unity. The convergence
is dominated by the gamma function in the denominator, which scales like a
factorial. The sum thus converges to something in the vicinity of
\begin{align}
    \sum\limits_{k = 7}^\infty \binom{8/3}{k}
            \frac{2\pi/\sqrt{3}}{\Gamma\p*{k - 20/3}}\p*{2\beta}^k
        &\lesssim 2\pi/\sqrt{3}(2\beta)^7 e^{-2\beta}.
\end{align}
Is this enlightening at all? Well, there is some critical $\beta_c \sim 1$
(corresponding to some critical $\Omega_{s,c}$) for which $\smiley$ and thus the
torque is zero, the pseudosynchronizated spin rate. For $\beta$ just below/above
this $\beta_c$, I guess we can see that the torque grows exponentially, before
the dominant power law behavior takes over, from the piecewise approximations.
And this was thoroughly useless, albeit pretty fun. Note that the signs do work
out, a pretty big surprise to me at least.

The original hope would have been to determine how near pseudosynchronization
our piecewise approximation breaks down, but this seems somewhat hard. We
probably can guesstimate by looking at the values of $\beta$ for which the above
is either half of its maximum value. We know it is maximized for $\beta \sim
3.5$, and the half max apparently happens around $2.2$ or $5.2$, so our power
law approximation should only be good to within about $70\%$ of the
pseudosynchronization frequency.

\subsection{Discrete Modes}

This is very tenuous, let's give it a shot. When we have discrete modes (cf.
Michelle's paper 2019, VL19), we generally have torques of form
\begin{equation}
    \tau \propto \sum\limits_{\alpha N}
        \frac{\gamma F_{Nm}^2 \omega_{Nm}^2}{
            \p*{\omega_\alpha - \omega_{Nm}}^2 + \Gamma_\alpha^2}.
\end{equation}

Consider just the torque on a single mode for now, so take the sum only over $N$
not $\alpha$. Let's furthermore assume that the eccentricity is sufficiently
large such that $\Omega$ is much slower than all other scales of interest, and
we have a continuum $N$ of driving frequencies. Then ($\hat{\tau}_\alpha$ drops
all proportionality factors)
\begin{equation}
    \tau_\alpha \approx \int\limits_{-\infty}^\infty
        \frac{\gamma F_{Nm}^2 \omega_{Nm}^2}{
            \p*{\omega_\alpha - \omega_{Nm}}^2 + \Gamma_\alpha^2}\;\mathrm{d}N.
\end{equation}
Note that $\omega_{Nm} \equiv N\Omega - m\Omega_s$.

If we further assume that $\Gamma_{\alpha}$ is small, the Lorentzion in the
denominator might be approximated by a delta function. Its height is fixed by
normalization
\begin{align}
    \int\limits_{-\infty}^\infty \frac{1}{\p*{\omega_\alpha - \omega_{Nm}}^2
            + \Gamma_\alpha^2}\;\mathrm{d}\omega_{Nm}
        &= \int\limits_{-\infty}^\infty \frac{1}{\Gamma}
            \frac{1}{(x/\Gamma)^2 + 1}\;\mathrm{d}(x/\Gamma),\\
        &= \frac{\pi}{\Gamma},\\
    \frac{1}{\p*{\omega_\alpha - \omega_{Nm}}^2 + \Gamma_\alpha^2}
        &\approx \frac{\pi}{\Gamma_\alpha}\delta\p*{\omega_\alpha - \omega_{Nm}}
            ,\\
        &= \frac{\delta\p*{N - N_\alpha}\pi}{\Gamma_\alpha\Omega},
\end{align}
where $\omega_\alpha - (N_\alpha \Omega - m\Omega_s) = 0$. Thus, we are able to
turn the Lorentzion into a delta function in terms of $N$, which allows us to
write
\begin{equation}
    \tau_\alpha \approx \s*{\gamma F_{Nm}^2 \omega_{Nm}^2}_{N = N_\alpha}
        \frac{\pi}{\Gamma_\alpha \Omega}.
\end{equation}
With our approximation for $F_{Nm}^2$, we can easily evaluate for any arbitrary
$N$, so maybe given a particular spectrum of modes $\omega_\alpha$ it would be
possible to sum $\tau$ in closed form? Seems somewhat unlikely.

In an alternative world, when $\Gamma$ is much \emph{larger} than
$\Omega_{peri}$, the Lorentzian is fat and we can probably just set it equal to
its peak value $1/\Gamma^2$ and perform the integration as usual?

\section{J0045--7319}

Let's try to plug in some realistic numbers. Consider J0045+7319, which has
parameters $e = 0.808$, $M_t = 10.2M_{\odot}$, $a = 126R_{\odot}$, $\Omega_o =
\scinot{1.42}{-6}\;\mathrm{rad/s}$, $\Omega_p/2\pi = 3.61\upmu\mathrm{Hz}$. We
can assume the mass of one star is $1.4M_{\odot}$, the NS, so the other is
$8.8M_{\odot}$.

Some fiducial numbers from KQ include that the B star has RCB at $0.23R_*$,
where the star is taken to be roughly $6.4R_{\odot}$, and the mass of the
convective core is $3M_{\odot}$, the density and BV frequency outside the core
ar $2\;\mathrm{g/cm^3}$ and $100\upmu\mathrm{Hz}$ respectively. Finally, the
star is observed to have $\frac{\dot{a}}{a} \sim \scinot{2}{-6}\;\mathrm{/yr}$.

Let's first evaluate $\hat{T}$, the circular orbit limit:
\begin{equation}
    \hat{T}(r_c, \omega) \equiv \frac{GM_2^2r_c^5}{a^6}
        \p*{\frac{\omega}{\sqrt{GM_c/r_c^3}}}^{8/3}
        \s*{\frac{r_c}{g_c}\p*{\rd{N^2}{\ln r}}_{r = r_c}}^{-1/3}
            \frac{\rho_c}{\bar{\rho}_c} \p*{1 - \frac{\rho_c}{\bar{\rho}_c}}^2
            \s*{\frac{3}{2}\frac{3^{2/3}\Gamma^2(1/3)}{5 \cdot
                6^{4/3}} \frac{3}{4\pi}\alpha^2}.
\end{equation}
Let's take Kushnir's prescription and set $\beta_2 = 1$, so $\hat{T}$ is now
\begin{equation}
    \hat{T}(r_c, \omega) \approx \beta_2\frac{GM_2^2r_c^5}{a^6}
        \p*{\frac{\omega}{\sqrt{GM_c/r_c^3}}}^{8/3}
            \frac{\rho_c}{\bar{\rho}_c} \p*{1 - \frac{\rho_c}{\bar{\rho}_c}}^2.
\end{equation}
We evaluate in pieces:
\begin{itemize}
% 0.23 * 6.4 * solar radius = 1.024e11 cm
% G * (1.4 solar mass)^2 * (1.024e11 cm)^5 / (126 * solar radius)^6
    \item Note that $\frac{GM_2^2 r_c^5}{a^6} = \scinot{1.284}{30}\;\mathrm{N
        \cdot m}$ for the given parameters.

% (1.42 * 10^{-6} (Hz) / sqrt(G * 3 * solar mass / (1.024e11 cm)^3))^(8/3)
    \item Then evaluating $\hat{T}$ at the orbital frequency, we can evaluate
        the $\omega$ term, which is $\scinot{9.563}{-8}$.

% 0.76 * (1 - 0.76)^2
    \item Let's just take $\rho_c / \bar{\rho}_c \sim 0.76$, then we can evaluate
        the final piece $0.0438$.
\end{itemize}
This tells us that $\hat{T} \approx \scinot{5.38}{21}\;\mathrm{N \cdot m}$, and
so
% G * (8.8 solar masses) * (1.4 solar masses) / (2 * 126 * solar radius)
\begin{align}
    % (9.563e-8 * 0.0438 * 1.284e30) * (Newton * m) (1.42e-6 Hz)
    \hat{T}\Omega &\approx \scinot{7.636}{15}\;\mathrm{W},\\
    E &\approx \scinot{1.85}{40}\;\mathrm{J},\\
    \dot{E}_{obs} = -\frac{\dot{a}}{a}E &= \scinot{1.173}{27}\;\mathrm{W}.
\end{align}
% sqrt(G * 3 * solar mass / (1.024e11 cm)^3) / (1.42e-6 Hz)
Thus, we need an enhancement $\dot{E}_{obs} / \p*{\hat{T}\Omega} =
\scinot{1.536}{11}$. The breakup frequency in units of the orbital frequency is
$428.822 \Omega_o$, and it turns out the two values for the spin of the core
evaluate to be $-306 \Omega_o$ or $341 \Omega_o$ depending on whether retrograde
or prograde rotation is used. This is quite a large fraction of breakup!

For reference, the heating in the inertial frame for these two values (given
$\dot{E}_{\rm rot} = \dot{E}_{\rm in} - \Omega_s \tau$) is $\sim \scinot{4}{12}
\hat{T}\Omega = \scinot{3}{28}\;\mathrm{W} = 80L_{\odot}$. This is still
somewhat negligible compared to the luminosity of the host star\dots

Let's try running this via MESA\@. It looks like the metallicity should be
somewhere around $0.1\times$ or $0.2\times$
solar\footnote{https://academic.oup.com/mnras/article/422/2/1109/1032973\#18429474}.
Hilariously, the old mass estimate comes from requiring a $1.4M_{\odot}$ NS,
which is not accurate; Thorsett \& Chakrabatty 1999 give a new estimate $\sim 10
M_{\odot}$. We want a star of the correct metallicity to reproduce the only
observational constraints, $L \sim \scinot{1.2}{4}L_{\odot}$ and $T \sim 24000
\pm 100\;\mathrm{K}$. This seems to be hard to satisfy, my original MESA
simulations have higher temperatures and lower luminosities, at least on the MS
down to a central hydrogen abundance $\sim 0.5$. This seems to suggest possible
signatures of tidal heating, which could puff up the star (lowering the Teff and
raising the luminosity), but this seems somewhat unlikely from the $\dot{E}_{\rm
rot}$ calculation using the OG values above. See plots for comparison of MESA
data. % TODO

Taking $M = 11.4M_{\odot}$ (which matches at least the luminosity) and companion
mass $1.81 M_{\odot}$, we can plug in some values
% G * (1.81 solar mass)^2 * (5.69e10 cm)^5 / (126 solar radii)^6
$GM_2^2 r_c^5 / a^6 = \scinot{1.137}{29}\;\mathrm{N \cdot m}$,
% (1.42 * 10^{-6} (Hz) / sqrt(G * 3.05 * solar mass / (5.69e10 cm)^3))^(8/3)
$\omega^{8/3} \approx \scinot{8.9181}{-9}$, and
$\rho_c / \bar{\rho}_c \sim 0.742$ so the final piece comes out $0.0494$. This
again gives substantially lower
\begin{equation}
    % 7.992e28 (Newton * meter) * 8.190e-9 * 0.0459 * (1.42e-6 Hz)
    \hat{T}\Omega \approx \scinot{7.113}{13}\;\mathrm{W}.
\end{equation}
% sqrt(G * 3.05 * solar mass / (5.69e10 cm)^3) / (1.42 * 10^{-6} (Hz))
The new breakup frequency is $\Omega_* / \Omega_o = 1043.87$. The new required
enhancement $\dot{E}_{\rm obs} / \hat{T}\Omega = \scinot{1.649}{13}$. The core
must rotate faster than breakup for this to be the case though!

\end{document}

