    \documentclass[11pt,
        usenames, % allows access to some tikz colors
        dvipsnames % more colors: https://en.wikibooks.org/wiki/LaTeX/Colors
    ]{article}
    \usepackage{
        amsmath,
        amssymb,
        fouriernc, % fourier font w/ new century book
        fancyhdr, % page styling
        lastpage, % footer fanciness
        hyperref, % various links
        setspace, % line spacing
        amsthm, % newtheorem and proof environment
        mathtools, % \Aboxed for boxing inside aligns, among others
        float, % Allow [H] figure env alignment
        enumerate, % Allow custom enumerate numbering
        graphicx, % allow includegraphics with more filetypes
        wasysym, % \smiley!
        upgreek, % \upmu for \mum macro
        listings, % writing TrueType fonts and including code prettily
        tikz, % drawing things
        booktabs, % \bottomrule instead of hline apparently
        cancel % can cancel things out!
    }
    \usepackage[margin=1in]{geometry} % page geometry
    \usepackage[
        labelfont=bf, % caption names are labeled in bold
        font=scriptsize % smaller font for captions
    ]{caption}
    \usepackage[font=scriptsize]{subcaption} % subfigures

    \newcommand*{\scinot}[2]{#1\times10^{#2}}
    \newcommand*{\dotp}[2]{\left<#1\,\middle|\,#2\right>}
    \newcommand*{\rd}[2]{\frac{\mathrm{d}#1}{\mathrm{d}#2}}
    \newcommand*{\pd}[2]{\frac{\partial#1}{\partial#2}}
    \newcommand*{\rtd}[2]{\frac{\mathrm{d}^2#1}{\mathrm{d}#2^2}}
    \newcommand*{\ptd}[2]{\frac{\partial^2 #1}{\partial#2^2}}
    \newcommand*{\md}[2]{\frac{\mathrm{D}#1}{\mathrm{D}#2}}
    \newcommand*{\pvec}[1]{\vec{#1}^{\,\prime}}
    \newcommand*{\svec}[1]{\vec{#1}\;\!}
    \newcommand*{\bm}[1]{\boldsymbol{\mathbf{#1}}}
    \newcommand*{\ang}[0]{\;\text{\AA}}
    \newcommand*{\mum}[0]{\;\upmu \mathrm{m}}
    \newcommand*{\at}[1]{\left.#1\right|}

    \newtheorem{theorem}{Theorem}[section]

    \let\Re\undefined
    \let\Im\undefined
    \DeclareMathOperator{\Res}{Res}
    \DeclareMathOperator{\Re}{Re}
    \DeclareMathOperator{\Im}{Im}
    \DeclareMathOperator{\Log}{Log}
    \DeclareMathOperator{\Arg}{Arg}
    \DeclareMathOperator{\Tr}{Tr}
    \DeclareMathOperator{\E}{E}
    \DeclareMathOperator{\Var}{Var}
    \DeclareMathOperator*{\argmin}{argmin}
    \DeclareMathOperator*{\argmax}{argmax}
    \DeclareMathOperator{\sgn}{sgn}
    \DeclareMathOperator{\diag}{diag\;}

    \DeclarePairedDelimiter\bra{\langle}{\rvert}
    \DeclarePairedDelimiter\ket{\lvert}{\rangle}
    \DeclarePairedDelimiter\abs{\lvert}{\rvert}
    \DeclarePairedDelimiter\ev{\langle}{\rangle}
    \DeclarePairedDelimiter\p{\lparen}{\rparen}
    \DeclarePairedDelimiter\s{\lbrack}{\rbrack}
    \DeclarePairedDelimiter\z{\lbrace}{\rbrace}

    % \everymath{\displaystyle} % biggify limits of inline sums and integrals
    \tikzstyle{circ} % usage: \node[circ, placement] (label) {text};
        = [draw, circle, fill=white, node distance=3cm, minimum height=2em]
    \definecolor{commentgreen}{rgb}{0,0.6,0}
    \lstset{
        basicstyle=\ttfamily\footnotesize,
        frame=single,
        numbers=left,
        showstringspaces=false,
        keywordstyle=\color{blue},
        stringstyle=\color{purple},
        commentstyle=\color{commentgreen},
        morecomment=[l][\color{magenta}]{\#}
    }

\begin{document}

\def\Snospace~{\S{}} % hack to remove the space left after autorefs
\renewcommand*{\sectionautorefname}{\Snospace}
\renewcommand*{\appendixautorefname}{\Snospace}
\renewcommand*{\figureautorefname}{Fig.}
\renewcommand*{\equationautorefname}{Eq.}
\renewcommand*{\tableautorefname}{Tab.}

\onehalfspacing

\pagestyle{fancy}
\rfoot{Yubo Su}
\rhead{}
\cfoot{\thepage/\pageref{LastPage}}

\title{Eccentric Tides}
\author{Yubo Su}

\maketitle

\section{Kushnir et.\ al., 2016}

We coarsely follow the derivation of Kushnir et.\ al., 2016 (KZ) to express the
traveling wave regime of dynamical tides in high-mass stars (convective core,
radiative envelope) in analytical form.

\subsection{Plane Parallel Case}

We will consider IGW in a plane parallel atmosphere in the Boussinesq
approximation. Consider buoyancy frequency
\begin{equation}
    N^2 = -g\p*{\rd{\ln \rho}{r} + \frac{g}{c_s^2}},
\end{equation}
where $c_s \to \infty$ is the sound speed in the fluid. Then the Boussinesq
equations can be written in terms of some \emph{buoyancy} variable $b$
\begin{subequations}\label{eq:bouss}
    \begin{align}
        \md{\vec{u}}{t} &= \frac{\vec{\nabla}P}{\rho_0} + b\hat{z},\\
        \md{b}{t} &= -N^2 u_z,\\
        \vec{\nabla} \cdot \vec{u} &= 0.
    \end{align}
\end{subequations}
Note that $b \equiv -\frac{\rho'}{\rho_0} g$, as can be verified via
direct substitution into the Euler equations:
\begin{subequations}\label{eq:bouss}
    \begin{align}
        0 &= \md{\rho'}{t} + \vec{u} \cdot \vec{\nabla}\rho_0 =
            \md{\rho'}{t} - u_z \frac{N^2}{g} \rho',\\
        \md{\vec{u}}{t} &= \frac{\vec{\nabla}P'}{\rho_0}
                - \frac{\rho'}{\rho_0^2} \vec{\nabla} P_0
            = \frac{\vec{\nabla}P'}{\rho_0} - \frac{\rho'}{\rho_0}g\hat{z}.
    \end{align}
\end{subequations}

These equations can be solved for $u_z$, or we can just recall the IGW
dispersion relation $\omega^2 k^2 = N^2k_{\perp}^2$ and write down PDE
\begin{equation}
    \ptd{}{t} \nabla^2 u_z = -N^2\nabla_{\perp}^2 u_z.\label{eq:igw_pde}
\end{equation}

Now, we might recall that in tidally-forced stars, $\omega$ the tidal forcing
frequency obeys $\omega \ll N$, or $k \gg k_\perp$. But the tidal potential, the
quadrupolar expansion of the gravitational perturbation from the companion, has
no quickly-varying directions, or can only excite $k \simeq k_\perp$ modes.
Thus, we intuit that waves must be excited where $N$ is much smaller than its
typical value, or near the \emph{radiative-convective boundary} (RCB). At the
RCB, $N^2 = 0$, and we are concerned with the turning point where $\omega^2 =
N^2$. We perform linear expansion about this turning point $z_c$, and for
convenience we set $z_c = 0$, then
\begin{equation}
    N^2 \approx \omega^2 + \rd{N^2}{z}z,
\end{equation}
where compared to KZ I've taken $N_0^2 = \omega^2$, there seems to be little
harm here. Making then general ansatz $u_z(z, \vec{r}_\perp, t) = \tilde{u}_z(z)
e^{i\p*{\vec{k}_{\perp} \cdot \vec{r}_{\perp} - \omega t}}$ we obtain
\begin{align}
    -\omega^2 \p*{-k_\perp^2 \tilde{u}_z + \tilde{u}_z''}
            &= N^2k_\perp^2 u_z,\\
        \tilde{u}_z'' + k_\perp^2\p*{\frac{N^2}{\omega^2} - 1}\tilde{u}_z &= 0
            ,\\
        \tilde{u}_z'' + k_\perp^2\rd{N^2}{z}\frac{z}{\omega^2}\tilde{u}_z &= 0.
\end{align}
It's easiest now to rescale $\tilde{k}_\perp^2 \equiv k_\perp^23
\rd{N^2}{z}\frac{1}{\omega^2}$ so that
\begin{equation}
    \tilde{u}_z'' + \tilde{k}_\perp z\tilde{u}_z = 0.
\end{equation}
The general solution to this ODE is written in terms of Airy functions for
arbitrary constants $a, b$
\begin{equation}
    \tilde{u}_z(z) = a\mathrm{Ai}\p*{-\frac{z}{\lambda}}
        + b\mathrm{Bi}\p*{-\frac{z}{\lambda}},\label{eq:gen_uz}
\end{equation}
where $\lambda = \tilde{k}_\perp^{-2/3}$. For large $-z$, it turns out that
\begin{align}
    \mathrm{Ai}\p*{-z} &\sim \frac{\sin\p*{\frac{2}{3}z^{3/2} +
        \frac{\pi}{4}}}{z^{1/4}} + \mathcal{O}\p*{z^{-7/4}},\\
    \mathrm{Bi}\p*{-z} &\sim \frac{\cos\p*{\frac{2}{3}z^{3/2} +
        \frac{\pi}{4}}}{z^{1/4}} + \mathcal{O}\p*{z^{-7/4}}.
\end{align}
In order for us to get traveling waves with \emph{group velocity} going outwards
(towards $z > 0$), we need $\tilde{u}_z(z) \sim e^{-ik_zz}$ such that $u(z, t)
\propto e^{i\p*{-k_zz - \omega t}}$ (phase velocity goes inwards, group velocity
goes outwards for IGW). Thus, $a = -ib$ in \autoref{eq:gen_uz}, and we obtain
\begin{equation}
    \tilde{u}_z(z) = b\p*{-i\mathrm{Ai}\p*{-\frac{z}{\lambda}}
        + \mathrm{Bi}\p*{-\frac{z}{\lambda}}},\label{eq:gen_uz}
\end{equation}

Now we just need to fix $b$. This is traditionally accomplished by mandating a
particular $\rd{\delta z}{z}$ the displacement of the mode at the turning point
$z = 0$. That the forcing results in a constraint on $\rd{\delta z}{z}$ is
similar to what I did in my IGW breaking forcing, where forcing induces a jump
in the $\rd{u_z}{z}$ above/below $z_c$ whose magnitude is fixed by the strength
of the forcing term. In the stellar problem, it appears the correct way to
obtain the $\delta z$ is to solve the inhomogeneous problem in the convective
zone where $N^2 = 0$ including the tidal potential, so it's not a perfect
analogy. But since $\mathrm{Ai}'(0) = -\frac{1}{3^{1/3}\Gamma(1/3)},
\mathrm{Bi}'(0) = \frac{3^{1/6}}{\Gamma(1/3)}$, this is not so difficult to
evaluate, and I cite the KZ result
\begin{equation}
    \rd{\delta z}{z} = -\frac{ib}{\lambda \omega}\frac{2}{3^{1/3}\Gamma(1/3)}
        \frac{3^{1/2} + i}{2}.
\end{equation}

Finally, we impose one more step: we will compute the luminosity or \emph{energy
flux} associated with the wave, since this is the easiest way to get the
resulting torque. We can easily write down the energy density of the wave
$\frac{\rho_0}{2}\p*{v^2 + \frac{b^2}{N^2}}$, for which the energy flux is
$\vec{F} = \vec{v}P$. Noting furthermore that $\pd{u_z}{z} = -ik_\perp u_x =
-\frac{ik_\perp P}{\rho_0}\frac{k_\perp }{\omega}$, we can explicitly express
$P$ in terms of $u_z'$, and so the energy flux density is then simply (I'm not
evaluating this, but KZ do)
\begin{align}
    \frac{\delta L}{\delta A} &= \frac{1}{2}\Re\p*{P u_z^*}
            = \frac{\rho_0 \omega}{2k_\perp^2} \Re\p*{iu_z'u_z^*},\\
        &= \frac{3^{2/3}\Gamma^2(1/3) \lambda \omega^3 \rho_0}{8\pi k_\perp^2}
            \p*{\rd{\delta z}{z}}^2.
\end{align}
We would then compute $L = \int \frac{\delta L}{\delta A}\;\mathrm{d}A$, which
for us is just $\frac{\delta L}{\delta A}A$ where $A$ is the surface area of the
wave.

Finally, we would compute the total torque from $L = \tau \omega$ (the same
as $E = \vec{F} \cdot \vec{v}$).

\subsection{Spherical Case}

To go to the spherical case, we simply replace $z \to r$ and $k_\perp^2 \to l(l
+ 1)/r^2$, which gives
\begin{equation}
    \lambda = \p*{\frac{l(l + 1)}{r^2\omega^2} \rd{N^2}{r}}^{-1/3}.
\end{equation}
Then to get $\rd{\delta z}{z} \to \rd{\delta r}{r}$, we use prescription
\begin{equation}
    \rd{\delta r}{r} = \alpha \frac{\Phi}{gr}
        \p*{1 - \frac{\rho(r)}{\bar{\rho}(r)}}.
\end{equation}
Here, $\alpha \sim 1$ depends on the specific stellar structure in the
convective zone, while $\bar{\rho}$ is the average density inside $r$. Finally,
instead of getting a clean $L = \frac{\delta L}{\delta A}A$, we have to actually
do the integral of $L = \int \frac{\delta L}{\delta A}\;\mathrm{d}A = \int
(\dots) \abs*{Y_{lm}}^2 r^2 \;\mathrm{d}\cos\theta \mathrm{d}\phi = r_c^2 L$
(note that it's not $4\pi r_c^2$, thanks to the $Y_{lm}$ normalization).
Finally, the $\ell = 2$ potential is taken to be
\begin{equation}
    \Phi_{\mathrm{ext}} = -\sqrt{\frac{6\pi}{5}}\frac{GM_2R_c^2}{D^3}.
\end{equation}
I guess the angular dependency is just dropped. With all these things together,
we obtain the final KZ result (I omit the derivation, this part is grungy and
not very physically interesting)
\begin{align}
    \tau = \dot{J}_z &= \frac{GM_2^2R_c^5}{D^6} \sigma_c^{8/3}
        \s*{\frac{r_c}{g_c}\p*{\rd{N^2}{\ln R}}_{r = r_c}}^{-1/3}
            \frac{\rho_c}{\bar{\rho}_c} \p*{1 - \frac{\rho_c}{\bar{\rho}_c}}^2
            \s*{\frac{3}{2}\frac{3^{2/3}\Gamma^2(1/3)}{5 \cdot
                6^{4/3}} \frac{3}{4\pi}\alpha^2},\\
        &= \frac{GM_2^2R_c^5}{D^6}2\hat{F}\p*{r_c, \sigma_c}.
            \label{eq:tau_fhat}
\end{align}
Note that $\hat{F}$ follows the convention from Equation 42 of Fuller \& Lai's
second paper (FL2) and Vick et.\ al's paper as well (VLF), while $\sigma_c =
2\abs*{\Omega - \Omega_s} / \sqrt{GM_c / r_c^3}$ is the ratio of the forcing
frequency to the breakup frequency \emph{of the core}. Finally, I've replaced
$M_2$ the mass of the companion, $R_c$ the radius of the core, and $D$ the
separation, while retaining $\Omega_s$ spin angular frequency and $\Omega$
orbital angular frequency.

\textbf{NB:} The exact definition of $\hat{F}$ for a given $m$ is given in
VLF.23 as
\begin{equation}
    \dot{J} = G\frac{M_2R^5}{a^3}\frac{\abs*{m}}{2}\hat{F}(\omega)
        = T_0 \frac{\abs*{m}}{2} \hat{F}(\omega).
\end{equation}
Since the total torque $\tau$ has already summed over $m = \pm 2$, we incur the
extra factor of $2$ above in \autoref{eq:tau_fhat}. That $m$ has already been
summed over is visible in the $\sqrt{6\pi/5}$ prefactor used in $\Phi_{ext}$,
compared to $W_{2\pm 2} = \sqrt{3\pi/10}$ as seen below.

\section{Vick et.\ al., 2016}

We now consider eccentric forcing. We will remove subscript compared to VLF and
just call $\vec{r}_i = \p*{r, \theta, \phi + \Omega_s t}$ the position
coordinate in the inertial frame. Then the $\ell = 2$ tidal forcing potential is
generally a sum over $m \in [-2, 2]$
\begin{align}
    U &= \sum\limits_m U_{2m} \p*{\vec{r}, t},\\
    U_{2m}\p*{\vec{r}} &= -\frac{GM_2 W_{2m} r^2}{D(t)^3}
        e^{-imf(t)} Y_{2m}(\theta, \phi).
\end{align}
Note that $f$ is the true anomaly here. Note that $W_{2m}$ is just a constant:
$W_{20} = \sqrt{\pi/5}$, $W_{2 \pm 1} = 0$, and $W_{2 \pm 2} = \sqrt{3\pi /
10}$.

This is complicated since $f(t)$ does not evolve uniformly, ando also since
$D(t)$ is time-varying! The easiest treatment is to decompose
\begin{equation}
    U_{2m} = -\frac{GM_2W_{2m}r^2}{a^3}Y_{2m}\p*{\theta, \phi}
        \sum\limits_{N = -\infty}^\infty F_{Nm}e^{-iN\Omega t}.
\end{equation}
Note that the $F_{Nm}$ here are \emph{Hansen coefficients} given by
\begin{equation}
    F_{Nm} = \frac{1}{\pi}\int\limits_{0}^{\pi}
        \frac{\cos\s*{N\p*{E - e\sin E} - mf(E)}}
            {\p*{1 - e\cos E}^2}\;\mathrm{d}E.
\end{equation}
Note $E$ is the eccentric anomaly. This differs from the VLF definition in a few
places but is in agreement with Natalia's paper w/ Dong (SD), such that $F_{Nm}
= \delta_{Nm}$ for $e = 0$. It bears noting that VLF's formula normalizes to
$F_{Nm} = 2\delta_{Nm}$, so we use the restricted domain of integration for
numerical speed (the integrand is symmetric since the argument of the cosine is
antisymmetric in $E$, so both the numerator/denominator are even in $E$).

Let's explicitly write out the $U_{2\pm 2}$, since they are the only ones that
contribute to the tidal torque
\begin{align}
    U_{22} &= -\frac{GM_2 \sqrt{\frac{3\pi}{10}}r^2}{a^3}
        \sum\limits_{N = 1}^\infty \s*{
            F_{N2}Y_{22}\p*{\theta, \phi}e^{-i\p*{N\Omega - 2\Omega_s}t}
            + F_{-N2}Y_{22}\p*{\theta, \phi}e^{i\p*{N\Omega + 2\Omega_s}t}},\\
    U_{2-2} &= -\frac{GM_2 \sqrt{\frac{3\pi}{10}} r^2}{a^3}
        \sum\limits_{N = 1}^\infty \s*{
            F_{-N2}Y_{2-2}\p*{\theta, \phi}e^{-i\p*{N\Omega - 2\Omega_s}t}
            + F_{N2}Y_{2-2}\p*{\theta, \phi}e^{i\p*{N\Omega + 2\Omega_s}t}},\\
    U_{22} + U_{2-2} &= -\frac{GM_2 \sqrt{\frac{3\pi}{10}} r^2}{a^3}
        \sum\limits_{N = 1}^\infty \s*{
            F_{N2}Y_{22}\p*{\theta, \phi}e^{-i\p*{N\Omega - 2\Omega_s}t}
            + c.c.}
\end{align}
We can verify that if the perturbing orbit is circular $e = 0$, then the Hansen
coefficient $F_{Nm} = \delta_{Nm}$, and we obtain
\begin{equation}
    U_{22} + U_{2-2} = -\frac{GM_2r^2}{a^3}\sqrt{\frac{6\pi}{5}}
        \Re \s*{Y_{22}\p*{\theta, \phi}e^{-2i\p*{\Omega - \Omega_s}t}}.
\end{equation}
This is the same torque used in KZ\@. Finally, this yields torque
\begin{equation}
    \dot{J} = \tau = T_0 \sum\limits_{N = -\infty}^\infty
        F_{N2}^2 \mathrm{sgn}\p*{N\Omega - 2\Omega_s} \hat{F}
            \p*{\omega = \abs*{N\Omega - 2\Omega_s}}.
\end{equation}

\subsection{Hansen Coefficients}

Maybe someday follow \url{https://arxiv.org/pdf/1308.0607.pdf} and get the
derivation of the Hansen coefficients? One fast way to calculate them is to take
an FFT of the $F^{lm} = \p*{\frac{r}{a}}^l e^{imf}$, per
\url{https://www.aanda.org/articles/aa/pdf/2014/11/aa24211-14.pdf}% chktex 8
(CBLR). Basically, the Hansen coefficients are just the FT of the disturbing
function. Consider that we want to make jump from
\begin{equation}
    U(r, t) = -GM_rr^2 \sum\limits_m \frac{W_{2m}}{D(t)^3}e^{-imf(t)}
        Y_{2m}\p*{\theta, \phi},
\end{equation}
to
\begin{equation}
    U(r, t) = -\frac{GM_2r^2}{a^3}\sum\limits_{m, N} W_{2m}
        F_{Nm}(e) Y_{2m}\p*{\theta, \phi} e^{-in\Omega t}.
\end{equation}
Thus, we seek coefficients such that
\begin{equation}
    \frac{a^3}{D(t)^3} e^{-imf} = \p*{\frac{1 + e\cos f}{1 - e^2}}^3
            e^{-imf}
        = \sum\limits_N F_{Nm} e^{-iN\Omega t}.
\end{equation}
Thus, it's clear the Hansen coefficients are defined by computing Fourier series
coefficients (NB\@: In hindsight, using $r = a\p*{1 - e\cos E}$ probably would
have been much faster/easier)
\begin{align}
    F_{Nm} &\equiv \frac{1}{T}\int\limits_{0}^T
        \frac{e^{-imf}}{\p*{1 - e^2}^3}\p*{1 + e\cos f}^3 e^{iN\Omega t}
            \;\mathrm{d}t,\\
        &= \frac{1}{2\pi} \int\limits_{0}^{2\pi}
            \frac{e^{-imf}}{\p*{1 - e^2}^3}\p*{1 + e\cos f}^3 e^{iN\Omega t}
                \;\mathrm{d}M
\end{align}
We have notated $T$ the period, and $M$ the mean anomaly. Then one just
evaluates using $\cos f = \frac{\cos E - e}{1 - e\cos E}$ and $M = E - e\sin E$
or more usefully $\mathrm{d}M = \p*{1 - e\cos E}\mathrm{d}E$ and obtains
\begin{align}
    F_{Nm} &= \frac{1}{2\pi}\int\limits_0^{2\pi}
            \p*{\frac{1 + e\cos f}{1 - e^2}}^3 e^{-imf + iN\Omega t}
                \;\mathrm{d}M,\\
        &= \frac{1}{2\pi}\int\limits_0^{2\pi}
            \p*{\frac{1}{1 - e\cos E}}^3 e^{-imf + iN M} \p*{1 - e\cos E}
                \;\mathrm{d}E,\\
        &= \frac{1}{2\pi}\int\limits_0^{2\pi}
            \frac{\exp\s*{i\p*{N\p*{E - e\sin E} - mf}}}{(1 - e \cos E)^2}
                \;\mathrm{d}E.
\end{align}
Now, as we observed above, the integrand is symmetric with respect to $E$, but
it had to be, since in an elliptical orbit the first half and second half are
obviously symmetric. Thus, we arrive at final expression as promised
\begin{equation}
    F_{Nm} = \frac{1}{\pi}\int\limits_{0}^{\pi}
        \frac{\cos\s*{N\p*{E - e\sin E} - mf(E)}}
            {\p*{1 - e\cos E}^2}\;\mathrm{d}E.
\end{equation}

\section{Combined Results}

We have been somewhat careful in checking the agreement between the VLF and KZ
forms. Note now that \autoref{eq:tau_fhat} has $\hat{F}$ for a single $m$
contribution, just as $\hat{F}$ is defined in VLF\@. Thus, we should be able to
simply plug in
\begin{align}
    \tau &= T_0 \sum\limits_{N = -\infty}^\infty
        F_{N2}^2 \frac{\mathrm{sgn}\p*{N\Omega - 2\Omega_s}}{2} \sigma_c^{8/3}
        \s*{\frac{r_c}{g_c}\p*{\rd{N^2}{\ln R}}_{r = r_c}}^{-1/3}
            \frac{\rho_c}{\bar{\rho}_c} \p*{1 - \frac{\rho_c}{\bar{\rho}_c}}^2
            \s*{\frac{3}{2}\frac{3^{2/3}\Gamma^2(1/3)}{5 \cdot
                6^{4/3}} \frac{3}{4\pi}\alpha^2},\\
        &= T_0 C(r_c) \sum\limits_{N = -\infty}^\infty
            F_{N2}^2 \mathrm{sgn}\p*{N\Omega - 2\Omega_s}
                \sigma_c^{8/3}.\label{eq:tau_sum}
\end{align}
Note that now $\sigma_c = \abs*{N\Omega - 2\Omega_s} / \sqrt{GM_c / r_c^3}$, and
I've defined $C(r_c)$ to be some constant defined at the RCB and does not change
with $N$.

Thus, the relative significance of each $N$ term is given by the summand of
\autoref{eq:tau_sum}, or
\begin{equation}
    \tau_N \equiv F_{N2}^2 \mathrm{sgn}\p*{N\Omega - 2\Omega_s} \sigma_c^{8/3},
\end{equation}
$F_{N2}$ is easiest to evaluate via an integral for now, but can probably be
done via a sampling + FFT when speed is necessary (CBLR).

One guess is that the sum is dominated by the contribution of the frequency at
pericenter. Since pericenter occurs at $a_p = a\p*{1 - e}$, using scaling $P^2
\propto r^3 \propto \Omega^{-2}$ we find that $\frac{\Omega_p}{\Omega} =
\p*{\frac{a}{a_p}}^{3/2} = \p*{1 - e}^{-3/2}$. Thus, we should expect the
dominant term to come at $N \sim \p*{1 - e}^{-3/2}$. This indeed roughly
maximizes $F_{N2}$ but probably won't maximize $\tau_N$.

\textbf{NB:} It appears that the maximum $N$ to sum to is (according to
Michelle/Chris)
\begin{equation}
    N_{\max} = 10 \p*{1 - e}^{-3/2}.
\end{equation}

To understand exactly to what $N$ we should sum, we should recall $\abs*{\tau_N}
\propto \abs*{F_{N2}}^2 \abs*{N\Omega - 2\Omega_s}^{8/3}$. If $\Omega_s \gg
\Omega_p$, then the latter term $\sigma_c^2$ is roughly independent of $N$ and
indeed we get that $\tau_N$ turns over similarly to where $F_{N2}$ turns over.
On the other hand, if $\Omega_s \lesssim \Omega_p$, then $\tau_N \propto
\abs*{F_{N2}}^2 N^{8/3}$. Plots of these two are given in \autoref{fig:hansens}.
\begin{figure}[t]
    \centering
    \begin{subfigure}{0.45\textwidth}
        \centering
        \includegraphics[width=\textwidth]{../scripts/eccentric_tides/hansens.png}
    \end{subfigure}
    \begin{subfigure}{0.45\textwidth}
        \centering
        \includegraphics[width=\textwidth]{../scripts/eccentric_tides/hansens_83.png}
    \end{subfigure}
    \caption{$F_{N2}$ and $F_{N2} N^{8/3}$ as we integrate straightforwardly.
    Note the vertical blue line is $N = \p*{1 - e}^{-3/2}$ while the vertical
    black/green line are the actual $\argmax_N F_{N\pm 2}$ resrpectively.
    }\label{fig:hansens}
\end{figure}

A plot of the actual maximum $F_{N \pm 2}$ is provided below in
\autoref{fig:maxes}. Note that $F_{N2}$ seems to be a constant multiple of
$\p*{1 - e}^{-3/2}$ our prediction, while $F_{N-2}$ seems to be impacted by the
shallowness of the fit at smaller eccentricities.
\begin{figure}[t]
    \centering
    \includegraphics[width=\textwidth]{../scripts/eccentric_tides/hansen_maxes.png}
    \caption{Maxima of $F_{\pm 2}$ with fits.}\label{fig:maxes}
\end{figure}

\end{document}

