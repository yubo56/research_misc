    \documentclass[10pt]{article}% chktex 8
    \usepackage{
        amsmath,
        amssymb,
        hyperref, % various links
        amsthm, % newtheorem and proof environment
        mathtools, % \Aboxed for boxing inside aligns, among others
        enumerate, % Allow custom enumerate numbering
        graphicx, % allow includegraphics with more filetypes
        booktabs, % \bottomrule instead of hline apparently
        xcolor, % colored text
        wasysym
    }
    \usepackage[
        labelfont=bf, % caption names are labeled in bold
        font=scriptsize % smaller font for captions
    ]{caption}
    \usepackage[margin=1.0in]{geometry}

    \newcommand*{\scinot}[2]{#1\times10^{#2}}
    \newcommand*{\dotp}[2]{\left<#1\,\middle|\,#2\right>}
    \newcommand*{\rd}[2]{\frac{\mathrm{d}#1}{\mathrm{d}#2}}
    \newcommand*{\pd}[2]{\frac{\partial#1}{\partial#2}}
    \newcommand*{\rdil}[2]{\mathrm{d}#1 / \mathrm{d}#2}
    \newcommand*{\pdil}[2]{\partial#1 / \partial#2}
    \newcommand*{\rtd}[2]{\frac{\mathrm{d}^2#1}{\mathrm{d}#2^2}}
    \newcommand*{\ptd}[2]{\frac{\partial^2 #1}{\partial#2^2}}
    \newcommand*{\md}[2]{\frac{\mathrm{D}#1}{\mathrm{D}#2}}
    \newcommand*{\pvec}[1]{\vec{#1}^{\,\prime}}
    \newcommand*{\svec}[1]{\vec{#1}\;\!}
    \let\bm\undefined
    \newcommand*{\bm}[1]{\boldsymbol{\mathbf{#1}}}
    \newcommand*{\uv}[1]{\hat{\bm{#1}}}
    \newcommand*{\ang}[0]{\;\text{\AA}}
    \newcommand*{\mum}[0]{\;\upmu \mathrm{m}}
    \newcommand*{\at}[1]{\left.#1\right|}
    \newcommand*{\bra}[1]{\left<#1\right|}
    \newcommand*{\ket}[1]{\left|#1\right>}
    \newcommand*{\abs}[1]{\left|#1\right|}
    \newcommand*{\ev}[1]{\langle#1\rangle}
    \newcommand*{\p}[1]{\left(#1\right)}
    \newcommand*{\s}[1]{\left[#1\right]}
    \newcommand*{\z}[1]{\left\{#1\right\}}

    \let\Re\undefined
    \let\Im\undefined
    \DeclareMathOperator{\Res}{Res}
    \DeclareMathOperator{\Re}{Re}
    \DeclareMathOperator{\Im}{Im}
    \DeclareMathOperator{\Log}{Log}
    \DeclareMathOperator{\Arg}{Arg}
    \DeclareMathOperator{\Tr}{Tr}
    \DeclareMathOperator{\E}{E}
    \DeclareMathOperator{\Var}{Var}
    \DeclareMathOperator*{\argmin}{argmin}
    \DeclareMathOperator*{\argmax}{argmax}
    \DeclareMathOperator{\sgn}{sgn}
    \DeclareMathOperator{\diag}{diag\;}

    \colorlet{Corr}{red}

\begin{document}

\linespread{1.15}

\title{Octupole-order Lidov-Kozai Population Statistics}
\author{Yubo Su}
\date{Today}

\maketitle

\section{10/22/20---Initial Thoughts}

\subsection{Equations}

The equations of motion we want to study come from LML15. Describe the inner
binary by $\p{a, e, I, \ascnode, \omega}$ and the outer binary with ``out''
subscripts, and denote $I_{\rm tot} = I + I_{\rm out}$. Call the inner binary
component masses $m_1$, $m_2$, and the tertiary mass $m_3$, define the inner
binary total and reduced masses $m_{12} = m_1 + m_2$ and $\mu = m_1m_2 /
m_{12}$, and define the tertiary orbit total and reduced masses $m_{123} =
m_{12} + m_3$ and $\mu_{\rm out} = m_{12}m_3 / m_{123}$. The equations of motion
are ($j(e) = \sqrt{1 - e^2}$)
{\small
\begin{align}
    \rd{a}{t} ={}& -\frac{64}{5}\frac{a}{t_{\rm GW}j^7(e)}
            \p{1 + \frac{73}{24}e^2 +
            \frac{37}{96}e^4},
\end{align}
\begin{align}
    \rd{e}{t} ={}& \frac{j(e)}{64t_{\rm LK}}\Big\{
            120 e \sin^2 I_{\rm tot} \sin \p{2\omega}\nonumber\\
        &+ \frac{15 \epsilon_{\rm oct}}{8}\cos \omega_{\rm out}
            \big[\p{4 + 3e^2}\p{3 + 5\cos (2I_{\rm tot})}\nonumber\\
        &\times \sin \omega + 210 e^2\sin^2I_{\rm tot} \sin 3\omega\big]
            \nonumber\\
        &- \frac{15 \epsilon_{\rm oct}}{4} \cos I_{\rm tot} \cos \omega
            \big[15(2 + 5e^2) \cos \p{2I_{\rm tot}}\nonumber\\
        &+ 7\p{30 e^2 \cos(2\omega) \sin^2 I_{\rm tot} - 2 - 9e^2}]
                \sin \omega_{\rm out}\Big\}\nonumber\\
        &- \frac{304}{15}\frac{e}{t_{\rm GW}j^5(e)}
            \p{1 + \frac{121}{304}e^2},
\end{align}
\begin{align}
    \rd{I}{t} ={}& -\frac{3e}{32t_{\rm LK}j(e)}\Big\{
            10\sin\p{2I_{\rm tot}}\big[e \sin (2\omega)\nonumber\\
        &+ \frac{5\epsilon_{\rm oct}}{8}\p{2 + 5e^2 + 7e^2\cos(2\omega)}
            \cos \omega_{\rm out}\sin \omega\big]\nonumber\\
        &+ \frac{5\epsilon_{\rm oct}}{8} \cos \omega \big[
            26 + 37e^2 -35 e^2\cos(2\omega)\nonumber\\
        &- 15\cos\p{2I_{\rm tot}}\p{7e^2\cos (2\omega) - 2 - 5e^2}\big]\nonumber\\
        &\times \sin I_{\rm tot}\sin \omega_{\rm out}\Big\}
\end{align}
\begin{align}
    \rd{\ascnode}{t} ={}& \rd{\ascnode_{\rm out}}{t}
            = -\frac{3\csc I}{32 t_{\rm LK}j(e)}\Big\{
            2\big[(2 + 3e^2 - 5e^2 \cos (2\omega))\nonumber\\
        &+ \frac{25 \epsilon_{\rm oct}e}{8} \cos \omega
            \p{2 + 5e^2 - 7e^2\cos(2\omega)}\cos \omega_{\rm out}\big]\nonumber\\
        &\times\sin (2I_{\rm tot}) - \frac{5\epsilon_{\rm oct}e}{8}
            \big[35e^2(1 + 3\cos\p{2I_{\rm tot}})\cos 2\omega\nonumber\\
        &- 46 - 17e^2 - 15\p{6 + e_1^2}\cos\p{2I_{\rm tot}}\big]\nonumber\\
        &\times \sin I_{\rm tot} \sin \omega \sin \omega_{\rm out}\Big\},
\end{align}
\begin{align}
    \rd{\omega}{t} ={}& \frac{3}{8t_{\rm LK}}
            \Big\{\frac{1}{j(e)} \big[4 \cos^2 I_{\rm tot}
            + \p{5 \cos (2\omega) - 1}\nonumber\\
        &\times \p{1 - e^2 - \cos^2 I_{\rm tot}}\big] + \frac{L \cos I_{\rm tot}}{
            L_{\rm out}j(e_{\rm out})}\big[2 + e^2(3\nonumber\\
        &- 5\cos(2\omega))\big]\Big\}
        + \frac{15 \epsilon_{\rm oct}}{64t_{\rm LK}}\Big\{
            \p{\frac{L}{L_{\rm out}j(e_{\rm out})}
                + \frac{\cos I_{\rm tot}}{j(e)}}\nonumber\\
        &\times e\big[\sin \omega \sin \omega_{\rm out}
            \s{10(3\cos^2 I_{\rm tot} - 1)(1 - e^2) + A}\nonumber\\
        &- 5B\cos I_{\rm tot}\cos \Theta\big] - \frac{j(e)}{e}
            \big[10 \sin \omega \sin \omega_{\rm out}\cos I_{\rm tot}\nonumber\\
        &\times \sin^2 I_{\rm tot}\p{1 - 3e^2} + \cos \Theta\p{3A -
            10\cos^2 I_{\rm tot} + 2}\big]\Big\}\nonumber\\
        &+ \Omega_{\rm GR},
\end{align}
\begin{align}
    \rd{e_{\rm out}}{t} ={}& \frac{15eL\,j(e_{\rm out})\epsilon_{\rm oct}}{
            256t_{\rm LK}e_{\rm out}L_{\rm out}}\Big\{ \cos \omega
                \big[6 - 13e^2\nonumber\\
        &+ 5(2 + 5e^2)\cos (2I_{\rm tot}) + 70e^2 \cos (2\omega) \sin^2 I_{\rm
            tot}\big]\nonumber\\
        &\times \sin \omega_{\rm out} - \cos I_{\rm tot} \cos \omega_{\rm out}
            \big[5(6 + e^2)\cos \p{2I_{\rm tot}}\nonumber\\
        &+ 7\p{10e^2 \cos (2\omega) \sin^2 I_{\rm tot} - 2 + e^2}\big]
            \sin \omega\Big\},
\end{align}
\begin{align}
    \rd{I_{\rm out}}{t} ={}& -\frac{3eL}{32t_{\rm LK}j(e_{\rm out}) L_{\rm out}}
            \Big\{ 10\big[2e \sin I_{\rm tot} \sin (2\omega)\nonumber\\
        &+ \frac{5\epsilon_{\rm oct}}{8}\cos \omega\p{
            2 + 5e^2 - 7e^2\cos (2\omega)}\sin (2I_{\rm tot})\sin \omega_{\rm
            out}\big]\nonumber\\
        &+ \frac{5\epsilon_{\rm oct}}{8}\big[26 + 107e^2
            + 5(6 + e^2)\cos(2I_{\rm tot})\nonumber\\
        &- 35 e^2\p{\cos 2(I_{\rm tot}) - 5} \cos (2\omega)\big]
            \cos \omega_{\rm out}\sin I_{\rm tot} \sin \omega\Big\},
\end{align}
\begin{align}
    \rd{\omega_{\rm out}}{t} ={}&
        \frac{3}{16t_{\rm LK}}\Big\{\frac{2\cos I_{\rm tot}}{j(e)}
            \s{2 + e^2(3 - 5\cos(2\omega))}\nonumber\\
        &+ \frac{L}{L_{\rm out}j(e_{\rm out})}\big[4 + 6e^2
            + (5\cos^2 I_{\rm tot} - 3)\nonumber\\
        &\times [2 + e^2(3 - 5\cos(2\omega))]\big]\Big\}
            - \frac{15 \epsilon_{\rm oct}e}{64t_{\rm LK}e_{\rm out}}\nonumber\\
        &\times \Big\{ \sin \omega \sin \omega_{\rm out}\Big[
            \frac{L(4e_{\rm out}^2 + 1)}{e_{\rm out}L_{\rm out} j(e_{\rm out})}
            10 \cos I_{\rm tot}\sin^2 I_{\rm tot}\nonumber\\
        &\times(1 - e^2) - e_{\rm out}\p{\frac{1}{j(e)} +
            \frac{L \cos I_{\rm tot}}{L_{\rm out}j(e_{\rm out})}}\nonumber\\
        &\times\s{A + 10\p{3\cos^2 I_{\rm tot} - 1}\p{1 - e^2}}\Big]
            + \cos \Theta\nonumber\\
        &\times \Big[5B\cos I_{\rm tot}e_{\rm out}\p{\frac{1}{j(e)} +
            \frac{L \cos I_{\rm tot}}{L_{\rm out}j(e_{\rm out})}}\nonumber\\
        &+ \frac{L\p{4e_{\rm out}^2 + 1}}{e_{\rm out}L_{\rm out}j(e_{\rm out})}
            A\Big]\Big\}.
\end{align}
}
where $n = \sqrt{Gm_{12} / a^3}$ is the mean motion, $L = \mu \sqrt{Gm_{12}a}$
and $L_{\rm out} = \mu_{\rm out}\sqrt{Gm_{123}a_{\rm out}}$ are the circular
angular momenta, and
\begin{align}
    t_{\rm LK}^{-1} &= n\p{\frac{m_3}{m_{12}}}\p{\frac{a}{a_{\rm out}j(e_{\rm
        out})}}^3,\\
    t_{\rm GW}^{-1} &= \frac{G^3\mu m_{12}^2}{c^5a^4},\\
    \Omega_{\rm GR} &= \frac{3Gnm_{12}}{c^2aj^2(e)},\\
    \epsilon_{\rm oct} &= \frac{m_2 - m_1}{m_{12}}\frac{a}{a_{\rm out}}
        \frac{e_{\rm out}}{1 - e_{\rm out}^2},\\
    A &\equiv 4 + 3e^2 - \frac{5}{2}B\sin^2 I_{\rm tot},\\
    B &\equiv 2 + 5e^2 - 7e^2 \cos(2\omega),\\
    \cos \Theta &\equiv -\cos \omega \cos \omega_{\rm out}
        - \cos I_{\rm tot} \sin \omega \sin \omega_{\rm out}.
\end{align}

These equations can be nondimensionalized via the following steps (I won't
rewrite the equations): (i) multiply through by $t_{\rm LK, 0}$ ($a = a_0$ and
$e_{\rm out} = 0$), and call $\tau \equiv t / t_{\rm LK, 0}$ the new variable of
differentiation, (ii) re-express all of the timescales as
\begin{align}
    \frac{t_{\rm LK, 0}}{t_{\rm LK}} &= \p{\frac{a}{a_0}}^{3/2}
        j^{-3}\p{e_{\rm out}},\\
    \frac{t_{\rm LK, 0}}{t_{\rm GW}} &=
        \frac{G^3\mu m_{12}^3}{m_3c^5a^4}\frac{1}{n_0}\p{\frac{a_{\rm out}
            }{a_0}}^3,\nonumber\\
        &= \epsilon_{\rm GW}\p{\frac{a_0}{a}}^4,\\
    \epsilon_{\rm GW} &\equiv \frac{G^3\mu m_{12}^3
        a_{\rm out}^3}{m_3c^5a_0^7 n_0},\\
    \Omega_{\rm GR}t_{\rm LK, 0} &= \frac{3Gnm_{12}^2}{m_3c^2a}
        \frac{1}{n_0}\p{\frac{a_{\rm out}}{a_0}}^3,\nonumber\\
        &= \epsilon_{\rm GR} \p{\frac{a_0}{a}}^{5/2},\\
    \epsilon_{\rm GR} &= \frac{3Gm_{12}^2a_{\rm out}^3}{m_3c^2a_0^4}.
\end{align}
(iii) re-express $\rdil{a}{t}$ as
\begin{equation}
    \rd{(a / a_0)}{\tau} = -\frac{64}{5}\frac{\epsilon_{\rm GW}}{
        j^{7/2}(e)}\p{\frac{a_0}{a}}^3\p{1 + \frac{73}{24}e^2 +
        \frac{37}{96}e^4}.
\end{equation}
As such, the natural unit of length is $a_0 = 1$, the natural unit of time is
$t_{\rm LK, 0} = 1$, and everything else is dimensionless. When computing these
$\epsilon$, I use convention where $1 M_{\odot} = 1\;\mathrm{AU} = c = 1$, under
which $G = 9.87 \times 10^{-9}$.

\subsection{Points of Inquiry}

The goal is to understand how the merger window varies as a function of $q
\equiv m_1 / m_2$ ($m_1 < m_2$) when the octupole order LK effects are
important.
\begin{itemize}
    \item First, let's set $\epsilon_{\rm GW} = 0$. It is well known that the
        octupole order LK is nonintegrable. What does the Fourier spectrum of
        the eccentricity look like? Will this help us get a delay time
        distribution between high-e phases?

        When the octupole effect is unimportant, the spectrum falls off
        exponentially over scales $\tau \simeq P_{\rm LK} j^{-1}(e_{\max})$,
        where $P_{\rm LK}$ is the quadrupole LK period. One imagines the
        tail of the spectrum gets heavier when $\epsilon_{\rm oct}$ is
        increased, and this might help us get the delay time distribution.

        A second way we can postprocess this is to take a histogram of $e(t)$.
        If there is some regular structure, it's likely this will allow us to
        compute the average rate of binary coalescence due to GW radiation.

    \item The goal is to understand the size of the merger window, $\Delta I$,
        as a function of $q$. To do this, we numerically sample the merger time
        function $T_{\rm m}\p{I_0, q}$. At each $I_0$, the natural thing to do
        would be to try for $\sim 5-10$ random $\Omega, \omega$, and define the
        merger window to be where $T_{\rm m} \leq 10^10\;\mathrm{yr}$.
\end{itemize}

\section{11/02/20}

We've done a lot more inquiry on this, read the Nov 3 weekly for a recap. In
summary though, there should roughly be two ways to have LK-induced mergers:
\begin{itemize}
    \item Quadrupole LK-induced mergers. The $e_{\max}$ of these systems is well
        understood, however, and the merger fraction should be easy to
        calculate: we can get the merger time using the $T_m \propto \p{1 -
        e_{\max}}^{-3}$ physically-justified fitting law from LL18 in the
        LK-induced regime, and we just have to evaluate where it crosses
        $10\;\mathrm{Gyr}$.

    \item Octupole LK-induced mergers. For these, there is a characteristic
        initial inclination range for which orbit flipping occurs, which is a
        function of $\epsilon_{\rm oct}$. This can likely be calculated
        analytically, but I'm not sure yet.

        For these systems, there is a characteristic orbit flipping timescale
        that is robust up to a factor of a few (since $K$ oscillates on a fixed
        timescale, and orbit flips occur whenever $K$ crosses $-\eta/2$), call
        this $t_{\rm ELK}$. Thus, octupole LK-induced mergers occur over
        characteristic times $t_{\rm ELK}$ within the desired inclination window
        (as once the system reaches an orbit flipping eccentricity, it executes
        a one-shot merger, approximately). It is well known that $t_{\rm ELK}$
        depends on $\epsilon_{\rm oct}$, see e.g.\ Antognini 2015.

        There is some small variability, however, in this picture, according to
        my plots. This can likely be attributed to the exact history of
        eccentricity maxima prior to the one-shot maximum, since particularly
        dissipative sequences (i.e.\ many large eccentricity maxima prior to
        undergoing ``the big one'') can shrink the orbit and change the LK cycle
        pattern.
\end{itemize}

Armed with this, we should have enough information to compute the indicator
function $\mathbb{I}_{\rm merge}\p{a_{\rm in}, I_0, t_{\rm LK}, \epsilon_{\rm
oct}}$, whether a system will merge, by simply computing the two ranges of $I_0$
from above. This should be valid wherever we are in a strongly LK-induced
regime, i.e.\ very large eccentricities are necessary to merge.

In fact, there's a possibility the critical $\epsilon_{\rm oct}$ can be
calculated analytically, see Katz et.\ al.\ 2011 for the test particle case. Is
this generalizable?

\subsection{The Quadrupole Conserved Quantity}

This is a quick derivation. To quadrupolar order, $L_{\rm out}$ is conserved, as
is the total angular momentum, so
\begin{align}
    L_{\rm tot}^2 &= L_{\rm out}^2 + L_{\rm in}^2 + 2L_{\rm out}L_{\rm in}
            \cos I,\\
    \frac{L_{\rm tot}^2 - L_{\rm out}^2}{L_{\rm out}L_{\rm in, 0}} &=
        \eta j_{\rm in}^2 + 2j_{\rm in}\cos I,\\
    \frac{1}{2}\s{\frac{L_{\rm tot}^2 - L_{\rm out}^2}{L_{\rm out}L_{\rm in, 0}}
        - \eta} &= j_{\rm in}\cos I - \eta \frac{e_{\rm in}^2}{2}.
\end{align}
This is the form of the constant given in LL18. Note that $\eta = \eta_0 /
j_{\rm out}$, where $\eta_0$ is evaluated at $e_{\rm out} = 0$ as well.

\section{11/03/20}

We found out that the critical inclination window has a clean fitting formula in
the test particle limit, MLL16 (which agrees with the leading order expansion of
Katz et al 2011). If we take $\epsilon_{\rm oct}$ to be small, then
\begin{align}
    \cos^2 I_0 &\leq \frac{\epsilon_{\rm oct}}{0.4},\\
    \Delta I_0 &\leq \sqrt{\frac{\epsilon_{\rm oct}}{0.4}}.
\end{align}
This marks the inclination window in which the octupole-order effects are
expected to be dominant, if the inclination window is similar to the test
particle case (it's not exactly, since with finite $\eta$ the window is offset
to larger $I_0$ than $I_{0, \lim}$)

We will make two simplifying assumptions:
\begin{itemize}
    \item DA is valid.
    \item $e_{\lim}$ is sufficiently large that one-shot mergers occur.
\end{itemize}
If we assume that DA is valid, then
\begin{align}
    t_{\rm LK}\sqrt{1 - e_{\max}^2} &\gtrsim \frac{1}{n_{\rm out}},\\
    \frac{a^{3/2}}{a_{\rm out}^{3/2}}\frac{m_{12}}{m_3}
        \p{\frac{a_{\rm out}}{a}}^{3}\sqrt{1 - e_{\max}^2}
            \p{1 - e_{\rm out}^2}^{3/2} &\gtrsim 1,\\
    \frac{a_{\rm out}}{a} &\gtrsim \frac{1}{j_{\min}^{2/3}}.
\end{align}
In the last line, we take $e_{\rm out} \sim 0.6$ for which $\p{1 - e_{\rm
out}^2}^{3/2} = 0.5$, and $m_{12} / m_3 = 2$. For some reference $e_{\max} =
10^{-4}$, $j_{\min}^{2/3} = 0.058$ (it's okay for DA to break down at
$e_{\lim}$, since one shot mergers should occur). TODO this should probably be
evaluated for $e_{\max}$ near the edge of the quadrupole merger window.

Recall also that
\begin{equation}
    \epsilon_{\rm oct} = \frac{1 - q}{1 + q}\frac{a}{a_{\rm out}}
        \frac{e_{\rm out}}{1 - e_{\rm out}^2}.
\end{equation}
Thus, in the DA regime, we can explicitly write down the merger window for
ELK-induced mergers
\begin{equation}
    f_{\rm oct} \equiv \frac{2\Delta I_0}{\pi}
        \lesssim \sqrt{\frac{1 - q}{1 + q}\frac{1}{2}\frac{e_{\rm out}}{1 -
            e_{\rm out}^2}}.
\end{equation}
This isn't great, since my data suggest that $\Delta I_0$ should be linear in
$(1-q)/(1+q)$, where $\epsilon_{\rm oct} \simeq 0.01$ should be small enough to
satisfy the assumptions. Bin's data also agree with this, at least up to $e_{\rm
out} = 0.6$ ($25^\circ$, $15^\circ$, $10^\circ$ degree merger windows for $q =
0.2$, $0.4$, $0.6$ respectively).

So it's clear that the ELK-active window should be revised for the finite-$\eta$
case.

\subsection{Differentiating $K$}

We showed earlier that the quadrupole-conserved quantity is
\begin{equation}
    K = j \cos I + j^2\eta = \frac{1}{2}\frac{L_{\rm tot}^2 - L_{\rm out}^2}{
        L_{\rm out}L_{\rm in, 0}}.
\end{equation}
Differentiating the second expression, we obtain that
\begin{align}
    \rd{K}{t} &= \frac{1}{2L_{\in, 0}}
        \p{\frac{-L_{\rm tot}^2}{L_{\rm out}^2} \rd{L_{\rm out}}{t}
            - \rd{L_{\rm out}}{t}},\\
        &= -\frac{1 + L_{\rm tot}^2 / L_{\rm out}^2}{2L_{\rm in, 0}}
            \uv{L}_{\rm out} \cdot \rd{\bm{L}_{\rm out}}{t}.
\end{align}
Indeed, from LML15, we see that this term vanishes to quadrupolar order, and
that the only terms that survive are (I use $1$ for in and $2$ for out somewhat
interchangeably, due to the source materials I'm pulling from)
\begin{align}
    \frac{\uv{L}_{\rm out}}{L_{\rm in}} \cdot \rd{\bm{L}_{\rm out}}{t}
        &= -\frac{75\epsilon_{\rm oct}}{64}\s{
            2\p{\bm{e}_1 \cdot \uv{n}_2}\p{\bm{j}_1 \cdot \uv{n}_2}
                \uv{L}_2 \cdot \p{\uv{u}_2 \times \bm{j}_1}
            + \s{\frac{8}{5}e_1^2 - \frac{1}{5}
                - 7\p{\bm{e}_1 \cdot \uv{n}_2}^2
                + \p{\bm{j}_1 \cdot \uv{n}_2}^2}
                \uv{L}_2 \cdot \p{\uv{u}_2 \times \bm{e}_1}
            }.
\end{align}
Triple products let us simplify a little bit to be in terms of $\uv{v}_2 =
\uv{L}_2 \times \uv{e}_2$ as in LML15. But this won't be useful: if we don't
have the second derivative as well, we can't compute a characteristic
oscillation frequency.

We can make one more approximation though: all of the $\bm{j}_1 \cdot \uv{n}_2$
terms are generally small, since they are of order $j_{1z}$ which is not
conserved exactly but remains small (this is similar to Katz's approximation).
This means we're left with
\begin{align}
    \rd{K}{t}
        &\approx \p{1 + \frac{\eta^2}{4}}
        \frac{75\epsilon_{\rm oct}}{64}
            \s{\frac{8}{5}e_1^2 - \frac{1}{5}
                - 7\p{\bm{e}_1 \cdot \uv{n}_2}^2}
                \bm{e}_1 \cdot \uv{v}_2.
\end{align}
We've assumed $L_{\rm tot}^2 / L_{\rm out}^2 \sim 1 + \eta^2/2$, since typically
$\bm{L}_{\rm in}$ and $\bm{L}_{\rm out}$ are misaligned by $I \simeq 90^\circ$.
This is again in agreement with the Katz formula except now $\bm{n}_2$ and
$\bm{v}_2$ are permitted to vary (and our prefactor).

Now, their $\Omega_{\rm e}$ is such that $e_z / e = \cos \Omega_{\rm e}$. This
is a bit more difficult to generalize for us, but it shouldn't be impossible. We
will return later if it proves interesting.

\subsection{Timescale Analysis}

Let's suppose we only consider systems with $T_{\rm m, 0} > 10\;\mathrm{Gyr}$,
that cannot merge in a Hubble time. For simplicity, we also require DA hold, and
all three masses be comparable ($m_{12} = 2m_3$, say). Then this places
constraint
\begin{equation}
    \frac{5c^5a_0^4}{256G^3m_{12}^2\mu} > 10\;\mathrm{Gyr}.
\end{equation}
If we fix $m_3 = 30M_{\odot}$, so $m_{12}^2\mu = q(1+q)\p{30M_{\odot}}^3$, then
\begin{equation}
    \frac{a_0}{0.202\;\mathrm{AU}}\p{\frac{2}{q(1+q)}}^{1/4} > 1.
\end{equation}
This is a very weak constraint.

Nevertheless, if we are firmly in the LK-induced regime, LL18 can be used to
easily compute the quadrupole merger window, $I_{\rm 0, merger}^+ - I_{\rm 0,
merger}^-$ by just using their Eq.~(42) to compute $e_{\max}$. We know that in
the LK-induced regime, $\epsilon_{\rm GR}$ is very weak except for at very large
$e_{\max}$, so we can likely omit it when computing the $I_{\rm 0,
merger}^{\pm}$ since these are very smooth mergers. Then, since $j_{\min}^6 =
10\;\mathrm{Gyr} / T_{\rm m, 0}$, we can obtain
\begin{align}
    \cos I_{\rm 0, merger}^- - \cos I_{\rm 0, merger}^+
        &= \frac{1}{5}\sqrt{
            \p{5\eta - 4\eta j_{\min}^2}^2
            - 20\p{\frac{5\eta^2}{4}
                - j_{\min}^2\p{3 + \frac{9\eta^2}{4}}
                + \eta^2j_{\min}^4}},\\
        &\approx \frac{j_{\min}\sqrt{60}}{5}
            + \mathcal{O}\p{\eta j_{\min}},\\
        &\approx \frac{\sqrt{60}}{5}\p{\frac{10\;\mathrm{Gyr}}{T_{m, 0}}}^{1/6}
            + \mathcal{O}\p{\eta j_{\min}},\\
        &\approx \frac{\sqrt{60}}{5}\p{\frac{0.202\;\mathrm{AU}}{a_0}}^{2/3}
            \p{\frac{q(1+q)}{2}}^{1/6}
            + \mathcal{O}\p{\eta j_{\min}}.
\end{align}
For $I_{\rm 0, merger} \sim 90^\circ$, we can replace $\cos(x) \approx 90^\circ
- x$, and so the difference of these cosines is just the negative difference of
their arguments. For $q = 2/3$ and $a_0 = 100\;\mathrm{AU}$, this gives $\Delta
I_{\rm 0, merger} = 1.231^\circ$, while LL18 obtain $1.20^\circ$. Thus, this is
the right scaling.

\end{document}

