    \documentclass[10pt,
        twocolumn,
        fleqn,
    ]{revtex4-2}% chktex 8
    \usepackage{
        amsmath,
        amssymb,
        hyperref, % various links
        amsthm, % newtheorem and proof environment
        mathtools, % \Aboxed for boxing inside aligns, among others
        enumerate, % Allow custom enumerate numbering
        graphicx, % allow includegraphics with more filetypes
        booktabs, % \bottomrule instead of hline apparently
        xcolor, % colored text
        wasysym
    }
    \usepackage[
        labelfont=bf, % caption names are labeled in bold
        font=scriptsize % smaller font for captions
    ]{caption}

    \newcommand*{\scinot}[2]{#1\times10^{#2}}
    \newcommand*{\dotp}[2]{\left<#1\,\middle|\,#2\right>}
    \newcommand*{\rd}[2]{\frac{\mathrm{d}#1}{\mathrm{d}#2}}
    \newcommand*{\pd}[2]{\frac{\partial#1}{\partial#2}}
    \newcommand*{\rdil}[2]{\mathrm{d}#1 / \mathrm{d}#2}
    \newcommand*{\pdil}[2]{\partial#1 / \partial#2}
    \newcommand*{\rtd}[2]{\frac{\mathrm{d}^2#1}{\mathrm{d}#2^2}}
    \newcommand*{\ptd}[2]{\frac{\partial^2 #1}{\partial#2^2}}
    \newcommand*{\md}[2]{\frac{\mathrm{D}#1}{\mathrm{D}#2}}
    \newcommand*{\pvec}[1]{\vec{#1}^{\,\prime}}
    \newcommand*{\svec}[1]{\vec{#1}\;\!}
    \let\bm\undefined
    \newcommand*{\bm}[1]{\boldsymbol{\mathbf{#1}}}
    \newcommand*{\uv}[1]{\hat{\bm{#1}}}
    \newcommand*{\ang}[0]{\;\text{\AA}}
    \newcommand*{\mum}[0]{\;\upmu \mathrm{m}}
    \newcommand*{\at}[1]{\left.#1\right|}
    \newcommand*{\bra}[1]{\left<#1\right|}
    \newcommand*{\ket}[1]{\left|#1\right>}
    \newcommand*{\abs}[1]{\left|#1\right|}
    \newcommand*{\ev}[1]{\langle#1\rangle}
    \newcommand*{\p}[1]{\left(#1\right)}
    \newcommand*{\s}[1]{\left[#1\right]}
    \newcommand*{\z}[1]{\left\{#1\right\}}

    \let\Re\undefined
    \let\Im\undefined
    \DeclareMathOperator{\Res}{Res}
    \DeclareMathOperator{\Re}{Re}
    \DeclareMathOperator{\Im}{Im}
    \DeclareMathOperator{\Log}{Log}
    \DeclareMathOperator{\Arg}{Arg}
    \DeclareMathOperator{\Tr}{Tr}
    \DeclareMathOperator{\E}{E}
    \DeclareMathOperator{\Var}{Var}
    \DeclareMathOperator*{\argmin}{argmin}
    \DeclareMathOperator*{\argmax}{argmax}
    \DeclareMathOperator{\sgn}{sgn}
    \DeclareMathOperator{\diag}{diag\;}

    \colorlet{Corr}{red}

\begin{document}

\linespread{1.15}
\setlength{\mathindent}{0pt}

\title{Octupole-order Lidov-Kozai Population Statistics}
\author{Yubo Su}
\date{Today}

\maketitle

\section{10/22/20---Initial Thoughts}

\subsection{Equations}

The equations of motion we want to study come from LML15. Describe the inner
binary by $\p{a, e, I, \ascnode, \omega}$ and the outer binary with ``out''
subscripts, and denote $I_{\rm tot} = I + I_{\rm out}$. Call the inner binary
component masses $m_1$, $m_2$, and the tertiary mass $m_3$, define the inner
binary total and reduced masses $m_{12} = m_1 + m_2$ and $\mu = m_1m_2 /
m_{12}$, and define the tertiary orbit total and reduced masses $m_{123} =
m_{12} + m_3$ and $\mu_{\rm out} = m_{12}m_3 / m_{123}$. The equations of motion
are ($j(e) = \sqrt{1 - e^2}$)
{\small
\begin{align}
    \rd{a}{t} ={}& -\frac{64}{5}\frac{a}{t_{\rm GW}j^7(e)}
            \p{1 + \frac{73}{24}e^2 +
            \frac{37}{96}e^4},
\end{align}
\begin{align}
    \rd{e}{t} ={}& \frac{j(e)}{64t_{\rm LK}}\Big\{
            120 e \sin^2 I_{\rm tot} \sin \p{2\omega}\nonumber\\
        &+ \frac{15 \epsilon_{\rm oct}}{8}\cos \omega_{\rm out}
            \big[\p{4 + 3e^2}\p{3 + 5\cos (2I_{\rm tot})}\nonumber\\
        &\times \sin \omega + 210 e^2\sin^2I_{\rm tot} \sin 3\omega\big]
            \nonumber\\
        &- \frac{15 \epsilon_{\rm oct}}{4} \cos I_{\rm tot} \cos \omega
            \big[15(2 + 5e^2) \cos \p{2I_{\rm tot}}\nonumber\\
        &+ 7\p{30 e^2 \cos(2\omega) \sin^2 I_{\rm tot} - 2 - 9e^2}]
                \sin \omega_{\rm out}\Big\}\nonumber\\
        &- \frac{304}{15}\frac{e}{t_{\rm GW}j^5(e)}
            \p{1 + \frac{121}{304}e^2},
\end{align}
\begin{align}
    \rd{I}{t} ={}& -\frac{3e}{32t_{\rm LK}j(e)}\Big\{
            10\sin\p{2I_{\rm tot}}\big[e \sin (2\omega)\nonumber\\
        &+ \frac{5\epsilon_{\rm oct}}{8}\p{2 + 5e^2 + 7e^2\cos(2\omega)}
            \cos \omega_{\rm out}\sin \omega\big]\nonumber\\
        &+ \frac{5\epsilon_{\rm oct}}{8} \cos \omega \big[
            26 + 37e^2 -35 e^2\cos(2\omega)\nonumber\\
        &- 15\cos\p{2I_{\rm tot}}\p{7e^2\cos (2\omega) - 2 - 5e^2}\big]\nonumber\\
        &\times \sin I_{\rm tot}\sin \omega_{\rm out}\Big\}
\end{align}
\begin{align}
    \rd{\ascnode}{t} ={}& \rd{\ascnode_{\rm out}}{t}
            = -\frac{3\csc I}{32 t_{\rm LK}j(e)}\Big\{
            2\big[(2 + 3e^2 - 5e^2 \cos (2\omega))\nonumber\\
        &+ \frac{25 \epsilon_{\rm oct}e}{8} \cos \omega
            \p{2 + 5e^2 - 7e^2\cos(2\omega)}\cos \omega_{\rm out}\big]\nonumber\\
        &\times\sin (2I_{\rm tot}) - \frac{5\epsilon_{\rm oct}e}{8}
            \big[35e^2(1 + 3\cos\p{2I_{\rm tot}})\cos 2\omega\nonumber\\
        &- 46 - 17e^2 - 15\p{6 + e_1^2}\cos\p{2I_{\rm tot}}\big]\nonumber\\
        &\times \sin I_{\rm tot} \sin \omega \sin \omega_{\rm out}\Big\},
\end{align}
\begin{align}
    \rd{\omega}{t} ={}& \frac{3}{8t_{\rm LK}}
            \Big\{\frac{1}{j(e)} \big[4 \cos^2 I_{\rm tot}
            + \p{5 \cos (2\omega) - 1}\nonumber\\
        &\times \p{1 - e^2 - \cos^2 I_{\rm tot}}\big] + \frac{L \cos I_{\rm tot}}{
            L_{\rm out}j(e_{\rm out})}\big[2 + e^2(3\nonumber\\
        &- 5\cos(2\omega))\big]\Big\}
        + \frac{15 \epsilon_{\rm oct}}{64t_{\rm LK}}\Big\{
            \p{\frac{L}{L_{\rm out}j(e_{\rm out})}
                + \frac{\cos I_{\rm tot}}{j(e)}}\nonumber\\
        &\times e\big[\sin \omega \sin \omega_{\rm out}
            \s{10(3\cos^2 I_{\rm tot} - 1)(1 - e^2) + A}\nonumber\\
        &- 5B\cos I_{\rm tot}\cos \Theta\big] - \frac{j(e)}{e}
            \big[10 \sin \omega \sin \omega_{\rm out}\cos I_{\rm tot}\nonumber\\
        &\times \sin^2 I_{\rm tot}\p{1 - 3e^2} + \cos \Theta\p{3A -
            10\cos^2 I_{\rm tot} + 2}\big]\Big\}\nonumber\\
        &+ \Omega_{\rm GR},
\end{align}
\begin{align}
    \rd{e_{\rm out}}{t} ={}& \frac{15eL\,j(e_{\rm out})\epsilon_{\rm oct}}{
            256t_{\rm LK}e_{\rm out}L_{\rm out}}\Big\{ \cos \omega
                \big[6 - 13e^2\nonumber\\
        &+ 5(2 + 5e^2)\cos (2I_{\rm tot}) + 70e^2 \cos (2\omega) \sin^2 I_{\rm
            tot}\big]\nonumber\\
        &\times \sin \omega_{\rm out} - \cos I_{\rm tot} \cos \omega_{\rm out}
            \big[5(6 + e^2)\cos \p{2I_{\rm tot}}\nonumber\\
        &+ 7\p{10e^2 \cos (2\omega) \sin^2 I_{\rm tot} - 2 + e^2}\big]
            \sin \omega\Big\},
\end{align}
\begin{align}
    \rd{I_{\rm out}}{t} ={}& -\frac{3eL}{32t_{\rm LK}j(e_{\rm out}) L_{\rm out}}
            \Big\{ 10\big[2e \sin I_{\rm tot} \sin (2\omega)\nonumber\\
        &+ \frac{5\epsilon_{\rm oct}}{8}\cos \omega\p{
            2 + 5e^2 - 7e^2\cos (2\omega)}\sin (2I_{\rm tot})\sin \omega_{\rm
            out}\big]\nonumber\\
        &+ \frac{5\epsilon_{\rm oct}}{8}\big[26 + 107e^2
            + 5(6 + e^2)\cos(2I_{\rm tot})\nonumber\\
        &- 35 e^2\p{\cos 2(I_{\rm tot}) - 5} \cos (2\omega)\big]
            \cos \omega_{\rm out}\sin I_{\rm tot} \sin \omega\Big\},
\end{align}
\begin{align}
    \rd{\omega_{\rm out}}{t} ={}&
        \frac{3}{16t_{\rm LK}}\Big\{\frac{2\cos I_{\rm tot}}{j(e)}
            \s{2 + e^2(3 - 5\cos(2\omega))}\nonumber\\
        &+ \frac{L}{L_{\rm out}j(e_{\rm out})}\big[4 + 6e^2
            + (5\cos^2 I_{\rm tot} - 3)\nonumber\\
        &\times [2 + e^2(3 - 5\cos(2\omega))]\big]\Big\}
            - \frac{15 \epsilon_{\rm oct}e}{64t_{\rm LK}e_{\rm out}}\nonumber\\
        &\times \Big\{ \sin \omega \sin \omega_{\rm out}\Big[
            \frac{L(4e_{\rm out}^2 + 1)}{e_{\rm out}L_{\rm out} j(e_{\rm out})}
            10 \cos I_{\rm tot}\sin^2 I_{\rm tot}\nonumber\\
        &\times(1 - e^2) - e_{\rm out}\p{\frac{1}{j(e)} +
            \frac{L \cos I_{\rm tot}}{L_{\rm out}j(e_{\rm out})}}\nonumber\\
        &\times\s{A + 10\p{3\cos^2 I_{\rm tot} - 1}\p{1 - e^2}}\Big]
            + \cos \Theta\nonumber\\
        &\times \Big[5B\cos I_{\rm tot}e_{\rm out}\p{\frac{1}{j(e)} +
            \frac{L \cos I_{\rm tot}}{L_{\rm out}j(e_{\rm out})}}\nonumber\\
        &+ \frac{L\p{4e_{\rm out}^2 + 1}}{e_{\rm out}L_{\rm out}j(e_{\rm out})}
            A\Big]\Big\}.
\end{align}
}
where $n = \sqrt{Gm_{12} / a^3}$ is the mean motion, $L = \mu \sqrt{Gm_{12}a}$
and $L_{\rm out} = \mu_{\rm out}\sqrt{Gm_{123}a_{\rm out}}$ are the circular
angular momenta, and
\begin{align}
    t_{\rm LK}^{-1} &= n\p{\frac{m_3}{m_{12}}}\p{\frac{a}{a_{\rm out}j(e_{\rm
        out})}}^3,\\
    t_{\rm GW}^{-1} &= \frac{G^3\mu m_{12}^2}{c^5a^4},\\
    \Omega_{\rm GR} &= \frac{3Gnm_{12}}{c^2aj^2(e)},\\
    \epsilon_{\rm oct} &= \frac{m_2 - m_1}{m_{12}}\frac{a}{a_{\rm out}}
        \frac{e_{\rm out}}{1 - e_{\rm out}^2},\\
    A &\equiv 4 + 3e^2 - \frac{5}{2}B\sin^2 I_{\rm tot},\\
    B &\equiv 2 + 5e^2 - 7e^2 \cos(2\omega),\\
    \cos \Theta &\equiv -\cos \omega \cos \omega_{\rm out}
        - \cos I_{\rm tot} \sin \omega \sin \omega_{\rm out}.
\end{align}

These equations can be nondimensionalized via the following steps (I won't
rewrite the equations): (i) multiply through by $t_{\rm LK, 0}$ ($a = a_0$ and
$e_{\rm out} = 0$), and call $\tau \equiv t / t_{\rm LK, 0}$ the new variable of
differentiation, (ii) re-express all of the timescales as
\begin{align}
    \frac{t_{\rm LK, 0}}{t_{\rm LK}} &= \p{\frac{a}{a_0}}^{3/2}
        j^{-3}\p{e_{\rm out}},\\
    \frac{t_{\rm LK, 0}}{t_{\rm GW}} &=
        \frac{G^3\mu m_{12}^3}{m_3c^5a^4}\frac{1}{n_0}\p{\frac{a_{\rm out}
            }{a_0}}^3,\nonumber\\
        &= \epsilon_{\rm GW}\p{\frac{a_0}{a}}^4,\\
    \epsilon_{\rm GW} &\equiv \frac{G^3\mu m_{12}^3
        a_{\rm out}^3}{m_3c^5a_0^7 n_0},\\
    \Omega_{\rm GR}t_{\rm LK, 0} &= \frac{3Gnm_{12}^2}{m_3c^2a}
        \frac{1}{n_0}\p{\frac{a_{\rm out}}{a_0}}^3,\nonumber\\
        &= \epsilon_{\rm GR} \p{\frac{a_0}{a}}^{5/2},\\
    \epsilon_{\rm GR} &= \frac{3Gm_{12}^2a_{\rm out}^3}{m_3c^2a_0^4}.
\end{align}
(iii) re-express $\rdil{a}{t}$ as
\begin{equation}
    \rd{(a / a_0)}{\tau} = -\frac{64}{5}\frac{\epsilon_{\rm GW}}{
        j^{7/2}(e)}\p{\frac{a_0}{a}}^3\p{1 + \frac{73}{24}e^2 +
        \frac{37}{96}e^4}.
\end{equation}
As such, the natural unit of length is $a_0 = 1$, the natural unit of time is
$t_{\rm LK, 0} = 1$, and everything else is dimensionless. When computing these
$\epsilon$, I use convention where $1 M_{\odot} = 1\;\mathrm{AU} = c = 1$, under
which $G = 9.87 \times 10^{-9}$.

\subsection{Points of Inquiry}

The goal is to understand how the merger window varies as a function of $q
\equiv m_1 / m_2$ ($m_1 < m_2$) when the octupole order LK effects are
important.
\begin{itemize}
    \item First, let's set $\epsilon_{\rm GW} = 0$. It is well known that the
        octupole order LK is nonintegrable. What does the Fourier spectrum of
        the eccentricity look like? Will this help us get a delay time
        distribution between high-e phases?

        When the octupole effect is unimportant, the spectrum falls off
        exponentially over scales $\tau \simeq P_{\rm LK} j^{-1}(e_{\max})$,
        where $P_{\rm LK}$ is the quadrupole LK period. One imagines the
        tail of the spectrum gets heavier when $\epsilon_{\rm oct}$ is
        increased, and this might help us get the delay time distribution.

        A second way we can postprocess this is to take a histogram of $e(t)$.
        If there is some regular structure, it's likely this will allow us to
        compute the average rate of binary coalescence due to GW radiation.

    \item The goal is to understand the size of the merger window, $\Delta I$,
        as a function of $q$. To do this, we numerically sample the merger time
        function $T_{\rm m}\p{I_0, q}$. At each $I_0$, the natural thing to do
        would be to try for $\sim 5-10$ random $\Omega, \omega$, and define the
        merger window to be where $T_{\rm m} \leq 10^10\;\mathrm{yr}$.
\end{itemize}

\section{11/02/20}

We've done a lot more inquiry on this, read the Nov 3 weekly for a recap. In
summary though, there should roughly be two ways to have LK-induced mergers:
\begin{itemize}
    \item Quadrupole LK-induced mergers. The $e_{\max}$ of these systems is well
        understood, however, and the merger fraction should be easy to
        calculate: we can get the merger time using the $T_m \propto \p{1 -
        e_{\max}}^{-3}$ physically-justified fitting law from LL18 in the
        LK-induced regime, and we just have to evaluate where it crosses
        $10\;\mathrm{Gyr}$.

    \item Octupole LK-induced mergers. For these, there is a characteristic
        initial inclination range for which orbit flipping occurs, which is a
        function of $\epsilon_{\rm oct}$. This can likely be calculated
        analytically, but I'm not sure yet.

        For these systems, there is a characteristic orbit flipping timescale
        that is robust up to a factor of a few (since $K$ oscillates on a fixed
        timescale, and orbit flips occur whenever $K$ crosses $-\eta/2$), call
        this $t_{\rm ELK}$. Thus, octupole LK-induced mergers occur over
        characteristic times $t_{\rm ELK}$ within the desired inclination window
        (as once the system reaches an orbit flipping eccentricity, it executes
        a one-shot merger, approximately). It is well known that $t_{\rm ELK}$
        depends on $\epsilon_{\rm oct}$, see e.g.\ Antognini 2015.

        There is some small variability, however, in this picture, according to
        my plots. This can likely be attributed to the exact history of
        eccentricity maxima prior to the one-shot maximum, since particularly
        dissipative sequences (i.e.\ many large eccentricity maxima prior to
        undergoing ``the big one'') can shrink the orbit and change the LK cycle
        pattern.
\end{itemize}

Armed with this, we should have enough information to compute the indicator
function $\mathbb{I}_{\rm merge}\p{a_{\rm in}, I_0, t_{\rm LK}, \epsilon_{\rm
oct}}$, whether a system will merge, by simply computing the two ranges of $I_0$
from above. This should be valid wherever we are in a strongly LK-induced
regime, i.e.\ very large eccentricities are necessary to merge.

In fact, there's a possibility the critical $\epsilon_{\rm oct}$ can be
calculated analytically, see Katz et.\ al.\ 2011 for the test particle case. Is
this generalizable?

\end{document}

