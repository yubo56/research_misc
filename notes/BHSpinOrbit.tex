    \documentclass[11pt,
        usenames, % allows access to some tikz colors
        dvipsnames % more colors: https://en.wikibooks.org/wiki/LaTeX/Colors
    ]{article}
    \usepackage{
        amsmath,
        amssymb,
        fouriernc, % fourier font w/ new century book
        fancyhdr, % page styling
        lastpage, % footer fanciness
        hyperref, % various links
        setspace, % line spacing
        amsthm, % newtheorem and proof environment
        mathtools, % \Aboxed for boxing inside aligns, among others
        float, % Allow [H] figure env alignment
        enumerate, % Allow custom enumerate numbering
        graphicx, % allow includegraphics with more filetypes
        wasysym, % \smiley!
        upgreek, % \upmu for \mum macro
        listings, % writing TrueType fonts and including code prettily
        tikz, % drawing things
        booktabs, % \bottomrule instead of hline apparently
        xcolor, % colored text
        cancel % can cancel things out!
    }
    \usepackage[margin=1in]{geometry} % page geometry
    \usepackage[
        labelfont=bf, % caption names are labeled in bold
        font=scriptsize % smaller font for captions
    ]{caption}
    \usepackage[font=scriptsize]{subcaption} % subfigures

    \newcommand*{\scinot}[2]{#1\times10^{#2}}
    \newcommand*{\dotp}[2]{\left<#1\,\middle|\,#2\right>}
    \newcommand*{\rd}[2]{\frac{\mathrm{d}#1}{\mathrm{d}#2}}
    \newcommand*{\pd}[2]{\frac{\partial#1}{\partial#2}}
    \newcommand*{\rdil}[2]{\mathrm{d}#1 / \mathrm{d}#2}
    \newcommand*{\pdil}[2]{\partial#1 / \partial#2}
    \newcommand*{\rtd}[2]{\frac{\mathrm{d}^2#1}{\mathrm{d}#2^2}}
    \newcommand*{\ptd}[2]{\frac{\partial^2 #1}{\partial#2^2}}
    \newcommand*{\md}[2]{\frac{\mathrm{D}#1}{\mathrm{D}#2}}
    \newcommand*{\pvec}[1]{\vec{#1}^{\,\prime}}
    \newcommand*{\svec}[1]{\vec{#1}\;\!}
    \newcommand*{\bm}[1]{\boldsymbol{\mathbf{#1}}}
    \newcommand*{\uv}[1]{\hat{\bm{#1}}}
    \newcommand*{\ang}[0]{\;\text{\AA}}
    \newcommand*{\mum}[0]{\;\upmu \mathrm{m}}
    \newcommand*{\at}[1]{\left.#1\right|}
    \newcommand*{\bra}[1]{\left<#1\right|}
    \newcommand*{\ket}[1]{\left|#1\right>}
    \newcommand*{\abs}[1]{\left|#1\right|}
    \newcommand*{\ev}[1]{\left\langle#1\right\rangle}
    \newcommand*{\p}[1]{\left(#1\right)}
    \newcommand*{\s}[1]{\left[#1\right]}
    \newcommand*{\z}[1]{\left\{#1\right\}}

    \newtheorem{theorem}{Theorem}[section]

    \let\Re\undefined
    \let\Im\undefined
    \DeclareMathOperator{\Res}{Res}
    \DeclareMathOperator{\Re}{Re}
    \DeclareMathOperator{\Im}{Im}
    \DeclareMathOperator{\Log}{Log}
    \DeclareMathOperator{\Arg}{Arg}
    \DeclareMathOperator{\Tr}{Tr}
    \DeclareMathOperator{\E}{E}
    \DeclareMathOperator{\Var}{Var}
    \DeclareMathOperator*{\argmin}{argmin}
    \DeclareMathOperator*{\argmax}{argmax}
    \DeclareMathOperator{\sgn}{sgn}
    \DeclareMathOperator{\diag}{diag\;}

    \colorlet{Corr}{red}

    % \everymath{\displaystyle} % biggify limits of inline sums and integrals
    \tikzstyle{circ} % usage: \node[circ, placement] (label) {text};
        = [draw, circle, fill=white, node distance=3cm, minimum height=2em]
    \definecolor{commentgreen}{rgb}{0,0.6,0}
    \lstset{
        basicstyle=\ttfamily\footnotesize,
        frame=single,
        numbers=left,
        showstringspaces=false,
        keywordstyle=\color{blue},
        stringstyle=\color{purple},
        commentstyle=\color{commentgreen},
        morecomment=[l][\color{magenta}]{\#}
    }

\begin{document}

\section{05/29/21}

\subsection{Equal Masses: Following Racine}

The paper is \url{https://arxiv.org/pdf/0803.1820.pdf}. The EOM and definitions
in the equal mass case are considerably simplified and we obtain:
\begin{align}
    \rd{\bm{S}_1}{\psi} &= \uv{J} \times \bm{S}_1
        - \alpha \bm{S}_2 \times \bm{S}_1,\\
    \rd{\bm{S}_2}{\psi} &= \uv{J} \times \bm{S}_2
        - \alpha \bm{S}_1 \times \bm{S}_2,\\
    \rd{\psi}{t} &= \frac{1}{2a_{\rm eff}^3}\p{7 - \frac{3}{2}\lambda}J,\\
    \alpha &= \frac{3}{J}\frac{4 - \lambda}{14 - 3\lambda},\\
    \lambda &= \frac{\bm{L}}{L^2} \cdot
        \s{\p{1 + \frac{M_2}{M_1}}\bm{S}_1
            + \p{1 + \frac{M_1}{M_2}}\bm{S}_2}.
\end{align}
Here, $a_{\rm eff} = a\sqrt{1 - e^2}$, and $\bm{J} = J\uv{J} = \bm{L} + \bm{S}_1
+ \bm{S}_2$ is the total angular momentum.

The Hamiltonian is then probably
\begin{equation}
    H = -\uv{J} \cdot \p{\bm{S}_1 + \bm{S}_2}
        + \alpha\p{\bm{S}_1 \cdot \bm{S}_2},
\end{equation}
I think (double check the signs, check that it reproduces Racine's EOM?).

Ansatz: adiabatic invariant, can we set up $\oint p\;dq$? Why is $\psi$ not
uniformly advancing and is so complicated? $\lambda$ is not invariant since
$d\lambda/dt$ directly has a GW term.

Up to rotational invariance, equilibria occur when each of $\theta_1$,
$\theta_2$ and $\Delta \phi = \phi_2 - \phi_1$ are stationary. If we go to the
corotating frame about $\uv{J}$, then it's very obvious what is going on:
\begin{equation}
    H_{\rm rot} = \alpha \p{\bm{S}_1 \cdot \bm{S}_2}.
\end{equation}
Clearly, the equilibria occur when $\bm{S}_1$ and $\bm{S}_2$ are
aligned/antialigned. I don't know why the phase portrait for them has to be so
messy, probably a difficulty with coordinates?

This checks out in coordinate form, even in the inertial frame:
\begin{equation}
    H\p{\theta_1, \theta_2, \Delta \phi}
        = -S_1 \cos \theta_1 - S_2 \cos \theta_2
            + \alpha\p{\cos \theta_1 \cos \theta_2
                + \sin \theta_1 \sin \theta_2 \cos \Delta \phi}.
\end{equation}
Since centers of libration occur when $H$ attains an extremal value, it's clear
that the only extremal values can occur when $\Delta \phi = 0$ or $\pi$. If we
set that to be the case, then an extremal value can only occur if $\theta_1 \pm
\theta_2 = 0$, depending on which $\phi$ we use, so indeed, we recover
alignment/anti-alignment as the two possible centers of libration.

To try and extract an adiabatic invariant, we consider the evolution of the subs
and differences of $\bm{S}_i$, as pointed out by Racine. It's clear that
$\bm{S}_1 + \bm{S}_2$ just precesses about $\uv{J}$ (with the opening angle an
adiabatic invariant), while $\bm{\Delta} \equiv \bm{S}_1 - \bm{S}_2$ evolves
with ($\bm{S}$ is the total spin)
\begin{equation}
    \rd{\bm{\Delta}}{\psi} = \p{\uv{J} - \alpha \bm{S}} \times \bm{\Delta}.
\end{equation}
The obvious guess of adiabatic invariant in this case is then just going to be:
\begin{equation}
    \cos \theta_\Delta \equiv \p{\uv{J} - \alpha \bm{S}} \cdot \bm{\Delta}
\end{equation}

With these two adiabatic invariants, what happens during GW emission then? I
don't think that the total spin one will do anything interesting (the most
pathological case is when $\bm{S}_1 \parallel \bm{S}_2$: go to the co-rotating
frame about $\bm{L}$ initially, then the enclosed phase space area is zero, the
precession axis just moves). However, during GW decay, we expect $\alpha$ to
increase, as $J$ decreases, and so $\bm{\Delta}$ will initially precess about
$\bm{J}$ but eventually precess about the sum of the total spin and $\uv{J}$
(note that even at the very late stages of inspiral, I think $\bm{L} \gtrsim
\bm{S}_i$, so it will never be fully ignorable). Nevertheless, this is clearly
the mechanism for making $\bm{\Delta}$ tilt

\section{05/31/21}

We first try to find the equilibria of the system before seeking adiabatic
invariants. As Schnittman 2004 points out, we should consider the zeros of
\begin{align}
    \rd{}{t}\p{\bm{S}_1 \cdot \bm{S}_2}
        &= \rd{\psi}{t} \beta_- \uv{J} \cdot \p{\bm{S}_1 \times \bm{S}_2}.
        \label{eq:06/01/21.eq}
\end{align}
Here, $\beta_-$ is a constant from Racine's paper, defined to be $\beta_1 -
\beta_2$ where
\begin{equation}
    \beta_{1,2} = \p{4 + 3\frac{M_{2, 1}}{M_{1, 2}}
        - 3\frac{\mu}{M_{1, 2}}\lambda}\Bigg/\p{7 - \frac{3\lambda}{2}}.
\end{equation}
In order for Eq.~\eqref{eq:06/01/21.eq} to vanish, we need two conditions:
\begin{align}
    \bm{S}_2 \cdot \p{\uv{J} \times \bm{S}_1} =
        \sin \theta_1 \sin \theta_2 \sin \Delta \phi &= 0,\\
    \rd{}{t}\s{\bm{S}_2 \cdot \p{\uv{J} \times \bm{S}_1}}
        &= 0.
\end{align}
To achieve some comparability with the Schnittman equations, we can replace
$\uv{J}$ with $\bm{J}$ above without changing the equilibria. Note that this
is equivalent to the Schnittman expression
\begin{align}
    \bm{S}_2 \cdot \s{\p{\bm{L} + \bm{S}_1 + \bm{S}_2} \times \bm{S}_1}
        &= \bm{S}_2 \cdot \s{\p{\bm{L} + \bm{S}_2} \times \bm{S}_1},\\
        &= \bm{S}_2 \cdot \s{\bm{L} \times \bm{S}_1}
            + \bm{S}_2 \cdot \p{\bm{S}_2 \times \bm{S}_1},\\
        &= \bm{S}_2 \cdot \s{\bm{L} \times \bm{S}_1}.
\end{align}
Thus, we find that the Schnittman equilibria are probably correct, but the
detailed EOM about them may not necessarily be.

\end{document}

