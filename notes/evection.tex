    \documentclass[11pt,
        usenames, % allows access to some tikz colors
        dvipsnames % more colors: https://en.wikibooks.org/wiki/LaTeX/Colors
    ]{article}
    \usepackage{
        amsmath,
        amssymb,
        fouriernc, % fourier font w/ new century book
        fancyhdr, % page styling
        lastpage, % footer fanciness
        hyperref, % various links
        setspace, % line spacing
        amsthm, % newtheorem and proof environment
        mathtools, % \Aboxed for boxing inside aligns, among others
        float, % Allow [H] figure env alignment
        enumerate, % Allow custom enumerate numbering
        graphicx, % allow includegraphics with more filetypes
        wasysym, % \smiley!
        upgreek, % \upmu for \mum macro
        listings, % writing TrueType fonts and including code prettily
        tikz, % drawing things
        booktabs, % \bottomrule instead of hline apparently
        xcolor, % colored text
        cancel % can cancel things out!
    }
    \usepackage[margin=1in]{geometry} % page geometry
    \usepackage[
        labelfont=bf, % caption names are labeled in bold
        font=scriptsize % smaller font for captions
    ]{caption}
    \usepackage[font=scriptsize]{subcaption} % subfigures

    \newcommand*{\scinot}[2]{#1\times10^{#2}}
    \newcommand*{\dotp}[2]{\left<#1\,\middle|\,#2\right>}
    \newcommand*{\rd}[2]{\frac{\mathrm{d}#1}{\mathrm{d}#2}}
    \newcommand*{\pd}[2]{\frac{\partial#1}{\partial#2}}
    \newcommand*{\rdil}[2]{\mathrm{d}#1 / \mathrm{d}#2}
    \newcommand*{\pdil}[2]{\partial#1 / \partial#2}
    \newcommand*{\rtd}[2]{\frac{\mathrm{d}^2#1}{\mathrm{d}#2^2}}
    \newcommand*{\ptd}[2]{\frac{\partial^2 #1}{\partial#2^2}}
    \newcommand*{\md}[2]{\frac{\mathrm{D}#1}{\mathrm{D}#2}}
    \newcommand*{\pvec}[1]{\vec{#1}^{\,\prime}}
    \newcommand*{\svec}[1]{\vec{#1}\;\!}
    \newcommand*{\bm}[1]{\boldsymbol{\mathbf{#1}}}
    \newcommand*{\uv}[1]{\hat{\bm{#1}}}
    \newcommand*{\ang}[0]{\;\text{\AA}}
    \newcommand*{\mum}[0]{\;\upmu \mathrm{m}}
    \newcommand*{\at}[1]{\left.#1\right|}
    \newcommand*{\bra}[1]{\left<#1\right|}
    \newcommand*{\ket}[1]{\left|#1\right>}
    \newcommand*{\abs}[1]{\left|#1\right|}
    \newcommand*{\ev}[1]{\left\langle#1\right\rangle}
    \newcommand*{\p}[1]{\left(#1\right)}
    \newcommand*{\s}[1]{\left[#1\right]}
    \newcommand*{\z}[1]{\left\{#1\right\}}

    \newtheorem{theorem}{Theorem}[section]

    \let\Re\undefined
    \let\Im\undefined
    \DeclareMathOperator{\Res}{Res}
    \DeclareMathOperator{\Re}{Re}
    \DeclareMathOperator{\Im}{Im}
    \DeclareMathOperator{\Log}{Log}
    \DeclareMathOperator{\Arg}{Arg}
    \DeclareMathOperator{\Tr}{Tr}
    \DeclareMathOperator{\E}{E}
    \DeclareMathOperator{\Var}{Var}
    \DeclareMathOperator*{\argmin}{argmin}
    \DeclareMathOperator*{\argmax}{argmax}
    \DeclareMathOperator{\sgn}{sgn}
    \DeclareMathOperator{\diag}{diag\;}

    \colorlet{Corr}{red}

    % \everymath{\displaystyle} % biggify limits of inline sums and integrals
    \tikzstyle{circ} % usage: \node[circ, placement] (label) {text};
        = [draw, circle, fill=white, node distance=3cm, minimum height=2em]
    \definecolor{commentgreen}{rgb}{0,0.6,0}
    \lstset{
        basicstyle=\ttfamily\footnotesize,
        frame=single,
        numbers=left,
        showstringspaces=false,
        keywordstyle=\color{blue},
        stringstyle=\color{purple},
        commentstyle=\color{commentgreen},
        morecomment=[l][\color{magenta}]{\#}
    }

\begin{document}

\onehalfspacing

\pagestyle{fancy}
\rfoot{Yubo Su}
\rhead{}
\cfoot{\thepage/\pageref{LastPage}}

\title{Evection Resonances in BH Triples}
\author{Yubo Su}
\date{}

\maketitle

\section{03/15/21---Basics \& Introduction}

\subsection{Writing Down the Hamiltonian}

We assume a triple system $m_{1,2,3}$ and $a$, $a_{\rm out}$ with mutual
inclination $I$. The 1PN apsidal precession of the inner binary has
energy/Hamiltonian
\begin{equation}
    H_{\rm GR} = -\frac{3G^2m_1m_2m_{12}}{c^2a^2j(e)},
\end{equation}
while the external companion has averaged energy
\begin{equation}
    H_{\rm out} = -\frac{Gm_3\mu_{12} a^2}{a_{\rm out}^3}\s{
        \frac{1}{16}\s{\p{6 + 9e^2}\cos^2 I
            - (2 + 3e^2)} + \frac{15}{32}
                \p{1 + \cos I}^2 e^2 \cos\p{2\varpi - 2\lambda_{\rm out}}}.
\end{equation}
Here, we have averaged over: $\varpi = \Omega + \omega$ is the longitude of
pericenter of the inner orbit, so $\uv{e} = \cos \varpi \uv{x} + \sin \varpi
\uv{y}$, and $\lambda_{\rm out} = \varpi_{\rm out} + M_{\rm out}$ is the mean
longitude of $m_3$, where $M_{\rm out}$ is the outer mean anomaly. Recall that
$n_{\rm out} = \dot{\lambda}_{\rm out} = \dot{M}_{\rm out}$, and the useful
component form
\begin{equation}
    \uv{r}_{\rm out} = \cos \lambda_{\rm out}\uv{x}
        + \sin \lambda_{\rm out} \cos I \uv{y}
        + \sin \lambda_{\rm out} \sin I \uv{z}.
\end{equation}

Why is this interesting? Well, let's write $\epsilon \equiv
\frac{Gm_3\mu_{12}a^2}{a_{\rm out}^3} \big/ H_{\rm GR, 0}$, where $H_{\rm GR, 0}
= \s{H_{\rm GR}}_{e = 0}$, or
\begin{align}
    \epsilon &= \frac{m_3\mu_{12}a^2 c^2a^2}{3G^2m_1m_2m_{12}a_{\rm out}^3},\\
        &= \frac{m_3a^4 c^2}{3G^2m_{12}^2a_{\rm out}^3}.
\end{align}
This is like $\epsilon_{\rm GR}^{-1}$ from our previous LK work. We are
interested in the regime where $\epsilon \ll 1$. The total Hamiltonian of the
system is then
\begin{equation}
    \frac{H}{H_{\rm GR, 0}} = -\frac{1}{j(e)}
        - \epsilon\s{
        \frac{1}{16}\s{\p{6 + 9e^2}\cos^2 I
            - (2 + 3e^2)} + \frac{15}{32}
                \p{1 + \cos I}^2 e^2 \cos\p{2\varpi - 2\lambda_{\rm out}}}.
\end{equation}
We will eventually expand this Hamiltonian in terms of the conjugate variables
$-\varpi$ and $1 - \p{1 - e^2}^{1/2} \approx e^2/2$ and obtain a separatrix'd
Hamiltonian [Xu \& Lai (26)]. But for now, we can be satisfied that some sort of
separatrix might appear at $\epsilon \sim 1$? It's not clear yet.

\subsection{Timescale Comparison}

This section mostly follows Dong's notes, for completeness.

We need $n_{\rm out} \equiv \sqrt{Gm_{123} / a_{\rm out}^3}$ to be of order
$\dot{\varpi} \equiv 3Gnm_{12} / (c^2aj^2)$. Assuming the eccentricity is
already mostly damped (when $\epsilon_{\rm GR} \gg 1$, we expect this), then
this gives
\begin{align}
    \frac{3Gm_{12}}{c^2a} &\simeq \frac{n_{\rm out}}{n}
        = \sqrt{\frac{m_{123}}{m_{12}}\frac{a^3}{a_{\rm out}^3}},\\
    \p{\frac{a}{a_{\rm out}}}^{5/2} &\simeq
        \frac{3Gm_{12}}{c^2a_{\rm out}}\sqrt{\frac{m_{12}}{m_{123}}}.
\end{align}
Indeed, since everything is fixed, as $a$ decays, the evection resonance will be
crossed.

Will there be enough time to excite eccentricity? The eccentricity growth rate
due to the evection resonance must of order $t_{\rm ZLK}^{-1} \sim n \p{m_3 /
m_{12}} \p{a / a_{\rm out}}^3$. On the other hand, orbital decay due to GW is of
order
\begin{equation}
    t_{\rm GW}^{-1} \simeq \frac{64}{5}\frac{G^3m_{12}^2\mu}{c^5a^4}
        = \frac{64}{5}n\frac{G^{5/2}m_{12}^{3/2}\mu}{c^5a^{5/2}}.
\end{equation}
Thus, the resonance has time to grow if (in the third line, we invoke the
resonance condition above)
\begin{align}
    t_{\rm GW}^{-1} &\ll t_{\rm ZLK}^{-1},\\
    \frac{64}{5}\frac{G^{5/2}m_{12}^{3/2}\mu}{c^5a^{5/2}} &\ll
        \frac{m_3}{m_{12}}\p{\frac{a}{a_{\rm out}}}^3,\\
    \frac{64}{5}\frac{G^{5/2}m_{12}^{3/2}\mu}{c^5a_{\rm out}^{5/2}}
        \frac{c^2a_{\rm out}}{3Gm_{12}}\sqrt{\frac{m_{123}}{m_{12}}}
            &\ll\\
    \frac{64}{15} \frac{G^{3/2}m_{123}^{1/2}\mu}{c^3a_{\rm out}^{3/2}}&\ll\\
    \frac{64}{15}\p{\frac{v_{\rm out}}{c}}^3 \frac{m_{12} / 4}{m_{123}}
        &\ll\\
    \p{\frac{v_{\rm out}}{c}}^3\p{\frac{a_{\rm out}}{a}}^3
        \frac{m_{12}^2}{m_{123}m_3} &\ll 1.
\end{align}
Indeed, this must be the case. Another check requires
\begin{align}
    t_{\rm ZLK}^{-1} &\ll \dot{\varpi},\\
    \frac{m_3}{m_{12}}\p{\frac{a}{a_{\rm out}}}^3 &\ll \frac{3Gm_{12}}{c^2a}
        \sim \frac{n_{\rm out}}{n},\\
    &\ll \p{\frac{m_{123}}{m_{12}}}^{1/2}\p{\frac{a}{a_{\rm out}}}^{3/2},\\
    \frac{m_3}{m_{12}}\p{\frac{m_{12}}{m_{123}}}^{1/2}\p{\frac{a}{a_{\rm
        out}}}^{3/2} &\ll 1.
\end{align}
This is also satisfied. Thus, resonance excitation should be possible.

What are the kinds of systems that are interacting? If we want LISA band, we
need $n / \pi \sim 10^{-3}\;\mathrm{Hz}$, and:
\begin{align}
    n_{\rm out} &\simeq \frac{3Gnm_{12}}{c^2a},\\
        &\simeq \frac{3n^3a^2}{c^2},\\
        &\simeq \frac{3n^3}{c^2}\p{\frac{Gm_{12}}{n^2}}^{2/3},\\
        &\simeq 10^{-7}
            \p{\frac{P}{10^3\;\mathrm{s}}}^{-5/3}
            \p{\frac{m_{12}}{2 M_{\odot}}}^{2/3}\;\mathrm{s^{-1}},\\
    a_{\rm out} &= \p{\frac{Gm_{123}}{n_{\rm out}^2}}^{1/3},\\
        &= 2.4\p{\frac{m_{123}}{3 M_{\odot}}}^{1/3}
            \p{\frac{P}{10^3\;\mathrm{s}}}^{10/9}
            \p{\frac{m_{12}}{2M_{\odot}}}^{-4/9}\;\mathrm{AU}.
\end{align}
Indeed then, this is not going to be super useful unless $m_3$ is a SMBH, in
which case $a_{\rm out} \sim 100$--$1000\;\mathrm{AU}$. Note that $a \sim
\scinot{3}{8}\;\mathrm{m}$.

The other scenario then is that we cross this resonance, get a large
eccentricity, and it doesn't completely damp by the time it crosses the LISA
band? Well, we saw above that $\p{a / a_{\rm out}}^{5/2} \propto a_{\rm
out}^{-1}$, so if we fix the masses then increasing $a_{\rm out}$ by a factor of
$4$ increases $a$ by a factor of $32$, i.e.\ $a \sim 0.05\;\mathrm{AU}$ while
$a_{\rm out} = 10\;\mathrm{AU}$, somewhat believable. Since the rates of change
of $\ln a$ and $\ln e$ differ only by a factor of $j^2(e)$, if $e$ is only
modest, then it will have to also decay by $\sim 30$ by the time $a$ enters the
LISA band. However, if we can excite a substantial $e$ like $j^2(e) = 0.1$
(corresponding to $e = 0.95$), then $e$ will only decay by $\sim 3$ upon
entering the LISA band, which leaves us with an $e = 0.3$. Wenrui's paper
suggests evection isn't quite this strong, but maybe some sort of scenario is
possible.

The final solution is to use evection to pump an existing large eccentricity up
a little bit. But it's becoming clear that we aren't going to cleanly get
excitation in the LISA band, and that we will really need to consider dynamics
during \emph{and after} the resonance.

\subsection{Hamiltonian Level Curves}

Let's try to nondimensionalize the Hamiltonian now, like Wenrui's paper. Call
$\gamma = -\varpi$ and $\Gamma = 1 - \sqrt{1 - e^2}$, so that $j(e) = 1 -
\Gamma$ and $e^2 = 1 - \p{1 - \Gamma}^2 = 2\Gamma + \Gamma^2$, then
\begin{align}
    \frac{H}{H_{\rm GR, 0}} ={}&
        -\frac{1}{1 - \Gamma} - \frac{\epsilon}{16}
            \s{\p{6 + 9\p{2\Gamma + \Gamma^2}}\cos^2 I
                - \p{2 + 3\p{2\Gamma + \Gamma^2}}}\nonumber\\
            &- \frac{15\epsilon}{32}\p{1 + \cos I}^2 \p{2\Gamma + \Gamma^2}
                \cos\p{2\gamma + 2\lambda_{\rm out}},\\
    H' ={}& -\frac{1}{1 - \Gamma} - \frac{\epsilon}{16}\s{
        \p{9\cos^2 I - 3}\p{2\Gamma + \Gamma^2}}
        - \frac{15\epsilon}{32}\p{1 + \cos I}^2
            \p{2\Gamma + \Gamma^2}\cos\p{2\gamma + 2\lambda_{\rm out}},\\
        \approx{}&
            \Gamma\p{-1 - 2\epsilon A}
            + \Gamma^2\p{-1 + \epsilon A}
            - \Gamma \epsilon B\cos \theta
            + \mathcal{O}\p{\Gamma^3, \Gamma^2 \cos \theta},\\
    A ={}& \frac{9\cos^2 I - 3}{16},\\
    B ={}& \frac{15}{16}\p{1 + \cos I}^2,\\
    \theta ={}& 2\varpi - 2\lambda_{\rm out}.
\end{align}
This is not quite as clean as Wenrui's form, but it has the advantage (for me)
that the $\epsilon$ dependence is still explicit, while $A, B$ are almost always
positive (except for when $\cos^2 I < 1/3$).

\end{document}

