    \documentclass[10pt,
        usenames, % allows access to some tikz colors
        dvipsnames % more colors: https://en.wikibooks.org/wiki/LaTeX/Colors
    ]{article}
    \usepackage{
        amsmath,
        amssymb,
        fouriernc, % fourier font w/ new century book
        fancyhdr, % page styling
        lastpage, % footer fanciness
        hyperref, % various links
        setspace, % line spacing
        amsthm, % newtheorem and proof environment
        mathtools, % \Aboxed for boxing inside aligns, among others
        float, % Allow [H] figure env alignment
        enumerate, % Allow custom enumerate numbering
        graphicx, % allow includegraphics with more filetypes
        wasysym, % \smiley!
        upgreek, % \upmu for \mum macro
        listings, % writing TrueType fonts and including code prettily
        tikz, % drawing things
        booktabs, % \bottomrule instead of hline apparently
        xcolor, % colored text
        cancel % can cancel things out!
    }
    \usepackage[margin=1in, left=0.5in, right=0.5in]{geometry} % page geometry
    \usepackage[
        labelfont=bf, % caption names are labeled in bold
        font=scriptsize % smaller font for captions
    ]{caption}
    \usepackage[font=scriptsize]{subcaption} % subfigures

    \newcommand*{\scinot}[2]{#1\times10^{#2}}
    \newcommand*{\dotp}[2]{\left<#1\,\middle|\,#2\right>}
    \newcommand*{\rd}[2]{\frac{\mathrm{d}#1}{\mathrm{d}#2}}
    \newcommand*{\pd}[2]{\frac{\partial#1}{\partial#2}}
    \newcommand*{\rdil}[2]{\mathrm{d}#1 / \mathrm{d}#2}
    \newcommand*{\pdil}[2]{\partial#1 / \partial#2}
    \newcommand*{\rtd}[2]{\frac{\mathrm{d}^2#1}{\mathrm{d}#2^2}}
    \newcommand*{\ptd}[2]{\frac{\partial^2 #1}{\partial#2^2}}
    \newcommand*{\md}[2]{\frac{\mathrm{D}#1}{\mathrm{D}#2}}
    \newcommand*{\pvec}[1]{\vec{#1}^{\,\prime}}
    \newcommand*{\svec}[1]{\vec{#1}\;\!}
    \newcommand*{\bm}[1]{\boldsymbol{\mathbf{#1}}}
    \newcommand*{\uv}[1]{\hat{\bm{#1}}}
    \newcommand*{\ang}[0]{\;\text{\AA}}
    \newcommand*{\mum}[0]{\;\upmu \mathrm{m}}
    \newcommand*{\at}[1]{\left.#1\right|}
    \newcommand*{\bra}[1]{\left<#1\right|}
    \newcommand*{\ket}[1]{\left|#1\right>}
    \newcommand*{\abs}[1]{\left|#1\right|}
    \newcommand*{\ev}[1]{\left\langle#1\right\rangle}
    \newcommand*{\p}[1]{\left(#1\right)}
    \newcommand*{\s}[1]{\left[#1\right]}
    \newcommand*{\z}[1]{\left\{#1\right\}}

    \newtheorem{theorem}{Theorem}[section]

    \let\Re\undefined
    \let\Im\undefined
    \DeclareMathOperator{\Res}{Res}
    \DeclareMathOperator{\Re}{Re}
    \DeclareMathOperator{\Im}{Im}
    \DeclareMathOperator{\Log}{Log}
    \DeclareMathOperator{\Arg}{Arg}
    \DeclareMathOperator{\Tr}{Tr}
    \DeclareMathOperator{\E}{E}
    \DeclareMathOperator{\Var}{Var}
    \DeclareMathOperator*{\argmin}{argmin}
    \DeclareMathOperator*{\argmax}{argmax}
    \DeclareMathOperator{\sgn}{sgn}
    \DeclareMathOperator{\diag}{diag\;}

    \colorlet{Corr}{red}

    % \everymath{\displaystyle} % biggify limits of inline sums and integrals
    \tikzstyle{circ} % usage: \node[circ, placement] (label) {text};
        = [draw, circle, fill=white, node distance=3cm, minimum height=2em]
    \definecolor{commentgreen}{rgb}{0,0.6,0}
    \lstset{
        basicstyle=\ttfamily\footnotesize,
        frame=single,
        numbers=left,
        showstringspaces=false,
        keywordstyle=\color{blue},
        stringstyle=\color{purple},
        commentstyle=\color{commentgreen},
        morecomment=[l][\color{magenta}]{\#}
    }

\begin{document}

\onehalfspacing

\pagestyle{fancy}
\rfoot{Yubo Su}
\rhead{}
\cfoot{\thepage/\pageref{LastPage}}

\title{Evection Resonances in BH Triples}
\author{Yubo Su}
\date{}

\maketitle

\tableofcontents

\section{03/15/21---Basics \& Introduction}

\subsection{Writing Down the Hamiltonian}

We assume a triple system $m_{1,2,3}$ and $a$, $a_{\rm out}$ with mutual
inclination $I$. The 1PN apsidal precession of the inner binary has
energy/Hamiltonian
\begin{equation}
    H_{\rm GR} = -\frac{3G^2m_1m_2m_{12}}{c^2a^2j(e)},
\end{equation}
while the external companion has averaged energy
\begin{equation}
    H_{\rm out} = -\frac{Gm_3\mu_{12} a^2}{a_{\rm out}^3}\s{
        \frac{1}{16}\s{\p{6 + 9e^2}\cos^2 I
            - (2 + 3e^2)} + \frac{15}{32}
                \p{1 + \cos I}^2 e^2 \cos\p{2\varpi - 2\lambda_{\rm out}}}.
\end{equation}
Here, we have averaged over: $\varpi = \ascnode + \omega$ is the longitude of
pericenter of the inner orbit, so $\uv{e} = \cos \varpi \uv{x} + \sin \varpi
\uv{y}$, and $\lambda_{\rm out} = \varpi_{\rm out} + M_{\rm out}$ is the mean
longitude of $m_3$, where $M_{\rm out}$ is the outer mean anomaly. Recall that
$\Omega_{\rm out} = \dot{\lambda}_{\rm out} = \dot{M}_{\rm out}$, and the useful
component form
\begin{equation}
    \uv{r}_{\rm out} = \cos \lambda_{\rm out}\uv{x}
        + \sin \lambda_{\rm out} \cos I \uv{y}
        + \sin \lambda_{\rm out} \sin I \uv{z}.
\end{equation}

Why is this interesting? Well, let's write $\epsilon \equiv
\frac{Gm_3\mu_{12}a^2}{a_{\rm out}^3} \big/ H_{\rm GR, 0}$, where $H_{\rm GR, 0}
= \s{H_{\rm GR}}_{e = 0}$, or
\begin{align}
    \epsilon &= \frac{m_3\mu_{12}a^2 c^2a^2}{3G^2m_1m_2m_{12}a_{\rm out}^3},\\
        &= \frac{m_3a^4 c^2}{3Gm_{12}^2a_{\rm out}^3}.
\end{align}
This is like $\epsilon_{\rm GR}^{-1}$ from our previous LK work. We are
interested in the regime where $\epsilon \ll 1$. The total Hamiltonian of the
system is then
\begin{equation}
    \frac{H}{H_{\rm GR, 0}} = -\frac{1}{j(e)}
        - \epsilon\s{
        \frac{1}{16}\s{\p{6 + 9e^2}\cos^2 I
            - (2 + 3e^2)} + \frac{15}{32}
                \p{1 + \cos I}^2 e^2 \cos\p{2\varpi - 2\lambda_{\rm out}}}.
\end{equation}
We will eventually expand this Hamiltonian in terms of the conjugate variables
$-\varpi$ and $1 - \p{1 - e^2}^{1/2} \approx e^2/2$ and obtain a separatrix'd
Hamiltonian [Xu \& Lai (26)]. But for now, we can be satisfied that some sort of
separatrix might appear at $\epsilon \sim 1$? It's not clear yet.

\subsection{Timescale Comparison}

This section mostly follows Dong's notes, for completeness.

We need $\Omega_{\rm out} \equiv \sqrt{Gm_{123} / a_{\rm out}^3}$ to be of order
$\dot{\varpi} \equiv 3Gnm_{12} / (c^2aj^2)$. Assuming the eccentricity is
already mostly damped (when $\epsilon_{\rm GR} \gg 1$, we expect this), then
this gives
\begin{align}
    \frac{3Gm_{12}}{c^2a} &\simeq \frac{\Omega_{\rm out}}{n}
        = \sqrt{\frac{m_{123}}{m_{12}}\frac{a^3}{a_{\rm out}^3}},\\
    \p{\frac{a}{a_{\rm out}}}^{5/2} &\simeq
        \frac{3Gm_{12}}{c^2a_{\rm out}}\sqrt{\frac{m_{12}}{m_{123}}}.
\end{align}
Indeed, since everything is fixed, as $a$ decays, the evection resonance will be
crossed.

Will there be enough time to excite eccentricity? The eccentricity growth rate
due to the evection resonance must of order $t_{\rm ZLK}^{-1} \sim n \p{m_3 /
m_{12}} \p{a / a_{\rm out}}^3$. On the other hand, orbital decay due to GW is of
order
\begin{equation}
    t_{\rm GW}^{-1} \simeq \frac{64}{5}\frac{G^3m_{12}^2\mu}{c^5a^4}
        = \frac{64}{5}n\frac{G^{5/2}m_{12}^{3/2}\mu}{c^5a^{5/2}}.
\end{equation}
Thus, the resonance has time to grow if (in the third line, we invoke the
resonance condition above)
\begin{align}
    t_{\rm GW}^{-1} &\ll t_{\rm ZLK}^{-1},\\
    \frac{64}{5}\frac{G^{5/2}m_{12}^{3/2}\mu}{c^5a^{5/2}} &\ll
        \frac{m_3}{m_{12}}\p{\frac{a}{a_{\rm out}}}^3,\\
    \frac{64}{5}\frac{G^{5/2}m_{12}^{3/2}\mu}{c^5a_{\rm out}^{5/2}}
        \frac{c^2a_{\rm out}}{3Gm_{12}}\sqrt{\frac{m_{123}}{m_{12}}}
            &\ll\\
    \frac{64}{15} \frac{G^{3/2}m_{123}^{1/2}\mu}{c^3a_{\rm out}^{3/2}}&\ll\\
    \frac{64}{15}\p{\frac{v_{\rm out}}{c}}^3 \frac{m_{12} / 4}{m_{123}}
        &\ll\\
    \p{\frac{v_{\rm out}}{c}}^3\p{\frac{a_{\rm out}}{a}}^3
        \frac{m_{12}^2}{m_{123}m_3} &\ll 1.
\end{align}
Indeed, this must be the case. Another check requires
\begin{align}
    t_{\rm ZLK}^{-1} &\ll \dot{\varpi},\\
    \frac{m_3}{m_{12}}\p{\frac{a}{a_{\rm out}}}^3 &\ll \frac{3Gm_{12}}{c^2a}
        \sim \frac{\Omega_{\rm out}}{n},\\
    &\ll \p{\frac{m_{123}}{m_{12}}}^{1/2}\p{\frac{a}{a_{\rm out}}}^{3/2},\\
    \frac{m_3}{m_{12}}\p{\frac{m_{12}}{m_{123}}}^{1/2}\p{\frac{a}{a_{\rm
        out}}}^{3/2} &\ll 1.
\end{align}
This is also satisfied. Thus, resonance excitation should be possible.

What are the kinds of systems that are interacting? If we want LISA band, we
need $n / \pi \sim 10^{-3}\;\mathrm{Hz}$, and:
\begin{align}
    \Omega_{\rm out} &\simeq \frac{3Gnm_{12}}{c^2a},\\
        &\simeq \frac{3n^3a^2}{c^2},\\
        &\simeq \frac{3n^3}{c^2}\p{\frac{Gm_{12}}{n^2}}^{2/3},\\
        &\simeq 10^{-7}
            \p{\frac{P}{10^3\;\mathrm{s}}}^{-5/3}
            \p{\frac{m_{12}}{2 M_{\odot}}}^{2/3}\;\mathrm{s^{-1}},\\
    a_{\rm out} &= \p{\frac{Gm_{123}}{\Omega_{\rm out}^2}}^{1/3},\\
        &= 2.4\p{\frac{m_{123}}{3 M_{\odot}}}^{1/3}
            \p{\frac{P}{10^3\;\mathrm{s}}}^{10/9}
            \p{\frac{m_{12}}{2M_{\odot}}}^{-4/9}\;\mathrm{AU}.
\end{align}
Indeed then, this is not going to be super useful unless $m_3$ is a SMBH, in
which case $a_{\rm out} \sim 100$--$1000\;\mathrm{AU}$. Note that $a \sim
\scinot{3}{8}\;\mathrm{m}$.

The other scenario then is that we cross this resonance, get a large
eccentricity, and it doesn't completely damp by the time it crosses the LISA
band? Well, we saw above that $\p{a / a_{\rm out}}^{5/2} \propto a_{\rm
out}^{-1}$, so if we fix the masses then increasing $a_{\rm out}$ by a factor of
$4$ increases $a$ by a factor of $32$, i.e.\ $a \sim 0.05\;\mathrm{AU}$ while
$a_{\rm out} = 10\;\mathrm{AU}$, somewhat believable. Since the rates of change
of $\ln a$ and $\ln e$ differ only by a factor of $j^2(e)$, if $e$ is only
modest, then it will have to also decay by $\sim 30$ by the time $a$ enters the
LISA band. However, if we can excite a substantial $e$ like $j^2(e) = 0.1$
(corresponding to $e = 0.95$), then $e$ will only decay by $\sim 3$ upon
entering the LISA band, which leaves us with an $e = 0.3$. Wenrui's paper
suggests evection isn't quite this strong, but maybe some sort of scenario is
possible.

The final solution is to use evection to pump an existing large eccentricity up
a little bit. But it's becoming clear that we aren't going to cleanly get
excitation in the LISA band, and that we will really need to consider dynamics
during \emph{and after} the resonance.

\subsection{Hamiltonian Level Curves}

Let's try to nondimensionalize the Hamiltonian now, like Wenrui's paper. Call
$\gamma = -\varpi$ and $\Gamma = 1 - \sqrt{1 - e^2}$, so that $j(e) = 1 -
\Gamma$ and $e^2 = 1 - \p{1 - \Gamma}^2 = 2\Gamma + \Gamma^2$, then
\begin{align}
    \frac{H}{H_{\rm GR, 0}} ={}&
        -\frac{1}{1 - \Gamma} - \frac{\epsilon}{16}
            \s{\p{6 + 9\p{2\Gamma + \Gamma^2}}\cos^2 I
                - \p{2 + 3\p{2\Gamma + \Gamma^2}}}\nonumber\\
            &- \frac{15\epsilon}{32}\p{1 + \cos I}^2 \p{2\Gamma + \Gamma^2}
                \cos\p{2\gamma + 2\lambda_{\rm out}},\\
    H' ={}& -\frac{1}{1 - \Gamma} - \frac{\epsilon}{16}\s{
        \p{9\cos^2 I - 3}\p{2\Gamma + \Gamma^2}}
        - \frac{15\epsilon}{32}\p{1 + \cos I}^2
            \p{2\Gamma + \Gamma^2}\cos\p{2\gamma + 2\lambda_{\rm out}},\\
        \approx{}&
            \Gamma\p{-1 - 2\epsilon A}
            + \Gamma^2\p{-1 + \epsilon A}
            - \Gamma \epsilon B\cos \theta
            + \mathcal{O}\p{\Gamma^3, \Gamma^2 \cos \theta},\\
    A ={}& \frac{9\cos^2 I - 3}{16},\\
    B ={}& \frac{15}{16}\p{1 + \cos I}^2,\\
    \theta ={}& 2\varpi - 2\lambda_{\rm out}.
\end{align}
This is not quite as clean as Wenrui's form, but it has the advantage (for me)
that the $\epsilon$ dependence is still explicit, while $A, B$ are almost always
positive (except for when $\cos^2 I < 1/3$).

\section{03/18/21--03/21/21}

\subsection{Deriving the Hamiltonian Carefully: Circular Perturber}

After doing some simulations, it is pretty clear that we will need a precise
derivation of the Hamiltonian. We start with the full Hamiltonian
\begin{equation}
    H = -\frac{3G^2m_1m_2m_{12}}{c^2a^2j(e)}
        - \frac{Gm_3\mu_{12}a^2}{r_{\rm out}^3}\s{
        \frac{1}{16}\s{\p{6 + 9e^2}\cos^2 I
            - (2 + 3e^2)} + \frac{15}{32}
                \p{1 + \cos I}^2 e^2 \cos\p{2\varpi - 2\lambda_{\rm out}}}.
\end{equation}
Note that when the outer orbit is circular, $r_{\rm out} = a_{\rm out}$ and
$\lambda_{\rm out} = f_{\rm out}$. I can't seem to figure out how to
non-dimensionalize the Hamiltonian and the time right now, so let's just factor
out $H_{\rm GR, 0} \equiv 3G^2 m_1m_2m_{12} / c^2a^2$, and drop constant terms
\begin{equation}
    H = -H_{\rm GR, 0}\z{\frac{1}{j(e)}
        + \epsilon\s{
        \frac{(6 + 9e^2)\cos^2 I - 3e^2}{16} + \frac{15}{32}
                \p{1 + \cos I}^2 e^2 \cos\p{2\varpi - 2\lambda_{\rm out}}}}.
\end{equation}
Here, again, $\epsilon$ was given above
\begin{equation}
    \epsilon = \frac{m_3a^4 c^2}{3Gm_{12}^2a_{\rm out}^3}.
\end{equation}
Okay, I give up, we non-dimensionalize the Hamiltonian by dividing by $H_{\rm
GR, 0}$ and scale time via
\begin{equation}
    \tau \equiv \dot{\varpi}_{\rm GR}t =
        \frac{3Gnm_{12}}{c^2aj^2}t.
\end{equation}

We now seek the appropriate canonical transformation for our rescaled $H$. We
can first directly recast $H$ in terms of the modified Delaunay variables; since
the generating function is time-independent, we just need to re-express $H$ in
terms of the new variables ($\theta_1 = \lambda$ does not appear in the original
$H$, so $L_D = J_1 = \sqrt{Gm_{12}a}$ is conserved, which we've just used in
renormalizing the Hamiltonian and by convention set $L_D = J_1 = 1$)
\begin{align}
    J_2 &= 1 - \sqrt{1 - e^2} & \theta_2 &= -\varpi,\\
    J_3 &= \sqrt{1 - e^2}\p{1 - \cos I} & \theta_3 &= -\ascnode,
\end{align}
and so $\sqrt{1 - e^2} = 1 - J_2$, $e^2 = 1 - \p{1 - J_2}^2 = 2J_2 - J_2^2$, and
$\cos I = 1 - \s{J_3 / \p{1 - J_2}}$ (note that $J_3$ is also a constant, but
since $e$ is not constant, neither is $\cos I$, strictly speaking)
\begin{align}
    H\p{J_2, \theta_2, J_3, \theta_3}
        &= -\frac{1}{1 - J_2} - \epsilon\s{
            A + B \cos\p{-2\theta_2 - 2\lambda_{\rm out}}
        },\\
    A &= \frac{3\cos^2 I - 1}{16}3e^2 + \frac{3}{8}\cos^2 I =
        \frac{3}{16}\p{2J_2 - J_2^2}\p{
            2 - 6\frac{J_3}{1 - J_2} + 3\p{\frac{J_3}{1 - J_2}}^2}
            + \frac{3}{8}\cos^2 I,\\
    B &= \frac{15}{32}\p{1 + \cos I}^2e^2 =
        \frac{15}{32}\s{4 - 4\frac{J_3}{1 - J_2} + \p{\frac{J_3}{1 - J_2}}^2}
            \p{2J_2 - J_2^2}.
\end{align}
We further want to transform the Hamiltonian for some canonical variable $\phi =
-2\theta_2 - 2\lambda_{\rm out}$. The canonical variable conjugate to $\phi$ is
just $\Gamma = -J_2 / 2$, as we can verify via the Poisson bracket:
\begin{equation}
    \z{\phi, \Gamma} = \pd{\phi}{\theta_2}\pd{\Gamma}{J_2}
            - \pd{\phi}{J_2}\pd{\Gamma}{\theta_2}
        = \p{-2}\p{-\frac{1}{2}} = 1.
\end{equation}
The generating function for this canonical transformation is just $S\p{p, Q} =
-J_2\phi / 2$, so the Hamiltonian for the resonant angle $\phi$ becomes
\begin{align}
    H\p{\Gamma, \phi; J_3} &= -\frac{1}{1 + 2\Gamma} - \epsilon\p{
        A + B \cos \phi} + \pd{S}{t},\\
        &= -\frac{1}{1 + 2\Gamma} - \epsilon\p{
            A + B \cos \phi} - \frac{J_2}{2} \p{-2\rd{\lambda_{\rm out}}{t}},\\
        &= -\frac{1}{1 + 2\Gamma} - \epsilon\p{
            A + B \cos \phi} - 2\Gamma \frac{\Omega_{\rm out}}{\Omega_{\rm GR, 0}},\\
    A &= \frac{3}{16}\p{-4\Gamma - 4\Gamma^2}\p{
        2 - 6\frac{J_3}{1 + 2\Gamma} + 3\p{\frac{J_3}{1 + 2\Gamma}}^2}
        + \frac{3}{8}\p{1 - \frac{J_3}{1 + 2\Gamma}}^2,\\
    B &= \frac{15}{32}\s{
        4 - 4\frac{J_3}{1 + 2\Gamma} + \p{\frac{J_3}{1 + 2\Gamma}}^2}
        \p{-4\Gamma - 4\Gamma^2}.
\end{align}
Note that $\lambda_{\rm out} = \omega_{\rm out} + \ascnode_{\rm out} +
\mathcal{M}_{\rm out}$, so the time derivative is just the mean motion (at the
current order of approximation) in nondimensional units.

Up until here, everything is still exact; we can now compute the Hamiltonian to
leading order in $\Gamma$, $\Gamma^2$ and $\cos \phi$ (we drop constant terms in
$\epsilon$ and $J_3$)
\begin{align}
    H\p{\Gamma, \phi} &=
        2\Gamma - 4\Gamma^2 - \epsilon\p{A + B\cos\phi} - 2\Gamma\frac{\Omega_{\rm
            out}}{\Omega_{\rm GR, 0}} + \mathcal{O}\p{\Gamma^3},\\
        &= 2\Gamma\p{1 - \frac{\Omega_{\rm out}}{\Omega_{\rm GR, 0}}}
            - 4\Gamma^2 - \epsilon\p{A + B \cos \phi} +
            \mathcal{O}\p{\Gamma^3},\\
    A &= -\frac{3\p{\Gamma + \Gamma^2}}{4}\p{2 - 6J_3}
        + \frac{3}{8}\p{4J_3\Gamma}
        + \mathcal{O}\p{\Gamma^3, \Gamma^2J_3},\\
        &= \frac{3}{2}\p{4J_3\Gamma - \Gamma - \Gamma^2}
            + \mathcal{O}\p{\Gamma^3, \Gamma^2J_3},\\
    B &= -\frac{15}{2}\Gamma + \mathcal{O}\p{\Gamma^2, J_3\Gamma},\\
    H\p{\Gamma, \phi}
        &= 2\Gamma\p{1 - \frac{\Omega_{\rm out}}{\Omega_{\rm GR, 0}}}
            - 4\Gamma^2 - \frac{\epsilon}{2}\s{
                3\p{4J_3\Gamma - \Gamma - \Gamma^2}
                - 15 \Gamma \cos \phi} +
            \mathcal{O}\p{\Gamma^3, \Gamma^2 J_3, \Gamma^2 \cos \phi},\\
        &= 2\Gamma\p{1 - \frac{\Omega_{\rm out}}{\Omega_{\rm GR, 0}}
            -\frac{\epsilon\p{12J_3 - 3}}{4}}
            - \Gamma^2\p{4 - \frac{3\epsilon}{2}} + \frac{\epsilon}{2}
                 15 \Gamma \cos \phi +
            \mathcal{O}\p{\Gamma^3, \Gamma^2 J_3, \Gamma^2 \cos \phi}.
\end{align}
Note that, to leading order,
\begin{equation}
    J_3 \approx \p{1 - J_2}\p{1 - \cos I} = \p{1 + 2\Gamma}\p{1 - \cos I}
        \approx 1 + 2\Gamma - \cos I.
\end{equation}
This substitution cannot be made into the Hamiltonian directly, however, since
$J_3$ is an independent momentum from $\Gamma$ and is conserved. Thus, we obtain
\begin{align}
    H\p{\Gamma,\phi} &\approx \Gamma P - \Gamma^2 Q + R \Gamma \cos \phi,&&\\
        P &\approx 2\s{1 - \frac{\Omega_{\rm out}}{\Omega_{\rm GR, 0}} -
            \frac{\epsilon\p{12 J_3 - 3}}{4}},&
        Q &\approx 4 - \frac{3\epsilon}{2} \approx 4,\nonumber\\
        R &\approx \frac{15\epsilon}{2},\\
        \epsilon &= \frac{m_3a^4 c^2}{3Gm_{12}^2a_{\rm out}^3},&
        \frac{\Omega_{\rm out}}{\Omega_{\rm GR, 0}}
            &= \frac{\p{a / a_{\rm out}}^{3/2}\p{m_{123} / m_{12}}^{1/2}}{
                3Gm_{12} / (c^2a)}
\end{align}
Here, $\Gamma \in [-0.5, 0]$, and $\Gamma \simeq -e^2/4$ for $e \ll 1$; these
are checked with \lstinline{sympy}.

This seems to largely agree with Wenrui's Hamiltonian, except he omitted the
$\epsilon$ contribution to the $\Gamma^2$ coefficient (XL Eq.~17). There are
then three questions to answer about this Hamiltonian:
\begin{itemize}
    \item Are there any bifurcations ([dis]appearances of equilibria)?
    \item What does the phase portrait look like, qualitatively?
    \item What is the resonance width?
\end{itemize}

To answer these, we use Hamilton's equations to compute the EOM and equilibria:
\begin{align}
    \dot{\phi} &= \pd{H}{\Gamma} = P - 2\Gamma Q + R \cos \phi,\\
    \dot{\Gamma} &= -\pd{H}{\phi} = -R\Gamma \sin \phi.
\end{align}
From the second equation, there are only zeros when $\phi = 0, \pi$, and from
the first equation, these occur at
\begin{equation}
    \Gamma_{\phi = 0, \pi} = \frac{P \pm R}{2Q},
\end{equation}
where the positive sign corresponds to $\phi = 0$. Recalling that $\Gamma \in
[-0.5, 0]$, we see that the solutions only exist when $P \pm 15\epsilon/2 \in
[-Q, 0]$. Ps the inspiral progresses, $P$ is increasing, so we see that the
$\Gamma_{\phi = 0}$ equilibrium will disappear when $P = -15\epsilon/2$ and the
$\Gamma_{\phi = \pi}$ equilibrium will disappear when $P = 15\epsilon/2$.

\textbf{TODO:} Resonance width.

\subsection{Eccentric Perturber}

The goal here is to find the correct way to average the single-averaged
expression
\begin{equation}
    \tilde{H}_{\rm out} \equiv \frac{H_{\rm out}}{Gm_3\mu_{12}a^2/a_{\rm out}^3}
        = \frac{1}{4}\p{\frac{a_{\rm out}}{r_{\rm out}}}^3 \s{
            -1 + 6e^2 + 3\p{1 - e^2}\p{\uv{n} \cdot \uv{r}_{\rm out}}^2
            - 15 e^2\p{\uv{e} \cdot \uv{r}_{\rm out}}^2}.
\end{equation}
The coordinate system we choose matches XL16, where $\uv{L}_{\rm in} \propto
\uv{z}$ while $\ascnode_{\rm out} = 0$. This gives component form (see MD 2.20 and
2.122)
\begin{align}
    r_{\rm out} &= \frac{a_{\rm out}\p{1 - e_{\rm out}^2}}{1 + e_{\rm out} \cos
        f_{\rm out}} &
    \uv{r}_{\rm out} &= \begin{pmatrix}
        \cos v_{\rm out}\\
        \sin v_{\rm out} \cos I\\
        \sin v_{\rm out} \sin I
    \end{pmatrix}.
\end{align}
Here, $v_{\rm out} = \ascnode_{\rm out} + \omega_{\rm out} + f_{\rm out} =
\omega_{\rm out} + f_{\rm out}$ is the \emph{true longitude}. Averaging is done
via the identities
\begin{align}
    \ev{\frac{\cos^2 f_{\rm out}}{r_{\rm out}^3}} &=
        \ev{\frac{\sin^2 f_{\rm out}}{r_{\rm out}^3}}
        = \frac{1}{2a_{\rm out}^3\p{1 - e_{\rm out}^2}^{3/2}},\\
    \ev{\frac{1}{r_{\rm out}^3}} &= \frac{1}{a_{\rm out}^3\p{1 - e_{\rm
        out}^2}^{3/2}},\\
    \ev{\frac{\cos f_{\rm out} \sin f_{\rm out}}{r_{\rm out}^3}} &= 0.
\end{align}

\subsubsection{Circular Averaging}

Just to review the averaging procedure, let's consider the case where $e_{\rm
out} = 0$, then $v_{\rm out} = \lambda_{\rm out}$ the \emph{mean longitude} and
we can write
\begin{align}
    \tilde{H}_{\rm out} &= \frac{1}{4}\s{
        -1 + 6e^2 + 3\p{1 - e^2}\sin^2 \lambda_{\rm out} \sin^2 I
        - 15e^2\p{\cos \varpi \cos \lambda_{\rm out}
            + \sin \varpi \sin \lambda_{\rm out}\cos I}^2},\\
    \uv{e} \cdot \uv{r}_{\rm out} &=
        \cos\p{\varpi - \lambda_{\rm out}}\p{\frac{1 + \cos I}{2}}
            + \cos\p{\varpi - \lambda_{\rm out}}\p{\frac{1 - \cos I}{2}},\\
    \ev{\p{\dots}^2} &=
        \cos^2\p{\varpi - \lambda_{\rm out}}\p{\frac{1 + \cos I}{2}}^2
            + \frac{\p{1 - \cos I}^2}{8},\\
    \ev{\tilde{H}_{\rm out}} &= \frac{1}{4}\s{
        -1 + 6e^2 + \frac{3}{2}\p{1 - e^2}\p{1 - \cos^2 I}
            - \frac{15}{2} e^2 \p{1 + \cos\p{2\varpi - 2\lambda_{\rm out}}}
                \p{\frac{1 + \cos I}{2}}^2
                - \frac{15e^2\p{1 - \cos I}^2}{8}},\\
        &= \frac{1}{16}\s{-4 + 24 e^2 + 6\p{1 - e^2 - \cos^2 I
            + e^2\cos^2 I} - 15e^2\p{1 + \cos I^2} - \frac{15}{2}
                e^2\cos \phi\p{1 + \cos I}^2},\\
        &= \frac{1}{16}\s{2 + 3e^2 - 6\cos^2 I - 9e^2\cos^2 I
            - \frac{15}{2}\p{1 + \cos I}^2 e^2 \cos \phi}.
\end{align}
Success!

\subsection{Eccentric Averaging}

We cannot just substitute $a_{\rm out} \Rightarrow a_{\rm out, eff} \equiv
a_{\rm out}\sqrt{1 - e_{\rm out}^2}$ because $\phi$ is no longer a resonant
angle. All of the other terms are the same, however, so we effectively just need
to compute the expression
\begin{equation}
    a_{\rm out}\ev{\frac{\p{\uv{e} \cdot \uv{r}_{\rm out}}^2}{r_{\rm out}^3}}
\end{equation}
Consider a Fourier decomposition of $\uv{r}_{\rm out} / r_{\rm out}^{3/2}$,
where $\Omega_{\rm out}$ is the outer mean motion (just in the $x$--$y$ plane),
then (we assume $\omega_{\rm out}\p{t = 0} = 0$ and give an arbitrary phase
offset $\varpi_0$ to the inner eccentricity vector)
\begin{align}
    \uv{e}(t) &= \cos \p{\Omega_{\rm GR}t + \varpi_0} \uv{x} +
        \sin\p{\Omega_{\rm GR}t + \varpi_0}\uv{y},\\
    \frac{\uv{r}_{\rm out}(t)_{\perp}}{r_{\rm out}^{3/2}} &=
        \frac{1}{r_{\rm out}^{3/2}}\s{\cos v_{\rm out}
            \uv{x} + \sin v_{\rm out}\cos I \uv{y}},\\
        &= \sum\limits_{N = 1}^\infty
            \frac{c_N}{a_{\rm out}^{3/2}}\z{\cos\p{N \Omega_{\rm out}t} \uv{x}
            + \sin\p{N \Omega_{\rm out}t} \cos I\uv{y}},\\
    \uv{e} \cdot \frac{\uv{r}_{\rm out}}{r_{\rm out}^{3/2}}
        &= \sum\limits_{N = 1}^\infty
            \frac{c_N}{a_{\rm out}^{3/2}}
                \z{\cos\p{N \Omega_{\rm out}t} \cos\p{\Omega_{\rm GR}t +
                    \varpi_0}
            + \sin\p{N \Omega_{\rm out}t} \cos I \sin\p{\Omega_{\rm GR}t +
                \varpi_0}},\\
        &= \sum\limits_{N = 1}^\infty
            \frac{c_N}{a_{\rm out}^{3/2}}
                \z{
                    \cos\p{\p{N\Omega_{\rm out} - \Omega_{\rm GR}}t - \varpi_0}
                        \p{\frac{1 + \cos I}{2}}
                    + \cos\p{\p{N\Omega_{\rm out} + \Omega_{\rm GR}}t + \varpi_0}
                        \p{\frac{1 - \cos I}{2}}
                    },\\
    \ev{\frac{\p{\uv{e} \cdot \uv{r}_{\rm out}}^2}{r_{\rm out}^3}}
        &= \sum\limits_{M = 0}^{N - 1}
            f_{NM}
            \frac{c_{N_{\rm GR} + M}c_{N_{\rm GR} - M}}{2a_{\rm out}^3}
            \p{\frac{1 + \cos I}{2}}^2 \cos\p{\p{2N_{\rm GR}\Omega_{\rm out}
                - 2\Omega_{\rm GR}}t - 2\varpi_0}.
\end{align}
Here, $N_{\rm GR} \equiv \lfloor \Omega_{\rm GR} / \Omega_{\rm out}$, and
$f_{NM} = 1$ if $M = N / 2$ else $2$ (double counting factor). Since the $c_N$
should fall off for $N \gtrsim N_{\rm p}$ where
\begin{equation}
    N_{\rm p} \equiv \frac{\sqrt{1 + e}}{\p{1 - e_{\rm out}}^{3/2}},
\end{equation}
we see that there will generally be resonances for all $N\Omega_{\rm out} \sim
\Omega_{\rm GR}$ as long as $N \lesssim N_{\rm p}$. Furthermore, we guess that
the $c_{N}$ are expected to scale like $N^2$ for $N \lesssim N_{\rm p}$, so the
sum is dominated by the $M = 0$ contribution.

Does this agree with simulations yet? Well, we do observe resonances for both $N
\approx 1$ and $N \approx N_{\rm p}$, todo more in depth exploration.


\end{document}

