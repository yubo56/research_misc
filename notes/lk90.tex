    \documentclass[11pt,
        usenames, % allows access to some tikz colors
        dvipsnames % more colors: https://en.wikibooks.org/wiki/LaTeX/Colors
    ]{article}
    \usepackage{
        amsmath,
        amssymb,
        fouriernc, % fourier font w/ new century book
        fancyhdr, % page styling
        lastpage, % footer fanciness
        hyperref, % various links
        setspace, % line spacing
        amsthm, % newtheorem and proof environment
        mathtools, % \Aboxed for boxing inside aligns, among others
        float, % Allow [H] figure env alignment
        enumerate, % Allow custom enumerate numbering
        graphicx, % allow includegraphics with more filetypes
        wasysym, % \smiley!
        upgreek, % \upmu for \mum macro
        listings, % writing TrueType fonts and including code prettily
        tikz, % drawing things
        booktabs, % \bottomrule instead of hline apparently
        cancel % can cancel things out!
    }
    \usepackage[margin=1in]{geometry} % page geometry
    \usepackage[
        labelfont=bf, % caption names are labeled in bold
        font=scriptsize % smaller font for captions
    ]{caption}
    \usepackage[font=scriptsize]{subcaption} % subfigures

    \newcommand*{\scinot}[2]{#1\times10^{#2}}
    \newcommand*{\dotp}[2]{\left<#1\,\middle|\,#2\right>}
    \newcommand*{\rd}[2]{\frac{\mathrm{d}#1}{\mathrm{d}#2}}
    \newcommand*{\pd}[2]{\frac{\partial#1}{\partial#2}}
    \newcommand*{\rtd}[2]{\frac{\mathrm{d}^2#1}{\mathrm{d}#2^2}}
    \newcommand*{\ptd}[2]{\frac{\partial^2 #1}{\partial#2^2}}
    \newcommand*{\md}[2]{\frac{\mathrm{D}#1}{\mathrm{D}#2}}
    \newcommand*{\pvec}[1]{\vec{#1}^{\,\prime}}
    \newcommand*{\svec}[1]{\vec{#1}\;\!}
    \newcommand*{\bm}[1]{\boldsymbol{\mathbf{#1}}}
    \newcommand*{\ang}[0]{\;\text{\AA}}
    \newcommand*{\mum}[0]{\;\upmu \mathrm{m}}
    \newcommand*{\at}[1]{\left.#1\right|}

    \newtheorem{theorem}{Theorem}[section]

    \let\Re\undefined
    \let\Im\undefined
    \DeclareMathOperator{\Res}{Res}
    \DeclareMathOperator{\Re}{Re}
    \DeclareMathOperator{\Im}{Im}
    \DeclareMathOperator{\Log}{Log}
    \DeclareMathOperator{\Arg}{Arg}
    \DeclareMathOperator{\Tr}{Tr}
    \DeclareMathOperator{\E}{E}
    \DeclareMathOperator{\Var}{Var}
    \DeclareMathOperator*{\argmin}{argmin}
    \DeclareMathOperator*{\argmax}{argmax}
    \DeclareMathOperator{\sgn}{sgn}
    \DeclareMathOperator{\diag}{diag\;}

    \DeclarePairedDelimiter\bra{\langle}{\rvert}
    \DeclarePairedDelimiter\ket{\lvert}{\rangle}
    \DeclarePairedDelimiter\abs{\lvert}{\rvert}
    \DeclarePairedDelimiter\ev{\langle}{\rangle}
    \DeclarePairedDelimiter\p{\lparen}{\rparen}
    \DeclarePairedDelimiter\s{\lbrack}{\rbrack}
    \DeclarePairedDelimiter\z{\lbrace}{\rbrace}

    % \everymath{\displaystyle} % biggify limits of inline sums and integrals
    \tikzstyle{circ} % usage: \node[circ, placement] (label) {text};
        = [draw, circle, fill=white, node distance=3cm, minimum height=2em]
    \definecolor{commentgreen}{rgb}{0,0.6,0}
    \lstset{
        basicstyle=\ttfamily\footnotesize,
        frame=single,
        numbers=left,
        showstringspaces=false,
        keywordstyle=\color{blue},
        stringstyle=\color{purple},
        commentstyle=\color{commentgreen},
        morecomment=[l][\color{magenta}]{\#}
    }

\begin{document}

\def\Snospace~{\S{}} % hack to remove the space left after autorefs
\renewcommand*{\sectionautorefname}{\Snospace}
\renewcommand*{\appendixautorefname}{\Snospace}
\renewcommand*{\figureautorefname}{Fig.}
\renewcommand*{\equationautorefname}{Eq.}
\renewcommand*{\tableautorefname}{Tab.}

\onehalfspacing

\pagestyle{fancy}
\rfoot{Yubo Su}
\rhead{}
\cfoot{\thepage/\pageref{LastPage}}

\title{Lidov-Kozai $90^\circ$ Attractor}
\author{Yubo Su}
\date{Date}

\maketitle

\section{Equations}

\subsection{Bin's Papers}

Our major references will be Bin's paper with Diego + Dong in 2015 (LML15) and
Bin's later paper with Dong on spin-orbit misalignment (LL18). The target of
study is \S4.3 of LL18, where a $90^\circ$ attractor in spin-orbit misalignment
seems to appear when the octupole effect is negligible.

When the octupole effect is negligible, we define vectors
\begin{align}
    \bm{j} &= \sqrt{1 - e^2}\hat{n},\\
    \bm{e} &= e\hat{u}.
\end{align}
Here, $\bm{j}$ is the dimensionless angular momentum vector and $\bm{e}$ is the
eccentricity vector; see LML15 for precise definitions. Note that $\bm{j} \cdot
\bm{e} = 0$, $j^2 + e^2 = 1$. Then, the EOM for the inner and outer vectors
satisfy to quadrupolar order
\begin{align}
    \rd{\bm{j}}{t} &= \frac{3}{4t_{LK}} \s*{
        \p*{\bm{j} \cdot \hat{n}_2} \p*{\bm{j} \times \hat{n}_2}
        - 5\p*{\bm{e} \cdot \hat{n}_2}\p*{\bm{e} \times \hat{n}_2}
        },\\
    \rd{\bm{e}}{t} &= \frac{3}{4t_{LK}} \s*{
        \p*{\bm{j} \cdot \hat{n}_2} \p*{\bm{e} \times \hat{n}_2}
        + 2\bm{j} \times \bm{e}
        - 5\p*{\bm{e} \cdot \hat{n}_2}\p*{\bm{j} \times \hat{n}_2}
        }.
\end{align}
Let's assume for the time being that $L_1 \ll L_2$, so the system is
sufficiently hierarchical that $\bm{j}_2$, $\bm{e}_2$ are constants. Note for
reference that
\begin{equation}
    t_{LK} \equiv \frac{L_1}{\mu_1 \Phi_0}
        = \frac{1}{n_1}\p*{\frac{m_0 + m_1}{m_2}}
            \p*{\frac{a_2}{a}}^3
            \p*{1 - e_2^2}^{3/2}.
\end{equation}
Here, $n_1 \equiv \sqrt{G\p*{m_0 + m_1}/a^3}$. Finally, the GR effects
(Peters 1964) cause decays of $\bm{L}$ and $\bm{e}$ as
\begin{align}
    \at{\rd{\bm{L}}{t}}_{GW} &= -\frac{32}{5}\frac{G^{7/2}}{c^5}
        \frac{\mu^2 m_{12}^{5/2}}{a^{7/2}}
        \frac{1 + 7e^2/8}{\p*{1 - e^2}^2}\hat{L},\\
    \at{\rd{\bm{e}}{t}}_{GW} &= -\frac{304}{15} \frac{G^3}{c^5}
        \frac{\mu m_{12}^2}{a^4\p*{1 - e^2}^{5/2}}\p*{1 + \frac{121}{304}
            e^2}\bm{e},\\
    \at{\frac{\dot{a}}{a}}_{GW} &= -\frac{64}{5} \frac{G^3 \mu m_{12}^2}{c^5a^4}
        \frac{1}{\p*{1 - e^2}^{7/2}}\p*{1 + \frac{73}{24}e^2
            + \frac{37}{96}e^4}.
\end{align}
Here, $m_{12} \equiv m_1 + m_2$.

Given this system (from LML15), we can then add the spin-orbit coupling term,
which is given in LL18 to be
\begin{align}
    \rd{\hat{S}}{t} &= \Omega_{SL}\hat{L} \times \hat{S},\\
    \Omega_{SL} &\equiv \frac{3Gn\p*{m_2 + \mu/3}}{2c^2a\p*{1 - e^2}}.
\end{align}
Note that $\mu$ is the reduced mass of the inner binary. We can drop the
back-reaction term since $S \ll L$. What is observed is that, as this system
is evolved forward in time and GR coalesces the inner binary, $\theta_{sl}
\equiv \arccos\p*{\hat{S} \cdot \hat{L}}$ goes to $90^\circ$ consistently. The
relevant figure is Fig.~19 of LL18, which shows that for a close-in,
low-eccentricity perturber ($\bar{a}_{\rm out, eff} \propto a_{out}$), the
focusing is significantly stronger. Note that initially, $I \equiv
\arccos\p*{\hat{L} \cdot \hat{L}_2} \approx 90^\circ$ while $\theta_{sl}
\approx 0$.

In LL18, an adiabaticity parameter is defined:
\begin{equation}
    \mathcal{A} \equiv \abs*{\frac{\Omega_{SL}}{\Omega_L}},
\end{equation}
where $\Omega_{L} \simeq \ev*{\rd{\hat{L}}{t}}_{LK}$ to quadrupolar order. As
the inner binary coalesces, $\mathcal{A}$ transitions from $\ll 1$ to $\gg 1$
(as $\Omega_{SL}$ is a GR effect so ramps up very quickly as orbital separation
decreases).

\subsection{Simulations}

First, we run GR-less simulations, so let's take $t_{LK} = 1$ (no semimajor axis
evolution), and we reproduce LK oscillations.

Next, when accounting for GR, we should let $a$ evolve as above. Note that since
$\bm{j}$ and $\vec{e}$ are our dynamical variables, we should use $\bm{j} \equiv
\sqrt{1 - e^2} \hat{L} = \sqrt{1 - e^2}\frac{\bm{L}}{\mu \sqrt{Gm_{12}a\p*{1 -
e^2}}}$ and rewrite
\begin{equation}
    \at{\rd{\bm{j}}{t}}_{GW} = \frac{1}{\mu\sqrt{GMa}}\at{\rd{\bm{L}}{t}}_{GW}
        - \frac{\hat{L}}{2a}\at{\rd{a}{t}}_{GW}.
\end{equation}

For convenience, let's just define $t_{LK} = t_{LK,0} \frac{a_0^3}{a^3}$
and set $t_{LK, 0} = 1$. Furthermore, the timescale of relevance for the GW
terms is $t_{GW}^{-1} \sim \frac{G^3 \mu m_{12}^2}{c^5a^4}$. Let's express this
as some ratio $t_{GW} = \epsilon t_{LK, 0}\frac{a_0^4}{a^4}$. Thus, everything
should be nondimensionalized this way.

We lastly add de-Sitter precession of the spin of one of the inner binary
components, call this $\hat{S}$. Similarly, let's just define a proportionality
constant $t_{SL} = \delta t_{LK, 0}\frac{a_0}{a}$, then
\begin{equation}
    \rd{\hat{S}}{(t / t_{LK, 0})} = \delta \frac{a_0}{a} \hat{L} \times \hat{S}.
\end{equation}

Our final simulation equations are thus ($\tau = t / t_{LK, 0}$)
\begin{align}
    \rd{\bm{j}}{\tau} ={}& \frac{3}{4}\p*{\frac{a_0^3}{a^3}} \s*{
        \p*{\bm{j} \cdot \hat{n}_2} \p*{\bm{j} \times \hat{n}_2}
        - 5\p*{\bm{e} \cdot \hat{n}_2}\p*{\bm{e} \times \hat{n}_2}
        }\nonumber\\
        &- \p*{\epsilon \frac{a_0^4}{a^4}}\p*{
            \frac{32}{5}
                \frac{1 + 7e^2/8}{\p*{1 - e^2}^2}
            - \frac{32}{5} \frac{1}{\p*{1 - e^2}^{7/2}}\p*{1 + \frac{73}{24}e^2
                    + \frac{37}{96}e^4}
            }\frac{\bm{j}}{\sqrt{1 - e^{2}}},\\
    \rd{\bm{e}}{\tau} ={}&
        \frac{3}{4}\p*{\frac{a_0^3}{a^3}} \s*{
            \p*{\bm{j} \cdot \hat{n}_2} \p*{\bm{e} \times \hat{n}_2}
            + 2\bm{j} \times \bm{e}
            - 5\p*{\bm{e} \cdot \hat{n}_2}\p*{\bm{j} \times \hat{n}_2}
            }
        -
            \p*{\epsilon\frac{a_0^4}{a^4}}
            \frac{304}{15}
            \frac{1}{\p*{1 - e^2}^{5/2}}
            \p*{1 + \frac{121}{304} e^2}\bm{e},\\
    \rd{\hat{S}}{\tau} ={}& \delta \frac{a_0}{a}
        \frac{\bm{j}}{\sqrt{1 - e^2}} \times \hat{S},\\
    \rd{a}{\tau} ={}& -a\p*{\epsilon\frac{a_0^4}{a^4}}
        \frac{64}{5} \frac{1}{\p*{1 - e^2}^{7/2}}\p*{1 + \frac{73}{24}e^2
            + \frac{37}{96}e^4}.
\end{align}

\end{document}

