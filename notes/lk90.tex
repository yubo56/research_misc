    \documentclass[11pt,
        usenames, % allows access to some tikz colors
        dvipsnames % more colors: https://en.wikibooks.org/wiki/LaTeX/Colors
    ]{article}
    \usepackage{
        amsmath,
        % multicol,
        amssymb,
        fouriernc, % fourier font w/ new century book
        fancyhdr, % page styling
        lastpage, % footer fanciness
        hyperref, % various links
        setspace, % line spacing
        amsthm, % newtheorem and proof environment
        mathtools, % \Aboxed for boxing inside aligns, among others
        float, % Allow [H] figure env alignment
        enumerate, % Allow custom enumerate numbering
        graphicx, % allow includegraphics with more filetypes
        wasysym, % \smiley!
        upgreek, % \upmu for \mum macro
        listings, % writing TrueType fonts and including code prettily
        tikz, % drawing things
        booktabs, % \bottomrule instead of hline apparently
        cancel % can cancel things out!
    }
    \usepackage[margin=1in]{geometry} % page geometry
    \usepackage[
        labelfont=bf, % caption names are labeled in bold
        font=scriptsize % smaller font for captions
    ]{caption}
    \usepackage[font=scriptsize]{subcaption} % subfigures

    \newcommand*{\scinot}[2]{#1\times10^{#2}}
    \newcommand*{\dotp}[2]{\left<#1\,\middle|\,#2\right>}
    \newcommand*{\rd}[2]{\frac{\mathrm{d}#1}{\mathrm{d}#2}}
    \newcommand*{\pd}[2]{\frac{\partial#1}{\partial#2}}
    \newcommand*{\rtd}[2]{\frac{\mathrm{d}^2#1}{\mathrm{d}#2^2}}
    \newcommand*{\ptd}[2]{\frac{\partial^2 #1}{\partial#2^2}}
    \newcommand*{\md}[2]{\frac{\mathrm{D}#1}{\mathrm{D}#2}}
    \newcommand*{\pvec}[1]{\vec{#1}^{\,\prime}}
    \newcommand*{\svec}[1]{\vec{#1}\;\!}
    \newcommand*{\bm}[1]{\boldsymbol{\mathbf{#1}}}
    \newcommand*{\uv}[1]{\hat{\bm{#1}}}
    \newcommand*{\ang}[0]{\;\text{\AA}}
    \newcommand*{\mum}[0]{\;\upmu \mathrm{m}}
    \newcommand*{\at}[1]{\left.#1\right|}

    \newtheorem{theorem}{Theorem}[section]

    \let\Re\undefined
    \let\Im\undefined
    \DeclareMathOperator{\Res}{Res}
    \DeclareMathOperator{\Re}{Re}
    \DeclareMathOperator{\Im}{Im}
    \DeclareMathOperator{\Log}{Log}
    \DeclareMathOperator{\Arg}{Arg}
    \DeclareMathOperator{\Tr}{Tr}
    \DeclareMathOperator{\E}{E}
    \DeclareMathOperator{\Var}{Var}
    \DeclareMathOperator*{\argmin}{argmin}
    \DeclareMathOperator*{\argmax}{argmax}
    \DeclareMathOperator{\sgn}{sgn}
    \DeclareMathOperator{\diag}{diag\;}

    \newcommand*{\bra}[1]{\left<#1\right|}
    \newcommand*{\ket}[1]{\left|#1\right>}
    \newcommand*{\abs}[1]{\left|#1\right|}
    \newcommand*{\ev}[1]{\langle#1\rangle}
    \newcommand*{\p}[1]{\left(#1\right)}
    \newcommand*{\s}[1]{\left[#1\right]}
    \newcommand*{\z}[1]{\left\{#1\right\}}

    % \everymath{\displaystyle} % biggify limits of inline sums and integrals
    \tikzstyle{circ} % usage: \node[circ, placement] (label) {text};
        = [draw, circle, fill=white, node distance=3cm, minimum height=2em]
    \definecolor{commentgreen}{rgb}{0,0.6,0}
    \lstset{
        basicstyle=\ttfamily\footnotesize,
        frame=single,
        numbers=left,
        showstringspaces=false,
        keywordstyle=\color{blue},
        stringstyle=\color{purple},
        commentstyle=\color{commentgreen},
        morecomment=[l][\color{magenta}]{\#}
    }

\begin{document}

\def\Snospace~{\S{}} % hack to remove the space left after autorefs
\renewcommand*{\sectionautorefname}{\Snospace}
\renewcommand*{\appendixautorefname}{\Snospace}
\renewcommand*{\figureautorefname}{Fig.}
\renewcommand*{\equationautorefname}{Eq.}
\renewcommand*{\tableautorefname}{Tab.}

% \singlespacing

\pagestyle{fancy}
\rfoot{Yubo Su}
\rhead{}
\cfoot{\thepage/\pageref{LastPage}}

\title{Lidov-Kozai $90^\circ$ Attractor}
\author{Yubo Su}
\date{Date}

\maketitle

\section{Equations}

\subsection{Bin's Papers}

Our major references will be Bin's paper with Diego + Dong in 2015 (LML15) and
Bin's later paper with Dong on spin-orbit misalignment (LL18). The target of
study is \S4.3 of LL18, where a $90^\circ$ attractor in spin-orbit misalignment
seems to appear when the octupole effect is negligible.

The easiest formulation is just to express everything in terms of $\bm{L}$ and
$\bm{e}$, following LL18. We drop octupole terms and hold the third perturber
constant. These equations come out to be (Eqs.~4--5 w/ substitutions)
\begin{align}
    \rd{\bm{L}}{t} &= \frac{3}{4t_{LK}} \mu \sqrt{Gm_{12}a} \s{
        \p{\bm{j} \cdot \hat{n}_2} \p{\bm{j} \times \hat{n}_2}
        - 5\p{\bm{e} \cdot \hat{n}_2}\p{\bm{e} \times \hat{n}_2}
        },\\
    \rd{\bm{e}}{t} &= \frac{3}{4t_{LK}} \s{
        \p{\bm{j} \cdot \hat{n}_2} \p{\bm{e} \times \hat{n}_2}
        + 2\bm{j} \times \bm{e}
        - 5\p{\bm{e} \cdot \hat{n}_2}\p{\bm{j} \times \hat{n}_2}
        }.
\end{align}
Note that $\bm{j} \equiv \sqrt{1 - e^2}\hat{\bm{L}} = \frac{\bm{L}}{\mu
\sqrt{Gm_{12}a}}$. $m_{12} = m_1 + m_2$ and $\mu = m_1m_2/m_{12}$. We've defined
\begin{equation}
    t_{LK} \equiv \frac{L_1}{\mu_1 \Phi_0}
        = \frac{1}{n_1}\p{\frac{m_{12}}{m_3}}
            \p{\frac{a_2}{a}}^3
            \p{1 - e_2^2}^{3/2}.
\end{equation}
Here, $n_1 \equiv \sqrt{Gm_{12}/a^3}$. Thus, $1 / t_{LK} \propto a^{3/2}$.

GW radiation (Peters 1964) cause decays of $\bm{L}$ and $\bm{e}$ as
\begin{align}
    \at{\rd{\bm{L}}{t}}_{GW} &= -\frac{32}{5}\frac{G^{7/2}}{c^5}
        \frac{\mu^2 m_{12}^{5/2}}{a^{7/2}}
        \frac{1 + 7e^2/8}{\p{1 - e^2}^2}\hat{L},\\
    \at{\rd{\bm{e}}{t}}_{GW} &= -\frac{304}{15} \frac{G^3}{c^5}
        \frac{\mu m_{12}^2}{a^4\p{1 - e^2}^{5/2}}\p{1 + \frac{121}{304}
            e^2}\bm{e}.
\end{align}
Here, $m_{12} \equiv m_1 + m_2$, and $a$ is implicitly defined by $\bm{L}$ and
$e$. The last GR effect is precession of $\vec{e}$, which acts as
\begin{align}
    \at{\rd{\bm{e}}{t}}_{GR} &= \frac{1}{t_{GR}}\hat{\bm{L}} \times \bm{e},\\
    \frac{1}{t_{GR}} &\equiv \frac{3Gnm_{12}}{c^2a\p{1 - e^2}}.
\end{align}
Note that $t_{GR}^{-1} \propto a^{-5/2}$.

Given this system (from LML15 + LL18), we can then add the spin-orbit coupling
term (from de Sitter precession), which is given in LL18 to be
\begin{align}
    \rd{\hat{\bm{S}}}{t} &= \frac{1}{t_{SL}}\hat{\bm{L}} \times \hat{\bm{S}},\\
    \frac{1}{t_{SL}} &\equiv \frac{3Gn\p{m_2 + \mu/3}}{2c^2a\p{1 - e^2}}.
\end{align}
Note that $\mu$ is the reduced mass of the inner binary. We can drop the
back-reaction term since $S \ll L$. Thus, $t_{SL}^{-1} \propto a^{-5/2}$ as
well.

Finally, an adiabaticity parameter can be defined:
\begin{align}
    \mathcal{A} &\equiv \abs{\frac{\Omega_{SL}}{\Omega_L}}.
\end{align}
Here, $\Omega_L \simeq\frac{3\p{1 + 4e^2}}{8t_{LK}\sqrt{1 - e^2}}\abs{\sin
2I}$ is an approximate rate of change of $L$ during an LK cycle

It's natural to redimensionalize to the initial LK time such that
\begin{equation}
    \frac{1}{t_{LK, 0}} \equiv \p{\frac{a}{a_0}}^{3/2}\frac{1}{t_{LK}},
\end{equation}
since nothing else in $t_{LK}$ is changing. The next natural timescale for
gravitational waves is
\begin{equation}
    \frac{1}{t_{GW}} \equiv \frac{G^3 \mu m_{12}^2}{c^5a^4}
        \equiv \frac{1}{t_{GW, 0}} \p{\frac{a_0}{a}}^4
        \equiv \epsilon_{GW} \frac{1}{t_{LK, 0}} \p{\frac{a_0}{a}}^4.
\end{equation}
We can repeat the procedure for the GR precession term and the spin-orbit
coupling terms:
\begin{align}
    \frac{1}{t_{GR}} &= \epsilon_{GR}\frac{1}{t_{LK, 0}}
        \p{\frac{a_0}{a}}^{5/2},\\
    \frac{1}{t_{SL}} &= \epsilon_{SL}\frac{1}{t_{LK, 0}}
        \p{\frac{a_0}{a}}^{5/2}.
\end{align}
Thus, finally, if we let $\tau = t / t_{LK, 0}$, then we obtain full equations
of motion (note that $a_0 = 1$ below)
\begin{align}
    \rd{\bm{L}}{\tau} ={}& \p{\frac{a}{a_0}}^{3/2} \frac{3}{4}\sqrt{a}\s{
            \p{\bm{j} \cdot \hat{n}_2} \p{\bm{j} \times \hat{n}_2}
            - 5\p{\bm{e} \cdot \hat{n}_2}\p{\bm{e} \times \hat{n}_2}
            }\nonumber\\
        &- \epsilon_{GW}\p{\frac{a_0}{a}}^4
            \frac{32}{5} \frac{1 + 7e^2/8}{\p{1 - e^2}^{5/2}}\bm{L},\\
    \rd{\bm{e}}{\tau} ={}& \p{\frac{a}{a_0}}^{3/2} \frac{3}{4}\s{
            \p{\bm{j} \cdot \hat{n}_2} \p{\bm{e} \times \hat{n}_2}
            + 2\bm{j} \times \bm{e}
            - 5\p{\bm{e} \cdot \hat{n}_2}\p{\bm{j} \times \hat{n}_2}
            }\nonumber\\
        &-\epsilon_{GW}\p{\frac{a_0}{a}}^4
            \frac{304}{15} \frac{1}{\p{1 - e^2}^{5/2}}\p{1 + \frac{121}{304}
                e^2}\bm{e}\nonumber\\
        &+ \epsilon_{GR}\p{\frac{a_0}{a}}^{5/2}\frac{1}{1 - e^2}
            \hat{\bm{L}} \times \bm{e},\\
    \rd{\hat{\bm{S}}}{t} ={}&
        \epsilon_{SL}\p{\frac{a_0}{a}}^{5/2}\frac{1}{1 - e^2}
        \hat{\bm{L}} \times \hat{\bm{S}}.
\end{align}
For reference, we note that $a = \abs{\bm{L}}^2 / \p{\mu^2 Gm_{12}\p{1 -
e^2}}$, while $\bm{j} = \bm{L} / \p{\mu\sqrt{Gm_{12}a}}$. To invert $a(\bm{L})$
and $\bm{J}\p{\bm{L}}$ in this coordinate system where $a_0 = 1$, it is easiest
to choose the angular momentum dimensions such that $\mu \sqrt{Gm_{12}} = 1$,
such that now
\begin{align}
    \abs{\bm{L}\p{t = 0}} \equiv \mu \sqrt{Gm_{12}a_0\p{1 - e_0^2}}
        &= \sqrt{\p{1 - e_0^2}},\\
    a &= \frac{\abs{\bm{L}}^2}{1 - e^2},\\
    \bm{j} &= \frac{\bm{L}}{\sqrt{a}} = \hat{\bm{L}}\sqrt{1 - e^2}.
\end{align}

Finally, the timescales are
\begin{align}
    t_{LK, 0} &= \frac{1}{n}
        \frac{m_{12}}{m_3}\p{\frac{a_2}{a(t = 0)}}^3 \p{1 - e_2^2}^{3/2},\\
    \epsilon_{GW} \equiv \frac{t_{LK, 0}}{t_{GW, 0}}
        &= \frac{1}{n}
            \frac{m_{12}}{m_3}\frac{a_2^3}{\p{a(t = 0)}^7} \p{1 - e_2^2}^{3/2}
            \frac{G^3 \mu m_{12}^2}{c^5},\\
    \epsilon_{GR} \equiv \frac{t_{LK, 0}}{t_{GR, 0}}
        &= \frac{m_{12}}{m_3}\frac{a_2^3}{\p{a(t = 0)}^4} \p{1 - e_2^2}^{3/2}
            \frac{3Gm_{12}}{c^2},\\
    \epsilon_{SL} \equiv \frac{t_{SL, 0}}{t_{GR, 0}}
        &= \frac{m_{12}}{m_3}\frac{a_2^3}{\p{a(t = 0)}^4} \p{1 - e_2^2}^{3/2}
            \frac{3G\p{m_2 + \mu/3}}{2c^2}.
\end{align}

The adiabacitity parameter
\begin{equation}
    \mathcal{A} \equiv \abs{\frac{\Omega_{SL}}{\Omega_L}}
        = \frac{\epsilon_{SL}}{t_{LK, 0}}
            \p{\frac{a_0}{a}}^{5/2}\frac{1}{1 - e^2}
            \s{\frac{3(1 + 4e^2)}{8t_{LK, 0}\sqrt{1 - e^2}}
                \p{\frac{a}{a_0}}^{3/2}\abs{\sin 2I}}^{-1},
\end{equation}
(note that $\Omega_L$ is a somewhat averaged sense, see LL18) can be evaluated
in these units as
\begin{equation}
    \mathcal{A} = \epsilon_{SL}\p{\frac{a_0}{a}}^4\frac{1}{\sqrt{1 - e^2}}
        \frac{8}{3\p{1 + 4e^2}\abs{\sin 2I}}.
\end{equation}

Note also that the Hamiltonian is just
\begin{equation}
    H = \Omega_{SL} \hat{\bm{L}} \cdot \hat{\bm{S}},
        = \epsilon_{SL}\p{\frac{a_0}{a}}^{5/2} \frac{1}{1 - e^2}
            \hat{\bm{L}} \cdot \hat{\bm{S}}.
\end{equation}

\subsection{Maximum Eccentricity and Merger Time}

Note that, since we are only evolving $\bm{L}$ and $\bm{e}$, and not $\bm{L}_2$
and $\bm{e}_2$, we are in the test mass approximation, under which we set $\eta
= 0$ in Bin's equations. As such, the maximum eccentricity satisfies (Eq 42 of
LL18 with $\eta \to 0$)
\begin{equation}
    \frac{3}{8}\frac{j_{\min}^2 - 1}{j_{\min}^2}\s{
        5\cos I_0^2 - 3j_{\min}^2} + \epsilon_{GR}\p{1 - \frac{1}{j_{\min}}} =
        0.
\end{equation}
Note that $\epsilon_{GR}$ is exactly as we defined above, incidentally, and that
when GR is negligible, this reduces to the classic $j_{\min} \equiv \sqrt{1 -
e_{\max}^2} = \sqrt{\frac{5}{3}\cos^2 I_0}$. Since $\epsilon_{GR}$ is generally
very small for most of the evolution, this generally reduces to the well known
\begin{equation}
    e_{\max} = \sqrt{1 - \frac{5}{3}\cos^2I_0}.
\end{equation}
This only fails to saturate for extremely high eccentricies, so $I_0 \to
90^\circ$.

\subsection{Attractor Behavior}

Proposal: The reason the $90^\circ$ attractor appears is that the initial
$\theta_{sb}$ is roughly stationary for $\mathcal{A} \ll 1$ (only small kicks
during each LK cycle, as long as the maximum eccentricity isn't too large), then
as we enter the transadiabatic regime, the L-K cycles die down and we simply
have conservation of adiabatic invariant.

The latter half of this follows the LL18 claim, where the requirement that
$\epsilon_{GR} \lesssim 9/4$ (GR precession of pericenter is slow enough that
L-K survives) equates to $\mathcal{A} \lesssim 3$. The former half of this is
somewhat tricky, but we can understand what is happening if we consider what is
happening in the frame corotating with $\Omega_{SL, e = 0}$ about $\hat{z}$:
every time that a LK cycle appears, $\Omega_{SL}$ becomes much larger, and the
axis of precession changes from $\hat{z}$ to the location of $\hat{L}$ very
briefly. We can imagine this as a kick in this corotating frame (which is the
right frame to consider for $\mathcal{A} \ll 1$). In the limit that $I$ does not
change very much between L-K cycles, and the azimuthal angle of $\hat{L}$ is
roughly symmetric, the impulses roughly cancel out in the $\theta_{sb}$
direction. In other words, after two LK cycles, $\theta_{sb}$ does not change
much in the corotating frame. This is indeed the picture that we obtain when we
observe the plot.

As such, the hypothesis is that if $\mathcal{A} \gtrsim 1$ is satisfied while
the kicks are still \emph{small}, then deviations about $90^\circ$ cannot be
very large, and adiabatic invariance tilts us right over. On the other hand, if
the kicks have become \emph{large}, then $\theta_{sb}$ after any particular LK
cycle is far from $90^\circ$, and this is frozen in during the adiabatic
invariance phase. This explains the key observation that the initial
$\theta_{sb}$ eventually becomes the final $\theta_{sl}$, regardless of whether
it is $90^\circ$. Furthermore, it explains why the kicks to $\theta_{eff}$
become larger over time, but peak smaller for larger $I_0$.

The natural way to think about this is to consider the evolution of the
trajectory in $(a, e)$ space. There are two curves that can be drawn on here,
$\mathcal{A} \sim 1$ and $\abs{\Delta \theta_{sb}} \sim 1$ (the kick size), and
then we can see which one gets crossed first. The hypothesis is that the first
always gets crossed first, but if $e_{\max}$ is too large, then the second gets
crossed in the same LK cycle, and we get kicked far away from the starting
$\theta$, and have this frozen into $\theta_{sl}$. We need to find out how to
draw these boundaries in $(a, e)$ space. Drawing $\mathcal{A}$ is very easy,
since we have the explicit formulla for it.

To get the kick size, we have to integrate one of the LK peaks. This is easiest
done by considering the evolution of the $\delta e \equiv 1 - e$ variable by
dotting $\vec{e}$ into $\rd{\vec{e}}{t}$, such that
\begin{align}
    2e\rd{e}{t} &= \rd{\p{\vec{e} \cdot \vec{e}}}{t}
        = 2\vec{e} \cdot \rd{\vec{e}}{t},\\
        &= -\frac{15}{2t_{LK}}\p{\bm{e} \cdot \hat{n}_2}
            \p{\hat{n}_2 \cdot \p{\bm{e} \times \bm{j}}},\\
        &\lesssim \pm\frac{15}{2t_{LK}}e^2\sqrt{1 - e^2},\\
    \rd{e}{t} &\sim -\frac{15}{4t_{LK}}e\sqrt{1 - e^2},\\
    \rd{(\delta e)}{t} &\sim \frac{15}{4t_{LK}}\sqrt{2\delta e},\\
    \delta e(t) &\sim \p{\frac{15 t}{4\sqrt{2}t_{LK}}}^2.
\end{align}
The finding of a power law/quadratic seems in accordance w/ my simulations,
though I have to plot $\delta e - \delta e_{\min}$. Then, we can simply
integrate
\begin{align}
    \Delta \theta_{sb} &\sim \oint\limits_{LK}
            \Omega_{SL}\;\mathrm{d}t,\\
        &\sim \frac{\epsilon_{SL}}{2}\p{\frac{a_0}{a}}^{5/2}
            \oint\limits_{LK} \frac{1}{\delta e}\;\mathrm{d}t,\\
        &\sim \frac{\epsilon_{SL}}{2}\p{\frac{a_0}{a}}^{5/2}
            \oint\limits_{LK}
            \frac{1}{\delta e_{\min} + \p{\frac{15}{4\sqrt{2}t_{LK}}}^2
                t^2}\;\mathrm{d}t,\\
        &\sim \frac{\epsilon_{SL}}{2 \delta e_{\min}}\p{\frac{a_0}{a}}^{5/2}
            \pi \frac{4t_{LK} \sqrt{2 \delta e_{\min}}}{15},\\
        &\sim \frac{\epsilon_{SL}}{\sqrt{2\delta e_{\min}}}\frac{a_0}{a}
            \pi \frac{4}{15}.
\end{align}
In the last few steps, we've just taken the bounds of integration to be $t \in
[-\infty, \infty]$ for simplicity (they contribute negligibly), and used $t_{LK}
= (a / a_0)^{3/2}$ since $t_{LK, 0} = 0$.

If we now explicitly write down the criteria where $\mathcal{A} \sim 1$ and
$\Delta \theta_{sb} \sim 1$ in the $(a, e)$ plane, then we obtain
\begin{align}
    a_{c,\theta} &\sim \frac{\epsilon_{SL}}{\sqrt{2\delta e_{\min}}}
        \frac{4\pi}{15},\\
    a_{c, \mathcal{A}} &\sim
        \s{\epsilon_{SL}\frac{8/3}{\sqrt{1 - e_{\min}^2}
            \p{1 + 4e_{\min}^2}\abs{\sin 2I}}}^{1/4}.
\end{align}
The key difference between the two is that kicks occur at $e_{\max}$ or $\delta
e_{\min}$, while the adiabaticity parameter is moreso evaluated at $e_{\min}$.

\textbf{Update:} This cannot be the correct mechanism since it would generate
symmetric scatter of $\theta_{sl}$ about $\theta_{sb, 0}$, which is not the case
(see Fig 19 of LL18). Instead, it must really be how quickly the axis of
precession of $\rd{\hat{\bm{S}}}{t}$ moves compared to the precession
frequency, or indeed $\abs{\Omega_{eff}}$ as compared to
$\rd{\hat{\bm{\Omega}}_{eff}}{t}$.

\subsection{More analysis on LL18's proposal}

Note that, since $\theta_{sb}$ during the LK oscillations will receive a
sequence of kicks, it randomizes the ordering a bit, so exact conservation of
$\theta_{sb, i}$ to $\theta_{sl, f}$ is not maintained (i.e.\ the ordering can
change somewhat).

But ultimately, it must boil down indeed to comparison of the change in
precession axis vs the precession frequency. One of the key difficulties in this
conclusion in LL18 is neglect of nutation in Equations~64 and~65. However, in
the transadiabatic regime, the LK cycles are of small amplitude ($1 - e$
typically at least $\lesssim 0.1$, often $\lesssim 0.01$ throughout the cycles)
and are fast, and as a result $I$ is to good approximation constant and nutation
can likely be neglected; at worst an average value of $\bm{L}$ can be used. The
final spread in $\theta_{sl, f}$ probably comes from the spread in $\theta_{sb,
i}$ upon exiting the nonadiabatic regime, due to the kicks during the LK cycles.
\textbf{NB:} Another way to argue that the fast nutation can be ignored is if
$\Delta I \ll \theta_{\rm eff, S}$, since then the spin vector just precesses
around a fuzzy vector, which isn't a huge deal. If the precession frequencies
are equal, it's possible to hit an SHO-like resonance, which should probably be
dealt with TODO\@.

Let's suppose this is the case for the time being, where $e$, $I$, and $a$ are
all approximately slowly varying going into the transadiabatic regime. Then
let's go to the co-rotating frame with $\bm{L}$ (fix this in the $\hat{x},
\hat{z}$ plane) and look at the evolution of the components of $\bm{\Omega}_{\rm
eff}$:
\begin{align}
    \bm{\Omega}_{\rm eff} &= \Omega_{SL}\p{\sin I \hat{x} + \cos I \hat{z}}
            + \Omega_{pl} \hat{z},\\
    \hat{\bm{\Omega}}_{\rm eff} \cdot \hat{z}
        &= \frac{\Omega_{SL} \cos I + \Omega_{pl}}{
            \sqrt{\Omega_{SL}^2\sin^2 I + \p{\Omega_{SL}\cos I
                + \Omega_{pl}}^2}}.
\end{align}
Then, we just have to compare $\rd{\arccos\hat{\Omega}_{\rm eff, z}}{t}$ to
$\Omega_{\rm eff}$ the magnitude, and this tells us whether $\hat{S}$ can track
$\bm{\Omega}_{\rm eff}$ as it moves. This tracks the polar angle, and the $z$
component doesn't have a singularity during the evolution and is preferable
(compared to the $x$ component). If all quantities are slowly varying (at
roughly constant speeds), the characteristic speed at which the polar angle
varies occurs when it is $\sim 90^\circ$, or when $\Omega_{\rm eff, z} \approx
0$, so we can simply the expression a bit
\begin{equation}
    \rd{\arccos\hat{\Omega}_{\rm eff, z}}{t}
        \lesssim \rd{}{t}\p{\frac{\Omega_{SL} \cos I + \Omega_{pl}}{
            \Omega_{SL}\sin I}} \sim \frac{1}{\sin I}
                \rd{(\mathcal{A}^{-1})}{t}
\end{equation}

In summary, the picture is as follows:
\begin{itemize}
    \item Starting from some initial $\theta_{sb, 0}$, there are some random
        kicks (which cancel slightly better than a random walk, i.e.\ the
        variance does not seem to grow), we exit the nonadiabatic regime with
        some random value $\in \theta_{sb, 0} \pm \Delta \theta_{sb}$.

    \item Under the influences of $\epsilon_{GW}$ on $e_{\max}$ and
        $\epsilon_{GR}$ on $e_{\min}$, the trajectory flows towards a single
        point in $(a, e)$ space. Note that $I$ should be fixed by approximate
        conservation of the Kozai constant, since the GW effect is much weaker
        than the GR effect, and the GR effect preserves the Kozai constant.

    \item If the system hasn't exited the Kozai regime or merged at this point,
        it will evolve with small LK oscillations about GR decay of $e$ and $a$,
        coupled to convergence in $I$. As GR acts, this fuzz gets smaller and
        smaller amplitude until GR breaks the LK resonance.

        During this fuzzy phase, so long as $\Omega_{\rm eff} \gtrsim
        \rd{(\mathcal{A})^{-1}}{t}/\sin I$, then $\theta_{sb}$ gets sent to
        $\theta_{sl, f}$. The fuzz timescale is so short that it can be averaged
        over (the spin can't see it), so we should just have to consider the GR
        decay timescale when making this comparison.

    \item Regardless of whether the transadiabatic phase conserves the adiabatic
        invariant $\theta_{\rm eff}$, the final value is conserved once LK
        entirely disappears and it's just slow GR decay (which will evolve
        $\Omega_{\rm eff}$ slightly, but more obviously slowly).
\end{itemize}

\subsection{Timescales for My Picture}

\textbf{NB:} we are in the circulating regime of the L-K mechanism!

We now make some comments on the dynamics in each of these regimes:
\begin{itemize}
    \item During the initial pure-LK phase, there are small perturbations to
        $\theta_{sb}$ as we derived above.

        But furthermore, we can estimate the characteristic number of LK cycles
        by observing that the decay in the range of $\omega$ oscillation is what
        drives $e_{\min}$ to increase over time. We can integrate one Kozai
        cycle
        \begin{align}
            \Delta \omega &= \oint_{LK} \frac{3Gnm_{12}e}{c^2a\p{1 - e^2}}
                \;\mathrm{d}t,\\
            &\approx \frac{Gnm_{12}e}{2c^2a}
                \int\limits_{-\infty}^\infty \frac{1}{\delta e_{\min}
                    + \p{\frac{15 t}{4\sqrt{2}t_{LK}}^2}}\;\mathrm{d}t,\\
            &\approx \frac{3Gnm_{12} 4\pi t_{LK}}{c^2a\sqrt{2\delta e_{\min}}}.
        \end{align}
        It's likely we can replace $\sqrt{2 \delta e_{\min}} \to \sqrt{1 -
        e_{\max}^2}$.

        We should be able to determine the number of Kozai cycles before
        coalescence by computing $\pd{H}{\omega}$ at $e_{\max}$, which I haven't
        done.

        If we ignore GW effects, the final state for this phase is where $e
        \approx e_{\max}, I \approx I_{\min}$. There are small corrections due
        to (i) GW decay near the high-e phases; we can estimate the former,
        since we also know the number of high-e cycles, but it may not be very
        important.

    \item During the fuzz phase, let's assert that $\omega, I$ make small
        amplitude oscillations about mean values that evolve slowly under GW
        emission (which also affects $a$): note that $I$ is affected because
        GW emission is approximately adiabatic compared to the LK timescale.
        What sets the frequency and amplitude of these oscillations?

        \textbf{Gave up:} some online references seem to suggest that
        oscillations get to order $\sim t_{LK} / 6$ as we have defined
        it\footnote{\url{https://arxiv.org/pdf/1504.05957.pdf}}, while in our
        simulations, each LK cycle actually is much longer than this initially,
        so that gives us a decent idea of the timescale of the ``fuzz.''
        \textbf{Edit:} It's probably even faster, since this is the librating
        timescale, so let's just assume the fuzz is very short scale.
        \textbf{Edit 2:} It is bound from below by $\Omega_{GR}$, since that's
        one component of $\dot{\omega}$, and it must be at least as fast.

        Looking at the phase portrait, it's more clear that the GR precession
        will eventually just send entire trajectory to be roughly constant at
        $e_{\max} \to 1$. It's not clear that the amplitude of these
        oscillations ever saturates, but it's obvious that they are small and
        continue to decrease. One way to see that this is the case is to
        consider the $H(\omega, x)$ surface, where we drop constant of
        proportionality
        \begin{equation}
            H \propto \p{2 + 3e^2}\p{3\cos^2 I - 1}
                + 15e^2\sin^2I \cos 2\omega,
        \end{equation}
        and $I$ is implicitly defined by conservation of the Kozai constant $K =
        \sqrt{1 - e^2}\cos I$. We can see that along the separatrix, $H = -2$,
        if we give a kick at the location of maximum eccentricity ($\sin^2 I =
        2/5, e = 1, \omega = \pi/2$), the change in $H$ is quadratic like
        \begin{equation}
            \delta H = \frac{1}{2}\ptd{H}{\omega}\p{\delta \omega}^2,
        \end{equation}
        where $\delta \omega \sim \p{1 - e_{\max}^2}^{-1/2}$ was an earlier
        result we showed. The sign of this term is negative, so $H$ is being
        driven towards oscillating at large $e$ with small amplitude.

        In any case, the fuzz decreases in amplitude over time and oscillates
        faster and faster, probably $\ll t_{LK}$ (indeed so, according to my
        plots).

    \item As we evolve through the fuzz, we want to understand whether
        $\theta_{\rm eff, S}$ evolves adiabatically. We need to evaluate the
        precession frequency and the rate of change of the precession axis, but
        for this we need expressions for $\dot{e}, \dot{I}, \dot{a}$ through the
        fuzz. Based on the final observation that $\dot{\omega}_{GR}$ doesn't
        affect the mean eccentricity, we can assert that $\dot{e} =
        \dot{e}_{GW}$, $\dot{a} = \dot{a}_{GW}$, while $I$ is constrained
        implicitly by conservation of the Kozai constant (so long as Kozai still
        is active). Thus, to order of magnitude, $\dot{\hat{\Omega_{\rm eff}}} \sim
        \Omega_{GW}$ while $\Omega_{\rm eff} \sim \Omega_{GR}$, and since
        $\Omega_{GW} \propto 1/\p{a^4 x^{7/2}}$ while $\Omega_{GR} \propto
        1/\p{a^{5/2}x}$, it's clear that for sufficiently large eccentricities
        exiting the fuzz regime that $\hat{S}$ will not keep up with
        $\hat{\Omega}_{\rm eff, S}$.

        If so, what is the predicted $\theta_{\rm sl, f}$? Well, suppose that
        $\hat{L}$ ends up on the ring with uniform $I$ (probably\dots), and take
        the limit where $\hat{S}$ is not able to respond at all, then
        $\theta_{\rm sl, f} \in \s{I - \theta_{\rm sb, i}, 2\pi - I -
        \theta_{\rm sb, i}}$ and is roughly centered on $\pi - \theta_{\rm sb,
        i}$.

        \textbf{NB:} Above, we said $\Omega_{\rm eff} \sim \Omega_{GR}$ based on
        saying that $\hat{L}$ precesses around $\hat{L}_{\rm out}$ with
        $\dot{\omega}_{GR}$, but of course, if evolution is sufficiently abrupt,
        we should really use $\Omega_{\rm eff} \sim \Omega_{\rm pl}$, and if
        evolution is abrupt this is $\ll \Omega_{GR}$, further contributing to
        making the nonadiabatic criterion easy to satisfy.

    \item Note that there is one more way that this picture can break down, as
        we saw examples of in Bin's paper: we can get trapped in the LK
        resonance such that $\omega$ librates instead of circulating. This is
        not in general easy to do, since we start with $e \neq 0, \omega = 0$.
        Furthermore, to linear order, $\oint
        \pd{H}{\omega}\rd{\omega}{t}\;\mathrm{d}t = 0$ along the separatrix (it
        cancels during the increasing $e$ and decreasing $e$ phases).
        But we can imagine that if $\omega_{GR}$ is so strong, then
        $\dot{\omega}$ will drive the Kozai cycle inside the resonance during
        just the increasing $e$ phase alone, and takes a very different route
        back to low $e$ such that it is captured. That this is a nonlinear
        effect in $\delta\omega_{GR}$ might be important, since otherwise the
        resonance capture dynamics would only depend on the initial condition:
        the separatrix would open a gap like in the CS problem and for
        arbitrarily weak $\omega_{GR}$ we could still experience separatrix
        capture, which is obviously not the case?

        The advantage of invoking this mechanism is twofold: (i) if we look at
        Fig.~19 of LL18, it's clear that the distribution of $\theta_{sl, f}$ is
        roughly symmetric for a stronger companion (faster LK cycles), but
        becomes markedly asymmetric for a weaker companion (LK is weak). The
        violation of adiabiticity proposed above is generally expected to
        generate a $\theta_{sl, f}$ distribution symmetric about its mean. But
        capturing $\hat{L}$ into the $\omega = \pi/2$ resonance means $\hat{S}$
        precesses \emph{towards} it as it becomes dominant, meaning that
        $\theta_{sl, f} \lesssim 90^\circ$ is enforced. (ii), the above
        mechanism does not depend on the properties of the perturber or of the
        Kozai timescale, so there should be no change in distribution of
        $\theta_{sl, f}$ as a function of $a_{\rm out}$. This resonance capture
        mechanism provides a way for the outcome to be sensitive to the
        perturber properties.
\end{itemize}

Edit: Looks like I got $\Omega, \omega$ confused, and most of the above is
either wrong or not new.

\section{Fresh Start}

\textbf{NB:} I think these orbital elements Kozai actually give much slower
inspirals. It could be because my atol/rtol params were too loose when I was
doing the vector simulations, so we should prefer the \lstinline{4sims} line of
results, which qualitatively seem to agree with Bin's.

\subsection{Useful Kozai Results}

I have a bunch of formulas that I need to write down before I forget them, so
I'll do that here. We have begun analyzing the EOM (at quadrupolar order) in
Keplerian orbital elements, so I'll reproduce them here
\begin{align}
    \rd{a}{t} &= -\frac{64}{5}\frac{a}{t_{GW,0}} \frac{1}{\p{1 - e^2}^{7/2}}
        \p{1 + \frac{73}{24}e^2 + \frac{37}{96}e^4},\\
    \rd{e}{t} &= \frac{15}{8t_{LK}} e\sqrt{1 - e^2}\sin 2\omega
        \sin^2 I - \frac{304}{15}\frac{e}{t_{GW, 0}}\frac{1}{\p{1 - e^2}^{5/2}}
            \p{1 + \frac{121}{304}e^2},\\
    \rd{\Omega}{t} &= \frac{3}{4t_{LK}}
        \frac{\cos I\p{5e^2 \cos^2\omega - 4e^2 - 1}}{\sqrt{1 - e^2}},\\
    \rd{I}{t} &= -\frac{15}{16t_{LK}}\frac{e^2\sin 2\omega \sin 2I}{
        \sqrt{1 - e^2}},\\
    \rd{\omega}{t} &= \frac{3}{4t_{LK}}
        \frac{2\p{1 - e^2} + 5\sin^2\omega
            (e^2 - \sin^2 I)}{\sqrt{1 - e^2}} + \frac{\Omega_{GR,0}}{1 - e^2},\\
    \rd{\hat{\bm{S}}}{t} &= \frac{\Omega_{SL,0}}{1 - e^2}\hat{\bm{L}} \times \hat{\bm{S}}.
\end{align}
Here, we have defined
\begin{align}
    t_{LK}^{-1} &= n\p{\frac{m_3}{m_{12}}}\p{\frac{a}{\bar{a}_3}}^3,\\
    t_{GW,0}^{-1} &= \frac{G^3 \mu m_{12}^2}{c^5a^4},\\
    \Omega_{GR,0} &= \frac{3Gnm_{12}}{c^2a},\\
    \Omega_{SL,0} &= \frac{3Gn\p{m_2 + \mu/3}}{2c^2a}.
\end{align}
and $n = \sqrt{Gm_{12}/a^3}$ is the mean motion of the inner binary. We
define/recall the following:
\begin{itemize}
    \item $K = \sqrt{1 - e^2}\cos I$ is conserved, and we will sometimes write
        $x = 1 - e^2$.

    \item Kozai eccentricity excursions occur at $\omega = \pi/2, 3\pi/2$.

    \item If we ever need this, Natalia's paper gives ``closed'' forms for the
        eccentricity evolution

        \begin{align}
            x &= x_0 + \p{x_1 - x_0}\mathrm{cn}^2(\theta, k^2),\\
            \theta &= \frac{K}{\pi}\p{n_e t + \pi},\\
            n_e &= \frac{6\pi \sqrt{6}}{8Kt_{LK}}\sqrt{x_2 - x_1},\\
            k^2 &= \frac{x_0 - x_1}{x_2 - x_1}.
        \end{align}
        Here, $x_0$ and $x_1$ are the maximum/minimum $x$ respectively
        (corresponding to min/max eccentricity), and $x_2$ is the other root to
        the quadratic ($x_1$ is one of them).
        \begin{equation}
            x^2 - \frac{1}{3}\p{5 + 5K - 2x_0}x + \frac{5K}{3} = 0.
        \end{equation}

        Note that this implies $x_1 + x_2 = \frac{5 + 5K - 2x_0}{3}$.

    \item In terms of this $x$ parameter, the LK components of the EOM
        take on particularly simple form
        \begin{align}
            \dot{\Omega} &= \frac{3\sqrt{K}}{4t_{LK}}\p{1 - 2\frac{x_0 - h}{x -
                h}},\\
            \dot{I} &= \frac{\dot{x} \cos I}{2x \sin I}.
        \end{align}
\end{itemize}

It is not so hard to solve for the Kozai resonance location in the absence of GW
radiation; we know this occurs at $\omega = \pi/2$, which forces $\dot{e} =
\dot{I} = 0$, then we set $\dot{\omega} = 0$ and find
\begin{align}
    \rd{\omega}{t} &= \frac{3}{4t_{LK}} \frac{2(1 - e^2) + 5(e^2 - \sin^2 I)}{
            \sqrt{1 - e^2}} + \Omega_{GR},\\
    0 &= \frac{\Omega_{GR}\sqrt{1 - e^2}4t_{LK}}{3}
        + 2\p{1 - e^2} - 5\p{1 - \cos^2 I - e^2},\\
    5\cos^2 I &= 3\p{1 - e^2} + \mathcal{O}\p{\Omega_{GR}}.
\end{align}
Then, given some $K$, which is conserved even with GR precession, we know $(I,
e)$ the Kozai resonance. To obtain the $\mathcal{O}\p{\Omega_{GR}}$ correction,
we have to solve a quadratic, which yields
\begin{equation}
    \sqrt{1 - e^2} = \frac{5\cos^2I}{6}\p{2 +
        \sqrt{1 + \frac{16 \Omega_{GR}t_{LK}}{25 \cos^4 I}}}.
\end{equation}
If $\Omega_{GR}$ is strong, the equilibrium condition drives $\cos^2 I \to 0$,
and simplifying in this limit we get the familiar condition $\Omega_{GR}t_{LK}
\leq 9/4$ for the Kozai resonance itself to exist (of course, the separatrix
about which trajectories librate will begin shrinking much earlier).

We can use this to understand what $\mathcal{A}$ looks like when Kozai
disappears. Assuming $m_1 = m_2$, we can find $\Omega_{SL} =
\Omega_{GR}\frac{7}{24}$, and so when Kozai dies we obtain constraints
\begin{align}
    \Omega_{SL}t_{LK} &= \frac{21}{32},\\
    \mathcal{A} \simeq \abs{\frac{\Omega_{SL}}{\dot{\Omega}}}
        &= \frac{7}{8}\frac{\sqrt{1 - e^2}}{\cos I\p{4e^2 + 1}}.
\end{align}
If we plug in the values near the Kozai equilibrium when it disappears, we find
rough scaling
\begin{equation}
    \mathcal{A} \simeq \frac{7}{8\p{1 + 4e^2} \cos I}.
\end{equation}
Thus, indeed, typically $\mathcal{A} \sim 1$ when Kozai dies.

Finally, it bears noting that
\begin{align}
    \mathcal{A} &= \mathcal{A}_0 \frac{1}{\p{1 + 4e^2}
            \sqrt{1 - e^2}\abs{\sin 2I}} ,\\
        &= \mathcal{A}_0 \frac{1}{2\p{1 + 4e^2}K \sin I}.
\end{align}
Thus, over the course of a Kozai cycle, where $\sin I \in [\sqrt{2/5}, 1]$, and
$e \in [0, 1]$, the adiabaticity does not actually change very much, unless
$\mathcal{A}_0 \propto a^{-4}$ changes significantly due to $\dot{a}_{GW}$.

\subsection{Hamiltonian Approach 1: Natalia Style}

We go to the frame where $\uv{L}$ is stationary. The rotation vector is the
same as in SL15, and we obtain Hamiltonian ($\uv{L}_o$ is the outer angular
momentum, is constant in nonrotating frame)
\begin{align}
    H &= \Omega_{SL}\cos \theta - \bm{R} \cdot \uv{S},\\
    \bm{R} &\equiv \p{\dot{\Omega} \uv{L}_{o}
        + \dot{I}\p{\frac{\uv{L}_{o} \times \uv{L}}{\sin I}}}.
\end{align}
If we break down all the vectors into component form, such that $\uv{L} =
\uv{z}$, $\uv{L}_{o} = -\sin I \uv{x} + \cos I \uv{z}$, then we obtain
\begin{equation}
    H = \Omega_{SL}\cos \theta - \dot{\Omega}
        \p{-\sin I \sin \theta \cos \phi + \cos I \cos \theta}
        + \dot{I}\sin \theta \sin \phi.
\end{equation}
Note ICs $\theta = 0$, $\phi = 0$, $I = 90^\circ$. The EOM are
\begin{align}
    \dot{\phi} &= \pd{H}{\cos \theta} = \Omega_{SL} - \dot{\Omega}\cos I
        - \cot \theta\p{\dot{\Omega} \sin I \cos \phi - \dot{I}\sin \phi},\\
    \rd{(\cos \theta)}{t} &= -\pd{H}{\phi} =
        -\dot{\Omega} \sin I \sin \theta \sin \phi
        + \dot{I}\sin \theta \cos \phi,\\
    \rd{\theta}{t} &= -\frac{1}{\sin \theta}\rd{(\cos \theta)}{t}
        = \dot{\Omega}\sin I \sin \phi - \dot{I}\cos \phi.
\end{align}
If we assume $\Omega_{SL} \ll \dot{\Omega}, \dot{I}$ initially, even during LK
peaks (which is true by our experience), then we can imagine breaking down the
trajectory of $\uv{S}$ into a zeroth order precession about $\bm{R}$ (which is
very complicated, since $\bm{R}$ is both moving and changing in magnitude) and a
leading order perturbation due to $\Omega_{SL}$. The perturbation Hamiltonian is
then given
\begin{equation}
    H^{(1)} = \Omega_{SL}(t) \s{\cos \theta}(t).
\end{equation}
If we're brave like Natalia, we would expand $\cos \theta(t)$ in Fourier
components, and $\Omega_{SL}(t)$ in Fourier components, but there is clearly no
chance for a resonance here since there is no $\phi$ dependence, so the level
curves of this Hamiltonian are azimuthally symmetric and there can be no
resonance.

Conversely, if we're in the other regime $\Omega_{SL} \gg \dot{\Omega},
\dot{I}$, we must be in the regime where Kozai cycles have died out, which
implies $\dot{I} = 0$. Here, the Hamiltonian is much more similar to Natalia's
problem. Let's consider that $\theta = \theta_0$ and $\phi = \Omega_{SL}t$, then
the perturbing Hamiltonian is
\begin{equation}
    H^{(1)} = \dot{\Omega}\p{-\sin I \sin \theta_0 \cos \phi
        + \cos I \cos \theta_0}.
\end{equation}
Again, there is no resonance condition since there is only one $\phi$ dependent
term, and we really need two so we can get a form $\cos \p{\phi - Mt}$ like
Natalia's problem. Thus, there are no resonances to investigate here.

We can identify the key reasons that we don't have a similar problem:
\begin{itemize}
    \item LK is not a perturbation for us (compared to $\uv{S} \cdot \uv{L}$
        dynamics), it is significantly dominant. This corresponds to the
        $\mathcal{A} \ll 1$ regime of SL15. They obtain a neat bifurcation due
        to separatrix crossing, which is not observed in our LK simulations, so
        this cannot in spirit be a similar mechanism.

    \item SL15 focuses on adiabatically changing $\mathcal{A}$ and seeing how it
        encounters resonances. In our problem, nothing nontrivial can occur if
        $\mathcal{A}$ changes slowly.

    \item Our Hamiltonian takes on form $H = \p{\bm{\Omega}_{SL} - \bm{R}}
        \cdot \uv{S}$. This will never have any resonances since it's perfectly
        linear; anything that looks nonlinear is a pure consequence of
        coordinates (e.g.\ multiplication of $\theta$ and $\phi$ terms).
\end{itemize}

\subsection{Hamiltonian 2: Rotating Style}

Why is the previous Hamiltonian hard to use? Well, since there are no
resonances, and $H$ is linear in $\uv{S}$, it makes much more sense to just
analyze the EOM\@. As such, it's better to just find the right set of rotations
such that we have a convenient coordinate. LL18 proposed this $\theta_{\rm S,
eff}$, and I think this is the right idea, but it can be expounded on.

Let's consider the following: we clearly want to rotate by at least
$\dot{\Omega} \uv{L}_o$, so that $\bm{L}$ does not precess any more, but it
still nutates ($\dot{I}$) about fixed $\uv{L}_o = \uv{z}$. Let's first write
down the Hamiltonian and the EOM for this case:
\begin{align}
    H &= \Omega_{SL} \uv{S} \cdot \uv{L} - \dot{\Omega} \uv{L}_o \cdot \uv{S},\\
        &= \Omega_{SL}\p{\sin I \sin \theta \cos \phi
            + \cos I \cos \theta}
            - \dot{\Omega}\cos \theta,\\
    \rd{\cos \theta}{t} &= -\pd{H}{\phi}
        = -\Omega_{SL}\sin I\sin \theta \sin \phi,\\
    \rd{\theta}{t} &= \Omega_{SL}\sin I \sin \phi,\\
    \rd{\phi}{t} &= \pd{H}{\cos \theta}
        = -\Omega_{SL}\sin I \cot \theta \cos \phi + \Omega_{SL}\cos I
            - \dot{\Omega}.
\end{align}
This is an even stupider example than the previous section, since the desired
final angle $\theta_{sl, f}$ is almost impossible to measure, and writing down
$\dot{\theta}_{sl, f}$ would give huge excursions early in the evolution due to
$\dot{I}$ (we've seen this plot before, in LL18). But similarly, the EOM from
the previous section is also very difficult to use, since it's very unclear how
$\theta = \theta_{sl}$ evolves through the Kozai phase; with the benefit of
hindsight, we know that this $\dot{\theta}$ equation is just along a great
circle normal to $\bm{R}$, but it's hard to say anything quantitative other than
``this angle gets frozen by conservation of adiabatic invariant.''

Instead, let's consider applying an arbitrary rotation in the $\uv{y}$
direction for the time being, let's call it $\bm{R} = \dot{I}_o \uv{y}$; taking
this to equal either $0$ or $I$ equates to taking $\uv{L}_o$ or $\uv{L}$ as
$\uv{z}$ respectively. We can write down this Hamiltonian, calling $I_L = I -
I_o$ (these have interpretation of $\cos I_L = \uv{L} \cdot \uv{z}$ and $\cos
I_o = \uv{L}_o \cdot \uv{z}$ respectively)
\begin{align}
    H &= \Omega_{SL} \uv{S} \cdot \uv{L} - \uv{R} \cdot \uv{S},\\
        &= \Omega_{SL}\p{\sin I_L \sin \theta \cos \phi
            + \cos I_L \cos \theta}
            - \dot{\Omega}\p{-\sin I_{o} \sin \theta \cos \phi
                + \cos I_o \cos \theta}
            + \dot{I}_o \sin \theta \sin \phi,\\
    \rd{\cos \theta}{t}
        &= -\p{\Omega_{SL} \sin I_L + \dot{\Omega} \sin I_o}\p{-\sin \theta \sin
            \phi}
            - \dot{I}_o \sin \theta \cos \phi,\\
    \rd{\theta}{t}
        &= -\p{\Omega_{SL} \sin I_L - \dot{\Omega} \sin I_o}\sin \phi
            + \dot{I}_o \cos \phi,\\
    \rd{\phi}{t}
        &= -\cot \theta\p{\Omega_{SL}\sin I_L \cos \phi
                + \dot{\Omega}\sin I_o \cos \phi
                - \dot{I}_o \sin \phi}
            + \p{\Omega_{SL}\cos I_L - \dot{\Omega}\cos I_o}.
\end{align}
Immediately, we can see both of the terms that made the above equations hard to
work with: both $\Omega_{SL}\sin I_L$ and $\dot{\Omega} \sin I_{o}$ become
large at some point or another, making it hard to consider the effect on the
final $\theta$. But if we choose some $\dot{I}_o$ such that $\theta(t = 0) =
\theta_{sb}$ while $\theta(t = \infty) = \theta_{sl}$, then the EOM is easy to
analyze in both regimes.

Instead, the obvious thing to do is as follows: choose $\dot{I}_o$ such that
$\Omega_{SL} \sin I_L = \dot{\Omega} \sin I_o$, also satisfying $I_L + I_o = I$.
This ensures that the two terms $\Omega_{SL}$ and $\dot{\Omega}$ are almost
always small, while $\dot{I}_o$ is generally symmetric per cycle. On the other
hand, $\dot{\phi} \approx \max\p{\Omega_{SL}, \dot{\Omega}}$.

This immediately lets us read off one adiabaticity criterion: if $\cos \phi$
doesn't sweep through $2\pi$ much faster than it takes $\dot{I}_o$ to
accumulate, then $\theta$ will receive a kick. Another adiabaticity criterion is
obvious: the number of times we cross $\Omega_{SL} \simeq \dot{\Omega}$ will
yield a substantially larger kick than usual, since $\Omega_{SL}, \dot{\Omega}
\gg \dot{I}_o$.

Let's be a bit more quantitative and write down this $\dot{I}_o$ rotation. It
must satisfy (recall $I_L = I - I_o$)
\begin{align}
    \Omega_{SL}\sin I_L &= \dot{\Omega} \sin I_0,\\
    \dot{\Omega}_{SL}\sin I_L + \Omega_{SL}\cos I_L \p{\dot{I} - \dot{I}_o}
        &= \ddot{\Omega} \sin I_o + \dot{\Omega} \cos I_0 \dot{I}_o,\\
    \dot{\Omega}_{SL}\sin I_L - \ddot{\Omega} \sin I_o +
            \Omega_{SL}\cos I_L \dot{I}
        &= \dot{I}_o\p{\dot{\Omega} \cos I_0 + \Omega_{SL} \cos I_L},\\
    \dot{I}_o &= \frac{\dot{\Omega}_{SL}\sin I_L
            - \ddot{\Omega} \sin I_o
            + \Omega_{SL}\cos I_L \dot{I}}{
        \dot{\Omega} \cos I_0
            + \Omega_{SL} \cos I_L}
\end{align}
This is an absolute mess, but does it hold up?
\begin{itemize}
    \item Well, if we are in the $\dot{\Omega} \sim \dot{I} \gg \Omega_{SL}$
        limit, then $I_o \approx 0$, and things simplify to
        \begin{equation}
            \frac{\dot{I}_o}{\dot{I}} \approx
                \frac{\dot{\Omega}_{SL} \sin I / \dot{I} + \Omega_{SL}\cos I}
                {\dot{\Omega}} \ll 1,
        \end{equation}
        since all time derivatives are the same, $\sim 1/t_{LK}$. This is
        correct, since the rotation should basically not be acting in this
        limit.

    \item And in the other limit, $\dot{\Omega} \sim \dot{I} \ll
        \Omega_{SL}$ limit, then $I_l \approx 0$ and we have
        \begin{equation}
            \frac{\dot{I}_o}{\dot{I}} \approx
                \frac{-\ddot{\Omega}\sin I_0 / \dot{I}
                    + \dot{I} \Omega_{SL} \cos I}{
                        \Omega_{SL} \cos I}
                \approx 1.
        \end{equation}
        This is also correct, since in this limit we should have to rotate by
        $\dot{I}$.
\end{itemize}
Thus, it seems like we have the correct expression, as ugly as it is. In fact,
numerically, this turns out to be exactly $\bm{\Omega}_{\rm eff}$, which
shouldn't be a huge surprise, since that's exactly our definition. In this case,
it seems easier to just use Dong's suggestion along with Natalia's Hamiltonian
and EOM\@.

\section{Finding a Resonance}

Recall EOM from when we rotated $\hat{L} \propto \hat{z}$ (here, $\theta =
\theta_{sl}$):
\begin{align}
    \dot{\phi} &= \pd{H}{\cos \theta} = \Omega_{SL} - \dot{\Omega}\cos I
        - \cot \theta\p{\dot{\Omega} \sin I \cos \phi - \dot{I}\sin \phi},\\
    \rd{\theta}{t} &= -\frac{1}{\sin \theta}\rd{(\cos \theta)}{t}
        = \dot{\Omega}\sin I \sin \phi - \dot{I}\cos \phi.
\end{align}
If we directly substitute our known forms for $\dot{\Omega}$ and $\dot{I}$, we
obtain
\begin{align}
    \rd{\theta}{t}
        &= \frac{3\sin 2I}{8t_{LK}\sqrt{x}}\s{
            \p{5e^2\cos^2 \omega - 4e^2 - 1}\sin \phi
            - \p{\frac{5e^2\sin 2\omega}{2}}\cos \phi},\\
        &= \frac{3\sin 2I}{8t_{LK}\sqrt{x}}\p{
            -\sin \phi\p{\frac{3e^2}{2} - 1}
                + \frac{5e^2}{2}\sin\p{\phi - 2\omega}}.
\end{align}
If $\phi$ is slowly varying compared to $\omega$, the second term just becomes
an $\dot{I}$, and we indeed find the total change in $\theta$ is indeed just
$I$. Thus, we put together $\theta_{sl,i} + I = \theta_{sl, f}$, in the peaceful
limit.

But there does seem to be a resonance here, if $\dot{\phi} = 2\dot{\omega}$, or
if just $\dot{\phi} = 0$, then some kicks will add rather than cancel out. It's
not obvious what the final value will be, but this is a breakdown condition to
the above equality.

However, this doesn't seem to be the entire picture, since $\theta_{sb}$ seems
to be conserved to different extents in my $I = 90.45^\circ$ and $I =
90.5^\circ$ simulations. This again highlights the importance of choosing a good
coordinate system, since $\theta_{sb}$ is very difficult to analyze in this
coordinate system; we can't find the resonance that causes this.

We can see the origin of the $\theta_{sb}$ resonance as well: recall EOM
\begin{equation}
    \rd{\theta_{sb}}{t} = \Omega_{SL}\sin I \sin \phi.
\end{equation}
Since $\Omega_{SL}\sin I$ is periodic in $t_{LK}$, we can write
\begin{align}
    \rd{\theta_{sb}}{t} &= \sin \phi \sum\limits_{N = -\infty}^\infty
            \tilde{\Omega}_{SL}\cos\p{\frac{Nt}{t_{LK}}},\\
        &= \sum\limits_{N = -\infty}^\infty
            \frac{\tilde{\Omega}_{SL}}{2}\s{\sin\p{\phi - \frac{Nt}{t_{LK}}}
                + \sin \p{\phi + \frac{Nt}{t_{LK}}}},\\
    \rd{\phi_{sb}}{t} &= \pd{H}{\cos \theta}
        = -\Omega_{SL}\sin I \cot \theta \cos \phi + \Omega_{SL}\cos I
            - \dot{\Omega}.
\end{align}
Thus, there can be a resonance if $\dot{\phi}$ matches one of the harmonics of
$t_{LK}$. However, these resonances are indeed weaker, since they go with
$\Omega_{SL}$.

\textbf{Edit:} I don't think these are the resonances that I am seeing in the
simulations. Rather, it is much simpler: when $\theta_{sb} \approx 90^\circ$,
then $\rd{\phi_{\rm sb}}{t} \approx 0$ in between Kozai cycles. Then, when
$\phi_{sb}$ attains substantial values, the $\rd{\theta_{sb}}{t}$ term
activates even off LK peaks and $\theta_{sb}$ drifts from its initial
value. When $\theta_{sb} \neq 90^\circ$, then there is a slow and steady
$\Omega_{SL}$ term in $\dot{\phi}_{\rm sb}$ that prevents substantial drift of
$\theta_{\rm sb}$.

Is it possible to be more general than this?

\appendix

\section{$\bm{j}$ Equations}

We define vectors
\begin{align}
    \bm{j} &= \sqrt{1 - e^2}\hat{n},\\
    \bm{e} &= e\hat{u}.
\end{align}
Here, $\bm{j}$ is the dimensionless angular momentum vector and $\bm{e}$ is the
eccentricity vector; see LML15 for precise definitions. Note that $\bm{j} \cdot
\bm{e} = 0$, $j^2 + e^2 = 1$. Then, the EOM for the inner and outer vectors
satisfy to quadrupolar order
\begin{align}
    \rd{\bm{j}}{t} &= \frac{3}{4t_{LK}} \s{
        \p{\bm{j} \cdot \hat{n}_2} \p{\bm{j} \times \hat{n}_2}
        - 5\p{\bm{e} \cdot \hat{n}_2}\p{\bm{e} \times \hat{n}_2}
        },\\
    \rd{\bm{e}}{t} &= \frac{3}{4t_{LK}} \s{
        \p{\bm{j} \cdot \hat{n}_2} \p{\bm{e} \times \hat{n}_2}
        + 2\bm{j} \times \bm{e}
        - 5\p{\bm{e} \cdot \hat{n}_2}\p{\bm{j} \times \hat{n}_2}
        }.
\end{align}
Let's assume for the time being that $L_1 \ll L_2$, so the system is
sufficiently hierarchical that $\bm{j}_2$, $\bm{e}_2$ are constants. Note for
reference that
\begin{equation}
    t_{LK} \equiv \frac{L_1}{\mu_1 \Phi_0}
        = \frac{1}{n_1}\p{\frac{m_1 + m_2}{m_3}}
            \p{\frac{a_2}{a}}^3
            \p{1 - e_2^2}^{3/2}.
\end{equation}
Here, $n_1 \equiv \sqrt{G\p{m_1 + m_2}/a^3}$. Thus, $1 / t_{LK} \propto
a^{3/2}$.

GW radiation (Peters 1964) cause decays of $\bm{L}$ and $\bm{e}$ as
\begin{align}
    \at{\rd{\bm{L}}{t}}_{GW} &= -\frac{32}{5}\frac{G^{7/2}}{c^5}
        \frac{\mu^2 m_{12}^{5/2}}{a^{7/2}}
        \frac{1 + 7e^2/8}{\p{1 - e^2}^2}\hat{L},\\
    \at{\rd{\bm{e}}{t}}_{GW} &= -\frac{304}{15} \frac{G^3}{c^5}
        \frac{\mu m_{12}^2}{a^4\p{1 - e^2}^{5/2}}\p{1 + \frac{121}{304}
            e^2}\bm{e},\\
    \at{\frac{\dot{a}}{a}}_{GW} &= -\frac{64}{5} \frac{G^3 \mu m_{12}^2}{c^5a^4}
        \frac{1}{\p{1 - e^2}^{7/2}}\p{1 + \frac{73}{24}e^2
            + \frac{37}{96}e^4}.
\end{align}
Here, $m_{12} \equiv m_1 + m_2$. The last GR effect is precession of $\vec{e}$,
which acts as
\begin{equation}
    \at{\rd{\bm{e}}{t}}_{GR} = \frac{3Gnm_{12}}{c^2a\p{1 - e^2}}.
\end{equation}

Given this system (from LML15 + LL18), we can then add the spin-orbit coupling
term (from de Sitter precession), which is given in LL18 to be
\begin{align}
    \rd{\hat{S}}{t} &= \Omega_{SL}\hat{L} \times \hat{S},\\
    \Omega_{SL} &\equiv \frac{3Gn\p{m_2 + \mu/3}}{2c^2a\p{1 - e^2}}.
\end{align}
Note that $\mu$ is the reduced mass of the inner binary. We can drop the
back-reaction term since $S \ll L$. Thus, $\Omega_{SL} \propto a^{-5/2}$.

What is observed is that, as this system is evolved forward in time and GR
coalesces the inner binary, $\theta_{sl} \equiv \arccos\p{\hat{S} \cdot
\hat{L}}$ goes to $90^\circ$ consistently. The relevant figure is Fig.~19 of
LL18, which shows that for a close-in, low-eccentricity perturber ($\bar{a}_{\rm
out, eff} \propto a_{out}$), the focusing is significantly stronger. Note that
initially, $I \equiv \arccos\p{\hat{L} \cdot \hat{L}_2} \approx 90^\circ$ while
$\theta_{sl} \approx 0$.

Next, when accounting for GR, we should let $a$ evolve as above. Note that since
$\bm{j}$ and $\vec{e}$ are our dynamical variables, we should use $\bm{j} \equiv
\sqrt{1 - e^2} \hat{L} = \sqrt{1 - e^2}\frac{\bm{L}}{\mu \sqrt{Gm_{12}a\p{1 -
e^2}}}$ and rewrite
\begin{equation}
    \at{\rd{\bm{j}}{t}}_{GW} = \frac{1}{\mu\sqrt{GMa}}\at{\rd{\bm{L}}{t}}_{GW}
        - \frac{\bm{j}}{2a}\at{\rd{a}{t}}_{GW}.
\end{equation}
To double check, we should verify that $\at{\rd{(j^2 + e^2)}{t}}_{GW} = 0$,
which can be verified as (Let's set $G = M = \mu = a = c = 1$ for convenience)
\begin{align}
    \frac{1}{2}\rd{(j^2 + e^2)}{t}
        &= \bm{j} \cdot \rd{\bm{j}}{t} + \bm{e} \cdot \rd{\bm{e}}{t},\\
        &= \bm{j} \cdot \s{
            \p{-\frac{32}{5}\frac{1 + 7e^2/8}{\p{1 - e^2}^2}}\hat{L}
                - \frac{\bm{j}}{2}\p{-\frac{64}{5}
                    \frac{1 + 73e^2/24 + 37e^4/96}{\p{1 - e^2}^{7/2}}}}
            + \bm{e} \cdot
                \p{-\frac{304}{15}\frac{1 + 121e^2/304}{\p{1 - e^2}^{5/2}}}
                \bm{e},\\
        &= \p{-\frac{32}{5}\frac{1 + 7e^2/8}{\p{1 - e^2}^{3/2}}}
                + \p{\frac{32}{5}
                    \frac{1 + 73e^2/24 + 37e^4/96}{\p{1 - e^2}^{5/2}}}
            + e^2 \p{-\frac{304}{15}\frac{1 + 121e^2/304}{\p{1 - e^2}^{5/2}}}
                ,\\
        &= \frac{1}{15\p{1 - e^2}^{5/2}}\s{
            -96\p{1 - e^2}\p{1 + \frac{7e^2}{8}}
                + 96\p{1 + \frac{73e^2}{24} + \frac{37e^4}{96}}
                - 304e^2\p{1 + \frac{121e^2}{304}}}.
\end{align}
This can be verified to vanish upon term-by-term examination indeed.

\end{document}

