    \documentclass[11pt,
        usenames, % allows access to some tikz colors
        dvipsnames % more colors: https://en.wikibooks.org/wiki/LaTeX/Colors
    ]{article}
    \usepackage{
        amsmath,
        % multicol,
        amssymb,
        fouriernc, % fourier font w/ new century book
        fancyhdr, % page styling
        lastpage, % footer fanciness
        hyperref, % various links
        setspace, % line spacing
        amsthm, % newtheorem and proof environment
        mathtools, % \Aboxed for boxing inside aligns, among others
        float, % Allow [H] figure env alignment
        enumerate, % Allow custom enumerate numbering
        graphicx, % allow includegraphics with more filetypes
        wasysym, % \smiley!
        upgreek, % \upmu for \mum macro
        listings, % writing TrueType fonts and including code prettily
        tikz, % drawing things
        booktabs, % \bottomrule instead of hline apparently
        cancel % can cancel things out!
    }
    \usepackage[margin=1in]{geometry} % page geometry
    \usepackage[
        labelfont=bf, % caption names are labeled in bold
        font=scriptsize % smaller font for captions
    ]{caption}
    \usepackage[font=scriptsize]{subcaption} % subfigures

    \newcommand*{\scinot}[2]{#1\times10^{#2}}
    \newcommand*{\dotp}[2]{\left<#1\,\middle|\,#2\right>}
    \newcommand*{\rd}[2]{\frac{\mathrm{d}#1}{\mathrm{d}#2}}
    \newcommand*{\pd}[2]{\frac{\partial#1}{\partial#2}}
    \newcommand*{\rtd}[2]{\frac{\mathrm{d}^2#1}{\mathrm{d}#2^2}}
    \newcommand*{\ptd}[2]{\frac{\partial^2 #1}{\partial#2^2}}
    \newcommand*{\md}[2]{\frac{\mathrm{D}#1}{\mathrm{D}#2}}
    \newcommand*{\pvec}[1]{\vec{#1}^{\,\prime}}
    \newcommand*{\svec}[1]{\vec{#1}\;\!}
    \newcommand*{\bm}[1]{\boldsymbol{\mathbf{#1}}}
    \newcommand*{\ang}[0]{\;\text{\AA}}
    \newcommand*{\mum}[0]{\;\upmu \mathrm{m}}
    \newcommand*{\at}[1]{\left.#1\right|}

    \newtheorem{theorem}{Theorem}[section]

    \let\Re\undefined
    \let\Im\undefined
    \DeclareMathOperator{\Res}{Res}
    \DeclareMathOperator{\Re}{Re}
    \DeclareMathOperator{\Im}{Im}
    \DeclareMathOperator{\Log}{Log}
    \DeclareMathOperator{\Arg}{Arg}
    \DeclareMathOperator{\Tr}{Tr}
    \DeclareMathOperator{\E}{E}
    \DeclareMathOperator{\Var}{Var}
    \DeclareMathOperator*{\argmin}{argmin}
    \DeclareMathOperator*{\argmax}{argmax}
    \DeclareMathOperator{\sgn}{sgn}
    \DeclareMathOperator{\diag}{diag\;}

    \DeclarePairedDelimiter\bra{\langle}{\rvert}
    \DeclarePairedDelimiter\ket{\lvert}{\rangle}
    \DeclarePairedDelimiter\abs{\lvert}{\rvert}
    \DeclarePairedDelimiter\ev{\langle}{\rangle}
    \DeclarePairedDelimiter\p{\lparen}{\rparen}
    \DeclarePairedDelimiter\s{\lbrack}{\rbrack}
    \DeclarePairedDelimiter\z{\lbrace}{\rbrace}

    % \everymath{\displaystyle} % biggify limits of inline sums and integrals
    \tikzstyle{circ} % usage: \node[circ, placement] (label) {text};
        = [draw, circle, fill=white, node distance=3cm, minimum height=2em]
    \definecolor{commentgreen}{rgb}{0,0.6,0}
    \lstset{
        basicstyle=\ttfamily\footnotesize,
        frame=single,
        numbers=left,
        showstringspaces=false,
        keywordstyle=\color{blue},
        stringstyle=\color{purple},
        commentstyle=\color{commentgreen},
        morecomment=[l][\color{magenta}]{\#}
    }

\begin{document}

\def\Snospace~{\S{}} % hack to remove the space left after autorefs
\renewcommand*{\sectionautorefname}{\Snospace}
\renewcommand*{\appendixautorefname}{\Snospace}
\renewcommand*{\figureautorefname}{Fig.}
\renewcommand*{\equationautorefname}{Eq.}
\renewcommand*{\tableautorefname}{Tab.}

% \singlespacing

\pagestyle{fancy}
\rfoot{Yubo Su}
\rhead{}
\cfoot{\thepage/\pageref{LastPage}}

\title{Lidov-Kozai $90^\circ$ Attractor}
\author{Yubo Su}
\date{Date}

\maketitle

% \begin{multicols}{2}

\section{Equations}

\subsection{Bin's Papers}

Our major references will be Bin's paper with Diego + Dong in 2015 (LML15) and
Bin's later paper with Dong on spin-orbit misalignment (LL18). The target of
study is \S4.3 of LL18, where a $90^\circ$ attractor in spin-orbit misalignment
seems to appear when the octupole effect is negligible.

The easiest formulation is just to express everything in terms of $\bm{L}$ and
$\bm{e}$, following LL18. We drop octupole terms and hold the third perturber
constant. These equations come out to be (Eqs.~4--5 w/ substitutions)
\begin{align}
    \rd{\bm{L}}{t} &= \frac{3}{4t_{LK}} \mu \sqrt{Gm_{12}a} \s*{
        \p*{\bm{j} \cdot \hat{n}_2} \p*{\bm{j} \times \hat{n}_2}
        - 5\p*{\bm{e} \cdot \hat{n}_2}\p*{\bm{e} \times \hat{n}_2}
        },\\
    \rd{\bm{e}}{t} &= \frac{3}{4t_{LK}} \s*{
        \p*{\bm{j} \cdot \hat{n}_2} \p*{\bm{e} \times \hat{n}_2}
        + 2\bm{j} \times \bm{e}
        - 5\p*{\bm{e} \cdot \hat{n}_2}\p*{\bm{j} \times \hat{n}_2}
        }.
\end{align}
Note that $\bm{j} \equiv \sqrt{1 - e^2}\hat{\bm{L}} = \frac{\bm{L}}{\mu
\sqrt{Gm_{12}a}}$. $m_{12} = m_1 + m_2$ and $\mu = m_1m_2/m_{12}$. We've defined
\begin{equation}
    t_{LK} \equiv \frac{L_1}{\mu_1 \Phi_0}
        = \frac{1}{n_1}\p*{\frac{m_{12}}{m_3}}
            \p*{\frac{a_2}{a}}^3
            \p*{1 - e_2^2}^{3/2}.
\end{equation}
Here, $n_1 \equiv \sqrt{Gm_{12}/a^3}$. Thus, $1 / t_{LK} \propto a^{3/2}$.

GW radiation (Peters 1964) cause decays of $\bm{L}$ and $\bm{e}$ as
\begin{align}
    \at{\rd{\bm{L}}{t}}_{GW} &= -\frac{32}{5}\frac{G^{7/2}}{c^5}
        \frac{\mu^2 m_{12}^{5/2}}{a^{7/2}}
        \frac{1 + 7e^2/8}{\p*{1 - e^2}^2}\hat{L},\\
    \at{\rd{\bm{e}}{t}}_{GW} &= -\frac{304}{15} \frac{G^3}{c^5}
        \frac{\mu m_{12}^2}{a^4\p*{1 - e^2}^{5/2}}\p*{1 + \frac{121}{304}
            e^2}\bm{e}.
\end{align}
Here, $m_{12} \equiv m_1 + m_2$, and $a$ is implicitly defined by $\bm{L}$ and
$e$. The last GR effect is precession of $\vec{e}$, which acts as
\begin{align}
    \at{\rd{\bm{e}}{t}}_{GR} &= \frac{1}{t_{GR}}\hat{\bm{L}} \times \bm{e},\\
    \frac{1}{t_{GR}} &\equiv \frac{3Gnm_{12}}{c^2a\p*{1 - e^2}}.
\end{align}
Note that $t_{GR}^{-1} \propto a^{-5/2}$.

Given this system (from LML15 + LL18), we can then add the spin-orbit coupling
term (from de Sitter precession), which is given in LL18 to be
\begin{align}
    \rd{\hat{\bm{S}}}{t} &= \frac{1}{t_{SL}}\hat{\bm{L}} \times \hat{\bm{S}},\\
    \frac{1}{t_{SL}} &\equiv \frac{3Gn\p*{m_2 + \mu/3}}{2c^2a\p*{1 - e^2}}.
\end{align}
Note that $\mu$ is the reduced mass of the inner binary. We can drop the
back-reaction term since $S \ll L$. Thus, $t_{SL}^{-1} \propto a^{-5/2}$ as
well.

Finally, an adiabaticity parameter can be defined:
\begin{align}
    \mathcal{A} &\equiv \abs*{\frac{\Omega_{SL}}{\Omega_L}}.
\end{align}
Here, $\Omega_L \simeq\frac{3\p*{1 + 4e^2}}{8t_{LK}\sqrt{1 - e^2}}\abs*{\sin
2I}$ is an approximate rate of change of $L$ during an LK cycle

It's natural to redimensionalize to the initial LK time such that
\begin{equation}
    \frac{1}{t_{LK, 0}} \equiv \p*{\frac{a}{a_0}}^{3/2}\frac{1}{t_{LK}},
\end{equation}
since nothing else in $t_{LK}$ is changing. The next natural timescale for
gravitational waves is
\begin{equation}
    \frac{1}{t_{GW}} \equiv \frac{G^3 \mu m_{12}^2}{c^5a^4}
        \equiv \frac{1}{t_{GW, 0}} \p*{\frac{a_0}{a}}^4
        \equiv \epsilon_{GW} \frac{1}{t_{LK, 0}} \p*{\frac{a_0}{a}}^4.
\end{equation}
We can repeat the procedure for the GR precession term and the spin-orbit
coupling terms:
\begin{align}
    \frac{1}{t_{GR}} &= \epsilon_{GR}\frac{1}{t_{LK, 0}}
        \p*{\frac{a_0}{a}}^{5/2},\\
    \frac{1}{t_{SL}} &= \epsilon_{SL}\frac{1}{t_{LK, 0}}
        \p*{\frac{a_0}{a}}^{5/2}.
\end{align}
Thus, finally, if we let $\tau = t / t_{LK, 0}$, then we obtain full equations
of motion (note that $a_0 = 1$ below)
\begin{align}
    \rd{\bm{L}}{\tau} ={}& \p*{\frac{a}{a_0}}^{3/2} \frac{3}{4}\sqrt{a}\s*{
            \p*{\bm{j} \cdot \hat{n}_2} \p*{\bm{j} \times \hat{n}_2}
            - 5\p*{\bm{e} \cdot \hat{n}_2}\p*{\bm{e} \times \hat{n}_2}
            }\nonumber\\
        &- \epsilon_{GW}\p*{\frac{a_0}{a}}^4
            \frac{32}{5} \frac{1 + 7e^2/8}{\p*{1 - e^2}^{5/2}}\bm{L},\\
    \rd{\bm{e}}{\tau} ={}& \p*{\frac{a}{a_0}}^{3/2} \frac{3}{4}\s*{
            \p*{\bm{j} \cdot \hat{n}_2} \p*{\bm{e} \times \hat{n}_2}
            + 2\bm{j} \times \bm{e}
            - 5\p*{\bm{e} \cdot \hat{n}_2}\p*{\bm{j} \times \hat{n}_2}
            }\nonumber\\
        &-\epsilon_{GW}\p*{\frac{a_0}{a}}^4
            \frac{304}{15} \frac{1}{\p*{1 - e^2}^{5/2}}\p*{1 + \frac{121}{304}
                e^2}\bm{e}\nonumber\\
        &+ \epsilon_{GR}\p*{\frac{a_0}{a}}^{5/2}\frac{1}{1 - e^2}
            \hat{\bm{L}} \times \bm{e},\\
    \rd{\hat{\bm{S}}}{t} ={}&
        \epsilon_{SL}\p*{\frac{a_0}{a}}^{5/2}\frac{1}{1 - e^2}
        \hat{\bm{L}} \times \hat{\bm{S}}.
\end{align}
For reference, we note that $a = \abs*{\bm{L}}^2 / \p*{\mu^2 Gm_{12}\p*{1 -
e^2}}$, while $\bm{j} = \bm{L} / \p*{\mu\sqrt{Gm_{12}a}}$. To invert $a(\bm{L})$
and $\bm{J}\p*{\bm{L}}$ in this coordinate system where $a_0 = 1$, it is easiest
to choose the angular momentum dimensions such that $\mu \sqrt{Gm_{12}} = 1$,
such that now
\begin{align}
    \abs*{\bm{L}\p*{t = 0}} \equiv \mu \sqrt{Gm_{12}a_0\p*{1 - e_0^2}}
        &= \sqrt{\p*{1 - e_0^2}},\\
    a &= \frac{\abs*{\bm{L}}^2}{1 - e^2},\\
    \bm{j} &= \frac{\bm{L}}{\sqrt{a}} = \hat{\bm{L}}\sqrt{1 - e^2}.
\end{align}

Finally, the timescales are
\begin{align}
    t_{LK, 0} &= \frac{1}{n}
        \frac{m_{12}}{m_3}\p*{\frac{a_2}{a(t = 0)}}^3 \p*{1 - e_2^2}^{3/2},\\
    \epsilon_{GW} \equiv \frac{t_{LK, 0}}{t_{GW, 0}}
        &= \frac{1}{n}
            \frac{m_{12}}{m_3}\frac{a_2^3}{\p*{a(t = 0)}^7} \p*{1 - e_2^2}^{3/2}
            \frac{G^3 \mu m_{12}^2}{c^5},\\
    \epsilon_{GR} \equiv \frac{t_{LK, 0}}{t_{GR, 0}}
        &= \frac{m_{12}}{m_3}\frac{a_2^3}{\p*{a(t = 0)}^4} \p*{1 - e_2^2}^{3/2}
            \frac{3Gm_{12}}{c^2},\\
    \epsilon_{SL} \equiv \frac{t_{SL, 0}}{t_{GR, 0}}
        &= \frac{m_{12}}{m_3}\frac{a_2^3}{\p*{a(t = 0)}^4} \p*{1 - e_2^2}^{3/2}
            \frac{3G\p*{m_2 + \mu/3}}{2c^2}.
\end{align}

The adiabacitity parameter
\begin{equation}
    \mathcal{A} \equiv \abs*{\frac{\Omega_{SL}}{\Omega_L}}
        = \frac{\epsilon_{SL}}{t_{LK, 0}}
            \p*{\frac{a_0}{a}}^{5/2}\frac{1}{1 - e^2}
            \s*{\frac{3(1 + 4e^2)}{8t_{LK, 0}\sqrt{1 - e^2}}
                \p*{\frac{a}{a_0}}^{3/2}\abs*{\sin 2I}}^{-1},
\end{equation}
(note that $\Omega_L$ is a somewhat averaged sense, see LL18) can be evaluated
in these units as
\begin{equation}
    \mathcal{A} = \epsilon_{SL}\p*{\frac{a_0}{a}}^4\frac{1}{\sqrt{1 - e^2}}
        \frac{8}{3\p*{1 + 4e^2}\abs*{\sin 2I}}.
\end{equation}

Note also that the Hamiltonian is just
\begin{equation}
    H = \Omega_{SL} \hat{\bm{L}} \cdot \hat{\bm{S}},
        = \epsilon_{SL}\p*{\frac{a_0}{a}}^{5/2} \frac{1}{1 - e^2}
            \hat{\bm{L}} \cdot \hat{\bm{S}}.
\end{equation}

\subsection{Maximum Eccentricity and Merger Time}

Note that, since we are only evolving $\bm{L}$ and $\bm{e}$, and not $\bm{L}_2$
and $\bm{e}_2$, we are in the test mass approximation, under which we set $\eta
= 0$ in Bin's equations. As such, the maximum eccentricity satisfies (Eq 42 of
LL18 with $\eta \to 0$)
\begin{equation}
    \frac{3}{8}\frac{j_{\min}^2 - 1}{j_{\min}^2}\s*{
        5\cos I_0^2 - 3j_{\min}^2} + \epsilon_{GR}\p*{1 - \frac{1}{j_{\min}}} =
        0.
\end{equation}
Note that $\epsilon_{GR}$ is exactly as we defined above, incidentally, and that
when GR is negligible, this reduces to the classic $j_{\min} \equiv \sqrt{1 -
e_{\max}^2} = \sqrt{\frac{5}{3}\cos^2 I_0}$. Since $\epsilon_{GR}$ is generally
very small for most of the evolution, this generally reduces to the well known
\begin{equation}
    e_{\max} = \sqrt{1 - \frac{5}{3}\cos^2I_0}.
\end{equation}
This only fails to saturate for extremely high eccentricies, so $I_0 \to
90^\circ$.

\subsection{Attractor Behavior}

Proposal: The reason the $90^\circ$ attractor appears is that the initial
$\theta_{sb}$ is roughly stationary for $\mathcal{A} \ll 1$ (only small kicks
during each LK cycle, as long as the maximum eccentricity isn't too large), then
as we enter the transadiabatic regime, the L-K cycles die down and we simply
have conservation of adiabatic invariant.

The latter half of this follows the LL18 claim, where the requirement that
$\epsilon_{GR} \lesssim 9/4$ (GR precession of pericenter is slow enough that
L-K survives) equates to $\mathcal{A} \lesssim 3$. The former half of this is
somewhat tricky, but we can understand what is happening if we consider what is
happening in the frame corotating with $\Omega_{SL, e = 0}$ about $\hat{z}$:
every time that a LK cycle appears, $\Omega_{SL}$ becomes much larger, and the
axis of precession changes from $\hat{z}$ to the location of $\hat{L}$ very
briefly. We can imagine this as a kick in this corotating frame (which is the
right frame to consider for $\mathcal{A} \ll 1$). In the limit that $I$ does not
change very much between L-K cycles, and the azimuthal angle of $\hat{L}$ is
roughly symmetric, the impulses roughly cancel out in the $\theta_{sb}$
direction. In other words, after two LK cycles, $\theta_{sb}$ does not change
much in the corotating frame. This is indeed the picture that we obtain when we
observe the plot.

As such, the hypothesis is that if $\mathcal{A} \gtrsim 1$ is satisfied while
the kicks are still \emph{small}, then deviations about $90^\circ$ cannot be
very large, and adiabatic invariance tilts us right over. On the other hand, if
the kicks have become \emph{large}, then $\theta_{sb}$ after any particular LK
cycle is far from $90^\circ$, and this is frozen in during the adiabatic
invariance phase. This explains a few key observations:
\begin{itemize}
    \item If the initial $\theta_{sb} \approx 90^\circ$, e.g.\ if it is
        $70^\circ$, then the frozen in $\theta_{SL}$ at the end is also $\sim
        70^\circ$. Thus, it cannot be a dynamical attraction towards
        $\theta_{SL} = 90^\circ$, it's just conservation. This is a sensible
        condition to require though, since one might naively expect $\theta_{SL}
        \approx 0$ initially, and we know $I_0 \approx 90^\circ$ in order for
        this mechanism to be active at all.

    \item Two ways we can test this: if we intentionally weaken $\Omega_{SL}$,
        we should expect the width near $I_{0, \max}$ where the attractor breaks
        down to \emph{narrow}, i.e.\ e should be able to get closer to $I_{0,
        \max}$ while still seeing the attractor. The converse obviously holds.
        Secondly, we can do integrations farther away from $I_{0, \max}$ (so we
        can actually get some substantial $\hat{S}$ evolution) and plot the
        trajectory of $\hat{S}$ in the corotating frame. Some number tweaking
        may be necessary\dots
\end{itemize}

% \end{multicols}
\appendix

\section{$\bm{j}$ Equations}

We define vectors
\begin{align}
    \bm{j} &= \sqrt{1 - e^2}\hat{n},\\
    \bm{e} &= e\hat{u}.
\end{align}
Here, $\bm{j}$ is the dimensionless angular momentum vector and $\bm{e}$ is the
eccentricity vector; see LML15 for precise definitions. Note that $\bm{j} \cdot
\bm{e} = 0$, $j^2 + e^2 = 1$. Then, the EOM for the inner and outer vectors
satisfy to quadrupolar order
\begin{align}
    \rd{\bm{j}}{t} &= \frac{3}{4t_{LK}} \s*{
        \p*{\bm{j} \cdot \hat{n}_2} \p*{\bm{j} \times \hat{n}_2}
        - 5\p*{\bm{e} \cdot \hat{n}_2}\p*{\bm{e} \times \hat{n}_2}
        },\\
    \rd{\bm{e}}{t} &= \frac{3}{4t_{LK}} \s*{
        \p*{\bm{j} \cdot \hat{n}_2} \p*{\bm{e} \times \hat{n}_2}
        + 2\bm{j} \times \bm{e}
        - 5\p*{\bm{e} \cdot \hat{n}_2}\p*{\bm{j} \times \hat{n}_2}
        }.
\end{align}
Let's assume for the time being that $L_1 \ll L_2$, so the system is
sufficiently hierarchical that $\bm{j}_2$, $\bm{e}_2$ are constants. Note for
reference that
\begin{equation}
    t_{LK} \equiv \frac{L_1}{\mu_1 \Phi_0}
        = \frac{1}{n_1}\p*{\frac{m_1 + m_2}{m_3}}
            \p*{\frac{a_2}{a}}^3
            \p*{1 - e_2^2}^{3/2}.
\end{equation}
Here, $n_1 \equiv \sqrt{G\p*{m_1 + m_2}/a^3}$. Thus, $1 / t_{LK} \propto
a^{3/2}$.

GW radiation (Peters 1964) cause decays of $\bm{L}$ and $\bm{e}$ as
\begin{align}
    \at{\rd{\bm{L}}{t}}_{GW} &= -\frac{32}{5}\frac{G^{7/2}}{c^5}
        \frac{\mu^2 m_{12}^{5/2}}{a^{7/2}}
        \frac{1 + 7e^2/8}{\p*{1 - e^2}^2}\hat{L},\\
    \at{\rd{\bm{e}}{t}}_{GW} &= -\frac{304}{15} \frac{G^3}{c^5}
        \frac{\mu m_{12}^2}{a^4\p*{1 - e^2}^{5/2}}\p*{1 + \frac{121}{304}
            e^2}\bm{e},\\
    \at{\frac{\dot{a}}{a}}_{GW} &= -\frac{64}{5} \frac{G^3 \mu m_{12}^2}{c^5a^4}
        \frac{1}{\p*{1 - e^2}^{7/2}}\p*{1 + \frac{73}{24}e^2
            + \frac{37}{96}e^4}.
\end{align}
Here, $m_{12} \equiv m_1 + m_2$. The last GR effect is precession of $\vec{e}$,
which acts as
\begin{equation}
    \at{\rd{\bm{e}}{t}}_{GR} = \frac{3Gnm_{12}}{c^2a\p*{1 - e^2}}.
\end{equation}

Given this system (from LML15 + LL18), we can then add the spin-orbit coupling
term (from de Sitter precession), which is given in LL18 to be
\begin{align}
    \rd{\hat{S}}{t} &= \Omega_{SL}\hat{L} \times \hat{S},\\
    \Omega_{SL} &\equiv \frac{3Gn\p*{m_2 + \mu/3}}{2c^2a\p*{1 - e^2}}.
\end{align}
Note that $\mu$ is the reduced mass of the inner binary. We can drop the
back-reaction term since $S \ll L$. Thus, $\Omega_{SL} \propto a^{-5/2}$.

What is observed is that, as this system is evolved forward in time and GR
coalesces the inner binary, $\theta_{sl} \equiv \arccos\p*{\hat{S} \cdot
\hat{L}}$ goes to $90^\circ$ consistently. The relevant figure is Fig.~19 of
LL18, which shows that for a close-in, low-eccentricity perturber ($\bar{a}_{\rm
out, eff} \propto a_{out}$), the focusing is significantly stronger. Note that
initially, $I \equiv \arccos\p*{\hat{L} \cdot \hat{L}_2} \approx 90^\circ$ while
$\theta_{sl} \approx 0$.

In LL18, an adiabaticity parameter is defined:
\begin{equation}
    \mathcal{A} \equiv \abs*{\frac{\Omega_{SL}}{\Omega_L}},
\end{equation}
where $\Omega_{L} \simeq \ev*{\rd{\hat{L}}{t}}_{LK}$ to quadrupolar order. As
the inner binary coalesces, $\mathcal{A}$ transitions from $\ll 1$ to $\gg 1$
(as $\Omega_{SL}$ is a GR effect so ramps up very quickly as orbital separation
decreases).

The adiabaticity parameter $\mathcal{A}$ can be plotted upon rescaling in our
coordinates. Note that $\Omega_{\rm SL} = \frac{\delta a_0}{at_{LK, 0}}$, while
$\Omega_L \simeq \frac{3\p*{1 + 4e^2}}{8t_{LK}\sqrt{1 - e^2}}\abs*{\sin 2I}$ can
also be expressed in units of $t_{LK, 0}$. This gives us
\begin{equation}
    \mathcal{A} = \frac{8\delta \sqrt{1 - e^2}}{3\p*{1 +
        4e^2}}\p*{\frac{a_0}{a}}^4.
\end{equation}

\subsection{Simulations}

First, we run GR-less simulations, so let's take $t_{LK} = 1$ (no semimajor axis
evolution), and we reproduce LK oscillations.

Next, when accounting for GR, we should let $a$ evolve as above. Note that since
$\bm{j}$ and $\vec{e}$ are our dynamical variables, we should use $\bm{j} \equiv
\sqrt{1 - e^2} \hat{L} = \sqrt{1 - e^2}\frac{\bm{L}}{\mu \sqrt{Gm_{12}a\p*{1 -
e^2}}}$ and rewrite
\begin{equation}
    \at{\rd{\bm{j}}{t}}_{GW} = \frac{1}{\mu\sqrt{GMa}}\at{\rd{\bm{L}}{t}}_{GW}
        - \frac{\bm{j}}{2a}\at{\rd{a}{t}}_{GW}.
\end{equation}
To double check, we should verify that $\at{\rd{(j^2 + e^2)}{t}}_{GW} = 0$,
which can be verified as (Let's set $G = M = \mu = a = c = 1$ for convenience)
\begin{align}
    \frac{1}{2}\rd{(j^2 + e^2)}{t}
        &= \bm{j} \cdot \rd{\bm{j}}{t} + \bm{e} \cdot \rd{\bm{e}}{t},\\
        &= \bm{j} \cdot \s*{
            \p*{-\frac{32}{5}\frac{1 + 7e^2/8}{\p*{1 - e^2}^2}}\hat{L}
                - \frac{\bm{j}}{2}\p*{-\frac{64}{5}
                    \frac{1 + 73e^2/24 + 37e^4/96}{\p*{1 - e^2}^{7/2}}}}
            + \bm{e} \cdot
                \p*{-\frac{304}{15}\frac{1 + 121e^2/304}{\p*{1 - e^2}^{5/2}}}
                \bm{e},\\
        &= \p*{-\frac{32}{5}\frac{1 + 7e^2/8}{\p*{1 - e^2}^{3/2}}}
                + \p*{\frac{32}{5}
                    \frac{1 + 73e^2/24 + 37e^4/96}{\p*{1 - e^2}^{5/2}}}
            + e^2 \p*{-\frac{304}{15}\frac{1 + 121e^2/304}{\p*{1 - e^2}^{5/2}}}
                ,\\
        &= \frac{1}{15\p*{1 - e^2}^{5/2}}\s*{
            -96\p*{1 - e^2}\p*{1 + \frac{7e^2}{8}}
                + 96\p*{1 + \frac{73e^2}{24} + \frac{37e^4}{96}}
                - 304e^2\p*{1 + \frac{121e^2}{304}}}.
\end{align}
This can be verified to vanish upon term-by-term examination indeed.

For convenience, let's just define $t_{LK} = t_{LK,0} \frac{a_0^3}{a^3}$
and set $t_{LK, 0} = 1$. Furthermore, the timescale of relevance for the GW
terms is $t_{GW}^{-1} \sim \frac{G^3 \mu m_{12}^2}{c^5a^4}$. Let's express this
as some ratio $t_{GW} = \epsilon t_{LK, 0}\frac{a_0^4}{a^4}$. Thus, everything
should be nondimensionalized this way.

We lastly add de-Sitter precession of the spin of one of the inner binary
components, call this $\hat{S}$. Similarly, let's just define a proportionality
constant $t_{SL} = \delta t_{LK, 0}\frac{a_0}{a}$, then
\begin{equation}
    \rd{\hat{S}}{(t / t_{LK, 0})} = \delta \frac{a_0}{a} \hat{L} \times \hat{S}.
\end{equation}

Our final simulation equations are thus ($\tau = t / t_{LK, 0}$)
\begin{align}
    \rd{\bm{j}}{\tau} ={}& \frac{3}{4}\p*{\frac{a_0^3}{a^3}} \s*{
        \p*{\bm{j} \cdot \hat{n}_2} \p*{\bm{j} \times \hat{n}_2}
        - 5\p*{\bm{e} \cdot \hat{n}_2}\p*{\bm{e} \times \hat{n}_2}
        }\nonumber\\
        &- \p*{\epsilon \frac{a_0^4}{a^4}}\p*{
            \frac{32}{5}
                \frac{1 + 7e^2/8}{\p*{1 - e^2}^{5/2}}
            - \frac{32}{5} \frac{1}{\p*{1 - e^2}^{7/2}}\p*{1 + \frac{73}{24}e^2
                    + \frac{37}{96}e^4}
            }\bm{j},\\
    \rd{\bm{e}}{\tau} ={}&
        \frac{3}{4}\p*{\frac{a_0^3}{a^3}} \s*{
            \p*{\bm{j} \cdot \hat{n}_2} \p*{\bm{e} \times \hat{n}_2}
            + 2\bm{j} \times \bm{e}
            - 5\p*{\bm{e} \cdot \hat{n}_2}\p*{\bm{j} \times \hat{n}_2}
            }
        -
            \p*{\epsilon\frac{a_0^4}{a^4}}
            \frac{304}{15}
            \frac{1}{\p*{1 - e^2}^{5/2}}
            \p*{1 + \frac{121}{304} e^2}\bm{e},\\
    \rd{\hat{S}}{\tau} ={}& \delta \frac{a_0}{a}
        \frac{\bm{j}}{\sqrt{1 - e^2}} \times \hat{S},\\
    \rd{a}{\tau} ={}& -a\p*{\epsilon\frac{a_0^4}{a^4}}
        \frac{64}{5} \frac{1}{\p*{1 - e^2}^{7/2}}\p*{1 + \frac{73}{24}e^2
            + \frac{37}{96}e^4}.
\end{align}

\end{document}

